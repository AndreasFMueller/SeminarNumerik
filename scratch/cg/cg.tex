%
% cg.tex
%
% (c) 2020 Prof Dr Andreas Müller, Hochschule Rapperswil
%
\documentclass[a4paper,11pt]{article}
\usepackage[utf8]{inputenc}
\usepackage[T1]{fontenc}
\usepackage{times}
\usepackage{amsmath}
\usepackage{amssymb}
\usepackage{amsthm}
\usepackage{txfonts}
\usepackage{graphicx}
\usepackage[german]{babel}
\usepackage{hyperref}
\usepackage{tikz}
\usetikzlibrary{shapes.geometric,arrows,automata}
%\usepackage[backend=bibtex]{biblatex}
\makeatletter
\def\blx@maxline{77}
\makeatother
%\addbibresource{references.bib}
%
% macros.tex -- some common macro definitions
%
% (c) 2019 Prof Dr Andreas Müller, Hochschule Rapperswil
%
\hypersetup{
    linktoc=all,
    linkcolor=blue
}
\newcounter{beispiel}
\newenvironment{beispiele}{
\bgroup\smallskip\parindent0pt\bf Beispiele\egroup

\begin{list}{\arabic{beispiel}.}
  {\usecounter{beispiel}
  \setlength{\labelsep}{5mm}
  \setlength{\rightmargin}{0pt}
}}{\end{list}}
\newcounter{uebungsaufgabezaehler}
% environment fuer uebungsaufgaben
\newenvironment{uebungsaufgaben}{
\begin{list}{\arabic{uebungsaufgabezaehler}.}
  {\usecounter{uebungsaufgabezaehler}
  \setlength{\labelwidth}{2cm}
  \setlength{\leftmargin}{0pt}
  \setlength{\labelsep}{5mm}
  \setlength{\rightmargin}{0pt}
  \setlength{\itemindent}{0pt}
}}{\end{list}\vfill\pagebreak}
\newenvironment{teilaufgaben}{
\begin{enumerate}
\renewcommand{\labelenumi}{\alph{enumi})}
}{\end{enumerate}}
% Aufgabe
\newcounter{problemcounter}[chapter]
\def\aufgabenpath{chapters/uebungsaufgaben/}
\def\ainput#1{\input\aufgabenpath/#1}
\def\verbatimainput#1{\expandafter\verbatiminput{\aufgabenpath/#1}}
\def\aufgabetoplevel#1{%
\expandafter\def\expandafter\inputpath{#1}%
\let\aufgabepath=\inputpath
}
\def\includeagraphics[#1]#2{\expandafter\includegraphics[#1]{\aufgabepath#2}}
% \aufgabe
\newcommand{\uebungsaufgabe}[1]{%
\refstepcounter{problemcounter}%
\label{#1}%
\bigskip{\parindent0pt\strut}\hbox{\bf\arabic{problemcounter}. }%
\expandafter\def\csname aufgabenpath\endcsname{\inputpath/}%
\expandafter\input{chapters/uebungsaufgaben/#1.tex}
}
\renewcommand\theproblemcounter{\thechapter.\arabic{problemcounter}}
% linsys
\newcolumntype{\linsysR}{>{$}r<{$}}
\newcolumntype{\linsysL}{>{$}l<{$}}
\newcolumntype{\linsysC}{>{$}c<{$}}
\newenvironment{linsys}[1]{%
\begin{tabular}{*{#1}{\linsysR@{\;}\linsysC}@{\;}\linsysR}}%
{\end{tabular}}

% Loesung
\def\swallow#1{
%nothing
}
\NewEnviron{loesung}[1][Lösung]{%
\begin{proof}[#1]%
\renewcommand{\qedsymbol}{$\bigcirc$}
\BODY
\end{proof}
}
\NewEnviron{bewertung}{%
\begin{proof}[Bewertung]%
\renewcommand{\qedsymbol}{}
\BODY
\end{proof}
}
\NewEnviron{diskussion}{%
\begin{proof}[Diskussion]%
\renewcommand{\qedsymbol}{}
\BODY
\end{proof}
}
\NewEnviron{hinweis}{%
\begin{proof}[Hinweis]%
\renewcommand{\qedsymbol}{}
\BODY
\end{proof}
}
\def\keineloesungen{%
\RenewEnviron{loesung}{\relax}
\RenewEnviron{bewertung}{\relax}
\RenewEnviron{diskussion}{\relax}
}
\newenvironment{beispiel}{%
\begin{proof}[Beispiel]%
\renewcommand{\qedsymbol}{$\bigcirc$}
}{\end{proof}}

\allowdisplaybreaks

\lhead{Inhaltsverzeichnis}
\rhead{}
\tableofcontents
\newtheorem{satz}{Satz}[chapter]
\newtheorem{hilfssatz}[satz]{Hilfssatz}
\newtheorem{korollar}[satz]{Korollar}
\newtheorem{lemma}[satz]{Lemma}
\newtheorem{definition}[satz]{Definition}
\newtheorem{annahme}[satz]{Annahme}
\newtheorem{problem}[satz]{Problem}
\newtheorem{aufgabe}[satz]{Aufgabe}
\newtheorem*{problem*}{Problem}
\newtheorem{forderung}{Forderung}[chapter]
\newtheorem{konsequenz}[satz]{Konsequenz}
\newtheorem{algorithmus}[satz]{Algorithmus}
\renewcommand{\floatpagefraction}{0.7}

\def\sinc{\operatorname{sinc}}
\def\si{\operatorname{si}}
\definecolor{darkgreen}{rgb}{0,0.6,0}
\definecolor{orange}{rgb}{1,0.6,0}

% from https://tex.stackexchange.com/questions/156862/displaying-author-for-each-chapter-in-book

%\newcommand\chapterauthor[1]{\authortoc{#1}\printchapterauthor{#1}}
%\WithSuffix\newcommand\chapterauthor*[1]{\printchapterauthor{#1}}
%
%\makeatletter
%\newcommand{\printchapterauthor}[1]{%
%  {\parindent0pt\vspace*{-25pt}%
%  \linespread{1.1}\large\scshape#1%
%  \par\nobreak\vspace*{35pt}}
%  \@afterheading%
%}
%\newcommand{\authortoc}[1]{%
%  \addtocontents{toc}{\vskip-10pt}%
%  \addtocontents{toc}{%
%    \protect\contentsline{chapter}%
%    {\hskip1.3em\mdseries\scshape\protect\scriptsize#1}{}{}}
%  \addtocontents{toc}{\vskip5pt}%
%}
%\makeatother

\title{CG-Algorithmus und Orthogonalität}
\author{
Andreas Müller\footnote{\texttt{andreas.mueller@hsr.ch},
Hochschule Rapperswil}
}
\begin{document}
\maketitle
\renewcommand{\floatpagefraction}{0.7}

Der CG-Algorithmus löst ein lineares Gleichungssystem $Ax=b$ mit
einer symmetrischen positiv definition Koeffizientenmatrix $A$, indem
er es auf ein Minimalproblem reduziert, welches dann mit Gradientabstieg
gelöst wird.

\begin{satz}
Die Lösung des Gleichungssystems $Ax=b$ mit einer positiv definiten
symmetrischen Matrix $A$ ist das Minimum der Form
\[
J(x) = \frac12 x^t Ax - b^tx,
\]
deren Gradient
\[
\operatorname{grad}J(x) = Ax-b
\]
ist.
\end{satz}

\begin{definition}
Das {\em Residuum} $r$ eines Vektors $x$ ist der Vektor, um den das
Gleichungssystem nicht erfüllt ist, also
\[
r = Ax-b = \operatorname{grad}J(x).
\]
Der {\em Fehler} $f$ des Vektors $x$ ist der Vektor so, dass
$x-f$ Lösung des Gleichungssystems ist, also
\[
A(x-f)=b
\Rightarrow
Ax-b=Af=r.
\]
Für das Skalarprodukt von Vektoren schreiben wir
\[
x\cdot y
=
x^ty
=
\langle x,y\rangle.
\]
Die Matrix $A$ definiert ein weiteres Skalarprodukt
\[
\langle x,y\rangle_A
=
\langle x,Ay\rangle
=
\langle Ax,y\rangle
=
x^tAy.
\]
\end{definition}

\begin{proof}[Beweis]
Die partielle Ableitung von $J(x)$ nach $x_k$ ist
\begin{align*}
\frac{\partial J}{\partial x_k}
&=
\frac{\partial}{\partial x_k}
\biggl(
\frac12
\sum_{i,j} x_i a_{ij} x_j - \sum_i b_i x_i
\biggr)
\\
&=
\frac12
\sum_{i,j}
\delta_{ik} a_{ij}x_j
+
\frac12
\sum_{i,j}
x_i a_{ij} \delta_{jk}
-
\sum_i b_i \delta_{ik}
\\
&=
\frac12
\sum_{j} a_{kj} x_j
+
\frac12
\sum_{i} x_i a_{ik}
-
b_k
\\
&=
\frac12
\sum_{j} a_{kj} x_j
+
\frac12
\sum_{i} a_{ki} x_i
-b_k
\\
&=
\sum_{i} a_{ki} x_i - b_k
=
(Ax-b)_k.
\end{align*}
Daraus folgt $\operatorname{grad}J(x) = Ax-b$ und die Behauptung über
das Minimum.
\end{proof}

Der CG-Algorithmus löst das Gleichungssystem $Ax=b$ iterativ, indem er
durch Gradient-Abstieg das Minimum von $J(x)$ sucht.
Der Schritt $k$ beginnt mit einer approximativen Lösung $x_k$, die mit
Hilfe von Abstieg in Richtung $v_k$ verbessert werden soll.
Die Abstiegsrichtung ist der Grad
aus die
Dabei verwendet er im Schritt $k$ die Abstiegsrichtung $v_k$ und sucht
das Minimum von J.

Die Abstiegsrichtungen sind dabei so gewählt, dass sie wie man sagt
{\em konjugiert} sind.
Die geometrische Intuition dafür ist die folgende.
Die Gleichung $J(x)=c$ definiert ein Ellipsoid.
Verwendet man als Basis des Raumes konjugierte Richtungen $v_i$, dann lässt
sich das Ellipsoid durch eine Diagonalmatrix darstellen.
Damit dies möglich ist, muss für jedes Paar von Vektoren gelten
\[
\langle v_i , v_j\rangle_A
=
\langle v_i,Av_j\rangle
=
\langle Av_i,v_j\rangle
=
v_i^t Av_j = 0,
\]
die Vektoren $v_j$ müssen also orthogonal bezüglich des Skalarproduktes
$\langle\;\;,\;\rangle_A$ sein.
Dies lässt sich zum Beispiel durch Diagonalisierung erreichen, was
aber natürlich viel zu aufwändig wäre.

Man kann aber immer eine Basis aus konjugierten Richtungen finden,
indem man das Gram-Schmidt-Orthogonalisierungsverfahren bezüglich
des Skalar-Produktes $\langle\;\;,\;\rangle_A$ auf eine beliebige
Basis anwendet.

Der CG-Algorithmus verwendet nicht eine von Anfang an gegebene Basis,
sondern baut diese im Laufe des Algorithmus auf.
Jede neue Richtung soll dabei ermöglichen, die Lösung möglichst effizient
zu verbessern.
Dazu liegt nahe, als nächsten Basisvektor den Gradienten im Punkt
$x_k$ zu verwenden, also 
\[
\operatorname{grad}J(x_k) = Ax_k-b = r_k
\]
Dieser ist aber nicht zu den bisherigen Vektoren $A$-orthogonal,
muss also erst orthogonalisiert werden:
\begin{equation}
v_{k} = r_k - \sum_{i=1}^{k-1} \langle r_k,v_i\rangle_A v_i
\label{ortho}
\end{equation}
Mit diesen Abstiegsrichtungen als Basis ist das Ellipsoid achsenparallel,
der Abstieg erfolgt immer in Richtung einer Achse und bestimmt in
dieser Basis in jedem Schritt genau eine Koordinate exakt.
Der Algorithmus konvergiert also immer in maximal $n$ Schritten.
Weniger als $n$ Schritte sind allerdings auch möglich, dies ist
aber sofort zu erkennen, weil dann $r=0$ wird.

Die Orthogonalisierung \eqref{ortho} ist relativ aufwändig, aber die
spezielle Wahl der Abstiegsrichtungen erlaubt den Schluss, dass der
tatsächliche Aufwand viel geringer ist.

%
% r_i und v_i spannen den gleichen Raum auf
%
\begin{beobachtung}
Es gilt
\begin{equation}
\langle v_1,\dots,v_k\rangle
=
\langle r_1,\dots,r_k\rangle.
\label{reqv}
\end{equation}
\end{beobachtung}

\begin{proof}[Beweis]
Da die Vektoren $v_k$ jeweils aus $r_k$ und $v_1,\dots,v_{k-1}$
durch Orthogonalisieren entstehen, gilt
\begin{align*}
v_1 &= r_1
\\
v_2 &= r_2 + \alpha_{21}v_1
\\
v_3 &= r_3 + \alpha_{31}v_1 + \alpha_{32}v_2
\\
v_4 &= r_4 + \alpha_{41}v_1 + \alpha_{42}v_2 + \alpha_{43}v_3
\\
&\qquad\vdots
\\
v_k &= r_k + \sum_{i=1}^{k-1} \alpha_{ki} v_i
\end{align*}
Daraus kann man durch Umstellen der Gleichungen ableiten: $r_k$ lässt
sich immer aus $v_1,\dots,v_k$ linear kombinieren, es gilt also
\begin{equation}
r_k
\in
\langle v_1,\dots,v_k \rangle.
\label{rinv}
\end{equation}
Aus der ersten Gleichung liest man aber auch ab: $v_1 \in \langle r_1\rangle$.
Die zweite Gleichung sagt dann, dass
\[
v_2\in \langle r_2\rangle + \langle v_1\rangle =
\langle r_2\rangle + \langle r_1\rangle
=
\langle r_1,r_2\rangle.
\]
Indem man so weiter verfährt kann man schliessen, dass
\begin{equation}
v_k \in \langle r_1,\dots, r_k\rangle.
\label{vinr}
\end{equation}
Aus \eqref{rinv} kann man schliessen
$\langle r_1,\dots,r_k\rangle\subset\langle v_1,\dots,v_k\rangle$,
aus \eqref{vinr} folgt entsprechend
$ \langle v_1,\dots,v_k\rangle \subset \langle r_1,\dots,r_k\rangle $.
Beide zusammen ergeben \eqref{reqv}.
\end{proof}

%
% b=0
%
\begin{beobachtung}
Ist $x$ eine Lösung von $Ax=b$, dann konvergiert der CG-Algorithmus 
ausgehend von $x_1$ mit den Abstiegsrichtungen $v_i$ und Residuen $r_i$
gegen die Lösung $x$.
Der CG-Algorithmus für $b=0$ ausgehend von $f_1=x-x_1$ konvergiert dann
mit den gleichen Abstiegsrichtungen und Residuen gegen $0$.
\end{beobachtung}

\begin{proof}[Beweis]
\end{proof}

Diese Beobachtung besagt, dass man sich nicht um den Vektor $b$ kümmern
kann und stattdessen Aussagen zur Konvergenz aus dem Fall $b=0$
ablesen kann.
Wir können daher in den nachfolgenden Überlegungen davon ausgehen,
dass $b=0$ ist.
Es gilt jetzt also
\begin{align*}
r_k &= Ax_k
&
v_k &= r_k - \sum_{i=1}^{n-1}\langle v_i,r_k\rangle_A v_i
\end{align*}
Insbesondere ist also $r_k$ nicht einfach irgend etwas, sondern
etwas von der Form $Ax_k$.

%
% Krylov-Räume
%
\begin{beobachtung}
Der Vektor $r_1$ und die Bilder $Ar_1$, $A^2r_1$ usw.~spannen den gleichen
Raum auf wie die Vektoren $r_1,\dots,r_k$.
\end{beobachtung}

\begin{proof}[Beweis]
Zunächst verwenden wir wieder $r_1=Ax_1$.
Die erste Abstiegsrichtung ist $v_1=r_1$, also ist $x_2=x_1+tr_1$.
Wendet man darauf $A$ an, erhält man
\[
r_2=Ax_2 = Ax_1 + tAr_1 = r_1 + tAr_1 \in \langle r_1,Ar_1\rangle.
\]

Die nächste Abstiegsrichtung ist $v_2$, eine Linearkombination der Vektoren
$r_1$ und $r_2$.
Man kann also schreiben
\[
x_3 = x_2 + t_1v_1 + t_2v_2 \in x_2 + \langle r_1,r_2\rangle
\]
Wendet man wieder $A$ an, erhält man
\[
r_3 = Ax_3
=
Ax_2 + A\langle r_1,r_2\rangle
=
r_2 + A\langle r_1,Ar_1\rangle
=
\langle r_1,Ar_1\rangle + \langle Ar_1,A^2r_1\rangle
\]
Wenn also $r_k\in \langle r_1,Ar_1,\dots A^{k-1}r_1\rangle$ ist, dann
kann man
\[
x_{k+1} = x_{k} + t_{k1}r_1 + t_{k2}r_2 + t_{kk}r_k
\in
x_{k} + \langle r_1,\dots,r_{k}\rangle
\]
schreiben.
Wendet man darauf $A$ an, erhält man
\begin{align*}
r_{k+1}
&=
Ax_{k+1}
\in
Ax_{k} + A\langle r_1,\dots,r_k\rangle
\\
&=
r_k
+
A\langle r_1,Ar_1,\dots,A^{k-1}r_1\rangle
=
\langle r_1,Ar_1,\dots,A^{r-1}r_1\rangle
+
A\langle r_1,Ar_1,\dots,A^{k-1}r_1\rangle
\\
&=
\langle r_1,Ar_1,\dots,A^kr_1\rangle
\end{align*}
Damit ist gezeigt, dass $r_{k+1} \in \langle r_1,Ar_1,\dots,A^kr_1\rangle$.Nach vollständiger Induktion folgt daraus, 
\[
\langle r_1,\dots,r_k\rangle = \langle r_1,Ar_1,\dots,A^{k-1}r_1\rangle.
\]
\end{proof}

\begin{definition}
Die Räume $\langle r_1,Ar_1,\dots,A^{k-1}r_1\rangle$ werden auch
mit $\mathcal{K}_k(A,r_1)$ bezeichnet und heissen Krylov-Räume des
Vektors $r_1$ mit der Matrix $A$.
\end{definition}

%
% A-Orthogonalität der x_i auf den v_i
%
\begin{beobachtung}
Der Vektor $x_k$ sind $A$-orthogonal auf allen Vektoren $v_i$ mit
$i < k$.
\end{beobachtung}

\begin{proof}[Beweis]
Die Konstruktion des Vektors $x_k$ als Lösung eines Minimalproblems
erfolgt so, dass von $x_{k-1}$
ein Teilstück $tv_k$ subtrahiert derart, dass $x_{k}$ und $tv_k$ 
$A$-orthogonal sind.
Der Vektor $x_k$ entsteht daher aus $x_1$ durch Subtraktion
der $A$-orthogonalen Komponenten bezüglich der Vektoren
$v_1,\dots,v_{k-1}$, also ist $x_k$ auf allen Vektoren $v_1,\dots,v_{k-1}$
orthogonal.
\end{proof}

%
% Die Vektoren 
%
\begin{beobachtung}
\label{rortho}
Die Vektoren $r_k$ sind auf allen Vektoren $r_1,\dots,r_{k-1}$ 
orthogonal.
\end{beobachtung}

\begin{proof}[Beweis]
Aus der vorangegangenen Beobachtung weiss man
$x_k\perp_A \langle v_1,\dots,v_{k-1}\rangle =
\langle r_1,\dots, r_{k-1}\rangle$.
$A$-orthogonal heisst
\[
0
=
\langle x_k,r_i\rangle_A 
=
\langle Ax_k,r_i\rangle
=
\langle r_k,r_i\rangle
\qquad
\forall i < k.
\]
Damit ist gezeigt, dass $r_k$ auf allen $r_i$ mit $i<k$ orthogonal ist.
\end{proof}

\begin{beobachtung}
Das Residuum $r_{k+1}$ ist orthogonal auf der vorangegangene 
Abstiegsrichtung $v_{k}$: $\langle v_k,r_{k+1}\rangle=0$.
\end{beobachtung}

\begin{proof}[Beweis]
Der Abstieg führt immer zum Minimum, die Gerade $x_k+tv_k$ ist
also Tangential an das Ellipsoid durch $x_{k+1}$.
Daher ist der Gradient im Punkt $x_{k+1}$ orthogonal auf $v_k$.
Aber $\operatorname{grad}J(x_{k+1})=r_{k+1}$.
\end{proof}

\begin{beobachtung}
Das Residuum $r_{k+1}$ ist $A$-orthogonal zu allen Vektoren
$v_1,\dots,v_{k-1}$.
\end{beobachtung}

\begin{proof}[Beweis]
Wir müssen das $A$-Skalarprodukt von $r_{k+1}$ mit den Vektoren
$r_1,\dots,r_{k-1}$ berechnen.
\begin{equation}
\langle r_{k+1},r_i\rangle_A
=
\langle r_{k+1},Ar_i\rangle
\qquad\text{für $i<k$}
\label{zweitletzt}
\end{equation}
Die Vektoren $Ar_i$ mit $i<k$  sind alle in
\begin{align}
\langle Ar_1,\dots,Ar_{k-1}\rangle
&=
A\langle r_1,\dots,r_{k-1}\rangle
\notag
\\
&=
A\langle r_1,Ar_1,\dots,A^{k-2}r_1\rangle
\notag
\\
&=
\langle Ar_1,A^2r_1,\dots,A^{k-1}r_1\rangle
\notag
\\
&\subset
\langle r_1,Ar_1,\dots,A^{k-1}r_1\rangle
=
\langle r_1,\dots,r_{k}\rangle
\label{letzt}
\end{align}
Nach Beobachtung~\ref{rortho} ist $r_{k+1}$ orthogonal auf allen
Vektoren $r_1,\dots,r_{k}$. 
Nach \eqref{letzt} folgt jetzt, dass $r_{k+1}$ orthogonal ist auf allen
Vektoren $Ar_1,\dots Ar_{k-1}$, also ist nach \eqref{zweitletzt}
$r_{k+1}$ $A$-orthogonal auf den Vektoren $r_1,\dots,r_{k-1}$.

Da $\langle r_1,\dots,r_{k-1}\rangle = \langle v_1,\dots,v_{k-1}\rangle$ 
gilt, ist $r_{k+1}$ auch $A$-orthogonal auf allen Vektoren
$v_1,\dots,v_{k-1}$.
\end{proof}



%\printbibliography
\end{document}

