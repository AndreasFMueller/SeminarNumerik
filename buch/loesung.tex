%
% loesung.tex -- Beispiel-File für die Beschreibung der Loesung
%
% (c) 2020 Prof Dr Andreas Müller, Hochschule Rapperswil
%
\section{Lösung
\label{ableitung:section:loesung}}
\rhead{Lösung}
Wie bereits erklärt, basiert die Lösung darauf, dass ein geeignetes numerisches Verfahren anstelle des Backpropagation-Algorithmus verwendet wird um den Gradienten zu berchnen. Der Gradientenabstieg kann somit weiterhin vom Framework übernommen werden, obschon die Gradientenberechnung ersetzt wurde. Dies ermöglicht zu einem spätern Zeitpunkt einen Vergleich beider Verfahren.
\subsection{Finite Differenzen Methode}
Die Finite Differenzen Methode ist motiviert durch die klassische Ableitung. Die Ableitung (Tangente) wird als Sekante zweier Punkte der Funktion berechnet, wobei beide Punkte einen Abstand von $\Delta h \rightarrow 0$ annehmen.
Dieses Konzept ist als Differentialquotient
\begin{equation}
f'(x_0) = \lim_{{\Delta h} \rightarrow 0} \frac{f(x_0+\Delta h) - f(x_0)}{\Delta h}
\label{ableitung:equations:differentialquotient}
\end{equation}
bekannt. 
In der Numerik existiert die Möglichkeit von $\Delta h \rightarrow 0$ nicht, da die Datenpunkte diskretisiert sind. Stattdessen ist der kleinstmögliche Abstand zwei benachbarte Datenpunkte. Entsprechend kann man die numerische Ableitung (Differenzenquotient) als eine der Differentialquotienten
\begin{equation}
\begin{split}
f'(x_0) &= \frac{f(x_0 + h) - f(x_0)}{h} \\
f'(x_0) &= \frac{f(x_0) - f(x_0 - h)}{h} \\
f'(x_0) &= \frac{f(x_0 + h) - f(x_0 - h)}{2h} \\
\end{split}
\label{ableitung:equations:differenzenquotient}
\end{equation}
schreiben, wobei $h$ hier den Abstand zweier benachbarter Datenpunkte annimmt. Die Genauigkeit ist limitiert durch den Abstand der zwei Stützstellen.
\begin{figure}
	\centering
\begin{tikzpicture}[>=stealth',
dot/.style={circle,draw,fill=white,inner sep=0pt,minimum size=4pt}]

% draw axis lines
\draw[->,thick] (-0.5,0) -- ++(11,0) node[below left]{$x$};
\draw[->,thick] (0,-0.5) -- ++(0,7) node[below right]{$f(x)$};
\coordinate (O) at (0,0);

% create path for function curve
\path[thick,red] (-0.3,2) to[out=-25, in=200] coordinate[pos=0.2] (p) coordinate[pos=0.6] (q) (9,5);
% fill area
\fill[blue, opacity=.1] (p) -| (q);
% draw the secant line with fixed length
\draw[shorten <=-1.5cm] (p) -- ($ (p)!7.5cm!(q) $) node[below right, pos=0.9]{Sekante};
% draw function curve
\draw[thick,red] (-0.3,2) to[out=-25, in=200] (9,5);

% draw all points
\node[dot,label={above:$P$}] (P) at (p) {};
\node[dot,label={above:$Q$}] (Q) at (q) {};
\node[dot] (p1) at (P |- O) {};
\node[dot] (p2) at (Q |- O) {};
\node[dot] (p3) at (P -| Q) {};

% draw lines between nodes and place text
\draw (P) -- node[left]{$f(x_{0})$} (p1) node[dot,label={below:$x_{0}$}]{};
\draw (p2) node[dot,label={below:$x_{0} + \varepsilon$}]{} -- (p3);
\path (p1) -- node[below]{$\varepsilon$} (p2);

% draw blue arrows between nodes
\draw[<->,blue,thick] (P) -- node[below]{$\Delta h$} (p3);
\draw[<->,blue,thick] (Q) -- node[right]{$f(x_{0} + \Delta h) - f(x_{0})$} (p3);

% draw the explanation for the y-value of point Q
\draw[help lines] (Q) -- (Q -| {(9.5,0)}) ++(-0.5,0) coordinate (p4);
\draw[help lines, <->] (p4) -- node[fill=white,text=black]{$f(x_{0} + \varepsilon)$} (p4 |- O);

\end{tikzpicture}
\end{figure}

\subsection{Taylor Methode}

Mithilfe der Taylor Reihe lässt sich jede beliebige analytische Funktion $f(x)$ approxmieren. Dies machen wir uns für die Herleitung einer etwas stabileren numerischen Ableitung zu Nutze.

\begin{equation}
f(x) = f(a)+{\frac {f'(a)}{1!}}(x-a)+{\frac {f''(a)}{2!}}(x-a)^{2}+{\frac {f'''(a)}{3!}}(x-a)^{3}+\cdots
\label{ableitung:eqn:taylorseries}
\end{equation}
Um eine höhere Genauigkeit zu erreichen, kann die Anzahl der Stützstellen erhöhrt werden. 
Dies lässt sich mithilfe der Taylorreihe herleiten.
Im Grunde wird ein Polynom höherer Ördnung gesucht, welches Stützstellen in einem uniformen Abstand hat.
Die Ableitung dieses Polynoms sollte dann eine höhere Genauigkeit aufweisen, da durch die höhere Ordnung die Ursprungsfunktiuon besser um den Punkt der Ableitung approximiert werden kann.
Als Beispiel soll die erste Ableitung mit 5 Stützstellen in einem uniformen Abstand $h$ berechnet werden:
\begin{equation}
\frac{\delta f}{\delta x} \approx Af(x-2h) + Bf(x-h) + Cf(x) + Df(x+h) + Ef(x+2h)
\end{equation}
Die jeweiligen Stützstellen können nun mittels Taylor-Reihe erweitert werden:
\begin{equation}
\begin{split}
Af(x-2h) \approx & Af(x) + Af'(x)(-2h) + A\frac{1}{2}f''(x)(-2h)^2+A\frac{1}{6} + f'''(x)(-2h)^3+A\frac{1}{24} + \\ 
& f''''(x)(-2h)^4 + A\frac{1}{120}f'''''(x)(-2h)^5 \\
Bf(x-h) \approx & Bf(x) + Bf'(x)(-h) + B\frac{1}{2}f''(x)(-h)^2+B\frac{1}{6} + f'''(x)(-h)^3+B\frac{1}{24} + \\ 
& f''''(x)(-h)^4 + B\frac{1}{120}f'''''(x)(-h)^5 \\
Cf(x) = & Cf(x) \\
Df(x+h) \approx & Df(x) + Df'(x)(+h) + D\frac{1}{2}f''(x)(+h)^2+D\frac{1}{6} + f'''(x)(+h)^3+D\frac{1}{24} + \\ 
& f''''(x)(+h)^4 + D\frac{1}{120}f'''''(x)(+h)^5 \\
Ef(x+2h) \approx & Ef(x) + Df'(x)(+2h) + E\frac{1}{2}f''(x)(+2h)^2+E\frac{1}{6} + f'''(x)(+2h)^3+E\frac{1}{24} + \\ 
& f''''(x)(+2h)^4 + E\frac{1}{120}f'''''(x)(+2h)^5
\end{split}
\end{equation}
Bei einsetzen dieser Gleichungen in die erste Gleichung (siehe oben) ist ersichtlich, dass nur die Koeffizienten der 2. Ordnung bestehen bleiben müssen und folglich 1 ergeben müssen. Der Rest sollte verschwinden und 0 ergeben. Zur Unterstüzung der Lesbarkeit können die Terme nach Ableitungsgrad sortiert werden:
\begin{equation}
\begin{linsys}{6}
f(x)                        \cdot(&&  A&+&B&+&C&+&D&+&  E)&=&0 \\
\frac{1}{2} f'(x)(h)        \cdot(&&-2A&-&B& & &+&D&+& 2E)&=&\frac{2}{h}\\
\frac{1}{6} f''(x)(h^2)     \cdot(&&14A&+&B& & &+&D&+& 4E)&=&0 \\
\frac{1}{24} f''''(x)(h^3)  \cdot(&&-8A&-&B& & &+&D&+& 8E)&=&0 \\
\frac{1}{120} f'''''(x)(h^4)\cdot(&&16A&+&B& & &+&D&+&16E)&=&0 
\end{linsys}
\end{equation}
%
%\begin{equation}
%\begin{split}
%f(x) \cdot (A + B + C + D + E) & = 0 \\
%\frac{1}{2} f'(x)(h) \cdot (-2A - B + D + 2E) &= \frac{2}{h}\\
%\frac{1}{6} f''(x)(h^2) \cdot (4A + B + D + 4E) &= 0 \\
%\frac{1}{24} f''''(x)(h^3) \cdot (-8A - B + D + 8E) &= 0 \\
%\frac{1}{120} f'''''(x)(h^4) \cdot (16A + B + D + 16E) &= 0 
%\end{split}
%\end{equation}
Dies lässt sich auch in Matrix Form wie folgt schreiben:
\begin{align}
\begin{bmatrix}
1 & 1 & 1 & 1 & 1 \\
-2 & -1 & 0 & 1 & 2 \\
4 & 1 & 0 & 1 & 4 \\
-8 & -1 & 0 & 1 & 8 \\
16 & 1 & 0 & 1 & 16 \\
\end{bmatrix}
\cdot
\begin{bmatrix}
A \\
B \\
C \\
D \\
E \\
\end{bmatrix}
= \frac{1}{h} 
\begin{bmatrix}
0 \\
2 \\
0 \\
0 \\
0 \\
\end{bmatrix}
\end{align}
Diese Matrix lässt sich nach
\begin{align}
\begin{bmatrix}
A \\
B \\
C \\
D \\
E \\
\end{bmatrix}
=
\frac{1}{h}
\begin{bmatrix}
1 & 1 & 1 & 1 & 1 \\
-2 & -1 & 0 & 1 & 2 \\
4 & 1 & 0 & 1 & 4 \\
-8 & -1 & 0 & 1 & 8 \\
16 & 1 & 0 & 1 & 16 \\
\end{bmatrix}^{-1}
\begin{bmatrix}
0 \\
2 \\
0 \\
0 \\
0 \\
\end{bmatrix}
=
\frac{1}{h}
\begin{bmatrix}
\frac{1}{12} \\
\frac{-2}{3} \\
0 \\
\frac{2}{3} \\
\frac{1}{12} \\
\end{bmatrix}
\end{align}
invertieren.
Dies ergibt die erste Ableitung
\begin{align}
f'(x)  \approx \frac{1f(x-2h) - 8f(x-h) + 8f(x-h) - 1f(x+2h)}{12h}
\label{ableitung:eqn:aprox_5}
\end{align}
angenähert durch 5 Stützstellen.
Analog kann dies natürlich für höhere Ableitungen oder non-uniforme Stützstellen gemacht werden.
\subsubsection{Fehlerabschätzung}
Die Herleitung der finite Differenzen Methode geschieht über die Taylorreihe gemäss Gleichung \ref{ableitung:eqn:taylorseries}. Um die Approximation mit fünf Stützstellen zu erhalten wurden die Taylorpolynome bis und mit dem fünften Grad evaluiert, folglich kann man für Gleichung \ref{ableitung:eqn:aprox_5} den Fehler analog zu \ref{ableitung:eqn:error}
\begin{align}
\frac{f(x-2h) - 8f(x-h) + 8f(x-h) - f(x+2h)}{12h} = f'(x) - \frac{1}{30} f^{(5)} (x)h^{4}+R_n(x)
\label{ableitung:eqn:error}
\end{align}
Funktionen lassen sich somit mit mehr Stützstellen genauer approximieren, da der Rest $R$ mit steigender Taylor-Ordnung kleiner wird. Wenn aber die Terme höherer Ordnung schneller anwachsen als die Fakultät, so ist die Approximation nicht genau.
\subsection{Beispiele}
Es wurden wurden zwei Beispiele gesucht, welche spezielle Eigenschaften haben um zu verdeutlichen wie eine Approximation gut funktioniert und wie sie weniger gut funktioniert. Es existieren somit 2 Arten von Funktionen:
% TBD
%\begin{itemize}
%	\item {Funktionen mit verschwindendem Rest: \\
%		$\sin{x}$}
%	\item{Funktionen mit wachsendem Rest: \\
%		$\sqrt{(1-x^2)}}$}
%\end{itemize}
\label{ableitung:subsection:bonorum}

