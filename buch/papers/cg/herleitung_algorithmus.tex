\section{Herleitung des Algorithmus}
\label{cg:section:herleitung}
\rhead{Herleitung des Algorithmus}

\subsection{Intuition}
%TODO grafics "Abstieg auf Koordinatenachsen"

\subsection{Optimale Suchrichtung \label{cg:subsection:suchrichtung}}

Nun muss nur noch die optimale nächste Suchrichtung $d_{k+1}$ gefunden werden.
Damit jeder Schritt das Problem um eine Dimension reduziert, muss jede neue Suchrichtung orthogonal im Sinne von $A$ auf allen bisherigen Richtungen stehen. %TODO Grafik
Dies ist erreicht wenn das Residuum $r_k$ mithilfe des Gram-Schmidt-Orthogonalisierungsverfahren auf $d_k$ orthogonalisiert wird %TODO beweis dafür  (S. 80)
\begin{equation}
d_{k+1}
= 
r_{k+1} - \frac{\langle d_k , r_{k+1} \rangle_A}{\langle d_k , d_k \rangle_A} d_k.
\end{equation}

\subsection{Genügt Orthogonalisierung auf der letzten Richtung?}

