\section{Algorithmus\label{cg:section:voraussetzungen}}
\rhead{Algorithmus}

In diesem Abschnitt wird der ganze Algorithmus noch einmal formell aufgeschrieben.
Diese Formulierung des Algorithmus bildet die Grundlage für eine erfolgreiche Implementation.
Die Resultate einer solchen (einfachen) Implementation folgen im nächsten Abschnitt.

\begin{enumerate}
	\item Wähle initiales $x_1$ zufällig
	\item Berechne die erste Abstiegsrichtung als $d_1 = r_1 =  b-Ax_1$
	\item Berechne die optimale Schrittlänge  $ \alpha	= 	\displaystyle  \frac{\langle d_k , r_k \rangle}{\langle d_k , d_k \rangle_A} 
																			= \frac{d_k^T  r_k}{d_k^T A d_k }$
	\item Führe den Schritt aus $x_{k+1} = x_k + \alpha d_k$
	\item Berechne das neue Residuum $r_{k+1} = b-Ax_{k+1}$
	\item Falls $r_{k+1} = 0$: Beende den Algorithmus
	\item Berechne neue Abstiegsrichtung $d_{k+1} = d_{k+1}	= 	r_{k+1} - \displaystyle \frac{\langle d_k , r_{k+1} \rangle_A}{\langle d_k , d_k \rangle_A} d_k 
															= r_{k+1} - \displaystyle \frac{d_k^T A r_{k+1}}{d_k^T A d_k} d_k $
	\item Wiederhole ab Punkt 3.
\end{enumerate}