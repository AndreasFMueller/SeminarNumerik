\section{Voraussetzungen\label{cg:section:voraussetzungen}}
\rhead{Voraussetzungen}

Der CG-Algorithmus versucht ein (grosses) lineares Gleichungssystem der Form
\begin{equation}
Ax = b \quad x, b \in \mathbb{R}^N
\end{equation}
zu lösen.
Wir definieren folgende Voraussetzungen für die $N\times N$ Matrix $A$:
\begin{itemize}
	\item $A$ ist symmetrisch, d.h. $x^T A y = y^T A^T x = y^T A x$
	\item $A$ ist positiv definit, d.h. $x^T A x > 0$
\end{itemize}
Diese Voraussetzungen sind die Bedingung, damit das CG-Verfahren erfolgreich funktionieren kann.
Um eine schnelle Konvergenz zu erreichen, sind zudem gut konditionierte Matrizen von Vorteil.
Dies sind zum Beispiel dünn besetzte Matrizen mit wenigen Einträgen abseits der Diagonalen, wie sie häufig bei der Lösung von partiellen Differentialgleichung vorkommen.