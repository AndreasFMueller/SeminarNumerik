%
% problemstellung.tex -- Beispiel-File für die Beschreibung des Problems
%
% (c) 2020 Prof Dr Andreas Müller, Hochschule Rapperswil
%
\section{Folgerungen
\label{taylor:section:folgerungen}}
\rhead{Folgerungen}
Wie in Abbildung \ref{taylor:section:fig:FehlerRungeTaylor} ersichtlich ist, ist die Runge-Kutta Approximationen im Vergleich zur Taylor Approximation gleicher Ordnung bezüglich der Logistischen Funktion genauer.
Dies gilt allerdings nur für den ausgewerteten Bereich der Gleichtung als ganzes.
Da die Fehler Entwicklungen unterschiedlich verlaufen, sind beide Verfahren in gewissen Abschnitten besser.
Da man aber normalerweise die originale Funktion nicht kennt und somit auch der Fehler unbekannt ist, ziehen wir hier den defensiven Schluss und sagen, dass in diesem Fall die Runge-Kutte Approximation besser geeignet ist.

\subsection{Erklärung der Folgerung
\label{taylor:subsection:malorum}}
Wird eine Funktion nur in einem Punkt und seiner unmittelbaren Umgebung betrachtet, so kann er durch eine Funktion 1. Grades approximiert werden, bzw. einer Geraden mit der selben Steigung.
Die höheren Ordnungen spielen erst ab einem gewissen Abstand zum Auswertungspunkt eine Rolle.
Wenn wir also genug nahe bleiben, können wir eine Funktion nur durch die erste Ableitung ziemlich gut approximieren.
In diesem Fall sind die Taylor Approximation und die Runge-Kutta Approximation gleich gut, beziehungsweise die gleiche Formel.
Bei den höheren Ordnungen optimiert das Runge-Kutta Verfahren die Stelle von der es die Ableitung berechnet indem versucht wird einen räpresentativen Auswertungspunkt zu finden, welcher die Steigung zwischen zwei Auswertungspunkten bestmöglich trifft.
Das Taylorverfahren hingegen nimmt höhere Ableitungen dazu, welche (bei kleinen Schritten!) kaum eine Rolle spielen. 
Der Vorteil von der Taylor-Approximation ist, dass die Approxiamtionsfunktionen wesentlich länger mit der originalen Funktion übereinstimmen und somit bei einer grossen Schrittweite deutlich besser sein müssten.


