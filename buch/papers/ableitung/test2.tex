\subsection{Gradientenabstieg im neuronalen Netzwerk}
Um den Gradientenabstieg zu erklären reicht ein einfaches neuronales Netzwerk mit nur einem Neuron.
Ein Neuron entspricht der kleinsten Einheit eines neuronalen Netzwerks, welches in der Abbildung \ref{ableitung:fig:neuronal_network} dargestelt ist.
\begin{figure}
	\begin{center}
		%\documentclass[crop, tikz]{standalone}
%\usepackage{tikz}

\tikzstyle{inputNode}=[draw,circle,minimum size=17pt,inner sep=0pt]
\tikzstyle{stateTransition}=[-stealth, thick]

%\begin{document}
\begin{tikzpicture}
	\node[draw,circle,minimum size=25pt,inner sep=0pt] (x) at (0,0) {$\Sigma$ $\sigma$};

	\node[inputNode] (x0) at (-1, 2) {$\tiny +1$};
	\node[inputNode] (x1) at (-2, 0.75) {$\tiny x_0$};
	\node[inputNode] (x2) at (-2, 0) {$\tiny x_1$};
	\node[inputNode] (x3) at (-2, -0.75) {$\tiny x_2$};
	\node[inputNode] (xn) at (-2, -1.75) {$\tiny x_n$};

	\draw[stateTransition] (x0) to[out=0,in=90] node [midway, sloped, above] {$b^{l}$} (x);
	\draw[stateTransition] (x1) to[out=0,in=150] node [midway, sloped, above] {$w^{l}_0$} (x);
	\draw[stateTransition] (x2) to[out=0,in=180] node [midway, sloped, above] {$w^{l}_1$} (x);
	\draw[stateTransition] (x3) to[out=0,in=210] node [midway, sloped, above] {$w^{l}_2$} (x);
	\draw[stateTransition] (xn) to[out=0,in=240] node [midway, sloped, above] {$w^{l}_n$} (x);
	\draw[stateTransition] (x) -- (2,0) node [midway,above] {};
	\draw[dashed] (0,-0.43) -- (0,0.43);
	\node (dots) at (-2, -1.15) {$\vdots$};
	\node[inputNode, thick] (i1) at (5, 0.75) {};
	\node[inputNode, thick] (i2) at (5, 0) {};
	\node[inputNode, thick] (i3) at (5, -0.75) {};
	
	\node[inputNode, thick] (h1) at (7, 1.5) {};
	\node[inputNode, thick] (h2) at (7, 0.75) {};
	\node[inputNode, thick] (h3) at (7, 0) {$a^{2}_{3}$};
	\node[inputNode, thick] (h4) at (7, -0.75) {};
	\node[inputNode, thick] (h5) at (7, -1.5) {};
	
	\node[inputNode, thick] (o1) at (9, 0.75) {$a^{3}_{1}$};
	\node[inputNode, thick] (o2) at (9, -0.75) {};
	
	\draw[stateTransition] (4, 0.75) -- node[above] {$I_1$} (i1);
	\draw[stateTransition] (4, 0) -- node[above] {$I_2$} (i2);
	\draw[stateTransition] (4, -0.75) -- node[above] {$I_3$} (i3);
	
	\draw[stateTransition] (i1) -- (h1);
	\draw[stateTransition] (i1) -- (h2);
	\draw[stateTransition] (i1) -- (h3);
	\draw[stateTransition] (i1) -- (h4);
	\draw[stateTransition] (i1) -- (h5);
	\draw[stateTransition] (i2) -- (h1);
	\draw[stateTransition] (i2) -- (h2);
	\draw[stateTransition] (i2) -- (h3);
	\draw[stateTransition] (i2) -- (h4);
	\draw[stateTransition] (i2) -- (h5);
	\draw[stateTransition] (i3) -- (h1);
	\draw[stateTransition] (i3) -- (h2);
	\draw[stateTransition] (i3) -- (h3);
	\draw[stateTransition] (i3) -- (h4);
	\draw[stateTransition] (i3) -- (h5);
	
	\draw[stateTransition] (h1) -- (o1);
	\draw[stateTransition] (h1) -- (o2);
	\draw[stateTransition] (h2) -- (o1);
	\draw[stateTransition] (h2) -- (o2);
	\draw[stateTransition] (h3) -- (o1);
	\draw[stateTransition] (h3) -- (o2);
	\draw[stateTransition] (h4) -- (o1);
	\draw[stateTransition] (h4) -- (o2);
	\draw[stateTransition] (h5) -- (o1);
	\draw[stateTransition] (h5) -- (o2);
	
	\node[above=of i1, align=center] (l1) {Layer \\ $l=1$};
	\node[right=2.3em of l1, align=center] (l2) {Layer \\ $l=2$};
	\node[right=2.3em of l2, align=center] (l3) {Layer \\ $l=3$};
	
	\draw[stateTransition] (o1) -- node[above] {$O_1$} (10, 0.75);
	\draw[stateTransition] (o2) -- node[above] {$O_2$} (10, -0.75);
	
	\path[dashed, double, ultra thick, gray] (x.north) edge[bend left=0] (h5.north);
	\path[dashed, double, ultra thick, gray] (x.south) edge[bend right=0] (h5.south);
\end{tikzpicture}
%\end{document}
		\caption{Übersicht eines Neuronalen Netzwerks}
		\label{ableitung:fig:neuronal_network}
	\end{center}
\end{figure}
Um das verfahren der numerischen Ableitung zu erklären reicht ein Beispiel mit einem Neuron.
Im Grunde ist dies das einfachste implementierbare Netzwerk.
Dieses ein Neuron Netzwerk soll eine uns bekannte Abbildungsfunktion trainieren für welche wir einfach die Daten generieren können.
Daher verwendn wir als Beispiel ein digitales Not-Gate \ref{ableitung:fig:not_gate} die genauera Abbildung von Input $x$ und Output $y$ kann der Tabelle entnommen werden.
\begin{figure}
	\begin{center}

		\begin{tikzpicture}
		
		\node (x) at (0, 1) {$x$};
		\node (y) at (2, 1) {$y$};
		
		\node[not gate US, draw] at ($(x) + (0.8, 0)$) (notx) {};
		
		\draw (x) -- (notx.input);
		\draw (notx.output) -- (y);
		\end{tikzpicture}	
		\begin{tabular}{cc}
			\hline
			$x$ & $y$ \\
			\hline
			0 & 1 \\ 
			1 & 0 \\ 
			\hline
			
		\end{tabular}

		\caption{Einfaches Not-Gate mit zugehöriger Logik Tabelle}
		\label{ableitung:fig:not_gate}
	\end{center}
\end{figure}

Das einzelne Neruron umfasst 2 Paramter ($w$, $b$) und eine Aktivierungsfunktion $\sigma$. Das Trainieren des neuronalen Netzwerks geschieht nun in 3 Schritten.
\begin{itemize}
	\item \textbf{Initialisierung}
	\item \textbf{Forward Propagation:} Der Input $x$ wird durch das neuronale Netzwerk gerechnet und ein Output $\hat{y}$ wird erzeugt.
	\item \textbf{Loss Berechnung:} Die Loss-Funktion berechnet die Abweichung zwischen $\hat{y}$ und dem von uns zu erwartendem Output $y$.
	\item \textbf{back prop?}
\end{itemize}

\subsection{Lernen als Optimierungsproblem}
Ein Netzwerk und der Traininsprozess kann mathematisch formalisiert werden, um die Notation zu vereinfachen halten wir folgende Parameter und Komponenten fest:
\begin{itemize}
	\item{\textbf{$x$:} Input ins Neuronale Netzwerk}
	\item{\textbf{$y$:} Erwarteter Output aus dem Neuronalen Netzwerk gem. dem Trainingsdatensatz}
	\item{\textbf{$\hat{y}$:} Effektiver gerechneter Output aus den neuronalen Netzwerk}
	\item{\textbf{Loss Funktion $\Lb$:}  Die Funktion welche den Fehler zwischen dem gerechenten und tatsächlechen Output berechnet, als Bsp MSE: $\sum_{i=0}^{n}\left( L(y,\hat{y}) \right)$ }
\end{itemize}
%			L &= MSE(y,\hat{y}) = \left(\hat{y}-y\right)^{2}