%
% problemstellung.tex -- Beispiel-File für die Beschreibung des Problems
%
% (c) 2020 Prof Dr Andreas Müller, Hochschule Rapperswil
%
\section{Folgerungen
\label{ableitung:section:folgerungen}}
\rhead{Folgerungen}
Die Resultate zeigen, dass anhand eines einfachen Netzwerk die hier vorgestellte Implementation, für die Berechnung des Gradienten, funktioniert.
Im direkten Vergleich zwischen dem FDM- und dem Backpropagation-Algorithmus, hat ersterer früher zu einem kleineren Fehler konvergiert. Vorausgesetzt, dass die Parameter gut gewählt wurden. Nach der gleichen Anzahl an Iterationen lag der Fehler der FDM-Implementation um einen Faktor $10^3$ niedriger. Dabei spielt die Anzahl an Stützstellen eine wichtigere Rolle, als die Wahl des Abstandes $h$. Ohne an dieser Stelle zu stark im Detail auf die Fehlerrechnung einzugehen, kann die folgende Intuition dieses Phänomen erklären.

Die meisten Aktivierungsfunktionen im neuronalen Netzwerk haben einen Abbildungsbereich zwischen $[-1, 1]$ oder $[0, 1]$. Die Kettenregel hat zur Folge, dass für die Berechnung des Fehlers in den vorderen Schichten diese kleine Zahlen oft miteinander multipliziert werden müssen. Der Fehler in der Gradientenberechnung nimmt zu, da das numerische Rauschen durch die Gleitkomma-Darstellung von Schicht zu Schicht grösser wird. Dieses Phänomen wird auch als \textit{Vanishing Gradient Problem} bezeichnet und ist ein bekanntes Problem, welches bei Netzwerken mit hoher Tiefe besonders stark ausgeprägt ist. Durch den Verzicht der Rückwärtsberechnung mittels Kettenregel entfällt dieser Fehler bei der FDM vollständig. 

Weiter verbessert eine höhere Anzahl an Stütztstellen die Genauigkeit des Gradienten, da gewisse Toleranzen in der Gleitkomma-Darstellung weniger stark ins Gewicht fällt. Dies kann wie folgt erklärt werden: Das Zeichnen einer Parabel durch drei Punkte mit gewisser Unschärfe ist deutlich ungenauer, als das Zeichnen einer Parabel durch sieben Punkte mit ähnlicher Unschärfe. Als Unschärfe ist die Genauigkeit der Position der Punkte gemeint, durch welche die gezeichnete Parabel verlaufen soll.

Anhand eines Experiments wurde gezeigt, dass die FDM durchaus Potential zum Trainieren von tiefen neuronalen Netzwerken hat. An diesem Punkt wäre es ausserordentlich spannend gewesen tiefere Netzwerke zu trainieren. An einem etwas älteren aber gut dokumentierten Netzwerk (LeCun Net5) sollte dies probiert werden. Das Netzwerk wurde in den achziger Jahren verwendet um Handschriften auf dem bekannten MNIST-Datensatz zu erkennen. Es bietet eine gute Grundlage um qualitative Aussagen zu machen. 

Die erfreuliche Genauigkeit, die mit der FDM erreicht werden konnte, hat aber auch ihren Preis. Das Training mittels Backpropagation Algorithmus war in wenigen Minuten beendet, während das Training mittels FDM nach mehreren Tagen noch nicht genügend Iterationen durchlaufen hat um eine abschliessende qualitative Aussage zu machen. Es war nie Ziel dieses Projektes den Algorithmus soweit zu optimieren, dass ein Vergleich einer solchen Struktur möglich gewesen wäre. Da die Laufzeit um einen sehr grossen Faktor sich unterschieden hat, wurde auf eine Optimierung verzichtet. 