\subsection{Der Backpropagation-Algorithmus}

%5. 25.2.4 Backprop: Dieser Abschnitt ist immer noch sehr schwer verständlich. Im Moment ist für mich der grösste Stolperstein, dass Sie nicht erklären, was z eigentlich ist. Auch ist der erste Absatz von 25.2.4 über eine halbe Seite lang, da braucht man einen ziemlich langen Atem, das durchzustehen. Oft ist das auch ein Hinweis darauf, dass man die Ideen noch nicht genügend aufgeschlüsselt hat. Vielleicht sollten Sie den Absatz unterteilen:
%
%1. Teil/Absatz: Das Problem klären. Gesicht ist eine Parameter-Änderung, die eine Output-Änderung hervorruft, die den Fehler reduziert. Statt den Fehler in einem Schritt zu reduzieren, reduziert man ihn Schrittweise, indem man in jedem Layer versucht, die Parameter so zu ändern, dass der Fehler reduziert wird. Dies führt auf einen Fehler in der vorangegangenen Schicht, den man im nächsten Schritt reduzieren will.
%
%2. Teil/Absatz: Bezeichnungen. Da man jetzt weiss, was der Plan ist, kann man sagen, wie man diese Fehler und Fehleränderungen bezeichnen will. Damit ist dann klar, was \delta und z bedeuten.
%
%3. Teil/Absatz: Berechnung der Änderung in jedem Layer. Hier kann man jetzt die partiellen Ableitungen und das Hadamard-Produkt bringen.
%
%4. Teil/Absatz: Backprop, da haben Sie ja schon einen eigenen Absatz.
Die analytische Berechnung aller partiellen Ableitung nach allen Parametern 
ist nicht wirklich praktikabel.
Es sind einfach zu viele Dimensionen, welche mittels der Kettenregel
abgeleitet werden müssen.
Weiter können die Zusammenhänge zwischen
den Schichten nicht wirklich genutzt werden, da nach jedem Parameter
einzeln abgeleitet wird.

Aus diesem Grund ist der Backpropagation-Algorithmus sehr verbreitet.
Er löst genau dieses Problem, in dem er den Fehler pro Schicht berechnet und dann den Fehler der davor liegende Schicht davon abhängig macht.
Dies spart einen Teil des Rechenaufwandes der Gradientenberechnung und macht den Algorithmus so einiges effizienter.

Das Ziel des Backpropagation-Algorithmus ist nun die partiellen Ableitungen $\partial \Lb / \partial w^l_{jk}$ und $\partial \Lb / \partial b^l_j$ für den Fehler eines neuronalen Netzwerks der Form
\begin{equation}
\Lb(y_{i}, \sigma^{l}(W^{l}\sigma^{l-1}(W^{l-1}\cdots \sigma^{1}(W^{1}x_i+b^{1}) + b^{l-1} \cdots ) )
\end{equation}
zu finden, wobei $W^{n}$ und $b^{n}$ jeweils die Gewichte im $n$-ten Layer des neuronalen Netzwerks darstellen.
Dies gilt unter der Annahme, dass die Fehlerfunktion $\Lb$ einen Skalar
als Resultat liefert und die einzelnen Schichten $l$ jeweils nur mit
ihrer benachbarten Schicht $l+1$ verbunden sind.

Für den Zusammenhang der Layer führen wir die Notation
\begin{equation}
	\begin{split}
		a^l_j &= \sigma(z^l_j) \\
		z^l_j &= \sum\limits_{k}w^{l}_{jk} \cdot a^{l-1}_{k}+b^{l}_{j}
	\end{split}
\end{equation}
ein.
$a$ ist bereits aus der Gleichung~\eqref{ableitung:eqn:simple_neuron_eqn}
bekannt, $z$ ist im Grunde der gewichtete Input in ein Neuron
ohne die Aktivierungsfunktion $\sigma$.
Diese Notation vereinfacht die weitere Herleitung.

Die Änderung des Fehlers in einem einzelnen Neuron
\begin{equation}
	\delta^{l}_{j} = \frac{\partial \Lb}{\partial a^{l}_{j}} \sigma'(z^l_j)
\end{equation}
lässt sich Ausdrücken als Ableitung der Fehlerfunktion $\Lb$ nach dem jeweiligen Neuron.
In der letzten Schicht mit Index $L$ ({\em last}) gilt somit in
Matrixschreibweise
\begin{equation}
	\delta^{L} = \nabla_a \Lb \odot \sigma'(z^L) \text{,}
\end{equation}
wobei $\odot$ der komponentenweise Multiplikation der Matrix (Hadamard
Produkt) entspricht.
\index{Hadamard-Produkt}%
Um den Fehler zu berechnen, legen wir uns für die Funktion $\Lb$ auf
die quadratische Fehlerfunktion
\begin{equation}
	\Lb = \frac{1}{2} \sum_j \left( \hat{y} - y \right)^2 \text{.}
\end{equation}
fest, auch bekannt als 
$L_2$ Mean-Square-Error.
Somit resultiert nach der partiellen Ableitung von $\Lb$ und
einsetzten von $a^L$ die Gleichung
\begin{equation}
	\delta^{L} = (\hat{y} - a^L) \odot \sigma'(z^L).
\end{equation}
$\delta^L$ ist somit ein Ausdruck der Änderung des Fehlers in der
letzten Schicht.

Als nächstes muss nun rückwärts gerechnet --- back propagiert --- werden.
Die Änderung des Fehlers
\begin{equation}
\delta^{l} = ((W^{l+1} \delta^{l+1})^T \odot \sigma'(z^l)
\end{equation}
beschreibt nun den Zusammenhang zwischen den Schichten $l+1$ und $l$, was an den Ausdrücken $\delta^{l+1}$ und $ \delta^{l}$ abzulesen ist.
Die Transponierte der Matrix $W^{l+1}$ kann man sich intuitiv vorstellen, als das 'Rückwärtsrechnen' des gewichteten Fehlers, da $W^{l+1} \cdot (W^{l+1})^T = I$, wobei $I$ für die Einheitsmatrix steht.
Dies ermöglicht die Änderung des Fehlers pro Neuron in der davor liegenden Schicht zu bestimmen.
Abschliesend müssen noch die Gradienten
\begin{equation}
\begin{split}
\frac{\partial \Lb}{\partial b} & = \delta \\
\frac{\partial \Lb}{\partial w} & = a^{l-1}_{k} \delta^{l}_j = a_{\text{in}} \delta_{\text{out}}
\end{split}
\end{equation}
berechnet werden.
Das Resultat wird verwendet, um mittels Gradientenabstieg die neuen
inneren Parameter $w_\text{new}$ und $b_\text{new}$ zu bestimmen.

Die Herleitung zeigt sehr schön, dass eine hohe Netztiefe und die
rekursive Eigenschaft des Algorithmus viele Multiplikationen als
Konsequenz hat.
Dies hat zur Folge, dass durch numerische Effekte die Gradientenberechnung rauschbehaftet sein kann, weiter sind die vorderen Schichten viel stärker Betroffen, da mehr Multiplikationen nötig sind um den Gradienten zu bestimmen.
Auf dieses Problem wird in den folgenden Kapiteln mit einem Gedankenexperiment eingegangen.
\index{Gedankenexperiment}%

