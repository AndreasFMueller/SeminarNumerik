%
% einleitung.tex -- Beispiel-File für die Einleitung
%
% (c) 2020 Prof Dr Andreas Müller, Hochschule Rapperswil
%
\section{Einleitung\label{ableitung:section:einleitung}}
\rhead{Einleitung}
Dieses Kapitel befasst sich mit der numerischen Ableitungim Speziellen mit der Finite Differenzen Methode (FDM).
Bei dieser Methode handelt es sich um ein Verfahren zur Bestimmung der Ableitung. Die Methode war schon Gauss und Boltzmann bekannt, war aber bis in die 1940er nicht sehr verbreitet um angewandte Probleme zu lösen. Das Verfahren ist mathematisch sehr einfach zu erklären und ist ebenfalls einfach umzusetzen. Die Einfachheit der Methode ermöglicht ein schwieriges Beispiel zu verwenden, da die Umsetzung in Code unkompliziert ist.

Als Einführung der Methode wird aus diesem Grunde der Gradient im neuronalen Netzwerk berechnet, welcher von zentraler Rolle für den Lernprozess ist.
Neuronale Netzwerke sind durch die kostengünstig verfügbare Computerleistung in den letzten Jahrzehnten sehr populär geworden. Moderne Forschung in diesem Bereich konzentriert sich auf eine Vielzahl von Problemen, meistens im Bereich der Bild-, Signal- oder Mustererkennung, aber auch Sprachsynthese und Übersetzung können von neuronalen Netzwerken profitieren. Eine weitere Konsequenz der günstigen Computerleistung ist, dass die Netzwerke immer grösser werden und durch die wachsende Tiefe auch anfälliger auf numerische Probleme sind.