\section{Das Problem der numerischen Ableitung}
\rhead{Das Problem der numerischen Ableitung}
Wir gehen von einer in einem Intervall $I=[a,b]$ definierten differenzierbaren
Funktion $f\colon[a,b]\to\mathbb R$ aus. 
Zu bestimmen ist die Ableitung von $f$ in einem Punkt $x\in[a,b]$.
Der Differenzenquotient
\[
f'(x)
\approx
\frac{f(x+h)-f(x)}{h}
\]
ist die naheliegende Approximation, doch für kleines $h$ werden $f(x+h)$ und
$f(x)$ fast gleich gross, die Differenz ist daher stark verschmiert.
Grosses $h$ wiederum beschränkt die Genauigkeit, denn die Taylor-Entwicklung
liefert für den Differenzenquotienten
\index{Differenzenquotient}%
\[
f'(x)
\approx
\frac{f(x) + hf'(x) + h^2f''(x)/2 + O(h^3) - f(x)}{h}
=
f'(x) + h\frac{f''(x)}2 + O(h^2),
\]
für nicht allzu kleines $h$ wird der Einfluss des $f''(x)$-Terms
merklich.
Es muss also angestrebt werden, genaue Resultate auch bei relativ grossem
$h$ bekommen zu können.

Eine erste Verbesserung kann mit dem symmetrischen Differenzenquotienten
\[
f'(x)
\approx
\frac{f(x+\frac{h}2) -f(x-\frac{h}2)}{h}
\]
erreicht werden.
Setzt man wieder die Taylor-Reihe ein, erhält man
\begin{align*}
f'(x) &\approx
\frac{1}{h} \biggl(
f(x) + \frac{h}2 f'(x) + \frac{h^2}{8} f''(x) + \frac{h^3}{48}f'''(x) + O(h^4)
\\
&\qquad\qquad
-
\biggl(
f(x) - \frac{h}2 f'(x) + \frac{h^2}{8} f''(x) - \frac{h^3}{48}f'''(x) + O(h^4)
\biggr)
\biggr)
\\
&=
f'(x) + \frac{h^2}{24}f'''(x) + O(h^3).
\end{align*}
Danke der Symmetrie ist der $f''(x)$-Term verschwunden, der Fehler ist nur
noch von Ordnung $h^2$.

Den symmetrischen Differenzenquotienten kann man auch als Wert eines
Interpolationspolynoms ansehen.
\index{symmetrische Differenz}%
Die beiden Stützstellen $x_{0,1}=x\pm \frac{h}2$ definieren die lineare
Interpolationsfunktion
\begin{align*}
p(x)
&=
f(x_0) l_0(x) + f(x_1) l_1(x)
=
f(x_0) \frac{x-x_1}{x_0-x_1}
+
f(x_1) \frac{x-x_0}{x_1-x_0}
=
-f(x_0) \frac{x-x_1}{h}
+
f(x_1) \frac{x-x_0}{h}
\\
&=
\frac{f(x_1) - f(x_0)}{h}
x
+
\text{const}.
\intertext{%
Die symmetrische Differenz ist daher auch die Ableitung}
p'(x)
&=
\frac{f(x_1)-f(x_0)}{h}
\end{align*}
des Interpolationspolynoms.
Es liegt daher nahe, die Ableitung $f'(x)$ als Ableitung eines
Interpolationspolynoms höherer Ordnung von $f(x)$ zu berechnen.

