\section{Ableitungsverfahren
\label{section:interdiff:ableitung}}
\rhead{Ableitungsverfahren}
Wir wenden die Formeln
\eqref{interdiff:ableitung}
und
\eqref{interdiff:koeffizienten}
zunächst auf den Fall zweier Stützstellen $x_0=\xi-\frac{h}2$ und
$x_1=\xi+\frac{h}2$ an.
Es folgt
\begin{align*}
l_0(x)
&=
-\frac{1}{h} (x-x_1)
&
l_0'(x)
&=
-\frac1h \cdot 1
&
l_0'(\xi)
&=
-\frac1h \cdot 1
\\
l_1(x)
&=
\frac1{h} (x-x_0)
&
l_1'(x)
&=
\frac1h \cdot 1
&
l_1'(\xi)
&=
\frac1h \cdot 1
\\
f'(x)
&\approx
\frac1h(- f(x_0)+f(x_1))
\rlap{$\displaystyle=
= \frac{f(x+\frac{h}2)-f(x-\frac{h}2)}{2},$}
\end{align*}
der bereits bekannte symmetrische Differenzenquotient.

Für $n=2$ und die Stützstellen $x_0=x-h$, $x_1=x$ und $x_2=x+h$ erhalt man
dagegen
\begin{align*}
l_0(x)
&=
\frac{1}{h\cdot 2h} (x-x_1)(x-x_2)
&
l_0'(x)
&=
\frac{1}{2h^2}(2x-x_2-x_1)
&
l_0'(\xi)
&=
\frac{1}{2h^2}(2\xi -(\xi+h) -\xi)
=
-\frac{1}{2h}
\\
l_1(x)
&=
\frac{1}{h^2} (x-x_0)(x-x_2)
&
l'_1(x)
&=
\frac{1}{h^2} (2x-x_0-x_2)
&
l'_1(\xi)
&=
\frac{1}{h^2}(2\xi-(\xi-h)-(\xi+h)) = 0
\\
l_2(x)
&=
\frac{1}{2h\cdot h} (x-x_0)(x-x_1)
&
l'_2(x)
&=
\frac{1}{2h^2} (2x-x_0-x_1)
&
l'_2(\xi)
&=
\frac{1}{2h^2} (2\xi-(\xi-h)-\xi)
=
\frac{1}{2h}
\\
f'(\xi)
&\approx
\frac{1}{2h}(f(x_2) - f(x_0))
\rlap{$\displaystyle= \frac{f(\xi+h)-f(\xi-h)}{2h},$}
\end{align*}
also wieder eine symmetrische Differenz.
Die zusätzliche Stützstelle in der Mitte bringt keinen Genauigkeitsgewinn.

\begin{table}
\centering
\renewcommand\arraystretch{2}
\setcounter{MaxMatrixCols}{20}
\begin{tabular}{>{$}c<{$}|>{$\displaystyle}c<{$}>{$\displaystyle}c<{$}>{$\displaystyle}c<{$}>{$\displaystyle}c<{$}>{$\displaystyle}c<{$}>{$\displaystyle}c<{$}>{$\displaystyle}c<{$}>{$\displaystyle}c<{$}>{$\displaystyle}c<{$}}
\text{Stützstellen}&
-4&
-3&
-2&
-1&
0&
1&
2&
3&
4
\\
\hline
3&
&
&
&
-\frac{1}{2h} &
0 &
\frac{1}{2h} &
&
\\
5&
&
&
\frac{1}{12h} &
-\frac{2}{3h} &
0 &
\frac{2}{3h} &
-\frac{1}{12h} &
&
\\
7&
&
\frac{1}{60h} &
\frac{3}{20h} &
\frac{3}{4h} &
0 &
\frac{3}{4h} &
\frac{3}{20h} &
\frac{1}{60h} &
\\
9&
\frac{1}{280h} &
-\frac{4}{105h} &
\frac{1}{5h} &
-\frac{4}{5h} &
0 &
\frac{4}{5h} &
-\frac{1}{5h} &
\frac{4}{105h} &
-\frac{1}{280h}
\\[3pt]
\hline
\text{Taylor}&
&
&
\frac{1}{12h}&
-\frac{8}{12h}&
0&
\frac{8}{12h}&
-\frac{1}{12h}&
&
\end{tabular}
\caption{Koeffizienten für die Berechnung der Ableitung aus den Stützstellen
$x_j=x+jh$ mit $-n\le j\le n$.
Die Koeffizienten, die aus der Taylorreihe gewonnen wurden sind in der letzten
Zeile dargestellt, sie stimmen mit den Koeffizienten überein, die sich für
fünf Stützstellen aus der Interpolation ergeben.
\label{interdiff:table:ungerade}}
\end{table}

\begin{table}
\centering
\renewcommand\arraystretch{2}
\setcounter{MaxMatrixCols}{20}
\begin{tabular}{>{$}c<{$}|>{$\displaystyle}c<{$}>{$\displaystyle}c<{$}>{$\displaystyle}c<{$}>{$\displaystyle}c<{$}>{$\displaystyle}c<{$}>{$\displaystyle}c<{$}>{$\displaystyle}c<{$}>{$\displaystyle}c<{$}>{$\displaystyle}c<{$}}
\text{Stützstellen}&
-7&
-5&
-3&
-1&
1&
3&
5&
7
\\
\hline
2&
&
&
&
-\frac{1}{h}&
\frac{1}{h}&
&
&
\\
4&
&
&
\frac1{24h}&
-\frac{9}{8h}&
\frac{9}{8h}&
-\frac{1}{24h}&
&
\\
6&
&
-\frac{3}{640h}&
\frac{25}{384h}&
-\frac{75}{64h}&
\frac{75}{64h}&
-\frac{25}{384h}&
\frac{3}{640h}&
\\
8&
\frac{5}{7168h}&
-\frac{49}{5120h}&
\frac{245}{3072h}&
-\frac{1225}{1024h}&
\frac{1225}{1024h}&
-\frac{245}{3072h}&
\frac{49}{5120h}&
-\frac{5}{7168h}
\end{tabular}
\caption{Koeffizienten für die Berechnung der Ableitung aus den Stützstellen
$x+j\frac{h}2$.
\label{interdiff:table:gerade}}
\end{table}

Für grössere Anzahlen von Stützstellen wird die Berechnung etwas mühsam
und kann mit Computeralgebra vereinfacht werden.
In den Tabellen \ref{interdiff:table:ungerade} und \ref{interdiff:table:gerade}
sind die Koeffizienten zusammengestellt, die sich für eine ungerade
bzw.~gerade Anzahl von Stützstellen zusammengestellt.
\index{Maxima}%

In der letzten Zeile von Tabelle~\ref{interdiff:table:ungerade} sind
auch die Koeffizienten des in Kapitel~\ref{chapter:ableitung} aus
dem Taylor-Polynom abgeleiteten Verfahrens aufgelistet.
Nicht ganz überraschend stimmen Sie mit dem Verfahren überein, welches
hier aus dem Interpolationspolynom abgeleitet wurde.
\index{Taylor-Polynom}%


