\section{Ableitungsverfahren}
\rhead{Ableitungsverfahren}
Wir wenden die Formeln
\eqref{interdiff:ableitung}
und
\eqref{interdiff:koeffizienten}
zunächst auf den Fall zweier Stützstellen $x_0=\xi-\frac{h}2$ und
$x_1=\xi+\frac{h}2$ an.
Es folgt
\begin{align*}
l_0(x)
&=
-\frac{1}{h} (x-x_1)
&
l_0'(x)
&=
-\frac1h \cdot 1
\\
l_1(x)
&=
\frac1{h} (x-x_0)
&
l_1'(x)
&=
\frac1h \cdot 1
\\
f'(x)
&\approx
\frac1h(- f(x_0)+f(x_1)) = \frac{f(x+\frac{h}2)-f(x-\frac{h}2)}{2},
\end{align*}
der bereits bekannte symmetrische Differenzenquotient.

Für $n=2$ und die Stützstellen $x_0=x-h$, $x_1=x$ und $x_2=x+h$ erhalt man
dagegen
\begin{align*}
l_0(x)
&=
\frac{1}{h\cdot 2h} (x-x_1)(x-x_2)
&
l_0'(x)
&=
\frac{1}{2h^2}(2x-x_2-x_1)
&
l_0'(\xi)
&=
\frac{1}{2h^2}(2\xi -(\xi+h) -\xi)
=
-\frac{1}{2h}
\\
l_1(x)
&=
\frac{1}{h^2} (x-x_0)(x-x_2)
&
l'_1(x)
&=
\frac{1}{h^2} (2x-x_0-x_2)
&
l'_1(\xi)
&=
\frac{1}{h^2}(2\xi-(\xi-h)-(\xi+h)) = 0
\\
l_2(x)
&=
\frac{1}{2h\cdot h} (x-x_0)(x-x_1)
&
l'_2(x)
&=
\frac{1}{2h^2} (2x-x_0-x_1)
&
l'_2(\xi)
&=
\frac{1}{2h^2} (2\xi-(\xi-h)-\xi)
=
\frac{1}{2h}
\\
f'(\xi)
&\approx
\frac{1}{2h}(f(x_2) - f(x_0))
\rlap{$\displaystyle= \frac{f(\xi+h)-f(\xi-h)}{2h},$}
\end{align*}
also wieder eine symmetrische Differenz.
Die zusätzliche Stützstelle in der Mitte bringt keinen Genauigkeitsgewinn.


