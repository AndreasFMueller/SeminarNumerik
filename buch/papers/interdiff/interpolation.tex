\section{Die Ableitung des Interpolationspolynoms}
\rhead{Die Ableitung des Interpolationspolynoms}
In Kapitel~\ref{chapter:interpolation} wurde für das Interpolationspolynom
der Ausdruck
\[
p(x)
=
\sum_{j=0}^n f(x_j) l_j(x)
\qquad\text{mit}\quad
l_j(x)
=
\frac{(x-x_0)(x-x_1)\cdots\widehat{(x-x_j)}\cdot (x-x_n)}{(x_j-x_0)(x_j-x_1)\cdots\widehat{(x_j-x_j)}\cdot (x_j-x_n)}
\]
gefunden.
Die Ableitung seine Ableitung ist
\[
p'(x)
=
\sum_{j=0}^n f(x_j) l'_j(x).
\]
Die Ableitungen der speziellen Interpolationspolynome $l_j(x)$ kann
man ebenfalls direkt berechnen:
\begin{equation}
l_j'(x)
=
\frac{1}{(x_j-x_0)(x_j-x_1)\cdots\widehat{(x_j-x_j)}\cdot (x_j-x_n)}
\sum_{k\ne j} (x-x_0)\cdots \widehat{(x-x_k)}\cdots\widehat{(x-x_j)}\cdots (x-x_n).
\label{interdiff:koeffizienten}
\end{equation}
Durch Einsetzen der Stütztstellen lassen sich die $l_j(x)$ aus
Formel~\eqref{interdiff:koeffizienten} direkt berechnen, so dass
man die Näherungsformel 
\begin{equation}
f'(x) = \sum_{j=0}^n f(x_j) l'_j(x)
\label{interdiff:ableitung}
\end{equation}
erhält.

Für äquidistante Stützstellen mit Abstand $h$, alle Differenzen im
Nenner von $l_j(x)$ sind Vielfache von $h$.
Bei $n$ Stützstellen geben die Faktoren vor dem Faktor $(x_j-x_j)$, der
weggelassen muss, einen Beitrag $jh\cdot (j-1)h \cdot h$ im Nenner,
die Faktoren danach liefern den Beitrag $h\cdot 2h\cdot \dots\cdot (n-j)h$.
Der Nenner ist daher
\[
(-1)^j
j!\cdot (n-j)! h^n,
\]
die Ableitung kann jetzt als
\begin{equation}
l_j'(x)
=
\frac{(-1)^j}{j!\cdot (n-j)! h^n}
\sum_{k\ne j} (x-x_0)\cdots \widehat{(x-x_k)}\cdots\widehat{(x-x_j)}\cdots (x-x_n).
\label{interdiff:koeffizienten2}
\end{equation}
geschrieben werden.

Die Interpolationspolynome $l_j(x)$ können leicht mit einem
Computeralgebrasystem wie Maxima berechnet werden.

