%
% einleitung.tex -- Beispiel-File für die Einleitung
%
% (c) 2020 Prof Dr Andreas Müller, Hochschule Rapperswil
%
\section{Einleitung\label{qr:section:einleitung}}
Wie in Abschnitt \ref{buch:section:qr} schon genauer betrachtet, kann eine Matrix $A$ mit $l$ Zeilen und $n$ linear unabhängigen spalten ($l\ge n$) in zwei Matrizen zerlegt werden:
\begin{equation*}
A=QR.
\end{equation*}
$Q$ ist dabei wieder ein $l\times n$-Matrix mit, diesmal aber orthonormierten Spalten.
$R$ ist eine ober Dreiecksmatrix der Grösse $n\times n$.

\subsection{Anwedungsbeispiel Least-Squares}
Die $QR$-Zerlegung kann unter anderem beim lösen von Gleichungssystemen angewendet werden.
Ein in den Parametern lineares Gleichungssystem
\begin{equation}
Ax=b\label{qr:sle}
\end{equation}
hat in praktischen Fällen keine Lösung, wenn die $l\times n$ Matrix mehr Zeilen als Spalten hat, bzw. überdefiniert ist.
Mit der Methode der kleinsten Quadrate kann aber ein Vektor $\hat{x}$ gefunden werden welcher die $L^2$-Norm
\begin{equation*}
||A\hat{x}-b||_2
\end{equation*}
minimiert.
Dieses Problem kann geometrisch gelöst werden, indem man es wie folgt betrachtet:
\begin{equation*}
A\hat{x}=b-b_{\perp},
\end{equation*}
wobei $b_{\perp}$ die zu den Spalten in $A$ orthogonale Komponente von $b$ ist.
Multipliziert am beide Seiten mit $A^T$ erhält man
\begin{equation}
A^TA\hat{x}=A^Tb-\underbrace{A^Tb_{\perp}}_{=0}
\quad\Leftrightarrow\quad A^TA\hat{x}=A^Tb \label{qr:ls1}
\end{equation}
Wobei der letzte Term verschwindet, da alle Spalten von $A$ und $b_{\perp}$ orthogonal zueinander stehen und somit alle Skalarprodukte verschwinden.
Das Matrixprodukt $A^TA$ ist unter Umständen aufwändig zu berechnen.
Geht man nun aber den Umweg über die $QR$-Zerlegung kann die Gleichung \ref{qr:ls1} umgeschrieben werden:
\begin{equation*}
(QR)^TQR\hat{x}=(QR)^Tb
\end{equation*}
was sich zu 
\begin{equation}
R\hat{x}=Q^Tb\label{qr:ls2}
\end{equation}
vereinfacht.
