%
% einleitung.tex -- Beispiel-File für die Einleitung
%
% (c) 2020 Prof Dr Andreas Müller, Hochschule Rapperswil
%
\section{Einleitung\label{qr:section:einleitung}}
Wie in Abschnitt \ref{buch:section:qr} schon genauer betrachtet, kann eine Matrix $A$ mit $l$ Zeilen und $n$ linear unabhängigen Spalten ($l\ge n$) in zwei Matrizen zerlegt werden:
\begin{equation*}
A=QR.
\end{equation*}
$Q$ ist dabei wieder eine $l\times n$-Matrix, diesmal aber mit orthonormierten Spalten.
$R$ ist eine obere Dreiecksmatrix der Grösse $n\times n$.

\subsection{Anwendungsbeispiel Least-Squares\label{qr:section:ls}}
Die $QR$-Zerlegung kann unter anderem beim Lösen von Gleichungssystemen angewendet werden.
Ein lineares Gleichungssystem
\begin{equation}
Ax=b\label{qr:sle}
\end{equation}
hat in praktischen Fällen keine Lösung, wenn die $l\times n$-Matrix $A$ mehr Zeilen als Spalten hat, bzw. überdefiniert ist.

Mit der Methode der kleinsten Quadrate kann aber ein Vektor $\hat{x}$ gefunden werden welcher die $L^2$-Norm
\begin{equation*}
||A\hat{x}-b||_2
\end{equation*}
minimiert.
Dies ist das gleiche, wie wenn man einen Lösungsvektor $\hat{x}$ sucht, sodass
\begin{equation*}
A\hat{x}=b-b_{\perp}.
\end{equation*}
$b_{\perp}$ ist hier die zu den Spalten in $A$ orthogonale Komponente von $b$ und ist folglich der Teil von $b$, welcher nicht linear abhängig von den Spalten von $A$ ist.

Werden beide Seiten mit $A^T$ multipliziert, erhält man
\begin{equation}
A^TA\hat{x}=A^Tb-\underbrace{A^Tb_{\perp}}_{\displaystyle=0}
\quad\Leftrightarrow\quad A^TA\hat{x}=A^Tb \label{qr:ls1},
\end{equation}
wobei der letzte Term verschwindet, da alle Spalten von $A$ und $b_{\perp}$ orthogonal zueinander stehen und somit alle Skalarprodukte verschwinden.
Das Matrixprodukt $A^TA$ ist unter Umständen aufwändig zu berechnen und muss auch noch invertiert werden, um $\hat{x}$ zu erhalten.

Geht man nun aber den Umweg über die $QR$-Zerlegung, bzw. $A=QR$ kann die Gleichung \ref{qr:ls1} umgeschrieben werden:
\begin{equation*}
(QR)^TQR\hat{x}=(QR)^Tb \quad \Leftrightarrow \quad R^T\underbrace{Q^TQ}_{\displaystyle=E}R\hat{x}= R^TQ^Tb \quad \Leftrightarrow \quad
R^TR\hat{x}=R^TQ^Tb,
\end{equation*}
wobei $Q^TQ=E$, da $Q$ nach Definition orthogonal ist.
Die Determinante von $R^T$ ist
\begin{equation*}
\det(R^T)=\det(R)=\det
\begin{pmatrix}
r_{11}&r_{12}&\cdots &r_{1n}\\
0     &r_{22}&\cdots &r_{2n}\\
\vdots&\vdots&\ddots&\vdots\\
0&0&\cdots&r_{nn}
\end{pmatrix}=\prod_{k=1}^{n}r_{kk}.
\end{equation*}
Diese verschwindet nur, wenn eines der Diagonalelemente verschwindet.
Dies würde aber bedeuten, dass die Spalten von $Q$ und somit auch $A$ linear abhängig voneinander sind, womit wiederum die Voraussetzungen für die $QR$-Zerlegung nicht gegeben wären.
$R^T$ ist somit immer regulär und kann invertiert werden, was eine weitere Vereinfachung ermöglicht:
\begin{equation*}
(R^T)^{-1}R^TR\hat{x}=(R^T)^{-1}R^TQb\quad\Leftrightarrow\quad R\hat{x}=Q^Tb,
\end{equation*}
und weiter vereinfacht
\begin{equation}
\hat{x}=R^{-1}Q^Tb,
\end{equation}
was somit das Least-Squares-Problem mit dem Umweg über die $QR-$Zerlegung auf kompakte Weise löst.
