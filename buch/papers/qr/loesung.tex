%
% loesung.tex -- Beispiel-File für die Beschreibung der Loesung
%
% (c) 2020 Prof Dr Andreas Müller, Hochschule Rapperswil
%
\section{Lösung mit Givens-Rotationen
\label{qr:section:loesung}}
\rhead{Lösung}
Eine Drehung in der von $x_i$ und $x_j$ aufgespannten Ebene um den Winkel $\alpha$ kann mit der Matrix
\begin{equation}
D_{\alpha,i,j}=
\begin{pmatrix}
1     &\cdots&0     &\cdots&0     &\cdots&0\\
\vdots&\ddots&\vdots&      &\vdots&      &\vdots\\
0     &\cdots&c     &\cdots&-s    &\cdots&0\\
\vdots&      &\vdots&\ddots&\vdots&      &\vdots\\
0     &\cdots&s     &\cdots&c     &\cdots&0\\
\vdots&      &\vdots&      &\vdots&\ddots&\vdots\\
0     &\cdots&0     &\cdots&0     &\cdots&1
\end{pmatrix}
\end{equation}
beschrieben werden, wobei $c=\cos(\alpha)$ und $s=\sin(\alpha)$.
Drehungen dieser Art heissen Givens-Rotationen.

Es ist nun möglich, Elemente in der unterhalb der Diagonalen einer $l\times n $-Matrix $A$ mit geeigneten Givens-Rotationen auf Null zu bringen, und um so eine obere Dreiecksmatrix $R$ zu erhalten.
Das Produkt aller inversen Drehmatrizen ist dann eine orthonormierte Matrix $Q$, da alle Drehmatrizen $D_{\alpha_k,i,j}$ orthonormal sind.

Das wird nun an dieser Stelle im Detail betrachtet.
Zuerst wird mit Givens-Rotationen ($D_{\alpha_k, i, j}$ der Grösse $l\times l$) die Matrix $R$ berechnet: 
\begin{equation*}
R = D_{\alpha_k,n-1,l}...D_{\alpha_2,1,3}D_{\alpha_1,1,2}A.		
\end{equation*}
Auf die Matrix $Q$ kommt man dann mit
\begin{multline*}
Q=AR^{-1}=A(D_{\alpha_k,n-1,l}...D_{\alpha_2,1,3}D_{\alpha_1,1,2}A)^{-1}\\=
\underbrace{AA^{-1}}_{E}D_{\alpha_1,1,2}^{-1}D_{\alpha_2,1,3}^{-1}... D_{\alpha_k, n-1,l}^{-1}=
D_{\alpha_1,1,2}^{T}D_{\alpha_2,1,3}^{T}...D_{\alpha_k, n-1,l}^{T},
\end{multline*}
wobei im letzten Schritt ausgenutzt wurde, dass Drehmatrizen orthonormal sind.
Die Matrizen $Q$ und $R$ sind beide von der Grösse $l\times l$.
Für den Spezialfall $l=n$ können sie so stehen gelassen werden.
Für alle Fälle wo A aber die Grösse $l\times n$ hat, müssen die Dimensionen noch angepasst werden. um den Forderungen der $QR$-Zerlegung zu genügen.
In der Matrix $R$ stehen ab der n. Zeile nur noch Nullen.
Diese Zeilen können somit weggelassen werden und folglich alle Einträge ab der n. Spalte von Q.
Als Matrizen ausgeschrieben:
\begin{multline*}
A=QR=
\begin{pmatrix}
q_{11}&\cdots&q_{1n}&q_{1(n+1)}&\cdots&q_1l\\
\vdots&\ddots&\vdots&\vdots    &\ddots&\vdots\\
q_{l1}&\cdots&q_{ln}&q_{l(n+1)}&\cdots&q_{ll}
\end{pmatrix}
\begin{pmatrix}
r_{11}&r_{12}&\cdots&r_{1n}\\
0     &r_{22}&\cdots&r_{2n}\\
\vdots&\vdots&\ddots&\vdots\\
0     &0     &\cdots&r_{nn}\\
0     &0     &\cdots&0\\
\vdots&\vdots&\ddots&\vdots\\
0     &0     &\cdots&0
\end{pmatrix}\\=
\begin{pmatrix}
q_{11}&\cdots&q_{1n}&\\
\vdots&\ddots&\vdots&\\
q_{l1}&\cdots&q_{ln}\\
\end{pmatrix}
\begin{pmatrix}
r_{11}&r_{12}&\cdots&r_{1n}\\
0     &r_{22}&\cdots&r_{2n}\\
\vdots&\vdots&\ddots&\vdots\\
0     &0     &\cdots&r_{nn}.
\end{pmatrix}
\end{multline*}
$Q$ ist daraufhin also eine $l\times n$-Matrize und $R$ ist von der Grösse $n\times n$, wie verlangt.

