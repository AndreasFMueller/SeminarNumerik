%
% main.tex -- Paper zum Thema arctan
%
% (c) 2020 Hochschule Rapperswil
%
\chapter{Kettenbrüche und Padé-Approximation\label{chapter:arctan}}
\lhead{Kettenbrüche und Padé-Approximation}
\begin{refsection}
\chapterauthor{Andreas Müller}

{\parindent0pt
Kettenbrüche} versprechen eine bestmögliche Approximation einer Zahl durch
rationale Zahlen, die Padé-Approximation verspricht eine bestmögliche
Approximation einer Potenzreihe durch eine gebrochen rationale Funktion.
Besteht ein Zusammenhang?

In Kapitel~\ref{chapter:kettenbruch} wurde ohne Beweis eine Approximation der 
Taylor-Reihe der Arkustangens-Funktion
\begin{equation}
\arctan x
=
x-\frac{x^3}{3}+\frac{x^5}{5}-\frac{x^7}{7}+\frac{x^9}{9}-\frac{x^{11}}{11}
+\dots
\label{arctan:arctan}
\end{equation}
als Kettenbruch dargestellt.
Doch genau genommen ist das gar kein Kettenbruch in dem Sinne, wie er
im Rest von Kapitel~\ref{chapter:kettenbruch} betrachtet wurde, denn
bestimmt man die Approximationsbrüche, entsteht eine gebrochen rationale
Funktion.
In diesem Kapitel soll gezeigt werden, dass der Kettenbruch eigentlich
nur eine Schreibweise ist, mit der verschiedene Padé-Approximationen von
$\arctan x$ effizient berechnet werden können.

\section{Kettenbrüche und Matrizen
\label{arctan:section:matrizen}}
Wir verwenden wieder die schon in Kapitel \ref{chapter:kettenbruch}
eingeführte Notation
\[
\frac{p_n}{q_n}
=
a_0 +
\cfrac{b_1}{
a_1+\cfrac{b_2}{
a_2+\cfrac{b_3}{
\dots+\cfrac{\dots}{\dots+\cfrac{b_{n-1}}{a_{n-1}+\cfrac{b_n}{a_n}}}}}}.
\]
Für einen Kettenbruch wird verlangt, dass alle Zahlen $a_k$ und $b_k$ 
wie auch die Zähler $p_n$ und Nenner $q_n$ der Approximationsbrüche
ganze Zahlen sind.
Für die Approximation der Funktion $\arctan x$ müssen wir
diese Voraussetzung etwas aufweichen.
Wir verlangen nur noch, dass alle diese Ausdrücke Polynome in $x$ sind.

Wir brauchen eine Methode zur Berechnung der Approximationszähler und -nenner.
Dazu berechnen wir, wie sich ein Bruch ändert, wenn ein zusätzlicher
Bruchterm $\frac{p}{q}$ im Nenner hinzugefügt wird:
\begin{equation}
\frac{\tilde{p}}{\tilde{q}}
=
\cfrac{b}{a+\cfrac{p}{q}}
=
\frac{bq}{ab+p}.
\label{arctan:neuerbruch}
\end{equation}
Schreibt man die Brüche als Spaltenvektoren mit Zähler und Nenner als
Komponenten, dann kann man die Operation~\eqref{arctan:neuerbruch}
in Matrixform als
\[
\begin{pmatrix}
\tilde{p}\\\tilde{q}
\end{pmatrix}
=
\begin{pmatrix}
bq\\
aq+p
\end{pmatrix}
=
\begin{pmatrix}
0&b\\
1&a
\end{pmatrix}
\begin{pmatrix}
p\\
q
\end{pmatrix}
\]
schreiben.
Man kann daher den Näherungsbruch $p_n/q_n$ ebenfalls in Matrixform
schreiben als
\[
\begin{pmatrix}
p_n\\q_n
\end{pmatrix}
=
\begin{pmatrix} 0&b_1\\1&a_1\end{pmatrix}
\begin{pmatrix} 0&b_2\\1&a_2\end{pmatrix}
\begin{pmatrix} 0&b_3\\1&a_3\end{pmatrix}
\dots
\begin{pmatrix} 0&b_{n-1}\\1&a_{n-1}\end{pmatrix}
\begin{pmatrix} b_n\\a_n\end{pmatrix}.
\]
Will man den letzten Bruch $b_n/a_n$ in der Kettenbruchentwicklung
weglassen, also den Kettenbruch verkürzen, dann ist das gleichbedeutend
damit, $b_n=0$ und $a_n=1$ zu setzen.
Damit erhält man aber die Approximation $p_{n-1}/q_{n-1}$.
Kombiniert man beide Approximationen in eine Matrix, erhält man
\begin{equation}
\begin{pmatrix}
p_{n-1}&p_n\\
q_{n-1}&q_n
\end{pmatrix}
=
\begin{pmatrix} 0&b_1\\1&a_1\end{pmatrix}
\begin{pmatrix} 0&b_2\\1&a_2\end{pmatrix}
\begin{pmatrix} 0&b_3\\1&a_3\end{pmatrix}
\dots
\begin{pmatrix} 0&b_{n-1}\\1&a_{n-1}\end{pmatrix}
\begin{pmatrix} 0&b_n\\1&a_n\end{pmatrix}.
\label{arctan:produkt}
\end{equation}

Das Produkt~\eqref{arctan:produkt} wurde oben zunächst von rechts nach links
abgeleitet, es wurde der ``letzte'' Ausdruck in Matrixform gebracht und
dann nach links fortschreitend der ganze Kettenbruch aufgebaut.
Die Matrixmultiplikation ist jedoch assoziativ, so dass man ausgerüstet
mit Formel~\eqref{arctan:produkt} die Quotienten $p_n/q_n$ jetzt auch von
links nach rechts berechnen kann.

\section{Padé-Approximation von $\arctan x$
\label{arctan:section:kettenbruch}}
Wir versuchen jetzt zu verstehen, wie man die Taylor-Reihe~\ref{arctan:arctan}
als Kettenbruch verstehen kann.
Nach dem Prinzip der Padé-Approximation suchen wir einen Bruch $p_n/q_n$,
der mit der Taylor-Reihe bis zu einem gewissen Grad übereinstimmt.
Wir verlangen daher, dass $p_n$ und $q_n$ Polynome sind, für die gilt
\[
\frac{p_n}{q_n} = \arctan x + O(x^k),
\]
über die passende Ordnung $k$ machen wir uns später Gedanken.
Wie bei der Herleitung der Padé-Approximation in Kapitel~\ref{chapter:pade}
multiplizieren wir mit dem Nenner und erhalten die Bedingung
\begin{equation}
p_n - q_n\arctan x = O(x^k).
\label{arctan:bedingung}
\end{equation}
Auch diese Bedingung lässt sich in Vektorform fassen:
\begin{equation}
\begin{pmatrix}
1&-\arctan x
\end{pmatrix}
\begin{pmatrix} 
p_n\\q_n
\end{pmatrix}
=
O(x^k)
\qquad\text{oder}\qquad
\begin{pmatrix}
1&-\arctan x
\end{pmatrix}
\begin{pmatrix}
p_{n-1}&p_n\\
q_{n-1}&q_n
\end{pmatrix}
=
\begin{pmatrix}
O(x^{k-1})&O(x^k).
\end{pmatrix}
\label{arctan:bedingungmatrix}
\end{equation}

Die Rekursionsfolge startet mit der Matrix
\[
\begin{pmatrix}
1&-\arctan x
\end{pmatrix}
\begin{pmatrix}
p_0&p_1\\
q_0&q_1
\end{pmatrix}
=
\begin{pmatrix}
1&-\arctan x
\end{pmatrix}
\begin{pmatrix}
0&p_1\\
1&q_1
\end{pmatrix}
=
\begin{pmatrix}
0
&
p_1 - q_1\biggl(x-\frac{x^3}{3}+\frac{x^5}{5}-\frac{x^7}{7}+\dots
\end{pmatrix}.
\]
Damit auf der rechten Seite im zweiten Term die Terme erster Ordnung
wegfallen, müssen $p_1$ und $q_1$ als Polynome möglichst niedrigen Grades
so gewählt werden, dass
\[
p_1-xq_1 = O(x^2),
\]
dies kann zum Beispiel $q_1=1$ und $p_1=x$ geschehen.
Damit ist
\[
\begin{pmatrix}
p_0&p_1\\
q_0&q_1
\end{pmatrix}
=
\begin{pmatrix}
0&x\\
1&1
\end{pmatrix}
\]
der Start der Rekursion.

Die Bedingung \eqref{arctan:bedingungmatrix} kann jetzt dazu verwendet werden,
auch die weiteren $a_k$ und $b_k$ zu bestimmen.
Für $a_1$ und $b_1$ rechnen wir dazu zunächst das Produkt
\[
\begin{pmatrix}
p_0&p_1\\
q_0&q_1
\end{pmatrix}
\begin{pmatrix}
0&b_1\\
1&a_1
\end{pmatrix}
=
\begin{pmatrix}
0&x\\
1&1
\end{pmatrix}
\begin{pmatrix}
0&b_1\\
1&a_1
\end{pmatrix}
=
\begin{pmatrix}
x&a_1x\\
1&a_1+b_1
\end{pmatrix}
\]
aus. 
Multiplikation von links mit dem Zeilenvektor ergibt
\[
\begin{pmatrix}
1&-\arctan x
\end{pmatrix}
\begin{pmatrix}
x&a_1x\\
1&a_1+b_1
\end{pmatrix}
=
\begin{pmatrix}
O(x^2) & a_1x - (a_1+b_1)\biggl(\displaystyle x-\frac{x^3}3+\dots\biggr)
\end{pmatrix}.
\]
Die Polynome $a_1$ und $b_1$ müssen so gewählt werden, dass keine Terme
dritter Ordnung mehr stehen bleiben.
Dies ist natürlich nicht eindeutig möglich sondern nur bis auf ein
Vielfaches.
Den Term erster Ordnung kann man mit konstantem $a_1$ zum verschwinden
bringen, dann muss aber $b_1$ mindestens den Grad $2$ haben.
Es muss also
\[
a_1x-(a_1+cx^2)\biggl(x-\frac{x^3}3+O(x^5)\biggr)
=
a_1\frac{x^3}{3}-cx^3 + O(x^5)
\]
sein.
Dies lässt sich zum Beispiel mit $a_1=3$ und $c=1$ erreichen.
Der Näherungsbruch ist dann
\[
\begin{pmatrix}
x&3x\\
1&x^2+3
\end{pmatrix}
\qquad\Rightarrow\qquad
\frac{p_2}{q_2} = \frac{3x}{3+x^2}.
\]

Wir führen auch noch den Schritt $k=2$ durch, bei dem die Terme fünfter
Ordnung wegfallen sollten.
Das Matrizenprodukt ist
\[
\begin{pmatrix}
x&3x\\
1&x^2+3
\end{pmatrix}
\begin{pmatrix}
0&b_2\\
1&a_2
\end{pmatrix}
=
\begin{pmatrix}
3x    & b_2x + 3xa_2 \\
x^2+3 & b_2 + a_2(x^2+3)
\end{pmatrix}.
\]
Nach Linksmultiplikation mit dem Zeilenvektor ergibt sich wieder
\[
b_2p_1 + a_2p_2 - (b_2q_1+a_2q_2)
\biggl(x-\frac{x^3}{3} +\frac{x^5}5-\dots\biggr)
=
b_2(p_1-q_1 \arctan x)
+ a_2(p_2 - q_2\arctan x)
=
O(x^6)
\]
Im ersten Term wissen wir, dass wir, dass bereits die Terme der
Ordnung $\le 1$ wegfallen.
In der zweiten Klammer fallen die Terme bis zur dritten Ordnung weg.
Es folgt daher wieder, dass man für $b_2$ einen Term zweiten Grades
der Form $cx^2$ wählen muss und folglich für $a_2$ eine Konstante.
Setzt man alles ein, erhält man
\begin{align*}
cx^2\biggl(x- x+\frac{x^3}3-\frac{x^5}5+O(x^7)\biggr)
&+
a_2\biggl(3x -(x^2+3)\biggl(x-\frac{x^3}3+\frac{x^5}5-O(x^7)\biggr)\biggr)
\\
&=
c\frac{x^5}{3}
+O(x^6)
+a_2
\biggl(\frac{x^5}{3}-\frac{3}{5}x^5+O(x^6)\biggr)
\\
&=x^5\biggl(\frac{c}3-a_2\frac{4}{15}\biggr) + O(x^6),
\end{align*}
woraus als mögliche Lösung
\[
b_2=4x^2,\qquad
a_2=5
\]
folgt.
Der zugehörige Näherungsbruch ist
\[
\arctan x
\approx
\frac{p_3}{q_3}
=
\frac{xb_2+3xa_2}{b_2+a_2(x^2+3)}
=
\frac{4x^3+15}{9x^2+15}.
\]

Es lässt sich zum Beispiel mit einem Computer-Algebra-Programm wie
Maxima verifizieren, dass die Matrizen
\[
\begin{pmatrix}
0&n^2x^2\\
1&2n+1
\end{pmatrix}
\qquad\text{also}\qquad
b_n=n^2x^2\qquad\text{und}\qquad a_n=2n+1
\]
tatsächlich dafür sorgen, dass die Kettenbruchentwicklung tatsächlich
immer eine Approximation bis zur Ordnung $2n+1$ ist.

Wir ermitteln noch die Grade der Polynome $p_n$ und $q_n$.
Schreiben wir nur die Grade der Polynome in die Matrizen, erhalten wir
\begin{align*}
\begin{pmatrix}
n-1&n+1\\
n  &n
\end{pmatrix}
\begin{pmatrix}
-\infty & 0 \\
    0   & 2
\end{pmatrix}
&=
\begin{pmatrix}
n+1 & n+3 \\
n   & n+2
\end{pmatrix}
\end{align*}


\printbibliography[heading=subbibliography]
\end{refsection}
