\documentclass[tikz]{standalone}
\usepackage{amsmath}
\usepackage{times}
\usepackage{txfonts}
\usepackage{pgfplots}
\usepackage{csvsimple}
\usetikzlibrary{calc,shapes, positioning}
\begin{document}

\tikzset{
  treenode/.style = {shape=rectangle,
                     draw, anchor=center,
                     text width=8em, align=center,
                     top color=white, bottom color=white,
                     inner sep=1ex},
  decision/.style = {treenode, diamond, inner sep=0pt},
  root/.style     = {treenode, font=\Large, bottom color=white},
  env/.style      = {treenode, text width=18em, font=\ttfamily\normalsize},
}
\begin{tikzpicture}[-latex]
\def\sx{16}
\def\sy{8}
\def\rowcount{5}

\node [env] (A) at ({\sx/2}, {\sy/\rowcount*4}) {Eingabe von $x_0, u_0, h_\text{max};\quad n=0$};
\node [env] (B) at ({\sx/2}, {\sy/\rowcount*3}) {Ein Testschritt (ausgehend von $P_\text{start}=(x_n, u_n)$) der L\"ange $h_{\text{max}}$ mit dem Eulerverfahren $\implies P_\text{test}$};
\node [env] (C) at ({\sx/2}, {\sy/\rowcount*2}) {Berechnung der optimalen Schrittweite $h$ gem\"ass Formel~\eqref{buch:steps:equationSimpleStepSize}};
\node [env] (D) at ({\sx/2}, {\sy/\rowcount*1}) {falls berechnete Schrittweite $h > h_\text{max}: \implies h=h_\text{max}$};
\node [env] (E) at ({\sx/2}, {\sy/\rowcount*0}) {Runge-Kutta Schritt mit $h$ von $P_\text{start}$ aus durchf\"uhren\\ $\implies x_{n+1}, u_{n+1}$};

\draw[->] (A) -- (B);
\draw[->] (B) -- (C);
\draw[->] (C) -- (D);
\draw[->] (D) -- (E);

\draw[-] (E) |- ({0.27*\sx}, {-1});
\draw[-] ({0.27*\sx}, {-1}) -- node[above, rotate=90] {$n \implies n+1$} ({0.27*\sx}, {\sy/\rowcount*3});
\draw[->] ({0.27*\sx}, {\sy/\rowcount*3}) -- (B);

\end{tikzpicture}
\end{document}

