%
% einleitung.tex -- Beispiel-File für die Einleitung
%
% (c) 2020 Prof Dr Andreas Müller, Hochschule Rapperswil
%
\section{Einleitung\label{steps:section:einleitung}}
\rhead{Einleitung}

Numerische Verfahren zur Berechnung von Differentialgleichungen wie das Euler- oder Runge-
Kutta Verfahren bekanntlich auf iterativem Vorgehen in Schritten, das bedeutet, dass der Verlauf
der Kurve mit vielen Einzelschritten bis zum gesuchten Endpunkt angenähert wird und nicht exakt
bestimmt werden kann. Um den Fehler möglichst klein zu halten,
kann bei gegebenem Lösungsverfahren beispielsweise die Schrittlänge zwischen den einzelnen Punkten verkleinert werden, wodurch
jedoch der Rechenaufwand steigt. Die für die geforderte Genauigkeit nötige Schrittweite richtet sich
nach dem Punkt, an dem die "Krümmung" der Kurve am grössten ist. Die Aufgabe der Schrittlän-
gensteuerung ist es nun, die Schrittlänge laufend der dynamik der Kurve anzupassen, sodass der
Zielpunkt mit möglichst wenig Schritten und der erforderlichen Genauigket berechnet werden
kann.

