\section{Nummerische Lösung
\label{perturbation:section:nummerischeloesung}}
\rhead{Nummerische Lösung}
Die Gleichungen \ref{eq:x_diff} und \ref{eq:y_diff} lassen sich durch nummerische Verfahren lösen. 
Wenn man sich nun jedoch einen Satelliten oder eine Rakete vorstellt, haben diese nur begrenzte Ressourcen an Rechenleistung. 
Aus diesem Grund wird die komplexere Berechnung durch eine Bodenstation durchgeführt. 
Die Resultate der Berechnung werden jeweils dem Flugobjekt mitgeteilt, sodass dieses mit simpleren Formeln seine Bahn annähernd berechnen kann. 
Dieses Kapitel beschäftigt sich mit der exakten Lösung des Problems, welche durch die Bodenstation übernommen wird.

Da das Runge-Kutta Verfahren nur mit Differentialgleichungen erster Ordnung umgehen kann, muss das System \ref{eq:x_diff} und \ref{eq:y_diff} zuerst um eine Ordnung reduziert werden. 
Dies kann erreicht werden, indem zwei Hilfsvariablen $v_x = \dot{x}$ und $v_y = \dot{y}$ eingeführt werden. 
Das System wird somit um eine Ordnung reduziert, erhält aber im Austausch zusätzliche Dimensionen.
\[
\dot{r_x} = v_x
\]
\[
\dot{r_y} = v_y
\]
\[
\dot{v_x} = \frac{k}{m} \cdot \sqrt{v_x^2 + v_y^2} \cdot v_x
\]
\[
\dot{v_y} = \frac{k}{m} \cdot \sqrt{v_x^2 + v_y^2} \cdot v_y + g   
\]

Als Vektor formuliert:
\[
\frac{d}{dt}\begin{pmatrix}r_x\\r_y\\v_x\\v_y\end{pmatrix} = \begin{pmatrix}v_x\\v_y\\\frac{k}{m} \cdot \sqrt{v_x^2 + v_y^2} \cdot v_x\\\frac{k}{m} \cdot \sqrt{v_x^2 + v_y^2} \cdot v_y + g\end{pmatrix}
\]

Damit haben wir ein für Runge-Kutta lösbare Variante der Differentialgleichung gefunden. 
Folgender Python Code löst die Differentialgleichung mit Hilfe der Library \textit{SciPy}.

\begin{lstlisting}[language=Python]
    def dgl(t, y):
        '''The differential equation to solve.'''
        r_x = y[0]
        r_y = y[1]
        v_x = y[2]
        v_y = y[3]
        p0 = v_x
        p1 = v_y
        p2 = -k/m * np.sqrt(v_x**2 + v_y**2) * v_x
        p3 = -k/m * np.sqrt(v_x**2 + v_y**2) * v_y - g
        return [p0, p1, p2, p3]
    
    def runge_kutta(t):
        '''Runs Runge Kutta to find position and speed at time t.'''
        rk = sp.solve_ivp(fun=dgl, t_span = (0,t), 
        y0 = [r0x, r0y, v0x, v0y], method='RK45', vectorized = True)
        assert rk.status == 0
        time_at_end     = rk.t[-1]
        position_at_end = rk.y[0:2, -1]
        speed_at_end    = rk.y[2:4, -1]
        return time_at_end, position_at_end, speed_at_end
    \end{lstlisting}

    Die Bodenstation, die zu komplexen Kalkulationen in der Lage ist, führt also in einem regelmässigen Zeitintervall die Funktion \lstinline[language=Python]{runge_kutta(t)} auf dem Differentialgleichungssystem aus und erhält so die Position und die Geschwindigkeit zu einem bestimmten Zeitpunkt.
