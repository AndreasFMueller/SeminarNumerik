\section{Fazit}
Durch das Studium der Störungstheorie haben wir gesehen, dass auch komplexe Bahnverläufe mit begrenzter Rechenleistung sehr gut approximiert werden können.
Man benötigt allerdings eine Bodenstation, die fähig ist, exakte Werte an vorgegebenen Stützstellen zu berechnen.
Dies kann mit Hilfe numerischer Verfahren zur Lösung komplexer Differentialgleichungen erreicht werden, wie beispielsweise dem Runge-Kutta-Verfahren.
Es lassen sich somit Verfahren kreieren, sodass auch ein Satellit oder ein anderes Objekt mit begrenzter Rechenleistung die Bahn approximieren kann.
Dies, obwohl die Bahn von hunderten planetarischen Objekten gestört wird.
Es ist dabei möglich, die Genauigkeit nahezu beliebig zu erhöhen, indem man entweder die Bodenstation in kürzeren Intervallen um neue Werte bittet,
oder Polynome höherer Ordnung anstelle der Anfangswerte einsetzt.