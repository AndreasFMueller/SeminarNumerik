\section{Einleitung\label{perturbation:section:einleitung}}
\rhead{Einleitung}
Die Störungstheorie befasst sich mit den Effekten von kleinen Einflüssen auf sich bewegende Objekte.
Oft werden diese weggelassen, um die Probleme in geschlossener Form überhaupt lösen zu können.
Wenn es jedoch um die Berechnung von Satellitenbahnen geht, wo die Planeten eine kleine Anziehung auf die Satelliten ausüben, kann dies langfristig zu grossen Abweichungen führen.
\index{Satellitenbahnen}%
Auch bei Raketen gibt es Störungen der Bahn.
So wird diese unter anderem vom Luftdruck beeinflusst, der von der Höhe abhängig ist.
\index{Luftdruck}%

In der Störungstheorie wird mit einer vereinfachten Form des ursprünglichen Problems, welches sich exakt lösen lässt, gestartet.
Die Konstanten der Gleichung werden danach langsam angepasst, um die Lösung des Problems zu approximieren.
Dieses Vorgehen beschreiben wir anhand eines einfaches Beispiels in den nächsten Abschnitten.


