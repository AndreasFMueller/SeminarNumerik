\section{Problemstellung
\label{perturbation:section:problemstellung}}
\rhead{Problemstellung}
Eine Kanonenkugel wird in einem $45^{\circ}$ Winkel abgefeuert. 
Sie besitzt eine Startgeschwindigkeit von 100 m/s in $x$ sowie $y$-Richtung. 
Die Aufgabe besteht darin, die Position zum Zeitpunkt $t$ zu berechnen.

Dies ist ein gängiges Beispiel für den schiefen Wurf. 
In der Regel werden dabei folgende Formeln verwendet:

\begin{equation}\label{eq:x_simple}
\begin{aligned}
    x(t) &= x_0 + v_{0x}t\\
    y(t) &= y_0 + v_{0y}t - \frac{1}{2}gt^2
\end{aligned}
\end{equation}
%\begin{equation}\label{eq:vx_simple}
%    v(t)_x = v_{0_x} 
%\end{equation}
%\begin{equation}\label{eq:vy_simple}
%    v(t)_y = v_{0_y} - g \cdot t
%\end{equation}


Nun möchten wir jedoch ein genaueres Ergebnis erzielen und den Luftwiderstand mitberücksichtigen.  
Der Luftwiderstand $F_W$ ist proportional zum Quadrat der Geschwindigkeit. Es gilt folgende Formel:
\[
F_W = \underbrace{c_WA\frac{1}{2}\rho_{\text{Luft}}}_\text{k}v^2
\]
$c_W$ bezeichnet den Strömungswiderstand, $A$ die Widerstandsfläche und $\rho_{\text{Luft}}$ die Luftdichte. 
Um die Berechnung simpel zu gestalten, gehen wir von einer konstanten Luftdichte $\rho_{\text{Luft}}$ aus und können so den Luftwiderstand zusammen fassen als $F_W = k \cdot v^2$.

Formt man die Formel weiter um, erhält man das folgende Resultat:
\[
F_W = k \cdot |\vec{v}| \cdot \vec{v} = k \cdot \sqrt{v_x^2 + v_y^2} \cdot \begin{pmatrix}v_x\\v_y\end{pmatrix}
\]

Die ganze Kraft, die auf die Kugel wirkt, ist gemäss Newtons zweitem Axiom $F = m \cdot a$. 
Neben dem Luftwiderstand wirkt in $y$-Richtung zusätzlich die Gravitationskraft $F_G = m \cdot g$. 
Somit lassen sich folgende Gleichungen aufstellen:
\begin{align*}
m \cdot a_x &= k \cdot \sqrt{v_x^2 + v_y^2} \cdot v_x\\
m \cdot a_y &= k \cdot \sqrt{v_x^2 + v_y^2} \cdot v_y + m \cdot g
\end{align*}

Dividiert durch $m$ ergibt dies:
\begin{align*}
a_x &= \frac{k}{m} \cdot \sqrt{v_x^2 + v_y^2} \cdot v_x\\
a_y &= \frac{k}{m} \cdot \sqrt{v_x^2 + v_y^2} \cdot v_y + g
\end{align*}

Die Beschleunigung ist nicht bekannt. 
Sie kann jedoch als zweite Ableitung der Position nach der Zeit ausgedrückt werden. 
Die Geschwindigkeit entspricht der ersten Ableitung.  
Somit ergibt sich das folgende System von Differentialgleichungen zweiter Ordnung:

\begin{equation}\label{eq:x_diff}
\begin{aligned}
\Ddot{r_x} = \frac{k}{m} \cdot \sqrt{\dot{r_x^2} + \dot{r_y^2}} \cdot \dot{r_x}\\
\Ddot{r_y} =   \frac{k}{m} \cdot \sqrt{\dot{r_x^2} + \dot{r_y^2}} \cdot \dot{r_y} + g
\end{aligned}     
\end{equation}
%\begin{equation}\label{eq:vx_diff}
%\dot{v_x} = \frac{k}{m} \cdot \sqrt{v_x^2 + v_y^2} \cdot v_x
%\end{equation}
%\begin{equation}\label{eq:vy_diff}
%\dot{v_y} = \frac{k}{m} \cdot \sqrt{v_x^2 + v_y^2} \cdot v_y + g   
%\end{equation}





