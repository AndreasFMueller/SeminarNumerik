%
% problemstellung.tex -- Beispiel-File für die Beschreibung des Problems
%
% (c) 2020 Prof Dr Andreas Müller, Hochschule Rapperswil
%
\section{Folgerungen
\label{vanderpol:section:folgerungen}}
\rhead{Folgerungen}
Zusammenfassend kann man sagen, dass es möglich ist, Differentialgleichungen mit verschiedenen numerischen Methoden ohne allzu grosse Probleme zu lösen. Es kann auch entschieden werden, mit welcher Präzision man es tun möchte, natürlich auf Kosten längerer Rechenzeiten. Dieses Kapitel zeigt jedoch, dass einige Systeme, wie das von Van der Pol, sich unter bestimmten Umständen chaotisch verhalten können. Wenn man letztere in Betracht zieht, selbst wenn die Länge des Integrationsschrittes mit der Idee der Genauigkeitssteigerung verkürzt werden kann, wird das Ergebnis immer noch falsch sein. Der Wert des durch die numerische Methode eingeführten Fehlers wird mit fortschreitender Iteration grösser, was eher zu einer Divergenz als zu einer Annäherung der Lösung führt. Eine mögliche Methode, um die Auswirkungen der letzteren zu reduzieren, könnte zum Beispiel die Schrittlängensteuerung sein, d.h. eine variable Integrationsschritt zu haben. Kapitel \ref{chapter:steps} dieses Buches ist diesem letzten Thema gewidmet. 
\index{Schrittlängensteuerung}%
