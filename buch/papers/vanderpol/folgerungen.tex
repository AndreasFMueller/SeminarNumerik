%
% problemstellung.tex -- Beispiel-File für die Beschreibung des Problems
%
% (c) 2020 Prof Dr Andreas Müller, Hochschule Rapperswil
%
\section{Folgerungen
\label{vanderpol:section:folgerungen}}
\rhead{Folgerungen}
Das Kapitel zeigt daher, wie der Van der Pol Oszillator, indem bestimmte Werte erfüllt werden, ein chaotisches Verhalten haben kann. Durch den Einsatz numerischer Methoden zur Lösung von Differentialgleichungen ist es möglich, die Genauigkeit (kleine Schrittlänge) auf Kosten längerer Rechenzeiten zu entscheiden. Es ist wichtig, vorsichtig zu sein, denn eine lange und theoretisch genaue Berechnung zeigt kein exaktes Ergebnis, wenn das System chaotisch ist. 
Eine Methode zur Verbesserung der Berechnung der Differentialgleichung ist die Verwendung einer variablen Schrittlänge, wie in Kapitel \ref{chapter:steps} erläutert. 
\index{Schrittlängensteuerung}%
