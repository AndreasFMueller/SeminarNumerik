%
% uebersicht.tex -- Uebersicht ueber die Seminar-Arbeiten
%
% (c) 2020 Prof Dr Andreas Mueller, Hochschule Rapperswil
%
\chapter*{Übersicht}
\lhead{Übersicht}
\rhead{}
\label{buch:uebersicht}
Im zweiten Teil kommen die Teilnehmer des Seminars selbst zu Wort.
Die im ersten Teil dargelegten mathematischen Methoden und
grundlegenden Modelle werden dabei verfeinert, verallgemeinert
und auch numerisch überprüft.

Das Iterationsproblem wurde bereits in Kapitel~\ref{chapter:berechnung}
als zentral für das Verständnis der numerischen Berechnungsprozesse
etabliert.
{\em Michael Schneeberger} vertieft die Analyse in
Kapitel~\ref{chapter:logistic}
und zeigt das faszinierende
Verhalten der Iterationsfolgen bei der Variation des Parameters in
der logistischen Gleichung auf.
\index{logistische Gleichung}%
\index{Schneeberger, Michael}%
Die Periodenverdoppelungen sind ein universelle Eigenschaft chaotischer
Systeme.
\index{chaotisch}%
\index{Periodenverdoppelung}%

Das Beispiel der Legendre-Polynome zeigt, dass selbst die Berechnung
von Polynomen mit naheliegenden Formeln manchmal nicht einfach so 
funktioniert.
\index{Legendre-Polynom}%
{\em Patrick Elsener} analysiert in Kapitel~\ref{chapter:legendre}
die Gründe dafür, warum diese 
Berechnung instabil werden kann und erklärt damit die Fehler 
populärer Websites wie Wolfram Alpha.
\index{instabil}%
\index{Elsener, Patrick}%
\index{Wolfram Alpha}%

Integrale können natürlich auch mit Verfahren zur Lösung von
gewöhnlichen Differentialgleichungen bestimmt werden.
\index{Differentialgleichung!gewöhnliche}%
Mit besonders wenigen Auswertungen des Integranden kommt jedoch
ein ganzlich anderes Verfahren aus, welches sich aus der Interpolationstheorie
von Kapitel~\ref{chapter:interpolation} entwickeln lässt.
{\em Mike Schmid} stellt in Kapitel~\ref{chapter:quadratur} die
Gauss-Quadratur vor.
\index{Schmid, Mike}%
\index{Gauss-Quadratur}%

Ein mathematisches Modell für einen natürlichen Vorgang zu haben garantiert
noch nicht, dass man diesen Vorgang exakt vorhersagen kann.
{\em Manuel Cattaneo} und {\em Niccol\`o Galliani} untersuchen
in Kapitel~\ref{chapter:vanderpol} die
Van der Pol-Differentialgleichung, die einen nichtlinearen Oszillator
beschreibt.
\index{Cattaneo, Manuel}%
\index{Galliani, Niccol\`o}%
Sie heben besonders hervor, wie für gewisse Parameterwerte chaotisches
Verhalten auftritt.
\index{chaotisch}%
Dies bedeutet, dass die Rundungsfehler selbst so viel Unsicherheit in
die Problemlösung injizieren, dass keine zuverlässigen Langzeitvorhersagen 
mehr möglich sind.
\index{Rundungsfehler}%
\index{Langzeitvorhersage}%

Das Beispiel der Van der Pol-Gleichung zeigt, dass es sich lohnt,
etwas Aufwand in die Entwicklung guter numerischer Verfahren zu
investieren.
\index{Van der Pol-Gleichung}%
{\em Fabio Marti} zeigt in Kapitel~\ref{chapter:taylor}, wie man auf der
Basis der Taylor-Reihe im Prinzip ein Lösungsverfahren beliebig hoher Ordnung 
entwickeln kann.
\index{Marti, Fabio}%
\index{Taylor-Reihe}%
Die Wahl der Schrittweite hat einen entscheidenden Einfluss auf die
Genauigkeit des Resultats.
\index{Schrittweite}%
In Kapitel~\ref{chapter:steps} zeigt
{\em Reto Fritsche}, wie ein Lösungsalgorithmus die Schrittweite
anpassen und damit die Genauigkeit optimieren kann.
\index{Fritsche, Reto}%
Eine beliebte analytische Technik zur Lösung gewöhnlicher Differentialgleichung
verwendet die Laplace-Transformation, die jedoch oft schwierig zu
invertieren ist.
\index{Laplace-Transformation}%
\index{Laplace-Inverse}%
{\em Severin Weiss} erklärt in Kapitel~\ref{chapter:laplace}, wie 
das Verfahren von Talbot die Laplace-Inverse numerisch berechnen kann.
\index{Weiss, Severin}%

Der Aufwand für die Lösung einer Differentialgleichung, die alle
messbaren Einflüsse beinhaltet, ist manchmal nur mit grossem Aufwand
durchführbar.
Die Bahn eines Satelliten um die Erde wird von Sonne, Mond, den grossen
Planeten und vielen weiteren Kräften beeinflusst.
\index{Satellit}%
Eine CPU mit beschränkter Leistung kann diese Aufgabe nicht bewältigen.
Die Störungstheorie ist in der Lage, alle diese Einflüsse in ein einfacheres
Modell zusammenzufassen.
\index{Störungstheorie}%
{\em Daniel Bucher} und {\em Thomas Kistler} stellen diese interessante
Technik an einem einfachen Beispiel in Kapitel~\ref{chapter:perturbation}.
\index{Bucher, Daniel}%
\index{Kistler, Thomas}%
Die Störungsidee lässt sich auch auf andere Probleme anwenden.
Von besonderer Bedeutung ist das Eigenwertproblem für lineare Operatoren,
welches in der Quantenmechanik die Spektren von Atomen und Molekülen
zu berechnen gestattet.
\index{Quantenmechanik}%
\index{Spektrum}%
\index{Atom}%
\index{Molekül}%
Die vielfältigen Einflüsse sind jedoch in der Wellengleichung kaum
alle zu berücksichtigen.
\index{Wellengleichung}%
Die von {\em Nicolas Tobler} in Kapitel~\ref{chapter:ew} ausgeführte
Störungstheorie für das Eigenwertproblem löst dieses Problem mit
spektakulärem Erfolg, so dass wir heute auch die Feinheiten der
Spektrallinien zum Beispiel des Wasserstoffatoms verstehen.
\index{Tobler, Nicolas}%
\index{Wasserstoff}%

Polynome sind in manchen Fällen keine guten Approximationen,
Brüche können in solchen Fällen gute Näherungsbrüche liefern.
Kettenbrüche, von {\em Benjamin Bouhafs-Keller} in
Kapitel~\ref{chapter:kettenbruch} vorgestellt, liefern auf effiziente
Weise Näherungsbrüche für irrationale Zahlen, die sogar eine
Optimalitätseigenschaft besitzen.
\index{Kettenbruch}%
\index{Bouhafs-Keller, Benjamin}%
Die Padé-Approximation versucht Funktionen durch rationale
Funktionen, d.~h.~Brüche von Polynomen, anzunähern.
{\em Cédric Renda} zeigt in Kapitel~\ref{chapter:pade}, wie das
funktioniert.
Es zeigt sich, dass Kettenbrüche auf Padé-Approximationen führen.
Kapitel~\ref{chapter:arctan} rechnet nach, dass die Näherungsbrüche der
in Kapitel~\ref{chapter:kettenbruch} vorgestellt Kettenbruchentwicklung
der $\arctan x$-Funktion Padé-Approximanten sind.
\index{Renda, Cédric}%
\index{Polynom}%
\index{Padé-Approximation}%
\index{arctan@$\arctan x$}%

Die Gleichung von Burgers ist eine besonders einfache nichtlineare
partielle Differentialgleichung, die viele numerische Probleme
illustrieren kann.
\index{Burgers}%
\index{Gleichung von Burgers}%
{\em Michael Schmid} zeigt einige mögliche Ansätze in
Kapitel~\ref{chapter:burgers}.
\index{Schmid, Michael}%
In Abschnitt~\ref{section:finite-elemente} wurden die wichtigsten Ideen
der Methode der finiten Elemente an eindimensionalen Beispielen eingeführt.
{\em Joël Rechsteiner} zeigt in Kapitel~\ref{chapter:fem}, was sich in
zwei Dimensionen ändert.
\index{Rechsteiner, Joël}%
\index{finite Elemente}%

Die lineare Algebra ist ein besonders wichtiges Betätigungsfeld für
Numeriker.
Grundaufgaben der lineare Algebra wie Lösung eines Gleichungssystems,
Matrixzerlegungen oder das Eigenwertproblem werden in grossen
Simulationen immer wieder verwendet.
\index{Matrixzerlegung}%
\index{Unterer, Raphael}%
Entsprechend sind effiziente Algorithmen wichtig.
Das Kapitel~\ref{chapter:cg} von {\em Raphael Unterer} erklärt
ein besonderes erfolgreiches iteratives
Lösungsverfahren für Gleichungssysteme mit symmetrischer, positiv
definiter Koeffizienmatrix.
Der QR-Algorithmus löst zum Beispiel Least-Squares-Optimierungsprobleme,
\index{Least-Squares}%
{\em Manuel Tischhauser} zeigt in Kapitel~\ref{chapter:qr}, wie man mit
Givens-Drehungen eine effiziente Zerlegung finden kann.
\index{Tischhauser, Manuel}%
\index{QR-Algorithmus}%
Für das Eigenwertproblem hat John Francis in den 1960er Jahren einen
erfolgreichen Algorithmus formuliert, den {\em Tobias Grab} in
Kapitel~\ref{chapter:francis} vorstellt.
\index{Francis, John}%
\index{Grab, Tobias}%

Zum Trainieren von neuronalen Netzwerken mit Gradientabstieg muss 
der Gradient der Zielfunktion berechnet werden können.
{\em Martin Stypinsky} versucht in Kapitel~\ref{chapter:ableitung},
statt der üblichen Backpropagation
die Zielfunktion als Black-Box zu behandeln und mit numerischen
Ableitungsmethoden den Gradienten zu schätzen.
\index{Stypinsky, Martin}%
Alternativ zu dem auf der Taylor-Reihe abgeleitete Differetiationsverfahren
von Kapitel~\ref{chapter:ableitung} kann man auch versuchen, ähnliche
Verfahren aus der Interpolationstheorie abzuleiten, dies ist in
Kapitel~\ref{chapter:interdiff} durchgeführt.



