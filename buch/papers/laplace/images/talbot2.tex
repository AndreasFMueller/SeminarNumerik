%
% talbot2.tex -- Talbot Contour
%
% (c) 2020 Prof Dr Andreas Müller, Hochschule Rapperswil
%
\documentclass[tikz]{standalone}
\usepackage{amsmath}
\usepackage{times}
\usepackage{txfonts}
\usepackage{pgfplots}
\usepackage{csvsimple}
\usetikzlibrary{arrows,intersections,math}
\begin{document}
\def\skala{1}
\begin{tikzpicture}[>=latex,thick,scale=\skala]

\def\params{0.965}
\def\paraml{0.101}
\def\paramv{0.098953}
\def\s{9}

\pgfmathparse{\s*3.1415*\paraml*\paramv}
\xdef\y{\pgfmathresult}

\fill[color=gray!10] (-2.1,-\y) rectangle (10.1,\y);

\draw[->] (-2.1,0) -- (10.3,0) coordinate[label={\text{Re}}];
\draw[->] (0,-4.1) -- (0,4.3) coordinate[label={right:\text{Im}}];

\node at (10,4) {$\mathbb{C}$};

%\draw[line width=0.3pt] (-2.1,\y)--(2.1,\y);
%\draw[line width=0.3pt] (-2.1,-\y)--(2.1,-\y);
%\draw[line width=0.1pt] (-2.1,{0.5*\y})--(2.1,{0.5*\y});
%\draw[line width=0.1pt] (-2.1,{-0.5*\y})--(2.1,{-0.5*\y});
%\draw[line width=0.1pt] ({\params},{-0.5*\y}) -- ({\params},{0.5*\y});

%\draw ({\params+\paraml},{-0.1}) -- ({\params+\paraml},{0.1});

\node at (0,\y) [above left] {$i\lambda\nu\pi$};
\node at (0,-\y) [below left] {$-i\lambda\nu\pi$};

%\draw (\params,-0.1) -- (\params,0.1);
%\node at ({0.5*\params},-0.1) [below] {$\sigma\mathstrut$};

%\draw ({\params+\paraml},-0.1) -- ({\params+\paraml},0.1);
%\node at  ({\params+0.5*\paraml},-0.1) [below] {$\lambda\mathstrut$};

\xdef\x{2.5}
\xdef\y{3.0}
\pgfmathparse{atan(\y/\x)}
\xdef\winkel{\pgfmathresult}

\begin{scope}
\clip (-2.1,-4.1) rectangle (10.1,4.1);
\draw[color=red,line width=2pt]
	(\x,0) -- (\x,\y) arc (\winkel:{360-\winkel}:{sqrt(\x*\x+\y*\y)})
	-- (\x,0) -- cycle;
\end{scope}

\begin{scope}
\clip (-2.1,-4.1) rectangle (10.1,4.1);
\draw[color=blue,line width=2pt] plot[domain=-170:170,samples=100]
	({\s*(\params+\paraml*\x*cot(\x)*3.1415/180)},{\s*(\paraml*\paramv*\x*3.1415/180)});
\end{scope}

\definecolor{darkgreen}{rgb}{0,0.6,0}
\fill[color=darkgreen] (0,0) circle[radius=0.1];

\end{tikzpicture}
\end{document}

