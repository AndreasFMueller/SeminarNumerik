%
% einleitung.tex -- Beispiel-File für die Einleitung
%
% (c) 2020 Prof Dr Andreas Müller, Hochschule Rapperswil
%
Die Laplacetransformation ist eine Funktion $f(t)$, definiert auf dem Intervall von $[0, \infty)$ und absolut integrierbar auf jedem endlichen Intervall $[0, a)$. 
Die Laplacetransformation ist wie folgt definiert
\[
F(s) = \int_0^\infty e^{-st}f(t)\,dt,~~Re(s)>\gamma_{0}
\]
wobei $\gamma_{0}$ die Abzisse der Konvergenz der Laplacetransformation ist. 
Das inverse Laplaceproblem ist es Funktionen $f(t)$ im Zeitbereich aus jenen bekannten Funktionen $F(s)$ im Frequenzbereich zu rekonstruieren.

Mit der Riemmann inversions Formel ergibt sich $f(t)$ aus $F(s)$
\[
f(t) = \frac{1}{2\pi i} \oint_{B} e^{st}F(s)\,ds,~~t>0,~~i=\sqrt{-1}
\]
wobei $B$ die Bromwich Kontur von $\gamma-i\infty$ bis $\gamma+i\infty$, mit $\gamma>\gamma_{0}$, parallel zur imaginären Achse.
Diese Kontur befindet sich zur rechten aller Singularitäten von $F(s)$, falls $f(t)$ durch numerische quadratur berechnet wird.
Im Allgemeinen möchte man die Kontur zur linken Seite platzieren, sodass der Betrag des Faktors $e^{st}$ im Integrand kleiner wird.
Die Kontur sollte jedoch nicht zu nahe an Singularitäten von $F(s)$ angenähert werden, dies würde Spitzen der Integeralfunktion zur Folge haben.

