%
% loesung.tex -- Beispiel-File für die Beschreibung der Loesung
%
% (c) 2020 Prof Dr Andreas Müller, Hochschule Rapperswil
%

\documentclass{scrartcl}

\usepackage[utf8]{inputenc}
\usepackage[T1]{fontenc}
\usepackage{lmodern}
\usepackage[ngerman]{babel}
\usepackage{amsmath}
\usepackage{physics}
\usepackage{mathrsfs}

\begin{document}


\section{Lösung}
\label{laplace:section:Methode nach Talbot}
Mit der Riemmann inversions Formel ergibt sich $f(t)$ aus $F(s)$
\[
f(t) = \frac{1}{2\pi i} \oint_{B} \mathrm{e}^{st}F(s)\,\mathrm{d}s,~~t>0,~~i=\sqrt{-1}
\]
wobei $B$ die Bromwich Kontur von $\gamma-i\infty$ bis $\gamma+i\infty$, mit $\gamma>\gamma_{0}$, parallel zur imaginären Achse ist.
Diese Kontur befindet sich zur rechten aller Singularitäten von $F(s)$.
Falls $f(t)$ durch numerische quadratur berechnet wird.
Im Allgemeinen möchte man die Kontur zur linken Seite platzieren, sodass der Betrag des Faktors $\mathrm{e}^{st}$ im Integrand kleiner wird. Die Kontur sollte jedoch nicht zu nahe an Singularitäten von $F(s)$ angenähert werden, dies würde Spitzen der Integeralfunktion zur Folge haben.
Eine Bedingung um eine passende Kontur zu finden, wurde von Talbot gefunden.
Diese fordert, dass die Singularitäten von $F(s)$ bekannt sind, sowie folgendes gilt für $F(s)$

(a) $|F(s)|\rightarrow$ 0 mit $|s|\rightarrow$ $\infty$ in $Re(s)<\gamma_{0}$

(b) für alle Singularitäten $s_{j},~~|Im(s_{j})|<K$, wobei der Wert von K bekannt ist.

Die Talbots Kontur ist gegeben durch

\[
s(z) = \sigma+\lambda s_{\nu}(z),~~ z\in (-2\pi i,~~2\pi i)
\]
wobei

\[
s_{\nu}(z)=\frac{z}{1-e^{-z}}+\frac{z(\nu-1)}{2}
\]
In der Gleichung oben folgt mit $z=2i\theta$ die Parametrisierung
\[
s(\theta) = \sigma+\lambda s_{\nu}(\theta),~~ \theta\in (-\pi ,~~\pi)
\]
wobei

\[
s_{\nu}(\theta)=\theta \cot\theta+i\nu\theta,
\]
und das Riemannsche inversions Integral nimmt die folgende Form an
\[
f(t)=\frac{\lambda e^{\sigma}}{2\pi i}\int_{-\pi}^{\pi} \mathrm{e}^{-\lambda ts_{\nu}(\theta)}F[\sigma + \lambda s_{\nu}(\theta)]s'_{\nu}(\theta)\,\mathrm{d}\theta
\]
wobei
\[
s'_{\nu}(\theta) = i \Biggl\{\nu + \frac{\theta-\cos(\theta)\sin(\theta)}{\sin^{2}(\theta)}  \Biggr\}
\]
Unter berücksichtigung der Symmetrie, sowie der Trapez Approximationsregel erhält man
\[
\tilde{f}(t) = \frac{\lambda\mathrm{e}^{\sigma t}}{n}~T_{n}(t)
\]
wobei 
\[
T_{n}(t)=\sum_{j=0}^n " \mathrm{e}^{\lambda s_{\nu}(\theta_{j})} F[\sigma + \lambda s_{\nu}(\theta_{j})] \frac{1}{i} s'_{\nu}(\theta_{j})
\]

\[
\theta_{j} = j \frac{\pi}{n}.
\]
Die Notation $\sum_{j=0}^n "$ besagt, dass der Erste und Letzte Term der Summe halbiert werden. In diesem Falle wird der Term $j=n$ gleich 0 und der $j=0$ Term wird zu

\begin{center}$\frac{\nu}{2}\mathrm{e}^{\lambda t}F(\sigma + \lambda).$
\end{center}

Die obige beschriebene Methode wurde in Python implementiert.
Der Code ist auf Anfrage erhätlich.

\end{document}
