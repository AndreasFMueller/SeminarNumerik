%
% problemstellung.tex -- Beispiel-File für die Beschreibung des Problems
%
% (c) 2020 Severin Weiss
%



\section{Problemstellung
\label{laplace:section:problemstellung}}
\rhead{Problemstellung}
\subsection{Beispiel: Lösung einer linearen DGL mittels Laplacetransformation}
Wir wollen eine Lösung $f(t)$ aus $F(s)$ der linearen Differentialgleichung 
\[
\frac{df}{dt} + \lambda f(t) = 0
\]
Mit der Laplactransformation folgt
\[
(sF(s) - f_{0}) + \lambda F(s) = 0
\]
Die obige Gleichung im Frequenzbereich kann nach $F(s)$ aufgelöst werden.
Es folgt somit
\[
F(s) = \frac{f_{0}}{s + \lambda}
\]
Um die Laplaceinverse von $F(s)$ zu finden existieren für gewisse Funktionstypen tabellierte zugehörige Rücktransformationen.
Für die obige Funktion $F(s)$ ergibt sich zum Beispiel
\[
\mathcal{L}^{-1}\{F(s)\}=\mathcal{L}^{-1}\{\frac{f_{0}}{s+\lambda}\} = f_{0}e^{-\lambda t}
\]
Eine solche Tabelle ist jeweils in Literatur zu finden, welche sich mit der Laplacetransformation beschäftigt.
Wenn für die gesuchte Funktion $F(s)$ keine zugehörige Funktion tabelliert ist, muss das Integral wie in der Einleitung beschrieben numerisch berechnet werden.

