%
% problemstellung.tex -- Beispiel-File für die Beschreibung des Problems
%
% (c) 2020 Prof Dr Andreas Müller, Hochschule Rapperswil
%

\documentclass{scrartcl}

\usepackage[utf8]{inputenc}
\usepackage[T1]{fontenc}
\usepackage{lmodern}
\usepackage[ngerman]{babel}
\usepackage{amsmath}
\usepackage{physics}
\usepackage{mathrsfs}

\begin{document}
\section{Problemstellung
\label{laplace:section:problemstellung}}
\subsection{Beispiel: Lösung einer linearen DGL mittels Laplacetransformation}
Wir wollen eine Lösung $f(t)$ aus $F(s)$ der linearen Differentialgleichung 
\[
\dv{f}{t} + \lambda f(t) = 0
\]
Mit der Laplactransformation folgt
\[
(sF(s) - f_{0}) + \lambda F(s) = 0
\]
Die obige Gleichung im Frequenzbereich kann nach F(s) aufgelöst werden.
Es folgt somit
\[
F(s) = \frac{f_{0}}{s + \lambda}
\]
Um die Inverse von $F(s)$ zu finden existieren für gewisse Funktionstypen tabellierte zugehörige Rücktransformationen.
Für die obige Funktion $F(s)$ ergibt sich nämlich
\[
\mathscr{L}^{-1}\{F(s)\}=\mathscr{L}^{-1}\{\frac{f_{0}}{s+\lambda}\} = f_{0}\mathrm{e}^{-\lambda t}
\]
Eine solche Tabelle ist jeweils in Literatur zu finden, welche sich mit der Laplacetransformation beschäftigt.
Wenn für die gesuchte Funktion $F(s)$ keine zugehörige Funktion tabelliert ist, muss das Integral wie in der Einleitung beschrieben gelöst werden.

\end{document}
