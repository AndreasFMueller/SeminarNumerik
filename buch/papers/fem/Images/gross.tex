%
% gross.tex -- groesseres Gitter
%
% (c) 2020 Prof Dr Andreas Müller, Hochschule Rapperswil
%
\documentclass[tikz]{standalone}
\usepackage{amsmath}
\usepackage{times}
\usepackage{txfonts}
\usepackage{pgfplots}
\usepackage{csvsimple}
\usetikzlibrary{arrows,intersections,math}
\begin{document}
\def\skala{1}
\begin{tikzpicture}[>=latex,thick,scale=\skala]

\foreach \x in {0,...,4}{
	\foreach \y in {0,...,4}{
		\fill[color=red!20] (\x,\y) -- ({\x+1},{\y+1}) -- (\x,{\y+1}) --cycle;
		\fill[color=blue!20] (\x,\y) -- ({\x+1},\y) -- ({\x+1},{\y+1}) --cycle;
	}
}
\foreach \x in {0,...,5}{
	\draw (\x,0) -- (\x,5);
}
\foreach \y in {0,...,5}{
	\draw (0,\y) -- (5,\y);
}
\foreach \x in {0,...,4}{
	\foreach \y in {0,...,4}{
		\draw (\x,\y) -- ({\x+1},{\y+1});
	}
}

\foreach \x in {0,...,5}{
	\foreach \y in {0,...,5}{
		\fill (\x,\y) circle[radius=0.08];
	}
}

\foreach \x in {0,...,5}{
	\node at (\x,0) [above left] {$\x$};
}
\foreach \x in {6,...,11}{
	\node at ({\x-6},1) [above left] {$\x$};
}
\foreach \x in {12,...,17}{
	\node at ({\x-12},2) [above left] {$\x$};
}
\foreach \x in {18,...,23}{
	\node at ({\x-18},3) [above left] {$\x$};
}
\foreach \x in {24,...,29}{
	\node at ({\x-24},4) [above left] {$\x$};
}
\foreach \x in {30,...,35}{
	\node at ({\x-30},5) [above left] {$\x$};
}

\def\xstep{0.145}
\def\ystep{-0.145}

\def\quadrat#1#2#3{
	\fill[color=#3,opacity=0.2]
		({({#1}-0.5)*\xstep},{({#2}-0.5)*\ystep})
		rectangle
		({({#1}+0.5)*\xstep},{({#2}+0.5)*\ystep});
}

\def\oberesdreieck#1{
	\quadrat{#1}{#1}{red}
	\quadrat{#1+6}{#1}{red}
	\quadrat{#1+7}{#1}{red}
	\quadrat{#1}{#1+6}{red}
	\quadrat{#1+6}{#1+6}{red}
	\quadrat{#1+7}{#1+6}{red}
	\quadrat{#1}{#1+7}{red}
	\quadrat{#1+6}{#1+7}{red}
	\quadrat{#1+7}{#1+7}{red}
}

\def\unteresdreieck#1{
	\quadrat{#1}{#1}{blue}
	\quadrat{#1+1}{#1}{blue}
	\quadrat{#1+7}{#1}{blue}
	\quadrat{#1}{#1+1}{blue}
	\quadrat{#1+1}{#1+1}{blue}
	\quadrat{#1+7}{#1+1}{blue}
	\quadrat{#1}{#1+7}{blue}
	\quadrat{#1+1}{#1+7}{blue}
	\quadrat{#1+7}{#1+7}{blue}
}

\begin{scope}[xshift=7cm,yshift=5cm]

\foreach \i in {0,...,4}{
	\foreach \j in {0,...,4}{
		\oberesdreieck{\i+6*\j}
		\unteresdreieck{\i+6*\j}
	}
}

\end{scope}

\node at (9.2,2.45) {$\displaystyle 
B=\begin{pmatrix}
\hspace*{5.5cm}\\
\\
\\
\\
\\
\\
\\
\\
\\
\\
\\
\\
\\
\end{pmatrix}$};

\end{tikzpicture}
\end{document}

