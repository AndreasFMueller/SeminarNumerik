%
% dreiecke.tex -- Dreiecke Kombination der Matrizen
%
% (c) 2020 Prof Dr Andreas Müller, Hochschule Rapperswil
%
\documentclass[tikz]{standalone}
\usepackage{amsmath}
\usepackage{times}
\usepackage{txfonts}
\usepackage{pgfplots}
\usepackage{csvsimple}
\usetikzlibrary{arrows,intersections,math}
\begin{document}
\def\skala{1}
\begin{tikzpicture}[>=latex,thick,scale=\skala]

\definecolor{farbe1}{rgb}{0.8,0,0}
\definecolor{farbe2}{rgb}{0,0,1}

\coordinate (C1) at (0,0);
\coordinate (C2) at (2,0);
\coordinate (C3) at (4,1);
\coordinate (C4) at (1,2);

\coordinate (S1) at (1,0.6666);
\coordinate (S2) at (2.333,1);

\fill[color=farbe1!20] (C1) -- (C2) -- (C4) -- cycle;
\fill[color=farbe2!20] (C2) -- (C3) -- (C4) -- cycle;

\draw[color=farbe1] (C1)--(C2)--(C4)--cycle;
\draw[color=farbe2] (C2)--(C3)--(C4)--cycle;

\node[color=farbe1] at (S1) {$\triangle_1$};
\node[color=farbe2] at (S2) {$\triangle_2$};

\fill (C1) circle[radius=0.08];
\fill (C2) circle[radius=0.08];
\fill (C3) circle[radius=0.08];
\fill (C4) circle[radius=0.08];

\node at (C1) [below left] {$c_1$};
\node at (C2) [below] {$c_2$};
\node at (C3) [right] {$c_3$};
\node at (C4) [above] {$c_4$};

\end{tikzpicture}
\end{document}

