%
% problemstellung.tex -- Beispiel-File für die Beschreibung des Problems
%
% (c) 2020 Prof Dr Andreas Müller, Hochschule Rapperswil
%
\section{Problemstellung
\label{fem:section:problemstellung}}
\rhead{Problemstellung}
Anforderung an $h(x)$:

\begin{itemize}
	\item 1. stetig Differenzierbar (Steigung)
	\item 2. stetig Differenzierbar (Krümmung)
	\item an Knoten muss $h(x) = 0$ sein
\end{itemize}

Poisson-Gleichung:

\begin{equation}
\frac{\partial^2 u(x,y)}{\partial x^2} \frac{\partial^2 u(x,y)}{\partial y^2} = - u(x,y)  \in \Omega
\label{fem:equation5}
\end{equation}

Randbedingungen:
\begin{equation}
u = 0, (x,y)\in \Omega
\label{fem:rand1}
\end{equation}

\begin{equation}
u = U (x,y)\in \Omega
\label{fem:rand2}
\end{equation}

\begin{equation}
\frac{\partial u}{n} = 0, (x,y)\in \Omega
\label{fem:rand3}
\end{equation}
 
$\Rightarrow$ 4 Parameter $\Rightarrow$ Polynom 3. Grades\\



\begin{equation}
\iint_{\!\!\!\!\!\!\!\Omega} \limits (u_2^2 + u_y^2)) \,dx dy
\label{fem:equation1}
\end{equation}

\begin{equation}
\iint_{\!\!\!\!\!\!\!\Omega} \limits u^2  \,dx dy
\label{fem:equation2}
\end{equation}

\begin{equation}
\iint_{\!\!\!\!\!\!\!\Omega} \limits u  \,dx dy
\label{fem:equation3}
\end{equation}



Und daraus müsste dann die Lösung gefunden werden um die Koeffizienten der folgenden Gleichung zu finden.

\begin{equation}
\int g(x) \space dx = f_i \int h_0 \space dx+ f_{i+1}\int h_1 \space dx + s_i\int h_0^1 dx + s_{i+1}\int h_1^1 dx
\label{fem:equation5}
\end{equation}


\subsection{De finibus bonorum et malorum
\label{fem:subsection:finibus}}

\begin{equation}
\int_a^b x^2\, dx
=
\left[ \frac13 x^3 \right]_a^b
=
\frac{b^3-a^3}3.
\label{fem:equation1}
\end{equation}



