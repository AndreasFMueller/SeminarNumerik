%
% problemstellung.tex -- Beispiel-File für die Beschreibung des Problems
%
% (c) 2020 Prof Dr Andreas Müller, Hochschule Rapperswil
%
\section{Problemstellung
\label{fem:section:problemstellung}}
\rhead{Problemstellung}

\subsection{Was sind Ansatzfunktionen?}
Um eine gesuchte Funktion als Lösung für das Differentialgleichung (DGL) Problem in der Ebene zu finden, wird diese mit einfacheren Funktionen approximiert. Diese einfacheren Approximations- Funktionen werden in diesem Kapitel als Ansatzfunktion bezeichnet. 

Die Ansatzfunktion ist frei wählbar und gibt auch den Freiheitsgrad vor. Je höher die Ordnung der Ansatzfunktion desto besser soll die Approximation werden. So die Hoffnung. Allerdings hat dies auch Konsequenzen, die im Verlauf dieses Kapitels beschrieben werden. 


\subsection{Herausforderung FEM in der Ebene}
Wie bereits beschrieben und bewiesen im Kapitel xXx kann eine DGL in ein äquvalentes Minimalproblem übersetzt werden. Das Minimalproblem muss Element- bzw. Stückweise mit den Ansatzfunktionen formuliert werden. Die DGL in der Ebene hat allerdings zu der DGL im eindimensionalen Raum die Form:

\begin{equation}
	\Delta u = \lambda u
	\label{fem:DGL2D}
\end{equation} 
während der Laplace Operator die 2. Ableitungen nach den beiden Variablen darstellt.

\begin{equation}
	\Delta = \frac{\partial ^2}{\partial x^2} + \frac{\partial ^2}{\partial y^2}
\end{equation} 
Daraus lässt sich erkennen, dass sich die Vorgehensweise angepasst werden muss, da zweifache Differenzierbarkeit gefordert ist. Zudem muss die Ansatzfunktion die Approximation genügen genau beschreiben. Die Randbedingungen der DGL die erfüllt werden müssen lauten:
\begin{itemize}
	\item 1. stetig differenzierbar in den Stützstellen (Steigung) wie z.B: im Dreieck in den Ecken.
	\item 2. stetig differenzierbar an den Übergängen an den Rändern eines Elements zum anliegenden Rand eines benachbarten Elements. (Krümmung)
\end{itemize}
%Poisson-Gleichung:

%\begin{equation}
%\frac{\partial^2 u(x,y)}{\partial x^2} \frac{\partial^2 u(x,y)}{\partial y^2} = - u(x,y)  \in %\Omega
%\label{fem:equation5}
%\end{equation}

%Randbedingungen:
%\begin{equation}
%u = 0, (x,y)\in \Omega
%\label{fem:rand1}
%\end{equation}

%\begin{equation}
%u = U (x,y)\in \Omega
%\label{fem:rand2}
%\end{equation}

%\begin{equation}
%\frac{\partial u}{n} = 0, (x,y)\in \Omega
%\label{fem:rand3}
%\end{equation}
 
%$\Rightarrow$ 4 Parameter $\Rightarrow$ Polynom 3. Grades\\


%\begin{equation}
%\iint_{\!\!\!\!\!\!\!\Omega} \limits (u_2^2 + u_y^2)) \,dx dy
%\label{fem:equation1}
%\end{equation}

%\begin{equation}
%\iint_{\!\!\!\!\!\!\!\Omega} \limits u^2  \,dx dy
%\label{fem:equation2}
%\end{equation}

%\begin{equation}
%\iint_{\!\!\!\!\!\!\!\Omega} \limits u  \,dx dy
%\label{fem:equation3}
%\end{equation}
Daraus müsste dann die Lösung gefunden werden um die Koeffizienten der folgenden Gleichung zu finden.

\begin{equation}
\int g(x) \space dx = f_i \int h_0 \space dx+ f_{i+1}\int h_1 \space dx + s_i\int h_0^1 dx + s_{i+1}\int h_1^1 dx
\label{fem:equation5}
\end{equation}
Die gewählte Ansatzfunktion muss dann noch auf die gewählte Form z.B. ein Dreieck angewendet werden. Die Interpolation des Gebiet kann mit unterschiedlichen Formen vorgenommen werden. Jenach dem kann mit einem Dreieck oder einem Parallelogramm das Gebiet approximiert werden. Ein Kreis wäre auch denkbar, allerdings ist der Gewinn im Verhältniss zum Rechenaufwand nicht gerechtfertig. Da es verschiedene Dreieck- Arten gibt sowie auch verschiedene Parallelogramme, wird in xXx auch eine Lösung aufgezeigt wie mit Hilfe einer Transformation das Dreieck Flächenelements in ein weniger aufwändigere berechenbare Flächenelement überführt werden kann.

%\subsection{De finibus bonorum et malorum
%\label{fem:subsection:finibus}}

%\begin{equation}
%\int_a^b x^2\, dx
%=
%\left[ \frac13 x^3 \right]_a^b
%=
%\frac{b^3-a^3}3.
%\label{fem:equation1}
%\end{equation}



