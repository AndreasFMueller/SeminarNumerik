%
% loesung.tex -- Beispiel-File für die Beschreibung der Loesung
%
% (c) 2020 Prof Dr Andreas Müller, Hochschule Rapperswil
%
\section{Lösung
\label{fem:section:loesung}}
\rhead{Lösung}
Zuerst muss die DGL in ein äqualentes Minimalprobelm übersetzt werden in der Form

\begin{equation}
14.1 \Rightarrow \int_\Omega (\nabla u)^2 +\lambda u^2 dx
\label{fem:equation3}
\end{equation}

Approximation der unbekannten Funktion mit 

\begin{equation}
u(x,y) =  \sum_{k=0} N_k^e(x,y) \cdot u(x,y)_k^e
\label{fem:equation3}
\end{equation}

N = Formfunktion des Elementes e und Knoten k

\begin{equation}
A =  \sum_{k=0} \iint_{\!\!\!\!\!\!\!\Omega} \epsilon \cdot \nabla N_k(x,y) \cdot N_l(x,y) dS = \sum \iint_{\!\!\!\!\!\!\!\Omega} h(x) \cdot N_l(x,y) dS, l = 1,2,...N
\label{fem:equation3}
\end{equation}

Zu lösendes GL- System
\begin{equation}
Au(x,y) = b
\label{fem:GL}
\end{equation}

wobei A eine Matrix, u der Vektor der Unbekannten und b der Ansatzfunktionenvektor ist.

\subsection{linearer Ansatzfunktion
\label{fem:subsection:bonorum}}

\begin{equation}
u(x,y) = c_1 + c_2x + c_3y
\label{fem:equation3}
\end{equation}

mit einem Allgemeinen Dreieck:

\begin{equation}
N(x,y) = \int_a^b\int_c^{d-(d-c)(x-a)/(b-a)} u(x,y) dxdy
\label{fem:Dreieck_alg}
\end{equation}

Gemäss Abschnitt 2.2.2 Buch von Schwarz sind die Werte entlang einer Seite eines Dreiecks- Elements gleich  der angrenzenden Linie eines anderen Dreiecks, wenn die Eckpunktwerte gleich sind. Die Frage Warum soll dass so sein? Weil Funktion $h(x)$ bei jedem Element gleich ist sowie es sich bei $h(x)$ hierbei um eine Lineare Funktion handelt.

\subsection{Quadratischer Ansatz
\label{fem:subsection:bonorum}}
 Abschnitt 
