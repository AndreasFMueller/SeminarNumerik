%
% einleitung.tex -- Beispiel-File für die Einleitung
%
% (c) 2020 Prof Dr Andreas Müller, Hochschule Rapperswil
%
\section{Einleitung\label{fem:section:einleitung}}
\rhead{Einleitung}
Finite Elemente in der Ebene bzw. Finite Elemente Methode (FEM) in einer höheren Dimension als 1, haben komplexere Regeln, die beachtet werden müssen. Die zweidimensionale Elemente werden benötigt um stationäre oder instationäre Feldprobleme zu berechnen wie z.B. die Simulation von magnetischen Feldlinien. In diesem Kapitel sind Ideen beschrieben wie die komplexeren Reglen im 2D Fall angewendet werden. Das Rezept für das Vorgehen bei den Finiten Element Methode in der Ebene ist analog, zur FEM im 1 Dimensionalen Raum, vom Konzept hergesehen. Auch hier ist das Ziel ein allgemeines Vorgehen zu beschreiben, um die komplizierten Integrale in ein lineares Gleichungs- System zu überführen. Im Unterschied zur FEM in 1. Dimension, wird es bei FEM in der Ebene keine perfekte Lösung geben wie z.B. die Splines. Das Verstehens im 2 Dimensionalen Fall bringt den Vorteil mit sich, dass die Vorgehensweise zur FEM im 3 Dimensionalen Raum identisch ist. Dies gilt auch für noch höhere FEM Dimensionen. Dieses  Kapitel beschränkt sich auf die 2 Dimension bzw. FEM in der Ebene. \\
Das Grundkonzept der Finiten Element Methode ist im Kapitel 7 beschrieben, falls der Leser sich nochmals eine Übersicht zu den Finiten Element verschaffen möchte.

In den nächsten folgenden Zeilen wird drauf eingeangen, wie die einzelnen Schritte umgesetzt werden und insbesondere werden die jeweilige Nutzen dargestellt.

