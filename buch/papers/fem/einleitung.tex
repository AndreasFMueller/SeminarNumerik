%
% einleitung.tex -- Beispiel-File für die Einleitung
%
% (c) 2020 Prof Dr Andreas Müller, Hochschule Rapperswil
%
\section{Einleitung\label{fem:section:einleitung}}
\rhead{Einleitung}
Finite Elemente in der Ebene bzw. Finite Elemente Methode (FEM) in einer höheren Dimension als 1 haben ein paar zusätzliche Regeln die beachtet werden müssen. Die zweidimensionale Elemente werden benötigt stationäre oder instationäre Feldprobleme zu behandeln wie z.B. die Simulation des magnetischen Feldlinien. In diesem Kapitel finden sie die Idee bzw. die Beschreibung dieser zusätzlichen Reglen. Das Kochrezept für das Vorgehen bei den Finiten Element Methode in der Ebene ist analog zur FEM im 1 Dimensionalen Raum vom Konzept hergesehen. Die zusätzlichen komplexeren Regeln oder Ansprüche im mehrdimensionalen Fall, welche beachtet werden müssen, waren in der Dimension 1einfach gegeben oder sogar geschenkt. Der Vorteil des Verstehens im 2 Dimensionalen Fall bringt den Vorteil mit sich, dass die Vorgehensweise zur FEM im 3 Dimensionalen Raum identisch ist. Dies gilt auch für noch höhere FEM Dimensionen. Dieses  "Buch/Kapitel" beschränkt sich auf die 2. Ordnung bzw. FEM in der Ebene. Es handelt sich um Grund- Konzepte, welche Überlegungen erfordern, die in diesem Kapitel aufgezeigt werden.\\

Das Grundkonzept der Finiten Element Methode ist im Kapitel 7 beschrieben, falls der Leser sich nochmals eine Übersicht zu den Finiten Element verschaffen möchte.


In den nächsten folgenden Zeilen soll drauf eingeangen werden, wie die einzelnen Schritte umgesetzt werden in der Ebene und insbesondere die jeweilige Nutzung dargestellt werden.

