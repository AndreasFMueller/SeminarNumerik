%
% einleitung.tex -- Beispiel-File für die Einleitung
%
% (c) 2020 Prof Dr Andreas Müller, Hochschule Rapperswil
%
\section{Einleitung\label{fem:section:einleitung}}
\rhead{Einleitung}
Das Grundkonzept der Finiten Element Methode ist im Kapitel 7
beschrieben. Das aktuelle Kapitel beschränkt sich auf zwei Dimensionen
bzw. auf FEM in der Ebene.
Im Unterschied zur FEM in einer Dimension,
wird es bei FEM in der Ebene keine perfekte Lösung geben wie z.~B.~die
Splines.
Das Verstehen im zweidimensionalen Fall bringt den
Vorteil mit sich, dass die Vorgehensweise zum Lösen der FEM im
dreidimensionalen Raum identisch ist.
Dies gilt auch für noch höhere
FEM Dimensionen.
Auch hier ist das Ziel ein partielle Differentialgleichung
(PDE) numerisch zu lösen.
\index{partielle Differentialgleichung}%
\index{Differentialgleichung, partielle}%
Es wird ein allgemeines Vorgehen beschrieben,
um die umständlichen Ableitungen in Integralterme zu verwandeln und
diese komplizierten Integrale wiederum in ein lineares Gleichungssystem
zu überführen.
Die Gleichungssysteme können dann mit einem Rechner
gelöst  werden.
Solche numerische Approximationen werden z.~B. in
der Simulation von Feldlinien benötigt.
\index{Simulation}%
\index{Feldlinien}%
In dem folgenden Abschnitt wird drauf eingegangen, wie die einzelnen
Schritte umgesetzt werden und insbesondere werden die jeweiligen Nutzen
dargestellt.

