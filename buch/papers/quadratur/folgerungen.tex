%
% problemstellung.tex -- Beispiel-File für die Beschreibung des Problems
%
% (c) 2020 Prof Dr Andreas Müller, Hochschule Rapperswil
%
\section{Folgerungen
\label{quadratur:section:folgerungen}}
\rhead{Folgerungen}

\subsection{Vergleich Gauss-Quadratur mit Trapezregel}
In diesem Abschnitt werden die Resultate der Gauss-Quadratur mit Resultaten der Trapezregel
verglichen, um zu zeigen wie effizient die Integration mit der Gausss-Integration ist.
Dazu wird nochmals die Funktion $\sqrt{1-x^{2}}$ Integriert. 
Die Resultate sind in der Tabelle~\ref{buch:table:gaussvergleich} dargestellt.
Als Referenz: Das analytische Resultat des Integrals ist $\frac{\pi}{2} \approx 1.5707963268$.
\begin{table}
    \centering
    \begin{tabular}{|>{$}c<{$}|>{$}c<{$}|>{$}c<{$}|>{$}c<{$}|>{$}c<{$}|}
        \hline
        n & \text{Gauss-Quadratur} &  \text{Gauss-Quadratur} & \text{Trapezregel} & \text{Trapezregel} \\
         & \text{Resultat} &  \text{Fehler} & \text{Resultat} & \text{Fehler} \\
        \hline  
        2 & 1.6329931619 & 0.3670068381 & 0.0000000000 & 1.5707963268 \\
        3 & 1.5916172578 & 0.0413759040 & 1.0000000000 & 0.5707963268 \\
        4 & 1.5802775277 & 0.0113397301 & 1.2570787221 & 0.3137176047 \\
        5 & 1.5759063349 & 0.0043711928 & 1.3660254038 & 0.2047709230 \\
        7 & 1.5727819554 & 0.0010820282 & 1.4587766894 & 0.1120196374 \\
        10 & 1.5715139556 & 0.0002531675 & 1.5096159164 & 0.0611804104 \\
        20 & 1.5708921460 & 1.55366 \cdot 10^{-5} & 1.5507798233 & 0.020016503 \\
        30 & 1.5708253858 & 3.05870 \cdot 10^{-6} & 1.5601695280 & 0.010626799 \\
        40 & 1.5708087326 & 9.66660 \cdot 10^{-7} & 1.5639786384 & 0.006817688 \\
        50 & 1.5708027245 & 3.95730 \cdot 10^{-7} & 1.5659537235 & 0.004842603 \\
        100 & 1.5707971383 & 2.47153 \cdot 10^{-8} & 1.5691090196 & 0.001687307 \\
        \hline
    \end{tabular}
    \caption{Resultatvergleich zwischen der Gauss-Quadratur und der Trapezregel
\index{Resultatvergleich}
    \label{buch:table:gaussvergleich}}   
\end{table}
Man kann sehr schön sehen, dass die Resultate der Gauss-Quadratur mit nur wenigen Funktionsauswertungen
schon sehr geringe Abweichungen vom analytischen Resultat haben, während bei der Trapezregel
viel mehr Funktionsauswertungen benötigt werden, um die selbe Genauigkeit zu erreichen.
Man sieht, dass die Genauigkeit der Trapezregel mit $n = 100$ bei der Gauss-Quadratur bereits mit $n = 7$
erreicht wird.

\subsection{Programm für Gauss-Quadratur in Python}
\index{Python}%
In Python lässt sich die Gauss-Quadratur in gerade mal $25$ Zeilen Code umsetzen.
Dafür wird aus dem Modul \texttt{scipy} die Klasse \texttt{integrate} importiert. 
Ausserdem wird für die mathematischen Funktionen das Modul \texttt{numpy} verwendet
\index{numpy@\texttt{numpy}}
\begin{lstlisting}[float,language=Python,style=Python]
    import numpy as np
    from scipy import integrate

    # Warnungen werden erzeugt, wenn mit der gewaehlten Anzahl
    # Stuetzstellen die Toleranz  tol=1.49e-08 nicht erreicht werden kann

    # Zu integrierende Funktion
    func = lambda x: np.sqrt(1 - x ** 2)

    # Anzahl Stuetzstellen in einem Array. 
    nodes = [1, 2, 3, 4, 5, 7, 10, 20, 30, 40, 50, 100]

    # Quadraturen mit node = Anzahl Stuetzstellen
    for node in nodes:
        x = np.linspace(-1, 1, node)
        y = func(x)
        print(f"Number of nodes: {node}")
        # Berechnung mit Gauss-Quadratur
        print(f"Gauss-Legendre : {integrate.quadrature(func, -1, 1, maxiter=node)}")
        # Berechnung mit Trapezformel
        trapez = integrate.trapz(y, x)
        print(f"Trapezoidal Rule: {trapez}")
        # Berechnung mit Simpsonscher Regel
        print(f"Simpson's Rule : {integrate.simps(y, x)}")
        print()
\end{lstlisting}


\subsection{Fazit}
Die Gauss-Quadratur ist ein numerisches Integrationsverfahren, welches
mit wenigen Funktionsauswertungen das bestimmte Integral von beliebigen
Funktionen sehr genau berechnen kann. 
Polynome oder Annäherungen von Funktionen mittels Polynomen lassen sich 
sogar exakt berechnen und benötigen für ein Polynom vom Grad $2n+1$ nur 
$n$ Funktionsauswertungen.

Es gibt verschiedene Wege für die Bestimmung der Stützstellen und Gewichte.
Vorgerechnete Werte für $x_{i}$ und $A_{i}$ sind für grosse $n$ in Tabellen
vorhanden und erlauben schnelle Lookups. Für grössere $n$ existieren
Formeln, die die Berechnung der Stützstellen und Gewichte erlauben und
in Computerprogrammen einfach umgesetzt werden können.
Für verschiedene Integralgrenzen gibt es eigene Formen der Gauss-Quadratur
die auf orthogonalen Polynomen aufbauen, deren Nullstellen die Stützstellen
bestimmen.







