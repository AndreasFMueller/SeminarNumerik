%
% einleitung.tex -- Beispiel-File für die Einleitung
%
% (c) 2020 Prof Dr Andreas Müller, Hochschule Rapperswil
%
\section{Einleitung\label{quadratur:section:einleitung}}
\rhead{Einleitung}

Im Kapitel~\ref{chapter:integration}, Integration, wurden die 
Trapezregel und die Mittelpunktsregel für die numerische Integration, 
in der Mathematik auch Quadratur genannt, erklärt. 
In diesem Kapitel wird eine weitere Methode, die Gauss-Quadratur, erarbeitet.
\noindent
Die Gauss-Quadratatur ist ein Verfahren, welches ein bestimmtes Integral der Form

\begin{equation}
    \int_{a}^{b} f(x) \,dx
\end{equation}
\noindent
mit der approximierten Summenformel 

\begin{equation}
    I = \sum_{i=0}^{n} A_i f(x_i)
\end{equation}
\noindent
annähert, wobei die Stützstellen $x_i$ und die Gewichtung $A_i$ von der gewählten 
Methode abhängen. 
Es werden dabei im Vergleich mit ähnlichen Verfahren viel weniger Funktionsauswertungen benötigt.
\noindent
Methoden für die Quadratur lassen sich in zwei Gruppen unterteilen: 
Newton-Cotes Formeln und Gauss Quadratur.
Die im Kapitel \ref{buch:subsection:mittelpunkt} und \ref{buch:subsection:trapez} beschriebenen
Trapezregel und Mittelpunktsregel gehören zu der ersten Kategorie.
Newton-Cotes Formeln lassen sich dadurch erkennen, dass die Stützstellen auf der x-Achse 
gleichmässig verteilt sind und eignen sich besonders für die Integration, wenn sich $f(x)$ 
effizient in kleinen Intervallen berechnen lässt oder bereits von Computern berechnet wurden.

\newpage





