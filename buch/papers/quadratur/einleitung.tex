%
% einleitung.tex -- Beispiel-File für die Einleitung
%
% (c) 2020 Prof Dr Andreas Müller, Hochschule Rapperswil
%
\section{Einleitung\label{quadratur:section:einleitung}}
\rhead{Einleitung}

Die numerische Integration, auch Quadratur genannt, approximiert
 ein bestimmtes Integral der Form
\begin{equation}
    \int_{a}^{b} f(x) \,dx
\end{equation}
mit der approximierten summenformel 
\begin{equation}
    I = \sum_{i=0}^{n} A_i f(x_i)
\end{equation}
wobei die abscissen $x_i$ und die gewichtung $A_i$ von der gewählten 
Methode abhängen. 
Methoden für die Quadratur lassen sich in zwei Gruppen unterteilen: 
Newton-Cotes Formeln und Gauss Quadratur.
Die im Kapitel \ref{chapter:integration} beschriebene Trapezregel 
gehört zu der ersten Kategorie.
Newton-Cotes Formeln lassen sich dadurch erkennen, dass die Abscissenwerte 
gleichmässig verteilt sind und eignen sich besonders für die Integration, wenn sich $f(x)$ 
effizient in kleinen Intervallen berechnen lässt oder bereits von Computern berechnet wurden.
Das Gauss-Quadratur-Verfahren, oder auch Gauss-Integrations-Verfahren, 
ist ein numerisches Integrationsverfahren, welches beliebige Integrale der Form
\begin{equation}
\int_{a}^{b} w(x) f(x)\,dx
\end{equation}
mit nur wenigen Funktionsauswertungen sehr genau annähern kann, 
solange sich die Funktion $f(x)$ überhaupt integrieren lässt. 
Die Position der abscissen ist dabei so gewählt, 
dass die bestmögliche Genauigkeit bei der berechnung des Integrals erreicht wird.
Dadurch wird für die berechnung eines beliebigen Genauigkeitsgrades weniger Berechnungschritte benötigt.




