%
% problemstellung.tex -- Beispiel-File für die Beschreibung des Problems
%
% (c) 2020 Prof Dr Andreas Müller, Hochschule Rapperswil
%
\section{Problemstellung
\label{quadratur:section:problemstellung}}
\rhead{Problemstellung}
\subsection{Überlegung hinter der Gauss-Quadratur \label{quadratur:subsection:ueberlegung}}

Bei der Integration mit der Trapez-/Mittelpunktregel kann man die Genauigkeit
der Annäherung verbessern, indem man mehr Teilintervalle berechnet.
Dabei werden mehr Stützstellen auf der x-Achse benötigt und mehr Funktionsauswertungen
müssen berechnet werden.

Die Wahl der Position und Anzahl Stützstellen stellt hierbei einer der beiden Freiheitsgrade
in der Berechnung des Integrals dar.
Der zweite Freiheitsgrad ist die Verwendung einer Gewichtung für eine spezifische Stützstelle.
Das hinzuziehen einer Gewichtung erlaubt es, in einem Teilintervall mit hohem potenziellem 
Fehler, diesen Fehler zu minimieren. 

Die Gauss-Quadratur basiert auf der Idee, dass die Stützstellen und deren Gewichtung so gewählt werden,
dass das Resultat nun optimiert wird und möglichst genau berechnet werden kann.
Dieses Verfahren erlaubt es, ein Polynom vom Grad $n$ mit nur $\frac{n-1}{2}$
Funktionsauswertungen exakt zu berechnen.

\subsection{Anwendungsbereiche der numerischen Integration \label{quadratur:subsection:anwendungsbereiche}}
Die numerische Integration kann, wie im Kapitel~\ref{chapter:integration} beschrieben, dann verwendet
werden, wenn keine Stammfunktion in analytischer Form für eine Funktion gefunden werden kann.
Ein bekanntes Beispiel dafür ist die Wahrscheinlichkeitsverteilung der Standardnormalverteilung:
\begin{equation}
\Phi(x) 
=
\frac{1}{\sqrt{2\pi}}
\int_{-\infty}^x e^{-t^2/2}\,dt
\end{equation}

Weitere Beispiele mathematischer Funktionen ohne elementare Stammfunktionen können aus 
der Tabelle~\ref{buch:table:funktionenohnestammfunktion} entnommen werden.

\begin{table}[h!]
    \begin{itemize}
        \item $\frac{sin(x)}{x}$
        \item $\frac{x}{sin(x)}$
        \item $\frac{e^{x}}{x}$
        \item $\frac{1}{ln(x)}$
        \item $\frac{x}{ln(x)}$
        \item $ln(sin(x))$
        \item $e^{x}*ln(x)$
    \end{itemize}
    \caption{Funktionen ohne elementare Stammfunktion
    \label{buch:table:funktionenohnestammfunktion}}
    
\end{table}

\newpage


\subsection{Formen der Gauss-Quadratur
\label{quadratur:subsection:gaussformen}}
Es gibt verschiedene Ausprägungen der Gauss-Integration, abhängig vom jeweiligen Anwendungsbereich 
erkennbar an den Grenzen des Integrals und der Wahl der Gewichtung.
Die vier häufigsten Formen sind in der Tabelle~\ref{buch:table:gaussformen} abgebildet.

\begin{table}[h!]
    
    \begin{tabular}{|>{$}c<{$}|>{$}c<{$}|>{$}c<{$}|>{$}c<{$}|}
        \hline
            Name &  Untere Grenze & Obere Grenze & Formel \\
        \hline  
            Legendre & -1 & 1 & p_{n}(x) = \int_{-1}^{1} f(x)\,dx \approx \sum_{i=0}^{n} A_{i} f(x_{i}) \\
            Chebyshev &  -1 & 1 & T_{n}(x) = \int_{-1}^{1} (1-x^{2})^{-1/2} f(x)\,dy \approx \frac{\pi}{n+1} \sum_{i=0}^{n} f(x_{i}) \\
            Laguerre &  0 & \infty & L_{n}(x) = \int_{0}^{\infty} e^{-x} f(x)\,dx \approx \sum_{i=0}^{n} A_{i} f(x_{i}) \\
            Hermite & -\infty & \infty & H_{n}(x) = \int_{-\infty}^{\infty} f(x)\,dx \approx \sum_{i=0}^{n} A_{i} f(x_{i})\\
        \hline
    \end{tabular}
    \caption{Formen der Gauss-Quadratur
    \label{buch:table:gaussformen}}
    
\end{table}