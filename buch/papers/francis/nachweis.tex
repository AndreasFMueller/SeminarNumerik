\section{Nachweis\label{francis:section:nachweis}}
\rhead{Nachweis}

Da ein formaler Beweis zu komplex wäre, verweise ich an dieser Stelle gerne auf \cite{francis:watkins_book}.
Stattdessen wollen wir aber anhand einer einfachen Implementierung zeigen, dass der Algorithmus funktioniert.

Folgender Matlab-Code von \cite{francis:watkins_book} ist eine Schritt-für-Schritt Implementation einer Francis Iteration für eine symmetrische $n x n$ Matrix 

Zu Beginn muss eine Matrix generiert und auf Hessenberg Form gebracht werden.
	
\begin{lstlisting}
n=6;
a=randn(n);
a=a+a';
a=hess(a);
[Evec,Eval]=eig(a);
Eval
\end{lstlisting}

Anschliessend soll der Wilkinson-Shift berechnet werden.
Dies geschieht im folgenden Code-Abschnitt.

\begin{lstlisting}
htr= .5*(a(n-1,n-1) + a(n,n));                          
dscr= sqrt((.5*(a(n-1,n-1)-a(n,n)))^2 + a(n,n-1)^2);    
if htr < 0, dscr= -dscr; end                            
root1 = htr + dscr;                                     
if root1 == 0                                         
	root2 = 0;
else                                                    
	det= a(n-1,n-1)*a(n,n) -a(n,n-1)^2;                 
	root2= det/root1;
end
if abs(a(n,n)-root1) < abs(a(n,n)-root2)
	shift= root1;
else
	shift= root2;
end
\end{lstlisting}

Nun muss ein Rotator $q0$ gebildet werden.
Die anschliessende Transformation (Multiplikation von Rechts und Links) erzeugt eine Ausbuchtung in der Matrix.

\begin{lstlisting}
cs= a(1,1) -shift; sn=a(2,1);
r= norm([cs sn]);
cs= cs/r; sn=sn/r;              
q0= [ cs -sn; sn cs];           
a(:,1:2) = a(:,1:2) * q0;
a(1:2,:) = q0'*a(1:2,:);
\end{lstlisting}

Um die Francis Iteration zu beenden, muss nun die Ausbuchtung aus der Matrix hinausgeschoben werden.
Dies wird mit dem folgenden Loop erledigt.

\begin{lstlisting}
for ii = 1:n-2
	%Chase the bulge from position (ii+2, ii)
	cs=a(ii+1,ii);sn=a(ii+2,ii);
	r=norm([cs sn]);
	cs= cs/r; sn= sn/r;
	a(ii+1,ii) = r;
	a(ii+2,ii)=0;
	qi= [cs -sn; sn cs];
	
	%Gives the rotator to chase the bulge
	a(ii+1:ii+2,ii+1:n) =qi'*a(ii+1:ii+2,ii+1:n);
	a(:,ii+1:ii+2) = a(:,ii+1:ii+2)*qi;
end
\end{lstlisting}

Zum Schluss kann man die wichtigesten Matrixelemente ausgeben.

\begin{lstlisting}
format short e
subdiag = diag(a,-1);
format long
bottom_entry = a(n,n);
\end{lstlisting}

Mit Seed 1 des Random Generators (Einfügen von rng(1) am Anfang) ergibt sich folgende zufällige Matrix $a$:

\begin{equation}
\begin{bmatrix}
-3.228 &	1.689 &	0 &	0 &	0\\
1.689 &	-1.618 & 1.127 & 0 &	0\\
0 &	1.127&	0.994 &	-2.643 &	0\\
0 &	0 &	-2.643 &	0.769 &	-4.415\\
0 &	0 &	0	& -4.415 & -3.730\\
\end{bmatrix}
\end{equation}

Führt man nun mehrere Francis Iterationen gemäss obiger Implementation durch, ergibt sich nach 5 Iterationen die Matrix:

\begin{equation}
\begin{bmatrix}
1.974 & 2.417	& 3.078e-16 & 1.731e-16 & 4.339e-17\\
2.417 & 2.623 & 1.248 & 4.996e-16 & 6.150e-17\\
0 & 1.248 & -0.374 & 0.371 & -1.170e-16\\
0 & 0 & 0.371 & -4.325 & -2.015e-16\\
0 & 0 & 0 &-1.313e-46 & -6.712\\
\end{bmatrix}
\end{equation}

Nach 10 Iterationen ergibt sich:

\begin{equation}
\begin{bmatrix}
4.865 & 0.444 & 2.715e-16 & 5.243e-16 & 5.754e-17\\
0.444 & 0.472 & 0.341 & 4.783e-16 & -5.158e-17\\
0 & 0.341 &-1.076 & 0.003 & -1.364e-16\\
0 & 0 & 0.003 & -4.362 & -1.881e-16\\
0 & 0 & 0 & 0 & -6.712\\
\end{bmatrix}
\end{equation}

Nach 50 Iterationen ergibt sich:

\begin{equation}
\begin{bmatrix}
4.909 & 2.272e-09	& 2.146e-16 & 5.722e-16 & 5.130e-17\\
2.272e-09 & 0.500 & 1.088e-05 & 5.238e-16 & -8.465e-17\\
0 & 1.088e-05 & -1.149 & 4.264e-16 & -1.218e-16\\
0 & 0 & 4.2091e-18 & -4.362 & -1.879e-16\\
0 & 0 & 0 & 0 &-6.712\\
\end{bmatrix}
\end{equation}

Es ist ersichtlich, dass die Subdiagonalelemente von Iteration zu Iteration kleiner werden, bis sie schliesslich annähernd 0 sind und eine Dreiecksmatrix entstanden ist.
Da wir zur Manipulation unserer Matrix lediglich Ähnlichkeitstransformationen verwendet haben, entsprechen die Diagonalelemente dieser Dreiecksmatix den Eigenwerten unserer ursprünglichen Matrix.
