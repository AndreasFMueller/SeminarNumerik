\section{Nachweis\label{francis:section:nachweis}}
\rhead{Nachweis}

Da ein formaler Beweis zu komplex wäre, verweise ich an dieser Stelle auf \cite{francis:watkins_book}.
Stattdessen wollen wir aber anhand einer einfachen Implementierung zeigen, dass der Algorithmus funktioniert.

\subsection{Implementation\label{francis:section:nachweis:implementation}}
Folgender MATLAB-Code von \cite{francis:watkins_book} ist eine Schritt-für-Schritt Implementation einer Francis-Iteration für eine symmetrische $n \times n$ Matrix.
Zu Beginn muss eine Matrix generiert und auf Hessenberg-Form gebracht werden.
\lstinputlisting[language=Matlab, style=Matlab, firstline=1, lastline=6,caption={},label={francis:codes:code1}]{papers/francis/codes/code.m}
Anschliessend soll der Wilkinson-Shift berechnet werden.
Dies geschieht im folgenden Code-Abschnitt.
\lstinputlisting[language=Matlab, style=Matlab, firstline=7, lastline=23,caption={},label={francis:codes:code2}]{papers/francis/codes/code.m}
Nun muss ein Rotator $Q_0$ gebildet werden.
Die anschliessende Transformation (Multiplikation von rechts und links) erzeugt eine Ausbuchtung in der Matrix.
\lstinputlisting[language=Matlab, style=Matlab, firstline=24, lastline=29,caption={},label={francis:codes:code3}]{papers/francis/codes/code.m}
Um die Francis-Iteration zu beenden, muss nun die Ausbuchtung aus der Matrix hinausgeschoben werden.
Dies wird mit dem folgenden Loop erledigt.
\lstinputlisting[language=Matlab, style=Matlab, firstline=30, lastline=42,caption={},label={francis:codes:code4}]{papers/francis/codes/code.m}
Zum Schluss kann man die wichtigsten Matrixelemente ausgeben.
\lstinputlisting[language=Matlab, style=Matlab, firstline=43, lastline=46,caption={},label={francis:codes:code5}]{papers/francis/codes/code.m}

\subsection{Resultate\label{francis:section:nachweis:resultate}}
Nun wollen wir den in \ref{francis:section:nachweis:implementation} beschriebenen Code testen.
Zuerst eine Bemerkung zur Darstellung der Matrizen. 
Kleine Zahlen, welche nicht null sind, werden im folgenden als $0.0000$
oder $-0.0000$ geschrieben.
Handelt es sich bei einer Zahl um eine exakte Null, so wird sie nicht geschrieben.
Mit Samen \footnote{Wird ein Samen für einen Zufallsgenerator definiert, produziert dieser Zufallsgenerator immer die gleiche Sequenz von Zahlen, welche von der definierten Verteilungsfunktion gesampelt wurden.
\index{Zufallszahlgenerator}%
\index{Samen}%
So kann mit zufälligen Zahlen gerechnet werden und das Resultat ist dennoch reproduzierbar.
In Matlab kann das durch Einfügen von \texttt{rng(1)} vor dem Zufallsgenerator gemacht werden.}
1 für den Zufallszahlgenerator ergibt sich folgende zufällige Matrix $a$:

\begin{equation}
\begin{bmatrix*}[r]
-3.2289 &	1.6892 &	 &	 &	\\
1.6892 &	-1.6190 & 1.1276 &  &	\\
 &	1.1276&	0.9950 &	-2.6434 &	\\
 &	 &	-2.6434 &	0.7690 &	-4.4160\\
 &	 &		& -4.4160 & -3.7302\\
\end{bmatrix*}.
\end{equation}

Führt man nun mehrere Francis Iterationen gemäss obiger Implementation durch, ergibt sich nach 5 Iterationen die Matrix:

\begin{equation}
\begin{bmatrix*}[r]
1.9741 & 2.4170	& 0.0000 & 0.0000 & 0.0000\\
2.4170 & 2.6240 & 1.2488 & 0.0000 & 0.0000\\
 & 1.2488 & -0.3743 & 0.3716 & -0.0000\\
 &  & 0.3716 & -4.3253 & -0.0000\\
 &  &  &-0.0000 & -6.7126\\
\end{bmatrix*}.
\end{equation}
Nach 10 Iterationen ergibt sich:

\begin{equation}
\begin{bmatrix*}[r]
4.8651 & 0.4449 & 0.0000 & 0.0000 & 0.0000\\
0.4449 & 0.4722 & 0.3417 & 0.0000 & -0.0000\\
 & 0.3417 &-1.0764 & 0.0040 & -0.0000\\
 &  & 0.0040 & -4.3623 & -0.0000\\
 &  &  & 0.0000 & -6.7126\\
\end{bmatrix*}.
\end{equation}
Nach 50 Iterationen ergibt sich:

\begin{equation}
\begin{bmatrix*}[r]
\textcolor{darkgreen}{4.9099} & 0.0000	& 0.0000 & 0.0000 & 0.0000\\
0.0000 & \textcolor{darkgreen}{0.5008} & 0.0000 & 0.0000 & -0.0000\\
 & 0.0000 & \textcolor{darkgreen}{-1.1499} & 0.0000 & -0.0000\\
 &  & 0.0000 & \textcolor{darkgreen}{-4.3623} & -0.0000\\
 &  &  & 0.0000 &\textcolor{darkgreen}{-6.7126}\\
\end{bmatrix*}.
\end{equation}
Es ist ersichtlich, dass die Subdiagonalelemente von Iteration zu Iteration kleiner werden, bis sie schliesslich annähernd 0 sind und eine Dreiecksmatrix entstanden ist.
Da wir zur Manipulation unserer Matrix lediglich Ähnlichkeitstransformationen verwendet haben, entsprechen die Diagonalelemente dieser Dreiecksmatrix den Eigenwerten unserer ursprünglichen Matrix.
