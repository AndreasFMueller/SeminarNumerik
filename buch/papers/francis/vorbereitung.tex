%
% loesung.tex -- Beispiel-File für die Beschreibung der Loesung
%
% (c) 2020 Prof Dr Andreas Müller, Hochschule Rapperswil
%
\section{Vorbereitung der Matrix\label{francis:section:vorbereitung}}
\rhead{Vorbereitung}
Um den Francis Algorithmus effizient ausführen zu können, muss die Matrix zuerst auf Hessenberg Form gebracht werden.
\begin{equation}
	\begin{bmatrix}
	* & * & * & * & * \\
	* & * & * & * & * \\
	*& * & * & * & * \\
	*&  * & * & * & * \\
	*&  * &  * & * & *
	\end{bmatrix} \rightarrow
	\begin{bmatrix}
	* & * & * & * & * \\
	* & * & * & * & * \\
	& * & * & * & * \\
	&   & * & * & * \\
	&   &   & * & *
	\end{bmatrix}
\end{equation}

\begin{satz}
	\emph{Theorem: Jede Matrix $ A \subset \mathbb{R} ^{n x n} $ ist orthogonal ähnlich zu einer Matrix H in der oberen Hessenberg Form: $ H=Q^{-1}AQ $, wobei Q ein Produkt von n-2 Reflektoren ist.}
\end{satz}

Gegeben sei eine Matrix A mit einer Submatrix $\hat{A}$  und einer ersten Spalte $a_{11}$ und einem Vektor b.
Ziel ist es, alle Einträge dieses Vektors, ausser den ersten auf null zu setzen.
Dies kann mit einer Multiplikation mit einer geeigneten Matrix $Q$ gemacht werden. $\hat{Q}$ muss so gewählt werden, dass bei einer Multiplikation mit $b$ alle Werte bis auf den ersten verschwinden.
Die Matrix $Q$ soll so gewählt werden, dass die Länge des neuen Vektors, bzw. dieser einzelne Eintrag, der Länge des Vektors $b$ entspricht.
\begin{equation}
	A=
	\begin{bmatrix}
	a_{11} & *\\
	b & \hat{A}
	\end{bmatrix},
	Q_{1}=
	\begin{bmatrix}
	1 & 0\\
	0 & \hat{Q_1}
	\end{bmatrix},
	\text{mit } \hat{Q_1}b=
	\begin{bmatrix}
	-y\\
	0\\
	.\\
	.\\
	0
	\end{bmatrix}
\end{equation}


Wird $\hat{Q}$ richtig gewählt entsteht bei einer Multiplikation von links mit der Matrix Q der gewünschte Vektor in der ersten Spalte.
$\hat{Q}$ kann dabei zum Beispiel mit Givens Rotationsmatrizen \ref{francis:section:grundlagen:givens} oder einer Householder-Transformationsmatrix \ref{francis:section:grundlagen:householder} gebildet werden.

\begin{equation}
	A_{\text{Half}}=Q_{1}A=
	\begin{bmatrix}
	a_{11} & * & ... & *\\
	-y & \\
	0 & & & &\\
	. & &\hat{Q_1}\hat{A} & &\\
	. & & & &\\
	0 & & & &
	\end{bmatrix}
\end{equation}

Damit wir aber die Eigenwerte unserer Matrix nicht verändern, müssen wir die $Q$ Matrix nun auch von der rechten Seite hinzu multiplizieren.
Da unsere erste Spalte von $Q$ nur eine 1 und nullen hat, wird dabei die vorher erzeugte Spalte nicht beeinflusst.
Das Resultat ist also eine ähnliche Matrix, welche an den gewünschten Stellen in der ersten Spalte nullen hat.

\begin{equation}
	A_{1}=A_{\text{Half}}Q_{1}=
	\begin{bmatrix}
	a_{11} & * & ... & *\\
	-y & \\
	0 & & & &\\
	. & &\hat{Q_1}\hat{A}\hat{Q_1} & &\\
	. & & & &\\
	0 & & & &
	\end{bmatrix}
	=
	\begin{bmatrix}
	a_{11} & * & ... & *\\
	-y & \\
	0 & & & &\\
	. & &\hat{A1} & &\\
	. & & & &\\
	0 & & & &
	\end{bmatrix}
\end{equation}	

Das gleiche Prinzip wird wieder angewendet.
Durch den zweiten Reflektor entstehen Nullen in der zweiten Spalte.
Mit einem weiteren Schritt ebenfalls in der dritten Spalte usw.

\begin{equation}
	Q_{2}=
	\begin{bmatrix}
	1 & 0 & 0 & ... & 0\\
	0 & 1 & 0 & ... & 0\\
	0 & 0 &\\
	. & . & &\hat{Q_2} &\\
	. & . &\\
	0 & 0 &
	\end{bmatrix}
\end{equation}	
\begin{equation}
	A_{2}=
	\begin{bmatrix}
	a_{11} & * & * & ... & *\\
	-y_{1} & * & * & ... & *\\
	0 & -y_{2} &\\
	. & . & &\hat{Q_2}\hat{A_{2}}\hat{Q_2} &\\
	. & . &\\
	0 & 0 &
	\end{bmatrix}
\end{equation}

Das Resultat ist eine Matrix in der Hessenberg Form, welche ähnlich zu $A$ ist.
Das heisst für uns, sie besitzt dieselben Eigenwerte wie die Matrix $A$.
Die ganzen Multiplikation der Reflektoren kann zusammengefasst werden, sodass sich eine einzige Ähnlichkeitstransformation ergibt.

\begin{equation}
	H=Q_{n-2}Q_{n-1}...Q_{1}AQ_{1}...Q_{n-1}Q{n-2}
\end{equation}
mit 
\begin{equation}
	Q=Q_{1}...Q_{n-1}Q_{n-2}
\end{equation}
erhalten wir
\begin{equation}
	Q^{-1} = Q_{n-2}Q_{n-1}...Q_{1}
\end{equation}
\begin{equation}
H=Q^{-1}AQ
\end{equation}

Im Falle einer symmetrischen Matrix $A$, werden durch die Multiplikation von rechts dieselben Elemente im oberen Teil der Matrix eliminiert und das Ergebnis der Transformation tridiagonal.
Durch die Reduzierung werden die Kosten des anschliessenden Algorithmus stark reduziert.
