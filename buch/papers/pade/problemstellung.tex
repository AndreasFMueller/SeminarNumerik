%
% problemstellung.tex -- Beispiel-File für die Beschreibung des Problems
%
% (c) 2020 Prof Dr Andreas Müller, Hochschule Rapperswil
%

\section{Berechnung der Padé-Approximation
\label{pade:section:Problemstellung}}
Eine Padé-Approximation ist ein Bruch aus zwei Polynomen, welche aus den Koeffizienten der Taylor-Reihe einer Funktion gewonnen werden. 
Ziel dieses Kapitels ist, dem Leser den Nutzen der Padé-Approximation näher zu bringen und zu zeigen, wie man aus einer analytischen Funktion eine solche Approximation bildet.
Des Weiteren wird auf mehrere praktische Beispiele eingegangen und mögliche Fehlerquellen aufgezeigt. 



\subsection{Potenzreihen von analytischen Funktionen
\label{pade:subsection:Potenzreihen}}
\rhead{Potenzreihen}
Die Koeffizienten einer Potenzreihe können verwendet werden, um die Koeffizienten einer Padé-Approximation zu gewinnen. 
In der Analysis kann eine analytische Funktion mit einer Taylor-Reihe um eine Stelle $x_{0}$ durch eine Potenzreihe dargestellt werden. 
Diese Potenzreihen werden um eine vorgegebene Stelle $x_{0}$ als eine unendliche Summe 
\begin{equation}
f(x)=\sum_{n=0}^{\infty} c_{n} (x-x_{0})^{n} 
\label{pade:expofunk}
\end{equation}
gebildet.
Für viele Funktionen sind die dazugehörigen Potenzreihen bereits bekannt. 
Funktionen, welche durch eine Potenzreihe dargestellt werden können, werden auch analytische Funktionen genannt.
Es gibt verschiedene Methoden, um eine Potenzreihe einer Funktion zu erhalten. 
Sie können durch Differentialgleichungen oder eine andere Reihenentwicklungsmethoden hergeleitet werden.

\subsubsection{Beispiel Potenzreihen
\label{pade:section:Bsp_Potenzreihen}}
In diesem Beispiel möchten wir die Potenzreihe der Exponentialfunktion erhalten, welche an der Stelle $x_0 = 0$ entwickelt wird.
Um dies zu erreichen wird ein Potenzreihenansatz verwendet.
Dabei wird die gesuchte Funktion durch eine Potenzreihe mit unbekannten Koeffizienten dargestellt, diese Koeffizienten werden in eine Differentialgleichung eingesetzt und dann wird versucht mit Koeffizientenvergleich  eine Lösung zu finden.
Wir wissen welche Eigenschaften die Exponentialfunktion hat und nutzen diese, um die Differentialgleichungen aufzustellen. 
Gesucht ist eine Potenzreihe, welche die Differentialgleichung 
\begin{equation*}
	f^{\prime}(x) = f(x) , \qquad \text{ für alle } x \in \mathbb{R} 
\end{equation*}
erfüllt.

Die Form der Potenzreihe ist gegeben durch die schon gezeigte Summe \eqref{pade:expofunk}.
Weiter wissen wir, dass die Exponentialfunktion 
\begin{equation*}
f(0) = 1
\end{equation*}
erfüllen muss.
Daraus schliessen wir, dass
\begin{equation*}
f(x)=\sum_{n=0}^{\infty} c_{n} \cdot x^{n}
\qquad\Rightarrow\qquad
\sum_{n=0}^{\infty} c_{n} \cdot 0^{n} 
=
c_{0} \cdot 0^{0} + c_{1} \cdot 0^{1} + c_{2} \cdot 0^{2} \dots = 1
\qquad\Rightarrow\qquad
c_{0} = 1
\end{equation*}
und somit ist der erste Koeffizient $c_0$ gefunden.
Anschliessend wird das Polynom abgeleitet

\begin{equation*}
f^{\prime}(x)
=
\sum_{n=0}^{\infty}(n+1) \cdot c_{n+1} \cdot x^{n},
\end{equation*}
um die weiteren Koeffizienten $c_n$ zu erhalten.
Da immer noch die Anforderung $f(x) = f^{\prime}(x)$ gilt, wissen wir, dass die Koeffizienten von $x^n$  von der Form
\begin{equation*}
(n+1) \cdot c_{n+1} 
= 
c_{n} , \qquad \text{ für alle } n \in \mathbb{R}
\end{equation*}
sein müssen. 
Mit dem bereits bekannten $c_0 = 1$ folgt rekursiv
\begin{equation*}
c_{n+1} 
= 
\frac{1}{(n+1)!}
\qquad\Rightarrow\qquad
c_{n} 
= 
\frac{1}{n!}
\end{equation*}
und damit sind die Koeffizienten des Polynoms der Exponentialfunktion ermittelt.
Ausgeschrieben erhält man


\begin{equation}
e^{x}
=
\sum_{n=0}^{\infty} \frac{x^{n}}{n !}
=
\frac{x^{0}}{0 !}+\frac{x^{1}}{1 !}+\frac{x^{2}}{2 !}+\frac{x^{3}}{3 !}+\cdots .
\label{pade:potenzexp}
\end{equation}
Dieses Vorgehen wird auch Potenzreihenansatz für eine Differentialgleichung genannt und ist auch ein Lösungsansatz für Differentialgleichungen.

Aus der bekannten Potenzreihe der Exponentialfunktion \eqref{pade:potenzexp} können nun auch andere Potenzreihen gewonnen werden.
Aus der Exponentialfunktion kann die Potenzreihe für die Sinus- und Kosinus- Funktion ermittelt werden.
Dies kann durch beispielsweise durch die Koeffizientenvergleich Methode erreicht werden. 
Um dies zu erreichen wird, zuerst die Reihe komplex erweitert
\begin{equation*}
e^{x}
=
\sum_{n=0}^{\infty} \frac{x^{n}}{n !}
\qquad\Rightarrow\qquad
e^{ix}
=
\sum_{n=0}^{\infty} \frac{ix^{n}}{n !}.
\end{equation*}
und danach ausgeschrieben damit die Koeffizienten   
\begin{equation}
e^{ix}
=
\sum_{n=0}^{\infty} \frac{(ix)^{n}}{n !}
=
\frac{(ix)^{0}}{0 !}+\frac{(ix)^{1}}{1 !}+\frac{(ix)^{2}}{2 !}+\frac{(ix)^{3}}{3 !}+\cdots
\end{equation}
betrachtet werden können.
Man erkennt gleich, dass die komplexen Zahlen bei den geraden Exponenten verschwinden und bei den ungeraden Exponenten bestehen bleiben. 
Sortiert man nun das Polynom nach imaginären und reellen Termen erkennt man die $\operatorname{cis}$-Form 
\begin{align*}
e^{ix}
&=
\left(1-\frac{x^{2}}{2 !}+\frac{x^{4}}{4 !}-\ldots\right)+i \cdot\left(x-\frac{x^{3}}{3 !}+\frac{x^{5}}{5 !}-\ldots\right)
\\
&=
\cos(x)+i\cdot \sin(x) = \operatorname{cis}(x).
\end{align*}
und kann diese beiden Teile als Sinus und Kosinus aufschreiben,
wobei die Sinusfunktion 
\begin{equation*}
\sin (x)
=
\frac{x}{1 !}-\frac{x^{3}}{3 !}+\frac{x^{5}}{5 !} \mp \cdots
=
\sum_{n=0}^{\infty}(-1)^{n} \frac{x^{2 n+1}}{(2 n+1) !}
\end{equation*}
und die Kosinusfunktion 
\begin{equation*}
\cos (x)
=
\frac{x^{0}}{0 !}-\frac{x^{2}}{2 !}+\frac{x^{4}}{4 !} \mp \cdots
=
\sum_{n=0}^{\infty}(-1)^{n} \frac{x^{2 n}}{(2 n) !}
\end{equation*}
beide als einzelne Summen ausgeschrieben werden können.



\subsection{Padé-Approximation erstellen
	\label{pade:subsection:Pade_erstellen}}

Um eine Padé-Approximation zu erstellen, startet man immer mit der Potenzreihe der zu approximierenden Funktion.
Diese Potenzreihe der Form 
\begin{equation}
f(x)=\sum_{n=0}^{\infty} c_{n} (z)^{n} 
\end{equation} 
möchte man dann durch einen rationalen Bruch 
\begin{equation}
R_{[L/M]}
=
[L/M]
=
\frac{a_0 + a_1 z + \dots + a_L z^L}{b_0 + b_1 z + \dots + b_M z^M}
+O(z^{L+M+1})
\end{equation}
beschreiben.
Wir verwenden nun für den Approximanten die Notation $R_{[L/M]}$.
Der erste Koeffizient des Nennerpolynoms wird mit $b_0 = 1$ festgelegt.
Dank dieser Definition haben wir $L+1$ zu bestimmende Zähler- und $M$ zu bestimmende Nenner- Koeffizienten. 
Dies resultiert in $[L+M+1]$ unbekannten Koeffizienten, welche mit der Nummerierung den Termen $z^0, z^1, z^2,\dots , z^{L+M}$ der Potenzreihe übereinstimmt.

Um die einzelnen $a_k$ (mit $0\leq k\leq L$) und $b_k$ (mit $1\leq k\leq M $) Koeffizienten zu berechnen, wir die Approximation mit der Potenzreihe gleichgesetzt
\begin{equation}
\sum_{n=0}^{\infty} c_{n} (z)^{n} 
=
\frac{a_0 + a_1 z + \dots + a_L z^L}{b_0 + b_1 z + \dots + b_M z^M}
+O(z^{L+M+1}).
\end{equation}
Das ganze wird umgestellt und als lineares Gleichungssystem
\begin{equation}
\sum_{n=0}^{\infty} c_{n} (z)^{n} 
\cdot
(b_0 + b_1 z + \dots + b_M z^M)
=
(a_0 + a_1 z + \dots + a_L z^L) 
+
O(z^{L+M+1}),
\end{equation}
für einen späteren Koeffizientenvergleich betrachtet.
Diese linearen Gleichungen können in der Matrixform 
\[
\renewcommand\arraystretch{1.25}
\begin{pmatrix}
c_{L-M+1} & c_{L-M+2} & c_{L-M+3} &\dots & c_{L}\\
c_{L-M+2} & c_{L-M+3} & c_{L-M+4} &\dots & c_{L+1}\\
c_{L-M+3} & c_{L-M+4} & c_{L-M+5} &\dots & c_{L+2}\\
\vdots & \vdots  & \vdots  &  & \vdots \\
c_{L} & c_{L+1} & c_{L+2} &\dots & c_{L+M-1}\\
\end{pmatrix}
\cdot
\begin{pmatrix}
b_{M}\\
b_{M-1}\\
b_{M-2}\\
\vdots \\
b_{1}\\
\end{pmatrix}
=
-
\begin{pmatrix}
c_{L+1}\\
c_{L+2}\\
c_{L+3}\\
\vdots \\
c_{L+M}\\
\end{pmatrix}
\label{pade:bKoeff}
\]
aufgeschrieben und nach den $b_k$-Koeffizienten aufgelöst werden.
Die dazugehörigen $a_k$-Koeffizienten können dann aus den $b_k$ und den $c_n$ (mit $0\leq n \leq M+L$) Koeffizienten mittels Koeffizientenvergleich
\begin{equation}
\begin{array}{l}
a_{0}=c_{0} \\
a_{1}=c_{1}+b_{1} c_{0} \\
a_{2}=c_{2}+b_{1} c_{1}+b_{2} c_{0} \\
\vdots \\
a_{L}=c_{L}+\displaystyle\sum_{n=1}^{\min (L, M)} b_{n} c_{L-n}
\end{array}
\label{pade:aKoeff}
\end{equation}
berechnet werden.
Beim Lösen des Gleichungssystems können sich vor allem für grosse $L$ und $M$ numerische Fehler einschleichen, weil die Koeffizienten der Potenzreihe möglicherweise sehr klein werden. 
Gelingt die Lösung des Gleichungssystems, hat man einen $[L/M]$-Approximanten konstruiert.
Dieser Approximant stimmt mit 
\begin{equation}
\sum_{n=0}^{\infty} c_{n} z^{n}
\end{equation}
bis zu der Ordnung $z^{[L+M]}$ überein.

\begin{beispiel}


Nehmen wir das Beispiel der Einleitung,
\begin{equation}
f(x)
=
\left(\frac{1+2x}{1+x}\right)^{\frac{1}{2}}
\approx
1+\frac{1}{2}x - \frac{5}{8}x^2+\frac{13}{16}x^3 -\frac{141}{128}x^4 +\frac{399}{256}x^5 - \frac{2353}{1024}x^6 + \frac{7205}{2048}x^7 \mp \cdots
\label{pade:bspPotenz}
\end{equation}
von welchem wir nun den Padé-Approximanten der dritten Ordnung $R_{[3/3]}$ berechnen möchten.
Zuerst kreieren wir die Matrix, um die $b_k$ Koeffizienten zu berechnen.
\[
\renewcommand\arraystretch{1.25}
\begin{pmatrix}
c_{1} & c_{2} & c_{3}\\
c_{2} & c_{3} & c_{4}\\
c_{3} & c_{4} & c_{5} \\
\end{pmatrix}
\cdot
\begin{pmatrix}
b_{3}\\
b_{2}\\
b_{1}\\
\end{pmatrix}
=
-
\begin{pmatrix}
c_{4}\\
c_{5}\\
c_{6}\\
\end{pmatrix}
\]
Hier können nun die Koeffizienten der Potenzreihe \ref{pade:bspPotenz} eingesetzt werden.
\[
\renewcommand\arraystretch{1.25}
\begin{pmatrix}
\frac{1}{2} & -\frac{5}{8} & \frac{13}{16}\\
-\frac{5}{8} & \frac{13}{16}& -\frac{141}{128}\\
\frac{13}{16} & -\frac{141}{128} & \frac{399}{256} \\
\end{pmatrix}
\cdot
\begin{pmatrix}
b_{3}\\
b_{2}\\
b_{1}\\
\end{pmatrix}
=
-
\begin{pmatrix}
-\frac{141}{128}\\
\frac{399}{256}\\
-\frac{2353}{1024}\\
\end{pmatrix}
\]
Da es sich in diesem Fall um eine Matrix gleicher Dimensionen handelt, kann die Inverse der $C$-Matrix 
\begin{equation}
C\cdot b = -c
\qquad\Rightarrow\qquad
b = C^{-1} \cdot -c
\end{equation}
genommen werden, um die Gleichung 

\[
\renewcommand\arraystretch{1.25}
\begin{pmatrix}
b_{3}\\
b_{2}\\
b_{1}\\
\end{pmatrix}
=
\begin{pmatrix}
\frac{1}{2} & -\frac{5}{8} & \frac{13}{16}\\
-\frac{5}{8} & \frac{13}{16}& -\frac{141}{128}\\
\frac{13}{16} & -\frac{141}{128} & \frac{399}{256} \\
\end{pmatrix}^{-1}
\cdot
\begin{pmatrix}
\frac{141}{128}\\
-\frac{399}{256}\\
\frac{2353}{1024}\\
\end{pmatrix}
=
\begin{pmatrix}
\frac{169}{64}\\
\frac{47}{8}\\
\frac{17}{4}\\
\end{pmatrix}
\]
zu lösen.
Die $a_k$ Koeffizienten können nun aus den $b_k$ und den $c_n$ Koeffizienten berechnet werden.
\begin{equation}
\begin{array}{l}
a_{0}=c_{0} \\
a_{1}=c_{1}+b_{1} c_{0} \\
a_{2}=c_{2}+b_{1} c_{1}+b_{2} c_{0} \\
a_{3}=c_{3}+b_{1} c_{2}+b_{2} c_{1} + b_{3} c_{0} \\
\end{array}
\qquad\Rightarrow\qquad
\begin{array}{l}
a_{0}=1 \\
a_{1}=\frac{1}{2}+\frac{17}{4} \cdot 1 = \frac{19}{4}\\
a_{2}=-\frac{5}{8} +\frac{17}{4}\cdot \frac{1}{2} +\frac{47}{8} \cdot 1 = \frac{59}{8} \\
a_{3}=\frac{13}{16}+\frac{17}{4}\cdot -\frac{5}{8} + \frac{47}{8}\cdot \frac{1}{2} + \frac{169}{64} \cdot 1 = \frac{239}{64}\\
\end{array}
\end{equation}
Daraus resultiert der gesamte Padé-Approximant der dritten Ordnung
\begin{equation}
R_{[3/ 3]}(x)
=
\frac{a_0+a_1x+a_2 x^2+a_3x^3}
{1+b_1x+b_2 x^2+b_3 x^3}
=
\frac{1+\frac{19}{4}x+\frac{59}{8}x^2+\frac{239}{64}x^3}
{1+\frac{17}{4}x+\frac{47}{8}x^2+\frac{169}{64}x^3},
\end{equation}
welcher aus der Taylor-Reihe der sechsten Ordnung gewonnen werden konnte.

Auffallend ist, dass dieser Approximant der dritten Ordnung aus komplett anderen Polynomen zusammengesetzt ist, als der Approximant zweiter Ordnung des obigen Beispiels \ref{pade:bspordnung2}.
\end{beispiel}

Dieser Aufwand, um einen Approximanten zu berechnen, muss jedoch nicht immer betrieben werden.
Für mache Funktionen können Formeln zur Konstruktion der Nenner- und Zähler- Polynome der Padé-Approximanten gefunden werden.
Wie man auf diese Formeln kommt, ist zum Beispiel in \cite{pade:Baker2009} beschrieben.
Die Exponentialfunktion, welche für viele Anwendungen gebraucht wird, besitzt glücklicherweise eine solche Formel für die Konstruktion der Nenner- und Zähler- Polynome \cite{pade:moler}.
Um eine  Padé-Approximation beliebiger Ordnung für die Exponentialfunktion zu erhalten 
\begin{equation}
R_{[L/M]}(x)
=
\frac{P_{[L/ M]}(x)}{Q_{[L, M]}(x)} \approx e^{-x}
\end{equation}
können die zwei gegebenen Formeln aus \cite{pade:moler} verwendet werden.
Für $P_{[L/ M]}(x)$, also das Zählerpolynom, gilt die Formel
\begin{equation}
P_{[L/ M]}(x)
=
\sum_{n=0}^{L} \frac{(L+M-n) ! L !}{(L+M) ! n !(L-n) !}(-x)^{n}
\label{pade:expP}
\end{equation}
und für $Q_{[L/ M]}(x)$, das Nennerpolynom, gilt
\begin{equation}
Q_{[L/ M]}(x)
=
\sum_{n=0}^{M} \frac{(L+M-n) ! M !}{(L+M) ! n !(M-n) !} x^{n}.
\label{pade:expQ}
\end{equation}


Es ist gängige Praxis die Padé-Approximanten in einer Tabelle mit der Form
\begin{center}
	%\renewcommand\arraystrech{1.25}
	\begin{tabular}{c| c c c c}
		
		$[L/M]$ 	& 0 		& 1 		& 2 		& $\cdots$ \\
		\hline
		0 		& $[0/0]$ 	& $[1/0]$ 	& $[2/0]$ 	& $\dots$ \\
		1 		& $[0/1]$ 	& $[1/1]$ 	& $[2/1]$ 	& $\dots$ \\
		2 		& $[0/2]$ 	& $[1/2]$ 	& $[2/2]$ 	& $\dots$ \\
		$\vdots$ 	& $\vdots$ 	& $\vdots$  	& $\vdots$  	&  \\
	\end{tabular}
\end{center}
darzustellen.
Diese Darstellung dient zur Übersicht der einzelnen Approximanten und macht deutlich, dass alle Approximanten verschiedener Ordnung unabhängig von einander sind. 
Das heisst die Polynome sehen bei jeder Ordnung komplett verschieden aus und werden nicht wie bei der Taylor-Reihe einfach immer länger. 


In diese Tabelle können nun die einzelnen Approximanten einer Funktion eingefüllt werden.
Als Beispiel wurde die Exponentialfunktion, mit den dazugehörigen $[L/M]$ Approximanten bis zum Approximant $[2/2]$, mit Hilfe der Formeln \ref{pade:expP} und \ref{pade:expQ} berechnet und in die Tabelle
\begin{table}
\centering
\renewcommand\arraystretch{1.25}
	\begin{tabular}{c| c c c }
		$[L/M]$ 	& 0 		& 1 		& 2 	 \\
		\hline
		0 		&  $\displaystyle\frac{1}{1}$ 	& $\displaystyle\frac{1+z}{1}$ 	& $\displaystyle\frac{2+2 z+z^{2}}{2}$ \\
		1 		& $\displaystyle \frac{1}{1-z}$ 	& $\displaystyle\frac{2+z}{2-z}$ 	& $\displaystyle\frac{6+4 z+z^{2}}{6-2 z}$ 	 \\
		2 		& $\displaystyle\frac{2}{2-2 z+z^{2}} $ 	& $\displaystyle\frac{6+2 z}{6-4 z+z^{2}}$ 	& $\displaystyle\frac{12+6 z+z^{2}}{12-6 z+z^{2}}$ 	 \\
	\end{tabular}
\end{table}
eingefügt.













