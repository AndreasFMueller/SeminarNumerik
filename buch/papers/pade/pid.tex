%
% pid.tex -- pid schaltung
%
% (c) 2020 Prof Dr Andreas Müller, Hochschule Rapperswil
%
\documentclass[tikz]{standalone}
\usepackage{amsmath}
\usepackage{times}
\usepackage{txfonts}
\usepackage{pgfplots}
\usepackage{csvsimple}
\usepackage[european]{circuitikz}
\usetikzlibrary{arrows,intersections,math}
\begin{document}
\def\skala{1}
\begin{tikzpicture}[>=latex,thick,scale=\skala]

\draw (0,0) node[op amp] (opamp) {};
\draw (opamp.-) |- ++(0,1.5) to[C,l=$C_2$] ++(2.0,0) to[R,l=$R_2$]
	++(2.0,0) to[short,-*] ++(0,-1.99) |- (opamp.out) ;
\draw (opamp.out) to[short,-o] ++(2.3,0) coordinate (Output);
\node at (Output) [right] {Output};
\draw (opamp.-) to[short,*-] ++(-1,0) to[short,*-] ++(0,0.5) to[R,a=$R_1$] ++ (-2,0) to[short,-*] ++(0,-0.5);
\draw (opamp.-)    ++(-1,0) to ++(0,-0.5) to[C,l=$C_1$] ++ (-2,0) to ++(0,0.5) to[short,-o] ++(-1,0) coordinate(Input);
\draw (opamp.+) -| ++(0,-1) node[ground] {} ;

\node at (Input) [left] {Input};

\end{tikzpicture}
\end{document}

