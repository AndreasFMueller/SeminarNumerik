%
% problemstellung.tex -- Beispiel-File für die Beschreibung des Problems
%
% (c) 2020 Prof Dr Andreas Müller, Hochschule Rapperswil
%
\section{Folgerungen
\label{pade:section:folgerungen}}
\rhead{Folgerungen}
Die Padé-Approximation hat sich ergänzend zur Taylorreihe als ein sehr nützliches Werkzeug gezeigt.
Es ist jedoch nicht immer die beste Lösung, welche immer verwendet werden sollte.
Bei vielen Approximationen reicht eine Taylorreihe aus und die Padé-Approximanten ist dann nur einen Mehraufwand.
Hört man jedoch, dass eine Funktion als ein Bruch von rationalen Funktionen dargestellt werden soll oder  
eine Funktion, welche bei grossen Werten konvergiert, soll approximiert werden, dann sollte man als erstes an die Padé-Approximation denken.




