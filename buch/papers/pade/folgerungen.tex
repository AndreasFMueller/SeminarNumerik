%
% problemstellung.tex -- Beispiel-File für die Beschreibung des Problems
%
% (c) 2020 Prof Dr Andreas Müller, Hochschule Rapperswil
%
\section{Wann nun eine Padé-Approximation? 
\label{pade:section:folgerungen}}
\rhead{Folgerungen}
Die Padé-Approximation hat sich ergänzend zur Taylor-Reihe als ein sehr nützliches Werkzeug gezeigt.
Es ist jedoch nicht immer die beste Lösung.
Bei vielen Approximationen reicht eine Taylor-Reihe aus und die Padé-Approximation ist dann nur ein Mehraufwand.
Hört man jedoch, dass eine Funktion als ein Bruch von rationalen Funktionen dargestellt werden soll oder  
eine Funktion, welche bei grossen Werten konvergiert, approximiert werden soll, dann sollte man als erstes an die Padé-Approximation denken.




