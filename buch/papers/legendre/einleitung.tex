%
% einleitung.tex -- Beispiel-File für die Einleitung
%
% (c) 2020 Prof Dr Andreas Müller, Hochschule Rapperswil
%
\section{Einleitung\label{legendre:section:einleitung}}
\rhead{Einleitung}
Zur Berechnung respektive zur Evaluation von zugeordneten Legendrepolynome (\textit{engl.} associated Legendre polynomials \cmt{korrekt so?}) gibt es eine ganze Liste von Rekursionsbeziehungen (\textit{engl.} recurrence relations \cmt{korrekt so?}), mit denen aus vorgängig berechneten Legendrepolynome neue berechnet werden können.
Leider sind nicht alle dieser Rekursionsbeziehungen numerisch stabil.
Dieses Kapitel befasst sich mit der Stabilität oder eben Instabilität der Rekursionsbeziehungen für zugeordnete Legendrepolynome.
Es wird untersucht, wie numerisch instabile Rekursionsbeziehungen erkannt werden könne und wieso diese instabil sind.

% Unterkapitel Beispiel
%\subsection{Titel Unterkapitel
%\label{legendre:subsection:unterkapitellabel}}

% Quelle Zitieren Beispiel
%\cite{legendre:bibtex}

% Abschnittsverweis Beispiel
%\ref{legendre:section:loesung}

% Gleichung Beispiel
%\begin{equation}
%\int_a^b x^2\, dx
%=
%\left[ \frac13 x^3 \right]_a^b
%=
%\frac{b^3-a^3}3.
%\label{legendre:equation1}
%\end{equation}

% Gleichung Referenzieren Beispiel
%\eqref{legendre:equation1}


