%
% loesung.tex -- Beispiel-File für die Beschreibung der Loesung
%
% (c) 2020 Prof Dr Andreas Müller, Hochschule Rapperswil
%
\section{Lösung
\label{legendre:section:loesung}}
\rhead{Lösung}
\cmt{Lösung schreiben}

\cmt{Erste Rekursionsformel in l-Richtung, zweite in m-Richtung.}

\cmt{Bild mit Rekursionsrichtung und Beschrieb wie von Wertebereich von m.}

\cmt{Rekursionsformel in m-Richtung.}

\cmt{Eigener Plot Instabilität.}

\cmt{Wolfram Alpha Plot Instabilität und Referenz zu Dokumentation.}

\cmt{Untersuchen wieso instabil (Faktor wichtig für am Rand wird gross und multipliziert mit relativ kleinem Faktor -> numerisch schwierig.}

\cmt{Rekursionsformel in l-Richtung.}

\cmt{Eigener Plot Stabilität.}

\cmt{Untersuchen wieso dieser stabil und wo eventuell Probleme auftreten können.}

\cmt{Eventuell interessante Kenntnisse wieso trotzdem die Rekursionsformel in m-Richtung verwendet wird.}

\cmt{Geschwindigkeit untersuchen.}


