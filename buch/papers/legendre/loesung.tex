%
% loesung.tex -- Beispiel-File für die Beschreibung der Loesung
%
% (c) 2020 Prof Dr Andreas Müller, Hochschule Rapperswil
%
\section{Lösung
\label{legendre:section:loesung}}
\rhead{Lösung}
\subsection{Bedeutung $l$- und $m$-Richtung
\label{legendre:subsection:lmrichtung}}
Wie schon im vorherigen Abschnitt (\ref{legendre:section:problemstellung}) angesprochen, gibt es zwei Parameter, $l$ für den Grad und $m$ für die Ordnung.
Mit den Bedingungen für die beiden Parameter ($l>=0$ und $m=0, \ldots , l$) ergibt sich eine bestimmte Struktur von möglichen zugeordneten Legendrepolynomen.
Diese Struktur ist in Abbildung \cmt{Referenz zur Abbildung} dargestellt.

\cmt{Abbildung Struktur einfügen, Anfangswerte kennzeichnen}

Die beiden Gleichungen \eqref{legendre:recurrence-l} und \eqref{legendre:recurrence-m} aus dem vorherigen Abschnitt (\ref{legendre:section:problemstellung}) unterscheiden sich in der Rekursionsrichtung.
Während die Rekursion mittels Gleichung \eqref{legendre:recurrence-l} in $l$-Richtung verläuft, sprich in Richtung des Grades, verläuft die Rekursion mittels Gleichung \eqref{legendre:recurrence-m} in $m$-Richtung, sprich in Richtung der Ordnung.

\subsection{Rekursionsformel in $m$-Richtung
\label{legendre:subsection:mrichtung}}
Die Rekursion in $m$-Richtung verläuft in negativer $m$-Richtung, sprich nach abnehmenden $m$s.
Aus diesem Grund muss die Gleichung \eqref{legendre:recurrence-m} etwas umgeformt werden.
Daraus ergibt sich die neue Rekursionsgleichung \eqref{legendre:recurrence-m-1}.
% umgeformte Rekursionsformel in m-Richtung
\begin{equation}
P^{m-1}_{l}(x)
= \left[ \frac{2mxP^{m}_{l}(x)}{- \sqrt{1-x^2}}-P^{m+1}_{l} \right]
\frac{1}{(l+m)(l-m+1)}
\label{legendre:recurrence-m-1}
\end{equation}
Die benötigten Anfangswerte für ein $P^{m}_{l}$ für die Rekursion in $m$-Richtung sind demnach $P^{l}_{l}$ und $P^{l-1}_{l}$.
Wird die umgeformte Rekursionsgleichung \eqref{legendre:recurrence-m-1} auf numerische Instabilität untersucht, fällt sofort der Term $\sqrt{1-x^2}$ auf.
Dieser Term deutet darauf hin, dass nahe an den Intervallsgrenzen ($x \rightarrow \pm 1$) Auslöschung auftreten kann welche zu falschen Resultaten führt.
\cmt{Referenz zum Auslöschungskapitel}
\cmt{Untersuchen ob Division durch sehr sehr kleine Zahl auch zu Fehlern führen kann.}
Tatsächlich ist es so, dass die Implementation dieser Rekursionsformel zu Instabilitäten führt wie in Abbildung \cmt{Referenz zu Plot} zu sehen ist.

\cmt{Plot Instabilität einfügen}

Erstaunlicherweise scheint sogar Wolfram Alpha \cite{legendre:wolfram-alpha} diese Rekursionsformel zu verwenden.
Wird auf Wolfram Alpha der beispielsweise der Befehl \verb+plot[LegendreP[45, x], {x, -1, 1}]+ verwendet, um das Legendrepolynom $P^{0}_{45}$ darzustellen, treten ebenfalls in der Nähe der Intervallgrenzen Instabilitäten auf (siehe Abbildung \cmt{Referenz zu Wolfram Alpha Plot}).
\cmt{Wolfram Alpha Plot}
Dies ist umso erstaunlicher, als dass auf der Wolfram-Seite zu den zugeordneten Legendrepolynome \cite{legendre:assoc-legendre-poly-wolfram} die Rekursionsbeziehung in $l$-Richtung angegeben wird, welche keine Instabilitäten nahe den Intervallsgrenzen aufweist (siehe dazu den nächsten Abschnitt (\ref{legendre:subsection:lrichtung}).

\subsection{Rekursionsformel in $l$-Richtung
\label{legendre:subsection:lrichtung}}
Die Rekursion in $l$-Richtung verläuft in positiver $l$-Richtung, sprich nach zunehmenden $l$s.
Bei der Gleichung \eqref{legendre:recurrence-l} muss daher nur noch der Faktor $(l-m+1)$ auf die rechte Seite gebracht werden.
Daraus folgt die neue Rekursionsgleichung in $l$-Richtung \eqref{legendre:recurrence-l-neu}.
% umgeformte Rekursionsformel in l-Richtung
\begin{equation}
P^{m}_{l+1}(x)
= \frac{(2l+1)xP^{m}_{l}(x)-(l+m)P^{m}_{l-1}(x)}{(l-m+1)} 
\label{legendre:recurrence-l-neu}
\end{equation}
Die benötigten Anfangswerte für ein $P^{m}_{l}$ für die Rekursion in $l$-Richtung sind demnach $P^{m}_{m}$ und $P^{m}_{l+1}$.
Bei einem Vergleich mit der Rekursionsformel in $m$-Richtung (siehe Gleichung \eqref{legendre:recurrence-m-1} fällt auf, dass der Term $\sqrt{1-x^2}$ hier nicht mehr vorhanden ist.
Auch sonst ist kein Term ausfindig zu machen, der nahe den Intervallsgrenzen Instabilitäten verursachen könnte.
Das heisst, dass die Gleichung \eqref{legendre:recurrence-l-neu} an den Intervallsgrenzen deutlich stabiler sein sollte.
Dies bestätigt auch Implementation dieser Rekursionsformel, wie in Abbildung \cmt{Referenz zu Plot} zu sehen ist.

\cmt{Eigener Plot Stabilität}

\cmt{Können andere Probleme auftreten? Over-/Underflow bei grossen $l$s wurde schon erwähnt.}

\cmt{GNU Scientific Library verwendet diese Formel. Eventuell erwähnen, dass es eine gute Implementation gibt auch wenn die Lizenz nicht alles erlaubt.}

\subsection{Wieso trotzdem in $m$-Richtung?
\label{legendre:subsection:wiesom}}
\cmt{Vielleicht noch ein solches Unterkapitel, falls ja Titel überarbeiten}

\cmt{Eventuell interessante Kenntnisse wieso trotzdem die Rekursionsformel in m-Richtung verwendet wird.}

\cmt{Geschwindigkeit ist vielleicht besser? Anfangswerte?}


