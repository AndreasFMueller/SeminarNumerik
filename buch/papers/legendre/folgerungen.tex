%
% folgerung.tex -- Beispiel-File für die Beschreibung des Problems
%
% (c) 2020 Prof Dr Andreas Müller, Hochschule Rapperswil
%
\section{Folgerungen
\label{legendre:section:folgerungen}}
\rhead{Folgerungen}
In diesem Kapitel wurde aufgezeigt, dass sich nicht jede numerische Instabilität nur auf die Verwendung einer bestimmten Formel zurückführen lässt, dass jedoch eine Formel numerische Fehler aus einer anderen Quelle begünstigen kann.
Beide untersuchten Rekursionsbeziehungen offenbarten ihre Schwächen.
Es gibt jedoch Hinweise, dass die Formel für die $l$-Rekursion die bessere Wahl gegenüber der Formel für die $m$-Rekursion darstellt.
Daher ist zu empfehlen, nur Implementationen einer Programmbibliothek zu verwenden, die auf die $l$-Rekursion zurückgreifen.

Die Untersuchungen auf numerische Instabilität in diesem Kapitel zeigt klar auf, dass numerische Instabilitäten verschiedene Quellen haben können und erst einige Berechnungsschritte später auftauchen können.
Eine einfache Auslöschung kann somit eine ganze Kette von Effekten mit sich bringen, die erst ganz am Schluss einer Berechnung ein Resultat verfälscht.
