%
% folgerung.tex -- Beispiel-File für die Beschreibung des Problems
%
% (c) 2020 Prof Dr Andreas Müller, Hochschule Rapperswil
%
\section{Folgerungen
\label{legendre:section:folgerungen}}
\rhead{Folgerungen}
In diesem Kapitel wurde aufgezeigt, dass nicht jede Rekursionsgleichung für die zugeordneten Legendrepolynome numerisch stabil ist.
Dies gilt vor allem für Gleichungen die einen Term wie $1-x^2$ oder $x^2-1$ beinhalten.
Diese Terme verursachen numerische Instabilitäten nahe den Intervallsgrenzen von $-1$ respektive $1$.
Diese Art von Termen treten bei der Rekursion in $l$-Richtung nicht auf.
\cmt{Nur Library-Funktion verwenden, die in $l$-Richtung arbeiten. In der Dokumentation lassen sich Hinweise finden. Wenn alle Werte für jedes $l$ zurückgegeben wird ist ein deutlicher Hinweis.}


Diese Schlussfolgerung gilt jedoch nicht nur für die zugeordneten Legendrepolynome, sondern gilt auch ganz im Allgemeinen.
Demnach sollte immer im Hinterkopf behalten werden, dass jede mathematische Formel Probleme verursachen kann, wenn sie in Code umgewandelt wird.
