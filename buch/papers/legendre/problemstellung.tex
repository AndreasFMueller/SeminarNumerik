%
% problemstellung.tex -- Beispiel-File für die Beschreibung des Problems
%
% (c) 2020 Prof Dr Andreas Müller, Hochschule Rapperswil
%
\section{Problemstellung
\label{legendre:section:problemstellung}}
\rhead{Problemstellung}
Die zugeordneten Legendrepolynome sind die Lösungen der allgemeinen Legendregleichung (siehe Gleichung \eqref{legendre:legendregleichung}) \cite{legendre:assoc-legendre-poly-wolfram} \cite{legendre:assoc-legendre-diff-wolfram}.
% Legendregleichung
\begin{equation}
(1-x^2) \frac{d^2y}{dx^2}
-2x \frac{dy}{dx}
+ \left[ l(l+1)- \frac{m^2}{1-x^2} \right] y
=0
\label{legendre:legendregleichung}
\end{equation}
Dabei gilt für die beiden Parameter $l$ und $m$, dass $l\geq 0$ und $m=0, \ldots , l$.
Der Parameter $l$ ist für den Grad des zugeordneten Legendrepolynoms verantwortlich, $m$ hingegen für die Ordnung des Polynoms.
Die zugeordneten Legendrepolynome sind zudem nur auf dem Intervall $[-1, 1]$ definiert.
Verwendung finden diese Polynome vor allem als Teil der Kugelflächenfunktionen (\textit{engl.} spherical harmonics) \cite{legendre:spherical-harmonic-wolfram}.

Wie im Abschnitt \ref{legendre:section:einleitung} bereits erwähnt, gibt es Rekursionsbeziehungen für die zugeordneten Legendrepolynome.
Der englische Wikipedia-Artikel \cite{legendre:wikipedia} zu den zugeordneten Legendrepolynomen führt eine Liste solcher Rekursionsformeln, wobei $P^{m}_{l}$ das Legendrepolynom $l$-ten Grades und $m$-ter Ordnung ist.
Die ersten beiden Rekursionsformeln des Wikipedia-Artikels sind wohl die gängigsten.
Die erste Rekursionsformel \eqref{legendre:recurrence-l} verläuft in der Richtung des Parameters $l$, also in der Richtung des Grades des Legendrepolynoms.
Die zweite Rekursionsformel \eqref{legendre:recurrence-m} hingegen verläuft in der Richtung des Parameters $m$, also in der Richtung der Ordnung des Legendrepolynoms.
% Rekursionsformel in l-Richtung
\begin{equation}
(l-m+1)P^{m}_{l+1}(x)
=(2l+1)xP^{m}_{l}(x)
-(l+m)P^{m}_{l-1}(x)
\label{legendre:recurrence-l}
\end{equation}
% Rekursionsformel in m-Richtung
\begin{equation}
2mxP^{m}_{l}(x)
=-\sqrt{1-x^2}
\left[ P^{m+1}_{l}(x) + (l+m)(l-m+1)P^{m-1}_{l}(x) \right]
\label{legendre:recurrence-m}
\end{equation}
Beim Erstellen der Graphen (siehe Abbildung \ref{legendre:fig:plot-l} und Abbildung \ref{legendre:fig:plot-m} mittels diesen beiden Rekursionsformeln fällt auf, dass keine numerischen Instabilitäten ersichtlich sind, wie dies beim Graphen von Wolfram Alpha der Fall ist (vergleiche Abbildung \ref{legendre:fig:wolframalpha}).
% Plot in l-Richtung
\begin{figure}[!h]
\centering
\includegraphics[width=1.0\linewidth]{papers/legendre/plots/plot_l}
\caption{Zugeordnetes Legendrepolynom mit Grad 50 und Ordnung 3, berechnet mit der Rekursionsformel in \texorpdfstring{$l$}{l}-Richtung.}
\label{legendre:fig:plot-l}
\end{figure}
% Plot in m-Richtung
\begin{figure}[!h]
\centering
\includegraphics[width=1.0\linewidth]{papers/legendre/plots/plot_m}
\caption{Zugeordnetes Legendrepolynom mit Grad 50 und Ordnung 3, berechnet mit der Rekursionsformel in \texorpdfstring{$m$}{m}-Richtung.}
\label{legendre:fig:plot-m}
\end{figure}
Die beiden Graphen (Abbildungen \ref{legendre:fig:plot-l} und \ref{legendre:fig:plot-m}) zeigen, dass die Rekursionsformeln nicht die alleinige Ursache für die numerischen Instabilitäten des Legendrepolynom-Graphen von Wolfram Alpha (Abbildung \ref{legendre:fig:wolframalpha}) sein können.
Trotzdem lohnt es sich die beiden Rekursionsformeln etwas genauer zu untersuchen.
Sie können nämlich mitverantwortlich sein für numerische Instabilitäten, indem sie numerische Fehler aus einer anderen Quelle begünstigen.