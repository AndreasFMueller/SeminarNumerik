
%
% rationalezahlen.tex -- 
%
% (c) 2020 Benjamin Bouhars-Keller
%
\section{Rationale Zahlen
\label{kettenbruch:section:Zahlen}}
\rhead{Rationale Zahlen}
\subsection{Definition}
Die Kettenbruch entwicklung von $x \in \mathbb{R}$ bricht genau dann nach endlich
vielen Schritten ab, wenn $x$ rational ist.
Die Menge aller rationalen Zahlen wird mit $\mathbb{Q}$ bezeichnet.
Ein endlicher Kettenbruch ist ein Bruch der Form
\begin{equation}
a_0 + \cfrac{1}{a_1+\cfrac{1}{a_2+\cfrac{\cdots}{\cdots+\cfrac{1}{a_{n-1} + \cfrac{1}{a_n}}}}}
\end{equation}
in welchem $a_0, a_1,\dots,a_n$ und $b_1,b_2,\dots,b_n$ ganze Zahlen
darstellen, die mit Ausnahme möglicherweise von $a_0$ alle positiv sind.
Die Kettenbruchentwicklung von $x \in \mathbb{R}$ bricht genau dann
nach endlich vielen Schritten ab, wenn $x$ rational ist. So bilden
rationale Zahlen endliche Kettenbrüche.

\subsection{Euklidischer Algorithmus}
Die Umwandlung einer rationalen Zahl in einen Kettenbruch erfolgt
mit Hilfe des euklidischen Algorithmus.
Als Beispiel rechnen wir für $\frac{17}{10} = [1;1,2,3]$.
\begin{beispiel}
\begin{equation}
\frac{17}{10}
=
1 + \frac{7}{10}
=
1 + \cfrac{1}{\frac{10}{7}}
=
1 + \cfrac{1}{1+\frac{3}{7}}
=
1 + \cfrac{1}{1+\cfrac{1}{1+\cfrac{1}{\frac{7}{3}}}}
=
1 + \cfrac{1}{1+\cfrac{1}{2+\frac{1}{3}}}
\end{equation}
\begin{align*}
17 &= 1\cdot + 7 \\
10 &= 1\cdot 7 + 3 \\
7 &= 2\cdot 3 + 1 \\
3 &= 3\cdot 1 + 0
\end{align*}
Diese Methode kann auch umgekehrt angewendet werden.
Berechnen wir nun den Kettenbruch in einem Bruch zurück.
\begin{equation}
[1;1,2,3] \rightarrow	2 + \frac{1}{3} = \frac{7}{3}
						1 + \frac{3}{7} = \frac{10}{7}
						1 + \frac{7}{10} = \frac{17}{0}
\end{equation}
\end{beispiel}
Wie man im Beispiel sieht, ist die intuitive Berechnung der
Näherungsbrüche eines Kettenbruchs, indem man ihn von unten her
auflöst, sehr umständlich. Durch ein rekursives Bildungsgesetz für
Zähler und Nenner kann man die Berechnung erheblich vereinfachen.
Ausserdem kann man mit Hilfe dieser Rekursionsformel die Grenzwerte
unendlicher Kettenbrüche untersuchen.

Jede rationale Zahl $x$ lässt sich auf eindeutige Weise in einen
endlichen Kettenbruch entwicklen, dessen letzer Teilnenner grösser
oder gleich 2 ist.

\subsection{Rekursionsformel}
Die Berechnung der Konvergenten (siehe Abschnitt 10.3.3) eines Kettenbruches 
kann erheblich vereinfacht werden, indem eine Rekursionsformeln für den Zähler 
und den Nenner einführt wird.
\begin{beispiel}
\begin{align*}
A_0 &= b_0                     &   B_0 &= 1                                  \\
A_1 &= a_1a_0 + 1              &   B_1 &= a_1                                \\
A_k &= a_kA_{k-1} + A_{k-2}    &   B_k &= a_kB_{k-1} + B_{k-2} &(i &\ge 1),   \\
A_2 &= b_2b_1b_0 + b_2 + b_0   &   B_2 &= b_2b_1 + 1		             \\
A_3 &= b_3b_2b_1b_0 + b_3b_2 + b_3b_0 + b_1b_0 + 1 & B_3 &= b_3b_2b_1 + b_3 + b_1
\end{align*}
Dies lässt sich auch durch die folgende Matrizenschreibweise ausdrücken:
\begin{equation}
P_0 = 	\begin{pmatrix}
			A_{_1}&	A_{_2}\\
			B_{_1}&	A_{_2}
		\end{pmatrix}
		=\begin{pmatrix}
			1&	0\\
			0&	1
		\end{pmatrix}
\end{equation}
\begin{equation}
P_i = 	\begin{pmatrix}
			A_i&	A_{i-1}\\
			B_i&	B_{i-1}
		\end{pmatrix}
		=\begin{pmatrix}
			A_{i-1}&	A_{i-2}\\
			B_{i-1}&	B_{i-2}
		\end{pmatrix} 
		\begin{pmatrix}
			b_i	&	1\\
			1	&	0
		\end{pmatrix} 
\end{equation}
Rekursives Einsetzen ergibt:
\begin{equation}
		\begin{pmatrix}
			A_i&	A_{i-1}\\
			B_i&	B_{i-1}
		\end{pmatrix}
		=\begin{pmatrix}
			b_0	&	1\\
			1	&	0
		\end{pmatrix}
		\begin{pmatrix}
			b_1	&	1\\
			1	&	0
		\end{pmatrix}
		\cdots
		\begin{pmatrix}
			b_i	&	1\\
			1	&	0
		\end{pmatrix} 
		=\displaystyle\prod_{j=0}^{i}\begin{pmatrix}
			b_j	&	1\\
			1	&	0
		\end{pmatrix}
\end{equation}
Aus diesen Überlegungen resultiert eine Aussage, von deren letztem 
Teil wir noch sehr oft Gebrauch machen werden:
Somit gilt 
\begin{equation}
\det(P_i) = (-1)^i
\end{equation}
und für alle $i$:
\begin{equation}
A_iB_{i-1} - B_iA_{i-1} = (-1)^i
\end{equation}
\subsubsection{Beispiel}
Für $x = [b_0;b,\cdots,b_{i-1},\beta_1]$ gilt
\begin{equation}
x = \frac{A_{i-1}\beta_i + A_{i-2}}{B_{i-1}\beta_i + B_{i-2}}
\qquad \text{für alle $i \ge 0$.}
\end{equation}
Aus der Rekursionsformeln ergeben sich folgende Gleichung
\begin{equation}
[b_0;b_1,\cdots,b_n]
=
\frac{A_n}{B_n} = \frac{b_nA_{n-1} + A_{n-2}}{b_nB_{n-1} + B_{n-2}}
\end{equation}
und analog
\begin{equation}
[b_0;b,\cdots,b_{i-1},\beta_1] = [b_0;b_1,\cdots,b_{i-1},\beta_i]
=
\frac{A_{i-1}\beta_i + A_{i-2}}{B_{i-1}\beta_i + B_{i-2}}
\end{equation}
\end{beispiel}
Hiermit haben wir die wichtigsten Zusammenhänge in Bezug auf die Näherungszahlen herausgearbeitet.

