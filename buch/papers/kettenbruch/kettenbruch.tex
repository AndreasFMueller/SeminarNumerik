%
% kettenbruch.tex -- Paper zum Thema <kettenbruch>
%
% (c) 2020 Hochschule Rapperswil
%
\documentclass{book}
\usepackage{etex}
\usepackage{geometry}
\geometry{papersize={170mm,240mm},total={140mm,200mm},top=21mm,bindingoffset=10mm}
\usepackage[english,ngerman]{babel}
\usepackage[utf8]{inputenc}
\usepackage[T1]{fontenc}
\usepackage{cancel}
\usepackage{times}
\usepackage{amsmath,amscd}
\usepackage{amssymb}
\usepackage{amsfonts}
\usepackage{amsthm}
\usepackage{graphicx}
\usepackage{fancyhdr}
\usepackage{textcomp}
\usepackage{txfonts}
\newcommand\hmmax{0}
\newcommand\bmmax{0}
\usepackage{bm}
\usepackage{epic}
\usepackage{verbatim}
%\usepackage{suffix}
\usepackage{paralist}
\usepackage{makeidx}
\usepackage{array}
\usepackage{hyperref}
\usepackage{subfigure}
\usepackage{tikz}
\usepackage{pgfplots}
\usepackage{pgfplotstable}
\usepackage{pdftexcmds}
\usepackage{pgfmath}
\usepackage[autostyle=false,english=american]{csquotes}
\usepackage{wasysym}
\usepackage{environ}
\usepackage{appendix}
\usepackage[all]{xy}
\usetikzlibrary{calc,intersections,through,backgrounds,graphs,positioning,shapes,arrows,fit,math}
\usetikzlibrary{patterns,decorations.pathreplacing}
\usetikzlibrary{decorations.pathreplacing}
\usetikzlibrary{external}
\usetikzlibrary{datavisualization}
\usepackage[europeanvoltages,
            europeancurrents,
            europeanresistors,   % rectangular shape
            americaninductors,   % "4-bumbs" shape
            europeanports,       % rectangular logic ports
            siunitx,             % #1<#2>
            emptydiodes,
            noarrowmos,
            smartlabels]         % lables are rotated in a smart way
           {circuitikz}          %
\usepackage{siunitx}
\usepackage{tabularx}
\usetikzlibrary{arrows}
\usepackage{algpseudocode}
\usepackage{algorithm}
\usepackage{gensymb}
\usepackage{mathtools}

% import the listing styles

\usepackage{caption}
\usepackage[mode=buildnew]{standalone}
\usepackage[backend=bibtex]{biblatex}
\begin{document}
\def\chapterauthor#1{{\large #1}\bigskip\bigskip}

\newenvironment{beispiel}{%
\begin{proof}[Beispiel]%
\renewcommand{\qedsymbol}{$\bigcirc$}
}{\end{proof}}
\setcounter{page}{352}
\setcounter{chapter}{16}
\allowdisplaybreaks
\renewcommand{\floatpagefraction}{0.7}
\pagestyle{fancy}
\lhead{}
\rhead{}

\chapter{Kettenbrüche\label{chapter:kettenbruch}}
\lhead{Kettenbrüche}
\begin{refsection}
\chapterauthor{Benjamin Bouhafs-Keller}

%
% einleitung.tex -- Beispiel-File für die Einleitung
%
% (c) 2020 Prof Dr Andreas Müller, Hochschule Rapperswil
%
\section{Einleitung\label{laplace:section:einleitung}}
\rhead{Einleitung}
Lorem ipsum dolor sit amet, consetetur sadipscing elitr, sed diam
nonumy eirmod tempor invidunt ut labore et dolore magna aliquyam
erat, sed diam voluptua \cite{laplace:bibtex}.
At vero eos et accusam et justo duo dolores et ea rebum.
Stet clita kasd gubergren, no sea takimata sanctus est Lorem ipsum
dolor sit amet.

Lorem ipsum dolor sit amet, consetetur sadipscing elitr, sed diam
nonumy eirmod tempor invidunt ut labore et dolore magna aliquyam
erat, sed diam voluptua.
At vero eos et accusam et justo duo dolores et ea rebum.  Stet clita
kasd gubergren, no sea takimata sanctus est Lorem ipsum dolor sit
amet.




%
% rationalezahlen.tex -- 
%
% (c) 2020 Benjamin Bouhars-Keller
%
\section{Rationale Zahlen
\label{kettenbruch:section:Zahlen}}
\rhead{Rationale Zahlen}
\subsection{endliche Kettenbrüche}
Die Kettenbruchentwicklung von $x \in \mathbb{R}$ bricht genau dann nach endlich
vielen Schritten ab, wenn $x$ rational ist.
Die Menge aller rationalen Zahlen wird mit $\mathbb{Q}$ bezeichnet.
Ein endlicher Kettenbruch ist ein Bruch der Form
\begin{equation}
a_0 + \cfrac{b_1}{a_1+\cfrac{b_2}{a_2+\cfrac{\cdots}{\cdots+\cfrac{b_n}{a_{n-1} + \cfrac{1}{a_n}}}}}
\end{equation}
in welchem $a_0, a_1,\dots,a_n$ und $b_1,b_2,\dots,b_n$ ganze Zahlen
darstellen, die mit Ausnahme möglicherweise von $a_0$ alle positiv sind.
Die Kettenbruchentwicklung von $x \in \mathbb{R}$ bricht genau dann
nach endlich vielen Schritten ab, wenn $x$ rational ist. So bilden
rationale Zahlen endliche Kettenbrüche.

\subsection{Euklidischer Algorithmus}
Die Umwandlung einer rationalen Zahl in einen Kettenbruch erfolgt
mit Hilfe des euklidischen Algorithmus.
Als Beispiel rechnen wir für $\frac{17}{10} = [1;1,2,3]$.
Jede rationale Zahl $x$ lässt sich auf eindeutige Weise in einen
endlichen Kettenbruch entwicklen, dessen letzer Teilnenner grösser
oder gleich 2 ist.
\begin{beispiel}
\begin{equation}
\frac{17}{10}
=
1 + \frac{7}{10}
=
1 + \cfrac{1}{\frac{10}{7}}
=
1 + \cfrac{1}{1+\frac{3}{7}}
=
1 + \cfrac{1}{1+\cfrac{1}{1+\cfrac{1}{\frac{7}{3}}}}
=
1 + \cfrac{1}{1+\cfrac{1}{2+\frac{1}{3}}}
\end{equation}
\begin{align*}
17 &= 1\cdot 10 + 7 \\
10 &= 1\cdot 7 + 3 \\
7 &= 2\cdot 3 + 1 \\
3 &= 3\cdot 1 + 0
\end{align*}
Diese Methode kann auch umgekehrt angewendet werden.
Berechnen wir nun den Kettenbruch in einem Bruch zurück.
\begin{equation}
[1;1,2,3] 	
\rightarrow Es gilt		2 + \frac{1}{3} = \frac{7}{3} \\
\rightarrow Es gilt		1 + \frac{3}{7} = \frac{10}{7} \\
\rightarrow Es gilt		1 + \frac{7}{10} = \frac{17}{0} \\
\end{equation}
\end{beispiel}
Wie man im Beispiel sieht, ist die intuitive Berechnung der
Näherungsbrüche eines Kettenbruchs, indem man ihn von unten her
auflöst, sehr umständlich. Durch ein rekursives Bildungsgesetz für
Zähler und Nenner kann man die Berechnung erheblich vereinfachen.
Ausserdem kann man mit Hilfe dieser Rekursionsformel die Grenzwerte
unendlicher Kettenbrüche untersuchen.

\subsection{Rekursionsformel}
Die Berechnung der Konvergenten (siehe Abschnitt 10.3.3) eines Kettenbruches 
kann erheblich vereinfacht werden, indem eine Rekursionsformeln für den Zähler 
und den Nenner eingeführt wird. Anders formuliert können Grenzwerte für unendliche 
Zahlen oder Funktionen mit Hilfe der Rekursionsformel schneller definiert werden.
Betrachten wir zunächst die folgende Näherungsbrüche:

\begin{align*}
a_0 + \frac{b_1}{a_1} , a_0 + \cfrac{b_1}{a_1 + \frac{b_2}{a_2}} , \cdots
\end{align*}
Wie in Kapitel 17.1.1 erwähnt wird hier der Nenner wieder als Kettenbruch dargestellt.
Insofern  werden Terme der Form
\begin{align*}
a_k + \cfrac{b_{k + 1}}{a_{k + 1} + \frac{p}{q}}
\end{align*}
immer ausgerechnet
\begin{align*}
\cfrac{b_{k+1}}{a_{k+1} + \frac{p}{q}} = \frac{b_{k+1} \cdot q}{a_{k+1} \cdot q + p}
\end{align*}
Dies lässt sich auch durch die folgende Matrizenschreibweise ausdrücken:
\begin{equation}
		\begin{pmatrix}
			\textrm{neuer Zähler}\\
			\textrm{neuer Nenner}
		\end{pmatrix}
 = 		\begin{pmatrix}
			b_{k+1} \cdot q\\
			a_{k+1} \cdot q + p
		\end{pmatrix}
		=\begin{pmatrix}
			0&	b_{k+1}\\
			1&	a_{k+1}
		\end{pmatrix}
		\begin{pmatrix}
		p \\
		q
		\end{pmatrix}
\end{equation}
Der ``unterste'' Bruch ist $\frac{b_n}{a_n}$, also in Vektorschreibweise
\begin{align*}
a_0 + 	\begin{pmatrix}
			0& b_1\\
			1& a_1
		\end{pmatrix}
		\cdots
		\begin{pmatrix}
			0& b_{n-1}\\
			1& a_{n-1}
		\end{pmatrix}
		\begin{pmatrix}
			b_n\\
			a_n
		\end{pmatrix}
\textrm{Aufbau ``von rechts nach links''}
\end{align*}
aber das Matrizenprodukt kann man auch von links beginnend ausmultiplizieren, als
\begin{align*}
		\begin{pmatrix}
			0& b_1\\
			1& a_1
		\end{pmatrix}
		\cdots
		\begin{pmatrix}
			0& b_{n-1}\\
			1& a_{n-1}
		\end{pmatrix}
		\begin{pmatrix}
			b_n\\
			a_n
		\end{pmatrix}
\end{align*}
d.h. man kann bei $b_1, a_1$ beginnen.
Man kann die beiden Schritten
\begin{align*}
\frac{p}{q} \rightarrow a_k + \frac{p}{q} \rightarrow \cfrac{b_k}{a_k + \frac{p}{q}}
\end{align*}
auch separat als Matrizen schreiben.
\begin{align*}
a_k + \frac{p}{q} = \frac{a_k \cdot q + p}{q} \enspace d.h. \enspace	
	\begin{pmatrix}
		1& a_k\\
		0& 1
	\end{pmatrix}
	\begin{pmatrix}
		p\\
		q
	\end{pmatrix}
\end{align*}
\begin{align*}
cfrac{b_k}{\frac{p}{q}} = \frac{b_k \cdot q}{p} \textrm{d.h.}	
	\begin{pmatrix}
		1& b_k\\
		0& 1
	\end{pmatrix}
	\begin{pmatrix}
		p\\
		q
	\end{pmatrix}
\end{align*}
Tatsächlich ist die Kombination der beiden Schritte
\begin{align*}
		\begin{pmatrix}
			0& b_n\\
			1& 0
		\end{pmatrix}
		\begin{pmatrix}
			1& a_n\\
			0& 1
		\end{pmatrix}
	=	\begin{pmatrix}
			0& b_n\\
			1& a_n
		\end{pmatrix}
\end{align*}
wie auf der vorangegangene Seite gefunden.
Aufbau des Kettenbruchs von rechts
\begin{align*}
		\begin{pmatrix}
			1& a_0\\
			0& 1
		\end{pmatrix}
		\begin{pmatrix}
			0& b_1\\
			1& a_1
		\end{pmatrix}
		\begin{pmatrix}
			0& b_2\\
			1& a_2
		\end{pmatrix}
		\cdots
		\begin{pmatrix}
			0& b_{n-1}\\
			1& a_{n-1}
		\end{pmatrix}
		\begin{pmatrix}
			b_n\\
			a_n
		\end{pmatrix}
\end{align*}
Aufbau des Kettenbruchs von links: Matrizen von links her ausmultiplizieren
Start:
\begin{align*}
		\begin{pmatrix}
			1& a_0\\
			0& 1
		\end{pmatrix}
	=	\begin{pmatrix}
			A_{-1}& A_0\\
			B_{-1}& B_0
		\end{pmatrix}
	=	P_0
\end{align*}
Schritt 1:
\begin{align*}
		\begin{pmatrix}
			1& a_0\\
			0& 1
		\end{pmatrix}
		\begin{pmatrix}
			1& b_1\\
			0& a_1
		\end{pmatrix}
	=	\begin{pmatrix}
			a_0& b_1 + a_0 \cdot a_1\\
			1&	a_1
		\end{pmatrix}\\
	=	\begin{pmatrix}
			A_{-1}& A_0\\
			B_{-1}& B_0
		\end{pmatrix}
		\begin{pmatrix}
			1& b_1\\
			0& a_1
		\end{pmatrix}
	=	\begin{pmatrix}
			A_0& A_1\\
			B_0& B_1
		\end{pmatrix}
	=	P_1
\end{align*}
Schritt 2:
\begin{align*}
P_1 \cdot	\begin{pmatrix}
				0& b_2\\
				1& a_2
			\end{pmatrix}\\
		=	\begin{pmatrix}
				A_0& A_1\\
				B_0& B_1
			\end{pmatrix}
			\begin{pmatrix}
				0& b_2\\
				1& a_2
			\end{pmatrix}
		=	\begin{pmatrix}
				A_1& b_2 \cdot A_0 + a_2 \cdot A_1\\
				B_1& b_2 \cdot B_0 + a_2 \cdot B_1
			\end{pmatrix}
		= 	P_2
\end{align*}
Schritt 3:
\begin{align*}
P_2 \cdot	\begin{pmatrix}
				0& b_3\\
				1& a_3
			\end{pmatrix}\\
		=	\begin{pmatrix}
				A_1& A_2\\
				B_1& B_2
			\end{pmatrix}
			\begin{pmatrix}
				0& b_3\\
				1& a_3
			\end{pmatrix}
		=	\begin{pmatrix}
				A_2& b_3 \cdot A_1 + a_3 \cdot A_2\\
				B_2& b_3 \cdot B_1 + a_3 \cdot B_2
			\end{pmatrix}
		= 	P_2
\end{align*}
So macht man weiter bis zum Schritt $n_1$, der die Matrix
\begin{align*}
P_{n-1} = 	\begin{pmatrix}
				A_{n-2}& A_{n-1}\\
				B_{n-2}& B_{n-1}			
			\end{pmatrix}
\end{align*}
liefert.\\
Zähler und Nenner des Näherungsbruchs bekommt man dann durch Multiplikation mit $\begin{pmatrix}
b_n\\
a_n
\end{pmatrix}$. Dies ist aber die zweite Spalte der Matrix $\begin{pmatrix}
0& b_n\\
1& a_n
\end{pmatrix}$, d.h., Zähler und Nenner stehen in der zweite Spalte von $P_n$.

\underline{Folgerung:}\\

$\frac{A_n}{B_n}$ ist der Wert von
\begin{align*}
a_0 + \cfrac{b_1}{a_1+\cfrac{b_2}{a_2+\cfrac{\cdots}{\cdots+\cfrac{b_{n-1}}{a_{n-1} + \cfrac{b_n}{a_n}}}}}
\end{align*}
Die Berechnung von $A_n, B_n$ kann man jetzt auch ohne die Matrizenschreibweise aufschreiben:
Start:
\begin{align*}
A_{-1} = 0		\qquad		A_0 = a_0 \\
B_{-1} = 1		\qquad		B_0 = 1 \\
\end{align*}
$\rightarrow$ 0-te Näherung: $\frac{A_0}{B_0} = a_0$
Schritt:
\begin{align*}
k \rightarrow k + 1 \\
A_{k+1} = A_{k-1} \cdot b_k + A_k \cdot a_k \\
B_{k+1} = B_{k-1} \cdot b_k + B_k \cdot a_k
\end{align*}
Näherungsbruch $n: \frac{A_n}{B_n}$

Schliesslich sind für reguläre oder einfache Kettenbrüche die Formeln einfacher, weil $b_k = 1$
und so:
\begin{align*}
A_{k+1} = A_{k-1} + A_k \cdot a_k \\
B_{k+1} = B_{k-1} + B_k \cdot a_k
\end{align*}
Hiermit haben wir die wichtigsten Zusammenhänge in Bezug auf die Näherungszahlen herausgearbeitet.

%
% irrationalezahlen.tex 
%
% (c) 2020 Benjamin Bouhafs-Keller
%
\section{Irrationale Zahlen
\label{kettenbruch:section:Irrationale Zahlen}}
\rhead{Irrationale Zahlen}
\subsection{Definition}
Zahlen, die nicht rational sind, heissen irrational. Die rationale
und die irrationalen Zahlen bilden zusammen die reellen Zahlen.
Anders formuliert sind irrationale Zahlen von einem Quotienten der
nicht durch zweie ganzer Zahlen darstellbar ist gekenntzeichnet
(Bsp:  $\pi$, $e$)
Die Menge aller reellen Zahlen bezeichnet man mit $\mathbb{R}$.
Irrationale Zahlen bilden unendliche Kettenbrüche, d.~h.~sind durch
eine periodische oder nicht periodische Kettenbruchentwicklung
ausgezeichnet.
Ein unendlicher regelmässiger Kettenbruch wird in folgender Form dargestellt
\begin{equation}
a_0 + \cfrac{1}{a_1+\cfrac{1}{a_1+\cfrac{1}{a_3+\cfrac{1}{\cdots}}}}
\end{equation}
wobei $a_0,a_1,a_2,\dots$ eine unendliche Folge von positiven
ganzen Zahlen bilden. Sie sind auch wie beim endlichen Kettenbruch
alle bis auf möglicherweise $a_0$ positiv.


Zunächst sollen einige Beispiele für die Kettenbruchenentwicklung
irrationaler Zahlen betrachtet werden.

\subsection{Periodische Kettenbrüche}
In diesem Abschnitt wollen wir nun auf eine spezielle Form eingehen
und zwar auf unendliche regelmässige Kettebrüche, die ein bemerkenswertes
Bildungsgesetz befolgen. Das besondere an diesen Kettenbrüchen ist,
dass gleiche Teilnenner wiederholt auftreten.
Für den Kettenbruch $[3;1,2,1,6,1,2,1,6,\dots]$ heisst das: den
Kettenbruch die Periode 1,2,1,6 mit der Periodenlänge $n=4$ beträgt
und wird in der Form $[3;\overline{1,2,1,6}]$ geschrieben.
Kommen wir nun zu einem Beispiel. Wir betrachten den periodischen
einfachen Kettenbruch $[3;\bar{6}] = (3,6,6,6,\dots)$.
\begin{equation}
[3;\bar{6}]
=
3 + \cfrac{1}{6+\cfrac{1}{6+\cfrac{1}{6+\cfrac{1}{6\dots}}}}
=
x
\end{equation}
Die Euklidische Methode mit der rekursive Bildungsgesetz für Zähler
und Nenner würde hier unendlich sein und deshalb schwierig zu
berechnen.
Um diesen Kettenbruch vollständig darzustellen müssen wir ein System
erzeugen. Daher werden nur die ersten Brüche (Zahlen) betrachtet.

\begin{equation}
[3;6;6;6]
=
3 + \cfrac{1}{6+\cfrac{1}{6+\frac{1}{6}}}
=
3 + \cfrac{1}{6+\cfrac{1}{\frac{37}{6}}}
=
3 + \cfrac{1}{6+\frac{6}{37}}
=
3 + \cfrac{1}{\frac{228}{37}}
=
3 + \frac{37}{228}
=
\frac{684+37}{228}
=
\frac{721}{228}
\approx
3.162280702
\end{equation}
Wenn wir den Kettenbruchg so verändern das die Kettenbruchentwicklung 
immer mit 6 vortläuft dann können wir unser Kettenbruch als $x$ bezeichnen 
und darauf 3 addieren. Dies ergibt folgendes System das wir infolge quadratische
Gleichungen lösen können:
\begin{align*}
y = x+3 &= 6 + \cfrac{1}{6+\cfrac{1}{6+\dots}} = [6;\bar{6}]
\\
\Rightarrow y &= 6 + \frac{1}{y}	&&\vert\;\cdot y
\\
\Rightarrow y^2 &= 6y + 1
\\
\Rightarrow y &= 3\pm \sqrt{9+1} = 3 \pm \sqrt{10}\qquad\Rightarrow\qquad x = \sqrt{10}
\approx
3.16227766
\end{align*}
Jede irrationale Zahl $\Phi$ besitzt unendlich viele Näherungsbrüche
$\frac{p}{q}$ mit
\begin{equation}
\biggl|\Phi-\frac{p}{q}\biggr|<\frac{1}{\sqrt{5 q^2}}.
\end{equation}
Betrachten wir ein anderes System, $(\overline{2,3}) =  (2,3,2,3,\dots)$.
Sein Wert ist als unendlicher Kettenbruch irrational und lässt sich
berechne. Setzen wir $x:=(\overline{2,3})$, dann gilt
\begin{equation}
x
=
2 + \cfrac{1}{3+\cfrac{1}{2+\cfrac{1}{3+\dots}}}
=
2 + \cfrac{1}{3+\frac{1}{x}}.
\end{equation}
Dies führt auf die quadratische Gleichung $x^2 - 2x - \frac{2}{3}
= 0$, was die positive Lösung $x = \frac{3+\sqrt{15}}{3}$ liefert.
Der oben aufgeführte Kettenbruch $x$ ist ein Beispiel für periodische
einfache Kettenbrüche, die Nullstelle eines quadratischen Polynoms
mit rationalen Koeffizienten ist. Anders gesagt ist die reelle
Irrationalzahl $x$ Wurzel einer quadratischen Gleichung:
\begin{equation}
ax^2 + bx + c = 0
\end{equation}
dann ist die Kettenbruchentwicklung von $x$ periodisch, das bedeutet
die Existenz einer Schranke $n_0$ und einer Periode $k \in \mathbb{N}$
mit $x_n+k = x_n$ für alle $n\ge n_0$.

\subsubsection{Satz von Euler-Lagrange}
Jeder periodische einfache Kettenbruch ist eine quadratische
Irrationalzahl und umgekehrt. Dabei bezeichnet eine quadratische
Irrationalzahl eine irrationale Zahl und stellt eine algebraische
Zahl dar.

\subsection{Nicht periodische Kettenbrüche}
Es stellen sich dieselben Fragen wie im vorangegangenen Abschnitt.
Neu hinzu kommt das Problem, ob bzw. wann die Kettenbruchentwicklung
überhaupt konvergiert.
Für eine unendliche Folge $x_0,x_1,\dots$ ist der Kettenbruch
$[x_0,x_1,\dots]$ nur dann definiert wenn die Folge der Näherungsbrüche
$(\frac{p_n}{q_n})$ konvergiert. In diesem Fall hat der unendliche
Kettenbruch $[x_0,x_1,\dots]$ den Wert
\begin{equation}
\lim_{n\to\infty} [x_0;x_1;\cdots;x_n]
\end{equation}
oder anders dargestellt
\begin{equation}
\omega
=
x_0 + \cfrac{1}{x_1+\cfrac{1}{x_2+\frac{1}{x_n+\cdots}}}
\end{equation}
Folgt $\omega > 0$ durch einen unendliche Kettenbruch darstellbar
ist, wenn die endlichen Kettenbrüche $n$-ter Ordnung
$[x_0;x_1,x_2,\dots,x_n]$ gegen $\omega$ konvergieren.

\subsubsection{Beweis}
Betrachten wir folgenden Kettenbruch
\begin{align*}
\frac{19}{51} &= [0;2,1,2,6]
\\
	K_0 &= [0] = 0
\\
	K_1 &= [0;2] = 0 + \frac{1}{2} = \frac{1}{2}
\\
	K_2 &= [0;2,1] = 0 + \cfrac{1}{2+\frac{1}{1}} = \frac{1}{3}
\\
	K_3 &= [0;2,1,2] = 0 + \cfrac{1}{2+\cfrac{1}{1+\frac{1}{2}}} = \frac{3}{8}
\\
	K_4 &= [0;2,1,2,6] = \frac{19}{51}
\end{align*}
Folge der Näherungsbrüche
\begin{enumerate}
\item
$K_0 < K_2 < K_4 < \cdots$
\item
$K_1 > K_3 > K_5 > \cdots$
\item
$K_{2s} < K_{2r+1}, r,s \in \mathbb{N}$
\end{enumerate}

Es gilt offensichtlich
$K_0 < K_2 < K_4 < \cdots < K_{2n} < \cdots < K_{2n+1} < \cdots < K_5
< K_3 < K_1$
und $\frac{19}{51}$ wird von jeweils zwei aufeinanderfolgenden
Näherungsbrüchen eingeschlossen.
 
Die Folge der Näherungsbrüche mit geraden Index bilden eine streng
monoton steigende Folge und sind nach oben begrenzt. Also konvergiert
diese Folge von unten gegen einen Grenzwert, den wir als $\alpha$
bezeichnen. Anderseits bilden die Näherungsbrüche mit ungeradem
Index eine streng monoton fallende Folge und sind nach unten begrenzt.
Somit sind beide Folgen monoton und beschränkt und konvergieren in
$\alpha$.

Es gibt auch Zahlen, deren Kettenbruchdarstellung gewisse
Regelmässigkeiten aufweisen, ohne periodisch zu sein. Zum Beispiel
die Identität $e = [2;1,2,1,1,4,1,1,6,1,\cdots]$. Dieser
Kettenbruch ist nicht periodisch, die Teilnenner können aber durch
eine rekursive Folge bestimmt werden.

\subsubsection{Bemerkung}
\begin{itemize}
\item
Jede positive rationale Zahl lässt sich durch einen endlichen
Kettenbruch darstellen, und jeder endliche Kettenbruch stellt eine
positve rationale Zahl dar.
\item
Jeder unendliche Kettenbruch stellt eine positive irrationale Zahl
dar, und jede irrationale Zahl lässt sich durch einen unendliche
Kettenbruch darstellen.
\item
Jeder periodische Kettenbruch stellt eine quadratische Irrationalität
dar und jede quadratische Irrationalität lässt sich durch einen
periodischen Kettenbruch darstellen.
\end{itemize}
%
% loesung.tex -- Beispiel-File für die Beschreibung der Loesung
%
% (c) 2020 Prof Dr Andreas Müller, Hochschule Rapperswil
%
\section{Approximation
\label{kettenbruch:section:Approximation}}
\rhead{Approximation}

In der Einleitung wurde erwähnt, dass die Bestimmung von guten
Näherungsbrüchen eine wichtige Anwendung von Kettenbrüchen ist. Es
gilt nämlich, dass jeder Näherungsbruch der Kettenbruchentwicklung
einer reellen Zahl eine besonders gute rationale Näherung dieser
Zahl ist.

\subsection{Definition}

Eine rationale Zahl $\frac{a}{b}$ mit $b>0$ heisst Best approximation
erster Art an eine reelle Zahl $x$, wenn es keine von $\frac{a}{b}$
verschiedene rationale Zahl mit gleichem oder kleinerem Nenner gibt,
die bezüglich des euklidischen Absolutbetrages näher bei $x$ liegt.
Das heisst, dann gilt für alle rationalen Zahlen $\frac{c}{d} \ne
\frac{a}{b}$ mit $0<d\le b$:
\begin{equation}
\biggl|x-\frac{a}{b}\biggr| < \biggl| x-\frac{c}{d}\biggr|.
\end{equation}

\subsection{Näherungsgesetz}
Ziel dieses Abschnitt ist es, eine genügend gute Approximation der
Näherungsbrüche nachzuweisen. Gibt man sich eine beliebige Zahl $x$
vor, so kann man sich die Frage stellen, welche "unkürzbaren" Brüche
$\frac{p}{q}$ mit vorgegebenem Höchstnenner sich gut approximieren
lässt.

\subsubsection{Beispiel Nr.1}
Näherung von $\pi$ mit dem (unendliche Dezimalbruch):
$\pi = [3;7,15,1,292,1,1,1,2,1,3,1,14,2,\cdots]$
Die Näherung $3.14 = \frac{314}{100}$ ist eine Näherung. Aber
$\frac{22}{7} = 3.14285714\dots$ hat einen viel kleineren Nenner und
ist eine deutlich bessere Näherung von $\pi$.
Eine noch bessere Näherung ist der Kettenbruch
\begin{equation}
\frac{355}{113} = 3 + \cfrac{1}{7+\cfrac{1}{15+\frac{1}{1}}} = 3.1415\bar{92}
\end{equation}
Folgende Näherungswerte von $\pi$ können schnell und einfach gerechnet werden:
\begin{equation}
3,\frac{22}{7} \approx 3.143 ; \frac{333}{106} \approx 3.14151 ; \frac{355}{113} \approx 3.1415929 ; \frac{103993}{33102} \approx 3.1415926530 ; \cdots.
\end{equation}
Die Bestapproximation ist einfach formuliert durch die Bestimmung
derjenigen rationalen Brüchen, welche von einer gegebenen rationalen
oder irrationalen Zahl einen festgelegten minimalen Abstand haben
und dabei einen möglichst kleinen positiven Nenner besitzen.

\subsubsection{Beispiel Nr.2}
Die Kettenbruch von $\tan^{-1}(x)$ sieht folgendermassen aus

\begin{equation}
\tan^{-1}(x)
=
\cfrac{x}{1+\cfrac{x^2}{3+\cfrac{4x^2}{5+\cfrac{9x^2}{7+\frac{16x^2}{9+\cdots}}}}} 
\end{equation}
$(|x|< 1)$
Das Gleichungssystem kann umgeschrieben werden als Funktion $f_n$
\begin{equation}
f_n(x) = \frac{x}{1+}\frac{x^2}{3+}\frac{4x^2}{5+}\cdots\frac{(n-1)^2 x^2}{2n-1}
\end{equation}
$(|n|\ge 2)$
Hiermit kann nach $n$te Bildung der Kettenbruchreaktion eine Limite darstellen.
\begin{equation}
\tan^{-1}(x) = \lim_{n\to\infty} f_n(x)
\end{equation}
$(|x| < 1)$
Die Konvergenz der Funktion kann infolge einem Beispiel beurteilt werden. 
\begin{equation}
\tan^{-1}(1) = \pi/4 \approx 0.785398
\end{equation}

\begin{table}
\centering
\begin{tabular}{>{$}c<{$}>{$}l<{$}}
n	& f_n(1) 	\\
\hline
2	& 0.750000 	\\
3	& 0.791667 	\\
4	& 0.784314 	\\
5	& 0.785586 	\\
6	& 0.785366 	\\
7	& 0.785404	\\
8	& 0.785397	\\
9	& 0.785398	\\
\hline
\end{tabular}
\caption{XXX Beschreibung der Tabelle einfügen
\label{kettenbruch:tabelle}}
\end{table}

In wenigen und einfachen Schritten haben wir mit Hilfe einer
Kettenbruchentwicklung ein System gebildet das die Konvergenz der
Funktion $\tan^{-1}(x)$ vorantreibt und präzise Resultate liefert.

%
% problemstellung.tex -- Beispiel-File für die Beschreibung des Problems
%
% (c) 2020 Prof Dr Andreas Müller, Hochschule Rapperswil
%
\section{Folgerungen
\label{ew:section:folgerungen}}
\rhead{Folgerungen}
Sed ut perspiciatis unde omnis iste natus error sit voluptatem
accusantium doloremque laudantium, totam rem aperiam, eaque ipsa
quae ab illo inventore veritatis et quasi architecto beatae vitae
dicta sunt explicabo. Nemo enim ipsam voluptatem quia voluptas sit
aspernatur aut odit aut fugit, sed quia consequuntur magni dolores
eos qui ratione voluptatem sequi nesciunt. Neque porro quisquam
est, qui dolorem ipsum quia dolor sit amet, consectetur, adipisci
velit, sed quia non numquam eius modi tempora incidunt ut labore
et dolore magnam aliquam quaerat voluptatem. Ut enim ad minima
veniam, quis nostrum exercitationem ullam corporis suscipit laboriosam,
nisi ut aliquid ex ea commodi consequatur? Quis autem vel eum iure
reprehenderit qui in ea voluptate velit esse quam nihil molestiae
consequatur, vel illum qui dolorem eum fugiat quo voluptas nulla
pariatur?

\subsection{De finibus bonorum et malorum
\label{ew:subsection:malorum}}
At vero eos et accusamus et iusto odio dignissimos ducimus qui
blanditiis praesentium voluptatum deleniti atque corrupti quos
dolores et quas molestias excepturi sint occaecati cupiditate non
provident, similique sunt in culpa qui officia deserunt mollitia
animi, id est laborum et dolorum fuga. Et harum quidem rerum facilis
est et expedita distinctio. Nam libero tempore, cum soluta nobis
est eligendi optio cumque nihil impedit quo minus id quod maxime
placeat facere possimus, omnis voluptas assumenda est, omnis dolor
repellendus. Temporibus autem quibusdam et aut officiis debitis aut
rerum necessitatibus saepe eveniet ut et voluptates repudiandae
sint et molestiae non recusandae. Itaque earum rerum hic tenetur a
sapiente delectus, ut aut reiciendis voluptatibus maiores alias
consequatur aut perferendis doloribus asperiores repellat.




\printbibliography[heading=subbibliography]
\end{refsection}
\end{document}
