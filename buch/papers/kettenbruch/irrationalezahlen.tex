%
% irrationalezahlen.tex 
%
% (c) 2020 Benjamin Bouhafs-Keller
%
\section{Irrationale Zahlen
\label{kettenbruch:section:Irrationale Zahlen}}
\rhead{Irrationale Zahlen}
\subsection{Definition}
Zahlen, die nicht rational sind, heissen irrational. Die rationale
und die irrationalen Zahlen bilden zusammen die reellen Zahlen.
Anders formuliert sind irrationale Zahlen von einem Quotienten der
nicht durch zweie ganzer Zahlen darstellbar ist gekenntzeichnet
(Bsp:  $\pi$, $e$
Die Menge aller reellen Zahlen bezeichnet man mit $\mathbb{R}$.

Irrationale Zahlen bilden unendliche Kettenbrüche, d.~h.~sind durch
eine periodische oder nicht periodische Kettenbruchentwicklung
ausgezeichnet.

Ein unendlicher regelmässiger Kettenbruch wird in folgender Form dargestellt
\begin{equation}
a_0 + \cfrac{1}{a_1+\cfrac{1}{a_1+\cfrac{1}{a_3+\cfrac{1}{\cdots}}}}
\end{equation}
wobei die $a_0,a_1,a_2,\dots$ eine unendliche Folge von positiven
ganzen Zahlen bilden. Sie sind auch wie beim endlichen Kettenbruch
alle bis auf möglicherweise $a_0$ positiv.

Zunächst sollen einige Beispiele für die Kettenbruchenentwicklung
irrationaler Zahlen betrachtet werden.

\subsection{Periodische Kettenbrüche}
In diesem Abschnitt wollen wir nun auf eine spezielle Form eingehen
und zwar auf unendliche regelmässige Kettebrüche, die ein bemerkenswertes
Bildungsgesetz befolgen. Das besondere an diesen Kettenbrüchen ist,
dass gleiche Teilnenner wiederholt auftreten.

Für den Kettenbruch $[3;1,2,1,6,1,2,1,6,\dots]$ heisst das: den
Kettenbruch die Periode 1,2,1,6 mit der Periodenlänge $n=4$ beträgt
und wird in der Form $[3;\overline{1,2,1,6}]$ geschrieben.

Kommen wir nun zu einem Beispiel. Wir betrachten den periodischen
einfachen Kettenbruch $[3;\bar{6}] = (3,6,6,6,\dots)$.
\begin{equation}
[3;\bar{6}]
=
3 + \cfrac{1}{6+\cfrac{1}{6+\cfrac{1}{6+\cfrac{1}{6\dots}}}}
=
x
\end{equation}
Die Euklidische Methode mit der rekursive Bildungsgesetz für Zähler
und Nenner würde hier unendlich sein und deshalb schwierig zu
berechnen.

Um diesen Kettenbruch vollständig darzustellen müssen wir ein System
erzeugen. Daher werden nur die ersten Brüche (Zahlen) betrachtet.

\begin{equation}
[3;6;6;6]
=
3 + \cfrac{1}{6+\cfrac{1}{6+\frac{1}{6}}}
=
3 + \cfrac{1}{6+\cfrac{1}{\frac{37}{6}}}
=
3 + \cfrac{1}{6+\frac{6}{37}}
=
3 + \cfrac{1}{\frac{228}{37}}
=
3 + \frac{37}{228}
=
\frac{684+37}{228}
=
\frac{721}{228}
\approx
3.162280702
\end{equation}
Hier wird der Kettenbruch so verändert das eine 6 steht.
Wir können unser Kettenbruch als $x$ bezeichnet und addieren 3
darauf. Dies ergibt folgendes System das wir infolge quadratische
Gleichungen lösen können:
\begin{align*}
y = x+3 &= 6 + \cfrac{1}{6+\cfrac{1}{6+\dots}} = [6;\bar{6}]
\\
\Rightarrow y &= 6 + \frac{1}{y}	&&\vert\;\cdot y
\\
\Rightarrow y^2 &= 6y + 1
\\
\Rightarrow y &= 3\pm \sqrt{9+1} = 3 \pm \sqrt{10}\qquad\Rightarrow\qquad x = \sqrt{10}
\approx
3.16227766
\end{align*}
Jede irrationale Zahl $\Phi$ besitzt unendlich viele Näherungsbrüche
$\frac{p}{q}$ mit
\begin{equation}
\biggl|\Phi-\frac{p}{q}\biggr|<\frac{1}{\sqrt{5 q^2}}.
\end{equation}

Betrachten wir ein anderes System, $(\overline{2,3}) =  (2,3,2,3,\dots)$.
Sein Wert ist als unendlicher Kettenbruch irrational und lässt sich
berechnen, denn setzen wir $x:=(\overline{2,3})$, dann gilt
\begin{equation}
x
=
2 + \cfrac{1}{3+\cfrac{1}{2+\cfrac{1}{3+\dots}}}
=
2 + \cfrac{1}{3+\frac{1}{x}}.
\end{equation}
Dies führt auf die quadratische Gleichung $x^2 - 2x - \frac{2}{3}
= 0$, was die positive Lösung $x = \frac{3+\sqrt{15}}{3}$ liefert.

Der oben aufgeführte Kettenbruch $x$ ist ein Beispiel für periodische
einfache Kettenbrüche, die Nullstelle eines quadratischen Polynoms
mit rationalen Koeffizienten ist. Anders gesagt ist die reelle
Irrationalzahl $x$ Wurzel einer quadratischen Gleichung:
\begin{equation}
ax^2 + bx + c = 0
\end{equation}
dann ist die Kettenbruchentwicklung von $x$ periodisch, das bedeutet
die Existenz einer Schranke $n_0$ und einer Periode $k \in \mathbb{N}$
mit $x_n+k = x_n$ für alle $n\ge n_0$.

\subsubsection{Satz von Euler-Lagrange}
Jeder periodische einfache Kettenbruch ist eine quadratische
Irrationalzahl und umgekehrt. Dabei bezeichnet eine quadratische
Irrationalzahl eine irrationale Zahl und stellt eine algebraische
Zahl dar.

\subsection{Nicht periodische Kettenbrüche}
Es stellen sich dieselben Fragen wie im vorangegangenen Abschnitt.
Neu hinzu kommt das Problem, ob bzw. wann die Kettenbruchentwicklung
überhaupt konvergiert.

Für eine unendliche Folge $x_0,x_1,\dots$ ist der Kettenbruch
$[x_0,x_1,\dots]$ nur dann definiert wenn die Folge der Näherungsbrüche
$(\frac{p_n}{q_n})$ konvergiert. In diesem Fall hat der unendliche
Kettenbruch $[x_0,x_1,\dots]$ den Wert
\begin{equation}
\lim_{n\to\infty} [x_0;x_1;\cdots;x_n]
\end{equation}
oder anders dargestellt
\begin{equation}
\omega
=
x_0 + \cfrac{1}{x_1+\cfrac{1}{x_2+\frac{1}{x_n+\cdots}}}
\end{equation}

Folgt $\omega > 0$ durch einen unendliche Kettenbruch darstellbar
ist, wenn die endlichen Kettenbrüche $n$-ter Ordnung
$[x_0;x_1,x_2,\dots,x_n]$ gegen $\omega$ konvergieren.

\subsubsection{Beweis}
Betrachten wir folgenden Kettenbruch
\begin{align*}
\frac{19}{51} &= [0;2,1,2,6]
\\
	K_0 &= [0] = 0
\\
	K_1 &= [0;2] = 0 + \frac{1}{2} = \frac{1}{2}
\\
	K_2 &= [0;2,1] = 0 + \cfrac{1}{2+\frac{1}{1}} = \frac{1}{3}
\\
	K_3 &= [0;2,1,2] = 0 + \cfrac{1}{2+\cfrac{1}{1+\frac{1}{2}}} = \frac{3}{8}
\\
	K_4 &= [0;2,1,2,6] = \frac{19}{51}
\end{align*}

Folge der Näherungsbrüche
\begin{enumerate}
\item
$K_0 < K_2 < K_4 < \cdots$
\item
$K_1 > K_3 > K_5 > \cdots$
\item
$K_{2s} < K_{2r+1}, r,s \in \mathbb{N}$
\end{enumerate}

Es gilt offensichtlich
$K_0 < K_2 < K_4 < \cdots < K_{2n} < \cdots < K_{2n+1} < \cdots < K_5
< K_3 < K_1$
und $\frac{19}{51}$ wird von jeweils zwei aufeinanderfolgenden
Näherungsbrüchen eingeschlossen.
 
Die Folge der Näherungsbrüche mit geraden Index bilden eine streng
monoton steigende Folge und sind nach oben begrenzt. Also konvergiert
diese Folge von unten gegen einen Grenzwert, den wir als $\alpha$
bezeichnen. Anderseits bilden die Näherungsbrüche mit ungeradem
Index eine streng monoton fallende Folge und sind nach unten begrenzt.
Somit sind beide Folgen monoton und beschränkt und konvergieren in
$\alpha$.

Es gibt auch Zahlen, deren Kettenbruchdarstellung gewisse
Regelmässigkeiten aufweisen, ohne periodisch zu sein. Zum Beispiel
die Identität $e = [2;1,2,1,1,4,1,1,6,1,\cdots]$. Dieser
Kettenbruch ist nicht periodisch, die Teilnenner können aber durch
eine rekursive Folge bestimmt werden.

\subsubsection{Bemerkung}
\begin{itemize}
\item
Jede positive rationale Zahl lässt sich durch einen endlichen
Kettenbruch darstellen, und jeder endliche Kettenbruch stellt eine
positve rationale Zahl dar.
\item
Jeder unendliche Kettenbruch stellt eine positive irrationale Zahl
dar, und jede irrationale Zahl lässt sich durch einen unendliche
Kettenbruch darstellen.
\item
Jeder periodische Kettenbruch stellt eine quadratische Irrationalität
dar und jede quadratische Irrationalität lässt sich durch einen
periodischen Kettenbruch darstellen.
\end{itemize}

