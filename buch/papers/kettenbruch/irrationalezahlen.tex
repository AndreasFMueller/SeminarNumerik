%
% irrationalezahlen.tex 
%
% (c) 2020 Benjamin Bouhafs-Keller
%
\section{Irrationale Zahlen
\label{kettenbruch:section:Irrationale Zahlen}}
\rhead{Irrationale Zahlen}
\subsection{Definition}
Die Menge aller reellen Zahlen bezeichnet man mit $\mathbb{R}$.
Irrationale Zahlen bilden unendliche Kettenbrüche, d.~h.~sind durch
eine periodische oder nicht periodische Kettenbruchentwicklung
ausgezeichnet.
Ein unendlicher regelmässiger Kettenbruch wird in folgender Form dargestellt
\begin{equation}
a_0 + \cfrac{1}{a_1+\cfrac{1}{a_1+\cfrac{1}{a_3+\cfrac{1}{\dots}}}}
\end{equation}
wobei $a_0,a_1,a_2,\dots$ eine unendliche Folge von positiven
ganzen Zahlen bilden. Sie sind auch wie beim endlichen Kettenbruch
alle bis auf möglicherweise $a_0$ positiv.


Zunächst sollen einige Beispiele für die Kettenbruchenentwicklung
irrationaler Zahlen betrachtet werden.

\subsection{Periodische Kettenbrüche}
In diesem Abschnitt wollen wir nun auf eine spezielle Form eingehen
und zwar auf unendliche regelmässige Kettebrüche, die ein bemerkenswertes
Bildungsgesetz befolgen. Das besondere an diesen Kettenbrüchen ist,
dass gleiche Teilnenner wiederholt auftreten.
Für den Kettenbruch $[3;1,2,1,6,1,2,1,6,\dots]$ heisst das: den
Kettenbruch die Periode 1,2,1,6 mit der Periodenlänge $n=4$ beträgt
und wird in der Form $[3;\overline{1,2,1,6}]$ geschrieben.

\begin{beispiel}
Betrachten wir den periodischen einfachen Kettenbruch $[3;\bar{6}] = (3,6,6,6,\dots)$.
\begin{equation}
[3;\bar{6}]
=
3 + \cfrac{1}{6+\cfrac{1}{6+\cfrac{1}{6+\cfrac{1}{6+\dots}}}}
=
x
\end{equation}
Die Euklidische Methode mit der rekursive Bildungsgesetz für Zähler
und Nenner würde hier unendlich sein und deshalb schwierig zu
berechnen.
Um diesen Kettenbruch vollständig darzustellen müssen wir ein Muster
erzeugen. Daher werden nur die ersten Brüche (Zahlen) betrachtet.

\begin{equation}
[3;6,6,6]
=
3 + \cfrac{1}{6+\cfrac{1}{6+\cfrac{1}{6}}}
=
3 + \cfrac{1}{6+\cfrac{1}{\frac{37}{6}}}
=
3 + \cfrac{1}{6+\frac{6}{37}}
=
3 + \cfrac{1}{\frac{228}{37}}
=
3 + \frac{37}{228}
=
\frac{684+37}{228}
=
\frac{721}{228}
\approx
3.162280702
\end{equation}
Wenn wir den periodischen Kettenbruch so verändern das die Kettenbruchentwicklung 
immer mit 6 vortläuft dann können wir unser Kettenbruch als $x$ bezeichnen 
und darauf 3 addieren. Anders formuliert ist wie in einem periodischen Kettenbruch 
ein {\color{red}Teil oder inneren kettenbruch} der ähnlich ist wie der ganze Kettenbruch. Dies ergibt 
folgendes Muster das wir infolge quadratische
Gleichungen lösen können:
\begin{align*}
y = x+3 &= 6 + \cfrac{1}{\color{red}6+\cfrac{1}{6+\dots}} = [6;\bar{6}]
\\
\Rightarrow y &= 6 + \frac{1}{{\color{red}y}}	&&\vert\;\cdot y
\\
\Rightarrow y^2 &= 6y + 1
\\
\Rightarrow y &= 3\pm \sqrt{9+1} = 3 \pm \sqrt{10}\qquad\Rightarrow\qquad x = \sqrt{10}
\approx
3.16227766
\end{align*}
\end{beispiel}

Mit dieser Gleichung haben wir die Menge der Zahlen mit periodischer 
Kettenbruchentwicklung bestimmt. 

\begin{beispiel}
Betrachten wir einen neuen Kettenbruch, $[\overline{2,3}] =  [2;3,2,3,\dots]$.
Sein Wert ist als unendlicher Kettenbruch irrational und lässt sich
wie folgt berechnen. Setzen wir $x:=(\overline{2,3})$, dann gilt
\begin{equation}
x
=
2 + \cfrac{1}{3+\cfrac{1}{\color{red}2+\cfrac{1}{3+\dots}}}
=
2 + \cfrac{1}{3+\frac{1}{\color{red}x}}.
\end{equation}
Dies führt auf die quadratische Gleichung 
\begin{align*}
x &= 2 + \cfrac{1}{3+\frac{1}{{\color{red}x}}}
\\
x - 2 &= \cfrac{1}{3+\frac{1}{x}}
\\
(x - 2)\biggl(3 + \frac{1}{x}\biggr) &= 1
\\
3x - 6 - \frac{2}{x} &= 0 &&\vert \times\frac{x}{3}
\\
x^2 - 2x - \frac{2}{3} &= 0
\end{align*}
was die positive Lösung $x = \frac{3+\sqrt{15}}{3}$ liefert.
Der oben aufgeführte Kettenbruch $x$ ist ein Beispiel für periodische
einfache Kettenbrüche, die Nullstelle eines quadratischen Polynoms
mit rationalen Koeffizienten ist. Anders gesagt ist die reelle
Irrationalzahl $x$ Wurzel einer quadratischen Gleichung:
\begin{equation}
ax^2 + bx + c = 0
\end{equation}
dann ist die Kettenbruchentwicklung von $x$ periodisch, das bedeutet
die Existenz einer Schranke $n_0$ und einer Periode $k \in \mathbb{N}$
mit $x_n+k = x_n$ für alle $n\ge n_0$.
\end{beispiel}



\subsubsection{Satz von Euler-Lagrange}
Jeder periodische regulärer Kettenbruch ist eine quadratische
Irrationalzahl und umgekehrt. $x \in \mathbb{R}$ hat genau dann eine 
Darstellung als unendlicher, periodischer Kettenbruch, wenn $x$ eine
reell-quadratische Irrationalzahl ist (d.h. $x \notin \mathbb{Q}$ ist Lösung
einer quadratischen Gleichung $aX^2 + bX + c = 0, a \neq 0$ mit rationalen 
Koeffizienten $a,b,c$) \cite{kettenbruch:perron}.

\subsection{Nicht periodische Kettenbrüche}
Es stellen sich dieselben Fragen wie im vorangegangenen Abschnitt.
Neu hinzu kommt das Problem, ob bzw. wann die Kettenbruchentwicklung
überhaupt konvergiert.
Für eine unendliche Folge $x_0,x_1,\dots$ ist der Kettenbruch
$[x_0;x_1,\dots]$ nur dann definiert wenn die Folge der Näherungsbrüche
$(\frac{p_n}{q_n})$ konvergiert. In diesem Fall hat der unendliche
Kettenbruch $[x_0;x_1,\dots]$ den Wert
\begin{equation}
\lim_{n\to\infty} [x_0;x_1,\dots,x_n]
\end{equation}
oder anders dargestellt
\begin{equation}
\omega
=
x_0 + \cfrac{1}{x_1+\cfrac{1}{x_2+\frac{1}{x_n+\dots}}}
\end{equation}
Folgt $\omega > 0$ durch einen unendliche Kettenbruch darstellbar
ist, wenn die endlichen Kettenbrüche $n$-ter Ordnung
$[x_0;x_1,x_2,\dots,x_n]$ gegen $\omega$ konvergieren.

\subsubsection{Beispiel}
\begin{beispiel}
Betrachten wir folgenden Kettenbruch
\begin{align*}
\frac{1351}{3625} &= [0;2,1,2,6,2,1,1,2,1,3] = 0.372689655172
\\
	K_0 &= [0] = 0
\\
	K_1 &= [0;2] = 0 + \frac{1}{2} = \frac{1}{2} = 0.5
\\
	K_2 &= [0;2,1] = 0 + \cfrac{1}{2+\frac{1}{1}} = \frac{1}{3} = 0.\bar{3}
\\
	K_3 &= [0;2,1,2] = 0 + \cfrac{1}{2+\cfrac{1}{1+\frac{1}{2}}} = \frac{3}{8} = 0.375
\\
	K_4 &= [0;2,1,2,6] = 0 + \cfrac{1}{2+\cfrac{1}{1+\cfrac{1}{2+\frac{1}{6}}}} = \frac{19}{51} =0.37254
\\
	K_5 &= [0;2,1,2,6,2] = 0 + \cfrac{1}{2+\cfrac{1}{1+\cfrac{1}{2+\cfrac{1}{6+\frac{1}{2}}}}} = \frac{41}{110} = 0.37\bar{27}
\\
	K_6 &= [0;2,1,2,6,2,1] = 0 + \cfrac{1}{2+\cfrac{1}{1+\cfrac{1}{2+\cfrac{1}{6+\cfrac{1}{2+\frac{1}{1}}}}}} = \frac{60}{161} = 0.37267
\\
	K_7 &= [0;2,1,2,6,2,1,1] = 0 + \cfrac{1}{2+\cfrac{1}{1+\cfrac{1}{2+\cfrac{1}{6+\cfrac{1}{2+\cfrac{1}{1+\frac{1}{1}}}}}}} = \frac{101}{271} = 0.37269
\\
	K_8 &= [0;2,1,2,6,2,1,1,2] = 0 + \cfrac{1}{2+\cfrac{1}{1+\cfrac{1}{2+\cfrac{1}{6+\cfrac{1}{2+\cfrac{1}{1+\cfrac{1}{1+\frac{1}{2}}}}}}}} = \frac{262}{703} = 0.372688
\\
	K_9 &= [0;2,1,2,6,2,1,1,2,1] = 0 + \cfrac{1}{2+\cfrac{1}{1+\cfrac{1}{2+\cfrac{1}{6+\cfrac{1}{2+\cfrac{1}{1+\cfrac{1}{1+\cfrac{1}{2+\frac{1}{1}}}}}}}}} = \frac{363}{974} = 0.3726899
\\
	K_10 &= [0;2,1,2,6,2,1,1,2,1,3] = 0 + \cfrac{1}{2+\cfrac{1}{1+\cfrac{1}{2+\cfrac{1}{6+\cfrac{1}{2+\cfrac{1}{1+\cfrac{1}{1+\cfrac{1}{2+\cfrac{1}{1+\frac{1}{3}}}}}}}}}} = \frac{1351}{3625} = 0.372689655172
\end{align*}
Mit diesem Beispiel werden die Teilkettenbrüche und Näherung zum Endresultat ersichtlich.
Folge der Näherungsbrüche mit geradem (bzw. ungeradem) Index
ist streng monoton steigend (bzw. fallend) und jeder Näherungsbruch
mit geradem Index ist kleiner als jeder Näherungsbruch mit ungeradem Index\cite{kettenbruch:proseminar}.
\begin{enumerate}
\item
$K_0 < K_2 < K_4 < \cdots$
\item
$K_1 > K_3 > K_5 > \cdots$
\item
$K_{2s} < K_{2r+1}, r,s \in \mathbb{N}$
\end{enumerate}

Es gilt offensichtlich
$K_0 < K_2 < K_4 < \cdots < K_{2n} < \cdots < K_{2n+1} < \cdots < K_5
< K_3 < K_1$
und $\frac{1351}{3625}$ wird von jeweils zwei aufeinanderfolgenden
Näherungsbrüchen eingeschlossen.
\end{beispiel}

Die Folge der Näherungsbrüche des unendlichen Kettenbruchs
$[a_0;a_1,a_2,\dots]$ mit $a_0 \in \mathbb{Z}$ konvergiert 
von unten gegen einen Grenzwert, den wir als $\alpha$
bezeichnen. Anderseits sind die Näherungsbrüche mit ungeradem
Index nach unten begrenzt. Somit sind beide Folgen monoton und beschränkt 
und konvergieren in $\alpha$. Den erwähnten Ansatz gilt nur für reguläre Kettenbrüche, es 
wäre zum Beispiel nicht der Fall, wenn als Teilnenner beliebige positive reelle Zahlen 
zugelassen wären. Der Kettenbruch $[b_0;b_1,\dots]$ konvergiert genau dann, wenn die Summe
$\sum\limits_{i=0}^\infty b_i $ divergiert.\\
Hiermit haben wir beobachtet wie die $k_i$ den Grenzwert von oben und unten her approximieren.

Es gibt auch Zahlen, deren Kettenbruchdarstellung gewisse
Regelmässigkeiten aufweisen, ohne periodisch zu sein. Zum Beispiel
die Identität $e = [2;1,2,1,1,4,1,1,6,1,\dots]$. Dieser
Kettenbruch ist nicht periodisch, die Teilnenner können aber durch
eine rekursive Folge bestimmt werden.

\subsubsection{Zusammenfassung}
\begin{itemize}
\item
Jede positive rationale Zahl lässt sich durch einen endlichen
Kettenbruch darstellen, und jeder endliche Kettenbruch stellt eine
positve rationale Zahl dar.
\item
Jeder unendliche Kettenbruch stellt eine positive irrationale Zahl
dar, und jede irrationale Zahl lässt sich durch einen unendliche
Kettenbruch darstellen.
\item
Jeder periodische Kettenbruch stellt eine quadratische Irrationalität
dar und jede quadratische Irrationalität lässt sich durch einen
periodischen Kettenbruch darstellen.
\end{itemize}
