%
% problemstellung.tex 
%
% (c) 2020 Benjamin Bouhafs-Keller
%
\section{Folgerungen
\label{kettenbruch:section:folgerungen}}
\rhead{Folgerungen}
Die Arbeit soll ein kleinen Einblick in die Theorie der Kettenbrüche
geben.
Logarithmen, Kreisbögen, Quadraturen, andere Kurven und schwer
greifbare Zahlen können als Kettenbrüche ausgedruckt werden. Rationale
Zahlen können als endliche Kettenbrüche dargestellt werden und somit
irrationale Zahlen als unendliche Kettenbrüche.

Ein weiterer Aspekt der Kettenbruchtheorie ist, dass quadratische
Irrationalitäten endlich durch einen periodischen Kettenbruch
dargestellt werden können.

Besonders irrationale Zahlen können durch Kettenbrüche eine
Approximative Näherung des Endresultat liefern. Mit Hilfe der
Kettenbruchtheorie können Konvergente der Kettenbruchentwicklung
eine Abschätzung der gesuchte irrationale Zahl ermöglichen. Damit
kann der Reiz zur Unendliche Entwicklungen von Funktionen einen
Genauen Ausdruck gegeben werden.
