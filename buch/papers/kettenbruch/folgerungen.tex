%
% problemstellung.tex -- Beispiel-File für die Beschreibung des Problems
%
% (c) 2020 Prof Dr Andreas Müller, Hochschule Rapperswil
%
\section{Folgerungen
\label{kettenbruch:section:folgerungen}}
Die Arbeit soll ein kleinen Einblick in die Theorie der Kettenbrüche geben.
Logarithmen, Kreisbögen, Quadraturen, andere Kurven und schwer
greifbare Zahlen können als Kettenbrüche ausgedruckt werden. Rationale
\index{Logarithmus}%
\index{Kreisbogen}%
\index{Quadratur}%
Zahlen können als endliche Kettenbrüche dargestellt werden und somit
irrationale Zahlen als unendliche Kettenbrüche.

Ein weiterer Aspekt der Kettenbruchtheorie ist, dass quadratische Irrationalitäten endlich durch einen periodischen Kettenbruch dargestellt werden können.

Besonders irrationale Zahlen können durch Kettenbrüche eine Näherung
des Endresultats liefern.
Mit Hilfe der Kettenbruchtheorie können die Konvergenten von
Funktionen schneller ermittelt werden und eine Abschätzung der
gesuchten irrationale Zahl ermöglichen.
Damit kann der Wunsch zur unendliche Entwicklungen von Funktionen
einen genauen Ausdruck gegeben werden.
\rhead{Folgerungen}
