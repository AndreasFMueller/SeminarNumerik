%
% loesung.tex -- Beispiel-File für die Beschreibung der Loesung
%
% (c) 2020 Prof Dr Andreas Müller, Hochschule Rapperswil
%
\section{Approximation
\label{kettenbruch:section:Approximation}}
\rhead{Approximation}

In der Einleitung wurde erwähnt, dass die Bestimmung von guten
Näherungsbrüchen eine wichtige Anwendung von Kettenbrüchen ist. Es
gilt nämlich, dass jeder Näherungsbruch der Kettenbruchentwicklung
einer reellen Zahl eine besonders gute rationale Näherung dieser
Zahl ist.

\subsection{Definition}

Eine rationale Zahl $\frac{a}{b}$ mit $b>0$ heisst beste Näherung
erster Art an eine reelle Zahl $x$, wenn es keine von $\frac{a}{b}$
verschiedene rationale Zahl mit gleichem oder kleinerem Nenner gibt,
die bezüglich des Absolutbetrages näher bei $x$ liegt.
Das heisst, dann gilt für alle rationalen Zahlen $\frac{c}{d} \ne
\frac{a}{b}$ mit $0<d\le b$:
\begin{equation}
\biggl|x-\frac{a}{b}\biggr| < \biggl| x-\frac{c}{d}\biggr|.
\end{equation}

\subsection{Näherungsgesetz}
Ziel dieses Abschnitt ist es, eine genügend gute Approximation der
Näherungsbrüche nachzuweisen. Gibt man sich eine beliebige Zahl $x$
vor, so kann man sich die Frage stellen, welche "unkürzbaren" Brüche
$\frac{p}{q}$ mit vorgegebenem Höchstnenner sich gut approximieren
lässt.

\begin{beispiel}
Näherung von $\pi$ mit dem unendliche Dezimalbruch:
$\pi = [3;7,15,1,292,1,1,1,2,1,3,1,14,2,\cdots]$
Die Näherung $3.14 = \frac{314}{100}$ ist eine Näherung. Aber
$\frac{22}{7} = 3.14285714\dots$ hat einen viel kleineren Nenner und
ist eine deutlich bessere Näherung von $\pi$.
Eine noch bessere Näherung ist der Kettenbruch
\begin{equation}
\frac{355}{113} = 3 + \cfrac{1}{7+\cfrac{1}{15+\cfrac{1}{1}}} = 3.1415\overline{92}
\end{equation}
Folgende Näherungswerte von $\pi$ können schnell und einfach gerechnet werden:
\begin{equation}
3,\frac{22}{7} \approx 3.143 ; \frac{333}{106} \approx 3.14151 ; \frac{355}{113} 
\approx 3.1415929 ; \frac{103993}{33102} \approx 3.1415926530 ; \cdots.
\end{equation}
\end{beispiel}
Die Bestapproximation ist einfach formuliert durch die Bestimmung
derjenigen rationalen Brüchen, welche von einer gegebenen rationalen
oder irrationalen Zahl einen festgelegten minimalen Abstand haben
und dabei einen möglichst kleinen positiven Nenner besitzen.\cite{kettenbruch:numerical-analysis}

\subsubsection{Beispiel}
\begin{beispiel}
Die Funktion $\tan^{-1}(x)$ spielt bei der Berechnung von $\pi$ an vielen Stellen eine Rolle. 
Der Kettenbruch von $\tan^{-1}(x)$ kann folgendermassen dargestellt werden.
\begin{equation}
\tan^{-1}(x)
=
\cfrac{x}{1+\cfrac{x^2}{3+\cfrac{4x^2}{5+\cfrac{9x^2}{7+\cfrac{16x^2}{9+\cdots}}}}}
\qquad	(|x|< 1).
\end{equation}
Diese spezielle Darstellungsform der Arkustangens Funktion wird in Kapitel 19 begründet.
Somit kann der Kettenbruch auch umgeschrieben werden als Funktion $f_n$
\begin{equation}
f_n(x) = \frac{x}{1+}\frac{x^2}{3+}\frac{4x^2}{5+}\cdots\frac{(n-1)^2 x^2}{2n-1}
\qquad	(|n|\ge 2)
\end{equation}
\end{beispiel}
Hiermit kann nach $n$-te Bildung der Kettenbruchreaktion einen Grenzwert
erreichen:
\begin{equation}
\tan^{-1}(x) = \lim_{n\to\infty} f_n(x), \qquad (|x| < 1)
\end{equation}
Die Konvergenz der Funktion kann an einem Beispiel beurteilt werden. 
\begin{equation}
\tan^{-1}(1) = \pi/4 \approx 0.785398
\end{equation}

\begin{table}
\centering
\begin{tabular}{>{$}c<{$}>{$}l<{$}}
n	& f_n(1) 	\\
\hline
2	& 0.750000 	\\
3	& 0.791667 	\\
4	& 0.784314 	\\
5	& 0.785586 	\\
6	& 0.785366 	\\
7	& 0.785404	\\
8	& 0.785397	\\
9	& 0.785398	\\
\hline
\end{tabular}
\caption{Näherungsstufen mit Kettenbruchentwicklung von der Funktion $\tan^{-1}(1)$
\label{kettenbruch:tabelle}}
\end{table}

In wenigen und einfachen Schritten haben wir mit Hilfe einer
Kettenbruchentwicklung ein System gebildet das die Konvergenz der
Funktion $\tan^{-1}(x)$ vorantreibt und präzise Resultate liefert.
