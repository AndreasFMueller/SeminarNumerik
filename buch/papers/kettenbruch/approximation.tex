%
% loesung.tex -- Beispiel-File für die Beschreibung der Loesung
%
% (c) 2020 Prof Dr Andreas Müller, Hochschule Rapperswil
%
\section{Approximation
\label{kettenbruch:section:Approximation}}
\rhead{Approximation}

In der Einleitung wurde erwähnt, dass die Bestimmung von guten
Näherungsbrüchen eine wichtige Anwendung von Kettenbrüchen ist. Es
gilt nämlich, dass jeder Näherungsbruch der Kettenbruchentwicklung
einer reellen Zahl eine besonders gute rationale Näherung dieser
Zahl ist.

\subsection{Definition}

Eine rationale Zahl $\frac{a}{b}$ mit $b>0$ heisst beste Näherung
erster Art an eine reelle Zahl $x$, wenn es keine von $\frac{a}{b}$
verschiedene rationale Zahl mit gleichem oder kleinerem Nenner gibt,
die bezüglich des Absolutbetrages näher bei $x$ liegt.
Das heisst, dann gilt für alle rationalen Zahlen $\frac{c}{d} \ne
\frac{a}{b}$ mit $0<d\le b$:
\begin{equation}
\biggl|x-\frac{a}{b}\biggr| < \biggl| x-\frac{c}{d}\biggr|.
\end{equation}

\subsection{Näherungsgesetz}
Ziel dieses Abschnittes ist, eine genügend gute Approximation der
Näherungsbrüche zu zeigen. Wir werden hier Resultate zusammentragen und 
diese dokumentieren. Gibt man sich eine beliebige Zahl $x$
vor, so kann man sich die Frage stellen, welche ``unkürzbaren'' Brüche
$\frac{p}{q}$ mit vorgegebenem Höchstnenner sich gut approximieren
lässt. 
\begin{beispiel}
Näherung von $\pi$ mit dem unendliche Kettenbruch
\[
\pi
=
[3;7,15,1,292,1,1,1,2,1,3,1,14,2,\cdots].
\]
Die Näherung $3.14 = \frac{314}{100}$ ist eine Näherung.
Aber $\frac{22}{7} = 3.14285714\dots$ hat einen viel kleineren Nenner und
ist eine deutlich bessere Näherung von $\pi$.
Eine noch bessere Näherung ist der Näherungsbruch
\begin{equation}
\frac{355}{113}
=
3 + \cfrac{1}{7+\cfrac{1}{15+\cfrac{1}{1}}}
=
3.1415\overline{92}.
\end{equation}
Folgende Näherungswerte von $\pi$ können schnell und einfach berechnet werden:
\begin{equation*}
3, \quad
\frac{22}{7} \approx 3.143, \quad
\frac{333}{106} \approx 3.14151, \quad
\frac{355}{113} \approx 3.1415929, \quad
\frac{103993}{33102} \approx 3.1415926530, \quad
\cdots.
\qedhere
\end{equation*}
\end{beispiel}
Die Bestapproximation ist einfach formuliert durch die Bestimmung
derjenigen rationalen Brüchen, welche von einer gegebenen rationalen
oder irrationalen Zahl einen festgelegten minimalen Abstand haben
und dabei einen möglichst kleinen positiven Nenner besitzen. Diesbezüglich liefern Kettenbrüche
bestmögliche Approximationen durch rationale Zahlen
\cite{kettenbruch:numerical-analysis}.

\subsection{Approximation einer Funktion}
Die Funktion $\tan^{-1}(x)$ spielt bei der Berechnung von $\pi$ an
vielen Stellen eine Rolle. 
Der Kettenbruch von $\tan^{-1}(x)$ kann folgendermassen dargestellt werden.
\begin{equation}
\tan^{-1}(x)
=
\cfrac{x}{1+\cfrac{x^2}{3+\cfrac{4x^2}{5+\cfrac{9x^2}{7+\cfrac{16x^2}{9+\cfrac{\cdots}{\cdots}}}}}}
\qquad	(|x|< 1).
\end{equation}
Diese spezielle Darstellungsform der Arkustangensfunktion wird in Kapitel~19
begründet.
Somit kann der Kettenbruch auch umgeschrieben werden als Funktion $f_n$
\begin{equation}
f_n(x) = \cfrac{x}{1+\cfrac{x^2}{3+\cfrac{4x^2}{5+\cfrac{\cdots}{\cdots+\cfrac{(n-1)^2x^2}{2n-1}}}}}
\qquad	(|x|< 1).
\end{equation}
%\end{beispiel}
Hiermit kann nach der $n$-ten Bildung der Kettenbruchrekursion einen Grenzwert
erreichen:
\begin{equation}
\tan^{-1}(x) = \lim_{n\to\infty} f_n(x), \qquad (|x| < 1)
\end{equation}
Die Konvergenz der Funktion kann an einem Beispiel beurteilt werden. 
In Tabelle~\ref{kettenbruch:tabelle}
sind die Näherungsbrüche von 
\begin{equation}
\tan^{-1}(1) = \pi/4 \approx 0.785398
\end{equation}
zusammengestellt.

\begin{table}
\centering
\begin{tabular}{>{$}c<{$}>{$}l<{$}}
n	& f_n(1) 	\\
\hline
2	& 0.\underline{7}50000 	\\
3	& 0.\underline{7}91667 	\\
4	& 0.\underline{78}4314 	\\
5	& 0.\underline{785}586 	\\
6	& 0.\underline{7853}66 	\\
7	& 0.\underline{785}404	\\
8	& 0.\underline{78539}7	\\
9	& 0.\underline{785398}	\\
\hline
\end{tabular}
\caption{Näherungsstufen mit Kettenbruchentwicklung von der Funktion $\tan^{-1}(1)$
\label{kettenbruch:tabelle}}
\end{table}
In wenigen und einfachen Schritten haben wir mit Hilfe einer
Kettenbruchentwicklung eine Folge gebildet die gegen den Grenzwert
$\tan^{-1}(1)$ konvergiert und schnell präzise Resultate liefert.
