%
% einleitung.tex -- Beispiel-File für die Einleitung
%
% (c) 2020 Benjamin Bouhafs-Keller
%
\section{Einleitung\label{kettenbruch:section:einleitung}}
\rhead{Einleitung}
Die Entwicklung der Theorie der Kettenbrüche ist durch das Bedürfnis,
Brüche oder schwer fassbare Zahlen zu approximieren, motiviert
worden. Kettenbrüche fanden Verwendung bei der Annäherung von
Verhältnisgrössen in Form von Brüchen, zur Ermittlung von Schaltjahren
bei der Kalenderberechnung, zur Annäherung natürlicher Konstanten
wie $e$ und $\pi$ und zum Beweis der Irrationalität bestimmter
Zahlen. Mit Hilfe von Kettenbruchsysteme können Approximationsalgorithmen 
für Funktionen entwickelt werden. Die vielseiten Aspekten der Kettenbrüche
werden wir in den folgenden Kapiteln erläutern.

\subsection{Definition\label{kettenbruch:section:1s}}
\begin{itemize}
\item
Ein Kettenbruch ist eine eindeutige Darstellungsform der reellen Zahlen. 
Ein Allgemeiner Kettenbruch ist der Form:
\begin{beispiel}
\begin{equation}
a_0 + \cfrac{b_1}{a_1+\cfrac{b_2}{a_2+\cfrac{\cdots}{\cdots+\cfrac{b_{n-1}}{a_{n-1} + \cfrac{b_n}{a_n}}}}}
\end{equation}
wobei $a_i, b_i \in \mathbb{Z}$
\end{beispiel}
Hier ist eine Darstellung mit eckigen Klammern nicht möglich, da im Zähler keine Einsen mehr stehen.
\item
Ein regulärer oder einfacher Kettenbruch ist definiert als ein Bruch der Form:
\begin{align*}
[x_0;x_1,x_2,\cdots,x_n]
=
\frac{a}{b}=x_0+\cfrac{1}{x_1+\cfrac{1}{x_3+\cfrac{1}{x_4+\cdots}}}
\end{align*}
mit $x_0,x_1,x_2,x_3, \dots \in \mathbb{N}$.
Aus dieser Darstellung wird ersichtlich, dass ein Kettenbruch ein Bruch dessen 
Nenner wieder als Kettenbruch geschrieben ist.
Eine alternative Schreibweis für Kettenbrüche ist $[x_0;x_1,x_2,x_3,\dots]$ 
ist eine kompakte, Art irrationalen Zahlen  zu beschreiben. 
Bei Dezimalbrüchen würde dies unendlich gehen (Bspw. 0.333333\dots) da die Dezimalbruchentwicklung offensichtlich nicht als Bruch mit einer Zehnerpotenz (10, 100, 1000 etc.) im Nenner  geschrieben werden kann.
\end{itemize}

In der Mathematik können Bruchsysteme auch weitere Darstellung haben. 
Hier ein Paar Beispiele:

für unendliche Folge von $x_0,x_1,\dots$ gilt der Grenzwert
\begin{align*}
\lim_{x\to\infty} [x_0;x_1,\dots,x_n] 
\end{align*}  
oder auch in Matrixform
\begin{equation*}
	\begin{pmatrix}
		x_0&	1\\
		1  &	0
	\end{pmatrix}
	\begin{pmatrix}
		x_1&	1\\
		1  &	0
	\end{pmatrix}
	\cdots
	\begin{pmatrix}
		x_n&	1\\
		1  &	0
	\end{pmatrix}
	=\begin{pmatrix}
		p_n&	p_{n-1}\\
		q_n&	q_{n-1}
	\end{pmatrix} 
\end{equation*}
hier eine Form des Bilgungsgesetz für die Näherungsbrüche

Kettenbrüche können schnell konvergierende Approximationen liefern.
Damit also das, was ich unter der Bezeichnung Kettenbruch verstehe,
besser erläutert wird, biete ich weitreichende Beispiele für
rationalen und irrationalen Zahlen dar.
