%
% einleitung.tex -- Beispiel-File für die Einleitung
%
% (c) 2020 Benjamin Bouhafs-Keller
%
\section{Einleitung\label{kettenbruch:section:einleitung}}
\rhead{Einleitung}
Die Entwicklung der Theorie der Kettenbrüche ist durch das Bedürfnis,
Brüche oder schwer fassbare Zahlen zu approximieren, motiviert
worden. Kettenbrüche fanden Verwendung bei der Annäherung von
Verhältnisgrössen in Form von Brüchen, zur Ermittlung von Schaltjahren
bei der Kalenderberechnung, zur Annäherung natürlicher Konstanten
wie $e$ und $\pi$ und zum Beweis der Irrationalität bestimmter
Zahlen. Mit Hilfe von Kettenbruchsystemen können Approximationsalgorithmen 
für Funktionen entwickelt werden.
Einige der vielseitigen Aspekten der Kettenbrüche
werden wir in den folgenden Abschnitten erläutern.

\subsection{Definition\label{kettenbruch:section:1s}}
\begin{itemize}
\item
Ein {\em Kettenbruch} ist eine eindeutige Darstellungsform für reelle Zahlen. 
Ein allgemeiner Kettenbruch ist von der Form:
\begin{equation}
a_0 + \cfrac{b_1}{a_1+\cfrac{b_2}{a_2+\cfrac{\cdots}{\cdots+\cfrac{b_{n-1}}{a_{n-1} + \cfrac{b_n}{a_n}}}}},
\end{equation}
wobei $a_i, b_i \in \mathbb{Z}$.
\item
Ein {\em regulärer} oder {\em einfacher Kettenbruch} ist definiert als ein
Bruch der Form:
\index{Kettenbruch!regulär}%
\index{Kettenbruch!einfach}%
\begin{equation*}
[x_0;x_1,x_2,\cdots,x_n]
=
\frac{a}{b}=x_0+\cfrac{1}{x_1+\cfrac{1}{x_3+\cfrac{1}{x_4+\cfrac{1}{\cdots+\cfrac{1}{x_n}}}}}
\end{equation*}
mit $x_0,x_1,x_2,x_3, \dots \in \mathbb{N}$.
\end{itemize}

Eine alternative Schreibweise für reguläre Kettenbrüche ist
$[x_0;x_1,x_2,x_3,\dots]$.
Dies ist eine besonders kompakte Art, einen Kettenbruch zu beschreiben.
Bei allgemeinen Kettenbrüchen kann eine 
Darstellung mit eckigen Klammern nicht möglich sein, da im Zähler keine Einsen
mehr stehen.

%Bei Dezimalbrüchen würde dies unendlich gehen (Bspw. 0.333333\dots) da die Dezimalbruchentwicklung 
%offensichtlich nicht als Bruch mit einer Zehnerpotenz (10, 100, 1000 etc.) im Nenner  geschrieben werden kann.

Für unendliche Folge von $x_0,x_1,\dots$ kann der Wert des Kettenbruchs
durch den Grenzwert
\begin{align*}
\lim_{n\to\infty} [x_0;x_1,\dots,x_n] 
\end{align*}  
definiert werden.
Kettenbrüche können schnell konvergierende Approximationen liefern.

Damit also das, was ich unter der Bezeichnung Kettenbruch verstehe,
besser erläutert wird, biete ich Beispiele für
rationalen und irrationalen Zahlen dar.
