%
% einleitung.tex -- Beispiel-File für die Einleitung
%
% (c) 2020 Benjamin Bouhafs-Keller
%
\section{Einleitung\label{kettenbruch:section:einleitung}}
\rhead{Einleitung}
Die Entwicklung der Theorie der Kettenbrüche ist durch das Bedürfnis,
Brüche oder schwer fassbare Zahlen zu approximieren, motiviert
worden. Kettenbrüche fanden Verwendung bei der Annäherung von
Verhältnisgrössen in Form von Brüchen, zur Ermittlung von Schaltjahren
bei der Kalenderberechnung, zur Annäherung natürlicher Konstanten
wie $e$ und $\pi$ und zum Beweis der Irrationalität bestimmter
Zahlen.

\subsection{Definition\label{kettenbruch:section:1s}}
Ein Kettenbruch ist ein Bruch, dessen Nenner wieder als Kettenbruch
geschrieben ist.
Durch $n$-fache Wiederholung der genannten Injektionen entsteht
eine Projektive Abbildung.
Sie kann folgendermassern geschrieben werden

\begin{align*}
\frac{a}{b}=x_0+\cfrac{1}{x_1+\cfrac{1}{x_3+\cfrac{1}{x_4+\cdots}}}
\end{align*}
mit $x_0,x_1,x_2,x_3, \dots \in \mathbb{N}$


$[x_0;x_1;x_2;x_3;\dots]$ ist eine Kompakte Art irrationalen Zahlen
zu beschreiben. Bei Dezimalbrüchen würde dies unendlich gehen da
die Dezimalbruchentwicklung kein System ist.
In der Mathematik können Bruchsysteme auch weitere Darstellung haben. 
Hier ein Paar Beispiele:
\begin{align*}
\lim_{x\to\infty} [x_0;x_1;...;x_n]
\end{align*}  
oder
\begin{equation*}
	\begin{pmatrix}
		x_0&	1\\
		1  &	0
	\end{pmatrix}
	\begin{pmatrix}
		x_1&	1\\
		1  &	0
	\end{pmatrix}
	\cdots
	\begin{pmatrix}
		x_n&	1\\
		1  &	0
	\end{pmatrix}
	=\begin{pmatrix}
		p_n&	p_n-1\\
		q_n&	q_n-1
	\end{pmatrix}
\end{equation*}

Kettenbrüche können schnell konvergierende Approximationen liefern.
Damit also das, was ich unter der Bezeichnung Kettenbruch verstehe
besser erläutert wird, biete ich weitrenchende Beispiele für
Rationalen und Irrationalen Zahlen dar.
