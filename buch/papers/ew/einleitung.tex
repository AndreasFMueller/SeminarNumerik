%
% einleitung.tex -- Beispiel-File für die Einleitung
%
% (c) 2020 Prof Dr Andreas Müller, Hochschule Rapperswil
%
\section{Einleitung\label{ew:section:einleitung}}
\rhead{Einleitung}

Das Lösen des Eigenwertproblems ist  weit verbreitet mit vielen Anwendungsgebieten.
Umso wichtiger ist es, Eigenwerte und Eigenvektoren auch von grösseren Matrizen in kurzer Zeit zu berechnen.
Dazu gibt es diverse numerische Verfahren, die alle ihre Vor- und Nachteile haben.

Das klassische Verfahren für kleine Matrizen ist das Finden der Nullstellen des charakteristischen Polynoms.
Durch die Nullstellen können die Eigenwerte berechnet werden und anschliessend, durch Lösen eines Gleichungssystems auch die dazugehörigen Eigenvektoren.
Diese Methode, kann jedoch nur bis und mit Ordnung vier explizit gelöst werden, da alle Nullstellen nur bis zu dieser Ordnung explizit berechnet werden können.
Des Weiteren stellt sich heraus, dass der Weg über die Eigenwerte nummerisch impraktikabel für Floatingpoint berechnungen ist. 

Viel attraktiver ist das Anwenden von iterativen Methoden, welche auch für grössere Matrizen geeignet sind.
Einer der einfachsten Algorithmen ist die Potenzmethode.
Dabei wird ein zufällig gewählter Vektor mehrmals mit der Matrix multipliziert und normalisiert, bis er zu einem Eigenvektor konvergiert.
Diese Methode funktioniert generell für eine beliebige Matrix, doch sie liefert nur ein Eigenvektor auf einmal.

Falls man das Problem durch bestimmte Eigenschaften der Matrizen einschränken kann, können zum Teil schnellere Algorithmen angewendet werden.
Für reelle und symetrische Matrizen zum Beispiel, kann der Jacobi-Algorithmus angewendet werden. %TODO cite script???

Trotz vielen Methoden bleibt das Berechnen der Eigenwerte und Eigenvektoren sehr teuer.
Für manche Anwendungen genügen jedoch auch approximationen, wenn dadurch massenhaft Rechenleistung und dadurch auch Zeit gespart werden kann.
Die Störungstheorie, english \textit{perturbation}, befasst sich mit der approximation von Lösungen auf wenig abgeänderte, gestörte Probleme.
Auch für das Eigenwertproblem kann ein solches Verfahren entwickelt werden, welches angewendet werden kann, wenn Eigenwerte und Eigenvektoren von einer ähnlichen Matrix schon bekannt sind.
