%
% tschebasis.tex -- Basis-Bilder für Tschebyscheff Interpolationspolynom
%
% (c) 2020 Prof Dr Andreas Müller, Hochschule Rapperswil
%
\documentclass[tikz]{standalone}
\usepackage{amsmath}
\usepackage{times}
\usepackage{txfonts}
\usepackage{pgfplots}
\usepackage{csvsimple}
\usetikzlibrary{arrows,intersections,math}
\begin{document}
\def\skala{1.95}
\def\yskala{0.6}
\def\xskala{3}
\def\stuetz{
\draw ({-0.9749*\xskala},{-0.1/\skala})--({-0.9749*\xskala},{0.1/\skala});
\draw ({-0.7818*\xskala},{-0.1/\skala})--({-0.7818*\xskala},{0.1/\skala});
\draw ({-0.4339*\xskala},{-0.1/\skala})--({-0.4339*\xskala},{0.1/\skala});
\draw ({-0.0000*\xskala},{-0.1/\skala})--({-0.0000*\xskala},{0.1/\skala});
\draw ({0.4339*\xskala},{-0.1/\skala})--({0.4339*\xskala},{0.1/\skala});
\draw ({0.7818*\xskala},{-0.1/\skala})--({0.7818*\xskala},{0.1/\skala});
\draw ({0.9749*\xskala},{-0.1/\skala})--({0.9749*\xskala},{0.1/\skala});
}
\def\basiszero{
\draw[line width=1.4pt,color=red] ({-1.0000*\xskala},{1.2679*\yskala})
	--({-0.9900*\xskala},{1.1564*\yskala})
	--({-0.9801*\xskala},{1.0518*\yskala})
	--({-0.9701*\xskala},{0.9537*\yskala})
	--({-0.9602*\xskala},{0.8618*\yskala})
	--({-0.9502*\xskala},{0.7760*\yskala})
	--({-0.9403*\xskala},{0.6958*\yskala})
	--({-0.9303*\xskala},{0.6211*\yskala})
	--({-0.9204*\xskala},{0.5516*\yskala})
	--({-0.9104*\xskala},{0.4870*\yskala})
	--({-0.9005*\xskala},{0.4271*\yskala})
	--({-0.8905*\xskala},{0.3717*\yskala})
	--({-0.8806*\xskala},{0.3205*\yskala})
	--({-0.8706*\xskala},{0.2734*\yskala})
	--({-0.8607*\xskala},{0.2301*\yskala})
	--({-0.8507*\xskala},{0.1904*\yskala})
	--({-0.8408*\xskala},{0.1541*\yskala})
	--({-0.8308*\xskala},{0.1211*\yskala})
	--({-0.8209*\xskala},{0.0912*\yskala})
	--({-0.8109*\xskala},{0.0641*\yskala})
	--({-0.8010*\xskala},{0.0398*\yskala})
	--({-0.7910*\xskala},{0.0180*\yskala})
	--({-0.7811*\xskala},{-0.0014*\yskala})
	--({-0.7711*\xskala},{-0.0185*\yskala})
	--({-0.7612*\xskala},{-0.0335*\yskala})
	--({-0.7512*\xskala},{-0.0465*\yskala})
	--({-0.7413*\xskala},{-0.0577*\yskala})
	--({-0.7313*\xskala},{-0.0672*\yskala})
	--({-0.7214*\xskala},{-0.0751*\yskala})
	--({-0.7114*\xskala},{-0.0816*\yskala})
	--({-0.7015*\xskala},{-0.0866*\yskala})
	--({-0.6915*\xskala},{-0.0904*\yskala})
	--({-0.6816*\xskala},{-0.0931*\yskala})
	--({-0.6716*\xskala},{-0.0947*\yskala})
	--({-0.6617*\xskala},{-0.0954*\yskala})
	--({-0.6517*\xskala},{-0.0951*\yskala})
	--({-0.6418*\xskala},{-0.0941*\yskala})
	--({-0.6318*\xskala},{-0.0924*\yskala})
	--({-0.6219*\xskala},{-0.0900*\yskala})
	--({-0.6119*\xskala},{-0.0871*\yskala})
	--({-0.6020*\xskala},{-0.0837*\yskala})
	--({-0.5920*\xskala},{-0.0799*\yskala})
	--({-0.5821*\xskala},{-0.0756*\yskala})
	--({-0.5721*\xskala},{-0.0711*\yskala})
	--({-0.5622*\xskala},{-0.0663*\yskala})
	--({-0.5522*\xskala},{-0.0613*\yskala})
	--({-0.5423*\xskala},{-0.0562*\yskala})
	--({-0.5323*\xskala},{-0.0509*\yskala})
	--({-0.5224*\xskala},{-0.0456*\yskala})
	--({-0.5124*\xskala},{-0.0402*\yskala})
	--({-0.5025*\xskala},{-0.0348*\yskala})
	--({-0.4925*\xskala},{-0.0295*\yskala})
	--({-0.4826*\xskala},{-0.0242*\yskala})
	--({-0.4726*\xskala},{-0.0190*\yskala})
	--({-0.4627*\xskala},{-0.0139*\yskala})
	--({-0.4527*\xskala},{-0.0089*\yskala})
	--({-0.4428*\xskala},{-0.0041*\yskala})
	--({-0.4328*\xskala},{0.0005*\yskala})
	--({-0.4229*\xskala},{0.0049*\yskala})
	--({-0.4129*\xskala},{0.0091*\yskala})
	--({-0.4030*\xskala},{0.0131*\yskala})
	--({-0.3930*\xskala},{0.0169*\yskala})
	--({-0.3831*\xskala},{0.0204*\yskala})
	--({-0.3731*\xskala},{0.0237*\yskala})
	--({-0.3632*\xskala},{0.0267*\yskala})
	--({-0.3532*\xskala},{0.0295*\yskala})
	--({-0.3433*\xskala},{0.0320*\yskala})
	--({-0.3333*\xskala},{0.0342*\yskala})
	--({-0.3234*\xskala},{0.0362*\yskala})
	--({-0.3134*\xskala},{0.0379*\yskala})
	--({-0.3035*\xskala},{0.0394*\yskala})
	--({-0.2935*\xskala},{0.0406*\yskala})
	--({-0.2836*\xskala},{0.0416*\yskala})
	--({-0.2736*\xskala},{0.0423*\yskala})
	--({-0.2637*\xskala},{0.0427*\yskala})
	--({-0.2537*\xskala},{0.0430*\yskala})
	--({-0.2438*\xskala},{0.0430*\yskala})
	--({-0.2338*\xskala},{0.0428*\yskala})
	--({-0.2239*\xskala},{0.0423*\yskala})
	--({-0.2139*\xskala},{0.0417*\yskala})
	--({-0.2040*\xskala},{0.0409*\yskala})
	--({-0.1940*\xskala},{0.0399*\yskala})
	--({-0.1841*\xskala},{0.0387*\yskala})
	--({-0.1741*\xskala},{0.0373*\yskala})
	--({-0.1642*\xskala},{0.0359*\yskala})
	--({-0.1542*\xskala},{0.0342*\yskala})
	--({-0.1443*\xskala},{0.0325*\yskala})
	--({-0.1343*\xskala},{0.0306*\yskala})
	--({-0.1244*\xskala},{0.0286*\yskala})
	--({-0.1144*\xskala},{0.0266*\yskala})
	--({-0.1045*\xskala},{0.0244*\yskala})
	--({-0.0945*\xskala},{0.0222*\yskala})
	--({-0.0846*\xskala},{0.0199*\yskala})
	--({-0.0746*\xskala},{0.0176*\yskala})
	--({-0.0647*\xskala},{0.0153*\yskala})
	--({-0.0547*\xskala},{0.0129*\yskala})
	--({-0.0448*\xskala},{0.0105*\yskala})
	--({-0.0348*\xskala},{0.0082*\yskala})
	--({-0.0249*\xskala},{0.0058*\yskala})
	--({-0.0149*\xskala},{0.0035*\yskala})
	--({-0.0050*\xskala},{0.0011*\yskala})
	--({0.0050*\xskala},{-0.0011*\yskala})
	--({0.0149*\xskala},{-0.0033*\yskala})
	--({0.0249*\xskala},{-0.0055*\yskala})
	--({0.0348*\xskala},{-0.0076*\yskala})
	--({0.0448*\xskala},{-0.0096*\yskala})
	--({0.0547*\xskala},{-0.0115*\yskala})
	--({0.0647*\xskala},{-0.0134*\yskala})
	--({0.0746*\xskala},{-0.0151*\yskala})
	--({0.0846*\xskala},{-0.0168*\yskala})
	--({0.0945*\xskala},{-0.0183*\yskala})
	--({0.1045*\xskala},{-0.0197*\yskala})
	--({0.1144*\xskala},{-0.0210*\yskala})
	--({0.1244*\xskala},{-0.0222*\yskala})
	--({0.1343*\xskala},{-0.0232*\yskala})
	--({0.1443*\xskala},{-0.0241*\yskala})
	--({0.1542*\xskala},{-0.0249*\yskala})
	--({0.1642*\xskala},{-0.0255*\yskala})
	--({0.1741*\xskala},{-0.0260*\yskala})
	--({0.1841*\xskala},{-0.0264*\yskala})
	--({0.1940*\xskala},{-0.0266*\yskala})
	--({0.2040*\xskala},{-0.0267*\yskala})
	--({0.2139*\xskala},{-0.0267*\yskala})
	--({0.2239*\xskala},{-0.0265*\yskala})
	--({0.2338*\xskala},{-0.0262*\yskala})
	--({0.2438*\xskala},{-0.0258*\yskala})
	--({0.2537*\xskala},{-0.0252*\yskala})
	--({0.2637*\xskala},{-0.0245*\yskala})
	--({0.2736*\xskala},{-0.0237*\yskala})
	--({0.2836*\xskala},{-0.0228*\yskala})
	--({0.2935*\xskala},{-0.0218*\yskala})
	--({0.3035*\xskala},{-0.0207*\yskala})
	--({0.3134*\xskala},{-0.0195*\yskala})
	--({0.3234*\xskala},{-0.0182*\yskala})
	--({0.3333*\xskala},{-0.0168*\yskala})
	--({0.3433*\xskala},{-0.0153*\yskala})
	--({0.3532*\xskala},{-0.0138*\yskala})
	--({0.3632*\xskala},{-0.0122*\yskala})
	--({0.3731*\xskala},{-0.0106*\yskala})
	--({0.3831*\xskala},{-0.0089*\yskala})
	--({0.3930*\xskala},{-0.0072*\yskala})
	--({0.4030*\xskala},{-0.0054*\yskala})
	--({0.4129*\xskala},{-0.0037*\yskala})
	--({0.4229*\xskala},{-0.0019*\yskala})
	--({0.4328*\xskala},{-0.0002*\yskala})
	--({0.4428*\xskala},{0.0016*\yskala})
	--({0.4527*\xskala},{0.0033*\yskala})
	--({0.4627*\xskala},{0.0049*\yskala})
	--({0.4726*\xskala},{0.0066*\yskala})
	--({0.4826*\xskala},{0.0082*\yskala})
	--({0.4925*\xskala},{0.0097*\yskala})
	--({0.5025*\xskala},{0.0111*\yskala})
	--({0.5124*\xskala},{0.0125*\yskala})
	--({0.5224*\xskala},{0.0138*\yskala})
	--({0.5323*\xskala},{0.0150*\yskala})
	--({0.5423*\xskala},{0.0160*\yskala})
	--({0.5522*\xskala},{0.0170*\yskala})
	--({0.5622*\xskala},{0.0178*\yskala})
	--({0.5721*\xskala},{0.0185*\yskala})
	--({0.5821*\xskala},{0.0191*\yskala})
	--({0.5920*\xskala},{0.0195*\yskala})
	--({0.6020*\xskala},{0.0198*\yskala})
	--({0.6119*\xskala},{0.0199*\yskala})
	--({0.6219*\xskala},{0.0199*\yskala})
	--({0.6318*\xskala},{0.0197*\yskala})
	--({0.6418*\xskala},{0.0194*\yskala})
	--({0.6517*\xskala},{0.0189*\yskala})
	--({0.6617*\xskala},{0.0183*\yskala})
	--({0.6716*\xskala},{0.0174*\yskala})
	--({0.6816*\xskala},{0.0165*\yskala})
	--({0.6915*\xskala},{0.0154*\yskala})
	--({0.7015*\xskala},{0.0141*\yskala})
	--({0.7114*\xskala},{0.0127*\yskala})
	--({0.7214*\xskala},{0.0112*\yskala})
	--({0.7313*\xskala},{0.0096*\yskala})
	--({0.7413*\xskala},{0.0079*\yskala})
	--({0.7512*\xskala},{0.0060*\yskala})
	--({0.7612*\xskala},{0.0041*\yskala})
	--({0.7711*\xskala},{0.0022*\yskala})
	--({0.7811*\xskala},{0.0001*\yskala})
	--({0.7910*\xskala},{-0.0019*\yskala})
	--({0.8010*\xskala},{-0.0039*\yskala})
	--({0.8109*\xskala},{-0.0059*\yskala})
	--({0.8209*\xskala},{-0.0078*\yskala})
	--({0.8308*\xskala},{-0.0097*\yskala})
	--({0.8408*\xskala},{-0.0114*\yskala})
	--({0.8507*\xskala},{-0.0129*\yskala})
	--({0.8607*\xskala},{-0.0143*\yskala})
	--({0.8706*\xskala},{-0.0154*\yskala})
	--({0.8806*\xskala},{-0.0163*\yskala})
	--({0.8905*\xskala},{-0.0168*\yskala})
	--({0.9005*\xskala},{-0.0169*\yskala})
	--({0.9104*\xskala},{-0.0167*\yskala})
	--({0.9204*\xskala},{-0.0159*\yskala})
	--({0.9303*\xskala},{-0.0145*\yskala})
	--({0.9403*\xskala},{-0.0126*\yskala})
	--({0.9502*\xskala},{-0.0099*\yskala})
	--({0.9602*\xskala},{-0.0066*\yskala})
	--({0.9701*\xskala},{-0.0023*\yskala})
	--({0.9801*\xskala},{0.0028*\yskala})
	--({0.9900*\xskala},{0.0089*\yskala})
	--({1.0000*\xskala},{0.0161*\yskala});
}
\def\punktzero{
\fill[color=red] ({-0.9749*\xskala},\yskala) circle[radius={0.08/\skala}];
;
}
\def\basisone{
\draw[line width=1.4pt,color=red] ({-1.0000*\xskala},{-0.4083*\yskala})
	--({-0.9900*\xskala},{-0.2353*\yskala})
	--({-0.9801*\xskala},{-0.0769*\yskala})
	--({-0.9701*\xskala},{0.0678*\yskala})
	--({-0.9602*\xskala},{0.1994*\yskala})
	--({-0.9502*\xskala},{0.3186*\yskala})
	--({-0.9403*\xskala},{0.4260*\yskala})
	--({-0.9303*\xskala},{0.5224*\yskala})
	--({-0.9204*\xskala},{0.6082*\yskala})
	--({-0.9104*\xskala},{0.6841*\yskala})
	--({-0.9005*\xskala},{0.7506*\yskala})
	--({-0.8905*\xskala},{0.8083*\yskala})
	--({-0.8806*\xskala},{0.8577*\yskala})
	--({-0.8706*\xskala},{0.8993*\yskala})
	--({-0.8607*\xskala},{0.9337*\yskala})
	--({-0.8507*\xskala},{0.9613*\yskala})
	--({-0.8408*\xskala},{0.9824*\yskala})
	--({-0.8308*\xskala},{0.9977*\yskala})
	--({-0.8209*\xskala},{1.0075*\yskala})
	--({-0.8109*\xskala},{1.0121*\yskala})
	--({-0.8010*\xskala},{1.0121*\yskala})
	--({-0.7910*\xskala},{1.0076*\yskala})
	--({-0.7811*\xskala},{0.9992*\yskala})
	--({-0.7711*\xskala},{0.9872*\yskala})
	--({-0.7612*\xskala},{0.9718*\yskala})
	--({-0.7512*\xskala},{0.9534*\yskala})
	--({-0.7413*\xskala},{0.9323*\yskala})
	--({-0.7313*\xskala},{0.9088*\yskala})
	--({-0.7214*\xskala},{0.8831*\yskala})
	--({-0.7114*\xskala},{0.8554*\yskala})
	--({-0.7015*\xskala},{0.8261*\yskala})
	--({-0.6915*\xskala},{0.7954*\yskala})
	--({-0.6816*\xskala},{0.7634*\yskala})
	--({-0.6716*\xskala},{0.7305*\yskala})
	--({-0.6617*\xskala},{0.6967*\yskala})
	--({-0.6517*\xskala},{0.6622*\yskala})
	--({-0.6418*\xskala},{0.6273*\yskala})
	--({-0.6318*\xskala},{0.5922*\yskala})
	--({-0.6219*\xskala},{0.5568*\yskala})
	--({-0.6119*\xskala},{0.5215*\yskala})
	--({-0.6020*\xskala},{0.4863*\yskala})
	--({-0.5920*\xskala},{0.4514*\yskala})
	--({-0.5821*\xskala},{0.4168*\yskala})
	--({-0.5721*\xskala},{0.3827*\yskala})
	--({-0.5622*\xskala},{0.3492*\yskala})
	--({-0.5522*\xskala},{0.3164*\yskala})
	--({-0.5423*\xskala},{0.2843*\yskala})
	--({-0.5323*\xskala},{0.2531*\yskala})
	--({-0.5224*\xskala},{0.2227*\yskala})
	--({-0.5124*\xskala},{0.1933*\yskala})
	--({-0.5025*\xskala},{0.1650*\yskala})
	--({-0.4925*\xskala},{0.1376*\yskala})
	--({-0.4826*\xskala},{0.1114*\yskala})
	--({-0.4726*\xskala},{0.0863*\yskala})
	--({-0.4627*\xskala},{0.0624*\yskala})
	--({-0.4527*\xskala},{0.0397*\yskala})
	--({-0.4428*\xskala},{0.0182*\yskala})
	--({-0.4328*\xskala},{-0.0021*\yskala})
	--({-0.4229*\xskala},{-0.0211*\yskala})
	--({-0.4129*\xskala},{-0.0389*\yskala})
	--({-0.4030*\xskala},{-0.0555*\yskala})
	--({-0.3930*\xskala},{-0.0708*\yskala})
	--({-0.3831*\xskala},{-0.0849*\yskala})
	--({-0.3731*\xskala},{-0.0977*\yskala})
	--({-0.3632*\xskala},{-0.1094*\yskala})
	--({-0.3532*\xskala},{-0.1198*\yskala})
	--({-0.3433*\xskala},{-0.1291*\yskala})
	--({-0.3333*\xskala},{-0.1372*\yskala})
	--({-0.3234*\xskala},{-0.1442*\yskala})
	--({-0.3134*\xskala},{-0.1501*\yskala})
	--({-0.3035*\xskala},{-0.1550*\yskala})
	--({-0.2935*\xskala},{-0.1588*\yskala})
	--({-0.2836*\xskala},{-0.1616*\yskala})
	--({-0.2736*\xskala},{-0.1634*\yskala})
	--({-0.2637*\xskala},{-0.1644*\yskala})
	--({-0.2537*\xskala},{-0.1644*\yskala})
	--({-0.2438*\xskala},{-0.1636*\yskala})
	--({-0.2338*\xskala},{-0.1620*\yskala})
	--({-0.2239*\xskala},{-0.1596*\yskala})
	--({-0.2139*\xskala},{-0.1565*\yskala})
	--({-0.2040*\xskala},{-0.1528*\yskala})
	--({-0.1940*\xskala},{-0.1484*\yskala})
	--({-0.1841*\xskala},{-0.1434*\yskala})
	--({-0.1741*\xskala},{-0.1379*\yskala})
	--({-0.1642*\xskala},{-0.1319*\yskala})
	--({-0.1542*\xskala},{-0.1254*\yskala})
	--({-0.1443*\xskala},{-0.1186*\yskala})
	--({-0.1343*\xskala},{-0.1113*\yskala})
	--({-0.1244*\xskala},{-0.1038*\yskala})
	--({-0.1144*\xskala},{-0.0960*\yskala})
	--({-0.1045*\xskala},{-0.0880*\yskala})
	--({-0.0945*\xskala},{-0.0797*\yskala})
	--({-0.0846*\xskala},{-0.0714*\yskala})
	--({-0.0746*\xskala},{-0.0629*\yskala})
	--({-0.0647*\xskala},{-0.0544*\yskala})
	--({-0.0547*\xskala},{-0.0458*\yskala})
	--({-0.0448*\xskala},{-0.0373*\yskala})
	--({-0.0348*\xskala},{-0.0288*\yskala})
	--({-0.0249*\xskala},{-0.0204*\yskala})
	--({-0.0149*\xskala},{-0.0121*\yskala})
	--({-0.0050*\xskala},{-0.0040*\yskala})
	--({0.0050*\xskala},{0.0039*\yskala})
	--({0.0149*\xskala},{0.0117*\yskala})
	--({0.0249*\xskala},{0.0191*\yskala})
	--({0.0348*\xskala},{0.0263*\yskala})
	--({0.0448*\xskala},{0.0332*\yskala})
	--({0.0547*\xskala},{0.0398*\yskala})
	--({0.0647*\xskala},{0.0461*\yskala})
	--({0.0746*\xskala},{0.0519*\yskala})
	--({0.0846*\xskala},{0.0574*\yskala})
	--({0.0945*\xskala},{0.0625*\yskala})
	--({0.1045*\xskala},{0.0672*\yskala})
	--({0.1144*\xskala},{0.0715*\yskala})
	--({0.1244*\xskala},{0.0753*\yskala})
	--({0.1343*\xskala},{0.0787*\yskala})
	--({0.1443*\xskala},{0.0816*\yskala})
	--({0.1542*\xskala},{0.0841*\yskala})
	--({0.1642*\xskala},{0.0861*\yskala})
	--({0.1741*\xskala},{0.0877*\yskala})
	--({0.1841*\xskala},{0.0888*\yskala})
	--({0.1940*\xskala},{0.0894*\yskala})
	--({0.2040*\xskala},{0.0896*\yskala})
	--({0.2139*\xskala},{0.0893*\yskala})
	--({0.2239*\xskala},{0.0886*\yskala})
	--({0.2338*\xskala},{0.0874*\yskala})
	--({0.2438*\xskala},{0.0858*\yskala})
	--({0.2537*\xskala},{0.0838*\yskala})
	--({0.2637*\xskala},{0.0815*\yskala})
	--({0.2736*\xskala},{0.0787*\yskala})
	--({0.2836*\xskala},{0.0756*\yskala})
	--({0.2935*\xskala},{0.0721*\yskala})
	--({0.3035*\xskala},{0.0683*\yskala})
	--({0.3134*\xskala},{0.0642*\yskala})
	--({0.3234*\xskala},{0.0598*\yskala})
	--({0.3333*\xskala},{0.0552*\yskala})
	--({0.3433*\xskala},{0.0503*\yskala})
	--({0.3532*\xskala},{0.0452*\yskala})
	--({0.3632*\xskala},{0.0400*\yskala})
	--({0.3731*\xskala},{0.0346*\yskala})
	--({0.3831*\xskala},{0.0290*\yskala})
	--({0.3930*\xskala},{0.0234*\yskala})
	--({0.4030*\xskala},{0.0177*\yskala})
	--({0.4129*\xskala},{0.0120*\yskala})
	--({0.4229*\xskala},{0.0063*\yskala})
	--({0.4328*\xskala},{0.0006*\yskala})
	--({0.4428*\xskala},{-0.0050*\yskala})
	--({0.4527*\xskala},{-0.0106*\yskala})
	--({0.4627*\xskala},{-0.0160*\yskala})
	--({0.4726*\xskala},{-0.0213*\yskala})
	--({0.4826*\xskala},{-0.0264*\yskala})
	--({0.4925*\xskala},{-0.0312*\yskala})
	--({0.5025*\xskala},{-0.0359*\yskala})
	--({0.5124*\xskala},{-0.0402*\yskala})
	--({0.5224*\xskala},{-0.0443*\yskala})
	--({0.5323*\xskala},{-0.0480*\yskala})
	--({0.5423*\xskala},{-0.0514*\yskala})
	--({0.5522*\xskala},{-0.0545*\yskala})
	--({0.5622*\xskala},{-0.0571*\yskala})
	--({0.5721*\xskala},{-0.0593*\yskala})
	--({0.5821*\xskala},{-0.0610*\yskala})
	--({0.5920*\xskala},{-0.0624*\yskala})
	--({0.6020*\xskala},{-0.0632*\yskala})
	--({0.6119*\xskala},{-0.0636*\yskala})
	--({0.6219*\xskala},{-0.0634*\yskala})
	--({0.6318*\xskala},{-0.0628*\yskala})
	--({0.6418*\xskala},{-0.0617*\yskala})
	--({0.6517*\xskala},{-0.0601*\yskala})
	--({0.6617*\xskala},{-0.0580*\yskala})
	--({0.6716*\xskala},{-0.0554*\yskala})
	--({0.6816*\xskala},{-0.0523*\yskala})
	--({0.6915*\xskala},{-0.0487*\yskala})
	--({0.7015*\xskala},{-0.0447*\yskala})
	--({0.7114*\xskala},{-0.0403*\yskala})
	--({0.7214*\xskala},{-0.0355*\yskala})
	--({0.7313*\xskala},{-0.0303*\yskala})
	--({0.7413*\xskala},{-0.0248*\yskala})
	--({0.7512*\xskala},{-0.0190*\yskala})
	--({0.7612*\xskala},{-0.0130*\yskala})
	--({0.7711*\xskala},{-0.0068*\yskala})
	--({0.7811*\xskala},{-0.0005*\yskala})
	--({0.7910*\xskala},{0.0059*\yskala})
	--({0.8010*\xskala},{0.0123*\yskala})
	--({0.8109*\xskala},{0.0185*\yskala})
	--({0.8209*\xskala},{0.0246*\yskala})
	--({0.8308*\xskala},{0.0303*\yskala})
	--({0.8408*\xskala},{0.0357*\yskala})
	--({0.8507*\xskala},{0.0406*\yskala})
	--({0.8607*\xskala},{0.0448*\yskala})
	--({0.8706*\xskala},{0.0483*\yskala})
	--({0.8806*\xskala},{0.0510*\yskala})
	--({0.8905*\xskala},{0.0525*\yskala})
	--({0.9005*\xskala},{0.0529*\yskala})
	--({0.9104*\xskala},{0.0520*\yskala})
	--({0.9204*\xskala},{0.0495*\yskala})
	--({0.9303*\xskala},{0.0453*\yskala})
	--({0.9403*\xskala},{0.0392*\yskala})
	--({0.9502*\xskala},{0.0310*\yskala})
	--({0.9602*\xskala},{0.0204*\yskala})
	--({0.9701*\xskala},{0.0073*\yskala})
	--({0.9801*\xskala},{-0.0087*\yskala})
	--({0.9900*\xskala},{-0.0277*\yskala})
	--({1.0000*\xskala},{-0.0500*\yskala});
}
\def\punktone{
\fill[color=red] ({-0.7818*\xskala},\yskala) circle[radius={0.08/\skala}];
;
}
\def\basistwo{
\draw[line width=1.4pt,color=red] ({-1.0000*\xskala},{0.2274*\yskala})
	--({-0.9900*\xskala},{0.1273*\yskala})
	--({-0.9801*\xskala},{0.0403*\yskala})
	--({-0.9701*\xskala},{-0.0344*\yskala})
	--({-0.9602*\xskala},{-0.0977*\yskala})
	--({-0.9502*\xskala},{-0.1502*\yskala})
	--({-0.9403*\xskala},{-0.1927*\yskala})
	--({-0.9303*\xskala},{-0.2258*\yskala})
	--({-0.9204*\xskala},{-0.2503*\yskala})
	--({-0.9104*\xskala},{-0.2668*\yskala})
	--({-0.9005*\xskala},{-0.2758*\yskala})
	--({-0.8905*\xskala},{-0.2781*\yskala})
	--({-0.8806*\xskala},{-0.2740*\yskala})
	--({-0.8706*\xskala},{-0.2643*\yskala})
	--({-0.8607*\xskala},{-0.2493*\yskala})
	--({-0.8507*\xskala},{-0.2296*\yskala})
	--({-0.8408*\xskala},{-0.2057*\yskala})
	--({-0.8308*\xskala},{-0.1780*\yskala})
	--({-0.8209*\xskala},{-0.1469*\yskala})
	--({-0.8109*\xskala},{-0.1129*\yskala})
	--({-0.8010*\xskala},{-0.0763*\yskala})
	--({-0.7910*\xskala},{-0.0376*\yskala})
	--({-0.7811*\xskala},{0.0031*\yskala})
	--({-0.7711*\xskala},{0.0452*\yskala})
	--({-0.7612*\xskala},{0.0885*\yskala})
	--({-0.7512*\xskala},{0.1328*\yskala})
	--({-0.7413*\xskala},{0.1777*\yskala})
	--({-0.7313*\xskala},{0.2229*\yskala})
	--({-0.7214*\xskala},{0.2682*\yskala})
	--({-0.7114*\xskala},{0.3135*\yskala})
	--({-0.7015*\xskala},{0.3584*\yskala})
	--({-0.6915*\xskala},{0.4028*\yskala})
	--({-0.6816*\xskala},{0.4464*\yskala})
	--({-0.6716*\xskala},{0.4892*\yskala})
	--({-0.6617*\xskala},{0.5309*\yskala})
	--({-0.6517*\xskala},{0.5714*\yskala})
	--({-0.6418*\xskala},{0.6106*\yskala})
	--({-0.6318*\xskala},{0.6484*\yskala})
	--({-0.6219*\xskala},{0.6845*\yskala})
	--({-0.6119*\xskala},{0.7190*\yskala})
	--({-0.6020*\xskala},{0.7518*\yskala})
	--({-0.5920*\xskala},{0.7827*\yskala})
	--({-0.5821*\xskala},{0.8118*\yskala})
	--({-0.5721*\xskala},{0.8388*\yskala})
	--({-0.5622*\xskala},{0.8639*\yskala})
	--({-0.5522*\xskala},{0.8869*\yskala})
	--({-0.5423*\xskala},{0.9079*\yskala})
	--({-0.5323*\xskala},{0.9267*\yskala})
	--({-0.5224*\xskala},{0.9435*\yskala})
	--({-0.5124*\xskala},{0.9581*\yskala})
	--({-0.5025*\xskala},{0.9706*\yskala})
	--({-0.4925*\xskala},{0.9809*\yskala})
	--({-0.4826*\xskala},{0.9892*\yskala})
	--({-0.4726*\xskala},{0.9954*\yskala})
	--({-0.4627*\xskala},{0.9995*\yskala})
	--({-0.4527*\xskala},{1.0015*\yskala})
	--({-0.4428*\xskala},{1.0016*\yskala})
	--({-0.4328*\xskala},{0.9997*\yskala})
	--({-0.4229*\xskala},{0.9959*\yskala})
	--({-0.4129*\xskala},{0.9902*\yskala})
	--({-0.4030*\xskala},{0.9827*\yskala})
	--({-0.3930*\xskala},{0.9735*\yskala})
	--({-0.3831*\xskala},{0.9625*\yskala})
	--({-0.3731*\xskala},{0.9499*\yskala})
	--({-0.3632*\xskala},{0.9357*\yskala})
	--({-0.3532*\xskala},{0.9200*\yskala})
	--({-0.3433*\xskala},{0.9029*\yskala})
	--({-0.3333*\xskala},{0.8844*\yskala})
	--({-0.3234*\xskala},{0.8646*\yskala})
	--({-0.3134*\xskala},{0.8436*\yskala})
	--({-0.3035*\xskala},{0.8215*\yskala})
	--({-0.2935*\xskala},{0.7983*\yskala})
	--({-0.2836*\xskala},{0.7741*\yskala})
	--({-0.2736*\xskala},{0.7490*\yskala})
	--({-0.2637*\xskala},{0.7231*\yskala})
	--({-0.2537*\xskala},{0.6964*\yskala})
	--({-0.2438*\xskala},{0.6691*\yskala})
	--({-0.2338*\xskala},{0.6413*\yskala})
	--({-0.2239*\xskala},{0.6129*\yskala})
	--({-0.2139*\xskala},{0.5841*\yskala})
	--({-0.2040*\xskala},{0.5549*\yskala})
	--({-0.1940*\xskala},{0.5255*\yskala})
	--({-0.1841*\xskala},{0.4959*\yskala})
	--({-0.1741*\xskala},{0.4662*\yskala})
	--({-0.1642*\xskala},{0.4365*\yskala})
	--({-0.1542*\xskala},{0.4068*\yskala})
	--({-0.1443*\xskala},{0.3772*\yskala})
	--({-0.1343*\xskala},{0.3478*\yskala})
	--({-0.1244*\xskala},{0.3186*\yskala})
	--({-0.1144*\xskala},{0.2898*\yskala})
	--({-0.1045*\xskala},{0.2613*\yskala})
	--({-0.0945*\xskala},{0.2333*\yskala})
	--({-0.0846*\xskala},{0.2058*\yskala})
	--({-0.0746*\xskala},{0.1789*\yskala})
	--({-0.0647*\xskala},{0.1526*\yskala})
	--({-0.0547*\xskala},{0.1269*\yskala})
	--({-0.0448*\xskala},{0.1020*\yskala})
	--({-0.0348*\xskala},{0.0779*\yskala})
	--({-0.0249*\xskala},{0.0545*\yskala})
	--({-0.0149*\xskala},{0.0320*\yskala})
	--({-0.0050*\xskala},{0.0104*\yskala})
	--({0.0050*\xskala},{-0.0102*\yskala})
	--({0.0149*\xskala},{-0.0299*\yskala})
	--({0.0249*\xskala},{-0.0486*\yskala})
	--({0.0348*\xskala},{-0.0663*\yskala})
	--({0.0448*\xskala},{-0.0829*\yskala})
	--({0.0547*\xskala},{-0.0985*\yskala})
	--({0.0647*\xskala},{-0.1130*\yskala})
	--({0.0746*\xskala},{-0.1264*\yskala})
	--({0.0846*\xskala},{-0.1387*\yskala})
	--({0.0945*\xskala},{-0.1499*\yskala})
	--({0.1045*\xskala},{-0.1599*\yskala})
	--({0.1144*\xskala},{-0.1688*\yskala})
	--({0.1244*\xskala},{-0.1767*\yskala})
	--({0.1343*\xskala},{-0.1833*\yskala})
	--({0.1443*\xskala},{-0.1889*\yskala})
	--({0.1542*\xskala},{-0.1934*\yskala})
	--({0.1642*\xskala},{-0.1968*\yskala})
	--({0.1741*\xskala},{-0.1992*\yskala})
	--({0.1841*\xskala},{-0.2005*\yskala})
	--({0.1940*\xskala},{-0.2007*\yskala})
	--({0.2040*\xskala},{-0.2000*\yskala})
	--({0.2139*\xskala},{-0.1983*\yskala})
	--({0.2239*\xskala},{-0.1957*\yskala})
	--({0.2338*\xskala},{-0.1921*\yskala})
	--({0.2438*\xskala},{-0.1877*\yskala})
	--({0.2537*\xskala},{-0.1825*\yskala})
	--({0.2637*\xskala},{-0.1764*\yskala})
	--({0.2736*\xskala},{-0.1696*\yskala})
	--({0.2836*\xskala},{-0.1622*\yskala})
	--({0.2935*\xskala},{-0.1540*\yskala})
	--({0.3035*\xskala},{-0.1453*\yskala})
	--({0.3134*\xskala},{-0.1360*\yskala})
	--({0.3234*\xskala},{-0.1262*\yskala})
	--({0.3333*\xskala},{-0.1159*\yskala})
	--({0.3433*\xskala},{-0.1053*\yskala})
	--({0.3532*\xskala},{-0.0943*\yskala})
	--({0.3632*\xskala},{-0.0830*\yskala})
	--({0.3731*\xskala},{-0.0715*\yskala})
	--({0.3831*\xskala},{-0.0598*\yskala})
	--({0.3930*\xskala},{-0.0481*\yskala})
	--({0.4030*\xskala},{-0.0363*\yskala})
	--({0.4129*\xskala},{-0.0245*\yskala})
	--({0.4229*\xskala},{-0.0128*\yskala})
	--({0.4328*\xskala},{-0.0012*\yskala})
	--({0.4428*\xskala},{0.0102*\yskala})
	--({0.4527*\xskala},{0.0213*\yskala})
	--({0.4627*\xskala},{0.0321*\yskala})
	--({0.4726*\xskala},{0.0426*\yskala})
	--({0.4826*\xskala},{0.0526*\yskala})
	--({0.4925*\xskala},{0.0621*\yskala})
	--({0.5025*\xskala},{0.0711*\yskala})
	--({0.5124*\xskala},{0.0795*\yskala})
	--({0.5224*\xskala},{0.0873*\yskala})
	--({0.5323*\xskala},{0.0944*\yskala})
	--({0.5423*\xskala},{0.1008*\yskala})
	--({0.5522*\xskala},{0.1065*\yskala})
	--({0.5622*\xskala},{0.1113*\yskala})
	--({0.5721*\xskala},{0.1153*\yskala})
	--({0.5821*\xskala},{0.1184*\yskala})
	--({0.5920*\xskala},{0.1207*\yskala})
	--({0.6020*\xskala},{0.1220*\yskala})
	--({0.6119*\xskala},{0.1224*\yskala})
	--({0.6219*\xskala},{0.1219*\yskala})
	--({0.6318*\xskala},{0.1204*\yskala})
	--({0.6418*\xskala},{0.1180*\yskala})
	--({0.6517*\xskala},{0.1147*\yskala})
	--({0.6617*\xskala},{0.1104*\yskala})
	--({0.6716*\xskala},{0.1052*\yskala})
	--({0.6816*\xskala},{0.0991*\yskala})
	--({0.6915*\xskala},{0.0922*\yskala})
	--({0.7015*\xskala},{0.0845*\yskala})
	--({0.7114*\xskala},{0.0760*\yskala})
	--({0.7214*\xskala},{0.0668*\yskala})
	--({0.7313*\xskala},{0.0569*\yskala})
	--({0.7413*\xskala},{0.0465*\yskala})
	--({0.7512*\xskala},{0.0356*\yskala})
	--({0.7612*\xskala},{0.0243*\yskala})
	--({0.7711*\xskala},{0.0127*\yskala})
	--({0.7811*\xskala},{0.0009*\yskala})
	--({0.7910*\xskala},{-0.0110*\yskala})
	--({0.8010*\xskala},{-0.0227*\yskala})
	--({0.8109*\xskala},{-0.0342*\yskala})
	--({0.8209*\xskala},{-0.0453*\yskala})
	--({0.8308*\xskala},{-0.0559*\yskala})
	--({0.8408*\xskala},{-0.0657*\yskala})
	--({0.8507*\xskala},{-0.0745*\yskala})
	--({0.8607*\xskala},{-0.0822*\yskala})
	--({0.8706*\xskala},{-0.0885*\yskala})
	--({0.8806*\xskala},{-0.0931*\yskala})
	--({0.8905*\xskala},{-0.0959*\yskala})
	--({0.9005*\xskala},{-0.0965*\yskala})
	--({0.9104*\xskala},{-0.0946*\yskala})
	--({0.9204*\xskala},{-0.0899*\yskala})
	--({0.9303*\xskala},{-0.0822*\yskala})
	--({0.9403*\xskala},{-0.0710*\yskala})
	--({0.9502*\xskala},{-0.0560*\yskala})
	--({0.9602*\xskala},{-0.0369*\yskala})
	--({0.9701*\xskala},{-0.0131*\yskala})
	--({0.9801*\xskala},{0.0156*\yskala})
	--({0.9900*\xskala},{0.0497*\yskala})
	--({1.0000*\xskala},{0.0898*\yskala});
}
\def\punkttwo{
\fill[color=red] ({-0.4339*\xskala},\yskala) circle[radius={0.08/\skala}];
;
}
\def\basisthree{
\draw[line width=1.4pt,color=red] ({-1.0000*\xskala},{-0.1429*\yskala})
	--({-0.9900*\xskala},{-0.0794*\yskala})
	--({-0.9801*\xskala},{-0.0249*\yskala})
	--({-0.9701*\xskala},{0.0211*\yskala})
	--({-0.9602*\xskala},{0.0594*\yskala})
	--({-0.9502*\xskala},{0.0906*\yskala})
	--({-0.9403*\xskala},{0.1152*\yskala})
	--({-0.9303*\xskala},{0.1337*\yskala})
	--({-0.9204*\xskala},{0.1469*\yskala})
	--({-0.9104*\xskala},{0.1550*\yskala})
	--({-0.9005*\xskala},{0.1586*\yskala})
	--({-0.8905*\xskala},{0.1583*\yskala})
	--({-0.8806*\xskala},{0.1543*\yskala})
	--({-0.8706*\xskala},{0.1471*\yskala})
	--({-0.8607*\xskala},{0.1372*\yskala})
	--({-0.8507*\xskala},{0.1249*\yskala})
	--({-0.8408*\xskala},{0.1105*\yskala})
	--({-0.8308*\xskala},{0.0944*\yskala})
	--({-0.8209*\xskala},{0.0769*\yskala})
	--({-0.8109*\xskala},{0.0583*\yskala})
	--({-0.8010*\xskala},{0.0388*\yskala})
	--({-0.7910*\xskala},{0.0188*\yskala})
	--({-0.7811*\xskala},{-0.0015*\yskala})
	--({-0.7711*\xskala},{-0.0219*\yskala})
	--({-0.7612*\xskala},{-0.0423*\yskala})
	--({-0.7512*\xskala},{-0.0623*\yskala})
	--({-0.7413*\xskala},{-0.0818*\yskala})
	--({-0.7313*\xskala},{-0.1006*\yskala})
	--({-0.7214*\xskala},{-0.1187*\yskala})
	--({-0.7114*\xskala},{-0.1357*\yskala})
	--({-0.7015*\xskala},{-0.1517*\yskala})
	--({-0.6915*\xskala},{-0.1666*\yskala})
	--({-0.6816*\xskala},{-0.1801*\yskala})
	--({-0.6716*\xskala},{-0.1922*\yskala})
	--({-0.6617*\xskala},{-0.2029*\yskala})
	--({-0.6517*\xskala},{-0.2120*\yskala})
	--({-0.6418*\xskala},{-0.2196*\yskala})
	--({-0.6318*\xskala},{-0.2255*\yskala})
	--({-0.6219*\xskala},{-0.2297*\yskala})
	--({-0.6119*\xskala},{-0.2322*\yskala})
	--({-0.6020*\xskala},{-0.2330*\yskala})
	--({-0.5920*\xskala},{-0.2321*\yskala})
	--({-0.5821*\xskala},{-0.2294*\yskala})
	--({-0.5721*\xskala},{-0.2250*\yskala})
	--({-0.5622*\xskala},{-0.2188*\yskala})
	--({-0.5522*\xskala},{-0.2110*\yskala})
	--({-0.5423*\xskala},{-0.2014*\yskala})
	--({-0.5323*\xskala},{-0.1902*\yskala})
	--({-0.5224*\xskala},{-0.1774*\yskala})
	--({-0.5124*\xskala},{-0.1630*\yskala})
	--({-0.5025*\xskala},{-0.1471*\yskala})
	--({-0.4925*\xskala},{-0.1297*\yskala})
	--({-0.4826*\xskala},{-0.1108*\yskala})
	--({-0.4726*\xskala},{-0.0906*\yskala})
	--({-0.4627*\xskala},{-0.0691*\yskala})
	--({-0.4527*\xskala},{-0.0463*\yskala})
	--({-0.4428*\xskala},{-0.0224*\yskala})
	--({-0.4328*\xskala},{0.0027*\yskala})
	--({-0.4229*\xskala},{0.0287*\yskala})
	--({-0.4129*\xskala},{0.0558*\yskala})
	--({-0.4030*\xskala},{0.0836*\yskala})
	--({-0.3930*\xskala},{0.1123*\yskala})
	--({-0.3831*\xskala},{0.1417*\yskala})
	--({-0.3731*\xskala},{0.1716*\yskala})
	--({-0.3632*\xskala},{0.2022*\yskala})
	--({-0.3532*\xskala},{0.2331*\yskala})
	--({-0.3433*\xskala},{0.2645*\yskala})
	--({-0.3333*\xskala},{0.2961*\yskala})
	--({-0.3234*\xskala},{0.3279*\yskala})
	--({-0.3134*\xskala},{0.3598*\yskala})
	--({-0.3035*\xskala},{0.3918*\yskala})
	--({-0.2935*\xskala},{0.4236*\yskala})
	--({-0.2836*\xskala},{0.4554*\yskala})
	--({-0.2736*\xskala},{0.4869*\yskala})
	--({-0.2637*\xskala},{0.5180*\yskala})
	--({-0.2537*\xskala},{0.5488*\yskala})
	--({-0.2438*\xskala},{0.5792*\yskala})
	--({-0.2338*\xskala},{0.6089*\yskala})
	--({-0.2239*\xskala},{0.6381*\yskala})
	--({-0.2139*\xskala},{0.6665*\yskala})
	--({-0.2040*\xskala},{0.6942*\yskala})
	--({-0.1940*\xskala},{0.7210*\yskala})
	--({-0.1841*\xskala},{0.7469*\yskala})
	--({-0.1741*\xskala},{0.7719*\yskala})
	--({-0.1642*\xskala},{0.7958*\yskala})
	--({-0.1542*\xskala},{0.8186*\yskala})
	--({-0.1443*\xskala},{0.8403*\yskala})
	--({-0.1343*\xskala},{0.8608*\yskala})
	--({-0.1244*\xskala},{0.8800*\yskala})
	--({-0.1144*\xskala},{0.8980*\yskala})
	--({-0.1045*\xskala},{0.9146*\yskala})
	--({-0.0945*\xskala},{0.9298*\yskala})
	--({-0.0846*\xskala},{0.9436*\yskala})
	--({-0.0746*\xskala},{0.9559*\yskala})
	--({-0.0647*\xskala},{0.9668*\yskala})
	--({-0.0547*\xskala},{0.9762*\yskala})
	--({-0.0448*\xskala},{0.9840*\yskala})
	--({-0.0348*\xskala},{0.9903*\yskala})
	--({-0.0249*\xskala},{0.9951*\yskala})
	--({-0.0149*\xskala},{0.9982*\yskala})
	--({-0.0050*\xskala},{0.9998*\yskala})
	--({0.0050*\xskala},{0.9998*\yskala})
	--({0.0149*\xskala},{0.9982*\yskala})
	--({0.0249*\xskala},{0.9951*\yskala})
	--({0.0348*\xskala},{0.9903*\yskala})
	--({0.0448*\xskala},{0.9840*\yskala})
	--({0.0547*\xskala},{0.9762*\yskala})
	--({0.0647*\xskala},{0.9668*\yskala})
	--({0.0746*\xskala},{0.9559*\yskala})
	--({0.0846*\xskala},{0.9436*\yskala})
	--({0.0945*\xskala},{0.9298*\yskala})
	--({0.1045*\xskala},{0.9146*\yskala})
	--({0.1144*\xskala},{0.8980*\yskala})
	--({0.1244*\xskala},{0.8800*\yskala})
	--({0.1343*\xskala},{0.8608*\yskala})
	--({0.1443*\xskala},{0.8403*\yskala})
	--({0.1542*\xskala},{0.8186*\yskala})
	--({0.1642*\xskala},{0.7958*\yskala})
	--({0.1741*\xskala},{0.7719*\yskala})
	--({0.1841*\xskala},{0.7469*\yskala})
	--({0.1940*\xskala},{0.7210*\yskala})
	--({0.2040*\xskala},{0.6942*\yskala})
	--({0.2139*\xskala},{0.6665*\yskala})
	--({0.2239*\xskala},{0.6381*\yskala})
	--({0.2338*\xskala},{0.6089*\yskala})
	--({0.2438*\xskala},{0.5792*\yskala})
	--({0.2537*\xskala},{0.5488*\yskala})
	--({0.2637*\xskala},{0.5180*\yskala})
	--({0.2736*\xskala},{0.4869*\yskala})
	--({0.2836*\xskala},{0.4554*\yskala})
	--({0.2935*\xskala},{0.4236*\yskala})
	--({0.3035*\xskala},{0.3918*\yskala})
	--({0.3134*\xskala},{0.3598*\yskala})
	--({0.3234*\xskala},{0.3279*\yskala})
	--({0.3333*\xskala},{0.2961*\yskala})
	--({0.3433*\xskala},{0.2645*\yskala})
	--({0.3532*\xskala},{0.2331*\yskala})
	--({0.3632*\xskala},{0.2022*\yskala})
	--({0.3731*\xskala},{0.1716*\yskala})
	--({0.3831*\xskala},{0.1417*\yskala})
	--({0.3930*\xskala},{0.1123*\yskala})
	--({0.4030*\xskala},{0.0836*\yskala})
	--({0.4129*\xskala},{0.0558*\yskala})
	--({0.4229*\xskala},{0.0287*\yskala})
	--({0.4328*\xskala},{0.0027*\yskala})
	--({0.4428*\xskala},{-0.0224*\yskala})
	--({0.4527*\xskala},{-0.0463*\yskala})
	--({0.4627*\xskala},{-0.0691*\yskala})
	--({0.4726*\xskala},{-0.0906*\yskala})
	--({0.4826*\xskala},{-0.1108*\yskala})
	--({0.4925*\xskala},{-0.1297*\yskala})
	--({0.5025*\xskala},{-0.1471*\yskala})
	--({0.5124*\xskala},{-0.1630*\yskala})
	--({0.5224*\xskala},{-0.1774*\yskala})
	--({0.5323*\xskala},{-0.1902*\yskala})
	--({0.5423*\xskala},{-0.2014*\yskala})
	--({0.5522*\xskala},{-0.2110*\yskala})
	--({0.5622*\xskala},{-0.2188*\yskala})
	--({0.5721*\xskala},{-0.2250*\yskala})
	--({0.5821*\xskala},{-0.2294*\yskala})
	--({0.5920*\xskala},{-0.2321*\yskala})
	--({0.6020*\xskala},{-0.2330*\yskala})
	--({0.6119*\xskala},{-0.2322*\yskala})
	--({0.6219*\xskala},{-0.2297*\yskala})
	--({0.6318*\xskala},{-0.2255*\yskala})
	--({0.6418*\xskala},{-0.2196*\yskala})
	--({0.6517*\xskala},{-0.2120*\yskala})
	--({0.6617*\xskala},{-0.2029*\yskala})
	--({0.6716*\xskala},{-0.1922*\yskala})
	--({0.6816*\xskala},{-0.1801*\yskala})
	--({0.6915*\xskala},{-0.1666*\yskala})
	--({0.7015*\xskala},{-0.1517*\yskala})
	--({0.7114*\xskala},{-0.1357*\yskala})
	--({0.7214*\xskala},{-0.1187*\yskala})
	--({0.7313*\xskala},{-0.1006*\yskala})
	--({0.7413*\xskala},{-0.0818*\yskala})
	--({0.7512*\xskala},{-0.0623*\yskala})
	--({0.7612*\xskala},{-0.0423*\yskala})
	--({0.7711*\xskala},{-0.0219*\yskala})
	--({0.7811*\xskala},{-0.0015*\yskala})
	--({0.7910*\xskala},{0.0188*\yskala})
	--({0.8010*\xskala},{0.0388*\yskala})
	--({0.8109*\xskala},{0.0583*\yskala})
	--({0.8209*\xskala},{0.0769*\yskala})
	--({0.8308*\xskala},{0.0944*\yskala})
	--({0.8408*\xskala},{0.1105*\yskala})
	--({0.8507*\xskala},{0.1249*\yskala})
	--({0.8607*\xskala},{0.1372*\yskala})
	--({0.8706*\xskala},{0.1471*\yskala})
	--({0.8806*\xskala},{0.1543*\yskala})
	--({0.8905*\xskala},{0.1583*\yskala})
	--({0.9005*\xskala},{0.1586*\yskala})
	--({0.9104*\xskala},{0.1550*\yskala})
	--({0.9204*\xskala},{0.1469*\yskala})
	--({0.9303*\xskala},{0.1337*\yskala})
	--({0.9403*\xskala},{0.1152*\yskala})
	--({0.9502*\xskala},{0.0906*\yskala})
	--({0.9602*\xskala},{0.0594*\yskala})
	--({0.9701*\xskala},{0.0211*\yskala})
	--({0.9801*\xskala},{-0.0249*\yskala})
	--({0.9900*\xskala},{-0.0794*\yskala})
	--({1.0000*\xskala},{-0.1429*\yskala});
}
\def\punktthree{
\fill[color=red] ({-0.0000*\xskala},\yskala) circle[radius={0.08/\skala}];
;
}
\def\basisfour{
\draw[line width=1.4pt,color=red] ({-1.0000*\xskala},{0.0898*\yskala})
	--({-0.9900*\xskala},{0.0497*\yskala})
	--({-0.9801*\xskala},{0.0156*\yskala})
	--({-0.9701*\xskala},{-0.0131*\yskala})
	--({-0.9602*\xskala},{-0.0369*\yskala})
	--({-0.9502*\xskala},{-0.0560*\yskala})
	--({-0.9403*\xskala},{-0.0710*\yskala})
	--({-0.9303*\xskala},{-0.0822*\yskala})
	--({-0.9204*\xskala},{-0.0899*\yskala})
	--({-0.9104*\xskala},{-0.0946*\yskala})
	--({-0.9005*\xskala},{-0.0965*\yskala})
	--({-0.8905*\xskala},{-0.0959*\yskala})
	--({-0.8806*\xskala},{-0.0931*\yskala})
	--({-0.8706*\xskala},{-0.0885*\yskala})
	--({-0.8607*\xskala},{-0.0822*\yskala})
	--({-0.8507*\xskala},{-0.0745*\yskala})
	--({-0.8408*\xskala},{-0.0657*\yskala})
	--({-0.8308*\xskala},{-0.0559*\yskala})
	--({-0.8209*\xskala},{-0.0453*\yskala})
	--({-0.8109*\xskala},{-0.0342*\yskala})
	--({-0.8010*\xskala},{-0.0227*\yskala})
	--({-0.7910*\xskala},{-0.0110*\yskala})
	--({-0.7811*\xskala},{0.0009*\yskala})
	--({-0.7711*\xskala},{0.0127*\yskala})
	--({-0.7612*\xskala},{0.0243*\yskala})
	--({-0.7512*\xskala},{0.0356*\yskala})
	--({-0.7413*\xskala},{0.0465*\yskala})
	--({-0.7313*\xskala},{0.0569*\yskala})
	--({-0.7214*\xskala},{0.0668*\yskala})
	--({-0.7114*\xskala},{0.0760*\yskala})
	--({-0.7015*\xskala},{0.0845*\yskala})
	--({-0.6915*\xskala},{0.0922*\yskala})
	--({-0.6816*\xskala},{0.0991*\yskala})
	--({-0.6716*\xskala},{0.1052*\yskala})
	--({-0.6617*\xskala},{0.1104*\yskala})
	--({-0.6517*\xskala},{0.1147*\yskala})
	--({-0.6418*\xskala},{0.1180*\yskala})
	--({-0.6318*\xskala},{0.1204*\yskala})
	--({-0.6219*\xskala},{0.1219*\yskala})
	--({-0.6119*\xskala},{0.1224*\yskala})
	--({-0.6020*\xskala},{0.1220*\yskala})
	--({-0.5920*\xskala},{0.1207*\yskala})
	--({-0.5821*\xskala},{0.1184*\yskala})
	--({-0.5721*\xskala},{0.1153*\yskala})
	--({-0.5622*\xskala},{0.1113*\yskala})
	--({-0.5522*\xskala},{0.1065*\yskala})
	--({-0.5423*\xskala},{0.1008*\yskala})
	--({-0.5323*\xskala},{0.0944*\yskala})
	--({-0.5224*\xskala},{0.0873*\yskala})
	--({-0.5124*\xskala},{0.0795*\yskala})
	--({-0.5025*\xskala},{0.0711*\yskala})
	--({-0.4925*\xskala},{0.0621*\yskala})
	--({-0.4826*\xskala},{0.0526*\yskala})
	--({-0.4726*\xskala},{0.0426*\yskala})
	--({-0.4627*\xskala},{0.0321*\yskala})
	--({-0.4527*\xskala},{0.0213*\yskala})
	--({-0.4428*\xskala},{0.0102*\yskala})
	--({-0.4328*\xskala},{-0.0012*\yskala})
	--({-0.4229*\xskala},{-0.0128*\yskala})
	--({-0.4129*\xskala},{-0.0245*\yskala})
	--({-0.4030*\xskala},{-0.0363*\yskala})
	--({-0.3930*\xskala},{-0.0481*\yskala})
	--({-0.3831*\xskala},{-0.0598*\yskala})
	--({-0.3731*\xskala},{-0.0715*\yskala})
	--({-0.3632*\xskala},{-0.0830*\yskala})
	--({-0.3532*\xskala},{-0.0943*\yskala})
	--({-0.3433*\xskala},{-0.1053*\yskala})
	--({-0.3333*\xskala},{-0.1159*\yskala})
	--({-0.3234*\xskala},{-0.1262*\yskala})
	--({-0.3134*\xskala},{-0.1360*\yskala})
	--({-0.3035*\xskala},{-0.1453*\yskala})
	--({-0.2935*\xskala},{-0.1540*\yskala})
	--({-0.2836*\xskala},{-0.1622*\yskala})
	--({-0.2736*\xskala},{-0.1696*\yskala})
	--({-0.2637*\xskala},{-0.1764*\yskala})
	--({-0.2537*\xskala},{-0.1825*\yskala})
	--({-0.2438*\xskala},{-0.1877*\yskala})
	--({-0.2338*\xskala},{-0.1921*\yskala})
	--({-0.2239*\xskala},{-0.1957*\yskala})
	--({-0.2139*\xskala},{-0.1983*\yskala})
	--({-0.2040*\xskala},{-0.2000*\yskala})
	--({-0.1940*\xskala},{-0.2007*\yskala})
	--({-0.1841*\xskala},{-0.2005*\yskala})
	--({-0.1741*\xskala},{-0.1992*\yskala})
	--({-0.1642*\xskala},{-0.1968*\yskala})
	--({-0.1542*\xskala},{-0.1934*\yskala})
	--({-0.1443*\xskala},{-0.1889*\yskala})
	--({-0.1343*\xskala},{-0.1833*\yskala})
	--({-0.1244*\xskala},{-0.1767*\yskala})
	--({-0.1144*\xskala},{-0.1688*\yskala})
	--({-0.1045*\xskala},{-0.1599*\yskala})
	--({-0.0945*\xskala},{-0.1499*\yskala})
	--({-0.0846*\xskala},{-0.1387*\yskala})
	--({-0.0746*\xskala},{-0.1264*\yskala})
	--({-0.0647*\xskala},{-0.1130*\yskala})
	--({-0.0547*\xskala},{-0.0985*\yskala})
	--({-0.0448*\xskala},{-0.0829*\yskala})
	--({-0.0348*\xskala},{-0.0663*\yskala})
	--({-0.0249*\xskala},{-0.0486*\yskala})
	--({-0.0149*\xskala},{-0.0299*\yskala})
	--({-0.0050*\xskala},{-0.0102*\yskala})
	--({0.0050*\xskala},{0.0104*\yskala})
	--({0.0149*\xskala},{0.0320*\yskala})
	--({0.0249*\xskala},{0.0545*\yskala})
	--({0.0348*\xskala},{0.0779*\yskala})
	--({0.0448*\xskala},{0.1020*\yskala})
	--({0.0547*\xskala},{0.1269*\yskala})
	--({0.0647*\xskala},{0.1526*\yskala})
	--({0.0746*\xskala},{0.1789*\yskala})
	--({0.0846*\xskala},{0.2058*\yskala})
	--({0.0945*\xskala},{0.2333*\yskala})
	--({0.1045*\xskala},{0.2613*\yskala})
	--({0.1144*\xskala},{0.2898*\yskala})
	--({0.1244*\xskala},{0.3186*\yskala})
	--({0.1343*\xskala},{0.3478*\yskala})
	--({0.1443*\xskala},{0.3772*\yskala})
	--({0.1542*\xskala},{0.4068*\yskala})
	--({0.1642*\xskala},{0.4365*\yskala})
	--({0.1741*\xskala},{0.4662*\yskala})
	--({0.1841*\xskala},{0.4959*\yskala})
	--({0.1940*\xskala},{0.5255*\yskala})
	--({0.2040*\xskala},{0.5549*\yskala})
	--({0.2139*\xskala},{0.5841*\yskala})
	--({0.2239*\xskala},{0.6129*\yskala})
	--({0.2338*\xskala},{0.6413*\yskala})
	--({0.2438*\xskala},{0.6691*\yskala})
	--({0.2537*\xskala},{0.6964*\yskala})
	--({0.2637*\xskala},{0.7231*\yskala})
	--({0.2736*\xskala},{0.7490*\yskala})
	--({0.2836*\xskala},{0.7741*\yskala})
	--({0.2935*\xskala},{0.7983*\yskala})
	--({0.3035*\xskala},{0.8215*\yskala})
	--({0.3134*\xskala},{0.8436*\yskala})
	--({0.3234*\xskala},{0.8646*\yskala})
	--({0.3333*\xskala},{0.8844*\yskala})
	--({0.3433*\xskala},{0.9029*\yskala})
	--({0.3532*\xskala},{0.9200*\yskala})
	--({0.3632*\xskala},{0.9357*\yskala})
	--({0.3731*\xskala},{0.9499*\yskala})
	--({0.3831*\xskala},{0.9625*\yskala})
	--({0.3930*\xskala},{0.9735*\yskala})
	--({0.4030*\xskala},{0.9827*\yskala})
	--({0.4129*\xskala},{0.9902*\yskala})
	--({0.4229*\xskala},{0.9959*\yskala})
	--({0.4328*\xskala},{0.9997*\yskala})
	--({0.4428*\xskala},{1.0016*\yskala})
	--({0.4527*\xskala},{1.0015*\yskala})
	--({0.4627*\xskala},{0.9995*\yskala})
	--({0.4726*\xskala},{0.9954*\yskala})
	--({0.4826*\xskala},{0.9892*\yskala})
	--({0.4925*\xskala},{0.9809*\yskala})
	--({0.5025*\xskala},{0.9706*\yskala})
	--({0.5124*\xskala},{0.9581*\yskala})
	--({0.5224*\xskala},{0.9435*\yskala})
	--({0.5323*\xskala},{0.9267*\yskala})
	--({0.5423*\xskala},{0.9079*\yskala})
	--({0.5522*\xskala},{0.8869*\yskala})
	--({0.5622*\xskala},{0.8639*\yskala})
	--({0.5721*\xskala},{0.8388*\yskala})
	--({0.5821*\xskala},{0.8118*\yskala})
	--({0.5920*\xskala},{0.7827*\yskala})
	--({0.6020*\xskala},{0.7518*\yskala})
	--({0.6119*\xskala},{0.7190*\yskala})
	--({0.6219*\xskala},{0.6845*\yskala})
	--({0.6318*\xskala},{0.6484*\yskala})
	--({0.6418*\xskala},{0.6106*\yskala})
	--({0.6517*\xskala},{0.5714*\yskala})
	--({0.6617*\xskala},{0.5309*\yskala})
	--({0.6716*\xskala},{0.4892*\yskala})
	--({0.6816*\xskala},{0.4464*\yskala})
	--({0.6915*\xskala},{0.4028*\yskala})
	--({0.7015*\xskala},{0.3584*\yskala})
	--({0.7114*\xskala},{0.3135*\yskala})
	--({0.7214*\xskala},{0.2682*\yskala})
	--({0.7313*\xskala},{0.2229*\yskala})
	--({0.7413*\xskala},{0.1777*\yskala})
	--({0.7512*\xskala},{0.1328*\yskala})
	--({0.7612*\xskala},{0.0885*\yskala})
	--({0.7711*\xskala},{0.0452*\yskala})
	--({0.7811*\xskala},{0.0031*\yskala})
	--({0.7910*\xskala},{-0.0376*\yskala})
	--({0.8010*\xskala},{-0.0763*\yskala})
	--({0.8109*\xskala},{-0.1129*\yskala})
	--({0.8209*\xskala},{-0.1469*\yskala})
	--({0.8308*\xskala},{-0.1780*\yskala})
	--({0.8408*\xskala},{-0.2057*\yskala})
	--({0.8507*\xskala},{-0.2296*\yskala})
	--({0.8607*\xskala},{-0.2493*\yskala})
	--({0.8706*\xskala},{-0.2643*\yskala})
	--({0.8806*\xskala},{-0.2740*\yskala})
	--({0.8905*\xskala},{-0.2781*\yskala})
	--({0.9005*\xskala},{-0.2758*\yskala})
	--({0.9104*\xskala},{-0.2668*\yskala})
	--({0.9204*\xskala},{-0.2503*\yskala})
	--({0.9303*\xskala},{-0.2258*\yskala})
	--({0.9403*\xskala},{-0.1927*\yskala})
	--({0.9502*\xskala},{-0.1502*\yskala})
	--({0.9602*\xskala},{-0.0977*\yskala})
	--({0.9701*\xskala},{-0.0344*\yskala})
	--({0.9801*\xskala},{0.0403*\yskala})
	--({0.9900*\xskala},{0.1273*\yskala})
	--({1.0000*\xskala},{0.2274*\yskala});
}
\def\punktfour{
\fill[color=red] ({0.4339*\xskala},\yskala) circle[radius={0.08/\skala}];
;
}
\def\basisfive{
\draw[line width=1.4pt,color=red] ({-1.0000*\xskala},{-0.0500*\yskala})
	--({-0.9900*\xskala},{-0.0277*\yskala})
	--({-0.9801*\xskala},{-0.0087*\yskala})
	--({-0.9701*\xskala},{0.0073*\yskala})
	--({-0.9602*\xskala},{0.0204*\yskala})
	--({-0.9502*\xskala},{0.0310*\yskala})
	--({-0.9403*\xskala},{0.0392*\yskala})
	--({-0.9303*\xskala},{0.0453*\yskala})
	--({-0.9204*\xskala},{0.0495*\yskala})
	--({-0.9104*\xskala},{0.0520*\yskala})
	--({-0.9005*\xskala},{0.0529*\yskala})
	--({-0.8905*\xskala},{0.0525*\yskala})
	--({-0.8806*\xskala},{0.0510*\yskala})
	--({-0.8706*\xskala},{0.0483*\yskala})
	--({-0.8607*\xskala},{0.0448*\yskala})
	--({-0.8507*\xskala},{0.0406*\yskala})
	--({-0.8408*\xskala},{0.0357*\yskala})
	--({-0.8308*\xskala},{0.0303*\yskala})
	--({-0.8209*\xskala},{0.0246*\yskala})
	--({-0.8109*\xskala},{0.0185*\yskala})
	--({-0.8010*\xskala},{0.0123*\yskala})
	--({-0.7910*\xskala},{0.0059*\yskala})
	--({-0.7811*\xskala},{-0.0005*\yskala})
	--({-0.7711*\xskala},{-0.0068*\yskala})
	--({-0.7612*\xskala},{-0.0130*\yskala})
	--({-0.7512*\xskala},{-0.0190*\yskala})
	--({-0.7413*\xskala},{-0.0248*\yskala})
	--({-0.7313*\xskala},{-0.0303*\yskala})
	--({-0.7214*\xskala},{-0.0355*\yskala})
	--({-0.7114*\xskala},{-0.0403*\yskala})
	--({-0.7015*\xskala},{-0.0447*\yskala})
	--({-0.6915*\xskala},{-0.0487*\yskala})
	--({-0.6816*\xskala},{-0.0523*\yskala})
	--({-0.6716*\xskala},{-0.0554*\yskala})
	--({-0.6617*\xskala},{-0.0580*\yskala})
	--({-0.6517*\xskala},{-0.0601*\yskala})
	--({-0.6418*\xskala},{-0.0617*\yskala})
	--({-0.6318*\xskala},{-0.0628*\yskala})
	--({-0.6219*\xskala},{-0.0634*\yskala})
	--({-0.6119*\xskala},{-0.0636*\yskala})
	--({-0.6020*\xskala},{-0.0632*\yskala})
	--({-0.5920*\xskala},{-0.0624*\yskala})
	--({-0.5821*\xskala},{-0.0610*\yskala})
	--({-0.5721*\xskala},{-0.0593*\yskala})
	--({-0.5622*\xskala},{-0.0571*\yskala})
	--({-0.5522*\xskala},{-0.0545*\yskala})
	--({-0.5423*\xskala},{-0.0514*\yskala})
	--({-0.5323*\xskala},{-0.0480*\yskala})
	--({-0.5224*\xskala},{-0.0443*\yskala})
	--({-0.5124*\xskala},{-0.0402*\yskala})
	--({-0.5025*\xskala},{-0.0359*\yskala})
	--({-0.4925*\xskala},{-0.0312*\yskala})
	--({-0.4826*\xskala},{-0.0264*\yskala})
	--({-0.4726*\xskala},{-0.0213*\yskala})
	--({-0.4627*\xskala},{-0.0160*\yskala})
	--({-0.4527*\xskala},{-0.0106*\yskala})
	--({-0.4428*\xskala},{-0.0050*\yskala})
	--({-0.4328*\xskala},{0.0006*\yskala})
	--({-0.4229*\xskala},{0.0063*\yskala})
	--({-0.4129*\xskala},{0.0120*\yskala})
	--({-0.4030*\xskala},{0.0177*\yskala})
	--({-0.3930*\xskala},{0.0234*\yskala})
	--({-0.3831*\xskala},{0.0290*\yskala})
	--({-0.3731*\xskala},{0.0346*\yskala})
	--({-0.3632*\xskala},{0.0400*\yskala})
	--({-0.3532*\xskala},{0.0452*\yskala})
	--({-0.3433*\xskala},{0.0503*\yskala})
	--({-0.3333*\xskala},{0.0552*\yskala})
	--({-0.3234*\xskala},{0.0598*\yskala})
	--({-0.3134*\xskala},{0.0642*\yskala})
	--({-0.3035*\xskala},{0.0683*\yskala})
	--({-0.2935*\xskala},{0.0721*\yskala})
	--({-0.2836*\xskala},{0.0756*\yskala})
	--({-0.2736*\xskala},{0.0787*\yskala})
	--({-0.2637*\xskala},{0.0815*\yskala})
	--({-0.2537*\xskala},{0.0838*\yskala})
	--({-0.2438*\xskala},{0.0858*\yskala})
	--({-0.2338*\xskala},{0.0874*\yskala})
	--({-0.2239*\xskala},{0.0886*\yskala})
	--({-0.2139*\xskala},{0.0893*\yskala})
	--({-0.2040*\xskala},{0.0896*\yskala})
	--({-0.1940*\xskala},{0.0894*\yskala})
	--({-0.1841*\xskala},{0.0888*\yskala})
	--({-0.1741*\xskala},{0.0877*\yskala})
	--({-0.1642*\xskala},{0.0861*\yskala})
	--({-0.1542*\xskala},{0.0841*\yskala})
	--({-0.1443*\xskala},{0.0816*\yskala})
	--({-0.1343*\xskala},{0.0787*\yskala})
	--({-0.1244*\xskala},{0.0753*\yskala})
	--({-0.1144*\xskala},{0.0715*\yskala})
	--({-0.1045*\xskala},{0.0672*\yskala})
	--({-0.0945*\xskala},{0.0625*\yskala})
	--({-0.0846*\xskala},{0.0574*\yskala})
	--({-0.0746*\xskala},{0.0519*\yskala})
	--({-0.0647*\xskala},{0.0461*\yskala})
	--({-0.0547*\xskala},{0.0398*\yskala})
	--({-0.0448*\xskala},{0.0332*\yskala})
	--({-0.0348*\xskala},{0.0263*\yskala})
	--({-0.0249*\xskala},{0.0191*\yskala})
	--({-0.0149*\xskala},{0.0117*\yskala})
	--({-0.0050*\xskala},{0.0039*\yskala})
	--({0.0050*\xskala},{-0.0040*\yskala})
	--({0.0149*\xskala},{-0.0121*\yskala})
	--({0.0249*\xskala},{-0.0204*\yskala})
	--({0.0348*\xskala},{-0.0288*\yskala})
	--({0.0448*\xskala},{-0.0373*\yskala})
	--({0.0547*\xskala},{-0.0458*\yskala})
	--({0.0647*\xskala},{-0.0544*\yskala})
	--({0.0746*\xskala},{-0.0629*\yskala})
	--({0.0846*\xskala},{-0.0714*\yskala})
	--({0.0945*\xskala},{-0.0797*\yskala})
	--({0.1045*\xskala},{-0.0880*\yskala})
	--({0.1144*\xskala},{-0.0960*\yskala})
	--({0.1244*\xskala},{-0.1038*\yskala})
	--({0.1343*\xskala},{-0.1113*\yskala})
	--({0.1443*\xskala},{-0.1186*\yskala})
	--({0.1542*\xskala},{-0.1254*\yskala})
	--({0.1642*\xskala},{-0.1319*\yskala})
	--({0.1741*\xskala},{-0.1379*\yskala})
	--({0.1841*\xskala},{-0.1434*\yskala})
	--({0.1940*\xskala},{-0.1484*\yskala})
	--({0.2040*\xskala},{-0.1528*\yskala})
	--({0.2139*\xskala},{-0.1565*\yskala})
	--({0.2239*\xskala},{-0.1596*\yskala})
	--({0.2338*\xskala},{-0.1620*\yskala})
	--({0.2438*\xskala},{-0.1636*\yskala})
	--({0.2537*\xskala},{-0.1644*\yskala})
	--({0.2637*\xskala},{-0.1644*\yskala})
	--({0.2736*\xskala},{-0.1634*\yskala})
	--({0.2836*\xskala},{-0.1616*\yskala})
	--({0.2935*\xskala},{-0.1588*\yskala})
	--({0.3035*\xskala},{-0.1550*\yskala})
	--({0.3134*\xskala},{-0.1501*\yskala})
	--({0.3234*\xskala},{-0.1442*\yskala})
	--({0.3333*\xskala},{-0.1372*\yskala})
	--({0.3433*\xskala},{-0.1291*\yskala})
	--({0.3532*\xskala},{-0.1198*\yskala})
	--({0.3632*\xskala},{-0.1094*\yskala})
	--({0.3731*\xskala},{-0.0977*\yskala})
	--({0.3831*\xskala},{-0.0849*\yskala})
	--({0.3930*\xskala},{-0.0708*\yskala})
	--({0.4030*\xskala},{-0.0555*\yskala})
	--({0.4129*\xskala},{-0.0389*\yskala})
	--({0.4229*\xskala},{-0.0211*\yskala})
	--({0.4328*\xskala},{-0.0021*\yskala})
	--({0.4428*\xskala},{0.0182*\yskala})
	--({0.4527*\xskala},{0.0397*\yskala})
	--({0.4627*\xskala},{0.0624*\yskala})
	--({0.4726*\xskala},{0.0863*\yskala})
	--({0.4826*\xskala},{0.1114*\yskala})
	--({0.4925*\xskala},{0.1376*\yskala})
	--({0.5025*\xskala},{0.1650*\yskala})
	--({0.5124*\xskala},{0.1933*\yskala})
	--({0.5224*\xskala},{0.2227*\yskala})
	--({0.5323*\xskala},{0.2531*\yskala})
	--({0.5423*\xskala},{0.2843*\yskala})
	--({0.5522*\xskala},{0.3164*\yskala})
	--({0.5622*\xskala},{0.3492*\yskala})
	--({0.5721*\xskala},{0.3827*\yskala})
	--({0.5821*\xskala},{0.4168*\yskala})
	--({0.5920*\xskala},{0.4514*\yskala})
	--({0.6020*\xskala},{0.4863*\yskala})
	--({0.6119*\xskala},{0.5215*\yskala})
	--({0.6219*\xskala},{0.5568*\yskala})
	--({0.6318*\xskala},{0.5922*\yskala})
	--({0.6418*\xskala},{0.6273*\yskala})
	--({0.6517*\xskala},{0.6622*\yskala})
	--({0.6617*\xskala},{0.6967*\yskala})
	--({0.6716*\xskala},{0.7305*\yskala})
	--({0.6816*\xskala},{0.7634*\yskala})
	--({0.6915*\xskala},{0.7954*\yskala})
	--({0.7015*\xskala},{0.8261*\yskala})
	--({0.7114*\xskala},{0.8554*\yskala})
	--({0.7214*\xskala},{0.8831*\yskala})
	--({0.7313*\xskala},{0.9088*\yskala})
	--({0.7413*\xskala},{0.9323*\yskala})
	--({0.7512*\xskala},{0.9534*\yskala})
	--({0.7612*\xskala},{0.9718*\yskala})
	--({0.7711*\xskala},{0.9872*\yskala})
	--({0.7811*\xskala},{0.9992*\yskala})
	--({0.7910*\xskala},{1.0076*\yskala})
	--({0.8010*\xskala},{1.0121*\yskala})
	--({0.8109*\xskala},{1.0121*\yskala})
	--({0.8209*\xskala},{1.0075*\yskala})
	--({0.8308*\xskala},{0.9977*\yskala})
	--({0.8408*\xskala},{0.9824*\yskala})
	--({0.8507*\xskala},{0.9613*\yskala})
	--({0.8607*\xskala},{0.9337*\yskala})
	--({0.8706*\xskala},{0.8993*\yskala})
	--({0.8806*\xskala},{0.8577*\yskala})
	--({0.8905*\xskala},{0.8083*\yskala})
	--({0.9005*\xskala},{0.7506*\yskala})
	--({0.9104*\xskala},{0.6841*\yskala})
	--({0.9204*\xskala},{0.6082*\yskala})
	--({0.9303*\xskala},{0.5224*\yskala})
	--({0.9403*\xskala},{0.4260*\yskala})
	--({0.9502*\xskala},{0.3186*\yskala})
	--({0.9602*\xskala},{0.1994*\yskala})
	--({0.9701*\xskala},{0.0678*\yskala})
	--({0.9801*\xskala},{-0.0769*\yskala})
	--({0.9900*\xskala},{-0.2353*\yskala})
	--({1.0000*\xskala},{-0.4083*\yskala});
}
\def\punktfive{
\fill[color=red] ({0.7818*\xskala},\yskala) circle[radius={0.08/\skala}];
;
}
\def\basissix{
\draw[line width=1.4pt,color=red] ({-1.0000*\xskala},{0.0161*\yskala})
	--({-0.9900*\xskala},{0.0089*\yskala})
	--({-0.9801*\xskala},{0.0028*\yskala})
	--({-0.9701*\xskala},{-0.0023*\yskala})
	--({-0.9602*\xskala},{-0.0066*\yskala})
	--({-0.9502*\xskala},{-0.0099*\yskala})
	--({-0.9403*\xskala},{-0.0126*\yskala})
	--({-0.9303*\xskala},{-0.0145*\yskala})
	--({-0.9204*\xskala},{-0.0159*\yskala})
	--({-0.9104*\xskala},{-0.0167*\yskala})
	--({-0.9005*\xskala},{-0.0169*\yskala})
	--({-0.8905*\xskala},{-0.0168*\yskala})
	--({-0.8806*\xskala},{-0.0163*\yskala})
	--({-0.8706*\xskala},{-0.0154*\yskala})
	--({-0.8607*\xskala},{-0.0143*\yskala})
	--({-0.8507*\xskala},{-0.0129*\yskala})
	--({-0.8408*\xskala},{-0.0114*\yskala})
	--({-0.8308*\xskala},{-0.0097*\yskala})
	--({-0.8209*\xskala},{-0.0078*\yskala})
	--({-0.8109*\xskala},{-0.0059*\yskala})
	--({-0.8010*\xskala},{-0.0039*\yskala})
	--({-0.7910*\xskala},{-0.0019*\yskala})
	--({-0.7811*\xskala},{0.0001*\yskala})
	--({-0.7711*\xskala},{0.0022*\yskala})
	--({-0.7612*\xskala},{0.0041*\yskala})
	--({-0.7512*\xskala},{0.0060*\yskala})
	--({-0.7413*\xskala},{0.0079*\yskala})
	--({-0.7313*\xskala},{0.0096*\yskala})
	--({-0.7214*\xskala},{0.0112*\yskala})
	--({-0.7114*\xskala},{0.0127*\yskala})
	--({-0.7015*\xskala},{0.0141*\yskala})
	--({-0.6915*\xskala},{0.0154*\yskala})
	--({-0.6816*\xskala},{0.0165*\yskala})
	--({-0.6716*\xskala},{0.0174*\yskala})
	--({-0.6617*\xskala},{0.0183*\yskala})
	--({-0.6517*\xskala},{0.0189*\yskala})
	--({-0.6418*\xskala},{0.0194*\yskala})
	--({-0.6318*\xskala},{0.0197*\yskala})
	--({-0.6219*\xskala},{0.0199*\yskala})
	--({-0.6119*\xskala},{0.0199*\yskala})
	--({-0.6020*\xskala},{0.0198*\yskala})
	--({-0.5920*\xskala},{0.0195*\yskala})
	--({-0.5821*\xskala},{0.0191*\yskala})
	--({-0.5721*\xskala},{0.0185*\yskala})
	--({-0.5622*\xskala},{0.0178*\yskala})
	--({-0.5522*\xskala},{0.0170*\yskala})
	--({-0.5423*\xskala},{0.0160*\yskala})
	--({-0.5323*\xskala},{0.0150*\yskala})
	--({-0.5224*\xskala},{0.0138*\yskala})
	--({-0.5124*\xskala},{0.0125*\yskala})
	--({-0.5025*\xskala},{0.0111*\yskala})
	--({-0.4925*\xskala},{0.0097*\yskala})
	--({-0.4826*\xskala},{0.0082*\yskala})
	--({-0.4726*\xskala},{0.0066*\yskala})
	--({-0.4627*\xskala},{0.0049*\yskala})
	--({-0.4527*\xskala},{0.0033*\yskala})
	--({-0.4428*\xskala},{0.0016*\yskala})
	--({-0.4328*\xskala},{-0.0002*\yskala})
	--({-0.4229*\xskala},{-0.0019*\yskala})
	--({-0.4129*\xskala},{-0.0037*\yskala})
	--({-0.4030*\xskala},{-0.0054*\yskala})
	--({-0.3930*\xskala},{-0.0072*\yskala})
	--({-0.3831*\xskala},{-0.0089*\yskala})
	--({-0.3731*\xskala},{-0.0106*\yskala})
	--({-0.3632*\xskala},{-0.0122*\yskala})
	--({-0.3532*\xskala},{-0.0138*\yskala})
	--({-0.3433*\xskala},{-0.0153*\yskala})
	--({-0.3333*\xskala},{-0.0168*\yskala})
	--({-0.3234*\xskala},{-0.0182*\yskala})
	--({-0.3134*\xskala},{-0.0195*\yskala})
	--({-0.3035*\xskala},{-0.0207*\yskala})
	--({-0.2935*\xskala},{-0.0218*\yskala})
	--({-0.2836*\xskala},{-0.0228*\yskala})
	--({-0.2736*\xskala},{-0.0237*\yskala})
	--({-0.2637*\xskala},{-0.0245*\yskala})
	--({-0.2537*\xskala},{-0.0252*\yskala})
	--({-0.2438*\xskala},{-0.0258*\yskala})
	--({-0.2338*\xskala},{-0.0262*\yskala})
	--({-0.2239*\xskala},{-0.0265*\yskala})
	--({-0.2139*\xskala},{-0.0267*\yskala})
	--({-0.2040*\xskala},{-0.0267*\yskala})
	--({-0.1940*\xskala},{-0.0266*\yskala})
	--({-0.1841*\xskala},{-0.0264*\yskala})
	--({-0.1741*\xskala},{-0.0260*\yskala})
	--({-0.1642*\xskala},{-0.0255*\yskala})
	--({-0.1542*\xskala},{-0.0249*\yskala})
	--({-0.1443*\xskala},{-0.0241*\yskala})
	--({-0.1343*\xskala},{-0.0232*\yskala})
	--({-0.1244*\xskala},{-0.0222*\yskala})
	--({-0.1144*\xskala},{-0.0210*\yskala})
	--({-0.1045*\xskala},{-0.0197*\yskala})
	--({-0.0945*\xskala},{-0.0183*\yskala})
	--({-0.0846*\xskala},{-0.0168*\yskala})
	--({-0.0746*\xskala},{-0.0151*\yskala})
	--({-0.0647*\xskala},{-0.0134*\yskala})
	--({-0.0547*\xskala},{-0.0115*\yskala})
	--({-0.0448*\xskala},{-0.0096*\yskala})
	--({-0.0348*\xskala},{-0.0076*\yskala})
	--({-0.0249*\xskala},{-0.0055*\yskala})
	--({-0.0149*\xskala},{-0.0033*\yskala})
	--({-0.0050*\xskala},{-0.0011*\yskala})
	--({0.0050*\xskala},{0.0011*\yskala})
	--({0.0149*\xskala},{0.0035*\yskala})
	--({0.0249*\xskala},{0.0058*\yskala})
	--({0.0348*\xskala},{0.0082*\yskala})
	--({0.0448*\xskala},{0.0105*\yskala})
	--({0.0547*\xskala},{0.0129*\yskala})
	--({0.0647*\xskala},{0.0153*\yskala})
	--({0.0746*\xskala},{0.0176*\yskala})
	--({0.0846*\xskala},{0.0199*\yskala})
	--({0.0945*\xskala},{0.0222*\yskala})
	--({0.1045*\xskala},{0.0244*\yskala})
	--({0.1144*\xskala},{0.0266*\yskala})
	--({0.1244*\xskala},{0.0286*\yskala})
	--({0.1343*\xskala},{0.0306*\yskala})
	--({0.1443*\xskala},{0.0325*\yskala})
	--({0.1542*\xskala},{0.0342*\yskala})
	--({0.1642*\xskala},{0.0359*\yskala})
	--({0.1741*\xskala},{0.0373*\yskala})
	--({0.1841*\xskala},{0.0387*\yskala})
	--({0.1940*\xskala},{0.0399*\yskala})
	--({0.2040*\xskala},{0.0409*\yskala})
	--({0.2139*\xskala},{0.0417*\yskala})
	--({0.2239*\xskala},{0.0423*\yskala})
	--({0.2338*\xskala},{0.0428*\yskala})
	--({0.2438*\xskala},{0.0430*\yskala})
	--({0.2537*\xskala},{0.0430*\yskala})
	--({0.2637*\xskala},{0.0427*\yskala})
	--({0.2736*\xskala},{0.0423*\yskala})
	--({0.2836*\xskala},{0.0416*\yskala})
	--({0.2935*\xskala},{0.0406*\yskala})
	--({0.3035*\xskala},{0.0394*\yskala})
	--({0.3134*\xskala},{0.0379*\yskala})
	--({0.3234*\xskala},{0.0362*\yskala})
	--({0.3333*\xskala},{0.0342*\yskala})
	--({0.3433*\xskala},{0.0320*\yskala})
	--({0.3532*\xskala},{0.0295*\yskala})
	--({0.3632*\xskala},{0.0267*\yskala})
	--({0.3731*\xskala},{0.0237*\yskala})
	--({0.3831*\xskala},{0.0204*\yskala})
	--({0.3930*\xskala},{0.0169*\yskala})
	--({0.4030*\xskala},{0.0131*\yskala})
	--({0.4129*\xskala},{0.0091*\yskala})
	--({0.4229*\xskala},{0.0049*\yskala})
	--({0.4328*\xskala},{0.0005*\yskala})
	--({0.4428*\xskala},{-0.0041*\yskala})
	--({0.4527*\xskala},{-0.0089*\yskala})
	--({0.4627*\xskala},{-0.0139*\yskala})
	--({0.4726*\xskala},{-0.0190*\yskala})
	--({0.4826*\xskala},{-0.0242*\yskala})
	--({0.4925*\xskala},{-0.0295*\yskala})
	--({0.5025*\xskala},{-0.0348*\yskala})
	--({0.5124*\xskala},{-0.0402*\yskala})
	--({0.5224*\xskala},{-0.0456*\yskala})
	--({0.5323*\xskala},{-0.0509*\yskala})
	--({0.5423*\xskala},{-0.0562*\yskala})
	--({0.5522*\xskala},{-0.0613*\yskala})
	--({0.5622*\xskala},{-0.0663*\yskala})
	--({0.5721*\xskala},{-0.0711*\yskala})
	--({0.5821*\xskala},{-0.0756*\yskala})
	--({0.5920*\xskala},{-0.0799*\yskala})
	--({0.6020*\xskala},{-0.0837*\yskala})
	--({0.6119*\xskala},{-0.0871*\yskala})
	--({0.6219*\xskala},{-0.0900*\yskala})
	--({0.6318*\xskala},{-0.0924*\yskala})
	--({0.6418*\xskala},{-0.0941*\yskala})
	--({0.6517*\xskala},{-0.0951*\yskala})
	--({0.6617*\xskala},{-0.0954*\yskala})
	--({0.6716*\xskala},{-0.0947*\yskala})
	--({0.6816*\xskala},{-0.0931*\yskala})
	--({0.6915*\xskala},{-0.0904*\yskala})
	--({0.7015*\xskala},{-0.0866*\yskala})
	--({0.7114*\xskala},{-0.0816*\yskala})
	--({0.7214*\xskala},{-0.0751*\yskala})
	--({0.7313*\xskala},{-0.0672*\yskala})
	--({0.7413*\xskala},{-0.0577*\yskala})
	--({0.7512*\xskala},{-0.0465*\yskala})
	--({0.7612*\xskala},{-0.0335*\yskala})
	--({0.7711*\xskala},{-0.0185*\yskala})
	--({0.7811*\xskala},{-0.0014*\yskala})
	--({0.7910*\xskala},{0.0180*\yskala})
	--({0.8010*\xskala},{0.0398*\yskala})
	--({0.8109*\xskala},{0.0641*\yskala})
	--({0.8209*\xskala},{0.0912*\yskala})
	--({0.8308*\xskala},{0.1211*\yskala})
	--({0.8408*\xskala},{0.1541*\yskala})
	--({0.8507*\xskala},{0.1904*\yskala})
	--({0.8607*\xskala},{0.2301*\yskala})
	--({0.8706*\xskala},{0.2734*\yskala})
	--({0.8806*\xskala},{0.3205*\yskala})
	--({0.8905*\xskala},{0.3717*\yskala})
	--({0.9005*\xskala},{0.4271*\yskala})
	--({0.9104*\xskala},{0.4870*\yskala})
	--({0.9204*\xskala},{0.5516*\yskala})
	--({0.9303*\xskala},{0.6211*\yskala})
	--({0.9403*\xskala},{0.6958*\yskala})
	--({0.9502*\xskala},{0.7760*\yskala})
	--({0.9602*\xskala},{0.8618*\yskala})
	--({0.9701*\xskala},{0.9537*\yskala})
	--({0.9801*\xskala},{1.0518*\yskala})
	--({0.9900*\xskala},{1.1564*\yskala})
	--({1.0000*\xskala},{1.2679*\yskala});
}
\def\punktsix{
\fill[color=red] ({0.9749*\xskala},\yskala) circle[radius={0.08/\skala}];
;
}

\begin{tikzpicture}[>=latex,thick,scale=\skala]

\def\bild#1#2#3{
\begin{scope}[yshift=#1]
\begin{scope}
\clip ({-\xskala},-0.4) rectangle ({\xskala},0.75);
\expandafter\expandafter\csname basis#2\endcsname
\expandafter\expandafter\csname punkt#2\endcsname
\end{scope}

\draw[->] ({-\xskala-0.1/\skala},0)--({\xskala+0.1/\skala},0) coordinate[label={$x$}];
\draw[->] (0,-0.35)--(0,0.85) coordinate[label={right:$y$}];

\draw ({-0.1/\skala},\yskala)--({0.1/\skala},\yskala);
\node at ({-0.1/\skala},\yskala) [left] {$1$};
\node at (0.5,\yskala) [above] {$j=#3$};
\stuetz
\end{scope}
}

\bild{0cm}{zero}{0}
\bild{-1.4cm}{one}{1}
\bild{-2.8cm}{two}{2}
\bild{-4.2cm}{three}{3}
\bild{-5.6cm}{four}{4}
\bild{-7.0cm}{five}{5}
\bild{-8.4cm}{six}{6}

\end{tikzpicture}
\end{document}

