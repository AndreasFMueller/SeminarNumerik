%
% norm.tex -- absoluter und relativer Fehler des Interpolationspolynoms
%
% (c) 2020 Prof Dr Andreas Müller, Hochschule Rapperswil
%
\documentclass[tikz]{standalone}
\usepackage{amsmath}
\usepackage{times}
\usepackage{txfonts}
\usepackage{pgfplots}
\usepackage{csvsimple}
\usetikzlibrary{arrows,intersections,math}
\begin{document}
\def\skala{2}
\def\xwertea{
\fill[color=red] (0.0000,0) circle[radius={0.07/\skala}];
\fill[color=red] (2.5000,0) circle[radius={0.07/\skala}];
\fill[color=red] (5.0000,0) circle[radius={0.07/\skala}];
}
\def\punktea{2}
\def\maxfehlera{5.917\cdot 10^{-2}}
\def\fehlera{
\draw[color=red,line width=1.4pt,line join=round] ({\sx*(0.000)},{\sy*(0.0000)})
	--({\sx*(0.0100)},{\sy*(-0.0377)})
	--({\sx*(0.0200)},{\sy*(-0.0746)})
	--({\sx*(0.0300)},{\sy*(-0.1108)})
	--({\sx*(0.0400)},{\sy*(-0.1462)})
	--({\sx*(0.0500)},{\sy*(-0.1808)})
	--({\sx*(0.0600)},{\sy*(-0.2146)})
	--({\sx*(0.0700)},{\sy*(-0.2477)})
	--({\sx*(0.0800)},{\sy*(-0.2800)})
	--({\sx*(0.0900)},{\sy*(-0.3115)})
	--({\sx*(0.1000)},{\sy*(-0.3423)})
	--({\sx*(0.1100)},{\sy*(-0.3723)})
	--({\sx*(0.1200)},{\sy*(-0.4016)})
	--({\sx*(0.1300)},{\sy*(-0.4300)})
	--({\sx*(0.1400)},{\sy*(-0.4578)})
	--({\sx*(0.1500)},{\sy*(-0.4848)})
	--({\sx*(0.1600)},{\sy*(-0.5110)})
	--({\sx*(0.1700)},{\sy*(-0.5365)})
	--({\sx*(0.1800)},{\sy*(-0.5612)})
	--({\sx*(0.1900)},{\sy*(-0.5852)})
	--({\sx*(0.2000)},{\sy*(-0.6085)})
	--({\sx*(0.2100)},{\sy*(-0.6310)})
	--({\sx*(0.2200)},{\sy*(-0.6528)})
	--({\sx*(0.2300)},{\sy*(-0.6739)})
	--({\sx*(0.2400)},{\sy*(-0.6943)})
	--({\sx*(0.2500)},{\sy*(-0.7139)})
	--({\sx*(0.2600)},{\sy*(-0.7329)})
	--({\sx*(0.2700)},{\sy*(-0.7511)})
	--({\sx*(0.2800)},{\sy*(-0.7686)})
	--({\sx*(0.2900)},{\sy*(-0.7854)})
	--({\sx*(0.3000)},{\sy*(-0.8016)})
	--({\sx*(0.3100)},{\sy*(-0.8170)})
	--({\sx*(0.3200)},{\sy*(-0.8318)})
	--({\sx*(0.3300)},{\sy*(-0.8459)})
	--({\sx*(0.3400)},{\sy*(-0.8593)})
	--({\sx*(0.3500)},{\sy*(-0.8721)})
	--({\sx*(0.3600)},{\sy*(-0.8842)})
	--({\sx*(0.3700)},{\sy*(-0.8957)})
	--({\sx*(0.3800)},{\sy*(-0.9065)})
	--({\sx*(0.3900)},{\sy*(-0.9167)})
	--({\sx*(0.4000)},{\sy*(-0.9263)})
	--({\sx*(0.4100)},{\sy*(-0.9352)})
	--({\sx*(0.4200)},{\sy*(-0.9435)})
	--({\sx*(0.4300)},{\sy*(-0.9513)})
	--({\sx*(0.4400)},{\sy*(-0.9584)})
	--({\sx*(0.4500)},{\sy*(-0.9649)})
	--({\sx*(0.4600)},{\sy*(-0.9709)})
	--({\sx*(0.4700)},{\sy*(-0.9762)})
	--({\sx*(0.4800)},{\sy*(-0.9810)})
	--({\sx*(0.4900)},{\sy*(-0.9853)})
	--({\sx*(0.5000)},{\sy*(-0.9890)})
	--({\sx*(0.5100)},{\sy*(-0.9921)})
	--({\sx*(0.5200)},{\sy*(-0.9947)})
	--({\sx*(0.5300)},{\sy*(-0.9968)})
	--({\sx*(0.5400)},{\sy*(-0.9984)})
	--({\sx*(0.5500)},{\sy*(-0.9994)})
	--({\sx*(0.5600)},{\sy*(-0.9999)})
	--({\sx*(0.5700)},{\sy*(-1.0000)})
	--({\sx*(0.5800)},{\sy*(-0.9996)})
	--({\sx*(0.5900)},{\sy*(-0.9987)})
	--({\sx*(0.6000)},{\sy*(-0.9973)})
	--({\sx*(0.6100)},{\sy*(-0.9954)})
	--({\sx*(0.6200)},{\sy*(-0.9932)})
	--({\sx*(0.6300)},{\sy*(-0.9904)})
	--({\sx*(0.6400)},{\sy*(-0.9873)})
	--({\sx*(0.6500)},{\sy*(-0.9837)})
	--({\sx*(0.6600)},{\sy*(-0.9797)})
	--({\sx*(0.6700)},{\sy*(-0.9753)})
	--({\sx*(0.6800)},{\sy*(-0.9705)})
	--({\sx*(0.6900)},{\sy*(-0.9653)})
	--({\sx*(0.7000)},{\sy*(-0.9597)})
	--({\sx*(0.7100)},{\sy*(-0.9538)})
	--({\sx*(0.7200)},{\sy*(-0.9475)})
	--({\sx*(0.7300)},{\sy*(-0.9409)})
	--({\sx*(0.7400)},{\sy*(-0.9339)})
	--({\sx*(0.7500)},{\sy*(-0.9266)})
	--({\sx*(0.7600)},{\sy*(-0.9190)})
	--({\sx*(0.7700)},{\sy*(-0.9111)})
	--({\sx*(0.7800)},{\sy*(-0.9028)})
	--({\sx*(0.7900)},{\sy*(-0.8943)})
	--({\sx*(0.8000)},{\sy*(-0.8855)})
	--({\sx*(0.8100)},{\sy*(-0.8764)})
	--({\sx*(0.8200)},{\sy*(-0.8670)})
	--({\sx*(0.8300)},{\sy*(-0.8574)})
	--({\sx*(0.8400)},{\sy*(-0.8475)})
	--({\sx*(0.8500)},{\sy*(-0.8375)})
	--({\sx*(0.8600)},{\sy*(-0.8271)})
	--({\sx*(0.8700)},{\sy*(-0.8166)})
	--({\sx*(0.8800)},{\sy*(-0.8058)})
	--({\sx*(0.8900)},{\sy*(-0.7949)})
	--({\sx*(0.9000)},{\sy*(-0.7837)})
	--({\sx*(0.9100)},{\sy*(-0.7724)})
	--({\sx*(0.9200)},{\sy*(-0.7609)})
	--({\sx*(0.9300)},{\sy*(-0.7492)})
	--({\sx*(0.9400)},{\sy*(-0.7374)})
	--({\sx*(0.9500)},{\sy*(-0.7254)})
	--({\sx*(0.9600)},{\sy*(-0.7133)})
	--({\sx*(0.9700)},{\sy*(-0.7010)})
	--({\sx*(0.9800)},{\sy*(-0.6886)})
	--({\sx*(0.9900)},{\sy*(-0.6761)})
	--({\sx*(1.0000)},{\sy*(-0.6635)})
	--({\sx*(1.0100)},{\sy*(-0.6508)})
	--({\sx*(1.0200)},{\sy*(-0.6380)})
	--({\sx*(1.0300)},{\sy*(-0.6252)})
	--({\sx*(1.0400)},{\sy*(-0.6122)})
	--({\sx*(1.0500)},{\sy*(-0.5992)})
	--({\sx*(1.0600)},{\sy*(-0.5861)})
	--({\sx*(1.0700)},{\sy*(-0.5730)})
	--({\sx*(1.0800)},{\sy*(-0.5599)})
	--({\sx*(1.0900)},{\sy*(-0.5467)})
	--({\sx*(1.1000)},{\sy*(-0.5334)})
	--({\sx*(1.1100)},{\sy*(-0.5202)})
	--({\sx*(1.1200)},{\sy*(-0.5069)})
	--({\sx*(1.1300)},{\sy*(-0.4937)})
	--({\sx*(1.1400)},{\sy*(-0.4804)})
	--({\sx*(1.1500)},{\sy*(-0.4671)})
	--({\sx*(1.1600)},{\sy*(-0.4539)})
	--({\sx*(1.1700)},{\sy*(-0.4406)})
	--({\sx*(1.1800)},{\sy*(-0.4274)})
	--({\sx*(1.1900)},{\sy*(-0.4143)})
	--({\sx*(1.2000)},{\sy*(-0.4011)})
	--({\sx*(1.2100)},{\sy*(-0.3881)})
	--({\sx*(1.2200)},{\sy*(-0.3750)})
	--({\sx*(1.2300)},{\sy*(-0.3621)})
	--({\sx*(1.2400)},{\sy*(-0.3492)})
	--({\sx*(1.2500)},{\sy*(-0.3363)})
	--({\sx*(1.2600)},{\sy*(-0.3236)})
	--({\sx*(1.2700)},{\sy*(-0.3109)})
	--({\sx*(1.2800)},{\sy*(-0.2983)})
	--({\sx*(1.2900)},{\sy*(-0.2858)})
	--({\sx*(1.3000)},{\sy*(-0.2734)})
	--({\sx*(1.3100)},{\sy*(-0.2611)})
	--({\sx*(1.3200)},{\sy*(-0.2489)})
	--({\sx*(1.3300)},{\sy*(-0.2368)})
	--({\sx*(1.3400)},{\sy*(-0.2248)})
	--({\sx*(1.3500)},{\sy*(-0.2129)})
	--({\sx*(1.3600)},{\sy*(-0.2012)})
	--({\sx*(1.3700)},{\sy*(-0.1896)})
	--({\sx*(1.3800)},{\sy*(-0.1781)})
	--({\sx*(1.3900)},{\sy*(-0.1668)})
	--({\sx*(1.4000)},{\sy*(-0.1556)})
	--({\sx*(1.4100)},{\sy*(-0.1446)})
	--({\sx*(1.4200)},{\sy*(-0.1337)})
	--({\sx*(1.4300)},{\sy*(-0.1229)})
	--({\sx*(1.4400)},{\sy*(-0.1123)})
	--({\sx*(1.4500)},{\sy*(-0.1019)})
	--({\sx*(1.4600)},{\sy*(-0.0916)})
	--({\sx*(1.4700)},{\sy*(-0.0815)})
	--({\sx*(1.4800)},{\sy*(-0.0716)})
	--({\sx*(1.4900)},{\sy*(-0.0618)})
	--({\sx*(1.5000)},{\sy*(-0.0522)})
	--({\sx*(1.5100)},{\sy*(-0.0428)})
	--({\sx*(1.5200)},{\sy*(-0.0336)})
	--({\sx*(1.5300)},{\sy*(-0.0245)})
	--({\sx*(1.5400)},{\sy*(-0.0156)})
	--({\sx*(1.5500)},{\sy*(-0.0069)})
	--({\sx*(1.5600)},{\sy*(0.0016)})
	--({\sx*(1.5700)},{\sy*(0.0099)})
	--({\sx*(1.5800)},{\sy*(0.0181)})
	--({\sx*(1.5900)},{\sy*(0.0260)})
	--({\sx*(1.6000)},{\sy*(0.0337)})
	--({\sx*(1.6100)},{\sy*(0.0413)})
	--({\sx*(1.6200)},{\sy*(0.0487)})
	--({\sx*(1.6300)},{\sy*(0.0558)})
	--({\sx*(1.6400)},{\sy*(0.0628)})
	--({\sx*(1.6500)},{\sy*(0.0696)})
	--({\sx*(1.6600)},{\sy*(0.0761)})
	--({\sx*(1.6700)},{\sy*(0.0825)})
	--({\sx*(1.6800)},{\sy*(0.0887)})
	--({\sx*(1.6900)},{\sy*(0.0946)})
	--({\sx*(1.7000)},{\sy*(0.1004)})
	--({\sx*(1.7100)},{\sy*(0.1060)})
	--({\sx*(1.7200)},{\sy*(0.1113)})
	--({\sx*(1.7300)},{\sy*(0.1165)})
	--({\sx*(1.7400)},{\sy*(0.1214)})
	--({\sx*(1.7500)},{\sy*(0.1262)})
	--({\sx*(1.7600)},{\sy*(0.1307)})
	--({\sx*(1.7700)},{\sy*(0.1351)})
	--({\sx*(1.7800)},{\sy*(0.1392)})
	--({\sx*(1.7900)},{\sy*(0.1432)})
	--({\sx*(1.8000)},{\sy*(0.1469)})
	--({\sx*(1.8100)},{\sy*(0.1505)})
	--({\sx*(1.8200)},{\sy*(0.1538)})
	--({\sx*(1.8300)},{\sy*(0.1570)})
	--({\sx*(1.8400)},{\sy*(0.1599)})
	--({\sx*(1.8500)},{\sy*(0.1627)})
	--({\sx*(1.8600)},{\sy*(0.1652)})
	--({\sx*(1.8700)},{\sy*(0.1676)})
	--({\sx*(1.8800)},{\sy*(0.1697)})
	--({\sx*(1.8900)},{\sy*(0.1717)})
	--({\sx*(1.9000)},{\sy*(0.1735)})
	--({\sx*(1.9100)},{\sy*(0.1751)})
	--({\sx*(1.9200)},{\sy*(0.1765)})
	--({\sx*(1.9300)},{\sy*(0.1777)})
	--({\sx*(1.9400)},{\sy*(0.1787)})
	--({\sx*(1.9500)},{\sy*(0.1795)})
	--({\sx*(1.9600)},{\sy*(0.1802)})
	--({\sx*(1.9700)},{\sy*(0.1807)})
	--({\sx*(1.9800)},{\sy*(0.1809)})
	--({\sx*(1.9900)},{\sy*(0.1811)})
	--({\sx*(2.0000)},{\sy*(0.1810)})
	--({\sx*(2.0100)},{\sy*(0.1807)})
	--({\sx*(2.0200)},{\sy*(0.1803)})
	--({\sx*(2.0300)},{\sy*(0.1797)})
	--({\sx*(2.0400)},{\sy*(0.1790)})
	--({\sx*(2.0500)},{\sy*(0.1781)})
	--({\sx*(2.0600)},{\sy*(0.1770)})
	--({\sx*(2.0700)},{\sy*(0.1757)})
	--({\sx*(2.0800)},{\sy*(0.1743)})
	--({\sx*(2.0900)},{\sy*(0.1727)})
	--({\sx*(2.1000)},{\sy*(0.1710)})
	--({\sx*(2.1100)},{\sy*(0.1691)})
	--({\sx*(2.1200)},{\sy*(0.1671)})
	--({\sx*(2.1300)},{\sy*(0.1649)})
	--({\sx*(2.1400)},{\sy*(0.1625)})
	--({\sx*(2.1500)},{\sy*(0.1601)})
	--({\sx*(2.1600)},{\sy*(0.1574)})
	--({\sx*(2.1700)},{\sy*(0.1547)})
	--({\sx*(2.1800)},{\sy*(0.1517)})
	--({\sx*(2.1900)},{\sy*(0.1487)})
	--({\sx*(2.2000)},{\sy*(0.1455)})
	--({\sx*(2.2100)},{\sy*(0.1422)})
	--({\sx*(2.2200)},{\sy*(0.1388)})
	--({\sx*(2.2300)},{\sy*(0.1352)})
	--({\sx*(2.2400)},{\sy*(0.1315)})
	--({\sx*(2.2500)},{\sy*(0.1277)})
	--({\sx*(2.2600)},{\sy*(0.1238)})
	--({\sx*(2.2700)},{\sy*(0.1197)})
	--({\sx*(2.2800)},{\sy*(0.1155)})
	--({\sx*(2.2900)},{\sy*(0.1113)})
	--({\sx*(2.3000)},{\sy*(0.1069)})
	--({\sx*(2.3100)},{\sy*(0.1024)})
	--({\sx*(2.3200)},{\sy*(0.0978)})
	--({\sx*(2.3300)},{\sy*(0.0931)})
	--({\sx*(2.3400)},{\sy*(0.0883)})
	--({\sx*(2.3500)},{\sy*(0.0834)})
	--({\sx*(2.3600)},{\sy*(0.0784)})
	--({\sx*(2.3700)},{\sy*(0.0733)})
	--({\sx*(2.3800)},{\sy*(0.0681)})
	--({\sx*(2.3900)},{\sy*(0.0629)})
	--({\sx*(2.4000)},{\sy*(0.0575)})
	--({\sx*(2.4100)},{\sy*(0.0521)})
	--({\sx*(2.4200)},{\sy*(0.0466)})
	--({\sx*(2.4300)},{\sy*(0.0410)})
	--({\sx*(2.4400)},{\sy*(0.0354)})
	--({\sx*(2.4500)},{\sy*(0.0296)})
	--({\sx*(2.4600)},{\sy*(0.0238)})
	--({\sx*(2.4700)},{\sy*(0.0180)})
	--({\sx*(2.4800)},{\sy*(0.0120)})
	--({\sx*(2.4900)},{\sy*(0.0060)})
	--({\sx*(2.5000)},{\sy*(0.0000)})
	--({\sx*(2.5100)},{\sy*(-0.0061)})
	--({\sx*(2.5200)},{\sy*(-0.0123)})
	--({\sx*(2.5300)},{\sy*(-0.0185)})
	--({\sx*(2.5400)},{\sy*(-0.0247)})
	--({\sx*(2.5500)},{\sy*(-0.0311)})
	--({\sx*(2.5600)},{\sy*(-0.0374)})
	--({\sx*(2.5700)},{\sy*(-0.0438)})
	--({\sx*(2.5800)},{\sy*(-0.0503)})
	--({\sx*(2.5900)},{\sy*(-0.0567)})
	--({\sx*(2.6000)},{\sy*(-0.0632)})
	--({\sx*(2.6100)},{\sy*(-0.0698)})
	--({\sx*(2.6200)},{\sy*(-0.0764)})
	--({\sx*(2.6300)},{\sy*(-0.0830)})
	--({\sx*(2.6400)},{\sy*(-0.0896)})
	--({\sx*(2.6500)},{\sy*(-0.0963)})
	--({\sx*(2.6600)},{\sy*(-0.1030)})
	--({\sx*(2.6700)},{\sy*(-0.1097)})
	--({\sx*(2.6800)},{\sy*(-0.1164)})
	--({\sx*(2.6900)},{\sy*(-0.1231)})
	--({\sx*(2.7000)},{\sy*(-0.1299)})
	--({\sx*(2.7100)},{\sy*(-0.1367)})
	--({\sx*(2.7200)},{\sy*(-0.1434)})
	--({\sx*(2.7300)},{\sy*(-0.1502)})
	--({\sx*(2.7400)},{\sy*(-0.1570)})
	--({\sx*(2.7500)},{\sy*(-0.1638)})
	--({\sx*(2.7600)},{\sy*(-0.1706)})
	--({\sx*(2.7700)},{\sy*(-0.1774)})
	--({\sx*(2.7800)},{\sy*(-0.1842)})
	--({\sx*(2.7900)},{\sy*(-0.1910)})
	--({\sx*(2.8000)},{\sy*(-0.1978)})
	--({\sx*(2.8100)},{\sy*(-0.2046)})
	--({\sx*(2.8200)},{\sy*(-0.2114)})
	--({\sx*(2.8300)},{\sy*(-0.2181)})
	--({\sx*(2.8400)},{\sy*(-0.2249)})
	--({\sx*(2.8500)},{\sy*(-0.2316)})
	--({\sx*(2.8600)},{\sy*(-0.2383)})
	--({\sx*(2.8700)},{\sy*(-0.2450)})
	--({\sx*(2.8800)},{\sy*(-0.2517)})
	--({\sx*(2.8900)},{\sy*(-0.2584)})
	--({\sx*(2.9000)},{\sy*(-0.2650)})
	--({\sx*(2.9100)},{\sy*(-0.2717)})
	--({\sx*(2.9200)},{\sy*(-0.2782)})
	--({\sx*(2.9300)},{\sy*(-0.2848)})
	--({\sx*(2.9400)},{\sy*(-0.2913)})
	--({\sx*(2.9500)},{\sy*(-0.2979)})
	--({\sx*(2.9600)},{\sy*(-0.3043)})
	--({\sx*(2.9700)},{\sy*(-0.3108)})
	--({\sx*(2.9800)},{\sy*(-0.3172)})
	--({\sx*(2.9900)},{\sy*(-0.3236)})
	--({\sx*(3.0000)},{\sy*(-0.3299)})
	--({\sx*(3.0100)},{\sy*(-0.3362)})
	--({\sx*(3.0200)},{\sy*(-0.3425)})
	--({\sx*(3.0300)},{\sy*(-0.3487)})
	--({\sx*(3.0400)},{\sy*(-0.3548)})
	--({\sx*(3.0500)},{\sy*(-0.3610)})
	--({\sx*(3.0600)},{\sy*(-0.3671)})
	--({\sx*(3.0700)},{\sy*(-0.3731)})
	--({\sx*(3.0800)},{\sy*(-0.3791)})
	--({\sx*(3.0900)},{\sy*(-0.3850)})
	--({\sx*(3.1000)},{\sy*(-0.3909)})
	--({\sx*(3.1100)},{\sy*(-0.3968)})
	--({\sx*(3.1200)},{\sy*(-0.4026)})
	--({\sx*(3.1300)},{\sy*(-0.4083)})
	--({\sx*(3.1400)},{\sy*(-0.4140)})
	--({\sx*(3.1500)},{\sy*(-0.4196)})
	--({\sx*(3.1600)},{\sy*(-0.4252)})
	--({\sx*(3.1700)},{\sy*(-0.4307)})
	--({\sx*(3.1800)},{\sy*(-0.4362)})
	--({\sx*(3.1900)},{\sy*(-0.4416)})
	--({\sx*(3.2000)},{\sy*(-0.4469)})
	--({\sx*(3.2100)},{\sy*(-0.4522)})
	--({\sx*(3.2200)},{\sy*(-0.4574)})
	--({\sx*(3.2300)},{\sy*(-0.4625)})
	--({\sx*(3.2400)},{\sy*(-0.4676)})
	--({\sx*(3.2500)},{\sy*(-0.4727)})
	--({\sx*(3.2600)},{\sy*(-0.4776)})
	--({\sx*(3.2700)},{\sy*(-0.4825)})
	--({\sx*(3.2800)},{\sy*(-0.4873)})
	--({\sx*(3.2900)},{\sy*(-0.4921)})
	--({\sx*(3.3000)},{\sy*(-0.4968)})
	--({\sx*(3.3100)},{\sy*(-0.5014)})
	--({\sx*(3.3200)},{\sy*(-0.5059)})
	--({\sx*(3.3300)},{\sy*(-0.5104)})
	--({\sx*(3.3400)},{\sy*(-0.5148)})
	--({\sx*(3.3500)},{\sy*(-0.5191)})
	--({\sx*(3.3600)},{\sy*(-0.5234)})
	--({\sx*(3.3700)},{\sy*(-0.5276)})
	--({\sx*(3.3800)},{\sy*(-0.5317)})
	--({\sx*(3.3900)},{\sy*(-0.5357)})
	--({\sx*(3.4000)},{\sy*(-0.5397)})
	--({\sx*(3.4100)},{\sy*(-0.5436)})
	--({\sx*(3.4200)},{\sy*(-0.5474)})
	--({\sx*(3.4300)},{\sy*(-0.5511)})
	--({\sx*(3.4400)},{\sy*(-0.5548)})
	--({\sx*(3.4500)},{\sy*(-0.5583)})
	--({\sx*(3.4600)},{\sy*(-0.5618)})
	--({\sx*(3.4700)},{\sy*(-0.5652)})
	--({\sx*(3.4800)},{\sy*(-0.5686)})
	--({\sx*(3.4900)},{\sy*(-0.5718)})
	--({\sx*(3.5000)},{\sy*(-0.5750)})
	--({\sx*(3.5100)},{\sy*(-0.5781)})
	--({\sx*(3.5200)},{\sy*(-0.5811)})
	--({\sx*(3.5300)},{\sy*(-0.5840)})
	--({\sx*(3.5400)},{\sy*(-0.5868)})
	--({\sx*(3.5500)},{\sy*(-0.5896)})
	--({\sx*(3.5600)},{\sy*(-0.5923)})
	--({\sx*(3.5700)},{\sy*(-0.5949)})
	--({\sx*(3.5800)},{\sy*(-0.5974)})
	--({\sx*(3.5900)},{\sy*(-0.5998)})
	--({\sx*(3.6000)},{\sy*(-0.6021)})
	--({\sx*(3.6100)},{\sy*(-0.6044)})
	--({\sx*(3.6200)},{\sy*(-0.6065)})
	--({\sx*(3.6300)},{\sy*(-0.6086)})
	--({\sx*(3.6400)},{\sy*(-0.6106)})
	--({\sx*(3.6500)},{\sy*(-0.6125)})
	--({\sx*(3.6600)},{\sy*(-0.6143)})
	--({\sx*(3.6700)},{\sy*(-0.6160)})
	--({\sx*(3.6800)},{\sy*(-0.6176)})
	--({\sx*(3.6900)},{\sy*(-0.6192)})
	--({\sx*(3.7000)},{\sy*(-0.6206)})
	--({\sx*(3.7100)},{\sy*(-0.6220)})
	--({\sx*(3.7200)},{\sy*(-0.6233)})
	--({\sx*(3.7300)},{\sy*(-0.6245)})
	--({\sx*(3.7400)},{\sy*(-0.6256)})
	--({\sx*(3.7500)},{\sy*(-0.6266)})
	--({\sx*(3.7600)},{\sy*(-0.6275)})
	--({\sx*(3.7700)},{\sy*(-0.6283)})
	--({\sx*(3.7800)},{\sy*(-0.6290)})
	--({\sx*(3.7900)},{\sy*(-0.6297)})
	--({\sx*(3.8000)},{\sy*(-0.6302)})
	--({\sx*(3.8100)},{\sy*(-0.6307)})
	--({\sx*(3.8200)},{\sy*(-0.6311)})
	--({\sx*(3.8300)},{\sy*(-0.6313)})
	--({\sx*(3.8400)},{\sy*(-0.6315)})
	--({\sx*(3.8500)},{\sy*(-0.6316)})
	--({\sx*(3.8600)},{\sy*(-0.6316)})
	--({\sx*(3.8700)},{\sy*(-0.6315)})
	--({\sx*(3.8800)},{\sy*(-0.6313)})
	--({\sx*(3.8900)},{\sy*(-0.6310)})
	--({\sx*(3.9000)},{\sy*(-0.6307)})
	--({\sx*(3.9100)},{\sy*(-0.6302)})
	--({\sx*(3.9200)},{\sy*(-0.6296)})
	--({\sx*(3.9300)},{\sy*(-0.6290)})
	--({\sx*(3.9400)},{\sy*(-0.6282)})
	--({\sx*(3.9500)},{\sy*(-0.6274)})
	--({\sx*(3.9600)},{\sy*(-0.6264)})
	--({\sx*(3.9700)},{\sy*(-0.6254)})
	--({\sx*(3.9800)},{\sy*(-0.6243)})
	--({\sx*(3.9900)},{\sy*(-0.6231)})
	--({\sx*(4.0000)},{\sy*(-0.6217)})
	--({\sx*(4.0100)},{\sy*(-0.6203)})
	--({\sx*(4.0200)},{\sy*(-0.6188)})
	--({\sx*(4.0300)},{\sy*(-0.6172)})
	--({\sx*(4.0400)},{\sy*(-0.6155)})
	--({\sx*(4.0500)},{\sy*(-0.6137)})
	--({\sx*(4.0600)},{\sy*(-0.6118)})
	--({\sx*(4.0700)},{\sy*(-0.6098)})
	--({\sx*(4.0800)},{\sy*(-0.6078)})
	--({\sx*(4.0900)},{\sy*(-0.6056)})
	--({\sx*(4.1000)},{\sy*(-0.6033)})
	--({\sx*(4.1100)},{\sy*(-0.6009)})
	--({\sx*(4.1200)},{\sy*(-0.5985)})
	--({\sx*(4.1300)},{\sy*(-0.5959)})
	--({\sx*(4.1400)},{\sy*(-0.5933)})
	--({\sx*(4.1500)},{\sy*(-0.5905)})
	--({\sx*(4.1600)},{\sy*(-0.5877)})
	--({\sx*(4.1700)},{\sy*(-0.5847)})
	--({\sx*(4.1800)},{\sy*(-0.5817)})
	--({\sx*(4.1900)},{\sy*(-0.5785)})
	--({\sx*(4.2000)},{\sy*(-0.5753)})
	--({\sx*(4.2100)},{\sy*(-0.5720)})
	--({\sx*(4.2200)},{\sy*(-0.5685)})
	--({\sx*(4.2300)},{\sy*(-0.5650)})
	--({\sx*(4.2400)},{\sy*(-0.5614)})
	--({\sx*(4.2500)},{\sy*(-0.5577)})
	--({\sx*(4.2600)},{\sy*(-0.5538)})
	--({\sx*(4.2700)},{\sy*(-0.5499)})
	--({\sx*(4.2800)},{\sy*(-0.5459)})
	--({\sx*(4.2900)},{\sy*(-0.5418)})
	--({\sx*(4.3000)},{\sy*(-0.5376)})
	--({\sx*(4.3100)},{\sy*(-0.5333)})
	--({\sx*(4.3200)},{\sy*(-0.5289)})
	--({\sx*(4.3300)},{\sy*(-0.5244)})
	--({\sx*(4.3400)},{\sy*(-0.5198)})
	--({\sx*(4.3500)},{\sy*(-0.5151)})
	--({\sx*(4.3600)},{\sy*(-0.5103)})
	--({\sx*(4.3700)},{\sy*(-0.5054)})
	--({\sx*(4.3800)},{\sy*(-0.5004)})
	--({\sx*(4.3900)},{\sy*(-0.4954)})
	--({\sx*(4.4000)},{\sy*(-0.4902)})
	--({\sx*(4.4100)},{\sy*(-0.4849)})
	--({\sx*(4.4200)},{\sy*(-0.4795)})
	--({\sx*(4.4300)},{\sy*(-0.4740)})
	--({\sx*(4.4400)},{\sy*(-0.4685)})
	--({\sx*(4.4500)},{\sy*(-0.4628)})
	--({\sx*(4.4600)},{\sy*(-0.4570)})
	--({\sx*(4.4700)},{\sy*(-0.4512)})
	--({\sx*(4.4800)},{\sy*(-0.4452)})
	--({\sx*(4.4900)},{\sy*(-0.4392)})
	--({\sx*(4.5000)},{\sy*(-0.4330)})
	--({\sx*(4.5100)},{\sy*(-0.4267)})
	--({\sx*(4.5200)},{\sy*(-0.4204)})
	--({\sx*(4.5300)},{\sy*(-0.4139)})
	--({\sx*(4.5400)},{\sy*(-0.4074)})
	--({\sx*(4.5500)},{\sy*(-0.4007)})
	--({\sx*(4.5600)},{\sy*(-0.3940)})
	--({\sx*(4.5700)},{\sy*(-0.3871)})
	--({\sx*(4.5800)},{\sy*(-0.3802)})
	--({\sx*(4.5900)},{\sy*(-0.3732)})
	--({\sx*(4.6000)},{\sy*(-0.3660)})
	--({\sx*(4.6100)},{\sy*(-0.3588)})
	--({\sx*(4.6200)},{\sy*(-0.3515)})
	--({\sx*(4.6300)},{\sy*(-0.3440)})
	--({\sx*(4.6400)},{\sy*(-0.3365)})
	--({\sx*(4.6500)},{\sy*(-0.3289)})
	--({\sx*(4.6600)},{\sy*(-0.3211)})
	--({\sx*(4.6700)},{\sy*(-0.3133)})
	--({\sx*(4.6800)},{\sy*(-0.3054)})
	--({\sx*(4.6900)},{\sy*(-0.2974)})
	--({\sx*(4.7000)},{\sy*(-0.2893)})
	--({\sx*(4.7100)},{\sy*(-0.2810)})
	--({\sx*(4.7200)},{\sy*(-0.2727)})
	--({\sx*(4.7300)},{\sy*(-0.2643)})
	--({\sx*(4.7400)},{\sy*(-0.2558)})
	--({\sx*(4.7500)},{\sy*(-0.2472)})
	--({\sx*(4.7600)},{\sy*(-0.2385)})
	--({\sx*(4.7700)},{\sy*(-0.2297)})
	--({\sx*(4.7800)},{\sy*(-0.2208)})
	--({\sx*(4.7900)},{\sy*(-0.2118)})
	--({\sx*(4.8000)},{\sy*(-0.2027)})
	--({\sx*(4.8100)},{\sy*(-0.1935)})
	--({\sx*(4.8200)},{\sy*(-0.1842)})
	--({\sx*(4.8300)},{\sy*(-0.1748)})
	--({\sx*(4.8400)},{\sy*(-0.1653)})
	--({\sx*(4.8500)},{\sy*(-0.1557)})
	--({\sx*(4.8600)},{\sy*(-0.1460)})
	--({\sx*(4.8700)},{\sy*(-0.1362)})
	--({\sx*(4.8800)},{\sy*(-0.1263)})
	--({\sx*(4.8900)},{\sy*(-0.1163)})
	--({\sx*(4.9000)},{\sy*(-0.1062)})
	--({\sx*(4.9100)},{\sy*(-0.0961)})
	--({\sx*(4.9200)},{\sy*(-0.0858)})
	--({\sx*(4.9300)},{\sy*(-0.0754)})
	--({\sx*(4.9400)},{\sy*(-0.0649)})
	--({\sx*(4.9500)},{\sy*(-0.0544)})
	--({\sx*(4.9600)},{\sy*(-0.0437)})
	--({\sx*(4.9700)},{\sy*(-0.0329)})
	--({\sx*(4.9800)},{\sy*(-0.0220)})
	--({\sx*(4.9900)},{\sy*(-0.0111)})
	--({\sx*(5.0000)},{\sy*(0.0000)});
}
\def\relfehlera{
\draw[color=blue,line width=1.4pt,line join=round] ({\sx*(0.000)},{\sy*(0.0000)})
	--({\sx*(0.0100)},{\sy*(-0.0056)})
	--({\sx*(0.0200)},{\sy*(-0.0112)})
	--({\sx*(0.0300)},{\sy*(-0.0167)})
	--({\sx*(0.0400)},{\sy*(-0.0222)})
	--({\sx*(0.0500)},{\sy*(-0.0276)})
	--({\sx*(0.0600)},{\sy*(-0.0329)})
	--({\sx*(0.0700)},{\sy*(-0.0382)})
	--({\sx*(0.0800)},{\sy*(-0.0435)})
	--({\sx*(0.0900)},{\sy*(-0.0486)})
	--({\sx*(0.1000)},{\sy*(-0.0538)})
	--({\sx*(0.1100)},{\sy*(-0.0588)})
	--({\sx*(0.1200)},{\sy*(-0.0638)})
	--({\sx*(0.1300)},{\sy*(-0.0687)})
	--({\sx*(0.1400)},{\sy*(-0.0736)})
	--({\sx*(0.1500)},{\sy*(-0.0784)})
	--({\sx*(0.1600)},{\sy*(-0.0831)})
	--({\sx*(0.1700)},{\sy*(-0.0878)})
	--({\sx*(0.1800)},{\sy*(-0.0924)})
	--({\sx*(0.1900)},{\sy*(-0.0969)})
	--({\sx*(0.2000)},{\sy*(-0.1014)})
	--({\sx*(0.2100)},{\sy*(-0.1058)})
	--({\sx*(0.2200)},{\sy*(-0.1101)})
	--({\sx*(0.2300)},{\sy*(-0.1144)})
	--({\sx*(0.2400)},{\sy*(-0.1185)})
	--({\sx*(0.2500)},{\sy*(-0.1226)})
	--({\sx*(0.2600)},{\sy*(-0.1267)})
	--({\sx*(0.2700)},{\sy*(-0.1306)})
	--({\sx*(0.2800)},{\sy*(-0.1345)})
	--({\sx*(0.2900)},{\sy*(-0.1383)})
	--({\sx*(0.3000)},{\sy*(-0.1420)})
	--({\sx*(0.3100)},{\sy*(-0.1457)})
	--({\sx*(0.3200)},{\sy*(-0.1492)})
	--({\sx*(0.3300)},{\sy*(-0.1527)})
	--({\sx*(0.3400)},{\sy*(-0.1561)})
	--({\sx*(0.3500)},{\sy*(-0.1594)})
	--({\sx*(0.3600)},{\sy*(-0.1627)})
	--({\sx*(0.3700)},{\sy*(-0.1659)})
	--({\sx*(0.3800)},{\sy*(-0.1689)})
	--({\sx*(0.3900)},{\sy*(-0.1719)})
	--({\sx*(0.4000)},{\sy*(-0.1748)})
	--({\sx*(0.4100)},{\sy*(-0.1777)})
	--({\sx*(0.4200)},{\sy*(-0.1804)})
	--({\sx*(0.4300)},{\sy*(-0.1831)})
	--({\sx*(0.4400)},{\sy*(-0.1857)})
	--({\sx*(0.4500)},{\sy*(-0.1882)})
	--({\sx*(0.4600)},{\sy*(-0.1906)})
	--({\sx*(0.4700)},{\sy*(-0.1929)})
	--({\sx*(0.4800)},{\sy*(-0.1951)})
	--({\sx*(0.4900)},{\sy*(-0.1973)})
	--({\sx*(0.5000)},{\sy*(-0.1993)})
	--({\sx*(0.5100)},{\sy*(-0.2013)})
	--({\sx*(0.5200)},{\sy*(-0.2032)})
	--({\sx*(0.5300)},{\sy*(-0.2050)})
	--({\sx*(0.5400)},{\sy*(-0.2067)})
	--({\sx*(0.5500)},{\sy*(-0.2084)})
	--({\sx*(0.5600)},{\sy*(-0.2099)})
	--({\sx*(0.5700)},{\sy*(-0.2114)})
	--({\sx*(0.5800)},{\sy*(-0.2127)})
	--({\sx*(0.5900)},{\sy*(-0.2140)})
	--({\sx*(0.6000)},{\sy*(-0.2152)})
	--({\sx*(0.6100)},{\sy*(-0.2163)})
	--({\sx*(0.6200)},{\sy*(-0.2173)})
	--({\sx*(0.6300)},{\sy*(-0.2182)})
	--({\sx*(0.6400)},{\sy*(-0.2191)})
	--({\sx*(0.6500)},{\sy*(-0.2198)})
	--({\sx*(0.6600)},{\sy*(-0.2205)})
	--({\sx*(0.6700)},{\sy*(-0.2211)})
	--({\sx*(0.6800)},{\sy*(-0.2216)})
	--({\sx*(0.6900)},{\sy*(-0.2220)})
	--({\sx*(0.7000)},{\sy*(-0.2223)})
	--({\sx*(0.7100)},{\sy*(-0.2225)})
	--({\sx*(0.7200)},{\sy*(-0.2227)})
	--({\sx*(0.7300)},{\sy*(-0.2227)})
	--({\sx*(0.7400)},{\sy*(-0.2227)})
	--({\sx*(0.7500)},{\sy*(-0.2226)})
	--({\sx*(0.7600)},{\sy*(-0.2224)})
	--({\sx*(0.7700)},{\sy*(-0.2221)})
	--({\sx*(0.7800)},{\sy*(-0.2218)})
	--({\sx*(0.7900)},{\sy*(-0.2213)})
	--({\sx*(0.8000)},{\sy*(-0.2208)})
	--({\sx*(0.8100)},{\sy*(-0.2202)})
	--({\sx*(0.8200)},{\sy*(-0.2195)})
	--({\sx*(0.8300)},{\sy*(-0.2187)})
	--({\sx*(0.8400)},{\sy*(-0.2179)})
	--({\sx*(0.8500)},{\sy*(-0.2169)})
	--({\sx*(0.8600)},{\sy*(-0.2159)})
	--({\sx*(0.8700)},{\sy*(-0.2148)})
	--({\sx*(0.8800)},{\sy*(-0.2136)})
	--({\sx*(0.8900)},{\sy*(-0.2124)})
	--({\sx*(0.9000)},{\sy*(-0.2111)})
	--({\sx*(0.9100)},{\sy*(-0.2097)})
	--({\sx*(0.9200)},{\sy*(-0.2082)})
	--({\sx*(0.9300)},{\sy*(-0.2066)})
	--({\sx*(0.9400)},{\sy*(-0.2050)})
	--({\sx*(0.9500)},{\sy*(-0.2033)})
	--({\sx*(0.9600)},{\sy*(-0.2015)})
	--({\sx*(0.9700)},{\sy*(-0.1997)})
	--({\sx*(0.9800)},{\sy*(-0.1977)})
	--({\sx*(0.9900)},{\sy*(-0.1957)})
	--({\sx*(1.0000)},{\sy*(-0.1937)})
	--({\sx*(1.0100)},{\sy*(-0.1915)})
	--({\sx*(1.0200)},{\sy*(-0.1894)})
	--({\sx*(1.0300)},{\sy*(-0.1871)})
	--({\sx*(1.0400)},{\sy*(-0.1848)})
	--({\sx*(1.0500)},{\sy*(-0.1824)})
	--({\sx*(1.0600)},{\sy*(-0.1799)})
	--({\sx*(1.0700)},{\sy*(-0.1774)})
	--({\sx*(1.0800)},{\sy*(-0.1748)})
	--({\sx*(1.0900)},{\sy*(-0.1721)})
	--({\sx*(1.1000)},{\sy*(-0.1694)})
	--({\sx*(1.1100)},{\sy*(-0.1667)})
	--({\sx*(1.1200)},{\sy*(-0.1638)})
	--({\sx*(1.1300)},{\sy*(-0.1610)})
	--({\sx*(1.1400)},{\sy*(-0.1580)})
	--({\sx*(1.1500)},{\sy*(-0.1550)})
	--({\sx*(1.1600)},{\sy*(-0.1520)})
	--({\sx*(1.1700)},{\sy*(-0.1489)})
	--({\sx*(1.1800)},{\sy*(-0.1457)})
	--({\sx*(1.1900)},{\sy*(-0.1425)})
	--({\sx*(1.2000)},{\sy*(-0.1393)})
	--({\sx*(1.2100)},{\sy*(-0.1359)})
	--({\sx*(1.2200)},{\sy*(-0.1326)})
	--({\sx*(1.2300)},{\sy*(-0.1292)})
	--({\sx*(1.2400)},{\sy*(-0.1258)})
	--({\sx*(1.2500)},{\sy*(-0.1223)})
	--({\sx*(1.2600)},{\sy*(-0.1187)})
	--({\sx*(1.2700)},{\sy*(-0.1152)})
	--({\sx*(1.2800)},{\sy*(-0.1116)})
	--({\sx*(1.2900)},{\sy*(-0.1079)})
	--({\sx*(1.3000)},{\sy*(-0.1042)})
	--({\sx*(1.3100)},{\sy*(-0.1005)})
	--({\sx*(1.3200)},{\sy*(-0.0967)})
	--({\sx*(1.3300)},{\sy*(-0.0929)})
	--({\sx*(1.3400)},{\sy*(-0.0891)})
	--({\sx*(1.3500)},{\sy*(-0.0853)})
	--({\sx*(1.3600)},{\sy*(-0.0814)})
	--({\sx*(1.3700)},{\sy*(-0.0774)})
	--({\sx*(1.3800)},{\sy*(-0.0735)})
	--({\sx*(1.3900)},{\sy*(-0.0695)})
	--({\sx*(1.4000)},{\sy*(-0.0655)})
	--({\sx*(1.4100)},{\sy*(-0.0615)})
	--({\sx*(1.4200)},{\sy*(-0.0575)})
	--({\sx*(1.4300)},{\sy*(-0.0534)})
	--({\sx*(1.4400)},{\sy*(-0.0493)})
	--({\sx*(1.4500)},{\sy*(-0.0452)})
	--({\sx*(1.4600)},{\sy*(-0.0411)})
	--({\sx*(1.4700)},{\sy*(-0.0369)})
	--({\sx*(1.4800)},{\sy*(-0.0328)})
	--({\sx*(1.4900)},{\sy*(-0.0286)})
	--({\sx*(1.5000)},{\sy*(-0.0244)})
	--({\sx*(1.5100)},{\sy*(-0.0203)})
	--({\sx*(1.5200)},{\sy*(-0.0161)})
	--({\sx*(1.5300)},{\sy*(-0.0119)})
	--({\sx*(1.5400)},{\sy*(-0.0076)})
	--({\sx*(1.5500)},{\sy*(-0.0034)})
	--({\sx*(1.5600)},{\sy*(0.0008)})
	--({\sx*(1.5700)},{\sy*(0.0050)})
	--({\sx*(1.5800)},{\sy*(0.0092)})
	--({\sx*(1.5900)},{\sy*(0.0135)})
	--({\sx*(1.6000)},{\sy*(0.0177)})
	--({\sx*(1.6100)},{\sy*(0.0219)})
	--({\sx*(1.6200)},{\sy*(0.0261)})
	--({\sx*(1.6300)},{\sy*(0.0303)})
	--({\sx*(1.6400)},{\sy*(0.0345)})
	--({\sx*(1.6500)},{\sy*(0.0387)})
	--({\sx*(1.6600)},{\sy*(0.0429)})
	--({\sx*(1.6700)},{\sy*(0.0470)})
	--({\sx*(1.6800)},{\sy*(0.0512)})
	--({\sx*(1.6900)},{\sy*(0.0553)})
	--({\sx*(1.7000)},{\sy*(0.0594)})
	--({\sx*(1.7100)},{\sy*(0.0635)})
	--({\sx*(1.7200)},{\sy*(0.0676)})
	--({\sx*(1.7300)},{\sy*(0.0716)})
	--({\sx*(1.7400)},{\sy*(0.0757)})
	--({\sx*(1.7500)},{\sy*(0.0797)})
	--({\sx*(1.7600)},{\sy*(0.0836)})
	--({\sx*(1.7700)},{\sy*(0.0876)})
	--({\sx*(1.7800)},{\sy*(0.0915)})
	--({\sx*(1.7900)},{\sy*(0.0954)})
	--({\sx*(1.8000)},{\sy*(0.0992)})
	--({\sx*(1.8100)},{\sy*(0.1030)})
	--({\sx*(1.8200)},{\sy*(0.1068)})
	--({\sx*(1.8300)},{\sy*(0.1105)})
	--({\sx*(1.8400)},{\sy*(0.1142)})
	--({\sx*(1.8500)},{\sy*(0.1178)})
	--({\sx*(1.8600)},{\sy*(0.1214)})
	--({\sx*(1.8700)},{\sy*(0.1250)})
	--({\sx*(1.8800)},{\sy*(0.1285)})
	--({\sx*(1.8900)},{\sy*(0.1319)})
	--({\sx*(1.9000)},{\sy*(0.1353)})
	--({\sx*(1.9100)},{\sy*(0.1386)})
	--({\sx*(1.9200)},{\sy*(0.1419)})
	--({\sx*(1.9300)},{\sy*(0.1451)})
	--({\sx*(1.9400)},{\sy*(0.1482)})
	--({\sx*(1.9500)},{\sy*(0.1513)})
	--({\sx*(1.9600)},{\sy*(0.1543)})
	--({\sx*(1.9700)},{\sy*(0.1572)})
	--({\sx*(1.9800)},{\sy*(0.1601)})
	--({\sx*(1.9900)},{\sy*(0.1628)})
	--({\sx*(2.0000)},{\sy*(0.1655)})
	--({\sx*(2.0100)},{\sy*(0.1681)})
	--({\sx*(2.0200)},{\sy*(0.1706)})
	--({\sx*(2.0300)},{\sy*(0.1730)})
	--({\sx*(2.0400)},{\sy*(0.1754)})
	--({\sx*(2.0500)},{\sy*(0.1776)})
	--({\sx*(2.0600)},{\sy*(0.1797)})
	--({\sx*(2.0700)},{\sy*(0.1817)})
	--({\sx*(2.0800)},{\sy*(0.1836)})
	--({\sx*(2.0900)},{\sy*(0.1854)})
	--({\sx*(2.1000)},{\sy*(0.1870)})
	--({\sx*(2.1100)},{\sy*(0.1885)})
	--({\sx*(2.1200)},{\sy*(0.1899)})
	--({\sx*(2.1300)},{\sy*(0.1912)})
	--({\sx*(2.1400)},{\sy*(0.1922)})
	--({\sx*(2.1500)},{\sy*(0.1932)})
	--({\sx*(2.1600)},{\sy*(0.1940)})
	--({\sx*(2.1700)},{\sy*(0.1946)})
	--({\sx*(2.1800)},{\sy*(0.1950)})
	--({\sx*(2.1900)},{\sy*(0.1953)})
	--({\sx*(2.2000)},{\sy*(0.1953)})
	--({\sx*(2.2100)},{\sy*(0.1952)})
	--({\sx*(2.2200)},{\sy*(0.1948)})
	--({\sx*(2.2300)},{\sy*(0.1942)})
	--({\sx*(2.2400)},{\sy*(0.1934)})
	--({\sx*(2.2500)},{\sy*(0.1923)})
	--({\sx*(2.2600)},{\sy*(0.1909)})
	--({\sx*(2.2700)},{\sy*(0.1893)})
	--({\sx*(2.2800)},{\sy*(0.1873)})
	--({\sx*(2.2900)},{\sy*(0.1851)})
	--({\sx*(2.3000)},{\sy*(0.1825)})
	--({\sx*(2.3100)},{\sy*(0.1795)})
	--({\sx*(2.3200)},{\sy*(0.1762)})
	--({\sx*(2.3300)},{\sy*(0.1725)})
	--({\sx*(2.3400)},{\sy*(0.1683)})
	--({\sx*(2.3500)},{\sy*(0.1636)})
	--({\sx*(2.3600)},{\sy*(0.1585)})
	--({\sx*(2.3700)},{\sy*(0.1528)})
	--({\sx*(2.3800)},{\sy*(0.1465)})
	--({\sx*(2.3900)},{\sy*(0.1395)})
	--({\sx*(2.4000)},{\sy*(0.1319)})
	--({\sx*(2.4100)},{\sy*(0.1236)})
	--({\sx*(2.4200)},{\sy*(0.1144)})
	--({\sx*(2.4300)},{\sy*(0.1043)})
	--({\sx*(2.4400)},{\sy*(0.0933)})
	--({\sx*(2.4500)},{\sy*(0.0812)})
	--({\sx*(2.4600)},{\sy*(0.0679)})
	--({\sx*(2.4700)},{\sy*(0.0533)})
	--({\sx*(2.4800)},{\sy*(0.0372)})
	--({\sx*(2.4900)},{\sy*(0.0195)})
	--({\sx*(2.5000)},{\sy*(0.0000)})
	--({\sx*(2.5100)},{\sy*(-0.0216)})
	--({\sx*(2.5200)},{\sy*(-0.0455)})
	--({\sx*(2.5300)},{\sy*(-0.0721)})
	--({\sx*(2.5400)},{\sy*(-0.1018)})
	--({\sx*(2.5500)},{\sy*(-0.1350)})
	--({\sx*(2.5600)},{\sy*(-0.1723)})
	--({\sx*(2.5700)},{\sy*(-0.2145)})
	--({\sx*(2.5800)},{\sy*(-0.2624)})
	--({\sx*(2.5900)},{\sy*(-0.3172)})
	--({\sx*(2.6000)},{\sy*(-0.3803)})
	--({\sx*(2.6100)},{\sy*(-0.4536)})
	--({\sx*(2.6200)},{\sy*(-0.5397)})
	--({\sx*(2.6300)},{\sy*(-0.6421)})
	--({\sx*(2.6400)},{\sy*(-0.7654)})
	--({\sx*(2.6500)},{\sy*(-0.9167)})
	--({\sx*(2.6600)},{\sy*(-1.1063)})
	--({\sx*(2.6700)},{\sy*(-1.3504)})
	--({\sx*(2.6800)},{\sy*(-1.6759)})
	--({\sx*(2.6900)},{\sy*(-2.1308)})
	--({\sx*(2.7000)},{\sy*(-2.8101)})
	--({\sx*(2.7100)},{\sy*(-3.9319)})
	--({\sx*(2.7200)},{\sy*(-6.1330)})
	--({\sx*(2.7300)},{\sy*(-12.3973)})
	--({\sx*(2.7400)},{\sy*(-166.0998)})
	--({\sx*(2.7500)},{\sy*(16.1736)})
	--({\sx*(2.7600)},{\sy*(8.0844)})
	--({\sx*(2.7700)},{\sy*(5.5475)})
	--({\sx*(2.7800)},{\sy*(4.3081)})
	--({\sx*(2.7900)},{\sy*(3.5742)})
	--({\sx*(2.8000)},{\sy*(3.0895)})
	--({\sx*(2.8100)},{\sy*(2.7458)})
	--({\sx*(2.8200)},{\sy*(2.4897)})
	--({\sx*(2.8300)},{\sy*(2.2917)})
	--({\sx*(2.8400)},{\sy*(2.1342)})
	--({\sx*(2.8500)},{\sy*(2.0060)})
	--({\sx*(2.8600)},{\sy*(1.8999)})
	--({\sx*(2.8700)},{\sy*(1.8105)})
	--({\sx*(2.8800)},{\sy*(1.7344)})
	--({\sx*(2.8900)},{\sy*(1.6688)})
	--({\sx*(2.9000)},{\sy*(1.6118)})
	--({\sx*(2.9100)},{\sy*(1.5618)})
	--({\sx*(2.9200)},{\sy*(1.5177)})
	--({\sx*(2.9300)},{\sy*(1.4785)})
	--({\sx*(2.9400)},{\sy*(1.4435)})
	--({\sx*(2.9500)},{\sy*(1.4120)})
	--({\sx*(2.9600)},{\sy*(1.3836)})
	--({\sx*(2.9700)},{\sy*(1.3579)})
	--({\sx*(2.9800)},{\sy*(1.3346)})
	--({\sx*(2.9900)},{\sy*(1.3132)})
	--({\sx*(3.0000)},{\sy*(1.2937)})
	--({\sx*(3.0100)},{\sy*(1.2758)})
	--({\sx*(3.0200)},{\sy*(1.2593)})
	--({\sx*(3.0300)},{\sy*(1.2442)})
	--({\sx*(3.0400)},{\sy*(1.2301)})
	--({\sx*(3.0500)},{\sy*(1.2171)})
	--({\sx*(3.0600)},{\sy*(1.2050)})
	--({\sx*(3.0700)},{\sy*(1.1938)})
	--({\sx*(3.0800)},{\sy*(1.1833)})
	--({\sx*(3.0900)},{\sy*(1.1736)})
	--({\sx*(3.1000)},{\sy*(1.1645)})
	--({\sx*(3.1100)},{\sy*(1.1559)})
	--({\sx*(3.1200)},{\sy*(1.1479)})
	--({\sx*(3.1300)},{\sy*(1.1405)})
	--({\sx*(3.1400)},{\sy*(1.1334)})
	--({\sx*(3.1500)},{\sy*(1.1268)})
	--({\sx*(3.1600)},{\sy*(1.1206)})
	--({\sx*(3.1700)},{\sy*(1.1147)})
	--({\sx*(3.1800)},{\sy*(1.1092)})
	--({\sx*(3.1900)},{\sy*(1.1040)})
	--({\sx*(3.2000)},{\sy*(1.0991)})
	--({\sx*(3.2100)},{\sy*(1.0944)})
	--({\sx*(3.2200)},{\sy*(1.0901)})
	--({\sx*(3.2300)},{\sy*(1.0859)})
	--({\sx*(3.2400)},{\sy*(1.0820)})
	--({\sx*(3.2500)},{\sy*(1.0782)})
	--({\sx*(3.2600)},{\sy*(1.0747)})
	--({\sx*(3.2700)},{\sy*(1.0713)})
	--({\sx*(3.2800)},{\sy*(1.0681)})
	--({\sx*(3.2900)},{\sy*(1.0651)})
	--({\sx*(3.3000)},{\sy*(1.0623)})
	--({\sx*(3.3100)},{\sy*(1.0595)})
	--({\sx*(3.3200)},{\sy*(1.0569)})
	--({\sx*(3.3300)},{\sy*(1.0544)})
	--({\sx*(3.3400)},{\sy*(1.0521)})
	--({\sx*(3.3500)},{\sy*(1.0499)})
	--({\sx*(3.3600)},{\sy*(1.0477)})
	--({\sx*(3.3700)},{\sy*(1.0457)})
	--({\sx*(3.3800)},{\sy*(1.0437)})
	--({\sx*(3.3900)},{\sy*(1.0419)})
	--({\sx*(3.4000)},{\sy*(1.0401)})
	--({\sx*(3.4100)},{\sy*(1.0385)})
	--({\sx*(3.4200)},{\sy*(1.0368)})
	--({\sx*(3.4300)},{\sy*(1.0353)})
	--({\sx*(3.4400)},{\sy*(1.0338)})
	--({\sx*(3.4500)},{\sy*(1.0324)})
	--({\sx*(3.4600)},{\sy*(1.0311)})
	--({\sx*(3.4700)},{\sy*(1.0298)})
	--({\sx*(3.4800)},{\sy*(1.0286)})
	--({\sx*(3.4900)},{\sy*(1.0274)})
	--({\sx*(3.5000)},{\sy*(1.0263)})
	--({\sx*(3.5100)},{\sy*(1.0253)})
	--({\sx*(3.5200)},{\sy*(1.0242)})
	--({\sx*(3.5300)},{\sy*(1.0233)})
	--({\sx*(3.5400)},{\sy*(1.0223)})
	--({\sx*(3.5500)},{\sy*(1.0214)})
	--({\sx*(3.5600)},{\sy*(1.0206)})
	--({\sx*(3.5700)},{\sy*(1.0197)})
	--({\sx*(3.5800)},{\sy*(1.0190)})
	--({\sx*(3.5900)},{\sy*(1.0182)})
	--({\sx*(3.6000)},{\sy*(1.0175)})
	--({\sx*(3.6100)},{\sy*(1.0168)})
	--({\sx*(3.6200)},{\sy*(1.0161)})
	--({\sx*(3.6300)},{\sy*(1.0155)})
	--({\sx*(3.6400)},{\sy*(1.0149)})
	--({\sx*(3.6500)},{\sy*(1.0143)})
	--({\sx*(3.6600)},{\sy*(1.0137)})
	--({\sx*(3.6700)},{\sy*(1.0132)})
	--({\sx*(3.6800)},{\sy*(1.0127)})
	--({\sx*(3.6900)},{\sy*(1.0122)})
	--({\sx*(3.7000)},{\sy*(1.0117)})
	--({\sx*(3.7100)},{\sy*(1.0112)})
	--({\sx*(3.7200)},{\sy*(1.0108)})
	--({\sx*(3.7300)},{\sy*(1.0104)})
	--({\sx*(3.7400)},{\sy*(1.0100)})
	--({\sx*(3.7500)},{\sy*(1.0096)})
	--({\sx*(3.7600)},{\sy*(1.0092)})
	--({\sx*(3.7700)},{\sy*(1.0089)})
	--({\sx*(3.7800)},{\sy*(1.0085)})
	--({\sx*(3.7900)},{\sy*(1.0082)})
	--({\sx*(3.8000)},{\sy*(1.0079)})
	--({\sx*(3.8100)},{\sy*(1.0076)})
	--({\sx*(3.8200)},{\sy*(1.0073)})
	--({\sx*(3.8300)},{\sy*(1.0070)})
	--({\sx*(3.8400)},{\sy*(1.0068)})
	--({\sx*(3.8500)},{\sy*(1.0065)})
	--({\sx*(3.8600)},{\sy*(1.0062)})
	--({\sx*(3.8700)},{\sy*(1.0060)})
	--({\sx*(3.8800)},{\sy*(1.0058)})
	--({\sx*(3.8900)},{\sy*(1.0056)})
	--({\sx*(3.9000)},{\sy*(1.0054)})
	--({\sx*(3.9100)},{\sy*(1.0051)})
	--({\sx*(3.9200)},{\sy*(1.0050)})
	--({\sx*(3.9300)},{\sy*(1.0048)})
	--({\sx*(3.9400)},{\sy*(1.0046)})
	--({\sx*(3.9500)},{\sy*(1.0044)})
	--({\sx*(3.9600)},{\sy*(1.0043)})
	--({\sx*(3.9700)},{\sy*(1.0041)})
	--({\sx*(3.9800)},{\sy*(1.0039)})
	--({\sx*(3.9900)},{\sy*(1.0038)})
	--({\sx*(4.0000)},{\sy*(1.0037)})
	--({\sx*(4.0100)},{\sy*(1.0035)})
	--({\sx*(4.0200)},{\sy*(1.0034)})
	--({\sx*(4.0300)},{\sy*(1.0033)})
	--({\sx*(4.0400)},{\sy*(1.0031)})
	--({\sx*(4.0500)},{\sy*(1.0030)})
	--({\sx*(4.0600)},{\sy*(1.0029)})
	--({\sx*(4.0700)},{\sy*(1.0028)})
	--({\sx*(4.0800)},{\sy*(1.0027)})
	--({\sx*(4.0900)},{\sy*(1.0026)})
	--({\sx*(4.1000)},{\sy*(1.0025)})
	--({\sx*(4.1100)},{\sy*(1.0024)})
	--({\sx*(4.1200)},{\sy*(1.0023)})
	--({\sx*(4.1300)},{\sy*(1.0022)})
	--({\sx*(4.1400)},{\sy*(1.0022)})
	--({\sx*(4.1500)},{\sy*(1.0021)})
	--({\sx*(4.1600)},{\sy*(1.0020)})
	--({\sx*(4.1700)},{\sy*(1.0019)})
	--({\sx*(4.1800)},{\sy*(1.0019)})
	--({\sx*(4.1900)},{\sy*(1.0018)})
	--({\sx*(4.2000)},{\sy*(1.0017)})
	--({\sx*(4.2100)},{\sy*(1.0017)})
	--({\sx*(4.2200)},{\sy*(1.0016)})
	--({\sx*(4.2300)},{\sy*(1.0016)})
	--({\sx*(4.2400)},{\sy*(1.0015)})
	--({\sx*(4.2500)},{\sy*(1.0014)})
	--({\sx*(4.2600)},{\sy*(1.0014)})
	--({\sx*(4.2700)},{\sy*(1.0013)})
	--({\sx*(4.2800)},{\sy*(1.0013)})
	--({\sx*(4.2900)},{\sy*(1.0013)})
	--({\sx*(4.3000)},{\sy*(1.0012)})
	--({\sx*(4.3100)},{\sy*(1.0012)})
	--({\sx*(4.3200)},{\sy*(1.0011)})
	--({\sx*(4.3300)},{\sy*(1.0011)})
	--({\sx*(4.3400)},{\sy*(1.0011)})
	--({\sx*(4.3500)},{\sy*(1.0010)})
	--({\sx*(4.3600)},{\sy*(1.0010)})
	--({\sx*(4.3700)},{\sy*(1.0010)})
	--({\sx*(4.3800)},{\sy*(1.0009)})
	--({\sx*(4.3900)},{\sy*(1.0009)})
	--({\sx*(4.4000)},{\sy*(1.0009)})
	--({\sx*(4.4100)},{\sy*(1.0008)})
	--({\sx*(4.4200)},{\sy*(1.0008)})
	--({\sx*(4.4300)},{\sy*(1.0008)})
	--({\sx*(4.4400)},{\sy*(1.0008)})
	--({\sx*(4.4500)},{\sy*(1.0007)})
	--({\sx*(4.4600)},{\sy*(1.0007)})
	--({\sx*(4.4700)},{\sy*(1.0007)})
	--({\sx*(4.4800)},{\sy*(1.0007)})
	--({\sx*(4.4900)},{\sy*(1.0006)})
	--({\sx*(4.5000)},{\sy*(1.0006)})
	--({\sx*(4.5100)},{\sy*(1.0006)})
	--({\sx*(4.5200)},{\sy*(1.0006)})
	--({\sx*(4.5300)},{\sy*(1.0006)})
	--({\sx*(4.5400)},{\sy*(1.0006)})
	--({\sx*(4.5500)},{\sy*(1.0005)})
	--({\sx*(4.5600)},{\sy*(1.0005)})
	--({\sx*(4.5700)},{\sy*(1.0005)})
	--({\sx*(4.5800)},{\sy*(1.0005)})
	--({\sx*(4.5900)},{\sy*(1.0005)})
	--({\sx*(4.6000)},{\sy*(1.0005)})
	--({\sx*(4.6100)},{\sy*(1.0005)})
	--({\sx*(4.6200)},{\sy*(1.0004)})
	--({\sx*(4.6300)},{\sy*(1.0004)})
	--({\sx*(4.6400)},{\sy*(1.0004)})
	--({\sx*(4.6500)},{\sy*(1.0004)})
	--({\sx*(4.6600)},{\sy*(1.0004)})
	--({\sx*(4.6700)},{\sy*(1.0004)})
	--({\sx*(4.6800)},{\sy*(1.0004)})
	--({\sx*(4.6900)},{\sy*(1.0004)})
	--({\sx*(4.7000)},{\sy*(1.0004)})
	--({\sx*(4.7100)},{\sy*(1.0004)})
	--({\sx*(4.7200)},{\sy*(1.0004)})
	--({\sx*(4.7300)},{\sy*(1.0004)})
	--({\sx*(4.7400)},{\sy*(1.0003)})
	--({\sx*(4.7500)},{\sy*(1.0003)})
	--({\sx*(4.7600)},{\sy*(1.0003)})
	--({\sx*(4.7700)},{\sy*(1.0003)})
	--({\sx*(4.7800)},{\sy*(1.0003)})
	--({\sx*(4.7900)},{\sy*(1.0003)})
	--({\sx*(4.8000)},{\sy*(1.0003)})
	--({\sx*(4.8100)},{\sy*(1.0003)})
	--({\sx*(4.8200)},{\sy*(1.0003)})
	--({\sx*(4.8300)},{\sy*(1.0003)})
	--({\sx*(4.8400)},{\sy*(1.0003)})
	--({\sx*(4.8500)},{\sy*(1.0003)})
	--({\sx*(4.8600)},{\sy*(1.0003)})
	--({\sx*(4.8700)},{\sy*(1.0004)})
	--({\sx*(4.8800)},{\sy*(1.0004)})
	--({\sx*(4.8900)},{\sy*(1.0004)})
	--({\sx*(4.9000)},{\sy*(1.0004)})
	--({\sx*(4.9100)},{\sy*(1.0004)})
	--({\sx*(4.9200)},{\sy*(1.0004)})
	--({\sx*(4.9300)},{\sy*(1.0005)})
	--({\sx*(4.9400)},{\sy*(1.0005)})
	--({\sx*(4.9500)},{\sy*(1.0006)})
	--({\sx*(4.9600)},{\sy*(1.0007)})
	--({\sx*(4.9700)},{\sy*(1.0009)})
	--({\sx*(4.9800)},{\sy*(1.0013)})
	--({\sx*(4.9900)},{\sy*(1.0024)})
	--({\sx*(5.0000)},{\sy*(0.0000)});
}
\def\xwerteb{
\fill[color=red] (0.0000,0) circle[radius={0.07/\skala}];
\fill[color=red] (1.2500,0) circle[radius={0.07/\skala}];
\fill[color=red] (2.5000,0) circle[radius={0.07/\skala}];
\fill[color=red] (3.7500,0) circle[radius={0.07/\skala}];
\fill[color=red] (5.0000,0) circle[radius={0.07/\skala}];
}
\def\punkteb{4}
\def\maxfehlerb{3.011\cdot 10^{-2}}
\def\fehlerb{
\draw[color=red,line width=1.4pt,line join=round] ({\sx*(0.000)},{\sy*(0.0000)})
	--({\sx*(0.0100)},{\sy*(-0.0402)})
	--({\sx*(0.0200)},{\sy*(-0.0795)})
	--({\sx*(0.0300)},{\sy*(-0.1181)})
	--({\sx*(0.0400)},{\sy*(-0.1559)})
	--({\sx*(0.0500)},{\sy*(-0.1928)})
	--({\sx*(0.0600)},{\sy*(-0.2290)})
	--({\sx*(0.0700)},{\sy*(-0.2643)})
	--({\sx*(0.0800)},{\sy*(-0.2988)})
	--({\sx*(0.0900)},{\sy*(-0.3325)})
	--({\sx*(0.1000)},{\sy*(-0.3653)})
	--({\sx*(0.1100)},{\sy*(-0.3973)})
	--({\sx*(0.1200)},{\sy*(-0.4284)})
	--({\sx*(0.1300)},{\sy*(-0.4588)})
	--({\sx*(0.1400)},{\sy*(-0.4882)})
	--({\sx*(0.1500)},{\sy*(-0.5168)})
	--({\sx*(0.1600)},{\sy*(-0.5446)})
	--({\sx*(0.1700)},{\sy*(-0.5715)})
	--({\sx*(0.1800)},{\sy*(-0.5976)})
	--({\sx*(0.1900)},{\sy*(-0.6228)})
	--({\sx*(0.2000)},{\sy*(-0.6472)})
	--({\sx*(0.2100)},{\sy*(-0.6707)})
	--({\sx*(0.2200)},{\sy*(-0.6934)})
	--({\sx*(0.2300)},{\sy*(-0.7153)})
	--({\sx*(0.2400)},{\sy*(-0.7362)})
	--({\sx*(0.2500)},{\sy*(-0.7564)})
	--({\sx*(0.2600)},{\sy*(-0.7757)})
	--({\sx*(0.2700)},{\sy*(-0.7942)})
	--({\sx*(0.2800)},{\sy*(-0.8118)})
	--({\sx*(0.2900)},{\sy*(-0.8286)})
	--({\sx*(0.3000)},{\sy*(-0.8446)})
	--({\sx*(0.3100)},{\sy*(-0.8598)})
	--({\sx*(0.3200)},{\sy*(-0.8741)})
	--({\sx*(0.3300)},{\sy*(-0.8876)})
	--({\sx*(0.3400)},{\sy*(-0.9004)})
	--({\sx*(0.3500)},{\sy*(-0.9123)})
	--({\sx*(0.3600)},{\sy*(-0.9234)})
	--({\sx*(0.3700)},{\sy*(-0.9337)})
	--({\sx*(0.3800)},{\sy*(-0.9433)})
	--({\sx*(0.3900)},{\sy*(-0.9521)})
	--({\sx*(0.4000)},{\sy*(-0.9601)})
	--({\sx*(0.4100)},{\sy*(-0.9674)})
	--({\sx*(0.4200)},{\sy*(-0.9739)})
	--({\sx*(0.4300)},{\sy*(-0.9796)})
	--({\sx*(0.4400)},{\sy*(-0.9847)})
	--({\sx*(0.4500)},{\sy*(-0.9890)})
	--({\sx*(0.4600)},{\sy*(-0.9925)})
	--({\sx*(0.4700)},{\sy*(-0.9954)})
	--({\sx*(0.4800)},{\sy*(-0.9976)})
	--({\sx*(0.4900)},{\sy*(-0.9991)})
	--({\sx*(0.5000)},{\sy*(-0.9999)})
	--({\sx*(0.5100)},{\sy*(-1.0000)})
	--({\sx*(0.5200)},{\sy*(-0.9995)})
	--({\sx*(0.5300)},{\sy*(-0.9983)})
	--({\sx*(0.5400)},{\sy*(-0.9965)})
	--({\sx*(0.5500)},{\sy*(-0.9940)})
	--({\sx*(0.5600)},{\sy*(-0.9910)})
	--({\sx*(0.5700)},{\sy*(-0.9873)})
	--({\sx*(0.5800)},{\sy*(-0.9830)})
	--({\sx*(0.5900)},{\sy*(-0.9782)})
	--({\sx*(0.6000)},{\sy*(-0.9728)})
	--({\sx*(0.6100)},{\sy*(-0.9668)})
	--({\sx*(0.6200)},{\sy*(-0.9603)})
	--({\sx*(0.6300)},{\sy*(-0.9532)})
	--({\sx*(0.6400)},{\sy*(-0.9457)})
	--({\sx*(0.6500)},{\sy*(-0.9376)})
	--({\sx*(0.6600)},{\sy*(-0.9290)})
	--({\sx*(0.6700)},{\sy*(-0.9199)})
	--({\sx*(0.6800)},{\sy*(-0.9104)})
	--({\sx*(0.6900)},{\sy*(-0.9004)})
	--({\sx*(0.7000)},{\sy*(-0.8899)})
	--({\sx*(0.7100)},{\sy*(-0.8790)})
	--({\sx*(0.7200)},{\sy*(-0.8677)})
	--({\sx*(0.7300)},{\sy*(-0.8560)})
	--({\sx*(0.7400)},{\sy*(-0.8439)})
	--({\sx*(0.7500)},{\sy*(-0.8314)})
	--({\sx*(0.7600)},{\sy*(-0.8186)})
	--({\sx*(0.7700)},{\sy*(-0.8054)})
	--({\sx*(0.7800)},{\sy*(-0.7918)})
	--({\sx*(0.7900)},{\sy*(-0.7779)})
	--({\sx*(0.8000)},{\sy*(-0.7637)})
	--({\sx*(0.8100)},{\sy*(-0.7492)})
	--({\sx*(0.8200)},{\sy*(-0.7344)})
	--({\sx*(0.8300)},{\sy*(-0.7193)})
	--({\sx*(0.8400)},{\sy*(-0.7040)})
	--({\sx*(0.8500)},{\sy*(-0.6884)})
	--({\sx*(0.8600)},{\sy*(-0.6726)})
	--({\sx*(0.8700)},{\sy*(-0.6566)})
	--({\sx*(0.8800)},{\sy*(-0.6403)})
	--({\sx*(0.8900)},{\sy*(-0.6238)})
	--({\sx*(0.9000)},{\sy*(-0.6072)})
	--({\sx*(0.9100)},{\sy*(-0.5904)})
	--({\sx*(0.9200)},{\sy*(-0.5734)})
	--({\sx*(0.9300)},{\sy*(-0.5563)})
	--({\sx*(0.9400)},{\sy*(-0.5390)})
	--({\sx*(0.9500)},{\sy*(-0.5217)})
	--({\sx*(0.9600)},{\sy*(-0.5042)})
	--({\sx*(0.9700)},{\sy*(-0.4866)})
	--({\sx*(0.9800)},{\sy*(-0.4689)})
	--({\sx*(0.9900)},{\sy*(-0.4512)})
	--({\sx*(1.0000)},{\sy*(-0.4334)})
	--({\sx*(1.0100)},{\sy*(-0.4155)})
	--({\sx*(1.0200)},{\sy*(-0.3976)})
	--({\sx*(1.0300)},{\sy*(-0.3797)})
	--({\sx*(1.0400)},{\sy*(-0.3618)})
	--({\sx*(1.0500)},{\sy*(-0.3438)})
	--({\sx*(1.0600)},{\sy*(-0.3259)})
	--({\sx*(1.0700)},{\sy*(-0.3080)})
	--({\sx*(1.0800)},{\sy*(-0.2901)})
	--({\sx*(1.0900)},{\sy*(-0.2722)})
	--({\sx*(1.1000)},{\sy*(-0.2544)})
	--({\sx*(1.1100)},{\sy*(-0.2367)})
	--({\sx*(1.1200)},{\sy*(-0.2190)})
	--({\sx*(1.1300)},{\sy*(-0.2014)})
	--({\sx*(1.1400)},{\sy*(-0.1839)})
	--({\sx*(1.1500)},{\sy*(-0.1665)})
	--({\sx*(1.1600)},{\sy*(-0.1493)})
	--({\sx*(1.1700)},{\sy*(-0.1321)})
	--({\sx*(1.1800)},{\sy*(-0.1150)})
	--({\sx*(1.1900)},{\sy*(-0.0981)})
	--({\sx*(1.2000)},{\sy*(-0.0814)})
	--({\sx*(1.2100)},{\sy*(-0.0648)})
	--({\sx*(1.2200)},{\sy*(-0.0483)})
	--({\sx*(1.2300)},{\sy*(-0.0320)})
	--({\sx*(1.2400)},{\sy*(-0.0159)})
	--({\sx*(1.2500)},{\sy*(0.0000)})
	--({\sx*(1.2600)},{\sy*(0.0157)})
	--({\sx*(1.2700)},{\sy*(0.0313)})
	--({\sx*(1.2800)},{\sy*(0.0466)})
	--({\sx*(1.2900)},{\sy*(0.0617)})
	--({\sx*(1.3000)},{\sy*(0.0766)})
	--({\sx*(1.3100)},{\sy*(0.0912)})
	--({\sx*(1.3200)},{\sy*(0.1057)})
	--({\sx*(1.3300)},{\sy*(0.1199)})
	--({\sx*(1.3400)},{\sy*(0.1338)})
	--({\sx*(1.3500)},{\sy*(0.1475)})
	--({\sx*(1.3600)},{\sy*(0.1610)})
	--({\sx*(1.3700)},{\sy*(0.1741)})
	--({\sx*(1.3800)},{\sy*(0.1871)})
	--({\sx*(1.3900)},{\sy*(0.1997)})
	--({\sx*(1.4000)},{\sy*(0.2121)})
	--({\sx*(1.4100)},{\sy*(0.2242)})
	--({\sx*(1.4200)},{\sy*(0.2360)})
	--({\sx*(1.4300)},{\sy*(0.2475)})
	--({\sx*(1.4400)},{\sy*(0.2588)})
	--({\sx*(1.4500)},{\sy*(0.2697)})
	--({\sx*(1.4600)},{\sy*(0.2804)})
	--({\sx*(1.4700)},{\sy*(0.2907)})
	--({\sx*(1.4800)},{\sy*(0.3008)})
	--({\sx*(1.4900)},{\sy*(0.3105)})
	--({\sx*(1.5000)},{\sy*(0.3200)})
	--({\sx*(1.5100)},{\sy*(0.3291)})
	--({\sx*(1.5200)},{\sy*(0.3379)})
	--({\sx*(1.5300)},{\sy*(0.3465)})
	--({\sx*(1.5400)},{\sy*(0.3547)})
	--({\sx*(1.5500)},{\sy*(0.3625)})
	--({\sx*(1.5600)},{\sy*(0.3701)})
	--({\sx*(1.5700)},{\sy*(0.3773)})
	--({\sx*(1.5800)},{\sy*(0.3843)})
	--({\sx*(1.5900)},{\sy*(0.3909)})
	--({\sx*(1.6000)},{\sy*(0.3972)})
	--({\sx*(1.6100)},{\sy*(0.4032)})
	--({\sx*(1.6200)},{\sy*(0.4088)})
	--({\sx*(1.6300)},{\sy*(0.4142)})
	--({\sx*(1.6400)},{\sy*(0.4192)})
	--({\sx*(1.6500)},{\sy*(0.4239)})
	--({\sx*(1.6600)},{\sy*(0.4283)})
	--({\sx*(1.6700)},{\sy*(0.4323)})
	--({\sx*(1.6800)},{\sy*(0.4361)})
	--({\sx*(1.6900)},{\sy*(0.4395)})
	--({\sx*(1.7000)},{\sy*(0.4426)})
	--({\sx*(1.7100)},{\sy*(0.4454)})
	--({\sx*(1.7200)},{\sy*(0.4479)})
	--({\sx*(1.7300)},{\sy*(0.4501)})
	--({\sx*(1.7400)},{\sy*(0.4520)})
	--({\sx*(1.7500)},{\sy*(0.4536)})
	--({\sx*(1.7600)},{\sy*(0.4549)})
	--({\sx*(1.7700)},{\sy*(0.4559)})
	--({\sx*(1.7800)},{\sy*(0.4566)})
	--({\sx*(1.7900)},{\sy*(0.4570)})
	--({\sx*(1.8000)},{\sy*(0.4571)})
	--({\sx*(1.8100)},{\sy*(0.4569)})
	--({\sx*(1.8200)},{\sy*(0.4564)})
	--({\sx*(1.8300)},{\sy*(0.4557)})
	--({\sx*(1.8400)},{\sy*(0.4546)})
	--({\sx*(1.8500)},{\sy*(0.4533)})
	--({\sx*(1.8600)},{\sy*(0.4517)})
	--({\sx*(1.8700)},{\sy*(0.4499)})
	--({\sx*(1.8800)},{\sy*(0.4478)})
	--({\sx*(1.8900)},{\sy*(0.4454)})
	--({\sx*(1.9000)},{\sy*(0.4428)})
	--({\sx*(1.9100)},{\sy*(0.4399)})
	--({\sx*(1.9200)},{\sy*(0.4368)})
	--({\sx*(1.9300)},{\sy*(0.4334)})
	--({\sx*(1.9400)},{\sy*(0.4298)})
	--({\sx*(1.9500)},{\sy*(0.4260)})
	--({\sx*(1.9600)},{\sy*(0.4219)})
	--({\sx*(1.9700)},{\sy*(0.4176)})
	--({\sx*(1.9800)},{\sy*(0.4131)})
	--({\sx*(1.9900)},{\sy*(0.4083)})
	--({\sx*(2.0000)},{\sy*(0.4034)})
	--({\sx*(2.0100)},{\sy*(0.3982)})
	--({\sx*(2.0200)},{\sy*(0.3929)})
	--({\sx*(2.0300)},{\sy*(0.3873)})
	--({\sx*(2.0400)},{\sy*(0.3816)})
	--({\sx*(2.0500)},{\sy*(0.3756)})
	--({\sx*(2.0600)},{\sy*(0.3695)})
	--({\sx*(2.0700)},{\sy*(0.3632)})
	--({\sx*(2.0800)},{\sy*(0.3567)})
	--({\sx*(2.0900)},{\sy*(0.3501)})
	--({\sx*(2.1000)},{\sy*(0.3433)})
	--({\sx*(2.1100)},{\sy*(0.3364)})
	--({\sx*(2.1200)},{\sy*(0.3293)})
	--({\sx*(2.1300)},{\sy*(0.3220)})
	--({\sx*(2.1400)},{\sy*(0.3146)})
	--({\sx*(2.1500)},{\sy*(0.3071)})
	--({\sx*(2.1600)},{\sy*(0.2994)})
	--({\sx*(2.1700)},{\sy*(0.2916)})
	--({\sx*(2.1800)},{\sy*(0.2837)})
	--({\sx*(2.1900)},{\sy*(0.2757)})
	--({\sx*(2.2000)},{\sy*(0.2676)})
	--({\sx*(2.2100)},{\sy*(0.2594)})
	--({\sx*(2.2200)},{\sy*(0.2511)})
	--({\sx*(2.2300)},{\sy*(0.2427)})
	--({\sx*(2.2400)},{\sy*(0.2342)})
	--({\sx*(2.2500)},{\sy*(0.2256)})
	--({\sx*(2.2600)},{\sy*(0.2169)})
	--({\sx*(2.2700)},{\sy*(0.2082)})
	--({\sx*(2.2800)},{\sy*(0.1994)})
	--({\sx*(2.2900)},{\sy*(0.1905)})
	--({\sx*(2.3000)},{\sy*(0.1816)})
	--({\sx*(2.3100)},{\sy*(0.1727)})
	--({\sx*(2.3200)},{\sy*(0.1637)})
	--({\sx*(2.3300)},{\sy*(0.1546)})
	--({\sx*(2.3400)},{\sy*(0.1456)})
	--({\sx*(2.3500)},{\sy*(0.1365)})
	--({\sx*(2.3600)},{\sy*(0.1273)})
	--({\sx*(2.3700)},{\sy*(0.1182)})
	--({\sx*(2.3800)},{\sy*(0.1091)})
	--({\sx*(2.3900)},{\sy*(0.0999)})
	--({\sx*(2.4000)},{\sy*(0.0907)})
	--({\sx*(2.4100)},{\sy*(0.0816)})
	--({\sx*(2.4200)},{\sy*(0.0724)})
	--({\sx*(2.4300)},{\sy*(0.0633)})
	--({\sx*(2.4400)},{\sy*(0.0541)})
	--({\sx*(2.4500)},{\sy*(0.0450)})
	--({\sx*(2.4600)},{\sy*(0.0360)})
	--({\sx*(2.4700)},{\sy*(0.0269)})
	--({\sx*(2.4800)},{\sy*(0.0179)})
	--({\sx*(2.4900)},{\sy*(0.0089)})
	--({\sx*(2.5000)},{\sy*(0.0000)})
	--({\sx*(2.5100)},{\sy*(-0.0089)})
	--({\sx*(2.5200)},{\sy*(-0.0177)})
	--({\sx*(2.5300)},{\sy*(-0.0265)})
	--({\sx*(2.5400)},{\sy*(-0.0352)})
	--({\sx*(2.5500)},{\sy*(-0.0438)})
	--({\sx*(2.5600)},{\sy*(-0.0524)})
	--({\sx*(2.5700)},{\sy*(-0.0609)})
	--({\sx*(2.5800)},{\sy*(-0.0693)})
	--({\sx*(2.5900)},{\sy*(-0.0776)})
	--({\sx*(2.6000)},{\sy*(-0.0859)})
	--({\sx*(2.6100)},{\sy*(-0.0940)})
	--({\sx*(2.6200)},{\sy*(-0.1021)})
	--({\sx*(2.6300)},{\sy*(-0.1100)})
	--({\sx*(2.6400)},{\sy*(-0.1179)})
	--({\sx*(2.6500)},{\sy*(-0.1256)})
	--({\sx*(2.6600)},{\sy*(-0.1333)})
	--({\sx*(2.6700)},{\sy*(-0.1408)})
	--({\sx*(2.6800)},{\sy*(-0.1482)})
	--({\sx*(2.6900)},{\sy*(-0.1555)})
	--({\sx*(2.7000)},{\sy*(-0.1627)})
	--({\sx*(2.7100)},{\sy*(-0.1698)})
	--({\sx*(2.7200)},{\sy*(-0.1767)})
	--({\sx*(2.7300)},{\sy*(-0.1835)})
	--({\sx*(2.7400)},{\sy*(-0.1901)})
	--({\sx*(2.7500)},{\sy*(-0.1967)})
	--({\sx*(2.7600)},{\sy*(-0.2030)})
	--({\sx*(2.7700)},{\sy*(-0.2093)})
	--({\sx*(2.7800)},{\sy*(-0.2154)})
	--({\sx*(2.7900)},{\sy*(-0.2213)})
	--({\sx*(2.8000)},{\sy*(-0.2271)})
	--({\sx*(2.8100)},{\sy*(-0.2328)})
	--({\sx*(2.8200)},{\sy*(-0.2383)})
	--({\sx*(2.8300)},{\sy*(-0.2436)})
	--({\sx*(2.8400)},{\sy*(-0.2488)})
	--({\sx*(2.8500)},{\sy*(-0.2538)})
	--({\sx*(2.8600)},{\sy*(-0.2586)})
	--({\sx*(2.8700)},{\sy*(-0.2633)})
	--({\sx*(2.8800)},{\sy*(-0.2678)})
	--({\sx*(2.8900)},{\sy*(-0.2722)})
	--({\sx*(2.9000)},{\sy*(-0.2764)})
	--({\sx*(2.9100)},{\sy*(-0.2804)})
	--({\sx*(2.9200)},{\sy*(-0.2842)})
	--({\sx*(2.9300)},{\sy*(-0.2879)})
	--({\sx*(2.9400)},{\sy*(-0.2914)})
	--({\sx*(2.9500)},{\sy*(-0.2947)})
	--({\sx*(2.9600)},{\sy*(-0.2978)})
	--({\sx*(2.9700)},{\sy*(-0.3007)})
	--({\sx*(2.9800)},{\sy*(-0.3035)})
	--({\sx*(2.9900)},{\sy*(-0.3061)})
	--({\sx*(3.0000)},{\sy*(-0.3085)})
	--({\sx*(3.0100)},{\sy*(-0.3107)})
	--({\sx*(3.0200)},{\sy*(-0.3128)})
	--({\sx*(3.0300)},{\sy*(-0.3146)})
	--({\sx*(3.0400)},{\sy*(-0.3163)})
	--({\sx*(3.0500)},{\sy*(-0.3177)})
	--({\sx*(3.0600)},{\sy*(-0.3190)})
	--({\sx*(3.0700)},{\sy*(-0.3201)})
	--({\sx*(3.0800)},{\sy*(-0.3210)})
	--({\sx*(3.0900)},{\sy*(-0.3218)})
	--({\sx*(3.1000)},{\sy*(-0.3223)})
	--({\sx*(3.1100)},{\sy*(-0.3227)})
	--({\sx*(3.1200)},{\sy*(-0.3228)})
	--({\sx*(3.1300)},{\sy*(-0.3228)})
	--({\sx*(3.1400)},{\sy*(-0.3226)})
	--({\sx*(3.1500)},{\sy*(-0.3222)})
	--({\sx*(3.1600)},{\sy*(-0.3216)})
	--({\sx*(3.1700)},{\sy*(-0.3208)})
	--({\sx*(3.1800)},{\sy*(-0.3199)})
	--({\sx*(3.1900)},{\sy*(-0.3187)})
	--({\sx*(3.2000)},{\sy*(-0.3174)})
	--({\sx*(3.2100)},{\sy*(-0.3159)})
	--({\sx*(3.2200)},{\sy*(-0.3142)})
	--({\sx*(3.2300)},{\sy*(-0.3123)})
	--({\sx*(3.2400)},{\sy*(-0.3102)})
	--({\sx*(3.2500)},{\sy*(-0.3079)})
	--({\sx*(3.2600)},{\sy*(-0.3055)})
	--({\sx*(3.2700)},{\sy*(-0.3029)})
	--({\sx*(3.2800)},{\sy*(-0.3001)})
	--({\sx*(3.2900)},{\sy*(-0.2971)})
	--({\sx*(3.3000)},{\sy*(-0.2940)})
	--({\sx*(3.3100)},{\sy*(-0.2907)})
	--({\sx*(3.3200)},{\sy*(-0.2872)})
	--({\sx*(3.3300)},{\sy*(-0.2835)})
	--({\sx*(3.3400)},{\sy*(-0.2796)})
	--({\sx*(3.3500)},{\sy*(-0.2756)})
	--({\sx*(3.3600)},{\sy*(-0.2714)})
	--({\sx*(3.3700)},{\sy*(-0.2671)})
	--({\sx*(3.3800)},{\sy*(-0.2626)})
	--({\sx*(3.3900)},{\sy*(-0.2579)})
	--({\sx*(3.4000)},{\sy*(-0.2531)})
	--({\sx*(3.4100)},{\sy*(-0.2481)})
	--({\sx*(3.4200)},{\sy*(-0.2429)})
	--({\sx*(3.4300)},{\sy*(-0.2376)})
	--({\sx*(3.4400)},{\sy*(-0.2321)})
	--({\sx*(3.4500)},{\sy*(-0.2265)})
	--({\sx*(3.4600)},{\sy*(-0.2207)})
	--({\sx*(3.4700)},{\sy*(-0.2148)})
	--({\sx*(3.4800)},{\sy*(-0.2088)})
	--({\sx*(3.4900)},{\sy*(-0.2026)})
	--({\sx*(3.5000)},{\sy*(-0.1962)})
	--({\sx*(3.5100)},{\sy*(-0.1897)})
	--({\sx*(3.5200)},{\sy*(-0.1831)})
	--({\sx*(3.5300)},{\sy*(-0.1764)})
	--({\sx*(3.5400)},{\sy*(-0.1695)})
	--({\sx*(3.5500)},{\sy*(-0.1625)})
	--({\sx*(3.5600)},{\sy*(-0.1553)})
	--({\sx*(3.5700)},{\sy*(-0.1481)})
	--({\sx*(3.5800)},{\sy*(-0.1407)})
	--({\sx*(3.5900)},{\sy*(-0.1332)})
	--({\sx*(3.6000)},{\sy*(-0.1256)})
	--({\sx*(3.6100)},{\sy*(-0.1179)})
	--({\sx*(3.6200)},{\sy*(-0.1101)})
	--({\sx*(3.6300)},{\sy*(-0.1021)})
	--({\sx*(3.6400)},{\sy*(-0.0941)})
	--({\sx*(3.6500)},{\sy*(-0.0860)})
	--({\sx*(3.6600)},{\sy*(-0.0778)})
	--({\sx*(3.6700)},{\sy*(-0.0694)})
	--({\sx*(3.6800)},{\sy*(-0.0610)})
	--({\sx*(3.6900)},{\sy*(-0.0525)})
	--({\sx*(3.7000)},{\sy*(-0.0440)})
	--({\sx*(3.7100)},{\sy*(-0.0353)})
	--({\sx*(3.7200)},{\sy*(-0.0266)})
	--({\sx*(3.7300)},{\sy*(-0.0178)})
	--({\sx*(3.7400)},{\sy*(-0.0089)})
	--({\sx*(3.7500)},{\sy*(0.0000)})
	--({\sx*(3.7600)},{\sy*(0.0090)})
	--({\sx*(3.7700)},{\sy*(0.0180)})
	--({\sx*(3.7800)},{\sy*(0.0271)})
	--({\sx*(3.7900)},{\sy*(0.0363)})
	--({\sx*(3.8000)},{\sy*(0.0455)})
	--({\sx*(3.8100)},{\sy*(0.0547)})
	--({\sx*(3.8200)},{\sy*(0.0640)})
	--({\sx*(3.8300)},{\sy*(0.0733)})
	--({\sx*(3.8400)},{\sy*(0.0827)})
	--({\sx*(3.8500)},{\sy*(0.0920)})
	--({\sx*(3.8600)},{\sy*(0.1014)})
	--({\sx*(3.8700)},{\sy*(0.1108)})
	--({\sx*(3.8800)},{\sy*(0.1203)})
	--({\sx*(3.8900)},{\sy*(0.1297)})
	--({\sx*(3.9000)},{\sy*(0.1391)})
	--({\sx*(3.9100)},{\sy*(0.1486)})
	--({\sx*(3.9200)},{\sy*(0.1580)})
	--({\sx*(3.9300)},{\sy*(0.1674)})
	--({\sx*(3.9400)},{\sy*(0.1768)})
	--({\sx*(3.9500)},{\sy*(0.1862)})
	--({\sx*(3.9600)},{\sy*(0.1956)})
	--({\sx*(3.9700)},{\sy*(0.2049)})
	--({\sx*(3.9800)},{\sy*(0.2143)})
	--({\sx*(3.9900)},{\sy*(0.2235)})
	--({\sx*(4.0000)},{\sy*(0.2328)})
	--({\sx*(4.0100)},{\sy*(0.2420)})
	--({\sx*(4.0200)},{\sy*(0.2511)})
	--({\sx*(4.0300)},{\sy*(0.2602)})
	--({\sx*(4.0400)},{\sy*(0.2693)})
	--({\sx*(4.0500)},{\sy*(0.2782)})
	--({\sx*(4.0600)},{\sy*(0.2871)})
	--({\sx*(4.0700)},{\sy*(0.2960)})
	--({\sx*(4.0800)},{\sy*(0.3047)})
	--({\sx*(4.0900)},{\sy*(0.3134)})
	--({\sx*(4.1000)},{\sy*(0.3220)})
	--({\sx*(4.1100)},{\sy*(0.3305)})
	--({\sx*(4.1200)},{\sy*(0.3389)})
	--({\sx*(4.1300)},{\sy*(0.3471)})
	--({\sx*(4.1400)},{\sy*(0.3553)})
	--({\sx*(4.1500)},{\sy*(0.3634)})
	--({\sx*(4.1600)},{\sy*(0.3713)})
	--({\sx*(4.1700)},{\sy*(0.3791)})
	--({\sx*(4.1800)},{\sy*(0.3868)})
	--({\sx*(4.1900)},{\sy*(0.3944)})
	--({\sx*(4.2000)},{\sy*(0.4018)})
	--({\sx*(4.2100)},{\sy*(0.4091)})
	--({\sx*(4.2200)},{\sy*(0.4162)})
	--({\sx*(4.2300)},{\sy*(0.4231)})
	--({\sx*(4.2400)},{\sy*(0.4299)})
	--({\sx*(4.2500)},{\sy*(0.4366)})
	--({\sx*(4.2600)},{\sy*(0.4430)})
	--({\sx*(4.2700)},{\sy*(0.4493)})
	--({\sx*(4.2800)},{\sy*(0.4554)})
	--({\sx*(4.2900)},{\sy*(0.4613)})
	--({\sx*(4.3000)},{\sy*(0.4670)})
	--({\sx*(4.3100)},{\sy*(0.4725)})
	--({\sx*(4.3200)},{\sy*(0.4778)})
	--({\sx*(4.3300)},{\sy*(0.4829)})
	--({\sx*(4.3400)},{\sy*(0.4877)})
	--({\sx*(4.3500)},{\sy*(0.4924)})
	--({\sx*(4.3600)},{\sy*(0.4968)})
	--({\sx*(4.3700)},{\sy*(0.5009)})
	--({\sx*(4.3800)},{\sy*(0.5048)})
	--({\sx*(4.3900)},{\sy*(0.5085)})
	--({\sx*(4.4000)},{\sy*(0.5119)})
	--({\sx*(4.4100)},{\sy*(0.5151)})
	--({\sx*(4.4200)},{\sy*(0.5180)})
	--({\sx*(4.4300)},{\sy*(0.5206)})
	--({\sx*(4.4400)},{\sy*(0.5229)})
	--({\sx*(4.4500)},{\sy*(0.5250)})
	--({\sx*(4.4600)},{\sy*(0.5267)})
	--({\sx*(4.4700)},{\sy*(0.5282)})
	--({\sx*(4.4800)},{\sy*(0.5293)})
	--({\sx*(4.4900)},{\sy*(0.5302)})
	--({\sx*(4.5000)},{\sy*(0.5307)})
	--({\sx*(4.5100)},{\sy*(0.5309)})
	--({\sx*(4.5200)},{\sy*(0.5307)})
	--({\sx*(4.5300)},{\sy*(0.5303)})
	--({\sx*(4.5400)},{\sy*(0.5294)})
	--({\sx*(4.5500)},{\sy*(0.5283)})
	--({\sx*(4.5600)},{\sy*(0.5267)})
	--({\sx*(4.5700)},{\sy*(0.5248)})
	--({\sx*(4.5800)},{\sy*(0.5226)})
	--({\sx*(4.5900)},{\sy*(0.5199)})
	--({\sx*(4.6000)},{\sy*(0.5169)})
	--({\sx*(4.6100)},{\sy*(0.5135)})
	--({\sx*(4.6200)},{\sy*(0.5097)})
	--({\sx*(4.6300)},{\sy*(0.5055)})
	--({\sx*(4.6400)},{\sy*(0.5009)})
	--({\sx*(4.6500)},{\sy*(0.4958)})
	--({\sx*(4.6600)},{\sy*(0.4903)})
	--({\sx*(4.6700)},{\sy*(0.4844)})
	--({\sx*(4.6800)},{\sy*(0.4781)})
	--({\sx*(4.6900)},{\sy*(0.4713)})
	--({\sx*(4.7000)},{\sy*(0.4641)})
	--({\sx*(4.7100)},{\sy*(0.4564)})
	--({\sx*(4.7200)},{\sy*(0.4482)})
	--({\sx*(4.7300)},{\sy*(0.4396)})
	--({\sx*(4.7400)},{\sy*(0.4304)})
	--({\sx*(4.7500)},{\sy*(0.4208)})
	--({\sx*(4.7600)},{\sy*(0.4107)})
	--({\sx*(4.7700)},{\sy*(0.4001)})
	--({\sx*(4.7800)},{\sy*(0.3890)})
	--({\sx*(4.7900)},{\sy*(0.3774)})
	--({\sx*(4.8000)},{\sy*(0.3652)})
	--({\sx*(4.8100)},{\sy*(0.3525)})
	--({\sx*(4.8200)},{\sy*(0.3393)})
	--({\sx*(4.8300)},{\sy*(0.3255)})
	--({\sx*(4.8400)},{\sy*(0.3111)})
	--({\sx*(4.8500)},{\sy*(0.2962)})
	--({\sx*(4.8600)},{\sy*(0.2808)})
	--({\sx*(4.8700)},{\sy*(0.2647)})
	--({\sx*(4.8800)},{\sy*(0.2481)})
	--({\sx*(4.8900)},{\sy*(0.2308)})
	--({\sx*(4.9000)},{\sy*(0.2130)})
	--({\sx*(4.9100)},{\sy*(0.1946)})
	--({\sx*(4.9200)},{\sy*(0.1755)})
	--({\sx*(4.9300)},{\sy*(0.1559)})
	--({\sx*(4.9400)},{\sy*(0.1355)})
	--({\sx*(4.9500)},{\sy*(0.1146)})
	--({\sx*(4.9600)},{\sy*(0.0930)})
	--({\sx*(4.9700)},{\sy*(0.0708)})
	--({\sx*(4.9800)},{\sy*(0.0478)})
	--({\sx*(4.9900)},{\sy*(0.0243)})
	--({\sx*(5.0000)},{\sy*(0.0000)});
}
\def\relfehlerb{
\draw[color=blue,line width=1.4pt,line join=round] ({\sx*(0.000)},{\sy*(0.0000)})
	--({\sx*(0.0100)},{\sy*(-0.0030)})
	--({\sx*(0.0200)},{\sy*(-0.0060)})
	--({\sx*(0.0300)},{\sy*(-0.0090)})
	--({\sx*(0.0400)},{\sy*(-0.0119)})
	--({\sx*(0.0500)},{\sy*(-0.0148)})
	--({\sx*(0.0600)},{\sy*(-0.0176)})
	--({\sx*(0.0700)},{\sy*(-0.0204)})
	--({\sx*(0.0800)},{\sy*(-0.0231)})
	--({\sx*(0.0900)},{\sy*(-0.0258)})
	--({\sx*(0.1000)},{\sy*(-0.0285)})
	--({\sx*(0.1100)},{\sy*(-0.0311)})
	--({\sx*(0.1200)},{\sy*(-0.0337)})
	--({\sx*(0.1300)},{\sy*(-0.0362)})
	--({\sx*(0.1400)},{\sy*(-0.0387)})
	--({\sx*(0.1500)},{\sy*(-0.0411)})
	--({\sx*(0.1600)},{\sy*(-0.0434)})
	--({\sx*(0.1700)},{\sy*(-0.0458)})
	--({\sx*(0.1800)},{\sy*(-0.0480)})
	--({\sx*(0.1900)},{\sy*(-0.0503)})
	--({\sx*(0.2000)},{\sy*(-0.0525)})
	--({\sx*(0.2100)},{\sy*(-0.0546)})
	--({\sx*(0.2200)},{\sy*(-0.0567)})
	--({\sx*(0.2300)},{\sy*(-0.0587)})
	--({\sx*(0.2400)},{\sy*(-0.0607)})
	--({\sx*(0.2500)},{\sy*(-0.0626)})
	--({\sx*(0.2600)},{\sy*(-0.0645)})
	--({\sx*(0.2700)},{\sy*(-0.0663)})
	--({\sx*(0.2800)},{\sy*(-0.0681)})
	--({\sx*(0.2900)},{\sy*(-0.0698)})
	--({\sx*(0.3000)},{\sy*(-0.0714)})
	--({\sx*(0.3100)},{\sy*(-0.0731)})
	--({\sx*(0.3200)},{\sy*(-0.0746)})
	--({\sx*(0.3300)},{\sy*(-0.0761)})
	--({\sx*(0.3400)},{\sy*(-0.0776)})
	--({\sx*(0.3500)},{\sy*(-0.0790)})
	--({\sx*(0.3600)},{\sy*(-0.0803)})
	--({\sx*(0.3700)},{\sy*(-0.0816)})
	--({\sx*(0.3800)},{\sy*(-0.0829)})
	--({\sx*(0.3900)},{\sy*(-0.0841)})
	--({\sx*(0.4000)},{\sy*(-0.0852)})
	--({\sx*(0.4100)},{\sy*(-0.0863)})
	--({\sx*(0.4200)},{\sy*(-0.0873)})
	--({\sx*(0.4300)},{\sy*(-0.0883)})
	--({\sx*(0.4400)},{\sy*(-0.0892)})
	--({\sx*(0.4500)},{\sy*(-0.0900)})
	--({\sx*(0.4600)},{\sy*(-0.0908)})
	--({\sx*(0.4700)},{\sy*(-0.0916)})
	--({\sx*(0.4800)},{\sy*(-0.0923)})
	--({\sx*(0.4900)},{\sy*(-0.0929)})
	--({\sx*(0.5000)},{\sy*(-0.0935)})
	--({\sx*(0.5100)},{\sy*(-0.0940)})
	--({\sx*(0.5200)},{\sy*(-0.0945)})
	--({\sx*(0.5300)},{\sy*(-0.0949)})
	--({\sx*(0.5400)},{\sy*(-0.0953)})
	--({\sx*(0.5500)},{\sy*(-0.0956)})
	--({\sx*(0.5600)},{\sy*(-0.0959)})
	--({\sx*(0.5700)},{\sy*(-0.0961)})
	--({\sx*(0.5800)},{\sy*(-0.0962)})
	--({\sx*(0.5900)},{\sy*(-0.0963)})
	--({\sx*(0.6000)},{\sy*(-0.0964)})
	--({\sx*(0.6100)},{\sy*(-0.0964)})
	--({\sx*(0.6200)},{\sy*(-0.0963)})
	--({\sx*(0.6300)},{\sy*(-0.0962)})
	--({\sx*(0.6400)},{\sy*(-0.0960)})
	--({\sx*(0.6500)},{\sy*(-0.0958)})
	--({\sx*(0.6600)},{\sy*(-0.0955)})
	--({\sx*(0.6700)},{\sy*(-0.0952)})
	--({\sx*(0.6800)},{\sy*(-0.0948)})
	--({\sx*(0.6900)},{\sy*(-0.0944)})
	--({\sx*(0.7000)},{\sy*(-0.0939)})
	--({\sx*(0.7100)},{\sy*(-0.0933)})
	--({\sx*(0.7200)},{\sy*(-0.0927)})
	--({\sx*(0.7300)},{\sy*(-0.0921)})
	--({\sx*(0.7400)},{\sy*(-0.0914)})
	--({\sx*(0.7500)},{\sy*(-0.0907)})
	--({\sx*(0.7600)},{\sy*(-0.0899)})
	--({\sx*(0.7700)},{\sy*(-0.0890)})
	--({\sx*(0.7800)},{\sy*(-0.0882)})
	--({\sx*(0.7900)},{\sy*(-0.0872)})
	--({\sx*(0.8000)},{\sy*(-0.0862)})
	--({\sx*(0.8100)},{\sy*(-0.0852)})
	--({\sx*(0.8200)},{\sy*(-0.0841)})
	--({\sx*(0.8300)},{\sy*(-0.0830)})
	--({\sx*(0.8400)},{\sy*(-0.0818)})
	--({\sx*(0.8500)},{\sy*(-0.0806)})
	--({\sx*(0.8600)},{\sy*(-0.0793)})
	--({\sx*(0.8700)},{\sy*(-0.0780)})
	--({\sx*(0.8800)},{\sy*(-0.0766)})
	--({\sx*(0.8900)},{\sy*(-0.0752)})
	--({\sx*(0.9000)},{\sy*(-0.0738)})
	--({\sx*(0.9100)},{\sy*(-0.0723)})
	--({\sx*(0.9200)},{\sy*(-0.0708)})
	--({\sx*(0.9300)},{\sy*(-0.0692)})
	--({\sx*(0.9400)},{\sy*(-0.0676)})
	--({\sx*(0.9500)},{\sy*(-0.0659)})
	--({\sx*(0.9600)},{\sy*(-0.0642)})
	--({\sx*(0.9700)},{\sy*(-0.0625)})
	--({\sx*(0.9800)},{\sy*(-0.0607)})
	--({\sx*(0.9900)},{\sy*(-0.0589)})
	--({\sx*(1.0000)},{\sy*(-0.0570)})
	--({\sx*(1.0100)},{\sy*(-0.0551)})
	--({\sx*(1.0200)},{\sy*(-0.0532)})
	--({\sx*(1.0300)},{\sy*(-0.0512)})
	--({\sx*(1.0400)},{\sy*(-0.0492)})
	--({\sx*(1.0500)},{\sy*(-0.0472)})
	--({\sx*(1.0600)},{\sy*(-0.0451)})
	--({\sx*(1.0700)},{\sy*(-0.0430)})
	--({\sx*(1.0800)},{\sy*(-0.0408)})
	--({\sx*(1.0900)},{\sy*(-0.0387)})
	--({\sx*(1.1000)},{\sy*(-0.0364)})
	--({\sx*(1.1100)},{\sy*(-0.0342)})
	--({\sx*(1.1200)},{\sy*(-0.0319)})
	--({\sx*(1.1300)},{\sy*(-0.0296)})
	--({\sx*(1.1400)},{\sy*(-0.0273)})
	--({\sx*(1.1500)},{\sy*(-0.0250)})
	--({\sx*(1.1600)},{\sy*(-0.0226)})
	--({\sx*(1.1700)},{\sy*(-0.0202)})
	--({\sx*(1.1800)},{\sy*(-0.0177)})
	--({\sx*(1.1900)},{\sy*(-0.0153)})
	--({\sx*(1.2000)},{\sy*(-0.0128)})
	--({\sx*(1.2100)},{\sy*(-0.0103)})
	--({\sx*(1.2200)},{\sy*(-0.0077)})
	--({\sx*(1.2300)},{\sy*(-0.0052)})
	--({\sx*(1.2400)},{\sy*(-0.0026)})
	--({\sx*(1.2500)},{\sy*(0.0000)})
	--({\sx*(1.2600)},{\sy*(0.0026)})
	--({\sx*(1.2700)},{\sy*(0.0053)})
	--({\sx*(1.2800)},{\sy*(0.0079)})
	--({\sx*(1.2900)},{\sy*(0.0106)})
	--({\sx*(1.3000)},{\sy*(0.0133)})
	--({\sx*(1.3100)},{\sy*(0.0160)})
	--({\sx*(1.3200)},{\sy*(0.0187)})
	--({\sx*(1.3300)},{\sy*(0.0214)})
	--({\sx*(1.3400)},{\sy*(0.0242)})
	--({\sx*(1.3500)},{\sy*(0.0269)})
	--({\sx*(1.3600)},{\sy*(0.0297)})
	--({\sx*(1.3700)},{\sy*(0.0325)})
	--({\sx*(1.3800)},{\sy*(0.0353)})
	--({\sx*(1.3900)},{\sy*(0.0381)})
	--({\sx*(1.4000)},{\sy*(0.0409)})
	--({\sx*(1.4100)},{\sy*(0.0437)})
	--({\sx*(1.4200)},{\sy*(0.0465)})
	--({\sx*(1.4300)},{\sy*(0.0494)})
	--({\sx*(1.4400)},{\sy*(0.0522)})
	--({\sx*(1.4500)},{\sy*(0.0550)})
	--({\sx*(1.4600)},{\sy*(0.0579)})
	--({\sx*(1.4700)},{\sy*(0.0607)})
	--({\sx*(1.4800)},{\sy*(0.0636)})
	--({\sx*(1.4900)},{\sy*(0.0664)})
	--({\sx*(1.5000)},{\sy*(0.0692)})
	--({\sx*(1.5100)},{\sy*(0.0721)})
	--({\sx*(1.5200)},{\sy*(0.0749)})
	--({\sx*(1.5300)},{\sy*(0.0777)})
	--({\sx*(1.5400)},{\sy*(0.0806)})
	--({\sx*(1.5500)},{\sy*(0.0834)})
	--({\sx*(1.5600)},{\sy*(0.0862)})
	--({\sx*(1.5700)},{\sy*(0.0890)})
	--({\sx*(1.5800)},{\sy*(0.0918)})
	--({\sx*(1.5900)},{\sy*(0.0946)})
	--({\sx*(1.6000)},{\sy*(0.0973)})
	--({\sx*(1.6100)},{\sy*(0.1001)})
	--({\sx*(1.6200)},{\sy*(0.1028)})
	--({\sx*(1.6300)},{\sy*(0.1056)})
	--({\sx*(1.6400)},{\sy*(0.1083)})
	--({\sx*(1.6500)},{\sy*(0.1110)})
	--({\sx*(1.6600)},{\sy*(0.1136)})
	--({\sx*(1.6700)},{\sy*(0.1163)})
	--({\sx*(1.6800)},{\sy*(0.1189)})
	--({\sx*(1.6900)},{\sy*(0.1215)})
	--({\sx*(1.7000)},{\sy*(0.1241)})
	--({\sx*(1.7100)},{\sy*(0.1267)})
	--({\sx*(1.7200)},{\sy*(0.1292)})
	--({\sx*(1.7300)},{\sy*(0.1317)})
	--({\sx*(1.7400)},{\sy*(0.1342)})
	--({\sx*(1.7500)},{\sy*(0.1367)})
	--({\sx*(1.7600)},{\sy*(0.1391)})
	--({\sx*(1.7700)},{\sy*(0.1415)})
	--({\sx*(1.7800)},{\sy*(0.1438)})
	--({\sx*(1.7900)},{\sy*(0.1462)})
	--({\sx*(1.8000)},{\sy*(0.1484)})
	--({\sx*(1.8100)},{\sy*(0.1507)})
	--({\sx*(1.8200)},{\sy*(0.1529)})
	--({\sx*(1.8300)},{\sy*(0.1551)})
	--({\sx*(1.8400)},{\sy*(0.1572)})
	--({\sx*(1.8500)},{\sy*(0.1592)})
	--({\sx*(1.8600)},{\sy*(0.1613)})
	--({\sx*(1.8700)},{\sy*(0.1633)})
	--({\sx*(1.8800)},{\sy*(0.1652)})
	--({\sx*(1.8900)},{\sy*(0.1671)})
	--({\sx*(1.9000)},{\sy*(0.1689)})
	--({\sx*(1.9100)},{\sy*(0.1707)})
	--({\sx*(1.9200)},{\sy*(0.1724)})
	--({\sx*(1.9300)},{\sy*(0.1740)})
	--({\sx*(1.9400)},{\sy*(0.1756)})
	--({\sx*(1.9500)},{\sy*(0.1771)})
	--({\sx*(1.9600)},{\sy*(0.1786)})
	--({\sx*(1.9700)},{\sy*(0.1799)})
	--({\sx*(1.9800)},{\sy*(0.1813)})
	--({\sx*(1.9900)},{\sy*(0.1825)})
	--({\sx*(2.0000)},{\sy*(0.1837)})
	--({\sx*(2.0100)},{\sy*(0.1847)})
	--({\sx*(2.0200)},{\sy*(0.1857)})
	--({\sx*(2.0300)},{\sy*(0.1866)})
	--({\sx*(2.0400)},{\sy*(0.1875)})
	--({\sx*(2.0500)},{\sy*(0.1882)})
	--({\sx*(2.0600)},{\sy*(0.1888)})
	--({\sx*(2.0700)},{\sy*(0.1894)})
	--({\sx*(2.0800)},{\sy*(0.1898)})
	--({\sx*(2.0900)},{\sy*(0.1901)})
	--({\sx*(2.1000)},{\sy*(0.1903)})
	--({\sx*(2.1100)},{\sy*(0.1904)})
	--({\sx*(2.1200)},{\sy*(0.1904)})
	--({\sx*(2.1300)},{\sy*(0.1902)})
	--({\sx*(2.1400)},{\sy*(0.1899)})
	--({\sx*(2.1500)},{\sy*(0.1895)})
	--({\sx*(2.1600)},{\sy*(0.1889)})
	--({\sx*(2.1700)},{\sy*(0.1882)})
	--({\sx*(2.1800)},{\sy*(0.1873)})
	--({\sx*(2.1900)},{\sy*(0.1863)})
	--({\sx*(2.2000)},{\sy*(0.1851)})
	--({\sx*(2.2100)},{\sy*(0.1837)})
	--({\sx*(2.2200)},{\sy*(0.1822)})
	--({\sx*(2.2300)},{\sy*(0.1804)})
	--({\sx*(2.2400)},{\sy*(0.1784)})
	--({\sx*(2.2500)},{\sy*(0.1763)})
	--({\sx*(2.2600)},{\sy*(0.1739)})
	--({\sx*(2.2700)},{\sy*(0.1713)})
	--({\sx*(2.2800)},{\sy*(0.1684)})
	--({\sx*(2.2900)},{\sy*(0.1652)})
	--({\sx*(2.3000)},{\sy*(0.1618)})
	--({\sx*(2.3100)},{\sy*(0.1581)})
	--({\sx*(2.3200)},{\sy*(0.1541)})
	--({\sx*(2.3300)},{\sy*(0.1498)})
	--({\sx*(2.3400)},{\sy*(0.1451)})
	--({\sx*(2.3500)},{\sy*(0.1401)})
	--({\sx*(2.3600)},{\sy*(0.1347)})
	--({\sx*(2.3700)},{\sy*(0.1289)})
	--({\sx*(2.3800)},{\sy*(0.1226)})
	--({\sx*(2.3900)},{\sy*(0.1159)})
	--({\sx*(2.4000)},{\sy*(0.1087)})
	--({\sx*(2.4100)},{\sy*(0.1010)})
	--({\sx*(2.4200)},{\sy*(0.0927)})
	--({\sx*(2.4300)},{\sy*(0.0838)})
	--({\sx*(2.4400)},{\sy*(0.0742)})
	--({\sx*(2.4500)},{\sy*(0.0640)})
	--({\sx*(2.4600)},{\sy*(0.0530)})
	--({\sx*(2.4700)},{\sy*(0.0411)})
	--({\sx*(2.4800)},{\sy*(0.0284)})
	--({\sx*(2.4900)},{\sy*(0.0147)})
	--({\sx*(2.5000)},{\sy*(0.0000)})
	--({\sx*(2.5100)},{\sy*(-0.0159)})
	--({\sx*(2.5200)},{\sy*(-0.0330)})
	--({\sx*(2.5300)},{\sy*(-0.0516)})
	--({\sx*(2.5400)},{\sy*(-0.0716)})
	--({\sx*(2.5500)},{\sy*(-0.0934)})
	--({\sx*(2.5600)},{\sy*(-0.1170)})
	--({\sx*(2.5700)},{\sy*(-0.1427)})
	--({\sx*(2.5800)},{\sy*(-0.1707)})
	--({\sx*(2.5900)},{\sy*(-0.2014)})
	--({\sx*(2.6000)},{\sy*(-0.2351)})
	--({\sx*(2.6100)},{\sy*(-0.2721)})
	--({\sx*(2.6200)},{\sy*(-0.3130)})
	--({\sx*(2.6300)},{\sy*(-0.3584)})
	--({\sx*(2.6400)},{\sy*(-0.4089)})
	--({\sx*(2.6500)},{\sy*(-0.4654)})
	--({\sx*(2.6600)},{\sy*(-0.5290)})
	--({\sx*(2.6700)},{\sy*(-0.6011)})
	--({\sx*(2.6800)},{\sy*(-0.6832)})
	--({\sx*(2.6900)},{\sy*(-0.7777)})
	--({\sx*(2.7000)},{\sy*(-0.8874)})
	--({\sx*(2.7100)},{\sy*(-1.0161)})
	--({\sx*(2.7200)},{\sy*(-1.1691)})
	--({\sx*(2.7300)},{\sy*(-1.3539)})
	--({\sx*(2.7400)},{\sy*(-1.5811)})
	--({\sx*(2.7500)},{\sy*(-1.8671)})
	--({\sx*(2.7600)},{\sy*(-2.2375)})
	--({\sx*(2.7700)},{\sy*(-2.7360)})
	--({\sx*(2.7800)},{\sy*(-3.4419)})
	--({\sx*(2.7900)},{\sy*(-4.5174)})
	--({\sx*(2.8000)},{\sy*(-6.3539)})
	--({\sx*(2.8100)},{\sy*(-10.1960)})
	--({\sx*(2.8200)},{\sy*(-23.2586)})
	--({\sx*(2.8300)},{\sy*(121.0735)})
	--({\sx*(2.8400)},{\sy*(17.8462)})
	--({\sx*(2.8500)},{\sy*(9.9366)})
	--({\sx*(2.8600)},{\sy*(7.0252)})
	--({\sx*(2.8700)},{\sy*(5.5127)})
	--({\sx*(2.8800)},{\sy*(4.5867)})
	--({\sx*(2.8900)},{\sy*(3.9619)})
	--({\sx*(2.9000)},{\sy*(3.5122)})
	--({\sx*(2.9100)},{\sy*(3.1733)})
	--({\sx*(2.9200)},{\sy*(2.9088)})
	--({\sx*(2.9300)},{\sy*(2.6968)})
	--({\sx*(2.9400)},{\sy*(2.5233)})
	--({\sx*(2.9500)},{\sy*(2.3786)})
	--({\sx*(2.9600)},{\sy*(2.2564)})
	--({\sx*(2.9700)},{\sy*(2.1517)})
	--({\sx*(2.9800)},{\sy*(2.0611)})
	--({\sx*(2.9900)},{\sy*(1.9820)})
	--({\sx*(3.0000)},{\sy*(1.9124)})
	--({\sx*(3.0100)},{\sy*(1.8507)})
	--({\sx*(3.0200)},{\sy*(1.7957)})
	--({\sx*(3.0300)},{\sy*(1.7464)})
	--({\sx*(3.0400)},{\sy*(1.7019)})
	--({\sx*(3.0500)},{\sy*(1.6617)})
	--({\sx*(3.0600)},{\sy*(1.6251)})
	--({\sx*(3.0700)},{\sy*(1.5917)})
	--({\sx*(3.0800)},{\sy*(1.5612)})
	--({\sx*(3.0900)},{\sy*(1.5332)})
	--({\sx*(3.1000)},{\sy*(1.5074)})
	--({\sx*(3.1100)},{\sy*(1.4836)})
	--({\sx*(3.1200)},{\sy*(1.4616)})
	--({\sx*(3.1300)},{\sy*(1.4411)})
	--({\sx*(3.1400)},{\sy*(1.4222)})
	--({\sx*(3.1500)},{\sy*(1.4045)})
	--({\sx*(3.1600)},{\sy*(1.3881)})
	--({\sx*(3.1700)},{\sy*(1.3727)})
	--({\sx*(3.1800)},{\sy*(1.3584)})
	--({\sx*(3.1900)},{\sy*(1.3450)})
	--({\sx*(3.2000)},{\sy*(1.3324)})
	--({\sx*(3.2100)},{\sy*(1.3206)})
	--({\sx*(3.2200)},{\sy*(1.3095)})
	--({\sx*(3.2300)},{\sy*(1.2991)})
	--({\sx*(3.2400)},{\sy*(1.2893)})
	--({\sx*(3.2500)},{\sy*(1.2801)})
	--({\sx*(3.2600)},{\sy*(1.2715)})
	--({\sx*(3.2700)},{\sy*(1.2633)})
	--({\sx*(3.2800)},{\sy*(1.2556)})
	--({\sx*(3.2900)},{\sy*(1.2484)})
	--({\sx*(3.3000)},{\sy*(1.2416)})
	--({\sx*(3.3100)},{\sy*(1.2352)})
	--({\sx*(3.3200)},{\sy*(1.2292)})
	--({\sx*(3.3300)},{\sy*(1.2235)})
	--({\sx*(3.3400)},{\sy*(1.2182)})
	--({\sx*(3.3500)},{\sy*(1.2132)})
	--({\sx*(3.3600)},{\sy*(1.2086)})
	--({\sx*(3.3700)},{\sy*(1.2042)})
	--({\sx*(3.3800)},{\sy*(1.2002)})
	--({\sx*(3.3900)},{\sy*(1.1964)})
	--({\sx*(3.4000)},{\sy*(1.1929)})
	--({\sx*(3.4100)},{\sy*(1.1897)})
	--({\sx*(3.4200)},{\sy*(1.1867)})
	--({\sx*(3.4300)},{\sy*(1.1841)})
	--({\sx*(3.4400)},{\sy*(1.1817)})
	--({\sx*(3.4500)},{\sy*(1.1796)})
	--({\sx*(3.4600)},{\sy*(1.1777)})
	--({\sx*(3.4700)},{\sy*(1.1762)})
	--({\sx*(3.4800)},{\sy*(1.1749)})
	--({\sx*(3.4900)},{\sy*(1.1739)})
	--({\sx*(3.5000)},{\sy*(1.1733)})
	--({\sx*(3.5100)},{\sy*(1.1730)})
	--({\sx*(3.5200)},{\sy*(1.1731)})
	--({\sx*(3.5300)},{\sy*(1.1735)})
	--({\sx*(3.5400)},{\sy*(1.1744)})
	--({\sx*(3.5500)},{\sy*(1.1758)})
	--({\sx*(3.5600)},{\sy*(1.1778)})
	--({\sx*(3.5700)},{\sy*(1.1804)})
	--({\sx*(3.5800)},{\sy*(1.1837)})
	--({\sx*(3.5900)},{\sy*(1.1878)})
	--({\sx*(3.6000)},{\sy*(1.1930)})
	--({\sx*(3.6100)},{\sy*(1.1994)})
	--({\sx*(3.6200)},{\sy*(1.2074)})
	--({\sx*(3.6300)},{\sy*(1.2173)})
	--({\sx*(3.6400)},{\sy*(1.2297)})
	--({\sx*(3.6500)},{\sy*(1.2456)})
	--({\sx*(3.6600)},{\sy*(1.2661)})
	--({\sx*(3.6700)},{\sy*(1.2935)})
	--({\sx*(3.6800)},{\sy*(1.3312)})
	--({\sx*(3.6900)},{\sy*(1.3861)})
	--({\sx*(3.7000)},{\sy*(1.4723)})
	--({\sx*(3.7100)},{\sy*(1.6256)})
	--({\sx*(3.7200)},{\sy*(1.9703)})
	--({\sx*(3.7300)},{\sy*(3.4365)})
	--({\sx*(3.7400)},{\sy*(-2.7693)})
	--({\sx*(3.7500)},{\sy*(0.0000)})
	--({\sx*(3.7600)},{\sy*(0.4436)})
	--({\sx*(3.7700)},{\sy*(0.6242)})
	--({\sx*(3.7800)},{\sy*(0.7219)})
	--({\sx*(3.7900)},{\sy*(0.7828)})
	--({\sx*(3.8000)},{\sy*(0.8243)})
	--({\sx*(3.8100)},{\sy*(0.8543)})
	--({\sx*(3.8200)},{\sy*(0.8769)})
	--({\sx*(3.8300)},{\sy*(0.8945)})
	--({\sx*(3.8400)},{\sy*(0.9085)})
	--({\sx*(3.8500)},{\sy*(0.9200)})
	--({\sx*(3.8600)},{\sy*(0.9294)})
	--({\sx*(3.8700)},{\sy*(0.9373)})
	--({\sx*(3.8800)},{\sy*(0.9440)})
	--({\sx*(3.8900)},{\sy*(0.9498)})
	--({\sx*(3.9000)},{\sy*(0.9547)})
	--({\sx*(3.9100)},{\sy*(0.9590)})
	--({\sx*(3.9200)},{\sy*(0.9628)})
	--({\sx*(3.9300)},{\sy*(0.9661)})
	--({\sx*(3.9400)},{\sy*(0.9691)})
	--({\sx*(3.9500)},{\sy*(0.9717)})
	--({\sx*(3.9600)},{\sy*(0.9740)})
	--({\sx*(3.9700)},{\sy*(0.9761)})
	--({\sx*(3.9800)},{\sy*(0.9780)})
	--({\sx*(3.9900)},{\sy*(0.9797)})
	--({\sx*(4.0000)},{\sy*(0.9813)})
	--({\sx*(4.0100)},{\sy*(0.9827)})
	--({\sx*(4.0200)},{\sy*(0.9839)})
	--({\sx*(4.0300)},{\sy*(0.9851)})
	--({\sx*(4.0400)},{\sy*(0.9861)})
	--({\sx*(4.0500)},{\sy*(0.9871)})
	--({\sx*(4.0600)},{\sy*(0.9880)})
	--({\sx*(4.0700)},{\sy*(0.9888)})
	--({\sx*(4.0800)},{\sy*(0.9896)})
	--({\sx*(4.0900)},{\sy*(0.9902)})
	--({\sx*(4.1000)},{\sy*(0.9909)})
	--({\sx*(4.1100)},{\sy*(0.9915)})
	--({\sx*(4.1200)},{\sy*(0.9920)})
	--({\sx*(4.1300)},{\sy*(0.9925)})
	--({\sx*(4.1400)},{\sy*(0.9930)})
	--({\sx*(4.1500)},{\sy*(0.9934)})
	--({\sx*(4.1600)},{\sy*(0.9938)})
	--({\sx*(4.1700)},{\sy*(0.9942)})
	--({\sx*(4.1800)},{\sy*(0.9945)})
	--({\sx*(4.1900)},{\sy*(0.9949)})
	--({\sx*(4.2000)},{\sy*(0.9952)})
	--({\sx*(4.2100)},{\sy*(0.9954)})
	--({\sx*(4.2200)},{\sy*(0.9957)})
	--({\sx*(4.2300)},{\sy*(0.9959)})
	--({\sx*(4.2400)},{\sy*(0.9962)})
	--({\sx*(4.2500)},{\sy*(0.9964)})
	--({\sx*(4.2600)},{\sy*(0.9966)})
	--({\sx*(4.2700)},{\sy*(0.9968)})
	--({\sx*(4.2800)},{\sy*(0.9969)})
	--({\sx*(4.2900)},{\sy*(0.9971)})
	--({\sx*(4.3000)},{\sy*(0.9973)})
	--({\sx*(4.3100)},{\sy*(0.9974)})
	--({\sx*(4.3200)},{\sy*(0.9975)})
	--({\sx*(4.3300)},{\sy*(0.9977)})
	--({\sx*(4.3400)},{\sy*(0.9978)})
	--({\sx*(4.3500)},{\sy*(0.9979)})
	--({\sx*(4.3600)},{\sy*(0.9980)})
	--({\sx*(4.3700)},{\sy*(0.9981)})
	--({\sx*(4.3800)},{\sy*(0.9982)})
	--({\sx*(4.3900)},{\sy*(0.9983)})
	--({\sx*(4.4000)},{\sy*(0.9984)})
	--({\sx*(4.4100)},{\sy*(0.9985)})
	--({\sx*(4.4200)},{\sy*(0.9985)})
	--({\sx*(4.4300)},{\sy*(0.9986)})
	--({\sx*(4.4400)},{\sy*(0.9987)})
	--({\sx*(4.4500)},{\sy*(0.9987)})
	--({\sx*(4.4600)},{\sy*(0.9988)})
	--({\sx*(4.4700)},{\sy*(0.9989)})
	--({\sx*(4.4800)},{\sy*(0.9989)})
	--({\sx*(4.4900)},{\sy*(0.9990)})
	--({\sx*(4.5000)},{\sy*(0.9990)})
	--({\sx*(4.5100)},{\sy*(0.9990)})
	--({\sx*(4.5200)},{\sy*(0.9991)})
	--({\sx*(4.5300)},{\sy*(0.9991)})
	--({\sx*(4.5400)},{\sy*(0.9992)})
	--({\sx*(4.5500)},{\sy*(0.9992)})
	--({\sx*(4.5600)},{\sy*(0.9992)})
	--({\sx*(4.5700)},{\sy*(0.9993)})
	--({\sx*(4.5800)},{\sy*(0.9993)})
	--({\sx*(4.5900)},{\sy*(0.9993)})
	--({\sx*(4.6000)},{\sy*(0.9993)})
	--({\sx*(4.6100)},{\sy*(0.9994)})
	--({\sx*(4.6200)},{\sy*(0.9994)})
	--({\sx*(4.6300)},{\sy*(0.9994)})
	--({\sx*(4.6400)},{\sy*(0.9994)})
	--({\sx*(4.6500)},{\sy*(0.9995)})
	--({\sx*(4.6600)},{\sy*(0.9995)})
	--({\sx*(4.6700)},{\sy*(0.9995)})
	--({\sx*(4.6800)},{\sy*(0.9995)})
	--({\sx*(4.6900)},{\sy*(0.9995)})
	--({\sx*(4.7000)},{\sy*(0.9995)})
	--({\sx*(4.7100)},{\sy*(0.9996)})
	--({\sx*(4.7200)},{\sy*(0.9996)})
	--({\sx*(4.7300)},{\sy*(0.9996)})
	--({\sx*(4.7400)},{\sy*(0.9996)})
	--({\sx*(4.7500)},{\sy*(0.9996)})
	--({\sx*(4.7600)},{\sy*(0.9996)})
	--({\sx*(4.7700)},{\sy*(0.9996)})
	--({\sx*(4.7800)},{\sy*(0.9996)})
	--({\sx*(4.7900)},{\sy*(0.9996)})
	--({\sx*(4.8000)},{\sy*(0.9996)})
	--({\sx*(4.8100)},{\sy*(0.9996)})
	--({\sx*(4.8200)},{\sy*(0.9996)})
	--({\sx*(4.8300)},{\sy*(0.9997)})
	--({\sx*(4.8400)},{\sy*(0.9997)})
	--({\sx*(4.8500)},{\sy*(0.9997)})
	--({\sx*(4.8600)},{\sy*(0.9996)})
	--({\sx*(4.8700)},{\sy*(0.9996)})
	--({\sx*(4.8800)},{\sy*(0.9996)})
	--({\sx*(4.8900)},{\sy*(0.9996)})
	--({\sx*(4.9000)},{\sy*(0.9996)})
	--({\sx*(4.9100)},{\sy*(0.9996)})
	--({\sx*(4.9200)},{\sy*(0.9996)})
	--({\sx*(4.9300)},{\sy*(0.9996)})
	--({\sx*(4.9400)},{\sy*(0.9995)})
	--({\sx*(4.9500)},{\sy*(0.9994)})
	--({\sx*(4.9600)},{\sy*(0.9994)})
	--({\sx*(4.9700)},{\sy*(0.9992)})
	--({\sx*(4.9800)},{\sy*(0.9989)})
	--({\sx*(4.9900)},{\sy*(0.9979)})
	--({\sx*(5.0000)},{\sy*(0.0000)});
}
\def\xwertec{
\fill[color=red] (0.0000,0) circle[radius={0.07/\skala}];
\fill[color=red] (0.8333,0) circle[radius={0.07/\skala}];
\fill[color=red] (1.6667,0) circle[radius={0.07/\skala}];
\fill[color=red] (2.5000,0) circle[radius={0.07/\skala}];
\fill[color=red] (3.3333,0) circle[radius={0.07/\skala}];
\fill[color=red] (4.1667,0) circle[radius={0.07/\skala}];
\fill[color=red] (5.0000,0) circle[radius={0.07/\skala}];
}
\def\punktec{6}
\def\maxfehlerc{1.041\cdot 10^{-2}}
\def\fehlerc{
\draw[color=red,line width=1.4pt,line join=round] ({\sx*(0.000)},{\sy*(0.0000)})
	--({\sx*(0.0100)},{\sy*(0.0886)})
	--({\sx*(0.0200)},{\sy*(0.1720)})
	--({\sx*(0.0300)},{\sy*(0.2503)})
	--({\sx*(0.0400)},{\sy*(0.3238)})
	--({\sx*(0.0500)},{\sy*(0.3925)})
	--({\sx*(0.0600)},{\sy*(0.4567)})
	--({\sx*(0.0700)},{\sy*(0.5165)})
	--({\sx*(0.0800)},{\sy*(0.5720)})
	--({\sx*(0.0900)},{\sy*(0.6235)})
	--({\sx*(0.1000)},{\sy*(0.6710)})
	--({\sx*(0.1100)},{\sy*(0.7146)})
	--({\sx*(0.1200)},{\sy*(0.7546)})
	--({\sx*(0.1300)},{\sy*(0.7910)})
	--({\sx*(0.1400)},{\sy*(0.8241)})
	--({\sx*(0.1500)},{\sy*(0.8538)})
	--({\sx*(0.1600)},{\sy*(0.8804)})
	--({\sx*(0.1700)},{\sy*(0.9040)})
	--({\sx*(0.1800)},{\sy*(0.9247)})
	--({\sx*(0.1900)},{\sy*(0.9427)})
	--({\sx*(0.2000)},{\sy*(0.9579)})
	--({\sx*(0.2100)},{\sy*(0.9707)})
	--({\sx*(0.2200)},{\sy*(0.9810)})
	--({\sx*(0.2300)},{\sy*(0.9889)})
	--({\sx*(0.2400)},{\sy*(0.9947)})
	--({\sx*(0.2500)},{\sy*(0.9984)})
	--({\sx*(0.2600)},{\sy*(1.0000)})
	--({\sx*(0.2700)},{\sy*(0.9997)})
	--({\sx*(0.2800)},{\sy*(0.9977)})
	--({\sx*(0.2900)},{\sy*(0.9939)})
	--({\sx*(0.3000)},{\sy*(0.9885)})
	--({\sx*(0.3100)},{\sy*(0.9815)})
	--({\sx*(0.3200)},{\sy*(0.9731)})
	--({\sx*(0.3300)},{\sy*(0.9634)})
	--({\sx*(0.3400)},{\sy*(0.9524)})
	--({\sx*(0.3500)},{\sy*(0.9401)})
	--({\sx*(0.3600)},{\sy*(0.9268)})
	--({\sx*(0.3700)},{\sy*(0.9124)})
	--({\sx*(0.3800)},{\sy*(0.8970)})
	--({\sx*(0.3900)},{\sy*(0.8807)})
	--({\sx*(0.4000)},{\sy*(0.8636)})
	--({\sx*(0.4100)},{\sy*(0.8457)})
	--({\sx*(0.4200)},{\sy*(0.8271)})
	--({\sx*(0.4300)},{\sy*(0.8079)})
	--({\sx*(0.4400)},{\sy*(0.7881)})
	--({\sx*(0.4500)},{\sy*(0.7678)})
	--({\sx*(0.4600)},{\sy*(0.7470)})
	--({\sx*(0.4700)},{\sy*(0.7258)})
	--({\sx*(0.4800)},{\sy*(0.7042)})
	--({\sx*(0.4900)},{\sy*(0.6824)})
	--({\sx*(0.5000)},{\sy*(0.6603)})
	--({\sx*(0.5100)},{\sy*(0.6379)})
	--({\sx*(0.5200)},{\sy*(0.6154)})
	--({\sx*(0.5300)},{\sy*(0.5928)})
	--({\sx*(0.5400)},{\sy*(0.5701)})
	--({\sx*(0.5500)},{\sy*(0.5473)})
	--({\sx*(0.5600)},{\sy*(0.5245)})
	--({\sx*(0.5700)},{\sy*(0.5018)})
	--({\sx*(0.5800)},{\sy*(0.4791)})
	--({\sx*(0.5900)},{\sy*(0.4565)})
	--({\sx*(0.6000)},{\sy*(0.4341)})
	--({\sx*(0.6100)},{\sy*(0.4118)})
	--({\sx*(0.6200)},{\sy*(0.3897)})
	--({\sx*(0.6300)},{\sy*(0.3678)})
	--({\sx*(0.6400)},{\sy*(0.3462)})
	--({\sx*(0.6500)},{\sy*(0.3248)})
	--({\sx*(0.6600)},{\sy*(0.3037)})
	--({\sx*(0.6700)},{\sy*(0.2829)})
	--({\sx*(0.6800)},{\sy*(0.2624)})
	--({\sx*(0.6900)},{\sy*(0.2423)})
	--({\sx*(0.7000)},{\sy*(0.2225)})
	--({\sx*(0.7100)},{\sy*(0.2032)})
	--({\sx*(0.7200)},{\sy*(0.1842)})
	--({\sx*(0.7300)},{\sy*(0.1656)})
	--({\sx*(0.7400)},{\sy*(0.1474)})
	--({\sx*(0.7500)},{\sy*(0.1297)})
	--({\sx*(0.7600)},{\sy*(0.1124)})
	--({\sx*(0.7700)},{\sy*(0.0956)})
	--({\sx*(0.7800)},{\sy*(0.0792)})
	--({\sx*(0.7900)},{\sy*(0.0633)})
	--({\sx*(0.8000)},{\sy*(0.0479)})
	--({\sx*(0.8100)},{\sy*(0.0330)})
	--({\sx*(0.8200)},{\sy*(0.0185)})
	--({\sx*(0.8300)},{\sy*(0.0045)})
	--({\sx*(0.8400)},{\sy*(-0.0089)})
	--({\sx*(0.8500)},{\sy*(-0.0219)})
	--({\sx*(0.8600)},{\sy*(-0.0344)})
	--({\sx*(0.8700)},{\sy*(-0.0464)})
	--({\sx*(0.8800)},{\sy*(-0.0579)})
	--({\sx*(0.8900)},{\sy*(-0.0689)})
	--({\sx*(0.9000)},{\sy*(-0.0794)})
	--({\sx*(0.9100)},{\sy*(-0.0894)})
	--({\sx*(0.9200)},{\sy*(-0.0989)})
	--({\sx*(0.9300)},{\sy*(-0.1079)})
	--({\sx*(0.9400)},{\sy*(-0.1164)})
	--({\sx*(0.9500)},{\sy*(-0.1245)})
	--({\sx*(0.9600)},{\sy*(-0.1321)})
	--({\sx*(0.9700)},{\sy*(-0.1392)})
	--({\sx*(0.9800)},{\sy*(-0.1459)})
	--({\sx*(0.9900)},{\sy*(-0.1521)})
	--({\sx*(1.0000)},{\sy*(-0.1578)})
	--({\sx*(1.0100)},{\sy*(-0.1631)})
	--({\sx*(1.0200)},{\sy*(-0.1680)})
	--({\sx*(1.0300)},{\sy*(-0.1724)})
	--({\sx*(1.0400)},{\sy*(-0.1764)})
	--({\sx*(1.0500)},{\sy*(-0.1800)})
	--({\sx*(1.0600)},{\sy*(-0.1832)})
	--({\sx*(1.0700)},{\sy*(-0.1860)})
	--({\sx*(1.0800)},{\sy*(-0.1884)})
	--({\sx*(1.0900)},{\sy*(-0.1904)})
	--({\sx*(1.1000)},{\sy*(-0.1921)})
	--({\sx*(1.1100)},{\sy*(-0.1934)})
	--({\sx*(1.1200)},{\sy*(-0.1944)})
	--({\sx*(1.1300)},{\sy*(-0.1950)})
	--({\sx*(1.1400)},{\sy*(-0.1953)})
	--({\sx*(1.1500)},{\sy*(-0.1953)})
	--({\sx*(1.1600)},{\sy*(-0.1949)})
	--({\sx*(1.1700)},{\sy*(-0.1943)})
	--({\sx*(1.1800)},{\sy*(-0.1934)})
	--({\sx*(1.1900)},{\sy*(-0.1922)})
	--({\sx*(1.2000)},{\sy*(-0.1907)})
	--({\sx*(1.2100)},{\sy*(-0.1890)})
	--({\sx*(1.2200)},{\sy*(-0.1870)})
	--({\sx*(1.2300)},{\sy*(-0.1848)})
	--({\sx*(1.2400)},{\sy*(-0.1824)})
	--({\sx*(1.2500)},{\sy*(-0.1798)})
	--({\sx*(1.2600)},{\sy*(-0.1769)})
	--({\sx*(1.2700)},{\sy*(-0.1739)})
	--({\sx*(1.2800)},{\sy*(-0.1707)})
	--({\sx*(1.2900)},{\sy*(-0.1673)})
	--({\sx*(1.3000)},{\sy*(-0.1638)})
	--({\sx*(1.3100)},{\sy*(-0.1601)})
	--({\sx*(1.3200)},{\sy*(-0.1562)})
	--({\sx*(1.3300)},{\sy*(-0.1523)})
	--({\sx*(1.3400)},{\sy*(-0.1482)})
	--({\sx*(1.3500)},{\sy*(-0.1440)})
	--({\sx*(1.3600)},{\sy*(-0.1397)})
	--({\sx*(1.3700)},{\sy*(-0.1353)})
	--({\sx*(1.3800)},{\sy*(-0.1308)})
	--({\sx*(1.3900)},{\sy*(-0.1262)})
	--({\sx*(1.4000)},{\sy*(-0.1216)})
	--({\sx*(1.4100)},{\sy*(-0.1169)})
	--({\sx*(1.4200)},{\sy*(-0.1122)})
	--({\sx*(1.4300)},{\sy*(-0.1074)})
	--({\sx*(1.4400)},{\sy*(-0.1026)})
	--({\sx*(1.4500)},{\sy*(-0.0978)})
	--({\sx*(1.4600)},{\sy*(-0.0930)})
	--({\sx*(1.4700)},{\sy*(-0.0881)})
	--({\sx*(1.4800)},{\sy*(-0.0833)})
	--({\sx*(1.4900)},{\sy*(-0.0784)})
	--({\sx*(1.5000)},{\sy*(-0.0736)})
	--({\sx*(1.5100)},{\sy*(-0.0687)})
	--({\sx*(1.5200)},{\sy*(-0.0640)})
	--({\sx*(1.5300)},{\sy*(-0.0592)})
	--({\sx*(1.5400)},{\sy*(-0.0545)})
	--({\sx*(1.5500)},{\sy*(-0.0498)})
	--({\sx*(1.5600)},{\sy*(-0.0451)})
	--({\sx*(1.5700)},{\sy*(-0.0406)})
	--({\sx*(1.5800)},{\sy*(-0.0360)})
	--({\sx*(1.5900)},{\sy*(-0.0316)})
	--({\sx*(1.6000)},{\sy*(-0.0272)})
	--({\sx*(1.6100)},{\sy*(-0.0229)})
	--({\sx*(1.6200)},{\sy*(-0.0186)})
	--({\sx*(1.6300)},{\sy*(-0.0145)})
	--({\sx*(1.6400)},{\sy*(-0.0104)})
	--({\sx*(1.6500)},{\sy*(-0.0064)})
	--({\sx*(1.6600)},{\sy*(-0.0025)})
	--({\sx*(1.6700)},{\sy*(0.0013)})
	--({\sx*(1.6800)},{\sy*(0.0049)})
	--({\sx*(1.6900)},{\sy*(0.0085)})
	--({\sx*(1.7000)},{\sy*(0.0120)})
	--({\sx*(1.7100)},{\sy*(0.0154)})
	--({\sx*(1.7200)},{\sy*(0.0187)})
	--({\sx*(1.7300)},{\sy*(0.0219)})
	--({\sx*(1.7400)},{\sy*(0.0249)})
	--({\sx*(1.7500)},{\sy*(0.0279)})
	--({\sx*(1.7600)},{\sy*(0.0307)})
	--({\sx*(1.7700)},{\sy*(0.0334)})
	--({\sx*(1.7800)},{\sy*(0.0360)})
	--({\sx*(1.7900)},{\sy*(0.0385)})
	--({\sx*(1.8000)},{\sy*(0.0408)})
	--({\sx*(1.8100)},{\sy*(0.0431)})
	--({\sx*(1.8200)},{\sy*(0.0452)})
	--({\sx*(1.8300)},{\sy*(0.0472)})
	--({\sx*(1.8400)},{\sy*(0.0491)})
	--({\sx*(1.8500)},{\sy*(0.0508)})
	--({\sx*(1.8600)},{\sy*(0.0525)})
	--({\sx*(1.8700)},{\sy*(0.0540)})
	--({\sx*(1.8800)},{\sy*(0.0554)})
	--({\sx*(1.8900)},{\sy*(0.0567)})
	--({\sx*(1.9000)},{\sy*(0.0579)})
	--({\sx*(1.9100)},{\sy*(0.0590)})
	--({\sx*(1.9200)},{\sy*(0.0599)})
	--({\sx*(1.9300)},{\sy*(0.0608)})
	--({\sx*(1.9400)},{\sy*(0.0615)})
	--({\sx*(1.9500)},{\sy*(0.0621)})
	--({\sx*(1.9600)},{\sy*(0.0626)})
	--({\sx*(1.9700)},{\sy*(0.0630)})
	--({\sx*(1.9800)},{\sy*(0.0633)})
	--({\sx*(1.9900)},{\sy*(0.0636)})
	--({\sx*(2.0000)},{\sy*(0.0637)})
	--({\sx*(2.0100)},{\sy*(0.0637)})
	--({\sx*(2.0200)},{\sy*(0.0636)})
	--({\sx*(2.0300)},{\sy*(0.0634)})
	--({\sx*(2.0400)},{\sy*(0.0631)})
	--({\sx*(2.0500)},{\sy*(0.0628)})
	--({\sx*(2.0600)},{\sy*(0.0623)})
	--({\sx*(2.0700)},{\sy*(0.0618)})
	--({\sx*(2.0800)},{\sy*(0.0612)})
	--({\sx*(2.0900)},{\sy*(0.0605)})
	--({\sx*(2.1000)},{\sy*(0.0598)})
	--({\sx*(2.1100)},{\sy*(0.0589)})
	--({\sx*(2.1200)},{\sy*(0.0580)})
	--({\sx*(2.1300)},{\sy*(0.0571)})
	--({\sx*(2.1400)},{\sy*(0.0560)})
	--({\sx*(2.1500)},{\sy*(0.0550)})
	--({\sx*(2.1600)},{\sy*(0.0538)})
	--({\sx*(2.1700)},{\sy*(0.0526)})
	--({\sx*(2.1800)},{\sy*(0.0513)})
	--({\sx*(2.1900)},{\sy*(0.0500)})
	--({\sx*(2.2000)},{\sy*(0.0487)})
	--({\sx*(2.2100)},{\sy*(0.0473)})
	--({\sx*(2.2200)},{\sy*(0.0459)})
	--({\sx*(2.2300)},{\sy*(0.0444)})
	--({\sx*(2.2400)},{\sy*(0.0429)})
	--({\sx*(2.2500)},{\sy*(0.0413)})
	--({\sx*(2.2600)},{\sy*(0.0398)})
	--({\sx*(2.2700)},{\sy*(0.0382)})
	--({\sx*(2.2800)},{\sy*(0.0365)})
	--({\sx*(2.2900)},{\sy*(0.0349)})
	--({\sx*(2.3000)},{\sy*(0.0332)})
	--({\sx*(2.3100)},{\sy*(0.0316)})
	--({\sx*(2.3200)},{\sy*(0.0299)})
	--({\sx*(2.3300)},{\sy*(0.0282)})
	--({\sx*(2.3400)},{\sy*(0.0265)})
	--({\sx*(2.3500)},{\sy*(0.0247)})
	--({\sx*(2.3600)},{\sy*(0.0230)})
	--({\sx*(2.3700)},{\sy*(0.0213)})
	--({\sx*(2.3800)},{\sy*(0.0196)})
	--({\sx*(2.3900)},{\sy*(0.0179)})
	--({\sx*(2.4000)},{\sy*(0.0162)})
	--({\sx*(2.4100)},{\sy*(0.0145)})
	--({\sx*(2.4200)},{\sy*(0.0128)})
	--({\sx*(2.4300)},{\sy*(0.0111)})
	--({\sx*(2.4400)},{\sy*(0.0095)})
	--({\sx*(2.4500)},{\sy*(0.0078)})
	--({\sx*(2.4600)},{\sy*(0.0062)})
	--({\sx*(2.4700)},{\sy*(0.0046)})
	--({\sx*(2.4800)},{\sy*(0.0031)})
	--({\sx*(2.4900)},{\sy*(0.0015)})
	--({\sx*(2.5000)},{\sy*(0.0000)})
	--({\sx*(2.5100)},{\sy*(-0.0015)})
	--({\sx*(2.5200)},{\sy*(-0.0030)})
	--({\sx*(2.5300)},{\sy*(-0.0044)})
	--({\sx*(2.5400)},{\sy*(-0.0058)})
	--({\sx*(2.5500)},{\sy*(-0.0071)})
	--({\sx*(2.5600)},{\sy*(-0.0085)})
	--({\sx*(2.5700)},{\sy*(-0.0098)})
	--({\sx*(2.5800)},{\sy*(-0.0110)})
	--({\sx*(2.5900)},{\sy*(-0.0122)})
	--({\sx*(2.6000)},{\sy*(-0.0134)})
	--({\sx*(2.6100)},{\sy*(-0.0145)})
	--({\sx*(2.6200)},{\sy*(-0.0156)})
	--({\sx*(2.6300)},{\sy*(-0.0167)})
	--({\sx*(2.6400)},{\sy*(-0.0177)})
	--({\sx*(2.6500)},{\sy*(-0.0187)})
	--({\sx*(2.6600)},{\sy*(-0.0196)})
	--({\sx*(2.6700)},{\sy*(-0.0205)})
	--({\sx*(2.6800)},{\sy*(-0.0213)})
	--({\sx*(2.6900)},{\sy*(-0.0221)})
	--({\sx*(2.7000)},{\sy*(-0.0228)})
	--({\sx*(2.7100)},{\sy*(-0.0235)})
	--({\sx*(2.7200)},{\sy*(-0.0241)})
	--({\sx*(2.7300)},{\sy*(-0.0247)})
	--({\sx*(2.7400)},{\sy*(-0.0253)})
	--({\sx*(2.7500)},{\sy*(-0.0258)})
	--({\sx*(2.7600)},{\sy*(-0.0263)})
	--({\sx*(2.7700)},{\sy*(-0.0267)})
	--({\sx*(2.7800)},{\sy*(-0.0270)})
	--({\sx*(2.7900)},{\sy*(-0.0274)})
	--({\sx*(2.8000)},{\sy*(-0.0276)})
	--({\sx*(2.8100)},{\sy*(-0.0279)})
	--({\sx*(2.8200)},{\sy*(-0.0281)})
	--({\sx*(2.8300)},{\sy*(-0.0282)})
	--({\sx*(2.8400)},{\sy*(-0.0283)})
	--({\sx*(2.8500)},{\sy*(-0.0284)})
	--({\sx*(2.8600)},{\sy*(-0.0284)})
	--({\sx*(2.8700)},{\sy*(-0.0284)})
	--({\sx*(2.8800)},{\sy*(-0.0283)})
	--({\sx*(2.8900)},{\sy*(-0.0282)})
	--({\sx*(2.9000)},{\sy*(-0.0280)})
	--({\sx*(2.9100)},{\sy*(-0.0278)})
	--({\sx*(2.9200)},{\sy*(-0.0276)})
	--({\sx*(2.9300)},{\sy*(-0.0274)})
	--({\sx*(2.9400)},{\sy*(-0.0271)})
	--({\sx*(2.9500)},{\sy*(-0.0267)})
	--({\sx*(2.9600)},{\sy*(-0.0264)})
	--({\sx*(2.9700)},{\sy*(-0.0260)})
	--({\sx*(2.9800)},{\sy*(-0.0255)})
	--({\sx*(2.9900)},{\sy*(-0.0251)})
	--({\sx*(3.0000)},{\sy*(-0.0246)})
	--({\sx*(3.0100)},{\sy*(-0.0241)})
	--({\sx*(3.0200)},{\sy*(-0.0235)})
	--({\sx*(3.0300)},{\sy*(-0.0229)})
	--({\sx*(3.0400)},{\sy*(-0.0224)})
	--({\sx*(3.0500)},{\sy*(-0.0217)})
	--({\sx*(3.0600)},{\sy*(-0.0211)})
	--({\sx*(3.0700)},{\sy*(-0.0204)})
	--({\sx*(3.0800)},{\sy*(-0.0198)})
	--({\sx*(3.0900)},{\sy*(-0.0191)})
	--({\sx*(3.1000)},{\sy*(-0.0183)})
	--({\sx*(3.1100)},{\sy*(-0.0176)})
	--({\sx*(3.1200)},{\sy*(-0.0169)})
	--({\sx*(3.1300)},{\sy*(-0.0161)})
	--({\sx*(3.1400)},{\sy*(-0.0153)})
	--({\sx*(3.1500)},{\sy*(-0.0146)})
	--({\sx*(3.1600)},{\sy*(-0.0138)})
	--({\sx*(3.1700)},{\sy*(-0.0130)})
	--({\sx*(3.1800)},{\sy*(-0.0122)})
	--({\sx*(3.1900)},{\sy*(-0.0114)})
	--({\sx*(3.2000)},{\sy*(-0.0106)})
	--({\sx*(3.2100)},{\sy*(-0.0097)})
	--({\sx*(3.2200)},{\sy*(-0.0089)})
	--({\sx*(3.2300)},{\sy*(-0.0081)})
	--({\sx*(3.2400)},{\sy*(-0.0073)})
	--({\sx*(3.2500)},{\sy*(-0.0065)})
	--({\sx*(3.2600)},{\sy*(-0.0057)})
	--({\sx*(3.2700)},{\sy*(-0.0049)})
	--({\sx*(3.2800)},{\sy*(-0.0041)})
	--({\sx*(3.2900)},{\sy*(-0.0033)})
	--({\sx*(3.3000)},{\sy*(-0.0025)})
	--({\sx*(3.3100)},{\sy*(-0.0018)})
	--({\sx*(3.3200)},{\sy*(-0.0010)})
	--({\sx*(3.3300)},{\sy*(-0.0002)})
	--({\sx*(3.3400)},{\sy*(0.0005)})
	--({\sx*(3.3500)},{\sy*(0.0012)})
	--({\sx*(3.3600)},{\sy*(0.0019)})
	--({\sx*(3.3700)},{\sy*(0.0026)})
	--({\sx*(3.3800)},{\sy*(0.0033)})
	--({\sx*(3.3900)},{\sy*(0.0039)})
	--({\sx*(3.4000)},{\sy*(0.0046)})
	--({\sx*(3.4100)},{\sy*(0.0052)})
	--({\sx*(3.4200)},{\sy*(0.0058)})
	--({\sx*(3.4300)},{\sy*(0.0064)})
	--({\sx*(3.4400)},{\sy*(0.0069)})
	--({\sx*(3.4500)},{\sy*(0.0075)})
	--({\sx*(3.4600)},{\sy*(0.0080)})
	--({\sx*(3.4700)},{\sy*(0.0085)})
	--({\sx*(3.4800)},{\sy*(0.0090)})
	--({\sx*(3.4900)},{\sy*(0.0094)})
	--({\sx*(3.5000)},{\sy*(0.0098)})
	--({\sx*(3.5100)},{\sy*(0.0102)})
	--({\sx*(3.5200)},{\sy*(0.0106)})
	--({\sx*(3.5300)},{\sy*(0.0109)})
	--({\sx*(3.5400)},{\sy*(0.0113)})
	--({\sx*(3.5500)},{\sy*(0.0116)})
	--({\sx*(3.5600)},{\sy*(0.0118)})
	--({\sx*(3.5700)},{\sy*(0.0121)})
	--({\sx*(3.5800)},{\sy*(0.0123)})
	--({\sx*(3.5900)},{\sy*(0.0125)})
	--({\sx*(3.6000)},{\sy*(0.0127)})
	--({\sx*(3.6100)},{\sy*(0.0128)})
	--({\sx*(3.6200)},{\sy*(0.0129)})
	--({\sx*(3.6300)},{\sy*(0.0130)})
	--({\sx*(3.6400)},{\sy*(0.0131)})
	--({\sx*(3.6500)},{\sy*(0.0131)})
	--({\sx*(3.6600)},{\sy*(0.0131)})
	--({\sx*(3.6700)},{\sy*(0.0131)})
	--({\sx*(3.6800)},{\sy*(0.0131)})
	--({\sx*(3.6900)},{\sy*(0.0131)})
	--({\sx*(3.7000)},{\sy*(0.0130)})
	--({\sx*(3.7100)},{\sy*(0.0129)})
	--({\sx*(3.7200)},{\sy*(0.0128)})
	--({\sx*(3.7300)},{\sy*(0.0126)})
	--({\sx*(3.7400)},{\sy*(0.0124)})
	--({\sx*(3.7500)},{\sy*(0.0123)})
	--({\sx*(3.7600)},{\sy*(0.0121)})
	--({\sx*(3.7700)},{\sy*(0.0118)})
	--({\sx*(3.7800)},{\sy*(0.0116)})
	--({\sx*(3.7900)},{\sy*(0.0113)})
	--({\sx*(3.8000)},{\sy*(0.0111)})
	--({\sx*(3.8100)},{\sy*(0.0108)})
	--({\sx*(3.8200)},{\sy*(0.0105)})
	--({\sx*(3.8300)},{\sy*(0.0102)})
	--({\sx*(3.8400)},{\sy*(0.0098)})
	--({\sx*(3.8500)},{\sy*(0.0095)})
	--({\sx*(3.8600)},{\sy*(0.0091)})
	--({\sx*(3.8700)},{\sy*(0.0088)})
	--({\sx*(3.8800)},{\sy*(0.0084)})
	--({\sx*(3.8900)},{\sy*(0.0080)})
	--({\sx*(3.9000)},{\sy*(0.0077)})
	--({\sx*(3.9100)},{\sy*(0.0073)})
	--({\sx*(3.9200)},{\sy*(0.0069)})
	--({\sx*(3.9300)},{\sy*(0.0065)})
	--({\sx*(3.9400)},{\sy*(0.0061)})
	--({\sx*(3.9500)},{\sy*(0.0057)})
	--({\sx*(3.9600)},{\sy*(0.0053)})
	--({\sx*(3.9700)},{\sy*(0.0050)})
	--({\sx*(3.9800)},{\sy*(0.0046)})
	--({\sx*(3.9900)},{\sy*(0.0042)})
	--({\sx*(4.0000)},{\sy*(0.0038)})
	--({\sx*(4.0100)},{\sy*(0.0035)})
	--({\sx*(4.0200)},{\sy*(0.0031)})
	--({\sx*(4.0300)},{\sy*(0.0028)})
	--({\sx*(4.0400)},{\sy*(0.0025)})
	--({\sx*(4.0500)},{\sy*(0.0022)})
	--({\sx*(4.0600)},{\sy*(0.0019)})
	--({\sx*(4.0700)},{\sy*(0.0016)})
	--({\sx*(4.0800)},{\sy*(0.0013)})
	--({\sx*(4.0900)},{\sy*(0.0011)})
	--({\sx*(4.1000)},{\sy*(0.0009)})
	--({\sx*(4.1100)},{\sy*(0.0007)})
	--({\sx*(4.1200)},{\sy*(0.0005)})
	--({\sx*(4.1300)},{\sy*(0.0003)})
	--({\sx*(4.1400)},{\sy*(0.0002)})
	--({\sx*(4.1500)},{\sy*(0.0001)})
	--({\sx*(4.1600)},{\sy*(0.0000)})
	--({\sx*(4.1700)},{\sy*(-0.0000)})
	--({\sx*(4.1800)},{\sy*(-0.0000)})
	--({\sx*(4.1900)},{\sy*(-0.0000)})
	--({\sx*(4.2000)},{\sy*(0.0000)})
	--({\sx*(4.2100)},{\sy*(0.0001)})
	--({\sx*(4.2200)},{\sy*(0.0002)})
	--({\sx*(4.2300)},{\sy*(0.0004)})
	--({\sx*(4.2400)},{\sy*(0.0005)})
	--({\sx*(4.2500)},{\sy*(0.0007)})
	--({\sx*(4.2600)},{\sy*(0.0010)})
	--({\sx*(4.2700)},{\sy*(0.0013)})
	--({\sx*(4.2800)},{\sy*(0.0016)})
	--({\sx*(4.2900)},{\sy*(0.0020)})
	--({\sx*(4.3000)},{\sy*(0.0024)})
	--({\sx*(4.3100)},{\sy*(0.0028)})
	--({\sx*(4.3200)},{\sy*(0.0033)})
	--({\sx*(4.3300)},{\sy*(0.0038)})
	--({\sx*(4.3400)},{\sy*(0.0043)})
	--({\sx*(4.3500)},{\sy*(0.0049)})
	--({\sx*(4.3600)},{\sy*(0.0055)})
	--({\sx*(4.3700)},{\sy*(0.0062)})
	--({\sx*(4.3800)},{\sy*(0.0069)})
	--({\sx*(4.3900)},{\sy*(0.0076)})
	--({\sx*(4.4000)},{\sy*(0.0084)})
	--({\sx*(4.4100)},{\sy*(0.0092)})
	--({\sx*(4.4200)},{\sy*(0.0101)})
	--({\sx*(4.4300)},{\sy*(0.0110)})
	--({\sx*(4.4400)},{\sy*(0.0119)})
	--({\sx*(4.4500)},{\sy*(0.0128)})
	--({\sx*(4.4600)},{\sy*(0.0138)})
	--({\sx*(4.4700)},{\sy*(0.0148)})
	--({\sx*(4.4800)},{\sy*(0.0158)})
	--({\sx*(4.4900)},{\sy*(0.0169)})
	--({\sx*(4.5000)},{\sy*(0.0179)})
	--({\sx*(4.5100)},{\sy*(0.0190)})
	--({\sx*(4.5200)},{\sy*(0.0202)})
	--({\sx*(4.5300)},{\sy*(0.0213)})
	--({\sx*(4.5400)},{\sy*(0.0225)})
	--({\sx*(4.5500)},{\sy*(0.0236)})
	--({\sx*(4.5600)},{\sy*(0.0248)})
	--({\sx*(4.5700)},{\sy*(0.0260)})
	--({\sx*(4.5800)},{\sy*(0.0271)})
	--({\sx*(4.5900)},{\sy*(0.0283)})
	--({\sx*(4.6000)},{\sy*(0.0295)})
	--({\sx*(4.6100)},{\sy*(0.0306)})
	--({\sx*(4.6200)},{\sy*(0.0318)})
	--({\sx*(4.6300)},{\sy*(0.0329)})
	--({\sx*(4.6400)},{\sy*(0.0340)})
	--({\sx*(4.6500)},{\sy*(0.0351)})
	--({\sx*(4.6600)},{\sy*(0.0362)})
	--({\sx*(4.6700)},{\sy*(0.0372)})
	--({\sx*(4.6800)},{\sy*(0.0381)})
	--({\sx*(4.6900)},{\sy*(0.0391)})
	--({\sx*(4.7000)},{\sy*(0.0399)})
	--({\sx*(4.7100)},{\sy*(0.0407)})
	--({\sx*(4.7200)},{\sy*(0.0415)})
	--({\sx*(4.7300)},{\sy*(0.0421)})
	--({\sx*(4.7400)},{\sy*(0.0427)})
	--({\sx*(4.7500)},{\sy*(0.0432)})
	--({\sx*(4.7600)},{\sy*(0.0436)})
	--({\sx*(4.7700)},{\sy*(0.0439)})
	--({\sx*(4.7800)},{\sy*(0.0441)})
	--({\sx*(4.7900)},{\sy*(0.0442)})
	--({\sx*(4.8000)},{\sy*(0.0442)})
	--({\sx*(4.8100)},{\sy*(0.0440)})
	--({\sx*(4.8200)},{\sy*(0.0436)})
	--({\sx*(4.8300)},{\sy*(0.0431)})
	--({\sx*(4.8400)},{\sy*(0.0425)})
	--({\sx*(4.8500)},{\sy*(0.0416)})
	--({\sx*(4.8600)},{\sy*(0.0406)})
	--({\sx*(4.8700)},{\sy*(0.0394)})
	--({\sx*(4.8800)},{\sy*(0.0380)})
	--({\sx*(4.8900)},{\sy*(0.0363)})
	--({\sx*(4.9000)},{\sy*(0.0344)})
	--({\sx*(4.9100)},{\sy*(0.0323)})
	--({\sx*(4.9200)},{\sy*(0.0299)})
	--({\sx*(4.9300)},{\sy*(0.0273)})
	--({\sx*(4.9400)},{\sy*(0.0244)})
	--({\sx*(4.9500)},{\sy*(0.0211)})
	--({\sx*(4.9600)},{\sy*(0.0176)})
	--({\sx*(4.9700)},{\sy*(0.0137)})
	--({\sx*(4.9800)},{\sy*(0.0095)})
	--({\sx*(4.9900)},{\sy*(0.0049)})
	--({\sx*(5.0000)},{\sy*(0.0000)});
}
\def\relfehlerc{
\draw[color=blue,line width=1.4pt,line join=round] ({\sx*(0.000)},{\sy*(0.0000)})
	--({\sx*(0.0100)},{\sy*(0.0023)})
	--({\sx*(0.0200)},{\sy*(0.0045)})
	--({\sx*(0.0300)},{\sy*(0.0065)})
	--({\sx*(0.0400)},{\sy*(0.0084)})
	--({\sx*(0.0500)},{\sy*(0.0101)})
	--({\sx*(0.0600)},{\sy*(0.0118)})
	--({\sx*(0.0700)},{\sy*(0.0133)})
	--({\sx*(0.0800)},{\sy*(0.0148)})
	--({\sx*(0.0900)},{\sy*(0.0161)})
	--({\sx*(0.1000)},{\sy*(0.0173)})
	--({\sx*(0.1100)},{\sy*(0.0184)})
	--({\sx*(0.1200)},{\sy*(0.0194)})
	--({\sx*(0.1300)},{\sy*(0.0204)})
	--({\sx*(0.1400)},{\sy*(0.0212)})
	--({\sx*(0.1500)},{\sy*(0.0220)})
	--({\sx*(0.1600)},{\sy*(0.0227)})
	--({\sx*(0.1700)},{\sy*(0.0234)})
	--({\sx*(0.1800)},{\sy*(0.0239)})
	--({\sx*(0.1900)},{\sy*(0.0244)})
	--({\sx*(0.2000)},{\sy*(0.0249)})
	--({\sx*(0.2100)},{\sy*(0.0252)})
	--({\sx*(0.2200)},{\sy*(0.0255)})
	--({\sx*(0.2300)},{\sy*(0.0258)})
	--({\sx*(0.2400)},{\sy*(0.0260)})
	--({\sx*(0.2500)},{\sy*(0.0262)})
	--({\sx*(0.2600)},{\sy*(0.0263)})
	--({\sx*(0.2700)},{\sy*(0.0263)})
	--({\sx*(0.2800)},{\sy*(0.0264)})
	--({\sx*(0.2900)},{\sy*(0.0263)})
	--({\sx*(0.3000)},{\sy*(0.0263)})
	--({\sx*(0.3100)},{\sy*(0.0262)})
	--({\sx*(0.3200)},{\sy*(0.0260)})
	--({\sx*(0.3300)},{\sy*(0.0259)})
	--({\sx*(0.3400)},{\sy*(0.0256)})
	--({\sx*(0.3500)},{\sy*(0.0254)})
	--({\sx*(0.3600)},{\sy*(0.0251)})
	--({\sx*(0.3700)},{\sy*(0.0249)})
	--({\sx*(0.3800)},{\sy*(0.0245)})
	--({\sx*(0.3900)},{\sy*(0.0242)})
	--({\sx*(0.4000)},{\sy*(0.0238)})
	--({\sx*(0.4100)},{\sy*(0.0234)})
	--({\sx*(0.4200)},{\sy*(0.0230)})
	--({\sx*(0.4300)},{\sy*(0.0226)})
	--({\sx*(0.4400)},{\sy*(0.0221)})
	--({\sx*(0.4500)},{\sy*(0.0217)})
	--({\sx*(0.4600)},{\sy*(0.0212)})
	--({\sx*(0.4700)},{\sy*(0.0207)})
	--({\sx*(0.4800)},{\sy*(0.0202)})
	--({\sx*(0.4900)},{\sy*(0.0197)})
	--({\sx*(0.5000)},{\sy*(0.0191)})
	--({\sx*(0.5100)},{\sy*(0.0186)})
	--({\sx*(0.5200)},{\sy*(0.0180)})
	--({\sx*(0.5300)},{\sy*(0.0175)})
	--({\sx*(0.5400)},{\sy*(0.0169)})
	--({\sx*(0.5500)},{\sy*(0.0163)})
	--({\sx*(0.5600)},{\sy*(0.0158)})
	--({\sx*(0.5700)},{\sy*(0.0152)})
	--({\sx*(0.5800)},{\sy*(0.0146)})
	--({\sx*(0.5900)},{\sy*(0.0140)})
	--({\sx*(0.6000)},{\sy*(0.0134)})
	--({\sx*(0.6100)},{\sy*(0.0128)})
	--({\sx*(0.6200)},{\sy*(0.0122)})
	--({\sx*(0.6300)},{\sy*(0.0116)})
	--({\sx*(0.6400)},{\sy*(0.0110)})
	--({\sx*(0.6500)},{\sy*(0.0104)})
	--({\sx*(0.6600)},{\sy*(0.0098)})
	--({\sx*(0.6700)},{\sy*(0.0092)})
	--({\sx*(0.6800)},{\sy*(0.0086)})
	--({\sx*(0.6900)},{\sy*(0.0080)})
	--({\sx*(0.7000)},{\sy*(0.0074)})
	--({\sx*(0.7100)},{\sy*(0.0068)})
	--({\sx*(0.7200)},{\sy*(0.0062)})
	--({\sx*(0.7300)},{\sy*(0.0056)})
	--({\sx*(0.7400)},{\sy*(0.0050)})
	--({\sx*(0.7500)},{\sy*(0.0045)})
	--({\sx*(0.7600)},{\sy*(0.0039)})
	--({\sx*(0.7700)},{\sy*(0.0033)})
	--({\sx*(0.7800)},{\sy*(0.0028)})
	--({\sx*(0.7900)},{\sy*(0.0023)})
	--({\sx*(0.8000)},{\sy*(0.0017)})
	--({\sx*(0.8100)},{\sy*(0.0012)})
	--({\sx*(0.8200)},{\sy*(0.0007)})
	--({\sx*(0.8300)},{\sy*(0.0002)})
	--({\sx*(0.8400)},{\sy*(-0.0003)})
	--({\sx*(0.8500)},{\sy*(-0.0008)})
	--({\sx*(0.8600)},{\sy*(-0.0013)})
	--({\sx*(0.8700)},{\sy*(-0.0018)})
	--({\sx*(0.8800)},{\sy*(-0.0022)})
	--({\sx*(0.8900)},{\sy*(-0.0027)})
	--({\sx*(0.9000)},{\sy*(-0.0031)})
	--({\sx*(0.9100)},{\sy*(-0.0035)})
	--({\sx*(0.9200)},{\sy*(-0.0040)})
	--({\sx*(0.9300)},{\sy*(-0.0044)})
	--({\sx*(0.9400)},{\sy*(-0.0047)})
	--({\sx*(0.9500)},{\sy*(-0.0051)})
	--({\sx*(0.9600)},{\sy*(-0.0055)})
	--({\sx*(0.9700)},{\sy*(-0.0058)})
	--({\sx*(0.9800)},{\sy*(-0.0062)})
	--({\sx*(0.9900)},{\sy*(-0.0065)})
	--({\sx*(1.0000)},{\sy*(-0.0068)})
	--({\sx*(1.0100)},{\sy*(-0.0071)})
	--({\sx*(1.0200)},{\sy*(-0.0074)})
	--({\sx*(1.0300)},{\sy*(-0.0077)})
	--({\sx*(1.0400)},{\sy*(-0.0080)})
	--({\sx*(1.0500)},{\sy*(-0.0082)})
	--({\sx*(1.0600)},{\sy*(-0.0085)})
	--({\sx*(1.0700)},{\sy*(-0.0087)})
	--({\sx*(1.0800)},{\sy*(-0.0089)})
	--({\sx*(1.0900)},{\sy*(-0.0091)})
	--({\sx*(1.1000)},{\sy*(-0.0093)})
	--({\sx*(1.1100)},{\sy*(-0.0094)})
	--({\sx*(1.1200)},{\sy*(-0.0096)})
	--({\sx*(1.1300)},{\sy*(-0.0097)})
	--({\sx*(1.1400)},{\sy*(-0.0099)})
	--({\sx*(1.1500)},{\sy*(-0.0100)})
	--({\sx*(1.1600)},{\sy*(-0.0101)})
	--({\sx*(1.1700)},{\sy*(-0.0102)})
	--({\sx*(1.1800)},{\sy*(-0.0102)})
	--({\sx*(1.1900)},{\sy*(-0.0103)})
	--({\sx*(1.2000)},{\sy*(-0.0103)})
	--({\sx*(1.2100)},{\sy*(-0.0104)})
	--({\sx*(1.2200)},{\sy*(-0.0104)})
	--({\sx*(1.2300)},{\sy*(-0.0104)})
	--({\sx*(1.2400)},{\sy*(-0.0104)})
	--({\sx*(1.2500)},{\sy*(-0.0103)})
	--({\sx*(1.2600)},{\sy*(-0.0103)})
	--({\sx*(1.2700)},{\sy*(-0.0103)})
	--({\sx*(1.2800)},{\sy*(-0.0102)})
	--({\sx*(1.2900)},{\sy*(-0.0101)})
	--({\sx*(1.3000)},{\sy*(-0.0100)})
	--({\sx*(1.3100)},{\sy*(-0.0099)})
	--({\sx*(1.3200)},{\sy*(-0.0098)})
	--({\sx*(1.3300)},{\sy*(-0.0097)})
	--({\sx*(1.3400)},{\sy*(-0.0096)})
	--({\sx*(1.3500)},{\sy*(-0.0094)})
	--({\sx*(1.3600)},{\sy*(-0.0093)})
	--({\sx*(1.3700)},{\sy*(-0.0091)})
	--({\sx*(1.3800)},{\sy*(-0.0089)})
	--({\sx*(1.3900)},{\sy*(-0.0087)})
	--({\sx*(1.4000)},{\sy*(-0.0085)})
	--({\sx*(1.4100)},{\sy*(-0.0083)})
	--({\sx*(1.4200)},{\sy*(-0.0081)})
	--({\sx*(1.4300)},{\sy*(-0.0079)})
	--({\sx*(1.4400)},{\sy*(-0.0076)})
	--({\sx*(1.4500)},{\sy*(-0.0074)})
	--({\sx*(1.4600)},{\sy*(-0.0071)})
	--({\sx*(1.4700)},{\sy*(-0.0068)})
	--({\sx*(1.4800)},{\sy*(-0.0065)})
	--({\sx*(1.4900)},{\sy*(-0.0062)})
	--({\sx*(1.5000)},{\sy*(-0.0059)})
	--({\sx*(1.5100)},{\sy*(-0.0056)})
	--({\sx*(1.5200)},{\sy*(-0.0053)})
	--({\sx*(1.5300)},{\sy*(-0.0050)})
	--({\sx*(1.5400)},{\sy*(-0.0047)})
	--({\sx*(1.5500)},{\sy*(-0.0043)})
	--({\sx*(1.5600)},{\sy*(-0.0040)})
	--({\sx*(1.5700)},{\sy*(-0.0036)})
	--({\sx*(1.5800)},{\sy*(-0.0033)})
	--({\sx*(1.5900)},{\sy*(-0.0029)})
	--({\sx*(1.6000)},{\sy*(-0.0026)})
	--({\sx*(1.6100)},{\sy*(-0.0022)})
	--({\sx*(1.6200)},{\sy*(-0.0018)})
	--({\sx*(1.6300)},{\sy*(-0.0014)})
	--({\sx*(1.6400)},{\sy*(-0.0010)})
	--({\sx*(1.6500)},{\sy*(-0.0007)})
	--({\sx*(1.6600)},{\sy*(-0.0003)})
	--({\sx*(1.6700)},{\sy*(0.0001)})
	--({\sx*(1.6800)},{\sy*(0.0005)})
	--({\sx*(1.6900)},{\sy*(0.0009)})
	--({\sx*(1.7000)},{\sy*(0.0013)})
	--({\sx*(1.7100)},{\sy*(0.0017)})
	--({\sx*(1.7200)},{\sy*(0.0021)})
	--({\sx*(1.7300)},{\sy*(0.0025)})
	--({\sx*(1.7400)},{\sy*(0.0029)})
	--({\sx*(1.7500)},{\sy*(0.0033)})
	--({\sx*(1.7600)},{\sy*(0.0038)})
	--({\sx*(1.7700)},{\sy*(0.0042)})
	--({\sx*(1.7800)},{\sy*(0.0046)})
	--({\sx*(1.7900)},{\sy*(0.0050)})
	--({\sx*(1.8000)},{\sy*(0.0054)})
	--({\sx*(1.8100)},{\sy*(0.0057)})
	--({\sx*(1.8200)},{\sy*(0.0061)})
	--({\sx*(1.8300)},{\sy*(0.0065)})
	--({\sx*(1.8400)},{\sy*(0.0069)})
	--({\sx*(1.8500)},{\sy*(0.0073)})
	--({\sx*(1.8600)},{\sy*(0.0077)})
	--({\sx*(1.8700)},{\sy*(0.0080)})
	--({\sx*(1.8800)},{\sy*(0.0084)})
	--({\sx*(1.8900)},{\sy*(0.0088)})
	--({\sx*(1.9000)},{\sy*(0.0091)})
	--({\sx*(1.9100)},{\sy*(0.0094)})
	--({\sx*(1.9200)},{\sy*(0.0098)})
	--({\sx*(1.9300)},{\sy*(0.0101)})
	--({\sx*(1.9400)},{\sy*(0.0104)})
	--({\sx*(1.9500)},{\sy*(0.0107)})
	--({\sx*(1.9600)},{\sy*(0.0110)})
	--({\sx*(1.9700)},{\sy*(0.0113)})
	--({\sx*(1.9800)},{\sy*(0.0116)})
	--({\sx*(1.9900)},{\sy*(0.0119)})
	--({\sx*(2.0000)},{\sy*(0.0121)})
	--({\sx*(2.0100)},{\sy*(0.0124)})
	--({\sx*(2.0200)},{\sy*(0.0126)})
	--({\sx*(2.0300)},{\sy*(0.0128)})
	--({\sx*(2.0400)},{\sy*(0.0130)})
	--({\sx*(2.0500)},{\sy*(0.0132)})
	--({\sx*(2.0600)},{\sy*(0.0134)})
	--({\sx*(2.0700)},{\sy*(0.0136)})
	--({\sx*(2.0800)},{\sy*(0.0137)})
	--({\sx*(2.0900)},{\sy*(0.0138)})
	--({\sx*(2.1000)},{\sy*(0.0139)})
	--({\sx*(2.1100)},{\sy*(0.0140)})
	--({\sx*(2.1200)},{\sy*(0.0141)})
	--({\sx*(2.1300)},{\sy*(0.0142)})
	--({\sx*(2.1400)},{\sy*(0.0142)})
	--({\sx*(2.1500)},{\sy*(0.0143)})
	--({\sx*(2.1600)},{\sy*(0.0143)})
	--({\sx*(2.1700)},{\sy*(0.0142)})
	--({\sx*(2.1800)},{\sy*(0.0142)})
	--({\sx*(2.1900)},{\sy*(0.0142)})
	--({\sx*(2.2000)},{\sy*(0.0141)})
	--({\sx*(2.2100)},{\sy*(0.0140)})
	--({\sx*(2.2200)},{\sy*(0.0139)})
	--({\sx*(2.2300)},{\sy*(0.0137)})
	--({\sx*(2.2400)},{\sy*(0.0136)})
	--({\sx*(2.2500)},{\sy*(0.0134)})
	--({\sx*(2.2600)},{\sy*(0.0132)})
	--({\sx*(2.2700)},{\sy*(0.0129)})
	--({\sx*(2.2800)},{\sy*(0.0127)})
	--({\sx*(2.2900)},{\sy*(0.0124)})
	--({\sx*(2.3000)},{\sy*(0.0121)})
	--({\sx*(2.3100)},{\sy*(0.0117)})
	--({\sx*(2.3200)},{\sy*(0.0114)})
	--({\sx*(2.3300)},{\sy*(0.0110)})
	--({\sx*(2.3400)},{\sy*(0.0106)})
	--({\sx*(2.3500)},{\sy*(0.0101)})
	--({\sx*(2.3600)},{\sy*(0.0096)})
	--({\sx*(2.3700)},{\sy*(0.0091)})
	--({\sx*(2.3800)},{\sy*(0.0086)})
	--({\sx*(2.3900)},{\sy*(0.0080)})
	--({\sx*(2.4000)},{\sy*(0.0075)})
	--({\sx*(2.4100)},{\sy*(0.0069)})
	--({\sx*(2.4200)},{\sy*(0.0062)})
	--({\sx*(2.4300)},{\sy*(0.0055)})
	--({\sx*(2.4400)},{\sy*(0.0048)})
	--({\sx*(2.4500)},{\sy*(0.0041)})
	--({\sx*(2.4600)},{\sy*(0.0033)})
	--({\sx*(2.4700)},{\sy*(0.0025)})
	--({\sx*(2.4800)},{\sy*(0.0017)})
	--({\sx*(2.4900)},{\sy*(0.0009)})
	--({\sx*(2.5000)},{\sy*(0.0000)})
	--({\sx*(2.5100)},{\sy*(-0.0009)})
	--({\sx*(2.5200)},{\sy*(-0.0018)})
	--({\sx*(2.5300)},{\sy*(-0.0028)})
	--({\sx*(2.5400)},{\sy*(-0.0038)})
	--({\sx*(2.5500)},{\sy*(-0.0048)})
	--({\sx*(2.5600)},{\sy*(-0.0059)})
	--({\sx*(2.5700)},{\sy*(-0.0070)})
	--({\sx*(2.5800)},{\sy*(-0.0081)})
	--({\sx*(2.5900)},{\sy*(-0.0092)})
	--({\sx*(2.6000)},{\sy*(-0.0104)})
	--({\sx*(2.6100)},{\sy*(-0.0116)})
	--({\sx*(2.6200)},{\sy*(-0.0128)})
	--({\sx*(2.6300)},{\sy*(-0.0140)})
	--({\sx*(2.6400)},{\sy*(-0.0153)})
	--({\sx*(2.6500)},{\sy*(-0.0166)})
	--({\sx*(2.6600)},{\sy*(-0.0179)})
	--({\sx*(2.6700)},{\sy*(-0.0192)})
	--({\sx*(2.6800)},{\sy*(-0.0206)})
	--({\sx*(2.6900)},{\sy*(-0.0219)})
	--({\sx*(2.7000)},{\sy*(-0.0233)})
	--({\sx*(2.7100)},{\sy*(-0.0247)})
	--({\sx*(2.7200)},{\sy*(-0.0261)})
	--({\sx*(2.7300)},{\sy*(-0.0276)})
	--({\sx*(2.7400)},{\sy*(-0.0290)})
	--({\sx*(2.7500)},{\sy*(-0.0304)})
	--({\sx*(2.7600)},{\sy*(-0.0319)})
	--({\sx*(2.7700)},{\sy*(-0.0333)})
	--({\sx*(2.7800)},{\sy*(-0.0348)})
	--({\sx*(2.7900)},{\sy*(-0.0363)})
	--({\sx*(2.8000)},{\sy*(-0.0377)})
	--({\sx*(2.8100)},{\sy*(-0.0392)})
	--({\sx*(2.8200)},{\sy*(-0.0406)})
	--({\sx*(2.8300)},{\sy*(-0.0421)})
	--({\sx*(2.8400)},{\sy*(-0.0435)})
	--({\sx*(2.8500)},{\sy*(-0.0449)})
	--({\sx*(2.8600)},{\sy*(-0.0463)})
	--({\sx*(2.8700)},{\sy*(-0.0476)})
	--({\sx*(2.8800)},{\sy*(-0.0490)})
	--({\sx*(2.8900)},{\sy*(-0.0503)})
	--({\sx*(2.9000)},{\sy*(-0.0515)})
	--({\sx*(2.9100)},{\sy*(-0.0527)})
	--({\sx*(2.9200)},{\sy*(-0.0539)})
	--({\sx*(2.9300)},{\sy*(-0.0551)})
	--({\sx*(2.9400)},{\sy*(-0.0561)})
	--({\sx*(2.9500)},{\sy*(-0.0572)})
	--({\sx*(2.9600)},{\sy*(-0.0581)})
	--({\sx*(2.9700)},{\sy*(-0.0590)})
	--({\sx*(2.9800)},{\sy*(-0.0598)})
	--({\sx*(2.9900)},{\sy*(-0.0606)})
	--({\sx*(3.0000)},{\sy*(-0.0613)})
	--({\sx*(3.0100)},{\sy*(-0.0618)})
	--({\sx*(3.0200)},{\sy*(-0.0623)})
	--({\sx*(3.0300)},{\sy*(-0.0627)})
	--({\sx*(3.0400)},{\sy*(-0.0630)})
	--({\sx*(3.0500)},{\sy*(-0.0631)})
	--({\sx*(3.0600)},{\sy*(-0.0632)})
	--({\sx*(3.0700)},{\sy*(-0.0631)})
	--({\sx*(3.0800)},{\sy*(-0.0629)})
	--({\sx*(3.0900)},{\sy*(-0.0625)})
	--({\sx*(3.1000)},{\sy*(-0.0620)})
	--({\sx*(3.1100)},{\sy*(-0.0614)})
	--({\sx*(3.1200)},{\sy*(-0.0606)})
	--({\sx*(3.1300)},{\sy*(-0.0597)})
	--({\sx*(3.1400)},{\sy*(-0.0586)})
	--({\sx*(3.1500)},{\sy*(-0.0573)})
	--({\sx*(3.1600)},{\sy*(-0.0559)})
	--({\sx*(3.1700)},{\sy*(-0.0543)})
	--({\sx*(3.1800)},{\sy*(-0.0525)})
	--({\sx*(3.1900)},{\sy*(-0.0505)})
	--({\sx*(3.2000)},{\sy*(-0.0483)})
	--({\sx*(3.2100)},{\sy*(-0.0459)})
	--({\sx*(3.2200)},{\sy*(-0.0434)})
	--({\sx*(3.2300)},{\sy*(-0.0406)})
	--({\sx*(3.2400)},{\sy*(-0.0376)})
	--({\sx*(3.2500)},{\sy*(-0.0344)})
	--({\sx*(3.2600)},{\sy*(-0.0311)})
	--({\sx*(3.2700)},{\sy*(-0.0275)})
	--({\sx*(3.2800)},{\sy*(-0.0237)})
	--({\sx*(3.2900)},{\sy*(-0.0197)})
	--({\sx*(3.3000)},{\sy*(-0.0155)})
	--({\sx*(3.3100)},{\sy*(-0.0111)})
	--({\sx*(3.3200)},{\sy*(-0.0065)})
	--({\sx*(3.3300)},{\sy*(-0.0016)})
	--({\sx*(3.3400)},{\sy*(0.0034)})
	--({\sx*(3.3500)},{\sy*(0.0086)})
	--({\sx*(3.3600)},{\sy*(0.0139)})
	--({\sx*(3.3700)},{\sy*(0.0195)})
	--({\sx*(3.3800)},{\sy*(0.0252)})
	--({\sx*(3.3900)},{\sy*(0.0312)})
	--({\sx*(3.4000)},{\sy*(0.0372)})
	--({\sx*(3.4100)},{\sy*(0.0435)})
	--({\sx*(3.4200)},{\sy*(0.0498)})
	--({\sx*(3.4300)},{\sy*(0.0563)})
	--({\sx*(3.4400)},{\sy*(0.0630)})
	--({\sx*(3.4500)},{\sy*(0.0697)})
	--({\sx*(3.4600)},{\sy*(0.0766)})
	--({\sx*(3.4700)},{\sy*(0.0836)})
	--({\sx*(3.4800)},{\sy*(0.0906)})
	--({\sx*(3.4900)},{\sy*(0.0977)})
	--({\sx*(3.5000)},{\sy*(0.1049)})
	--({\sx*(3.5100)},{\sy*(0.1121)})
	--({\sx*(3.5200)},{\sy*(0.1194)})
	--({\sx*(3.5300)},{\sy*(0.1267)})
	--({\sx*(3.5400)},{\sy*(0.1340)})
	--({\sx*(3.5500)},{\sy*(0.1413)})
	--({\sx*(3.5600)},{\sy*(0.1486)})
	--({\sx*(3.5700)},{\sy*(0.1558)})
	--({\sx*(3.5800)},{\sy*(0.1630)})
	--({\sx*(3.5900)},{\sy*(0.1702)})
	--({\sx*(3.6000)},{\sy*(0.1773)})
	--({\sx*(3.6100)},{\sy*(0.1843)})
	--({\sx*(3.6200)},{\sy*(0.1912)})
	--({\sx*(3.6300)},{\sy*(0.1980)})
	--({\sx*(3.6400)},{\sy*(0.2046)})
	--({\sx*(3.6500)},{\sy*(0.2111)})
	--({\sx*(3.6600)},{\sy*(0.2175)})
	--({\sx*(3.6700)},{\sy*(0.2237)})
	--({\sx*(3.6800)},{\sy*(0.2297)})
	--({\sx*(3.6900)},{\sy*(0.2356)})
	--({\sx*(3.7000)},{\sy*(0.2412)})
	--({\sx*(3.7100)},{\sy*(0.2466)})
	--({\sx*(3.7200)},{\sy*(0.2517)})
	--({\sx*(3.7300)},{\sy*(0.2567)})
	--({\sx*(3.7400)},{\sy*(0.2613)})
	--({\sx*(3.7500)},{\sy*(0.2657)})
	--({\sx*(3.7600)},{\sy*(0.2697)})
	--({\sx*(3.7700)},{\sy*(0.2735)})
	--({\sx*(3.7800)},{\sy*(0.2769)})
	--({\sx*(3.7900)},{\sy*(0.2800)})
	--({\sx*(3.8000)},{\sy*(0.2828)})
	--({\sx*(3.8100)},{\sy*(0.2852)})
	--({\sx*(3.8200)},{\sy*(0.2871)})
	--({\sx*(3.8300)},{\sy*(0.2887)})
	--({\sx*(3.8400)},{\sy*(0.2898)})
	--({\sx*(3.8500)},{\sy*(0.2905)})
	--({\sx*(3.8600)},{\sy*(0.2907)})
	--({\sx*(3.8700)},{\sy*(0.2905)})
	--({\sx*(3.8800)},{\sy*(0.2897)})
	--({\sx*(3.8900)},{\sy*(0.2884)})
	--({\sx*(3.9000)},{\sy*(0.2865)})
	--({\sx*(3.9100)},{\sy*(0.2840)})
	--({\sx*(3.9200)},{\sy*(0.2809)})
	--({\sx*(3.9300)},{\sy*(0.2772)})
	--({\sx*(3.9400)},{\sy*(0.2727)})
	--({\sx*(3.9500)},{\sy*(0.2676)})
	--({\sx*(3.9600)},{\sy*(0.2617)})
	--({\sx*(3.9700)},{\sy*(0.2551)})
	--({\sx*(3.9800)},{\sy*(0.2477)})
	--({\sx*(3.9900)},{\sy*(0.2394)})
	--({\sx*(4.0000)},{\sy*(0.2303)})
	--({\sx*(4.0100)},{\sy*(0.2203)})
	--({\sx*(4.0200)},{\sy*(0.2094)})
	--({\sx*(4.0300)},{\sy*(0.1976)})
	--({\sx*(4.0400)},{\sy*(0.1849)})
	--({\sx*(4.0500)},{\sy*(0.1713)})
	--({\sx*(4.0600)},{\sy*(0.1569)})
	--({\sx*(4.0700)},{\sy*(0.1417)})
	--({\sx*(4.0800)},{\sy*(0.1258)})
	--({\sx*(4.0900)},{\sy*(0.1094)})
	--({\sx*(4.1000)},{\sy*(0.0926)})
	--({\sx*(4.1100)},{\sy*(0.0758)})
	--({\sx*(4.1200)},{\sy*(0.0592)})
	--({\sx*(4.1300)},{\sy*(0.0433)})
	--({\sx*(4.1400)},{\sy*(0.0286)})
	--({\sx*(4.1500)},{\sy*(0.0156)})
	--({\sx*(4.1600)},{\sy*(0.0052)})
	--({\sx*(4.1700)},{\sy*(-0.0020)})
	--({\sx*(4.1800)},{\sy*(-0.0050)})
	--({\sx*(4.1900)},{\sy*(-0.0032)})
	--({\sx*(4.2000)},{\sy*(0.0043)})
	--({\sx*(4.2100)},{\sy*(0.0181)})
	--({\sx*(4.2200)},{\sy*(0.0386)})
	--({\sx*(4.2300)},{\sy*(0.0658)})
	--({\sx*(4.2400)},{\sy*(0.0995)})
	--({\sx*(4.2500)},{\sy*(0.1391)})
	--({\sx*(4.2600)},{\sy*(0.1838)})
	--({\sx*(4.2700)},{\sy*(0.2323)})
	--({\sx*(4.2800)},{\sy*(0.2835)})
	--({\sx*(4.2900)},{\sy*(0.3360)})
	--({\sx*(4.3000)},{\sy*(0.3885)})
	--({\sx*(4.3100)},{\sy*(0.4401)})
	--({\sx*(4.3200)},{\sy*(0.4898)})
	--({\sx*(4.3300)},{\sy*(0.5370)})
	--({\sx*(4.3400)},{\sy*(0.5812)})
	--({\sx*(4.3500)},{\sy*(0.6221)})
	--({\sx*(4.3600)},{\sy*(0.6598)})
	--({\sx*(4.3700)},{\sy*(0.6941)})
	--({\sx*(4.3800)},{\sy*(0.7251)})
	--({\sx*(4.3900)},{\sy*(0.7532)})
	--({\sx*(4.4000)},{\sy*(0.7784)})
	--({\sx*(4.4100)},{\sy*(0.8009)})
	--({\sx*(4.4200)},{\sy*(0.8211)})
	--({\sx*(4.4300)},{\sy*(0.8391)})
	--({\sx*(4.4400)},{\sy*(0.8552)})
	--({\sx*(4.4500)},{\sy*(0.8696)})
	--({\sx*(4.4600)},{\sy*(0.8824)})
	--({\sx*(4.4700)},{\sy*(0.8938)})
	--({\sx*(4.4800)},{\sy*(0.9040)})
	--({\sx*(4.4900)},{\sy*(0.9130)})
	--({\sx*(4.5000)},{\sy*(0.9212)})
	--({\sx*(4.5100)},{\sy*(0.9284)})
	--({\sx*(4.5200)},{\sy*(0.9349)})
	--({\sx*(4.5300)},{\sy*(0.9408)})
	--({\sx*(4.5400)},{\sy*(0.9460)})
	--({\sx*(4.5500)},{\sy*(0.9507)})
	--({\sx*(4.5600)},{\sy*(0.9549)})
	--({\sx*(4.5700)},{\sy*(0.9587)})
	--({\sx*(4.5800)},{\sy*(0.9621)})
	--({\sx*(4.5900)},{\sy*(0.9652)})
	--({\sx*(4.6000)},{\sy*(0.9680)})
	--({\sx*(4.6100)},{\sy*(0.9705)})
	--({\sx*(4.6200)},{\sy*(0.9728)})
	--({\sx*(4.6300)},{\sy*(0.9749)})
	--({\sx*(4.6400)},{\sy*(0.9767)})
	--({\sx*(4.6500)},{\sy*(0.9784)})
	--({\sx*(4.6600)},{\sy*(0.9800)})
	--({\sx*(4.6700)},{\sy*(0.9814)})
	--({\sx*(4.6800)},{\sy*(0.9827)})
	--({\sx*(4.6900)},{\sy*(0.9838)})
	--({\sx*(4.7000)},{\sy*(0.9849)})
	--({\sx*(4.7100)},{\sy*(0.9859)})
	--({\sx*(4.7200)},{\sy*(0.9867)})
	--({\sx*(4.7300)},{\sy*(0.9876)})
	--({\sx*(4.7400)},{\sy*(0.9883)})
	--({\sx*(4.7500)},{\sy*(0.9889)})
	--({\sx*(4.7600)},{\sy*(0.9896)})
	--({\sx*(4.7700)},{\sy*(0.9901)})
	--({\sx*(4.7800)},{\sy*(0.9906)})
	--({\sx*(4.7900)},{\sy*(0.9910)})
	--({\sx*(4.8000)},{\sy*(0.9915)})
	--({\sx*(4.8100)},{\sy*(0.9918)})
	--({\sx*(4.8200)},{\sy*(0.9921)})
	--({\sx*(4.8300)},{\sy*(0.9924)})
	--({\sx*(4.8400)},{\sy*(0.9927)})
	--({\sx*(4.8500)},{\sy*(0.9929)})
	--({\sx*(4.8600)},{\sy*(0.9930)})
	--({\sx*(4.8700)},{\sy*(0.9932)})
	--({\sx*(4.8800)},{\sy*(0.9932)})
	--({\sx*(4.8900)},{\sy*(0.9933)})
	--({\sx*(4.9000)},{\sy*(0.9932)})
	--({\sx*(4.9100)},{\sy*(0.9931)})
	--({\sx*(4.9200)},{\sy*(0.9930)})
	--({\sx*(4.9300)},{\sy*(0.9926)})
	--({\sx*(4.9400)},{\sy*(0.9922)})
	--({\sx*(4.9500)},{\sy*(0.9914)})
	--({\sx*(4.9600)},{\sy*(0.9902)})
	--({\sx*(4.9700)},{\sy*(0.9880)})
	--({\sx*(4.9800)},{\sy*(0.9837)})
	--({\sx*(4.9900)},{\sy*(0.9705)})
	--({\sx*(5.0000)},{\sy*(0.0000)});
}
\def\xwerted{
\fill[color=red] (0.0000,0) circle[radius={0.07/\skala}];
\fill[color=red] (0.6250,0) circle[radius={0.07/\skala}];
\fill[color=red] (1.2500,0) circle[radius={0.07/\skala}];
\fill[color=red] (1.8750,0) circle[radius={0.07/\skala}];
\fill[color=red] (2.5000,0) circle[radius={0.07/\skala}];
\fill[color=red] (3.1250,0) circle[radius={0.07/\skala}];
\fill[color=red] (3.7500,0) circle[radius={0.07/\skala}];
\fill[color=red] (4.3750,0) circle[radius={0.07/\skala}];
\fill[color=red] (5.0000,0) circle[radius={0.07/\skala}];
}
\def\punkted{8}
\def\maxfehlerd{1.475\cdot 10^{-3}}
\def\fehlerd{
\draw[color=red,line width=1.4pt,line join=round] ({\sx*(0.000)},{\sy*(0.0000)})
	--({\sx*(0.0100)},{\sy*(-0.0674)})
	--({\sx*(0.0200)},{\sy*(-0.1277)})
	--({\sx*(0.0300)},{\sy*(-0.1812)})
	--({\sx*(0.0400)},{\sy*(-0.2284)})
	--({\sx*(0.0500)},{\sy*(-0.2697)})
	--({\sx*(0.0600)},{\sy*(-0.3056)})
	--({\sx*(0.0700)},{\sy*(-0.3364)})
	--({\sx*(0.0800)},{\sy*(-0.3625)})
	--({\sx*(0.0900)},{\sy*(-0.3843)})
	--({\sx*(0.1000)},{\sy*(-0.4021)})
	--({\sx*(0.1100)},{\sy*(-0.4162)})
	--({\sx*(0.1200)},{\sy*(-0.4269)})
	--({\sx*(0.1300)},{\sy*(-0.4345)})
	--({\sx*(0.1400)},{\sy*(-0.4392)})
	--({\sx*(0.1500)},{\sy*(-0.4414)})
	--({\sx*(0.1600)},{\sy*(-0.4413)})
	--({\sx*(0.1700)},{\sy*(-0.4390)})
	--({\sx*(0.1800)},{\sy*(-0.4349)})
	--({\sx*(0.1900)},{\sy*(-0.4290)})
	--({\sx*(0.2000)},{\sy*(-0.4217)})
	--({\sx*(0.2100)},{\sy*(-0.4130)})
	--({\sx*(0.2200)},{\sy*(-0.4031)})
	--({\sx*(0.2300)},{\sy*(-0.3923)})
	--({\sx*(0.2400)},{\sy*(-0.3806)})
	--({\sx*(0.2500)},{\sy*(-0.3681)})
	--({\sx*(0.2600)},{\sy*(-0.3551)})
	--({\sx*(0.2700)},{\sy*(-0.3416)})
	--({\sx*(0.2800)},{\sy*(-0.3277)})
	--({\sx*(0.2900)},{\sy*(-0.3135)})
	--({\sx*(0.3000)},{\sy*(-0.2991)})
	--({\sx*(0.3100)},{\sy*(-0.2846)})
	--({\sx*(0.3200)},{\sy*(-0.2702)})
	--({\sx*(0.3300)},{\sy*(-0.2557)})
	--({\sx*(0.3400)},{\sy*(-0.2414)})
	--({\sx*(0.3500)},{\sy*(-0.2272)})
	--({\sx*(0.3600)},{\sy*(-0.2133)})
	--({\sx*(0.3700)},{\sy*(-0.1996)})
	--({\sx*(0.3800)},{\sy*(-0.1863)})
	--({\sx*(0.3900)},{\sy*(-0.1733)})
	--({\sx*(0.4000)},{\sy*(-0.1607)})
	--({\sx*(0.4100)},{\sy*(-0.1484)})
	--({\sx*(0.4200)},{\sy*(-0.1366)})
	--({\sx*(0.4300)},{\sy*(-0.1253)})
	--({\sx*(0.4400)},{\sy*(-0.1144)})
	--({\sx*(0.4500)},{\sy*(-0.1041)})
	--({\sx*(0.4600)},{\sy*(-0.0942)})
	--({\sx*(0.4700)},{\sy*(-0.0848)})
	--({\sx*(0.4800)},{\sy*(-0.0759)})
	--({\sx*(0.4900)},{\sy*(-0.0675)})
	--({\sx*(0.5000)},{\sy*(-0.0596)})
	--({\sx*(0.5100)},{\sy*(-0.0522)})
	--({\sx*(0.5200)},{\sy*(-0.0452)})
	--({\sx*(0.5300)},{\sy*(-0.0388)})
	--({\sx*(0.5400)},{\sy*(-0.0329)})
	--({\sx*(0.5500)},{\sy*(-0.0274)})
	--({\sx*(0.5600)},{\sy*(-0.0224)})
	--({\sx*(0.5700)},{\sy*(-0.0178)})
	--({\sx*(0.5800)},{\sy*(-0.0137)})
	--({\sx*(0.5900)},{\sy*(-0.0100)})
	--({\sx*(0.6000)},{\sy*(-0.0066)})
	--({\sx*(0.6100)},{\sy*(-0.0037)})
	--({\sx*(0.6200)},{\sy*(-0.0011)})
	--({\sx*(0.6300)},{\sy*(0.0011)})
	--({\sx*(0.6400)},{\sy*(0.0029)})
	--({\sx*(0.6500)},{\sy*(0.0045)})
	--({\sx*(0.6600)},{\sy*(0.0058)})
	--({\sx*(0.6700)},{\sy*(0.0067)})
	--({\sx*(0.6800)},{\sy*(0.0074)})
	--({\sx*(0.6900)},{\sy*(0.0079)})
	--({\sx*(0.7000)},{\sy*(0.0081)})
	--({\sx*(0.7100)},{\sy*(0.0082)})
	--({\sx*(0.7200)},{\sy*(0.0080)})
	--({\sx*(0.7300)},{\sy*(0.0077)})
	--({\sx*(0.7400)},{\sy*(0.0072)})
	--({\sx*(0.7500)},{\sy*(0.0065)})
	--({\sx*(0.7600)},{\sy*(0.0057)})
	--({\sx*(0.7700)},{\sy*(0.0049)})
	--({\sx*(0.7800)},{\sy*(0.0039)})
	--({\sx*(0.7900)},{\sy*(0.0028)})
	--({\sx*(0.8000)},{\sy*(0.0017)})
	--({\sx*(0.8100)},{\sy*(0.0005)})
	--({\sx*(0.8200)},{\sy*(-0.0007)})
	--({\sx*(0.8300)},{\sy*(-0.0019)})
	--({\sx*(0.8400)},{\sy*(-0.0032)})
	--({\sx*(0.8500)},{\sy*(-0.0044)})
	--({\sx*(0.8600)},{\sy*(-0.0057)})
	--({\sx*(0.8700)},{\sy*(-0.0069)})
	--({\sx*(0.8800)},{\sy*(-0.0081)})
	--({\sx*(0.8900)},{\sy*(-0.0093)})
	--({\sx*(0.9000)},{\sy*(-0.0104)})
	--({\sx*(0.9100)},{\sy*(-0.0115)})
	--({\sx*(0.9200)},{\sy*(-0.0125)})
	--({\sx*(0.9300)},{\sy*(-0.0135)})
	--({\sx*(0.9400)},{\sy*(-0.0144)})
	--({\sx*(0.9500)},{\sy*(-0.0152)})
	--({\sx*(0.9600)},{\sy*(-0.0159)})
	--({\sx*(0.9700)},{\sy*(-0.0165)})
	--({\sx*(0.9800)},{\sy*(-0.0171)})
	--({\sx*(0.9900)},{\sy*(-0.0176)})
	--({\sx*(1.0000)},{\sy*(-0.0179)})
	--({\sx*(1.0100)},{\sy*(-0.0182)})
	--({\sx*(1.0200)},{\sy*(-0.0184)})
	--({\sx*(1.0300)},{\sy*(-0.0185)})
	--({\sx*(1.0400)},{\sy*(-0.0185)})
	--({\sx*(1.0500)},{\sy*(-0.0184)})
	--({\sx*(1.0600)},{\sy*(-0.0182)})
	--({\sx*(1.0700)},{\sy*(-0.0179)})
	--({\sx*(1.0800)},{\sy*(-0.0175)})
	--({\sx*(1.0900)},{\sy*(-0.0171)})
	--({\sx*(1.1000)},{\sy*(-0.0165)})
	--({\sx*(1.1100)},{\sy*(-0.0159)})
	--({\sx*(1.1200)},{\sy*(-0.0151)})
	--({\sx*(1.1300)},{\sy*(-0.0143)})
	--({\sx*(1.1400)},{\sy*(-0.0134)})
	--({\sx*(1.1500)},{\sy*(-0.0125)})
	--({\sx*(1.1600)},{\sy*(-0.0115)})
	--({\sx*(1.1700)},{\sy*(-0.0104)})
	--({\sx*(1.1800)},{\sy*(-0.0093)})
	--({\sx*(1.1900)},{\sy*(-0.0081)})
	--({\sx*(1.2000)},{\sy*(-0.0068)})
	--({\sx*(1.2100)},{\sy*(-0.0055)})
	--({\sx*(1.2200)},{\sy*(-0.0042)})
	--({\sx*(1.2300)},{\sy*(-0.0028)})
	--({\sx*(1.2400)},{\sy*(-0.0014)})
	--({\sx*(1.2500)},{\sy*(0.0000)})
	--({\sx*(1.2600)},{\sy*(0.0014)})
	--({\sx*(1.2700)},{\sy*(0.0029)})
	--({\sx*(1.2800)},{\sy*(0.0044)})
	--({\sx*(1.2900)},{\sy*(0.0058)})
	--({\sx*(1.3000)},{\sy*(0.0073)})
	--({\sx*(1.3100)},{\sy*(0.0088)})
	--({\sx*(1.3200)},{\sy*(0.0102)})
	--({\sx*(1.3300)},{\sy*(0.0117)})
	--({\sx*(1.3400)},{\sy*(0.0131)})
	--({\sx*(1.3500)},{\sy*(0.0145)})
	--({\sx*(1.3600)},{\sy*(0.0158)})
	--({\sx*(1.3700)},{\sy*(0.0172)})
	--({\sx*(1.3800)},{\sy*(0.0184)})
	--({\sx*(1.3900)},{\sy*(0.0197)})
	--({\sx*(1.4000)},{\sy*(0.0209)})
	--({\sx*(1.4100)},{\sy*(0.0220)})
	--({\sx*(1.4200)},{\sy*(0.0231)})
	--({\sx*(1.4300)},{\sy*(0.0242)})
	--({\sx*(1.4400)},{\sy*(0.0252)})
	--({\sx*(1.4500)},{\sy*(0.0261)})
	--({\sx*(1.4600)},{\sy*(0.0269)})
	--({\sx*(1.4700)},{\sy*(0.0277)})
	--({\sx*(1.4800)},{\sy*(0.0284)})
	--({\sx*(1.4900)},{\sy*(0.0291)})
	--({\sx*(1.5000)},{\sy*(0.0296)})
	--({\sx*(1.5100)},{\sy*(0.0301)})
	--({\sx*(1.5200)},{\sy*(0.0305)})
	--({\sx*(1.5300)},{\sy*(0.0309)})
	--({\sx*(1.5400)},{\sy*(0.0311)})
	--({\sx*(1.5500)},{\sy*(0.0313)})
	--({\sx*(1.5600)},{\sy*(0.0314)})
	--({\sx*(1.5700)},{\sy*(0.0314)})
	--({\sx*(1.5800)},{\sy*(0.0313)})
	--({\sx*(1.5900)},{\sy*(0.0311)})
	--({\sx*(1.6000)},{\sy*(0.0309)})
	--({\sx*(1.6100)},{\sy*(0.0306)})
	--({\sx*(1.6200)},{\sy*(0.0302)})
	--({\sx*(1.6300)},{\sy*(0.0297)})
	--({\sx*(1.6400)},{\sy*(0.0292)})
	--({\sx*(1.6500)},{\sy*(0.0285)})
	--({\sx*(1.6600)},{\sy*(0.0278)})
	--({\sx*(1.6700)},{\sy*(0.0271)})
	--({\sx*(1.6800)},{\sy*(0.0262)})
	--({\sx*(1.6900)},{\sy*(0.0253)})
	--({\sx*(1.7000)},{\sy*(0.0243)})
	--({\sx*(1.7100)},{\sy*(0.0233)})
	--({\sx*(1.7200)},{\sy*(0.0222)})
	--({\sx*(1.7300)},{\sy*(0.0211)})
	--({\sx*(1.7400)},{\sy*(0.0199)})
	--({\sx*(1.7500)},{\sy*(0.0186)})
	--({\sx*(1.7600)},{\sy*(0.0173)})
	--({\sx*(1.7700)},{\sy*(0.0159)})
	--({\sx*(1.7800)},{\sy*(0.0145)})
	--({\sx*(1.7900)},{\sy*(0.0131)})
	--({\sx*(1.8000)},{\sy*(0.0117)})
	--({\sx*(1.8100)},{\sy*(0.0102)})
	--({\sx*(1.8200)},{\sy*(0.0086)})
	--({\sx*(1.8300)},{\sy*(0.0071)})
	--({\sx*(1.8400)},{\sy*(0.0055)})
	--({\sx*(1.8500)},{\sy*(0.0040)})
	--({\sx*(1.8600)},{\sy*(0.0024)})
	--({\sx*(1.8700)},{\sy*(0.0008)})
	--({\sx*(1.8800)},{\sy*(-0.0008)})
	--({\sx*(1.8900)},{\sy*(-0.0024)})
	--({\sx*(1.9000)},{\sy*(-0.0040)})
	--({\sx*(1.9100)},{\sy*(-0.0056)})
	--({\sx*(1.9200)},{\sy*(-0.0071)})
	--({\sx*(1.9300)},{\sy*(-0.0087)})
	--({\sx*(1.9400)},{\sy*(-0.0102)})
	--({\sx*(1.9500)},{\sy*(-0.0117)})
	--({\sx*(1.9600)},{\sy*(-0.0132)})
	--({\sx*(1.9700)},{\sy*(-0.0146)})
	--({\sx*(1.9800)},{\sy*(-0.0160)})
	--({\sx*(1.9900)},{\sy*(-0.0174)})
	--({\sx*(2.0000)},{\sy*(-0.0187)})
	--({\sx*(2.0100)},{\sy*(-0.0200)})
	--({\sx*(2.0200)},{\sy*(-0.0212)})
	--({\sx*(2.0300)},{\sy*(-0.0224)})
	--({\sx*(2.0400)},{\sy*(-0.0236)})
	--({\sx*(2.0500)},{\sy*(-0.0246)})
	--({\sx*(2.0600)},{\sy*(-0.0256)})
	--({\sx*(2.0700)},{\sy*(-0.0266)})
	--({\sx*(2.0800)},{\sy*(-0.0275)})
	--({\sx*(2.0900)},{\sy*(-0.0283)})
	--({\sx*(2.1000)},{\sy*(-0.0290)})
	--({\sx*(2.1100)},{\sy*(-0.0297)})
	--({\sx*(2.1200)},{\sy*(-0.0303)})
	--({\sx*(2.1300)},{\sy*(-0.0308)})
	--({\sx*(2.1400)},{\sy*(-0.0313)})
	--({\sx*(2.1500)},{\sy*(-0.0317)})
	--({\sx*(2.1600)},{\sy*(-0.0320)})
	--({\sx*(2.1700)},{\sy*(-0.0322)})
	--({\sx*(2.1800)},{\sy*(-0.0323)})
	--({\sx*(2.1900)},{\sy*(-0.0324)})
	--({\sx*(2.2000)},{\sy*(-0.0324)})
	--({\sx*(2.2100)},{\sy*(-0.0323)})
	--({\sx*(2.2200)},{\sy*(-0.0321)})
	--({\sx*(2.2300)},{\sy*(-0.0318)})
	--({\sx*(2.2400)},{\sy*(-0.0315)})
	--({\sx*(2.2500)},{\sy*(-0.0311)})
	--({\sx*(2.2600)},{\sy*(-0.0306)})
	--({\sx*(2.2700)},{\sy*(-0.0300)})
	--({\sx*(2.2800)},{\sy*(-0.0294)})
	--({\sx*(2.2900)},{\sy*(-0.0286)})
	--({\sx*(2.3000)},{\sy*(-0.0278)})
	--({\sx*(2.3100)},{\sy*(-0.0270)})
	--({\sx*(2.3200)},{\sy*(-0.0260)})
	--({\sx*(2.3300)},{\sy*(-0.0250)})
	--({\sx*(2.3400)},{\sy*(-0.0239)})
	--({\sx*(2.3500)},{\sy*(-0.0228)})
	--({\sx*(2.3600)},{\sy*(-0.0216)})
	--({\sx*(2.3700)},{\sy*(-0.0203)})
	--({\sx*(2.3800)},{\sy*(-0.0190)})
	--({\sx*(2.3900)},{\sy*(-0.0176)})
	--({\sx*(2.4000)},{\sy*(-0.0162)})
	--({\sx*(2.4100)},{\sy*(-0.0148)})
	--({\sx*(2.4200)},{\sy*(-0.0133)})
	--({\sx*(2.4300)},{\sy*(-0.0117)})
	--({\sx*(2.4400)},{\sy*(-0.0101)})
	--({\sx*(2.4500)},{\sy*(-0.0085)})
	--({\sx*(2.4600)},{\sy*(-0.0068)})
	--({\sx*(2.4700)},{\sy*(-0.0052)})
	--({\sx*(2.4800)},{\sy*(-0.0035)})
	--({\sx*(2.4900)},{\sy*(-0.0017)})
	--({\sx*(2.5000)},{\sy*(0.0000)})
	--({\sx*(2.5100)},{\sy*(0.0017)})
	--({\sx*(2.5200)},{\sy*(0.0035)})
	--({\sx*(2.5300)},{\sy*(0.0053)})
	--({\sx*(2.5400)},{\sy*(0.0070)})
	--({\sx*(2.5500)},{\sy*(0.0088)})
	--({\sx*(2.5600)},{\sy*(0.0105)})
	--({\sx*(2.5700)},{\sy*(0.0122)})
	--({\sx*(2.5800)},{\sy*(0.0139)})
	--({\sx*(2.5900)},{\sy*(0.0156)})
	--({\sx*(2.6000)},{\sy*(0.0173)})
	--({\sx*(2.6100)},{\sy*(0.0189)})
	--({\sx*(2.6200)},{\sy*(0.0205)})
	--({\sx*(2.6300)},{\sy*(0.0220)})
	--({\sx*(2.6400)},{\sy*(0.0236)})
	--({\sx*(2.6500)},{\sy*(0.0250)})
	--({\sx*(2.6600)},{\sy*(0.0265)})
	--({\sx*(2.6700)},{\sy*(0.0278)})
	--({\sx*(2.6800)},{\sy*(0.0291)})
	--({\sx*(2.6900)},{\sy*(0.0304)})
	--({\sx*(2.7000)},{\sy*(0.0316)})
	--({\sx*(2.7100)},{\sy*(0.0327)})
	--({\sx*(2.7200)},{\sy*(0.0337)})
	--({\sx*(2.7300)},{\sy*(0.0347)})
	--({\sx*(2.7400)},{\sy*(0.0356)})
	--({\sx*(2.7500)},{\sy*(0.0364)})
	--({\sx*(2.7600)},{\sy*(0.0371)})
	--({\sx*(2.7700)},{\sy*(0.0378)})
	--({\sx*(2.7800)},{\sy*(0.0383)})
	--({\sx*(2.7900)},{\sy*(0.0388)})
	--({\sx*(2.8000)},{\sy*(0.0392)})
	--({\sx*(2.8100)},{\sy*(0.0395)})
	--({\sx*(2.8200)},{\sy*(0.0397)})
	--({\sx*(2.8300)},{\sy*(0.0398)})
	--({\sx*(2.8400)},{\sy*(0.0397)})
	--({\sx*(2.8500)},{\sy*(0.0396)})
	--({\sx*(2.8600)},{\sy*(0.0394)})
	--({\sx*(2.8700)},{\sy*(0.0391)})
	--({\sx*(2.8800)},{\sy*(0.0387)})
	--({\sx*(2.8900)},{\sy*(0.0382)})
	--({\sx*(2.9000)},{\sy*(0.0376)})
	--({\sx*(2.9100)},{\sy*(0.0369)})
	--({\sx*(2.9200)},{\sy*(0.0361)})
	--({\sx*(2.9300)},{\sy*(0.0352)})
	--({\sx*(2.9400)},{\sy*(0.0342)})
	--({\sx*(2.9500)},{\sy*(0.0331)})
	--({\sx*(2.9600)},{\sy*(0.0319)})
	--({\sx*(2.9700)},{\sy*(0.0306)})
	--({\sx*(2.9800)},{\sy*(0.0292)})
	--({\sx*(2.9900)},{\sy*(0.0277)})
	--({\sx*(3.0000)},{\sy*(0.0261)})
	--({\sx*(3.0100)},{\sy*(0.0244)})
	--({\sx*(3.0200)},{\sy*(0.0227)})
	--({\sx*(3.0300)},{\sy*(0.0209)})
	--({\sx*(3.0400)},{\sy*(0.0189)})
	--({\sx*(3.0500)},{\sy*(0.0170)})
	--({\sx*(3.0600)},{\sy*(0.0149)})
	--({\sx*(3.0700)},{\sy*(0.0128)})
	--({\sx*(3.0800)},{\sy*(0.0106)})
	--({\sx*(3.0900)},{\sy*(0.0083)})
	--({\sx*(3.1000)},{\sy*(0.0060)})
	--({\sx*(3.1100)},{\sy*(0.0036)})
	--({\sx*(3.1200)},{\sy*(0.0012)})
	--({\sx*(3.1300)},{\sy*(-0.0012)})
	--({\sx*(3.1400)},{\sy*(-0.0037)})
	--({\sx*(3.1500)},{\sy*(-0.0062)})
	--({\sx*(3.1600)},{\sy*(-0.0088)})
	--({\sx*(3.1700)},{\sy*(-0.0114)})
	--({\sx*(3.1800)},{\sy*(-0.0140)})
	--({\sx*(3.1900)},{\sy*(-0.0166)})
	--({\sx*(3.2000)},{\sy*(-0.0192)})
	--({\sx*(3.2100)},{\sy*(-0.0218)})
	--({\sx*(3.2200)},{\sy*(-0.0244)})
	--({\sx*(3.2300)},{\sy*(-0.0269)})
	--({\sx*(3.2400)},{\sy*(-0.0295)})
	--({\sx*(3.2500)},{\sy*(-0.0320)})
	--({\sx*(3.2600)},{\sy*(-0.0345)})
	--({\sx*(3.2700)},{\sy*(-0.0370)})
	--({\sx*(3.2800)},{\sy*(-0.0394)})
	--({\sx*(3.2900)},{\sy*(-0.0418)})
	--({\sx*(3.3000)},{\sy*(-0.0441)})
	--({\sx*(3.3100)},{\sy*(-0.0463)})
	--({\sx*(3.3200)},{\sy*(-0.0485)})
	--({\sx*(3.3300)},{\sy*(-0.0505)})
	--({\sx*(3.3400)},{\sy*(-0.0525)})
	--({\sx*(3.3500)},{\sy*(-0.0544)})
	--({\sx*(3.3600)},{\sy*(-0.0562)})
	--({\sx*(3.3700)},{\sy*(-0.0579)})
	--({\sx*(3.3800)},{\sy*(-0.0595)})
	--({\sx*(3.3900)},{\sy*(-0.0610)})
	--({\sx*(3.4000)},{\sy*(-0.0623)})
	--({\sx*(3.4100)},{\sy*(-0.0636)})
	--({\sx*(3.4200)},{\sy*(-0.0646)})
	--({\sx*(3.4300)},{\sy*(-0.0656)})
	--({\sx*(3.4400)},{\sy*(-0.0664)})
	--({\sx*(3.4500)},{\sy*(-0.0670)})
	--({\sx*(3.4600)},{\sy*(-0.0675)})
	--({\sx*(3.4700)},{\sy*(-0.0678)})
	--({\sx*(3.4800)},{\sy*(-0.0679)})
	--({\sx*(3.4900)},{\sy*(-0.0679)})
	--({\sx*(3.5000)},{\sy*(-0.0677)})
	--({\sx*(3.5100)},{\sy*(-0.0673)})
	--({\sx*(3.5200)},{\sy*(-0.0668)})
	--({\sx*(3.5300)},{\sy*(-0.0660)})
	--({\sx*(3.5400)},{\sy*(-0.0651)})
	--({\sx*(3.5500)},{\sy*(-0.0639)})
	--({\sx*(3.5600)},{\sy*(-0.0626)})
	--({\sx*(3.5700)},{\sy*(-0.0611)})
	--({\sx*(3.5800)},{\sy*(-0.0593)})
	--({\sx*(3.5900)},{\sy*(-0.0574)})
	--({\sx*(3.6000)},{\sy*(-0.0553)})
	--({\sx*(3.6100)},{\sy*(-0.0530)})
	--({\sx*(3.6200)},{\sy*(-0.0504)})
	--({\sx*(3.6300)},{\sy*(-0.0477)})
	--({\sx*(3.6400)},{\sy*(-0.0448)})
	--({\sx*(3.6500)},{\sy*(-0.0416)})
	--({\sx*(3.6600)},{\sy*(-0.0383)})
	--({\sx*(3.6700)},{\sy*(-0.0348)})
	--({\sx*(3.6800)},{\sy*(-0.0311)})
	--({\sx*(3.6900)},{\sy*(-0.0271)})
	--({\sx*(3.7000)},{\sy*(-0.0231)})
	--({\sx*(3.7100)},{\sy*(-0.0188)})
	--({\sx*(3.7200)},{\sy*(-0.0143)})
	--({\sx*(3.7300)},{\sy*(-0.0097)})
	--({\sx*(3.7400)},{\sy*(-0.0049)})
	--({\sx*(3.7500)},{\sy*(0.0000)})
	--({\sx*(3.7600)},{\sy*(0.0051)})
	--({\sx*(3.7700)},{\sy*(0.0103)})
	--({\sx*(3.7800)},{\sy*(0.0157)})
	--({\sx*(3.7900)},{\sy*(0.0212)})
	--({\sx*(3.8000)},{\sy*(0.0268)})
	--({\sx*(3.8100)},{\sy*(0.0325)})
	--({\sx*(3.8200)},{\sy*(0.0383)})
	--({\sx*(3.8300)},{\sy*(0.0442)})
	--({\sx*(3.8400)},{\sy*(0.0502)})
	--({\sx*(3.8500)},{\sy*(0.0562)})
	--({\sx*(3.8600)},{\sy*(0.0623)})
	--({\sx*(3.8700)},{\sy*(0.0684)})
	--({\sx*(3.8800)},{\sy*(0.0746)})
	--({\sx*(3.8900)},{\sy*(0.0807)})
	--({\sx*(3.9000)},{\sy*(0.0868)})
	--({\sx*(3.9100)},{\sy*(0.0930)})
	--({\sx*(3.9200)},{\sy*(0.0990)})
	--({\sx*(3.9300)},{\sy*(0.1050)})
	--({\sx*(3.9400)},{\sy*(0.1110)})
	--({\sx*(3.9500)},{\sy*(0.1168)})
	--({\sx*(3.9600)},{\sy*(0.1226)})
	--({\sx*(3.9700)},{\sy*(0.1282)})
	--({\sx*(3.9800)},{\sy*(0.1337)})
	--({\sx*(3.9900)},{\sy*(0.1389)})
	--({\sx*(4.0000)},{\sy*(0.1441)})
	--({\sx*(4.0100)},{\sy*(0.1490)})
	--({\sx*(4.0200)},{\sy*(0.1536)})
	--({\sx*(4.0300)},{\sy*(0.1581)})
	--({\sx*(4.0400)},{\sy*(0.1622)})
	--({\sx*(4.0500)},{\sy*(0.1661)})
	--({\sx*(4.0600)},{\sy*(0.1696)})
	--({\sx*(4.0700)},{\sy*(0.1729)})
	--({\sx*(4.0800)},{\sy*(0.1757)})
	--({\sx*(4.0900)},{\sy*(0.1782)})
	--({\sx*(4.1000)},{\sy*(0.1803)})
	--({\sx*(4.1100)},{\sy*(0.1819)})
	--({\sx*(4.1200)},{\sy*(0.1831)})
	--({\sx*(4.1300)},{\sy*(0.1839)})
	--({\sx*(4.1400)},{\sy*(0.1841)})
	--({\sx*(4.1500)},{\sy*(0.1838)})
	--({\sx*(4.1600)},{\sy*(0.1830)})
	--({\sx*(4.1700)},{\sy*(0.1817)})
	--({\sx*(4.1800)},{\sy*(0.1797)})
	--({\sx*(4.1900)},{\sy*(0.1772)})
	--({\sx*(4.2000)},{\sy*(0.1740)})
	--({\sx*(4.2100)},{\sy*(0.1702)})
	--({\sx*(4.2200)},{\sy*(0.1657)})
	--({\sx*(4.2300)},{\sy*(0.1606)})
	--({\sx*(4.2400)},{\sy*(0.1548)})
	--({\sx*(4.2500)},{\sy*(0.1482)})
	--({\sx*(4.2600)},{\sy*(0.1409)})
	--({\sx*(4.2700)},{\sy*(0.1328)})
	--({\sx*(4.2800)},{\sy*(0.1240)})
	--({\sx*(4.2900)},{\sy*(0.1144)})
	--({\sx*(4.3000)},{\sy*(0.1041)})
	--({\sx*(4.3100)},{\sy*(0.0929)})
	--({\sx*(4.3200)},{\sy*(0.0809)})
	--({\sx*(4.3300)},{\sy*(0.0680)})
	--({\sx*(4.3400)},{\sy*(0.0544)})
	--({\sx*(4.3500)},{\sy*(0.0399)})
	--({\sx*(4.3600)},{\sy*(0.0246)})
	--({\sx*(4.3700)},{\sy*(0.0084)})
	--({\sx*(4.3800)},{\sy*(-0.0086)})
	--({\sx*(4.3900)},{\sy*(-0.0265)})
	--({\sx*(4.4000)},{\sy*(-0.0451)})
	--({\sx*(4.4100)},{\sy*(-0.0647)})
	--({\sx*(4.4200)},{\sy*(-0.0850)})
	--({\sx*(4.4300)},{\sy*(-0.1061)})
	--({\sx*(4.4400)},{\sy*(-0.1281)})
	--({\sx*(4.4500)},{\sy*(-0.1508)})
	--({\sx*(4.4600)},{\sy*(-0.1743)})
	--({\sx*(4.4700)},{\sy*(-0.1985)})
	--({\sx*(4.4800)},{\sy*(-0.2234)})
	--({\sx*(4.4900)},{\sy*(-0.2490)})
	--({\sx*(4.5000)},{\sy*(-0.2752)})
	--({\sx*(4.5100)},{\sy*(-0.3020)})
	--({\sx*(4.5200)},{\sy*(-0.3294)})
	--({\sx*(4.5300)},{\sy*(-0.3573)})
	--({\sx*(4.5400)},{\sy*(-0.3857)})
	--({\sx*(4.5500)},{\sy*(-0.4145)})
	--({\sx*(4.5600)},{\sy*(-0.4437)})
	--({\sx*(4.5700)},{\sy*(-0.4732)})
	--({\sx*(4.5800)},{\sy*(-0.5029)})
	--({\sx*(4.5900)},{\sy*(-0.5327)})
	--({\sx*(4.6000)},{\sy*(-0.5626)})
	--({\sx*(4.6100)},{\sy*(-0.5925)})
	--({\sx*(4.6200)},{\sy*(-0.6223)})
	--({\sx*(4.6300)},{\sy*(-0.6519)})
	--({\sx*(4.6400)},{\sy*(-0.6813)})
	--({\sx*(4.6500)},{\sy*(-0.7102)})
	--({\sx*(4.6600)},{\sy*(-0.7385)})
	--({\sx*(4.6700)},{\sy*(-0.7663)})
	--({\sx*(4.6800)},{\sy*(-0.7933)})
	--({\sx*(4.6900)},{\sy*(-0.8193)})
	--({\sx*(4.7000)},{\sy*(-0.8443)})
	--({\sx*(4.7100)},{\sy*(-0.8681)})
	--({\sx*(4.7200)},{\sy*(-0.8905)})
	--({\sx*(4.7300)},{\sy*(-0.9114)})
	--({\sx*(4.7400)},{\sy*(-0.9306)})
	--({\sx*(4.7500)},{\sy*(-0.9479)})
	--({\sx*(4.7600)},{\sy*(-0.9630)})
	--({\sx*(4.7700)},{\sy*(-0.9759)})
	--({\sx*(4.7800)},{\sy*(-0.9863)})
	--({\sx*(4.7900)},{\sy*(-0.9939)})
	--({\sx*(4.8000)},{\sy*(-0.9986)})
	--({\sx*(4.8100)},{\sy*(-1.0000)})
	--({\sx*(4.8200)},{\sy*(-0.9980)})
	--({\sx*(4.8300)},{\sy*(-0.9922)})
	--({\sx*(4.8400)},{\sy*(-0.9824)})
	--({\sx*(4.8500)},{\sy*(-0.9682)})
	--({\sx*(4.8600)},{\sy*(-0.9495)})
	--({\sx*(4.8700)},{\sy*(-0.9258)})
	--({\sx*(4.8800)},{\sy*(-0.8969)})
	--({\sx*(4.8900)},{\sy*(-0.8623)})
	--({\sx*(4.9000)},{\sy*(-0.8218)})
	--({\sx*(4.9100)},{\sy*(-0.7750)})
	--({\sx*(4.9200)},{\sy*(-0.7214)})
	--({\sx*(4.9300)},{\sy*(-0.6608)})
	--({\sx*(4.9400)},{\sy*(-0.5926)})
	--({\sx*(4.9500)},{\sy*(-0.5164)})
	--({\sx*(4.9600)},{\sy*(-0.4318)})
	--({\sx*(4.9700)},{\sy*(-0.3384)})
	--({\sx*(4.9800)},{\sy*(-0.2356)})
	--({\sx*(4.9900)},{\sy*(-0.1230)})
	--({\sx*(5.0000)},{\sy*(0.0000)});
}
\def\relfehlerd{
\draw[color=blue,line width=1.4pt,line join=round] ({\sx*(0.000)},{\sy*(0.0000)})
	--({\sx*(0.0100)},{\sy*(-0.0002)})
	--({\sx*(0.0200)},{\sy*(-0.0005)})
	--({\sx*(0.0300)},{\sy*(-0.0007)})
	--({\sx*(0.0400)},{\sy*(-0.0008)})
	--({\sx*(0.0500)},{\sy*(-0.0010)})
	--({\sx*(0.0600)},{\sy*(-0.0011)})
	--({\sx*(0.0700)},{\sy*(-0.0012)})
	--({\sx*(0.0800)},{\sy*(-0.0013)})
	--({\sx*(0.0900)},{\sy*(-0.0014)})
	--({\sx*(0.1000)},{\sy*(-0.0015)})
	--({\sx*(0.1100)},{\sy*(-0.0016)})
	--({\sx*(0.1200)},{\sy*(-0.0016)})
	--({\sx*(0.1300)},{\sy*(-0.0016)})
	--({\sx*(0.1400)},{\sy*(-0.0016)})
	--({\sx*(0.1500)},{\sy*(-0.0017)})
	--({\sx*(0.1600)},{\sy*(-0.0017)})
	--({\sx*(0.1700)},{\sy*(-0.0016)})
	--({\sx*(0.1800)},{\sy*(-0.0016)})
	--({\sx*(0.1900)},{\sy*(-0.0016)})
	--({\sx*(0.2000)},{\sy*(-0.0016)})
	--({\sx*(0.2100)},{\sy*(-0.0016)})
	--({\sx*(0.2200)},{\sy*(-0.0015)})
	--({\sx*(0.2300)},{\sy*(-0.0015)})
	--({\sx*(0.2400)},{\sy*(-0.0015)})
	--({\sx*(0.2500)},{\sy*(-0.0014)})
	--({\sx*(0.2600)},{\sy*(-0.0014)})
	--({\sx*(0.2700)},{\sy*(-0.0013)})
	--({\sx*(0.2800)},{\sy*(-0.0013)})
	--({\sx*(0.2900)},{\sy*(-0.0012)})
	--({\sx*(0.3000)},{\sy*(-0.0012)})
	--({\sx*(0.3100)},{\sy*(-0.0011)})
	--({\sx*(0.3200)},{\sy*(-0.0011)})
	--({\sx*(0.3300)},{\sy*(-0.0010)})
	--({\sx*(0.3400)},{\sy*(-0.0009)})
	--({\sx*(0.3500)},{\sy*(-0.0009)})
	--({\sx*(0.3600)},{\sy*(-0.0008)})
	--({\sx*(0.3700)},{\sy*(-0.0008)})
	--({\sx*(0.3800)},{\sy*(-0.0007)})
	--({\sx*(0.3900)},{\sy*(-0.0007)})
	--({\sx*(0.4000)},{\sy*(-0.0006)})
	--({\sx*(0.4100)},{\sy*(-0.0006)})
	--({\sx*(0.4200)},{\sy*(-0.0006)})
	--({\sx*(0.4300)},{\sy*(-0.0005)})
	--({\sx*(0.4400)},{\sy*(-0.0005)})
	--({\sx*(0.4500)},{\sy*(-0.0004)})
	--({\sx*(0.4600)},{\sy*(-0.0004)})
	--({\sx*(0.4700)},{\sy*(-0.0004)})
	--({\sx*(0.4800)},{\sy*(-0.0003)})
	--({\sx*(0.4900)},{\sy*(-0.0003)})
	--({\sx*(0.5000)},{\sy*(-0.0002)})
	--({\sx*(0.5100)},{\sy*(-0.0002)})
	--({\sx*(0.5200)},{\sy*(-0.0002)})
	--({\sx*(0.5300)},{\sy*(-0.0002)})
	--({\sx*(0.5400)},{\sy*(-0.0001)})
	--({\sx*(0.5500)},{\sy*(-0.0001)})
	--({\sx*(0.5600)},{\sy*(-0.0001)})
	--({\sx*(0.5700)},{\sy*(-0.0001)})
	--({\sx*(0.5800)},{\sy*(-0.0001)})
	--({\sx*(0.5900)},{\sy*(-0.0000)})
	--({\sx*(0.6000)},{\sy*(-0.0000)})
	--({\sx*(0.6100)},{\sy*(-0.0000)})
	--({\sx*(0.6200)},{\sy*(-0.0000)})
	--({\sx*(0.6300)},{\sy*(0.0000)})
	--({\sx*(0.6400)},{\sy*(0.0000)})
	--({\sx*(0.6500)},{\sy*(0.0000)})
	--({\sx*(0.6600)},{\sy*(0.0000)})
	--({\sx*(0.6700)},{\sy*(0.0000)})
	--({\sx*(0.6800)},{\sy*(0.0000)})
	--({\sx*(0.6900)},{\sy*(0.0000)})
	--({\sx*(0.7000)},{\sy*(0.0000)})
	--({\sx*(0.7100)},{\sy*(0.0000)})
	--({\sx*(0.7200)},{\sy*(0.0000)})
	--({\sx*(0.7300)},{\sy*(0.0000)})
	--({\sx*(0.7400)},{\sy*(0.0000)})
	--({\sx*(0.7500)},{\sy*(0.0000)})
	--({\sx*(0.7600)},{\sy*(0.0000)})
	--({\sx*(0.7700)},{\sy*(0.0000)})
	--({\sx*(0.7800)},{\sy*(0.0000)})
	--({\sx*(0.7900)},{\sy*(0.0000)})
	--({\sx*(0.8000)},{\sy*(0.0000)})
	--({\sx*(0.8100)},{\sy*(0.0000)})
	--({\sx*(0.8200)},{\sy*(-0.0000)})
	--({\sx*(0.8300)},{\sy*(-0.0000)})
	--({\sx*(0.8400)},{\sy*(-0.0000)})
	--({\sx*(0.8500)},{\sy*(-0.0000)})
	--({\sx*(0.8600)},{\sy*(-0.0000)})
	--({\sx*(0.8700)},{\sy*(-0.0000)})
	--({\sx*(0.8800)},{\sy*(-0.0000)})
	--({\sx*(0.8900)},{\sy*(-0.0001)})
	--({\sx*(0.9000)},{\sy*(-0.0001)})
	--({\sx*(0.9100)},{\sy*(-0.0001)})
	--({\sx*(0.9200)},{\sy*(-0.0001)})
	--({\sx*(0.9300)},{\sy*(-0.0001)})
	--({\sx*(0.9400)},{\sy*(-0.0001)})
	--({\sx*(0.9500)},{\sy*(-0.0001)})
	--({\sx*(0.9600)},{\sy*(-0.0001)})
	--({\sx*(0.9700)},{\sy*(-0.0001)})
	--({\sx*(0.9800)},{\sy*(-0.0001)})
	--({\sx*(0.9900)},{\sy*(-0.0001)})
	--({\sx*(1.0000)},{\sy*(-0.0001)})
	--({\sx*(1.0100)},{\sy*(-0.0001)})
	--({\sx*(1.0200)},{\sy*(-0.0001)})
	--({\sx*(1.0300)},{\sy*(-0.0001)})
	--({\sx*(1.0400)},{\sy*(-0.0001)})
	--({\sx*(1.0500)},{\sy*(-0.0001)})
	--({\sx*(1.0600)},{\sy*(-0.0001)})
	--({\sx*(1.0700)},{\sy*(-0.0001)})
	--({\sx*(1.0800)},{\sy*(-0.0001)})
	--({\sx*(1.0900)},{\sy*(-0.0001)})
	--({\sx*(1.1000)},{\sy*(-0.0001)})
	--({\sx*(1.1100)},{\sy*(-0.0001)})
	--({\sx*(1.1200)},{\sy*(-0.0001)})
	--({\sx*(1.1300)},{\sy*(-0.0001)})
	--({\sx*(1.1400)},{\sy*(-0.0001)})
	--({\sx*(1.1500)},{\sy*(-0.0001)})
	--({\sx*(1.1600)},{\sy*(-0.0001)})
	--({\sx*(1.1700)},{\sy*(-0.0001)})
	--({\sx*(1.1800)},{\sy*(-0.0001)})
	--({\sx*(1.1900)},{\sy*(-0.0001)})
	--({\sx*(1.2000)},{\sy*(-0.0001)})
	--({\sx*(1.2100)},{\sy*(-0.0000)})
	--({\sx*(1.2200)},{\sy*(-0.0000)})
	--({\sx*(1.2300)},{\sy*(-0.0000)})
	--({\sx*(1.2400)},{\sy*(-0.0000)})
	--({\sx*(1.2500)},{\sy*(0.0000)})
	--({\sx*(1.2600)},{\sy*(0.0000)})
	--({\sx*(1.2700)},{\sy*(0.0000)})
	--({\sx*(1.2800)},{\sy*(0.0000)})
	--({\sx*(1.2900)},{\sy*(0.0000)})
	--({\sx*(1.3000)},{\sy*(0.0001)})
	--({\sx*(1.3100)},{\sy*(0.0001)})
	--({\sx*(1.3200)},{\sy*(0.0001)})
	--({\sx*(1.3300)},{\sy*(0.0001)})
	--({\sx*(1.3400)},{\sy*(0.0001)})
	--({\sx*(1.3500)},{\sy*(0.0001)})
	--({\sx*(1.3600)},{\sy*(0.0001)})
	--({\sx*(1.3700)},{\sy*(0.0002)})
	--({\sx*(1.3800)},{\sy*(0.0002)})
	--({\sx*(1.3900)},{\sy*(0.0002)})
	--({\sx*(1.4000)},{\sy*(0.0002)})
	--({\sx*(1.4100)},{\sy*(0.0002)})
	--({\sx*(1.4200)},{\sy*(0.0002)})
	--({\sx*(1.4300)},{\sy*(0.0002)})
	--({\sx*(1.4400)},{\sy*(0.0003)})
	--({\sx*(1.4500)},{\sy*(0.0003)})
	--({\sx*(1.4600)},{\sy*(0.0003)})
	--({\sx*(1.4700)},{\sy*(0.0003)})
	--({\sx*(1.4800)},{\sy*(0.0003)})
	--({\sx*(1.4900)},{\sy*(0.0003)})
	--({\sx*(1.5000)},{\sy*(0.0003)})
	--({\sx*(1.5100)},{\sy*(0.0003)})
	--({\sx*(1.5200)},{\sy*(0.0004)})
	--({\sx*(1.5300)},{\sy*(0.0004)})
	--({\sx*(1.5400)},{\sy*(0.0004)})
	--({\sx*(1.5500)},{\sy*(0.0004)})
	--({\sx*(1.5600)},{\sy*(0.0004)})
	--({\sx*(1.5700)},{\sy*(0.0004)})
	--({\sx*(1.5800)},{\sy*(0.0004)})
	--({\sx*(1.5900)},{\sy*(0.0004)})
	--({\sx*(1.6000)},{\sy*(0.0004)})
	--({\sx*(1.6100)},{\sy*(0.0004)})
	--({\sx*(1.6200)},{\sy*(0.0004)})
	--({\sx*(1.6300)},{\sy*(0.0004)})
	--({\sx*(1.6400)},{\sy*(0.0004)})
	--({\sx*(1.6500)},{\sy*(0.0004)})
	--({\sx*(1.6600)},{\sy*(0.0004)})
	--({\sx*(1.6700)},{\sy*(0.0004)})
	--({\sx*(1.6800)},{\sy*(0.0004)})
	--({\sx*(1.6900)},{\sy*(0.0004)})
	--({\sx*(1.7000)},{\sy*(0.0004)})
	--({\sx*(1.7100)},{\sy*(0.0004)})
	--({\sx*(1.7200)},{\sy*(0.0004)})
	--({\sx*(1.7300)},{\sy*(0.0003)})
	--({\sx*(1.7400)},{\sy*(0.0003)})
	--({\sx*(1.7500)},{\sy*(0.0003)})
	--({\sx*(1.7600)},{\sy*(0.0003)})
	--({\sx*(1.7700)},{\sy*(0.0003)})
	--({\sx*(1.7800)},{\sy*(0.0003)})
	--({\sx*(1.7900)},{\sy*(0.0002)})
	--({\sx*(1.8000)},{\sy*(0.0002)})
	--({\sx*(1.8100)},{\sy*(0.0002)})
	--({\sx*(1.8200)},{\sy*(0.0002)})
	--({\sx*(1.8300)},{\sy*(0.0001)})
	--({\sx*(1.8400)},{\sy*(0.0001)})
	--({\sx*(1.8500)},{\sy*(0.0001)})
	--({\sx*(1.8600)},{\sy*(0.0000)})
	--({\sx*(1.8700)},{\sy*(0.0000)})
	--({\sx*(1.8800)},{\sy*(-0.0000)})
	--({\sx*(1.8900)},{\sy*(-0.0001)})
	--({\sx*(1.9000)},{\sy*(-0.0001)})
	--({\sx*(1.9100)},{\sy*(-0.0001)})
	--({\sx*(1.9200)},{\sy*(-0.0002)})
	--({\sx*(1.9300)},{\sy*(-0.0002)})
	--({\sx*(1.9400)},{\sy*(-0.0002)})
	--({\sx*(1.9500)},{\sy*(-0.0003)})
	--({\sx*(1.9600)},{\sy*(-0.0003)})
	--({\sx*(1.9700)},{\sy*(-0.0004)})
	--({\sx*(1.9800)},{\sy*(-0.0004)})
	--({\sx*(1.9900)},{\sy*(-0.0005)})
	--({\sx*(2.0000)},{\sy*(-0.0005)})
	--({\sx*(2.0100)},{\sy*(-0.0006)})
	--({\sx*(2.0200)},{\sy*(-0.0006)})
	--({\sx*(2.0300)},{\sy*(-0.0007)})
	--({\sx*(2.0400)},{\sy*(-0.0007)})
	--({\sx*(2.0500)},{\sy*(-0.0007)})
	--({\sx*(2.0600)},{\sy*(-0.0008)})
	--({\sx*(2.0700)},{\sy*(-0.0008)})
	--({\sx*(2.0800)},{\sy*(-0.0009)})
	--({\sx*(2.0900)},{\sy*(-0.0009)})
	--({\sx*(2.1000)},{\sy*(-0.0010)})
	--({\sx*(2.1100)},{\sy*(-0.0010)})
	--({\sx*(2.1200)},{\sy*(-0.0011)})
	--({\sx*(2.1300)},{\sy*(-0.0011)})
	--({\sx*(2.1400)},{\sy*(-0.0011)})
	--({\sx*(2.1500)},{\sy*(-0.0012)})
	--({\sx*(2.1600)},{\sy*(-0.0012)})
	--({\sx*(2.1700)},{\sy*(-0.0013)})
	--({\sx*(2.1800)},{\sy*(-0.0013)})
	--({\sx*(2.1900)},{\sy*(-0.0013)})
	--({\sx*(2.2000)},{\sy*(-0.0013)})
	--({\sx*(2.2100)},{\sy*(-0.0014)})
	--({\sx*(2.2200)},{\sy*(-0.0014)})
	--({\sx*(2.2300)},{\sy*(-0.0014)})
	--({\sx*(2.2400)},{\sy*(-0.0014)})
	--({\sx*(2.2500)},{\sy*(-0.0014)})
	--({\sx*(2.2600)},{\sy*(-0.0015)})
	--({\sx*(2.2700)},{\sy*(-0.0015)})
	--({\sx*(2.2800)},{\sy*(-0.0015)})
	--({\sx*(2.2900)},{\sy*(-0.0015)})
	--({\sx*(2.3000)},{\sy*(-0.0015)})
	--({\sx*(2.3100)},{\sy*(-0.0014)})
	--({\sx*(2.3200)},{\sy*(-0.0014)})
	--({\sx*(2.3300)},{\sy*(-0.0014)})
	--({\sx*(2.3400)},{\sy*(-0.0014)})
	--({\sx*(2.3500)},{\sy*(-0.0013)})
	--({\sx*(2.3600)},{\sy*(-0.0013)})
	--({\sx*(2.3700)},{\sy*(-0.0012)})
	--({\sx*(2.3800)},{\sy*(-0.0012)})
	--({\sx*(2.3900)},{\sy*(-0.0011)})
	--({\sx*(2.4000)},{\sy*(-0.0011)})
	--({\sx*(2.4100)},{\sy*(-0.0010)})
	--({\sx*(2.4200)},{\sy*(-0.0009)})
	--({\sx*(2.4300)},{\sy*(-0.0008)})
	--({\sx*(2.4400)},{\sy*(-0.0007)})
	--({\sx*(2.4500)},{\sy*(-0.0006)})
	--({\sx*(2.4600)},{\sy*(-0.0005)})
	--({\sx*(2.4700)},{\sy*(-0.0004)})
	--({\sx*(2.4800)},{\sy*(-0.0003)})
	--({\sx*(2.4900)},{\sy*(-0.0001)})
	--({\sx*(2.5000)},{\sy*(0.0000)})
	--({\sx*(2.5100)},{\sy*(0.0002)})
	--({\sx*(2.5200)},{\sy*(0.0003)})
	--({\sx*(2.5300)},{\sy*(0.0005)})
	--({\sx*(2.5400)},{\sy*(0.0007)})
	--({\sx*(2.5500)},{\sy*(0.0008)})
	--({\sx*(2.5600)},{\sy*(0.0010)})
	--({\sx*(2.5700)},{\sy*(0.0012)})
	--({\sx*(2.5800)},{\sy*(0.0014)})
	--({\sx*(2.5900)},{\sy*(0.0016)})
	--({\sx*(2.6000)},{\sy*(0.0019)})
	--({\sx*(2.6100)},{\sy*(0.0021)})
	--({\sx*(2.6200)},{\sy*(0.0023)})
	--({\sx*(2.6300)},{\sy*(0.0026)})
	--({\sx*(2.6400)},{\sy*(0.0028)})
	--({\sx*(2.6500)},{\sy*(0.0031)})
	--({\sx*(2.6600)},{\sy*(0.0034)})
	--({\sx*(2.6700)},{\sy*(0.0036)})
	--({\sx*(2.6800)},{\sy*(0.0039)})
	--({\sx*(2.6900)},{\sy*(0.0042)})
	--({\sx*(2.7000)},{\sy*(0.0044)})
	--({\sx*(2.7100)},{\sy*(0.0047)})
	--({\sx*(2.7200)},{\sy*(0.0050)})
	--({\sx*(2.7300)},{\sy*(0.0053)})
	--({\sx*(2.7400)},{\sy*(0.0056)})
	--({\sx*(2.7500)},{\sy*(0.0059)})
	--({\sx*(2.7600)},{\sy*(0.0062)})
	--({\sx*(2.7700)},{\sy*(0.0064)})
	--({\sx*(2.7800)},{\sy*(0.0067)})
	--({\sx*(2.7900)},{\sy*(0.0070)})
	--({\sx*(2.8000)},{\sy*(0.0073)})
	--({\sx*(2.8100)},{\sy*(0.0075)})
	--({\sx*(2.8200)},{\sy*(0.0078)})
	--({\sx*(2.8300)},{\sy*(0.0080)})
	--({\sx*(2.8400)},{\sy*(0.0082)})
	--({\sx*(2.8500)},{\sy*(0.0084)})
	--({\sx*(2.8600)},{\sy*(0.0086)})
	--({\sx*(2.8700)},{\sy*(0.0088)})
	--({\sx*(2.8800)},{\sy*(0.0090)})
	--({\sx*(2.8900)},{\sy*(0.0091)})
	--({\sx*(2.9000)},{\sy*(0.0092)})
	--({\sx*(2.9100)},{\sy*(0.0093)})
	--({\sx*(2.9200)},{\sy*(0.0094)})
	--({\sx*(2.9300)},{\sy*(0.0094)})
	--({\sx*(2.9400)},{\sy*(0.0094)})
	--({\sx*(2.9500)},{\sy*(0.0094)})
	--({\sx*(2.9600)},{\sy*(0.0093)})
	--({\sx*(2.9700)},{\sy*(0.0092)})
	--({\sx*(2.9800)},{\sy*(0.0091)})
	--({\sx*(2.9900)},{\sy*(0.0089)})
	--({\sx*(3.0000)},{\sy*(0.0086)})
	--({\sx*(3.0100)},{\sy*(0.0083)})
	--({\sx*(3.0200)},{\sy*(0.0080)})
	--({\sx*(3.0300)},{\sy*(0.0075)})
	--({\sx*(3.0400)},{\sy*(0.0071)})
	--({\sx*(3.0500)},{\sy*(0.0065)})
	--({\sx*(3.0600)},{\sy*(0.0059)})
	--({\sx*(3.0700)},{\sy*(0.0052)})
	--({\sx*(3.0800)},{\sy*(0.0045)})
	--({\sx*(3.0900)},{\sy*(0.0036)})
	--({\sx*(3.1000)},{\sy*(0.0027)})
	--({\sx*(3.1100)},{\sy*(0.0017)})
	--({\sx*(3.1200)},{\sy*(0.0006)})
	--({\sx*(3.1300)},{\sy*(-0.0006)})
	--({\sx*(3.1400)},{\sy*(-0.0019)})
	--({\sx*(3.1500)},{\sy*(-0.0033)})
	--({\sx*(3.1600)},{\sy*(-0.0048)})
	--({\sx*(3.1700)},{\sy*(-0.0064)})
	--({\sx*(3.1800)},{\sy*(-0.0082)})
	--({\sx*(3.1900)},{\sy*(-0.0100)})
	--({\sx*(3.2000)},{\sy*(-0.0120)})
	--({\sx*(3.2100)},{\sy*(-0.0141)})
	--({\sx*(3.2200)},{\sy*(-0.0163)})
	--({\sx*(3.2300)},{\sy*(-0.0187)})
	--({\sx*(3.2400)},{\sy*(-0.0212)})
	--({\sx*(3.2500)},{\sy*(-0.0238)})
	--({\sx*(3.2600)},{\sy*(-0.0266)})
	--({\sx*(3.2700)},{\sy*(-0.0296)})
	--({\sx*(3.2800)},{\sy*(-0.0326)})
	--({\sx*(3.2900)},{\sy*(-0.0358)})
	--({\sx*(3.3000)},{\sy*(-0.0392)})
	--({\sx*(3.3100)},{\sy*(-0.0427)})
	--({\sx*(3.3200)},{\sy*(-0.0464)})
	--({\sx*(3.3300)},{\sy*(-0.0502)})
	--({\sx*(3.3400)},{\sy*(-0.0542)})
	--({\sx*(3.3500)},{\sy*(-0.0583)})
	--({\sx*(3.3600)},{\sy*(-0.0625)})
	--({\sx*(3.3700)},{\sy*(-0.0669)})
	--({\sx*(3.3800)},{\sy*(-0.0714)})
	--({\sx*(3.3900)},{\sy*(-0.0760)})
	--({\sx*(3.4000)},{\sy*(-0.0807)})
	--({\sx*(3.4100)},{\sy*(-0.0855)})
	--({\sx*(3.4200)},{\sy*(-0.0903)})
	--({\sx*(3.4300)},{\sy*(-0.0953)})
	--({\sx*(3.4400)},{\sy*(-0.1002)})
	--({\sx*(3.4500)},{\sy*(-0.1052)})
	--({\sx*(3.4600)},{\sy*(-0.1102)})
	--({\sx*(3.4700)},{\sy*(-0.1151)})
	--({\sx*(3.4800)},{\sy*(-0.1200)})
	--({\sx*(3.4900)},{\sy*(-0.1247)})
	--({\sx*(3.5000)},{\sy*(-0.1293)})
	--({\sx*(3.5100)},{\sy*(-0.1336)})
	--({\sx*(3.5200)},{\sy*(-0.1378)})
	--({\sx*(3.5300)},{\sy*(-0.1416)})
	--({\sx*(3.5400)},{\sy*(-0.1450)})
	--({\sx*(3.5500)},{\sy*(-0.1480)})
	--({\sx*(3.5600)},{\sy*(-0.1505)})
	--({\sx*(3.5700)},{\sy*(-0.1524)})
	--({\sx*(3.5800)},{\sy*(-0.1536)})
	--({\sx*(3.5900)},{\sy*(-0.1541)})
	--({\sx*(3.6000)},{\sy*(-0.1538)})
	--({\sx*(3.6100)},{\sy*(-0.1526)})
	--({\sx*(3.6200)},{\sy*(-0.1503)})
	--({\sx*(3.6300)},{\sy*(-0.1470)})
	--({\sx*(3.6400)},{\sy*(-0.1425)})
	--({\sx*(3.6500)},{\sy*(-0.1367)})
	--({\sx*(3.6600)},{\sy*(-0.1297)})
	--({\sx*(3.6700)},{\sy*(-0.1212)})
	--({\sx*(3.6800)},{\sy*(-0.1113)})
	--({\sx*(3.6900)},{\sy*(-0.0999)})
	--({\sx*(3.7000)},{\sy*(-0.0870)})
	--({\sx*(3.7100)},{\sy*(-0.0726)})
	--({\sx*(3.7200)},{\sy*(-0.0567)})
	--({\sx*(3.7300)},{\sy*(-0.0392)})
	--({\sx*(3.7400)},{\sy*(-0.0203)})
	--({\sx*(3.7500)},{\sy*(0.0000)})
	--({\sx*(3.7600)},{\sy*(0.0216)})
	--({\sx*(3.7700)},{\sy*(0.0445)})
	--({\sx*(3.7800)},{\sy*(0.0685)})
	--({\sx*(3.7900)},{\sy*(0.0934)})
	--({\sx*(3.8000)},{\sy*(0.1193)})
	--({\sx*(3.8100)},{\sy*(0.1458)})
	--({\sx*(3.8200)},{\sy*(0.1729)})
	--({\sx*(3.8300)},{\sy*(0.2004)})
	--({\sx*(3.8400)},{\sy*(0.2281)})
	--({\sx*(3.8500)},{\sy*(0.2560)})
	--({\sx*(3.8600)},{\sy*(0.2838)})
	--({\sx*(3.8700)},{\sy*(0.3114)})
	--({\sx*(3.8800)},{\sy*(0.3388)})
	--({\sx*(3.8900)},{\sy*(0.3657)})
	--({\sx*(3.9000)},{\sy*(0.3921)})
	--({\sx*(3.9100)},{\sy*(0.4179)})
	--({\sx*(3.9200)},{\sy*(0.4430)})
	--({\sx*(3.9300)},{\sy*(0.4674)})
	--({\sx*(3.9400)},{\sy*(0.4909)})
	--({\sx*(3.9500)},{\sy*(0.5136)})
	--({\sx*(3.9600)},{\sy*(0.5354)})
	--({\sx*(3.9700)},{\sy*(0.5563)})
	--({\sx*(3.9800)},{\sy*(0.5763)})
	--({\sx*(3.9900)},{\sy*(0.5954)})
	--({\sx*(4.0000)},{\sy*(0.6136)})
	--({\sx*(4.0100)},{\sy*(0.6309)})
	--({\sx*(4.0200)},{\sy*(0.6473)})
	--({\sx*(4.0300)},{\sy*(0.6628)})
	--({\sx*(4.0400)},{\sy*(0.6775)})
	--({\sx*(4.0500)},{\sy*(0.6913)})
	--({\sx*(4.0600)},{\sy*(0.7043)})
	--({\sx*(4.0700)},{\sy*(0.7165)})
	--({\sx*(4.0800)},{\sy*(0.7280)})
	--({\sx*(4.0900)},{\sy*(0.7387)})
	--({\sx*(4.1000)},{\sy*(0.7487)})
	--({\sx*(4.1100)},{\sy*(0.7580)})
	--({\sx*(4.1200)},{\sy*(0.7667)})
	--({\sx*(4.1300)},{\sy*(0.7747)})
	--({\sx*(4.1400)},{\sy*(0.7821)})
	--({\sx*(4.1500)},{\sy*(0.7888)})
	--({\sx*(4.1600)},{\sy*(0.7949)})
	--({\sx*(4.1700)},{\sy*(0.8004)})
	--({\sx*(4.1800)},{\sy*(0.8053)})
	--({\sx*(4.1900)},{\sy*(0.8096)})
	--({\sx*(4.2000)},{\sy*(0.8133)})
	--({\sx*(4.2100)},{\sy*(0.8163)})
	--({\sx*(4.2200)},{\sy*(0.8186)})
	--({\sx*(4.2300)},{\sy*(0.8202)})
	--({\sx*(4.2400)},{\sy*(0.8210)})
	--({\sx*(4.2500)},{\sy*(0.8208)})
	--({\sx*(4.2600)},{\sy*(0.8197)})
	--({\sx*(4.2700)},{\sy*(0.8173)})
	--({\sx*(4.2800)},{\sy*(0.8134)})
	--({\sx*(4.2900)},{\sy*(0.8076)})
	--({\sx*(4.3000)},{\sy*(0.7994)})
	--({\sx*(4.3100)},{\sy*(0.7878)})
	--({\sx*(4.3200)},{\sy*(0.7714)})
	--({\sx*(4.3300)},{\sy*(0.7478)})
	--({\sx*(4.3400)},{\sy*(0.7122)})
	--({\sx*(4.3500)},{\sy*(0.6547)})
	--({\sx*(4.3600)},{\sy*(0.5495)})
	--({\sx*(4.3700)},{\sy*(0.3034)})
	--({\sx*(4.3800)},{\sy*(-0.8745)})
	--({\sx*(4.3900)},{\sy*(3.0084)})
	--({\sx*(4.4000)},{\sy*(1.5986)})
	--({\sx*(4.4100)},{\sy*(1.3337)})
	--({\sx*(4.4200)},{\sy*(1.2227)})
	--({\sx*(4.4300)},{\sy*(1.1621)})
	--({\sx*(4.4400)},{\sy*(1.1244)})
	--({\sx*(4.4500)},{\sy*(1.0987)})
	--({\sx*(4.4600)},{\sy*(1.0803)})
	--({\sx*(4.4700)},{\sy*(1.0666)})
	--({\sx*(4.4800)},{\sy*(1.0560)})
	--({\sx*(4.4900)},{\sy*(1.0477)})
	--({\sx*(4.5000)},{\sy*(1.0410)})
	--({\sx*(4.5100)},{\sy*(1.0355)})
	--({\sx*(4.5200)},{\sy*(1.0310)})
	--({\sx*(4.5300)},{\sy*(1.0272)})
	--({\sx*(4.5400)},{\sy*(1.0240)})
	--({\sx*(4.5500)},{\sy*(1.0213)})
	--({\sx*(4.5600)},{\sy*(1.0190)})
	--({\sx*(4.5700)},{\sy*(1.0170)})
	--({\sx*(4.5800)},{\sy*(1.0152)})
	--({\sx*(4.5900)},{\sy*(1.0137)})
	--({\sx*(4.6000)},{\sy*(1.0124)})
	--({\sx*(4.6100)},{\sy*(1.0112)})
	--({\sx*(4.6200)},{\sy*(1.0102)})
	--({\sx*(4.6300)},{\sy*(1.0093)})
	--({\sx*(4.6400)},{\sy*(1.0085)})
	--({\sx*(4.6500)},{\sy*(1.0077)})
	--({\sx*(4.6600)},{\sy*(1.0071)})
	--({\sx*(4.6700)},{\sy*(1.0065)})
	--({\sx*(4.6800)},{\sy*(1.0060)})
	--({\sx*(4.6900)},{\sy*(1.0056)})
	--({\sx*(4.7000)},{\sy*(1.0051)})
	--({\sx*(4.7100)},{\sy*(1.0048)})
	--({\sx*(4.7200)},{\sy*(1.0044)})
	--({\sx*(4.7300)},{\sy*(1.0041)})
	--({\sx*(4.7400)},{\sy*(1.0039)})
	--({\sx*(4.7500)},{\sy*(1.0036)})
	--({\sx*(4.7600)},{\sy*(1.0034)})
	--({\sx*(4.7700)},{\sy*(1.0032)})
	--({\sx*(4.7800)},{\sy*(1.0030)})
	--({\sx*(4.7900)},{\sy*(1.0028)})
	--({\sx*(4.8000)},{\sy*(1.0027)})
	--({\sx*(4.8100)},{\sy*(1.0026)})
	--({\sx*(4.8200)},{\sy*(1.0024)})
	--({\sx*(4.8300)},{\sy*(1.0023)})
	--({\sx*(4.8400)},{\sy*(1.0023)})
	--({\sx*(4.8500)},{\sy*(1.0022)})
	--({\sx*(4.8600)},{\sy*(1.0021)})
	--({\sx*(4.8700)},{\sy*(1.0021)})
	--({\sx*(4.8800)},{\sy*(1.0020)})
	--({\sx*(4.8900)},{\sy*(1.0020)})
	--({\sx*(4.9000)},{\sy*(1.0020)})
	--({\sx*(4.9100)},{\sy*(1.0020)})
	--({\sx*(4.9200)},{\sy*(1.0021)})
	--({\sx*(4.9300)},{\sy*(1.0022)})
	--({\sx*(4.9400)},{\sy*(1.0023)})
	--({\sx*(4.9500)},{\sy*(1.0025)})
	--({\sx*(4.9600)},{\sy*(1.0029)})
	--({\sx*(4.9700)},{\sy*(1.0035)})
	--({\sx*(4.9800)},{\sy*(1.0047)})
	--({\sx*(4.9900)},{\sy*(1.0087)})
	--({\sx*(5.0000)},{\sy*(0.0000)});
}
\def\xwertee{
\fill[color=red] (0.0000,0) circle[radius={0.07/\skala}];
\fill[color=red] (0.5000,0) circle[radius={0.07/\skala}];
\fill[color=red] (1.0000,0) circle[radius={0.07/\skala}];
\fill[color=red] (1.5000,0) circle[radius={0.07/\skala}];
\fill[color=red] (2.0000,0) circle[radius={0.07/\skala}];
\fill[color=red] (2.5000,0) circle[radius={0.07/\skala}];
\fill[color=red] (3.0000,0) circle[radius={0.07/\skala}];
\fill[color=red] (3.5000,0) circle[radius={0.07/\skala}];
\fill[color=red] (4.0000,0) circle[radius={0.07/\skala}];
\fill[color=red] (4.5000,0) circle[radius={0.07/\skala}];
\fill[color=red] (5.0000,0) circle[radius={0.07/\skala}];
}
\def\punktee{10}
\def\maxfehlere{4.281\cdot 10^{-4}}
\def\fehlere{
\draw[color=red,line width=1.4pt,line join=round] ({\sx*(0.000)},{\sy*(0.0000)})
	--({\sx*(0.0100)},{\sy*(-0.1533)})
	--({\sx*(0.0200)},{\sy*(-0.2903)})
	--({\sx*(0.0300)},{\sy*(-0.4120)})
	--({\sx*(0.0400)},{\sy*(-0.5194)})
	--({\sx*(0.0500)},{\sy*(-0.6135)})
	--({\sx*(0.0600)},{\sy*(-0.6951)})
	--({\sx*(0.0700)},{\sy*(-0.7651)})
	--({\sx*(0.0800)},{\sy*(-0.8242)})
	--({\sx*(0.0900)},{\sy*(-0.8734)})
	--({\sx*(0.1000)},{\sy*(-0.9132)})
	--({\sx*(0.1100)},{\sy*(-0.9444)})
	--({\sx*(0.1200)},{\sy*(-0.9677)})
	--({\sx*(0.1300)},{\sy*(-0.9838)})
	--({\sx*(0.1400)},{\sy*(-0.9931)})
	--({\sx*(0.1500)},{\sy*(-0.9964)})
	--({\sx*(0.1600)},{\sy*(-0.9941)})
	--({\sx*(0.1700)},{\sy*(-0.9868)})
	--({\sx*(0.1800)},{\sy*(-0.9749)})
	--({\sx*(0.1900)},{\sy*(-0.9590)})
	--({\sx*(0.2000)},{\sy*(-0.9394)})
	--({\sx*(0.2100)},{\sy*(-0.9166)})
	--({\sx*(0.2200)},{\sy*(-0.8910)})
	--({\sx*(0.2300)},{\sy*(-0.8628)})
	--({\sx*(0.2400)},{\sy*(-0.8326)})
	--({\sx*(0.2500)},{\sy*(-0.8006)})
	--({\sx*(0.2600)},{\sy*(-0.7670)})
	--({\sx*(0.2700)},{\sy*(-0.7322)})
	--({\sx*(0.2800)},{\sy*(-0.6965)})
	--({\sx*(0.2900)},{\sy*(-0.6600)})
	--({\sx*(0.3000)},{\sy*(-0.6231)})
	--({\sx*(0.3100)},{\sy*(-0.5858)})
	--({\sx*(0.3200)},{\sy*(-0.5485)})
	--({\sx*(0.3300)},{\sy*(-0.5112)})
	--({\sx*(0.3400)},{\sy*(-0.4742)})
	--({\sx*(0.3500)},{\sy*(-0.4376)})
	--({\sx*(0.3600)},{\sy*(-0.4015)})
	--({\sx*(0.3700)},{\sy*(-0.3661)})
	--({\sx*(0.3800)},{\sy*(-0.3314)})
	--({\sx*(0.3900)},{\sy*(-0.2976)})
	--({\sx*(0.4000)},{\sy*(-0.2647)})
	--({\sx*(0.4100)},{\sy*(-0.2328)})
	--({\sx*(0.4200)},{\sy*(-0.2021)})
	--({\sx*(0.4300)},{\sy*(-0.1724)})
	--({\sx*(0.4400)},{\sy*(-0.1440)})
	--({\sx*(0.4500)},{\sy*(-0.1168)})
	--({\sx*(0.4600)},{\sy*(-0.0908)})
	--({\sx*(0.4700)},{\sy*(-0.0661)})
	--({\sx*(0.4800)},{\sy*(-0.0428)})
	--({\sx*(0.4900)},{\sy*(-0.0207)})
	--({\sx*(0.5000)},{\sy*(0.0000)})
	--({\sx*(0.5100)},{\sy*(0.0194)})
	--({\sx*(0.5200)},{\sy*(0.0375)})
	--({\sx*(0.5300)},{\sy*(0.0543)})
	--({\sx*(0.5400)},{\sy*(0.0697)})
	--({\sx*(0.5500)},{\sy*(0.0839)})
	--({\sx*(0.5600)},{\sy*(0.0969)})
	--({\sx*(0.5700)},{\sy*(0.1086)})
	--({\sx*(0.5800)},{\sy*(0.1191)})
	--({\sx*(0.5900)},{\sy*(0.1285)})
	--({\sx*(0.6000)},{\sy*(0.1367)})
	--({\sx*(0.6100)},{\sy*(0.1439)})
	--({\sx*(0.6200)},{\sy*(0.1499)})
	--({\sx*(0.6300)},{\sy*(0.1550)})
	--({\sx*(0.6400)},{\sy*(0.1590)})
	--({\sx*(0.6500)},{\sy*(0.1621)})
	--({\sx*(0.6600)},{\sy*(0.1644)})
	--({\sx*(0.6700)},{\sy*(0.1657)})
	--({\sx*(0.6800)},{\sy*(0.1663)})
	--({\sx*(0.6900)},{\sy*(0.1661)})
	--({\sx*(0.7000)},{\sy*(0.1652)})
	--({\sx*(0.7100)},{\sy*(0.1636)})
	--({\sx*(0.7200)},{\sy*(0.1614)})
	--({\sx*(0.7300)},{\sy*(0.1586)})
	--({\sx*(0.7400)},{\sy*(0.1552)})
	--({\sx*(0.7500)},{\sy*(0.1514)})
	--({\sx*(0.7600)},{\sy*(0.1471)})
	--({\sx*(0.7700)},{\sy*(0.1424)})
	--({\sx*(0.7800)},{\sy*(0.1373)})
	--({\sx*(0.7900)},{\sy*(0.1319)})
	--({\sx*(0.8000)},{\sy*(0.1261)})
	--({\sx*(0.8100)},{\sy*(0.1202)})
	--({\sx*(0.8200)},{\sy*(0.1140)})
	--({\sx*(0.8300)},{\sy*(0.1076)})
	--({\sx*(0.8400)},{\sy*(0.1011)})
	--({\sx*(0.8500)},{\sy*(0.0945)})
	--({\sx*(0.8600)},{\sy*(0.0878)})
	--({\sx*(0.8700)},{\sy*(0.0810)})
	--({\sx*(0.8800)},{\sy*(0.0743)})
	--({\sx*(0.8900)},{\sy*(0.0675)})
	--({\sx*(0.9000)},{\sy*(0.0608)})
	--({\sx*(0.9100)},{\sy*(0.0541)})
	--({\sx*(0.9200)},{\sy*(0.0475)})
	--({\sx*(0.9300)},{\sy*(0.0410)})
	--({\sx*(0.9400)},{\sy*(0.0346)})
	--({\sx*(0.9500)},{\sy*(0.0284)})
	--({\sx*(0.9600)},{\sy*(0.0223)})
	--({\sx*(0.9700)},{\sy*(0.0165)})
	--({\sx*(0.9800)},{\sy*(0.0108)})
	--({\sx*(0.9900)},{\sy*(0.0053)})
	--({\sx*(1.0000)},{\sy*(0.0000)})
	--({\sx*(1.0100)},{\sy*(-0.0050)})
	--({\sx*(1.0200)},{\sy*(-0.0099)})
	--({\sx*(1.0300)},{\sy*(-0.0144)})
	--({\sx*(1.0400)},{\sy*(-0.0187)})
	--({\sx*(1.0500)},{\sy*(-0.0228)})
	--({\sx*(1.0600)},{\sy*(-0.0266)})
	--({\sx*(1.0700)},{\sy*(-0.0301)})
	--({\sx*(1.0800)},{\sy*(-0.0334)})
	--({\sx*(1.0900)},{\sy*(-0.0363)})
	--({\sx*(1.1000)},{\sy*(-0.0391)})
	--({\sx*(1.1100)},{\sy*(-0.0415)})
	--({\sx*(1.1200)},{\sy*(-0.0437)})
	--({\sx*(1.1300)},{\sy*(-0.0456)})
	--({\sx*(1.1400)},{\sy*(-0.0473)})
	--({\sx*(1.1500)},{\sy*(-0.0487)})
	--({\sx*(1.1600)},{\sy*(-0.0498)})
	--({\sx*(1.1700)},{\sy*(-0.0507)})
	--({\sx*(1.1800)},{\sy*(-0.0514)})
	--({\sx*(1.1900)},{\sy*(-0.0518)})
	--({\sx*(1.2000)},{\sy*(-0.0520)})
	--({\sx*(1.2100)},{\sy*(-0.0520)})
	--({\sx*(1.2200)},{\sy*(-0.0518)})
	--({\sx*(1.2300)},{\sy*(-0.0513)})
	--({\sx*(1.2400)},{\sy*(-0.0507)})
	--({\sx*(1.2500)},{\sy*(-0.0499)})
	--({\sx*(1.2600)},{\sy*(-0.0489)})
	--({\sx*(1.2700)},{\sy*(-0.0478)})
	--({\sx*(1.2800)},{\sy*(-0.0465)})
	--({\sx*(1.2900)},{\sy*(-0.0451)})
	--({\sx*(1.3000)},{\sy*(-0.0435)})
	--({\sx*(1.3100)},{\sy*(-0.0418)})
	--({\sx*(1.3200)},{\sy*(-0.0400)})
	--({\sx*(1.3300)},{\sy*(-0.0381)})
	--({\sx*(1.3400)},{\sy*(-0.0361)})
	--({\sx*(1.3500)},{\sy*(-0.0340)})
	--({\sx*(1.3600)},{\sy*(-0.0319)})
	--({\sx*(1.3700)},{\sy*(-0.0297)})
	--({\sx*(1.3800)},{\sy*(-0.0274)})
	--({\sx*(1.3900)},{\sy*(-0.0252)})
	--({\sx*(1.4000)},{\sy*(-0.0228)})
	--({\sx*(1.4100)},{\sy*(-0.0205)})
	--({\sx*(1.4200)},{\sy*(-0.0182)})
	--({\sx*(1.4300)},{\sy*(-0.0158)})
	--({\sx*(1.4400)},{\sy*(-0.0135)})
	--({\sx*(1.4500)},{\sy*(-0.0111)})
	--({\sx*(1.4600)},{\sy*(-0.0088)})
	--({\sx*(1.4700)},{\sy*(-0.0066)})
	--({\sx*(1.4800)},{\sy*(-0.0043)})
	--({\sx*(1.4900)},{\sy*(-0.0021)})
	--({\sx*(1.5000)},{\sy*(0.0000)})
	--({\sx*(1.5100)},{\sy*(0.0021)})
	--({\sx*(1.5200)},{\sy*(0.0041)})
	--({\sx*(1.5300)},{\sy*(0.0060)})
	--({\sx*(1.5400)},{\sy*(0.0079)})
	--({\sx*(1.5500)},{\sy*(0.0097)})
	--({\sx*(1.5600)},{\sy*(0.0114)})
	--({\sx*(1.5700)},{\sy*(0.0130)})
	--({\sx*(1.5800)},{\sy*(0.0145)})
	--({\sx*(1.5900)},{\sy*(0.0159)})
	--({\sx*(1.6000)},{\sy*(0.0172)})
	--({\sx*(1.6100)},{\sy*(0.0184)})
	--({\sx*(1.6200)},{\sy*(0.0196)})
	--({\sx*(1.6300)},{\sy*(0.0206)})
	--({\sx*(1.6400)},{\sy*(0.0215)})
	--({\sx*(1.6500)},{\sy*(0.0223)})
	--({\sx*(1.6600)},{\sy*(0.0230)})
	--({\sx*(1.6700)},{\sy*(0.0236)})
	--({\sx*(1.6800)},{\sy*(0.0241)})
	--({\sx*(1.6900)},{\sy*(0.0245)})
	--({\sx*(1.7000)},{\sy*(0.0247)})
	--({\sx*(1.7100)},{\sy*(0.0249)})
	--({\sx*(1.7200)},{\sy*(0.0250)})
	--({\sx*(1.7300)},{\sy*(0.0250)})
	--({\sx*(1.7400)},{\sy*(0.0248)})
	--({\sx*(1.7500)},{\sy*(0.0246)})
	--({\sx*(1.7600)},{\sy*(0.0243)})
	--({\sx*(1.7700)},{\sy*(0.0239)})
	--({\sx*(1.7800)},{\sy*(0.0234)})
	--({\sx*(1.7900)},{\sy*(0.0229)})
	--({\sx*(1.8000)},{\sy*(0.0222)})
	--({\sx*(1.8100)},{\sy*(0.0215)})
	--({\sx*(1.8200)},{\sy*(0.0208)})
	--({\sx*(1.8300)},{\sy*(0.0199)})
	--({\sx*(1.8400)},{\sy*(0.0190)})
	--({\sx*(1.8500)},{\sy*(0.0180)})
	--({\sx*(1.8600)},{\sy*(0.0170)})
	--({\sx*(1.8700)},{\sy*(0.0160)})
	--({\sx*(1.8800)},{\sy*(0.0149)})
	--({\sx*(1.8900)},{\sy*(0.0137)})
	--({\sx*(1.9000)},{\sy*(0.0125)})
	--({\sx*(1.9100)},{\sy*(0.0113)})
	--({\sx*(1.9200)},{\sy*(0.0101)})
	--({\sx*(1.9300)},{\sy*(0.0089)})
	--({\sx*(1.9400)},{\sy*(0.0076)})
	--({\sx*(1.9500)},{\sy*(0.0063)})
	--({\sx*(1.9600)},{\sy*(0.0051)})
	--({\sx*(1.9700)},{\sy*(0.0038)})
	--({\sx*(1.9800)},{\sy*(0.0025)})
	--({\sx*(1.9900)},{\sy*(0.0012)})
	--({\sx*(2.0000)},{\sy*(0.0000)})
	--({\sx*(2.0100)},{\sy*(-0.0012)})
	--({\sx*(2.0200)},{\sy*(-0.0024)})
	--({\sx*(2.0300)},{\sy*(-0.0036)})
	--({\sx*(2.0400)},{\sy*(-0.0048)})
	--({\sx*(2.0500)},{\sy*(-0.0059)})
	--({\sx*(2.0600)},{\sy*(-0.0070)})
	--({\sx*(2.0700)},{\sy*(-0.0080)})
	--({\sx*(2.0800)},{\sy*(-0.0090)})
	--({\sx*(2.0900)},{\sy*(-0.0099)})
	--({\sx*(2.1000)},{\sy*(-0.0108)})
	--({\sx*(2.1100)},{\sy*(-0.0117)})
	--({\sx*(2.1200)},{\sy*(-0.0125)})
	--({\sx*(2.1300)},{\sy*(-0.0132)})
	--({\sx*(2.1400)},{\sy*(-0.0139)})
	--({\sx*(2.1500)},{\sy*(-0.0145)})
	--({\sx*(2.1600)},{\sy*(-0.0151)})
	--({\sx*(2.1700)},{\sy*(-0.0156)})
	--({\sx*(2.1800)},{\sy*(-0.0160)})
	--({\sx*(2.1900)},{\sy*(-0.0164)})
	--({\sx*(2.2000)},{\sy*(-0.0166)})
	--({\sx*(2.2100)},{\sy*(-0.0169)})
	--({\sx*(2.2200)},{\sy*(-0.0170)})
	--({\sx*(2.2300)},{\sy*(-0.0171)})
	--({\sx*(2.2400)},{\sy*(-0.0172)})
	--({\sx*(2.2500)},{\sy*(-0.0171)})
	--({\sx*(2.2600)},{\sy*(-0.0170)})
	--({\sx*(2.2700)},{\sy*(-0.0169)})
	--({\sx*(2.2800)},{\sy*(-0.0166)})
	--({\sx*(2.2900)},{\sy*(-0.0163)})
	--({\sx*(2.3000)},{\sy*(-0.0160)})
	--({\sx*(2.3100)},{\sy*(-0.0156)})
	--({\sx*(2.3200)},{\sy*(-0.0151)})
	--({\sx*(2.3300)},{\sy*(-0.0146)})
	--({\sx*(2.3400)},{\sy*(-0.0140)})
	--({\sx*(2.3500)},{\sy*(-0.0134)})
	--({\sx*(2.3600)},{\sy*(-0.0127)})
	--({\sx*(2.3700)},{\sy*(-0.0120)})
	--({\sx*(2.3800)},{\sy*(-0.0113)})
	--({\sx*(2.3900)},{\sy*(-0.0105)})
	--({\sx*(2.4000)},{\sy*(-0.0096)})
	--({\sx*(2.4100)},{\sy*(-0.0088)})
	--({\sx*(2.4200)},{\sy*(-0.0079)})
	--({\sx*(2.4300)},{\sy*(-0.0069)})
	--({\sx*(2.4400)},{\sy*(-0.0060)})
	--({\sx*(2.4500)},{\sy*(-0.0050)})
	--({\sx*(2.4600)},{\sy*(-0.0040)})
	--({\sx*(2.4700)},{\sy*(-0.0030)})
	--({\sx*(2.4800)},{\sy*(-0.0020)})
	--({\sx*(2.4900)},{\sy*(-0.0010)})
	--({\sx*(2.5000)},{\sy*(0.0000)})
	--({\sx*(2.5100)},{\sy*(0.0010)})
	--({\sx*(2.5200)},{\sy*(0.0020)})
	--({\sx*(2.5300)},{\sy*(0.0030)})
	--({\sx*(2.5400)},{\sy*(0.0040)})
	--({\sx*(2.5500)},{\sy*(0.0050)})
	--({\sx*(2.5600)},{\sy*(0.0059)})
	--({\sx*(2.5700)},{\sy*(0.0069)})
	--({\sx*(2.5800)},{\sy*(0.0078)})
	--({\sx*(2.5900)},{\sy*(0.0086)})
	--({\sx*(2.6000)},{\sy*(0.0095)})
	--({\sx*(2.6100)},{\sy*(0.0103)})
	--({\sx*(2.6200)},{\sy*(0.0111)})
	--({\sx*(2.6300)},{\sy*(0.0118)})
	--({\sx*(2.6400)},{\sy*(0.0125)})
	--({\sx*(2.6500)},{\sy*(0.0131)})
	--({\sx*(2.6600)},{\sy*(0.0137)})
	--({\sx*(2.6700)},{\sy*(0.0143)})
	--({\sx*(2.6800)},{\sy*(0.0147)})
	--({\sx*(2.6900)},{\sy*(0.0152)})
	--({\sx*(2.7000)},{\sy*(0.0156)})
	--({\sx*(2.7100)},{\sy*(0.0159)})
	--({\sx*(2.7200)},{\sy*(0.0161)})
	--({\sx*(2.7300)},{\sy*(0.0163)})
	--({\sx*(2.7400)},{\sy*(0.0165)})
	--({\sx*(2.7500)},{\sy*(0.0165)})
	--({\sx*(2.7600)},{\sy*(0.0166)})
	--({\sx*(2.7700)},{\sy*(0.0165)})
	--({\sx*(2.7800)},{\sy*(0.0164)})
	--({\sx*(2.7900)},{\sy*(0.0162)})
	--({\sx*(2.8000)},{\sy*(0.0160)})
	--({\sx*(2.8100)},{\sy*(0.0157)})
	--({\sx*(2.8200)},{\sy*(0.0153)})
	--({\sx*(2.8300)},{\sy*(0.0149)})
	--({\sx*(2.8400)},{\sy*(0.0144)})
	--({\sx*(2.8500)},{\sy*(0.0138)})
	--({\sx*(2.8600)},{\sy*(0.0132)})
	--({\sx*(2.8700)},{\sy*(0.0126)})
	--({\sx*(2.8800)},{\sy*(0.0119)})
	--({\sx*(2.8900)},{\sy*(0.0111)})
	--({\sx*(2.9000)},{\sy*(0.0103)})
	--({\sx*(2.9100)},{\sy*(0.0094)})
	--({\sx*(2.9200)},{\sy*(0.0085)})
	--({\sx*(2.9300)},{\sy*(0.0075)})
	--({\sx*(2.9400)},{\sy*(0.0066)})
	--({\sx*(2.9500)},{\sy*(0.0055)})
	--({\sx*(2.9600)},{\sy*(0.0045)})
	--({\sx*(2.9700)},{\sy*(0.0034)})
	--({\sx*(2.9800)},{\sy*(0.0023)})
	--({\sx*(2.9900)},{\sy*(0.0012)})
	--({\sx*(3.0000)},{\sy*(0.0000)})
	--({\sx*(3.0100)},{\sy*(-0.0012)})
	--({\sx*(3.0200)},{\sy*(-0.0023)})
	--({\sx*(3.0300)},{\sy*(-0.0035)})
	--({\sx*(3.0400)},{\sy*(-0.0047)})
	--({\sx*(3.0500)},{\sy*(-0.0059)})
	--({\sx*(3.0600)},{\sy*(-0.0071)})
	--({\sx*(3.0700)},{\sy*(-0.0082)})
	--({\sx*(3.0800)},{\sy*(-0.0094)})
	--({\sx*(3.0900)},{\sy*(-0.0105)})
	--({\sx*(3.1000)},{\sy*(-0.0116)})
	--({\sx*(3.1100)},{\sy*(-0.0127)})
	--({\sx*(3.1200)},{\sy*(-0.0137)})
	--({\sx*(3.1300)},{\sy*(-0.0147)})
	--({\sx*(3.1400)},{\sy*(-0.0157)})
	--({\sx*(3.1500)},{\sy*(-0.0166)})
	--({\sx*(3.1600)},{\sy*(-0.0175)})
	--({\sx*(3.1700)},{\sy*(-0.0183)})
	--({\sx*(3.1800)},{\sy*(-0.0190)})
	--({\sx*(3.1900)},{\sy*(-0.0197)})
	--({\sx*(3.2000)},{\sy*(-0.0204)})
	--({\sx*(3.2100)},{\sy*(-0.0209)})
	--({\sx*(3.2200)},{\sy*(-0.0214)})
	--({\sx*(3.2300)},{\sy*(-0.0218)})
	--({\sx*(3.2400)},{\sy*(-0.0222)})
	--({\sx*(3.2500)},{\sy*(-0.0224)})
	--({\sx*(3.2600)},{\sy*(-0.0226)})
	--({\sx*(3.2700)},{\sy*(-0.0227)})
	--({\sx*(3.2800)},{\sy*(-0.0227)})
	--({\sx*(3.2900)},{\sy*(-0.0226)})
	--({\sx*(3.3000)},{\sy*(-0.0224)})
	--({\sx*(3.3100)},{\sy*(-0.0221)})
	--({\sx*(3.3200)},{\sy*(-0.0218)})
	--({\sx*(3.3300)},{\sy*(-0.0213)})
	--({\sx*(3.3400)},{\sy*(-0.0208)})
	--({\sx*(3.3500)},{\sy*(-0.0201)})
	--({\sx*(3.3600)},{\sy*(-0.0194)})
	--({\sx*(3.3700)},{\sy*(-0.0185)})
	--({\sx*(3.3800)},{\sy*(-0.0176)})
	--({\sx*(3.3900)},{\sy*(-0.0166)})
	--({\sx*(3.4000)},{\sy*(-0.0155)})
	--({\sx*(3.4100)},{\sy*(-0.0143)})
	--({\sx*(3.4200)},{\sy*(-0.0130)})
	--({\sx*(3.4300)},{\sy*(-0.0116)})
	--({\sx*(3.4400)},{\sy*(-0.0102)})
	--({\sx*(3.4500)},{\sy*(-0.0087)})
	--({\sx*(3.4600)},{\sy*(-0.0071)})
	--({\sx*(3.4700)},{\sy*(-0.0054)})
	--({\sx*(3.4800)},{\sy*(-0.0037)})
	--({\sx*(3.4900)},{\sy*(-0.0019)})
	--({\sx*(3.5000)},{\sy*(0.0000)})
	--({\sx*(3.5100)},{\sy*(0.0019)})
	--({\sx*(3.5200)},{\sy*(0.0039)})
	--({\sx*(3.5300)},{\sy*(0.0058)})
	--({\sx*(3.5400)},{\sy*(0.0079)})
	--({\sx*(3.5500)},{\sy*(0.0099)})
	--({\sx*(3.5600)},{\sy*(0.0120)})
	--({\sx*(3.5700)},{\sy*(0.0141)})
	--({\sx*(3.5800)},{\sy*(0.0161)})
	--({\sx*(3.5900)},{\sy*(0.0182)})
	--({\sx*(3.6000)},{\sy*(0.0203)})
	--({\sx*(3.6100)},{\sy*(0.0224)})
	--({\sx*(3.6200)},{\sy*(0.0244)})
	--({\sx*(3.6300)},{\sy*(0.0264)})
	--({\sx*(3.6400)},{\sy*(0.0283)})
	--({\sx*(3.6500)},{\sy*(0.0302)})
	--({\sx*(3.6600)},{\sy*(0.0320)})
	--({\sx*(3.6700)},{\sy*(0.0338)})
	--({\sx*(3.6800)},{\sy*(0.0355)})
	--({\sx*(3.6900)},{\sy*(0.0371)})
	--({\sx*(3.7000)},{\sy*(0.0385)})
	--({\sx*(3.7100)},{\sy*(0.0399)})
	--({\sx*(3.7200)},{\sy*(0.0412)})
	--({\sx*(3.7300)},{\sy*(0.0423)})
	--({\sx*(3.7400)},{\sy*(0.0433)})
	--({\sx*(3.7500)},{\sy*(0.0442)})
	--({\sx*(3.7600)},{\sy*(0.0449)})
	--({\sx*(3.7700)},{\sy*(0.0454)})
	--({\sx*(3.7800)},{\sy*(0.0458)})
	--({\sx*(3.7900)},{\sy*(0.0460)})
	--({\sx*(3.8000)},{\sy*(0.0460)})
	--({\sx*(3.8100)},{\sy*(0.0459)})
	--({\sx*(3.8200)},{\sy*(0.0455)})
	--({\sx*(3.8300)},{\sy*(0.0449)})
	--({\sx*(3.8400)},{\sy*(0.0441)})
	--({\sx*(3.8500)},{\sy*(0.0431)})
	--({\sx*(3.8600)},{\sy*(0.0418)})
	--({\sx*(3.8700)},{\sy*(0.0404)})
	--({\sx*(3.8800)},{\sy*(0.0387)})
	--({\sx*(3.8900)},{\sy*(0.0368)})
	--({\sx*(3.9000)},{\sy*(0.0346)})
	--({\sx*(3.9100)},{\sy*(0.0322)})
	--({\sx*(3.9200)},{\sy*(0.0296)})
	--({\sx*(3.9300)},{\sy*(0.0267)})
	--({\sx*(3.9400)},{\sy*(0.0236)})
	--({\sx*(3.9500)},{\sy*(0.0202)})
	--({\sx*(3.9600)},{\sy*(0.0166)})
	--({\sx*(3.9700)},{\sy*(0.0128)})
	--({\sx*(3.9800)},{\sy*(0.0087)})
	--({\sx*(3.9900)},{\sy*(0.0045)})
	--({\sx*(4.0000)},{\sy*(0.0000)})
	--({\sx*(4.0100)},{\sy*(-0.0047)})
	--({\sx*(4.0200)},{\sy*(-0.0096)})
	--({\sx*(4.0300)},{\sy*(-0.0146)})
	--({\sx*(4.0400)},{\sy*(-0.0199)})
	--({\sx*(4.0500)},{\sy*(-0.0253)})
	--({\sx*(4.0600)},{\sy*(-0.0308)})
	--({\sx*(4.0700)},{\sy*(-0.0365)})
	--({\sx*(4.0800)},{\sy*(-0.0423)})
	--({\sx*(4.0900)},{\sy*(-0.0483)})
	--({\sx*(4.1000)},{\sy*(-0.0542)})
	--({\sx*(4.1100)},{\sy*(-0.0603)})
	--({\sx*(4.1200)},{\sy*(-0.0664)})
	--({\sx*(4.1300)},{\sy*(-0.0725)})
	--({\sx*(4.1400)},{\sy*(-0.0786)})
	--({\sx*(4.1500)},{\sy*(-0.0846)})
	--({\sx*(4.1600)},{\sy*(-0.0906)})
	--({\sx*(4.1700)},{\sy*(-0.0965)})
	--({\sx*(4.1800)},{\sy*(-0.1023)})
	--({\sx*(4.1900)},{\sy*(-0.1080)})
	--({\sx*(4.2000)},{\sy*(-0.1134)})
	--({\sx*(4.2100)},{\sy*(-0.1187)})
	--({\sx*(4.2200)},{\sy*(-0.1236)})
	--({\sx*(4.2300)},{\sy*(-0.1283)})
	--({\sx*(4.2400)},{\sy*(-0.1327)})
	--({\sx*(4.2500)},{\sy*(-0.1367)})
	--({\sx*(4.2600)},{\sy*(-0.1403)})
	--({\sx*(4.2700)},{\sy*(-0.1435)})
	--({\sx*(4.2800)},{\sy*(-0.1462)})
	--({\sx*(4.2900)},{\sy*(-0.1484)})
	--({\sx*(4.3000)},{\sy*(-0.1500)})
	--({\sx*(4.3100)},{\sy*(-0.1510)})
	--({\sx*(4.3200)},{\sy*(-0.1513)})
	--({\sx*(4.3300)},{\sy*(-0.1510)})
	--({\sx*(4.3400)},{\sy*(-0.1499)})
	--({\sx*(4.3500)},{\sy*(-0.1481)})
	--({\sx*(4.3600)},{\sy*(-0.1454)})
	--({\sx*(4.3700)},{\sy*(-0.1419)})
	--({\sx*(4.3800)},{\sy*(-0.1374)})
	--({\sx*(4.3900)},{\sy*(-0.1320)})
	--({\sx*(4.4000)},{\sy*(-0.1257)})
	--({\sx*(4.4100)},{\sy*(-0.1183)})
	--({\sx*(4.4200)},{\sy*(-0.1098)})
	--({\sx*(4.4300)},{\sy*(-0.1003)})
	--({\sx*(4.4400)},{\sy*(-0.0896)})
	--({\sx*(4.4500)},{\sy*(-0.0777)})
	--({\sx*(4.4600)},{\sy*(-0.0646)})
	--({\sx*(4.4700)},{\sy*(-0.0504)})
	--({\sx*(4.4800)},{\sy*(-0.0349)})
	--({\sx*(4.4900)},{\sy*(-0.0181)})
	--({\sx*(4.5000)},{\sy*(0.0000)})
	--({\sx*(4.5100)},{\sy*(0.0194)})
	--({\sx*(4.5200)},{\sy*(0.0400)})
	--({\sx*(4.5300)},{\sy*(0.0620)})
	--({\sx*(4.5400)},{\sy*(0.0853)})
	--({\sx*(4.5500)},{\sy*(0.1098)})
	--({\sx*(4.5600)},{\sy*(0.1357)})
	--({\sx*(4.5700)},{\sy*(0.1628)})
	--({\sx*(4.5800)},{\sy*(0.1911)})
	--({\sx*(4.5900)},{\sy*(0.2206)})
	--({\sx*(4.6000)},{\sy*(0.2513)})
	--({\sx*(4.6100)},{\sy*(0.2830)})
	--({\sx*(4.6200)},{\sy*(0.3158)})
	--({\sx*(4.6300)},{\sy*(0.3495)})
	--({\sx*(4.6400)},{\sy*(0.3841)})
	--({\sx*(4.6500)},{\sy*(0.4195)})
	--({\sx*(4.6600)},{\sy*(0.4555)})
	--({\sx*(4.6700)},{\sy*(0.4921)})
	--({\sx*(4.6800)},{\sy*(0.5290)})
	--({\sx*(4.6900)},{\sy*(0.5662)})
	--({\sx*(4.7000)},{\sy*(0.6035)})
	--({\sx*(4.7100)},{\sy*(0.6407)})
	--({\sx*(4.7200)},{\sy*(0.6776)})
	--({\sx*(4.7300)},{\sy*(0.7139)})
	--({\sx*(4.7400)},{\sy*(0.7495)})
	--({\sx*(4.7500)},{\sy*(0.7841)})
	--({\sx*(4.7600)},{\sy*(0.8174)})
	--({\sx*(4.7700)},{\sy*(0.8490)})
	--({\sx*(4.7800)},{\sy*(0.8788)})
	--({\sx*(4.7900)},{\sy*(0.9062)})
	--({\sx*(4.8000)},{\sy*(0.9310)})
	--({\sx*(4.8100)},{\sy*(0.9528)})
	--({\sx*(4.8200)},{\sy*(0.9710)})
	--({\sx*(4.8300)},{\sy*(0.9853)})
	--({\sx*(4.8400)},{\sy*(0.9951)})
	--({\sx*(4.8500)},{\sy*(1.0000)})
	--({\sx*(4.8600)},{\sy*(0.9993)})
	--({\sx*(4.8700)},{\sy*(0.9925)})
	--({\sx*(4.8800)},{\sy*(0.9789)})
	--({\sx*(4.8900)},{\sy*(0.9579)})
	--({\sx*(4.9000)},{\sy*(0.9288)})
	--({\sx*(4.9100)},{\sy*(0.8907)})
	--({\sx*(4.9200)},{\sy*(0.8429)})
	--({\sx*(4.9300)},{\sy*(0.7846)})
	--({\sx*(4.9400)},{\sy*(0.7149)})
	--({\sx*(4.9500)},{\sy*(0.6328)})
	--({\sx*(4.9600)},{\sy*(0.5373)})
	--({\sx*(4.9700)},{\sy*(0.4274)})
	--({\sx*(4.9800)},{\sy*(0.3020)})
	--({\sx*(4.9900)},{\sy*(0.1599)})
	--({\sx*(5.0000)},{\sy*(0.0000)});
}
\def\relfehlere{
\draw[color=blue,line width=1.4pt,line join=round] ({\sx*(0.000)},{\sy*(0.0000)})
	--({\sx*(0.0100)},{\sy*(-0.0002)})
	--({\sx*(0.0200)},{\sy*(-0.0003)})
	--({\sx*(0.0300)},{\sy*(-0.0004)})
	--({\sx*(0.0400)},{\sy*(-0.0006)})
	--({\sx*(0.0500)},{\sy*(-0.0007)})
	--({\sx*(0.0600)},{\sy*(-0.0007)})
	--({\sx*(0.0700)},{\sy*(-0.0008)})
	--({\sx*(0.0800)},{\sy*(-0.0009)})
	--({\sx*(0.0900)},{\sy*(-0.0009)})
	--({\sx*(0.1000)},{\sy*(-0.0010)})
	--({\sx*(0.1100)},{\sy*(-0.0010)})
	--({\sx*(0.1200)},{\sy*(-0.0010)})
	--({\sx*(0.1300)},{\sy*(-0.0011)})
	--({\sx*(0.1400)},{\sy*(-0.0011)})
	--({\sx*(0.1500)},{\sy*(-0.0011)})
	--({\sx*(0.1600)},{\sy*(-0.0011)})
	--({\sx*(0.1700)},{\sy*(-0.0011)})
	--({\sx*(0.1800)},{\sy*(-0.0011)})
	--({\sx*(0.1900)},{\sy*(-0.0010)})
	--({\sx*(0.2000)},{\sy*(-0.0010)})
	--({\sx*(0.2100)},{\sy*(-0.0010)})
	--({\sx*(0.2200)},{\sy*(-0.0010)})
	--({\sx*(0.2300)},{\sy*(-0.0010)})
	--({\sx*(0.2400)},{\sy*(-0.0009)})
	--({\sx*(0.2500)},{\sy*(-0.0009)})
	--({\sx*(0.2600)},{\sy*(-0.0009)})
	--({\sx*(0.2700)},{\sy*(-0.0008)})
	--({\sx*(0.2800)},{\sy*(-0.0008)})
	--({\sx*(0.2900)},{\sy*(-0.0007)})
	--({\sx*(0.3000)},{\sy*(-0.0007)})
	--({\sx*(0.3100)},{\sy*(-0.0007)})
	--({\sx*(0.3200)},{\sy*(-0.0006)})
	--({\sx*(0.3300)},{\sy*(-0.0006)})
	--({\sx*(0.3400)},{\sy*(-0.0005)})
	--({\sx*(0.3500)},{\sy*(-0.0005)})
	--({\sx*(0.3600)},{\sy*(-0.0005)})
	--({\sx*(0.3700)},{\sy*(-0.0004)})
	--({\sx*(0.3800)},{\sy*(-0.0004)})
	--({\sx*(0.3900)},{\sy*(-0.0003)})
	--({\sx*(0.4000)},{\sy*(-0.0003)})
	--({\sx*(0.4100)},{\sy*(-0.0003)})
	--({\sx*(0.4200)},{\sy*(-0.0002)})
	--({\sx*(0.4300)},{\sy*(-0.0002)})
	--({\sx*(0.4400)},{\sy*(-0.0002)})
	--({\sx*(0.4500)},{\sy*(-0.0001)})
	--({\sx*(0.4600)},{\sy*(-0.0001)})
	--({\sx*(0.4700)},{\sy*(-0.0001)})
	--({\sx*(0.4800)},{\sy*(-0.0001)})
	--({\sx*(0.4900)},{\sy*(-0.0000)})
	--({\sx*(0.5000)},{\sy*(0.0000)})
	--({\sx*(0.5100)},{\sy*(0.0000)})
	--({\sx*(0.5200)},{\sy*(0.0000)})
	--({\sx*(0.5300)},{\sy*(0.0001)})
	--({\sx*(0.5400)},{\sy*(0.0001)})
	--({\sx*(0.5500)},{\sy*(0.0001)})
	--({\sx*(0.5600)},{\sy*(0.0001)})
	--({\sx*(0.5700)},{\sy*(0.0001)})
	--({\sx*(0.5800)},{\sy*(0.0002)})
	--({\sx*(0.5900)},{\sy*(0.0002)})
	--({\sx*(0.6000)},{\sy*(0.0002)})
	--({\sx*(0.6100)},{\sy*(0.0002)})
	--({\sx*(0.6200)},{\sy*(0.0002)})
	--({\sx*(0.6300)},{\sy*(0.0002)})
	--({\sx*(0.6400)},{\sy*(0.0002)})
	--({\sx*(0.6500)},{\sy*(0.0002)})
	--({\sx*(0.6600)},{\sy*(0.0002)})
	--({\sx*(0.6700)},{\sy*(0.0002)})
	--({\sx*(0.6800)},{\sy*(0.0002)})
	--({\sx*(0.6900)},{\sy*(0.0002)})
	--({\sx*(0.7000)},{\sy*(0.0002)})
	--({\sx*(0.7100)},{\sy*(0.0002)})
	--({\sx*(0.7200)},{\sy*(0.0002)})
	--({\sx*(0.7300)},{\sy*(0.0002)})
	--({\sx*(0.7400)},{\sy*(0.0002)})
	--({\sx*(0.7500)},{\sy*(0.0002)})
	--({\sx*(0.7600)},{\sy*(0.0002)})
	--({\sx*(0.7700)},{\sy*(0.0002)})
	--({\sx*(0.7800)},{\sy*(0.0002)})
	--({\sx*(0.7900)},{\sy*(0.0002)})
	--({\sx*(0.8000)},{\sy*(0.0002)})
	--({\sx*(0.8100)},{\sy*(0.0002)})
	--({\sx*(0.8200)},{\sy*(0.0002)})
	--({\sx*(0.8300)},{\sy*(0.0002)})
	--({\sx*(0.8400)},{\sy*(0.0002)})
	--({\sx*(0.8500)},{\sy*(0.0001)})
	--({\sx*(0.8600)},{\sy*(0.0001)})
	--({\sx*(0.8700)},{\sy*(0.0001)})
	--({\sx*(0.8800)},{\sy*(0.0001)})
	--({\sx*(0.8900)},{\sy*(0.0001)})
	--({\sx*(0.9000)},{\sy*(0.0001)})
	--({\sx*(0.9100)},{\sy*(0.0001)})
	--({\sx*(0.9200)},{\sy*(0.0001)})
	--({\sx*(0.9300)},{\sy*(0.0001)})
	--({\sx*(0.9400)},{\sy*(0.0001)})
	--({\sx*(0.9500)},{\sy*(0.0000)})
	--({\sx*(0.9600)},{\sy*(0.0000)})
	--({\sx*(0.9700)},{\sy*(0.0000)})
	--({\sx*(0.9800)},{\sy*(0.0000)})
	--({\sx*(0.9900)},{\sy*(0.0000)})
	--({\sx*(1.0000)},{\sy*(0.0000)})
	--({\sx*(1.0100)},{\sy*(-0.0000)})
	--({\sx*(1.0200)},{\sy*(-0.0000)})
	--({\sx*(1.0300)},{\sy*(-0.0000)})
	--({\sx*(1.0400)},{\sy*(-0.0000)})
	--({\sx*(1.0500)},{\sy*(-0.0000)})
	--({\sx*(1.0600)},{\sy*(-0.0001)})
	--({\sx*(1.0700)},{\sy*(-0.0001)})
	--({\sx*(1.0800)},{\sy*(-0.0001)})
	--({\sx*(1.0900)},{\sy*(-0.0001)})
	--({\sx*(1.1000)},{\sy*(-0.0001)})
	--({\sx*(1.1100)},{\sy*(-0.0001)})
	--({\sx*(1.1200)},{\sy*(-0.0001)})
	--({\sx*(1.1300)},{\sy*(-0.0001)})
	--({\sx*(1.1400)},{\sy*(-0.0001)})
	--({\sx*(1.1500)},{\sy*(-0.0001)})
	--({\sx*(1.1600)},{\sy*(-0.0001)})
	--({\sx*(1.1700)},{\sy*(-0.0001)})
	--({\sx*(1.1800)},{\sy*(-0.0001)})
	--({\sx*(1.1900)},{\sy*(-0.0001)})
	--({\sx*(1.2000)},{\sy*(-0.0001)})
	--({\sx*(1.2100)},{\sy*(-0.0001)})
	--({\sx*(1.2200)},{\sy*(-0.0001)})
	--({\sx*(1.2300)},{\sy*(-0.0001)})
	--({\sx*(1.2400)},{\sy*(-0.0001)})
	--({\sx*(1.2500)},{\sy*(-0.0001)})
	--({\sx*(1.2600)},{\sy*(-0.0001)})
	--({\sx*(1.2700)},{\sy*(-0.0001)})
	--({\sx*(1.2800)},{\sy*(-0.0001)})
	--({\sx*(1.2900)},{\sy*(-0.0001)})
	--({\sx*(1.3000)},{\sy*(-0.0001)})
	--({\sx*(1.3100)},{\sy*(-0.0001)})
	--({\sx*(1.3200)},{\sy*(-0.0001)})
	--({\sx*(1.3300)},{\sy*(-0.0001)})
	--({\sx*(1.3400)},{\sy*(-0.0001)})
	--({\sx*(1.3500)},{\sy*(-0.0001)})
	--({\sx*(1.3600)},{\sy*(-0.0001)})
	--({\sx*(1.3700)},{\sy*(-0.0001)})
	--({\sx*(1.3800)},{\sy*(-0.0001)})
	--({\sx*(1.3900)},{\sy*(-0.0001)})
	--({\sx*(1.4000)},{\sy*(-0.0001)})
	--({\sx*(1.4100)},{\sy*(-0.0001)})
	--({\sx*(1.4200)},{\sy*(-0.0001)})
	--({\sx*(1.4300)},{\sy*(-0.0000)})
	--({\sx*(1.4400)},{\sy*(-0.0000)})
	--({\sx*(1.4500)},{\sy*(-0.0000)})
	--({\sx*(1.4600)},{\sy*(-0.0000)})
	--({\sx*(1.4700)},{\sy*(-0.0000)})
	--({\sx*(1.4800)},{\sy*(-0.0000)})
	--({\sx*(1.4900)},{\sy*(-0.0000)})
	--({\sx*(1.5000)},{\sy*(0.0000)})
	--({\sx*(1.5100)},{\sy*(0.0000)})
	--({\sx*(1.5200)},{\sy*(0.0000)})
	--({\sx*(1.5300)},{\sy*(0.0000)})
	--({\sx*(1.5400)},{\sy*(0.0000)})
	--({\sx*(1.5500)},{\sy*(0.0000)})
	--({\sx*(1.5600)},{\sy*(0.0000)})
	--({\sx*(1.5700)},{\sy*(0.0000)})
	--({\sx*(1.5800)},{\sy*(0.0001)})
	--({\sx*(1.5900)},{\sy*(0.0001)})
	--({\sx*(1.6000)},{\sy*(0.0001)})
	--({\sx*(1.6100)},{\sy*(0.0001)})
	--({\sx*(1.6200)},{\sy*(0.0001)})
	--({\sx*(1.6300)},{\sy*(0.0001)})
	--({\sx*(1.6400)},{\sy*(0.0001)})
	--({\sx*(1.6500)},{\sy*(0.0001)})
	--({\sx*(1.6600)},{\sy*(0.0001)})
	--({\sx*(1.6700)},{\sy*(0.0001)})
	--({\sx*(1.6800)},{\sy*(0.0001)})
	--({\sx*(1.6900)},{\sy*(0.0001)})
	--({\sx*(1.7000)},{\sy*(0.0001)})
	--({\sx*(1.7100)},{\sy*(0.0001)})
	--({\sx*(1.7200)},{\sy*(0.0001)})
	--({\sx*(1.7300)},{\sy*(0.0001)})
	--({\sx*(1.7400)},{\sy*(0.0001)})
	--({\sx*(1.7500)},{\sy*(0.0001)})
	--({\sx*(1.7600)},{\sy*(0.0001)})
	--({\sx*(1.7700)},{\sy*(0.0001)})
	--({\sx*(1.7800)},{\sy*(0.0001)})
	--({\sx*(1.7900)},{\sy*(0.0001)})
	--({\sx*(1.8000)},{\sy*(0.0001)})
	--({\sx*(1.8100)},{\sy*(0.0001)})
	--({\sx*(1.8200)},{\sy*(0.0001)})
	--({\sx*(1.8300)},{\sy*(0.0001)})
	--({\sx*(1.8400)},{\sy*(0.0001)})
	--({\sx*(1.8500)},{\sy*(0.0001)})
	--({\sx*(1.8600)},{\sy*(0.0001)})
	--({\sx*(1.8700)},{\sy*(0.0001)})
	--({\sx*(1.8800)},{\sy*(0.0001)})
	--({\sx*(1.8900)},{\sy*(0.0001)})
	--({\sx*(1.9000)},{\sy*(0.0001)})
	--({\sx*(1.9100)},{\sy*(0.0001)})
	--({\sx*(1.9200)},{\sy*(0.0001)})
	--({\sx*(1.9300)},{\sy*(0.0001)})
	--({\sx*(1.9400)},{\sy*(0.0001)})
	--({\sx*(1.9500)},{\sy*(0.0000)})
	--({\sx*(1.9600)},{\sy*(0.0000)})
	--({\sx*(1.9700)},{\sy*(0.0000)})
	--({\sx*(1.9800)},{\sy*(0.0000)})
	--({\sx*(1.9900)},{\sy*(0.0000)})
	--({\sx*(2.0000)},{\sy*(0.0000)})
	--({\sx*(2.0100)},{\sy*(-0.0000)})
	--({\sx*(2.0200)},{\sy*(-0.0000)})
	--({\sx*(2.0300)},{\sy*(-0.0000)})
	--({\sx*(2.0400)},{\sy*(-0.0000)})
	--({\sx*(2.0500)},{\sy*(-0.0001)})
	--({\sx*(2.0600)},{\sy*(-0.0001)})
	--({\sx*(2.0700)},{\sy*(-0.0001)})
	--({\sx*(2.0800)},{\sy*(-0.0001)})
	--({\sx*(2.0900)},{\sy*(-0.0001)})
	--({\sx*(2.1000)},{\sy*(-0.0001)})
	--({\sx*(2.1100)},{\sy*(-0.0001)})
	--({\sx*(2.1200)},{\sy*(-0.0001)})
	--({\sx*(2.1300)},{\sy*(-0.0001)})
	--({\sx*(2.1400)},{\sy*(-0.0001)})
	--({\sx*(2.1500)},{\sy*(-0.0002)})
	--({\sx*(2.1600)},{\sy*(-0.0002)})
	--({\sx*(2.1700)},{\sy*(-0.0002)})
	--({\sx*(2.1800)},{\sy*(-0.0002)})
	--({\sx*(2.1900)},{\sy*(-0.0002)})
	--({\sx*(2.2000)},{\sy*(-0.0002)})
	--({\sx*(2.2100)},{\sy*(-0.0002)})
	--({\sx*(2.2200)},{\sy*(-0.0002)})
	--({\sx*(2.2300)},{\sy*(-0.0002)})
	--({\sx*(2.2400)},{\sy*(-0.0002)})
	--({\sx*(2.2500)},{\sy*(-0.0002)})
	--({\sx*(2.2600)},{\sy*(-0.0002)})
	--({\sx*(2.2700)},{\sy*(-0.0002)})
	--({\sx*(2.2800)},{\sy*(-0.0002)})
	--({\sx*(2.2900)},{\sy*(-0.0002)})
	--({\sx*(2.3000)},{\sy*(-0.0002)})
	--({\sx*(2.3100)},{\sy*(-0.0002)})
	--({\sx*(2.3200)},{\sy*(-0.0002)})
	--({\sx*(2.3300)},{\sy*(-0.0002)})
	--({\sx*(2.3400)},{\sy*(-0.0002)})
	--({\sx*(2.3500)},{\sy*(-0.0002)})
	--({\sx*(2.3600)},{\sy*(-0.0002)})
	--({\sx*(2.3700)},{\sy*(-0.0002)})
	--({\sx*(2.3800)},{\sy*(-0.0002)})
	--({\sx*(2.3900)},{\sy*(-0.0002)})
	--({\sx*(2.4000)},{\sy*(-0.0002)})
	--({\sx*(2.4100)},{\sy*(-0.0002)})
	--({\sx*(2.4200)},{\sy*(-0.0002)})
	--({\sx*(2.4300)},{\sy*(-0.0001)})
	--({\sx*(2.4400)},{\sy*(-0.0001)})
	--({\sx*(2.4500)},{\sy*(-0.0001)})
	--({\sx*(2.4600)},{\sy*(-0.0001)})
	--({\sx*(2.4700)},{\sy*(-0.0001)})
	--({\sx*(2.4800)},{\sy*(-0.0000)})
	--({\sx*(2.4900)},{\sy*(-0.0000)})
	--({\sx*(2.5000)},{\sy*(0.0000)})
	--({\sx*(2.5100)},{\sy*(0.0000)})
	--({\sx*(2.5200)},{\sy*(0.0001)})
	--({\sx*(2.5300)},{\sy*(0.0001)})
	--({\sx*(2.5400)},{\sy*(0.0001)})
	--({\sx*(2.5500)},{\sy*(0.0001)})
	--({\sx*(2.5600)},{\sy*(0.0002)})
	--({\sx*(2.5700)},{\sy*(0.0002)})
	--({\sx*(2.5800)},{\sy*(0.0002)})
	--({\sx*(2.5900)},{\sy*(0.0003)})
	--({\sx*(2.6000)},{\sy*(0.0003)})
	--({\sx*(2.6100)},{\sy*(0.0003)})
	--({\sx*(2.6200)},{\sy*(0.0004)})
	--({\sx*(2.6300)},{\sy*(0.0004)})
	--({\sx*(2.6400)},{\sy*(0.0004)})
	--({\sx*(2.6500)},{\sy*(0.0005)})
	--({\sx*(2.6600)},{\sy*(0.0005)})
	--({\sx*(2.6700)},{\sy*(0.0005)})
	--({\sx*(2.6800)},{\sy*(0.0006)})
	--({\sx*(2.6900)},{\sy*(0.0006)})
	--({\sx*(2.7000)},{\sy*(0.0006)})
	--({\sx*(2.7100)},{\sy*(0.0007)})
	--({\sx*(2.7200)},{\sy*(0.0007)})
	--({\sx*(2.7300)},{\sy*(0.0007)})
	--({\sx*(2.7400)},{\sy*(0.0008)})
	--({\sx*(2.7500)},{\sy*(0.0008)})
	--({\sx*(2.7600)},{\sy*(0.0008)})
	--({\sx*(2.7700)},{\sy*(0.0008)})
	--({\sx*(2.7800)},{\sy*(0.0008)})
	--({\sx*(2.7900)},{\sy*(0.0009)})
	--({\sx*(2.8000)},{\sy*(0.0009)})
	--({\sx*(2.8100)},{\sy*(0.0009)})
	--({\sx*(2.8200)},{\sy*(0.0009)})
	--({\sx*(2.8300)},{\sy*(0.0009)})
	--({\sx*(2.8400)},{\sy*(0.0009)})
	--({\sx*(2.8500)},{\sy*(0.0009)})
	--({\sx*(2.8600)},{\sy*(0.0008)})
	--({\sx*(2.8700)},{\sy*(0.0008)})
	--({\sx*(2.8800)},{\sy*(0.0008)})
	--({\sx*(2.8900)},{\sy*(0.0008)})
	--({\sx*(2.9000)},{\sy*(0.0007)})
	--({\sx*(2.9100)},{\sy*(0.0007)})
	--({\sx*(2.9200)},{\sy*(0.0006)})
	--({\sx*(2.9300)},{\sy*(0.0006)})
	--({\sx*(2.9400)},{\sy*(0.0005)})
	--({\sx*(2.9500)},{\sy*(0.0005)})
	--({\sx*(2.9600)},{\sy*(0.0004)})
	--({\sx*(2.9700)},{\sy*(0.0003)})
	--({\sx*(2.9800)},{\sy*(0.0002)})
	--({\sx*(2.9900)},{\sy*(0.0001)})
	--({\sx*(3.0000)},{\sy*(0.0000)})
	--({\sx*(3.0100)},{\sy*(-0.0001)})
	--({\sx*(3.0200)},{\sy*(-0.0002)})
	--({\sx*(3.0300)},{\sy*(-0.0004)})
	--({\sx*(3.0400)},{\sy*(-0.0005)})
	--({\sx*(3.0500)},{\sy*(-0.0007)})
	--({\sx*(3.0600)},{\sy*(-0.0008)})
	--({\sx*(3.0700)},{\sy*(-0.0010)})
	--({\sx*(3.0800)},{\sy*(-0.0012)})
	--({\sx*(3.0900)},{\sy*(-0.0013)})
	--({\sx*(3.1000)},{\sy*(-0.0015)})
	--({\sx*(3.1100)},{\sy*(-0.0017)})
	--({\sx*(3.1200)},{\sy*(-0.0019)})
	--({\sx*(3.1300)},{\sy*(-0.0021)})
	--({\sx*(3.1400)},{\sy*(-0.0023)})
	--({\sx*(3.1500)},{\sy*(-0.0025)})
	--({\sx*(3.1600)},{\sy*(-0.0028)})
	--({\sx*(3.1700)},{\sy*(-0.0030)})
	--({\sx*(3.1800)},{\sy*(-0.0032)})
	--({\sx*(3.1900)},{\sy*(-0.0034)})
	--({\sx*(3.2000)},{\sy*(-0.0037)})
	--({\sx*(3.2100)},{\sy*(-0.0039)})
	--({\sx*(3.2200)},{\sy*(-0.0041)})
	--({\sx*(3.2300)},{\sy*(-0.0043)})
	--({\sx*(3.2400)},{\sy*(-0.0045)})
	--({\sx*(3.2500)},{\sy*(-0.0048)})
	--({\sx*(3.2600)},{\sy*(-0.0049)})
	--({\sx*(3.2700)},{\sy*(-0.0051)})
	--({\sx*(3.2800)},{\sy*(-0.0053)})
	--({\sx*(3.2900)},{\sy*(-0.0055)})
	--({\sx*(3.3000)},{\sy*(-0.0056)})
	--({\sx*(3.3100)},{\sy*(-0.0057)})
	--({\sx*(3.3200)},{\sy*(-0.0058)})
	--({\sx*(3.3300)},{\sy*(-0.0059)})
	--({\sx*(3.3400)},{\sy*(-0.0059)})
	--({\sx*(3.3500)},{\sy*(-0.0059)})
	--({\sx*(3.3600)},{\sy*(-0.0059)})
	--({\sx*(3.3700)},{\sy*(-0.0059)})
	--({\sx*(3.3800)},{\sy*(-0.0058)})
	--({\sx*(3.3900)},{\sy*(-0.0056)})
	--({\sx*(3.4000)},{\sy*(-0.0054)})
	--({\sx*(3.4100)},{\sy*(-0.0052)})
	--({\sx*(3.4200)},{\sy*(-0.0049)})
	--({\sx*(3.4300)},{\sy*(-0.0045)})
	--({\sx*(3.4400)},{\sy*(-0.0041)})
	--({\sx*(3.4500)},{\sy*(-0.0036)})
	--({\sx*(3.4600)},{\sy*(-0.0030)})
	--({\sx*(3.4700)},{\sy*(-0.0024)})
	--({\sx*(3.4800)},{\sy*(-0.0017)})
	--({\sx*(3.4900)},{\sy*(-0.0009)})
	--({\sx*(3.5000)},{\sy*(0.0000)})
	--({\sx*(3.5100)},{\sy*(0.0010)})
	--({\sx*(3.5200)},{\sy*(0.0020)})
	--({\sx*(3.5300)},{\sy*(0.0032)})
	--({\sx*(3.5400)},{\sy*(0.0044)})
	--({\sx*(3.5500)},{\sy*(0.0058)})
	--({\sx*(3.5600)},{\sy*(0.0072)})
	--({\sx*(3.5700)},{\sy*(0.0088)})
	--({\sx*(3.5800)},{\sy*(0.0104)})
	--({\sx*(3.5900)},{\sy*(0.0122)})
	--({\sx*(3.6000)},{\sy*(0.0140)})
	--({\sx*(3.6100)},{\sy*(0.0160)})
	--({\sx*(3.6200)},{\sy*(0.0180)})
	--({\sx*(3.6300)},{\sy*(0.0201)})
	--({\sx*(3.6400)},{\sy*(0.0224)})
	--({\sx*(3.6500)},{\sy*(0.0247)})
	--({\sx*(3.6600)},{\sy*(0.0271)})
	--({\sx*(3.6700)},{\sy*(0.0296)})
	--({\sx*(3.6800)},{\sy*(0.0321)})
	--({\sx*(3.6900)},{\sy*(0.0347)})
	--({\sx*(3.7000)},{\sy*(0.0374)})
	--({\sx*(3.7100)},{\sy*(0.0401)})
	--({\sx*(3.7200)},{\sy*(0.0428)})
	--({\sx*(3.7300)},{\sy*(0.0455)})
	--({\sx*(3.7400)},{\sy*(0.0482)})
	--({\sx*(3.7500)},{\sy*(0.0509)})
	--({\sx*(3.7600)},{\sy*(0.0536)})
	--({\sx*(3.7700)},{\sy*(0.0561)})
	--({\sx*(3.7800)},{\sy*(0.0586)})
	--({\sx*(3.7900)},{\sy*(0.0610)})
	--({\sx*(3.8000)},{\sy*(0.0632)})
	--({\sx*(3.8100)},{\sy*(0.0653)})
	--({\sx*(3.8200)},{\sy*(0.0671)})
	--({\sx*(3.8300)},{\sy*(0.0687)})
	--({\sx*(3.8400)},{\sy*(0.0701)})
	--({\sx*(3.8500)},{\sy*(0.0711)})
	--({\sx*(3.8600)},{\sy*(0.0717)})
	--({\sx*(3.8700)},{\sy*(0.0719)})
	--({\sx*(3.8800)},{\sy*(0.0716)})
	--({\sx*(3.8900)},{\sy*(0.0708)})
	--({\sx*(3.9000)},{\sy*(0.0694)})
	--({\sx*(3.9100)},{\sy*(0.0673)})
	--({\sx*(3.9200)},{\sy*(0.0644)})
	--({\sx*(3.9300)},{\sy*(0.0607)})
	--({\sx*(3.9400)},{\sy*(0.0561)})
	--({\sx*(3.9500)},{\sy*(0.0503)})
	--({\sx*(3.9600)},{\sy*(0.0434)})
	--({\sx*(3.9700)},{\sy*(0.0350)})
	--({\sx*(3.9800)},{\sy*(0.0252)})
	--({\sx*(3.9900)},{\sy*(0.0136)})
	--({\sx*(4.0000)},{\sy*(0.0000)})
	--({\sx*(4.0100)},{\sy*(-0.0158)})
	--({\sx*(4.0200)},{\sy*(-0.0343)})
	--({\sx*(4.0300)},{\sy*(-0.0558)})
	--({\sx*(4.0400)},{\sy*(-0.0807)})
	--({\sx*(4.0500)},{\sy*(-0.1098)})
	--({\sx*(4.0600)},{\sy*(-0.1437)})
	--({\sx*(4.0700)},{\sy*(-0.1835)})
	--({\sx*(4.0800)},{\sy*(-0.2302)})
	--({\sx*(4.0900)},{\sy*(-0.2856)})
	--({\sx*(4.1000)},{\sy*(-0.3516)})
	--({\sx*(4.1100)},{\sy*(-0.4312)})
	--({\sx*(4.1200)},{\sy*(-0.5282)})
	--({\sx*(4.1300)},{\sy*(-0.6483)})
	--({\sx*(4.1400)},{\sy*(-0.7998)})
	--({\sx*(4.1500)},{\sy*(-0.9956)})
	--({\sx*(4.1600)},{\sy*(-1.2570)})
	--({\sx*(4.1700)},{\sy*(-1.6210)})
	--({\sx*(4.1800)},{\sy*(-2.1594)})
	--({\sx*(4.1900)},{\sy*(-3.0316)})
	--({\sx*(4.2000)},{\sy*(-4.6750)})
	--({\sx*(4.2100)},{\sy*(-8.8854)})
	--({\sx*(4.2200)},{\sy*(-42.3532)})
	--({\sx*(4.2300)},{\sy*(18.2934)})
	--({\sx*(4.2400)},{\sy*(8.0817)})
	--({\sx*(4.2500)},{\sy*(5.4110)})
	--({\sx*(4.2600)},{\sy*(4.1862)})
	--({\sx*(4.2700)},{\sy*(3.4869)})
	--({\sx*(4.2800)},{\sy*(3.0375)})
	--({\sx*(4.2900)},{\sy*(2.7266)})
	--({\sx*(4.3000)},{\sy*(2.5008)})
	--({\sx*(4.3100)},{\sy*(2.3312)})
	--({\sx*(4.3200)},{\sy*(2.2009)})
	--({\sx*(4.3300)},{\sy*(2.0996)})
	--({\sx*(4.3400)},{\sy*(2.0206)})
	--({\sx*(4.3500)},{\sy*(1.9595)})
	--({\sx*(4.3600)},{\sy*(1.9136)})
	--({\sx*(4.3700)},{\sy*(1.8812)})
	--({\sx*(4.3800)},{\sy*(1.8617)})
	--({\sx*(4.3900)},{\sy*(1.8555)})
	--({\sx*(4.4000)},{\sy*(1.8643)})
	--({\sx*(4.4100)},{\sy*(1.8917)})
	--({\sx*(4.4200)},{\sy*(1.9446)})
	--({\sx*(4.4300)},{\sy*(2.0369)})
	--({\sx*(4.4400)},{\sy*(2.1984)})
	--({\sx*(4.4500)},{\sy*(2.5062)})
	--({\sx*(4.4600)},{\sy*(3.2349)})
	--({\sx*(4.4700)},{\sy*(6.5759)})
	--({\sx*(4.4800)},{\sy*(-5.8180)})
	--({\sx*(4.4900)},{\sy*(-0.8610)})
	--({\sx*(4.5000)},{\sy*(0.0000)})
	--({\sx*(4.5100)},{\sy*(0.3517)})
	--({\sx*(4.5200)},{\sy*(0.5399)})
	--({\sx*(4.5300)},{\sy*(0.6553)})
	--({\sx*(4.5400)},{\sy*(0.7324)})
	--({\sx*(4.5500)},{\sy*(0.7867)})
	--({\sx*(4.5600)},{\sy*(0.8266)})
	--({\sx*(4.5700)},{\sy*(0.8569)})
	--({\sx*(4.5800)},{\sy*(0.8804)})
	--({\sx*(4.5900)},{\sy*(0.8989)})
	--({\sx*(4.6000)},{\sy*(0.9139)})
	--({\sx*(4.6100)},{\sy*(0.9260)})
	--({\sx*(4.6200)},{\sy*(0.9360)})
	--({\sx*(4.6300)},{\sy*(0.9443)})
	--({\sx*(4.6400)},{\sy*(0.9512)})
	--({\sx*(4.6500)},{\sy*(0.9571)})
	--({\sx*(4.6600)},{\sy*(0.9621)})
	--({\sx*(4.6700)},{\sy*(0.9664)})
	--({\sx*(4.6800)},{\sy*(0.9700)})
	--({\sx*(4.6900)},{\sy*(0.9732)})
	--({\sx*(4.7000)},{\sy*(0.9759)})
	--({\sx*(4.7100)},{\sy*(0.9783)})
	--({\sx*(4.7200)},{\sy*(0.9804)})
	--({\sx*(4.7300)},{\sy*(0.9822)})
	--({\sx*(4.7400)},{\sy*(0.9838)})
	--({\sx*(4.7500)},{\sy*(0.9852)})
	--({\sx*(4.7600)},{\sy*(0.9865)})
	--({\sx*(4.7700)},{\sy*(0.9876)})
	--({\sx*(4.7800)},{\sy*(0.9885)})
	--({\sx*(4.7900)},{\sy*(0.9894)})
	--({\sx*(4.8000)},{\sy*(0.9902)})
	--({\sx*(4.8100)},{\sy*(0.9908)})
	--({\sx*(4.8200)},{\sy*(0.9914)})
	--({\sx*(4.8300)},{\sy*(0.9919)})
	--({\sx*(4.8400)},{\sy*(0.9924)})
	--({\sx*(4.8500)},{\sy*(0.9928)})
	--({\sx*(4.8600)},{\sy*(0.9931)})
	--({\sx*(4.8700)},{\sy*(0.9934)})
	--({\sx*(4.8800)},{\sy*(0.9936)})
	--({\sx*(4.8900)},{\sy*(0.9938)})
	--({\sx*(4.9000)},{\sy*(0.9939)})
	--({\sx*(4.9100)},{\sy*(0.9939)})
	--({\sx*(4.9200)},{\sy*(0.9939)})
	--({\sx*(4.9300)},{\sy*(0.9938)})
	--({\sx*(4.9400)},{\sy*(0.9935)})
	--({\sx*(4.9500)},{\sy*(0.9930)})
	--({\sx*(4.9600)},{\sy*(0.9922)})
	--({\sx*(4.9700)},{\sy*(0.9907)})
	--({\sx*(4.9800)},{\sy*(0.9875)})
	--({\sx*(4.9900)},{\sy*(0.9777)})
	--({\sx*(5.0000)},{\sy*(0.0000)});
}
\def\xwertef{
\fill[color=red] (0.0000,0) circle[radius={0.07/\skala}];
\fill[color=red] (0.4167,0) circle[radius={0.07/\skala}];
\fill[color=red] (0.8333,0) circle[radius={0.07/\skala}];
\fill[color=red] (1.2500,0) circle[radius={0.07/\skala}];
\fill[color=red] (1.6667,0) circle[radius={0.07/\skala}];
\fill[color=red] (2.0833,0) circle[radius={0.07/\skala}];
\fill[color=red] (2.5000,0) circle[radius={0.07/\skala}];
\fill[color=red] (2.9167,0) circle[radius={0.07/\skala}];
\fill[color=red] (3.3333,0) circle[radius={0.07/\skala}];
\fill[color=red] (3.7500,0) circle[radius={0.07/\skala}];
\fill[color=red] (4.1667,0) circle[radius={0.07/\skala}];
\fill[color=red] (4.5833,0) circle[radius={0.07/\skala}];
\fill[color=red] (5.0000,0) circle[radius={0.07/\skala}];
}
\def\punktef{12}
\def\maxfehlerf{1.462\cdot 10^{-4}}
\def\fehlerf{
\draw[color=red,line width=1.4pt,line join=round] ({\sx*(0.000)},{\sy*(0.0000)})
	--({\sx*(0.0100)},{\sy*(0.2018)})
	--({\sx*(0.0200)},{\sy*(0.3745)})
	--({\sx*(0.0300)},{\sy*(0.5208)})
	--({\sx*(0.0400)},{\sy*(0.6431)})
	--({\sx*(0.0500)},{\sy*(0.7436)})
	--({\sx*(0.0600)},{\sy*(0.8246)})
	--({\sx*(0.0700)},{\sy*(0.8880)})
	--({\sx*(0.0800)},{\sy*(0.9355)})
	--({\sx*(0.0900)},{\sy*(0.9690)})
	--({\sx*(0.1000)},{\sy*(0.9899)})
	--({\sx*(0.1100)},{\sy*(0.9998)})
	--({\sx*(0.1200)},{\sy*(1.0000)})
	--({\sx*(0.1300)},{\sy*(0.9917)})
	--({\sx*(0.1400)},{\sy*(0.9760)})
	--({\sx*(0.1500)},{\sy*(0.9540)})
	--({\sx*(0.1600)},{\sy*(0.9266)})
	--({\sx*(0.1700)},{\sy*(0.8948)})
	--({\sx*(0.1800)},{\sy*(0.8592)})
	--({\sx*(0.1900)},{\sy*(0.8206)})
	--({\sx*(0.2000)},{\sy*(0.7797)})
	--({\sx*(0.2100)},{\sy*(0.7371)})
	--({\sx*(0.2200)},{\sy*(0.6932)})
	--({\sx*(0.2300)},{\sy*(0.6485)})
	--({\sx*(0.2400)},{\sy*(0.6036)})
	--({\sx*(0.2500)},{\sy*(0.5587)})
	--({\sx*(0.2600)},{\sy*(0.5141)})
	--({\sx*(0.2700)},{\sy*(0.4702)})
	--({\sx*(0.2800)},{\sy*(0.4272)})
	--({\sx*(0.2900)},{\sy*(0.3854)})
	--({\sx*(0.3000)},{\sy*(0.3448)})
	--({\sx*(0.3100)},{\sy*(0.3057)})
	--({\sx*(0.3200)},{\sy*(0.2683)})
	--({\sx*(0.3300)},{\sy*(0.2325)})
	--({\sx*(0.3400)},{\sy*(0.1984)})
	--({\sx*(0.3500)},{\sy*(0.1663)})
	--({\sx*(0.3600)},{\sy*(0.1359)})
	--({\sx*(0.3700)},{\sy*(0.1075)})
	--({\sx*(0.3800)},{\sy*(0.0810)})
	--({\sx*(0.3900)},{\sy*(0.0564)})
	--({\sx*(0.4000)},{\sy*(0.0337)})
	--({\sx*(0.4100)},{\sy*(0.0129)})
	--({\sx*(0.4200)},{\sy*(-0.0061)})
	--({\sx*(0.4300)},{\sy*(-0.0234)})
	--({\sx*(0.4400)},{\sy*(-0.0389)})
	--({\sx*(0.4500)},{\sy*(-0.0527)})
	--({\sx*(0.4600)},{\sy*(-0.0650)})
	--({\sx*(0.4700)},{\sy*(-0.0757)})
	--({\sx*(0.4800)},{\sy*(-0.0849)})
	--({\sx*(0.4900)},{\sy*(-0.0927)})
	--({\sx*(0.5000)},{\sy*(-0.0992)})
	--({\sx*(0.5100)},{\sy*(-0.1044)})
	--({\sx*(0.5200)},{\sy*(-0.1085)})
	--({\sx*(0.5300)},{\sy*(-0.1114)})
	--({\sx*(0.5400)},{\sy*(-0.1134)})
	--({\sx*(0.5500)},{\sy*(-0.1144)})
	--({\sx*(0.5600)},{\sy*(-0.1145)})
	--({\sx*(0.5700)},{\sy*(-0.1138)})
	--({\sx*(0.5800)},{\sy*(-0.1125)})
	--({\sx*(0.5900)},{\sy*(-0.1104)})
	--({\sx*(0.6000)},{\sy*(-0.1078)})
	--({\sx*(0.6100)},{\sy*(-0.1046)})
	--({\sx*(0.6200)},{\sy*(-0.1010)})
	--({\sx*(0.6300)},{\sy*(-0.0970)})
	--({\sx*(0.6400)},{\sy*(-0.0926)})
	--({\sx*(0.6500)},{\sy*(-0.0880)})
	--({\sx*(0.6600)},{\sy*(-0.0831)})
	--({\sx*(0.6700)},{\sy*(-0.0781)})
	--({\sx*(0.6800)},{\sy*(-0.0729)})
	--({\sx*(0.6900)},{\sy*(-0.0676)})
	--({\sx*(0.7000)},{\sy*(-0.0622)})
	--({\sx*(0.7100)},{\sy*(-0.0569)})
	--({\sx*(0.7200)},{\sy*(-0.0515)})
	--({\sx*(0.7300)},{\sy*(-0.0462)})
	--({\sx*(0.7400)},{\sy*(-0.0410)})
	--({\sx*(0.7500)},{\sy*(-0.0359)})
	--({\sx*(0.7600)},{\sy*(-0.0309)})
	--({\sx*(0.7700)},{\sy*(-0.0261)})
	--({\sx*(0.7800)},{\sy*(-0.0214)})
	--({\sx*(0.7900)},{\sy*(-0.0170)})
	--({\sx*(0.8000)},{\sy*(-0.0127)})
	--({\sx*(0.8100)},{\sy*(-0.0086)})
	--({\sx*(0.8200)},{\sy*(-0.0048)})
	--({\sx*(0.8300)},{\sy*(-0.0012)})
	--({\sx*(0.8400)},{\sy*(0.0022)})
	--({\sx*(0.8500)},{\sy*(0.0054)})
	--({\sx*(0.8600)},{\sy*(0.0083)})
	--({\sx*(0.8700)},{\sy*(0.0109)})
	--({\sx*(0.8800)},{\sy*(0.0134)})
	--({\sx*(0.8900)},{\sy*(0.0155)})
	--({\sx*(0.9000)},{\sy*(0.0175)})
	--({\sx*(0.9100)},{\sy*(0.0192)})
	--({\sx*(0.9200)},{\sy*(0.0207)})
	--({\sx*(0.9300)},{\sy*(0.0219)})
	--({\sx*(0.9400)},{\sy*(0.0230)})
	--({\sx*(0.9500)},{\sy*(0.0238)})
	--({\sx*(0.9600)},{\sy*(0.0245)})
	--({\sx*(0.9700)},{\sy*(0.0249)})
	--({\sx*(0.9800)},{\sy*(0.0252)})
	--({\sx*(0.9900)},{\sy*(0.0253)})
	--({\sx*(1.0000)},{\sy*(0.0252)})
	--({\sx*(1.0100)},{\sy*(0.0250)})
	--({\sx*(1.0200)},{\sy*(0.0246)})
	--({\sx*(1.0300)},{\sy*(0.0241)})
	--({\sx*(1.0400)},{\sy*(0.0235)})
	--({\sx*(1.0500)},{\sy*(0.0228)})
	--({\sx*(1.0600)},{\sy*(0.0220)})
	--({\sx*(1.0700)},{\sy*(0.0211)})
	--({\sx*(1.0800)},{\sy*(0.0201)})
	--({\sx*(1.0900)},{\sy*(0.0190)})
	--({\sx*(1.1000)},{\sy*(0.0179)})
	--({\sx*(1.1100)},{\sy*(0.0168)})
	--({\sx*(1.1200)},{\sy*(0.0156)})
	--({\sx*(1.1300)},{\sy*(0.0143)})
	--({\sx*(1.1400)},{\sy*(0.0131)})
	--({\sx*(1.1500)},{\sy*(0.0118)})
	--({\sx*(1.1600)},{\sy*(0.0105)})
	--({\sx*(1.1700)},{\sy*(0.0093)})
	--({\sx*(1.1800)},{\sy*(0.0080)})
	--({\sx*(1.1900)},{\sy*(0.0068)})
	--({\sx*(1.2000)},{\sy*(0.0056)})
	--({\sx*(1.2100)},{\sy*(0.0044)})
	--({\sx*(1.2200)},{\sy*(0.0032)})
	--({\sx*(1.2300)},{\sy*(0.0021)})
	--({\sx*(1.2400)},{\sy*(0.0010)})
	--({\sx*(1.2500)},{\sy*(0.0000)})
	--({\sx*(1.2600)},{\sy*(-0.0010)})
	--({\sx*(1.2700)},{\sy*(-0.0019)})
	--({\sx*(1.2800)},{\sy*(-0.0028)})
	--({\sx*(1.2900)},{\sy*(-0.0036)})
	--({\sx*(1.3000)},{\sy*(-0.0043)})
	--({\sx*(1.3100)},{\sy*(-0.0050)})
	--({\sx*(1.3200)},{\sy*(-0.0057)})
	--({\sx*(1.3300)},{\sy*(-0.0062)})
	--({\sx*(1.3400)},{\sy*(-0.0068)})
	--({\sx*(1.3500)},{\sy*(-0.0072)})
	--({\sx*(1.3600)},{\sy*(-0.0076)})
	--({\sx*(1.3700)},{\sy*(-0.0079)})
	--({\sx*(1.3800)},{\sy*(-0.0082)})
	--({\sx*(1.3900)},{\sy*(-0.0084)})
	--({\sx*(1.4000)},{\sy*(-0.0085)})
	--({\sx*(1.4100)},{\sy*(-0.0086)})
	--({\sx*(1.4200)},{\sy*(-0.0086)})
	--({\sx*(1.4300)},{\sy*(-0.0086)})
	--({\sx*(1.4400)},{\sy*(-0.0086)})
	--({\sx*(1.4500)},{\sy*(-0.0084)})
	--({\sx*(1.4600)},{\sy*(-0.0083)})
	--({\sx*(1.4700)},{\sy*(-0.0081)})
	--({\sx*(1.4800)},{\sy*(-0.0078)})
	--({\sx*(1.4900)},{\sy*(-0.0076)})
	--({\sx*(1.5000)},{\sy*(-0.0073)})
	--({\sx*(1.5100)},{\sy*(-0.0069)})
	--({\sx*(1.5200)},{\sy*(-0.0066)})
	--({\sx*(1.5300)},{\sy*(-0.0062)})
	--({\sx*(1.5400)},{\sy*(-0.0058)})
	--({\sx*(1.5500)},{\sy*(-0.0053)})
	--({\sx*(1.5600)},{\sy*(-0.0049)})
	--({\sx*(1.5700)},{\sy*(-0.0045)})
	--({\sx*(1.5800)},{\sy*(-0.0040)})
	--({\sx*(1.5900)},{\sy*(-0.0035)})
	--({\sx*(1.6000)},{\sy*(-0.0030)})
	--({\sx*(1.6100)},{\sy*(-0.0026)})
	--({\sx*(1.6200)},{\sy*(-0.0021)})
	--({\sx*(1.6300)},{\sy*(-0.0016)})
	--({\sx*(1.6400)},{\sy*(-0.0012)})
	--({\sx*(1.6500)},{\sy*(-0.0007)})
	--({\sx*(1.6600)},{\sy*(-0.0003)})
	--({\sx*(1.6700)},{\sy*(0.0001)})
	--({\sx*(1.6800)},{\sy*(0.0006)})
	--({\sx*(1.6900)},{\sy*(0.0009)})
	--({\sx*(1.7000)},{\sy*(0.0013)})
	--({\sx*(1.7100)},{\sy*(0.0017)})
	--({\sx*(1.7200)},{\sy*(0.0020)})
	--({\sx*(1.7300)},{\sy*(0.0023)})
	--({\sx*(1.7400)},{\sy*(0.0026)})
	--({\sx*(1.7500)},{\sy*(0.0029)})
	--({\sx*(1.7600)},{\sy*(0.0031)})
	--({\sx*(1.7700)},{\sy*(0.0034)})
	--({\sx*(1.7800)},{\sy*(0.0036)})
	--({\sx*(1.7900)},{\sy*(0.0037)})
	--({\sx*(1.8000)},{\sy*(0.0039)})
	--({\sx*(1.8100)},{\sy*(0.0040)})
	--({\sx*(1.8200)},{\sy*(0.0041)})
	--({\sx*(1.8300)},{\sy*(0.0041)})
	--({\sx*(1.8400)},{\sy*(0.0042)})
	--({\sx*(1.8500)},{\sy*(0.0042)})
	--({\sx*(1.8600)},{\sy*(0.0042)})
	--({\sx*(1.8700)},{\sy*(0.0042)})
	--({\sx*(1.8800)},{\sy*(0.0041)})
	--({\sx*(1.8900)},{\sy*(0.0040)})
	--({\sx*(1.9000)},{\sy*(0.0039)})
	--({\sx*(1.9100)},{\sy*(0.0038)})
	--({\sx*(1.9200)},{\sy*(0.0037)})
	--({\sx*(1.9300)},{\sy*(0.0035)})
	--({\sx*(1.9400)},{\sy*(0.0034)})
	--({\sx*(1.9500)},{\sy*(0.0032)})
	--({\sx*(1.9600)},{\sy*(0.0030)})
	--({\sx*(1.9700)},{\sy*(0.0028)})
	--({\sx*(1.9800)},{\sy*(0.0026)})
	--({\sx*(1.9900)},{\sy*(0.0023)})
	--({\sx*(2.0000)},{\sy*(0.0021)})
	--({\sx*(2.0100)},{\sy*(0.0018)})
	--({\sx*(2.0200)},{\sy*(0.0016)})
	--({\sx*(2.0300)},{\sy*(0.0013)})
	--({\sx*(2.0400)},{\sy*(0.0011)})
	--({\sx*(2.0500)},{\sy*(0.0008)})
	--({\sx*(2.0600)},{\sy*(0.0006)})
	--({\sx*(2.0700)},{\sy*(0.0003)})
	--({\sx*(2.0800)},{\sy*(0.0001)})
	--({\sx*(2.0900)},{\sy*(-0.0002)})
	--({\sx*(2.1000)},{\sy*(-0.0004)})
	--({\sx*(2.1100)},{\sy*(-0.0006)})
	--({\sx*(2.1200)},{\sy*(-0.0009)})
	--({\sx*(2.1300)},{\sy*(-0.0011)})
	--({\sx*(2.1400)},{\sy*(-0.0013)})
	--({\sx*(2.1500)},{\sy*(-0.0015)})
	--({\sx*(2.1600)},{\sy*(-0.0017)})
	--({\sx*(2.1700)},{\sy*(-0.0018)})
	--({\sx*(2.1800)},{\sy*(-0.0020)})
	--({\sx*(2.1900)},{\sy*(-0.0021)})
	--({\sx*(2.2000)},{\sy*(-0.0023)})
	--({\sx*(2.2100)},{\sy*(-0.0024)})
	--({\sx*(2.2200)},{\sy*(-0.0025)})
	--({\sx*(2.2300)},{\sy*(-0.0026)})
	--({\sx*(2.2400)},{\sy*(-0.0027)})
	--({\sx*(2.2500)},{\sy*(-0.0027)})
	--({\sx*(2.2600)},{\sy*(-0.0028)})
	--({\sx*(2.2700)},{\sy*(-0.0028)})
	--({\sx*(2.2800)},{\sy*(-0.0028)})
	--({\sx*(2.2900)},{\sy*(-0.0028)})
	--({\sx*(2.3000)},{\sy*(-0.0028)})
	--({\sx*(2.3100)},{\sy*(-0.0027)})
	--({\sx*(2.3200)},{\sy*(-0.0027)})
	--({\sx*(2.3300)},{\sy*(-0.0026)})
	--({\sx*(2.3400)},{\sy*(-0.0025)})
	--({\sx*(2.3500)},{\sy*(-0.0024)})
	--({\sx*(2.3600)},{\sy*(-0.0023)})
	--({\sx*(2.3700)},{\sy*(-0.0022)})
	--({\sx*(2.3800)},{\sy*(-0.0021)})
	--({\sx*(2.3900)},{\sy*(-0.0019)})
	--({\sx*(2.4000)},{\sy*(-0.0018)})
	--({\sx*(2.4100)},{\sy*(-0.0016)})
	--({\sx*(2.4200)},{\sy*(-0.0015)})
	--({\sx*(2.4300)},{\sy*(-0.0013)})
	--({\sx*(2.4400)},{\sy*(-0.0011)})
	--({\sx*(2.4500)},{\sy*(-0.0009)})
	--({\sx*(2.4600)},{\sy*(-0.0008)})
	--({\sx*(2.4700)},{\sy*(-0.0006)})
	--({\sx*(2.4800)},{\sy*(-0.0004)})
	--({\sx*(2.4900)},{\sy*(-0.0002)})
	--({\sx*(2.5000)},{\sy*(0.0000)})
	--({\sx*(2.5100)},{\sy*(0.0002)})
	--({\sx*(2.5200)},{\sy*(0.0004)})
	--({\sx*(2.5300)},{\sy*(0.0006)})
	--({\sx*(2.5400)},{\sy*(0.0007)})
	--({\sx*(2.5500)},{\sy*(0.0009)})
	--({\sx*(2.5600)},{\sy*(0.0011)})
	--({\sx*(2.5700)},{\sy*(0.0013)})
	--({\sx*(2.5800)},{\sy*(0.0014)})
	--({\sx*(2.5900)},{\sy*(0.0016)})
	--({\sx*(2.6000)},{\sy*(0.0017)})
	--({\sx*(2.6100)},{\sy*(0.0018)})
	--({\sx*(2.6200)},{\sy*(0.0020)})
	--({\sx*(2.6300)},{\sy*(0.0021)})
	--({\sx*(2.6400)},{\sy*(0.0022)})
	--({\sx*(2.6500)},{\sy*(0.0022)})
	--({\sx*(2.6600)},{\sy*(0.0023)})
	--({\sx*(2.6700)},{\sy*(0.0024)})
	--({\sx*(2.6800)},{\sy*(0.0024)})
	--({\sx*(2.6900)},{\sy*(0.0025)})
	--({\sx*(2.7000)},{\sy*(0.0025)})
	--({\sx*(2.7100)},{\sy*(0.0025)})
	--({\sx*(2.7200)},{\sy*(0.0025)})
	--({\sx*(2.7300)},{\sy*(0.0025)})
	--({\sx*(2.7400)},{\sy*(0.0024)})
	--({\sx*(2.7500)},{\sy*(0.0024)})
	--({\sx*(2.7600)},{\sy*(0.0023)})
	--({\sx*(2.7700)},{\sy*(0.0022)})
	--({\sx*(2.7800)},{\sy*(0.0022)})
	--({\sx*(2.7900)},{\sy*(0.0021)})
	--({\sx*(2.8000)},{\sy*(0.0019)})
	--({\sx*(2.8100)},{\sy*(0.0018)})
	--({\sx*(2.8200)},{\sy*(0.0017)})
	--({\sx*(2.8300)},{\sy*(0.0015)})
	--({\sx*(2.8400)},{\sy*(0.0014)})
	--({\sx*(2.8500)},{\sy*(0.0012)})
	--({\sx*(2.8600)},{\sy*(0.0011)})
	--({\sx*(2.8700)},{\sy*(0.0009)})
	--({\sx*(2.8800)},{\sy*(0.0007)})
	--({\sx*(2.8900)},{\sy*(0.0005)})
	--({\sx*(2.9000)},{\sy*(0.0003)})
	--({\sx*(2.9100)},{\sy*(0.0001)})
	--({\sx*(2.9200)},{\sy*(-0.0001)})
	--({\sx*(2.9300)},{\sy*(-0.0003)})
	--({\sx*(2.9400)},{\sy*(-0.0005)})
	--({\sx*(2.9500)},{\sy*(-0.0007)})
	--({\sx*(2.9600)},{\sy*(-0.0009)})
	--({\sx*(2.9700)},{\sy*(-0.0011)})
	--({\sx*(2.9800)},{\sy*(-0.0012)})
	--({\sx*(2.9900)},{\sy*(-0.0014)})
	--({\sx*(3.0000)},{\sy*(-0.0016)})
	--({\sx*(3.0100)},{\sy*(-0.0018)})
	--({\sx*(3.0200)},{\sy*(-0.0020)})
	--({\sx*(3.0300)},{\sy*(-0.0021)})
	--({\sx*(3.0400)},{\sy*(-0.0023)})
	--({\sx*(3.0500)},{\sy*(-0.0024)})
	--({\sx*(3.0600)},{\sy*(-0.0025)})
	--({\sx*(3.0700)},{\sy*(-0.0026)})
	--({\sx*(3.0800)},{\sy*(-0.0027)})
	--({\sx*(3.0900)},{\sy*(-0.0028)})
	--({\sx*(3.1000)},{\sy*(-0.0029)})
	--({\sx*(3.1100)},{\sy*(-0.0029)})
	--({\sx*(3.1200)},{\sy*(-0.0030)})
	--({\sx*(3.1300)},{\sy*(-0.0030)})
	--({\sx*(3.1400)},{\sy*(-0.0030)})
	--({\sx*(3.1500)},{\sy*(-0.0030)})
	--({\sx*(3.1600)},{\sy*(-0.0030)})
	--({\sx*(3.1700)},{\sy*(-0.0029)})
	--({\sx*(3.1800)},{\sy*(-0.0029)})
	--({\sx*(3.1900)},{\sy*(-0.0028)})
	--({\sx*(3.2000)},{\sy*(-0.0027)})
	--({\sx*(3.2100)},{\sy*(-0.0026)})
	--({\sx*(3.2200)},{\sy*(-0.0025)})
	--({\sx*(3.2300)},{\sy*(-0.0023)})
	--({\sx*(3.2400)},{\sy*(-0.0022)})
	--({\sx*(3.2500)},{\sy*(-0.0020)})
	--({\sx*(3.2600)},{\sy*(-0.0018)})
	--({\sx*(3.2700)},{\sy*(-0.0016)})
	--({\sx*(3.2800)},{\sy*(-0.0014)})
	--({\sx*(3.2900)},{\sy*(-0.0011)})
	--({\sx*(3.3000)},{\sy*(-0.0009)})
	--({\sx*(3.3100)},{\sy*(-0.0006)})
	--({\sx*(3.3200)},{\sy*(-0.0004)})
	--({\sx*(3.3300)},{\sy*(-0.0001)})
	--({\sx*(3.3400)},{\sy*(0.0002)})
	--({\sx*(3.3500)},{\sy*(0.0005)})
	--({\sx*(3.3600)},{\sy*(0.0008)})
	--({\sx*(3.3700)},{\sy*(0.0011)})
	--({\sx*(3.3800)},{\sy*(0.0013)})
	--({\sx*(3.3900)},{\sy*(0.0016)})
	--({\sx*(3.4000)},{\sy*(0.0019)})
	--({\sx*(3.4100)},{\sy*(0.0022)})
	--({\sx*(3.4200)},{\sy*(0.0025)})
	--({\sx*(3.4300)},{\sy*(0.0028)})
	--({\sx*(3.4400)},{\sy*(0.0030)})
	--({\sx*(3.4500)},{\sy*(0.0033)})
	--({\sx*(3.4600)},{\sy*(0.0036)})
	--({\sx*(3.4700)},{\sy*(0.0038)})
	--({\sx*(3.4800)},{\sy*(0.0040)})
	--({\sx*(3.4900)},{\sy*(0.0042)})
	--({\sx*(3.5000)},{\sy*(0.0044)})
	--({\sx*(3.5100)},{\sy*(0.0046)})
	--({\sx*(3.5200)},{\sy*(0.0047)})
	--({\sx*(3.5300)},{\sy*(0.0048)})
	--({\sx*(3.5400)},{\sy*(0.0049)})
	--({\sx*(3.5500)},{\sy*(0.0050)})
	--({\sx*(3.5600)},{\sy*(0.0050)})
	--({\sx*(3.5700)},{\sy*(0.0050)})
	--({\sx*(3.5800)},{\sy*(0.0050)})
	--({\sx*(3.5900)},{\sy*(0.0050)})
	--({\sx*(3.6000)},{\sy*(0.0049)})
	--({\sx*(3.6100)},{\sy*(0.0048)})
	--({\sx*(3.6200)},{\sy*(0.0047)})
	--({\sx*(3.6300)},{\sy*(0.0045)})
	--({\sx*(3.6400)},{\sy*(0.0043)})
	--({\sx*(3.6500)},{\sy*(0.0041)})
	--({\sx*(3.6600)},{\sy*(0.0038)})
	--({\sx*(3.6700)},{\sy*(0.0035)})
	--({\sx*(3.6800)},{\sy*(0.0032)})
	--({\sx*(3.6900)},{\sy*(0.0028)})
	--({\sx*(3.7000)},{\sy*(0.0024)})
	--({\sx*(3.7100)},{\sy*(0.0020)})
	--({\sx*(3.7200)},{\sy*(0.0015)})
	--({\sx*(3.7300)},{\sy*(0.0010)})
	--({\sx*(3.7400)},{\sy*(0.0005)})
	--({\sx*(3.7500)},{\sy*(0.0000)})
	--({\sx*(3.7600)},{\sy*(-0.0006)})
	--({\sx*(3.7700)},{\sy*(-0.0011)})
	--({\sx*(3.7800)},{\sy*(-0.0017)})
	--({\sx*(3.7900)},{\sy*(-0.0023)})
	--({\sx*(3.8000)},{\sy*(-0.0029)})
	--({\sx*(3.8100)},{\sy*(-0.0036)})
	--({\sx*(3.8200)},{\sy*(-0.0042)})
	--({\sx*(3.8300)},{\sy*(-0.0048)})
	--({\sx*(3.8400)},{\sy*(-0.0055)})
	--({\sx*(3.8500)},{\sy*(-0.0061)})
	--({\sx*(3.8600)},{\sy*(-0.0067)})
	--({\sx*(3.8700)},{\sy*(-0.0073)})
	--({\sx*(3.8800)},{\sy*(-0.0079)})
	--({\sx*(3.8900)},{\sy*(-0.0085)})
	--({\sx*(3.9000)},{\sy*(-0.0091)})
	--({\sx*(3.9100)},{\sy*(-0.0096)})
	--({\sx*(3.9200)},{\sy*(-0.0101)})
	--({\sx*(3.9300)},{\sy*(-0.0105)})
	--({\sx*(3.9400)},{\sy*(-0.0109)})
	--({\sx*(3.9500)},{\sy*(-0.0113)})
	--({\sx*(3.9600)},{\sy*(-0.0116)})
	--({\sx*(3.9700)},{\sy*(-0.0119)})
	--({\sx*(3.9800)},{\sy*(-0.0121)})
	--({\sx*(3.9900)},{\sy*(-0.0122)})
	--({\sx*(4.0000)},{\sy*(-0.0122)})
	--({\sx*(4.0100)},{\sy*(-0.0122)})
	--({\sx*(4.0200)},{\sy*(-0.0121)})
	--({\sx*(4.0300)},{\sy*(-0.0120)})
	--({\sx*(4.0400)},{\sy*(-0.0117)})
	--({\sx*(4.0500)},{\sy*(-0.0113)})
	--({\sx*(4.0600)},{\sy*(-0.0109)})
	--({\sx*(4.0700)},{\sy*(-0.0104)})
	--({\sx*(4.0800)},{\sy*(-0.0097)})
	--({\sx*(4.0900)},{\sy*(-0.0090)})
	--({\sx*(4.1000)},{\sy*(-0.0082)})
	--({\sx*(4.1100)},{\sy*(-0.0072)})
	--({\sx*(4.1200)},{\sy*(-0.0062)})
	--({\sx*(4.1300)},{\sy*(-0.0051)})
	--({\sx*(4.1400)},{\sy*(-0.0038)})
	--({\sx*(4.1500)},{\sy*(-0.0025)})
	--({\sx*(4.1600)},{\sy*(-0.0010)})
	--({\sx*(4.1700)},{\sy*(0.0005)})
	--({\sx*(4.1800)},{\sy*(0.0022)})
	--({\sx*(4.1900)},{\sy*(0.0039)})
	--({\sx*(4.2000)},{\sy*(0.0057)})
	--({\sx*(4.2100)},{\sy*(0.0076)})
	--({\sx*(4.2200)},{\sy*(0.0096)})
	--({\sx*(4.2300)},{\sy*(0.0116)})
	--({\sx*(4.2400)},{\sy*(0.0137)})
	--({\sx*(4.2500)},{\sy*(0.0158)})
	--({\sx*(4.2600)},{\sy*(0.0180)})
	--({\sx*(4.2700)},{\sy*(0.0202)})
	--({\sx*(4.2800)},{\sy*(0.0225)})
	--({\sx*(4.2900)},{\sy*(0.0247)})
	--({\sx*(4.3000)},{\sy*(0.0270)})
	--({\sx*(4.3100)},{\sy*(0.0292)})
	--({\sx*(4.3200)},{\sy*(0.0313)})
	--({\sx*(4.3300)},{\sy*(0.0334)})
	--({\sx*(4.3400)},{\sy*(0.0355)})
	--({\sx*(4.3500)},{\sy*(0.0374)})
	--({\sx*(4.3600)},{\sy*(0.0393)})
	--({\sx*(4.3700)},{\sy*(0.0410)})
	--({\sx*(4.3800)},{\sy*(0.0425)})
	--({\sx*(4.3900)},{\sy*(0.0439)})
	--({\sx*(4.4000)},{\sy*(0.0450)})
	--({\sx*(4.4100)},{\sy*(0.0460)})
	--({\sx*(4.4200)},{\sy*(0.0467)})
	--({\sx*(4.4300)},{\sy*(0.0471)})
	--({\sx*(4.4400)},{\sy*(0.0472)})
	--({\sx*(4.4500)},{\sy*(0.0470)})
	--({\sx*(4.4600)},{\sy*(0.0464)})
	--({\sx*(4.4700)},{\sy*(0.0455)})
	--({\sx*(4.4800)},{\sy*(0.0441)})
	--({\sx*(4.4900)},{\sy*(0.0423)})
	--({\sx*(4.5000)},{\sy*(0.0401)})
	--({\sx*(4.5100)},{\sy*(0.0373)})
	--({\sx*(4.5200)},{\sy*(0.0341)})
	--({\sx*(4.5300)},{\sy*(0.0303)})
	--({\sx*(4.5400)},{\sy*(0.0259)})
	--({\sx*(4.5500)},{\sy*(0.0210)})
	--({\sx*(4.5600)},{\sy*(0.0154)})
	--({\sx*(4.5700)},{\sy*(0.0092)})
	--({\sx*(4.5800)},{\sy*(0.0024)})
	--({\sx*(4.5900)},{\sy*(-0.0051)})
	--({\sx*(4.6000)},{\sy*(-0.0132)})
	--({\sx*(4.6100)},{\sy*(-0.0220)})
	--({\sx*(4.6200)},{\sy*(-0.0315)})
	--({\sx*(4.6300)},{\sy*(-0.0417)})
	--({\sx*(4.6400)},{\sy*(-0.0525)})
	--({\sx*(4.6500)},{\sy*(-0.0640)})
	--({\sx*(4.6600)},{\sy*(-0.0762)})
	--({\sx*(4.6700)},{\sy*(-0.0890)})
	--({\sx*(4.6800)},{\sy*(-0.1024)})
	--({\sx*(4.6900)},{\sy*(-0.1163)})
	--({\sx*(4.7000)},{\sy*(-0.1308)})
	--({\sx*(4.7100)},{\sy*(-0.1458)})
	--({\sx*(4.7200)},{\sy*(-0.1611)})
	--({\sx*(4.7300)},{\sy*(-0.1768)})
	--({\sx*(4.7400)},{\sy*(-0.1927)})
	--({\sx*(4.7500)},{\sy*(-0.2088)})
	--({\sx*(4.7600)},{\sy*(-0.2250)})
	--({\sx*(4.7700)},{\sy*(-0.2410)})
	--({\sx*(4.7800)},{\sy*(-0.2569)})
	--({\sx*(4.7900)},{\sy*(-0.2723)})
	--({\sx*(4.8000)},{\sy*(-0.2873)})
	--({\sx*(4.8100)},{\sy*(-0.3015)})
	--({\sx*(4.8200)},{\sy*(-0.3148)})
	--({\sx*(4.8300)},{\sy*(-0.3269)})
	--({\sx*(4.8400)},{\sy*(-0.3376)})
	--({\sx*(4.8500)},{\sy*(-0.3466)})
	--({\sx*(4.8600)},{\sy*(-0.3536)})
	--({\sx*(4.8700)},{\sy*(-0.3583)})
	--({\sx*(4.8800)},{\sy*(-0.3603)})
	--({\sx*(4.8900)},{\sy*(-0.3592)})
	--({\sx*(4.9000)},{\sy*(-0.3547)})
	--({\sx*(4.9100)},{\sy*(-0.3463)})
	--({\sx*(4.9200)},{\sy*(-0.3334)})
	--({\sx*(4.9300)},{\sy*(-0.3156)})
	--({\sx*(4.9400)},{\sy*(-0.2923)})
	--({\sx*(4.9500)},{\sy*(-0.2630)})
	--({\sx*(4.9600)},{\sy*(-0.2268)})
	--({\sx*(4.9700)},{\sy*(-0.1832)})
	--({\sx*(4.9800)},{\sy*(-0.1314)})
	--({\sx*(4.9900)},{\sy*(-0.0706)})
	--({\sx*(5.0000)},{\sy*(0.0000)});
}
\def\relfehlerf{
\draw[color=blue,line width=1.4pt,line join=round] ({\sx*(0.000)},{\sy*(0.0000)})
	--({\sx*(0.0100)},{\sy*(0.0001)})
	--({\sx*(0.0200)},{\sy*(0.0001)})
	--({\sx*(0.0300)},{\sy*(0.0002)})
	--({\sx*(0.0400)},{\sy*(0.0002)})
	--({\sx*(0.0500)},{\sy*(0.0003)})
	--({\sx*(0.0600)},{\sy*(0.0003)})
	--({\sx*(0.0700)},{\sy*(0.0003)})
	--({\sx*(0.0800)},{\sy*(0.0003)})
	--({\sx*(0.0900)},{\sy*(0.0004)})
	--({\sx*(0.1000)},{\sy*(0.0004)})
	--({\sx*(0.1100)},{\sy*(0.0004)})
	--({\sx*(0.1200)},{\sy*(0.0004)})
	--({\sx*(0.1300)},{\sy*(0.0004)})
	--({\sx*(0.1400)},{\sy*(0.0004)})
	--({\sx*(0.1500)},{\sy*(0.0004)})
	--({\sx*(0.1600)},{\sy*(0.0003)})
	--({\sx*(0.1700)},{\sy*(0.0003)})
	--({\sx*(0.1800)},{\sy*(0.0003)})
	--({\sx*(0.1900)},{\sy*(0.0003)})
	--({\sx*(0.2000)},{\sy*(0.0003)})
	--({\sx*(0.2100)},{\sy*(0.0003)})
	--({\sx*(0.2200)},{\sy*(0.0003)})
	--({\sx*(0.2300)},{\sy*(0.0002)})
	--({\sx*(0.2400)},{\sy*(0.0002)})
	--({\sx*(0.2500)},{\sy*(0.0002)})
	--({\sx*(0.2600)},{\sy*(0.0002)})
	--({\sx*(0.2700)},{\sy*(0.0002)})
	--({\sx*(0.2800)},{\sy*(0.0002)})
	--({\sx*(0.2900)},{\sy*(0.0001)})
	--({\sx*(0.3000)},{\sy*(0.0001)})
	--({\sx*(0.3100)},{\sy*(0.0001)})
	--({\sx*(0.3200)},{\sy*(0.0001)})
	--({\sx*(0.3300)},{\sy*(0.0001)})
	--({\sx*(0.3400)},{\sy*(0.0001)})
	--({\sx*(0.3500)},{\sy*(0.0001)})
	--({\sx*(0.3600)},{\sy*(0.0001)})
	--({\sx*(0.3700)},{\sy*(0.0000)})
	--({\sx*(0.3800)},{\sy*(0.0000)})
	--({\sx*(0.3900)},{\sy*(0.0000)})
	--({\sx*(0.4000)},{\sy*(0.0000)})
	--({\sx*(0.4100)},{\sy*(0.0000)})
	--({\sx*(0.4200)},{\sy*(-0.0000)})
	--({\sx*(0.4300)},{\sy*(-0.0000)})
	--({\sx*(0.4400)},{\sy*(-0.0000)})
	--({\sx*(0.4500)},{\sy*(-0.0000)})
	--({\sx*(0.4600)},{\sy*(-0.0000)})
	--({\sx*(0.4700)},{\sy*(-0.0000)})
	--({\sx*(0.4800)},{\sy*(-0.0000)})
	--({\sx*(0.4900)},{\sy*(-0.0000)})
	--({\sx*(0.5000)},{\sy*(-0.0000)})
	--({\sx*(0.5100)},{\sy*(-0.0000)})
	--({\sx*(0.5200)},{\sy*(-0.0000)})
	--({\sx*(0.5300)},{\sy*(-0.0000)})
	--({\sx*(0.5400)},{\sy*(-0.0000)})
	--({\sx*(0.5500)},{\sy*(-0.0000)})
	--({\sx*(0.5600)},{\sy*(-0.0000)})
	--({\sx*(0.5700)},{\sy*(-0.0000)})
	--({\sx*(0.5800)},{\sy*(-0.0000)})
	--({\sx*(0.5900)},{\sy*(-0.0000)})
	--({\sx*(0.6000)},{\sy*(-0.0000)})
	--({\sx*(0.6100)},{\sy*(-0.0000)})
	--({\sx*(0.6200)},{\sy*(-0.0000)})
	--({\sx*(0.6300)},{\sy*(-0.0000)})
	--({\sx*(0.6400)},{\sy*(-0.0000)})
	--({\sx*(0.6500)},{\sy*(-0.0000)})
	--({\sx*(0.6600)},{\sy*(-0.0000)})
	--({\sx*(0.6700)},{\sy*(-0.0000)})
	--({\sx*(0.6800)},{\sy*(-0.0000)})
	--({\sx*(0.6900)},{\sy*(-0.0000)})
	--({\sx*(0.7000)},{\sy*(-0.0000)})
	--({\sx*(0.7100)},{\sy*(-0.0000)})
	--({\sx*(0.7200)},{\sy*(-0.0000)})
	--({\sx*(0.7300)},{\sy*(-0.0000)})
	--({\sx*(0.7400)},{\sy*(-0.0000)})
	--({\sx*(0.7500)},{\sy*(-0.0000)})
	--({\sx*(0.7600)},{\sy*(-0.0000)})
	--({\sx*(0.7700)},{\sy*(-0.0000)})
	--({\sx*(0.7800)},{\sy*(-0.0000)})
	--({\sx*(0.7900)},{\sy*(-0.0000)})
	--({\sx*(0.8000)},{\sy*(-0.0000)})
	--({\sx*(0.8100)},{\sy*(-0.0000)})
	--({\sx*(0.8200)},{\sy*(-0.0000)})
	--({\sx*(0.8300)},{\sy*(-0.0000)})
	--({\sx*(0.8400)},{\sy*(0.0000)})
	--({\sx*(0.8500)},{\sy*(0.0000)})
	--({\sx*(0.8600)},{\sy*(0.0000)})
	--({\sx*(0.8700)},{\sy*(0.0000)})
	--({\sx*(0.8800)},{\sy*(0.0000)})
	--({\sx*(0.8900)},{\sy*(0.0000)})
	--({\sx*(0.9000)},{\sy*(0.0000)})
	--({\sx*(0.9100)},{\sy*(0.0000)})
	--({\sx*(0.9200)},{\sy*(0.0000)})
	--({\sx*(0.9300)},{\sy*(0.0000)})
	--({\sx*(0.9400)},{\sy*(0.0000)})
	--({\sx*(0.9500)},{\sy*(0.0000)})
	--({\sx*(0.9600)},{\sy*(0.0000)})
	--({\sx*(0.9700)},{\sy*(0.0000)})
	--({\sx*(0.9800)},{\sy*(0.0000)})
	--({\sx*(0.9900)},{\sy*(0.0000)})
	--({\sx*(1.0000)},{\sy*(0.0000)})
	--({\sx*(1.0100)},{\sy*(0.0000)})
	--({\sx*(1.0200)},{\sy*(0.0000)})
	--({\sx*(1.0300)},{\sy*(0.0000)})
	--({\sx*(1.0400)},{\sy*(0.0000)})
	--({\sx*(1.0500)},{\sy*(0.0000)})
	--({\sx*(1.0600)},{\sy*(0.0000)})
	--({\sx*(1.0700)},{\sy*(0.0000)})
	--({\sx*(1.0800)},{\sy*(0.0000)})
	--({\sx*(1.0900)},{\sy*(0.0000)})
	--({\sx*(1.1000)},{\sy*(0.0000)})
	--({\sx*(1.1100)},{\sy*(0.0000)})
	--({\sx*(1.1200)},{\sy*(0.0000)})
	--({\sx*(1.1300)},{\sy*(0.0000)})
	--({\sx*(1.1400)},{\sy*(0.0000)})
	--({\sx*(1.1500)},{\sy*(0.0000)})
	--({\sx*(1.1600)},{\sy*(0.0000)})
	--({\sx*(1.1700)},{\sy*(0.0000)})
	--({\sx*(1.1800)},{\sy*(0.0000)})
	--({\sx*(1.1900)},{\sy*(0.0000)})
	--({\sx*(1.2000)},{\sy*(0.0000)})
	--({\sx*(1.2100)},{\sy*(0.0000)})
	--({\sx*(1.2200)},{\sy*(0.0000)})
	--({\sx*(1.2300)},{\sy*(0.0000)})
	--({\sx*(1.2400)},{\sy*(0.0000)})
	--({\sx*(1.2500)},{\sy*(0.0000)})
	--({\sx*(1.2600)},{\sy*(-0.0000)})
	--({\sx*(1.2700)},{\sy*(-0.0000)})
	--({\sx*(1.2800)},{\sy*(-0.0000)})
	--({\sx*(1.2900)},{\sy*(-0.0000)})
	--({\sx*(1.3000)},{\sy*(-0.0000)})
	--({\sx*(1.3100)},{\sy*(-0.0000)})
	--({\sx*(1.3200)},{\sy*(-0.0000)})
	--({\sx*(1.3300)},{\sy*(-0.0000)})
	--({\sx*(1.3400)},{\sy*(-0.0000)})
	--({\sx*(1.3500)},{\sy*(-0.0000)})
	--({\sx*(1.3600)},{\sy*(-0.0000)})
	--({\sx*(1.3700)},{\sy*(-0.0000)})
	--({\sx*(1.3800)},{\sy*(-0.0000)})
	--({\sx*(1.3900)},{\sy*(-0.0000)})
	--({\sx*(1.4000)},{\sy*(-0.0000)})
	--({\sx*(1.4100)},{\sy*(-0.0000)})
	--({\sx*(1.4200)},{\sy*(-0.0000)})
	--({\sx*(1.4300)},{\sy*(-0.0000)})
	--({\sx*(1.4400)},{\sy*(-0.0000)})
	--({\sx*(1.4500)},{\sy*(-0.0000)})
	--({\sx*(1.4600)},{\sy*(-0.0000)})
	--({\sx*(1.4700)},{\sy*(-0.0000)})
	--({\sx*(1.4800)},{\sy*(-0.0000)})
	--({\sx*(1.4900)},{\sy*(-0.0000)})
	--({\sx*(1.5000)},{\sy*(-0.0000)})
	--({\sx*(1.5100)},{\sy*(-0.0000)})
	--({\sx*(1.5200)},{\sy*(-0.0000)})
	--({\sx*(1.5300)},{\sy*(-0.0000)})
	--({\sx*(1.5400)},{\sy*(-0.0000)})
	--({\sx*(1.5500)},{\sy*(-0.0000)})
	--({\sx*(1.5600)},{\sy*(-0.0000)})
	--({\sx*(1.5700)},{\sy*(-0.0000)})
	--({\sx*(1.5800)},{\sy*(-0.0000)})
	--({\sx*(1.5900)},{\sy*(-0.0000)})
	--({\sx*(1.6000)},{\sy*(-0.0000)})
	--({\sx*(1.6100)},{\sy*(-0.0000)})
	--({\sx*(1.6200)},{\sy*(-0.0000)})
	--({\sx*(1.6300)},{\sy*(-0.0000)})
	--({\sx*(1.6400)},{\sy*(-0.0000)})
	--({\sx*(1.6500)},{\sy*(-0.0000)})
	--({\sx*(1.6600)},{\sy*(-0.0000)})
	--({\sx*(1.6700)},{\sy*(0.0000)})
	--({\sx*(1.6800)},{\sy*(0.0000)})
	--({\sx*(1.6900)},{\sy*(0.0000)})
	--({\sx*(1.7000)},{\sy*(0.0000)})
	--({\sx*(1.7100)},{\sy*(0.0000)})
	--({\sx*(1.7200)},{\sy*(0.0000)})
	--({\sx*(1.7300)},{\sy*(0.0000)})
	--({\sx*(1.7400)},{\sy*(0.0000)})
	--({\sx*(1.7500)},{\sy*(0.0000)})
	--({\sx*(1.7600)},{\sy*(0.0000)})
	--({\sx*(1.7700)},{\sy*(0.0000)})
	--({\sx*(1.7800)},{\sy*(0.0000)})
	--({\sx*(1.7900)},{\sy*(0.0000)})
	--({\sx*(1.8000)},{\sy*(0.0000)})
	--({\sx*(1.8100)},{\sy*(0.0000)})
	--({\sx*(1.8200)},{\sy*(0.0000)})
	--({\sx*(1.8300)},{\sy*(0.0000)})
	--({\sx*(1.8400)},{\sy*(0.0000)})
	--({\sx*(1.8500)},{\sy*(0.0000)})
	--({\sx*(1.8600)},{\sy*(0.0000)})
	--({\sx*(1.8700)},{\sy*(0.0000)})
	--({\sx*(1.8800)},{\sy*(0.0000)})
	--({\sx*(1.8900)},{\sy*(0.0000)})
	--({\sx*(1.9000)},{\sy*(0.0000)})
	--({\sx*(1.9100)},{\sy*(0.0000)})
	--({\sx*(1.9200)},{\sy*(0.0000)})
	--({\sx*(1.9300)},{\sy*(0.0000)})
	--({\sx*(1.9400)},{\sy*(0.0000)})
	--({\sx*(1.9500)},{\sy*(0.0000)})
	--({\sx*(1.9600)},{\sy*(0.0000)})
	--({\sx*(1.9700)},{\sy*(0.0000)})
	--({\sx*(1.9800)},{\sy*(0.0000)})
	--({\sx*(1.9900)},{\sy*(0.0000)})
	--({\sx*(2.0000)},{\sy*(0.0000)})
	--({\sx*(2.0100)},{\sy*(0.0000)})
	--({\sx*(2.0200)},{\sy*(0.0000)})
	--({\sx*(2.0300)},{\sy*(0.0000)})
	--({\sx*(2.0400)},{\sy*(0.0000)})
	--({\sx*(2.0500)},{\sy*(0.0000)})
	--({\sx*(2.0600)},{\sy*(0.0000)})
	--({\sx*(2.0700)},{\sy*(0.0000)})
	--({\sx*(2.0800)},{\sy*(0.0000)})
	--({\sx*(2.0900)},{\sy*(-0.0000)})
	--({\sx*(2.1000)},{\sy*(-0.0000)})
	--({\sx*(2.1100)},{\sy*(-0.0000)})
	--({\sx*(2.1200)},{\sy*(-0.0000)})
	--({\sx*(2.1300)},{\sy*(-0.0000)})
	--({\sx*(2.1400)},{\sy*(-0.0000)})
	--({\sx*(2.1500)},{\sy*(-0.0000)})
	--({\sx*(2.1600)},{\sy*(-0.0000)})
	--({\sx*(2.1700)},{\sy*(-0.0000)})
	--({\sx*(2.1800)},{\sy*(-0.0000)})
	--({\sx*(2.1900)},{\sy*(-0.0000)})
	--({\sx*(2.2000)},{\sy*(-0.0000)})
	--({\sx*(2.2100)},{\sy*(-0.0000)})
	--({\sx*(2.2200)},{\sy*(-0.0000)})
	--({\sx*(2.2300)},{\sy*(-0.0000)})
	--({\sx*(2.2400)},{\sy*(-0.0000)})
	--({\sx*(2.2500)},{\sy*(-0.0000)})
	--({\sx*(2.2600)},{\sy*(-0.0000)})
	--({\sx*(2.2700)},{\sy*(-0.0000)})
	--({\sx*(2.2800)},{\sy*(-0.0000)})
	--({\sx*(2.2900)},{\sy*(-0.0000)})
	--({\sx*(2.3000)},{\sy*(-0.0000)})
	--({\sx*(2.3100)},{\sy*(-0.0000)})
	--({\sx*(2.3200)},{\sy*(-0.0000)})
	--({\sx*(2.3300)},{\sy*(-0.0000)})
	--({\sx*(2.3400)},{\sy*(-0.0000)})
	--({\sx*(2.3500)},{\sy*(-0.0000)})
	--({\sx*(2.3600)},{\sy*(-0.0000)})
	--({\sx*(2.3700)},{\sy*(-0.0000)})
	--({\sx*(2.3800)},{\sy*(-0.0000)})
	--({\sx*(2.3900)},{\sy*(-0.0000)})
	--({\sx*(2.4000)},{\sy*(-0.0000)})
	--({\sx*(2.4100)},{\sy*(-0.0000)})
	--({\sx*(2.4200)},{\sy*(-0.0000)})
	--({\sx*(2.4300)},{\sy*(-0.0000)})
	--({\sx*(2.4400)},{\sy*(-0.0000)})
	--({\sx*(2.4500)},{\sy*(-0.0000)})
	--({\sx*(2.4600)},{\sy*(-0.0000)})
	--({\sx*(2.4700)},{\sy*(-0.0000)})
	--({\sx*(2.4800)},{\sy*(-0.0000)})
	--({\sx*(2.4900)},{\sy*(-0.0000)})
	--({\sx*(2.5000)},{\sy*(0.0000)})
	--({\sx*(2.5100)},{\sy*(0.0000)})
	--({\sx*(2.5200)},{\sy*(0.0000)})
	--({\sx*(2.5300)},{\sy*(0.0000)})
	--({\sx*(2.5400)},{\sy*(0.0000)})
	--({\sx*(2.5500)},{\sy*(0.0000)})
	--({\sx*(2.5600)},{\sy*(0.0000)})
	--({\sx*(2.5700)},{\sy*(0.0000)})
	--({\sx*(2.5800)},{\sy*(0.0000)})
	--({\sx*(2.5900)},{\sy*(0.0000)})
	--({\sx*(2.6000)},{\sy*(0.0000)})
	--({\sx*(2.6100)},{\sy*(0.0000)})
	--({\sx*(2.6200)},{\sy*(0.0000)})
	--({\sx*(2.6300)},{\sy*(0.0000)})
	--({\sx*(2.6400)},{\sy*(0.0000)})
	--({\sx*(2.6500)},{\sy*(0.0000)})
	--({\sx*(2.6600)},{\sy*(0.0000)})
	--({\sx*(2.6700)},{\sy*(0.0000)})
	--({\sx*(2.6800)},{\sy*(0.0000)})
	--({\sx*(2.6900)},{\sy*(0.0000)})
	--({\sx*(2.7000)},{\sy*(0.0000)})
	--({\sx*(2.7100)},{\sy*(0.0000)})
	--({\sx*(2.7200)},{\sy*(0.0000)})
	--({\sx*(2.7300)},{\sy*(0.0000)})
	--({\sx*(2.7400)},{\sy*(0.0000)})
	--({\sx*(2.7500)},{\sy*(0.0000)})
	--({\sx*(2.7600)},{\sy*(0.0000)})
	--({\sx*(2.7700)},{\sy*(0.0000)})
	--({\sx*(2.7800)},{\sy*(0.0000)})
	--({\sx*(2.7900)},{\sy*(0.0000)})
	--({\sx*(2.8000)},{\sy*(0.0000)})
	--({\sx*(2.8100)},{\sy*(0.0000)})
	--({\sx*(2.8200)},{\sy*(0.0000)})
	--({\sx*(2.8300)},{\sy*(0.0000)})
	--({\sx*(2.8400)},{\sy*(0.0000)})
	--({\sx*(2.8500)},{\sy*(0.0000)})
	--({\sx*(2.8600)},{\sy*(0.0000)})
	--({\sx*(2.8700)},{\sy*(0.0000)})
	--({\sx*(2.8800)},{\sy*(0.0000)})
	--({\sx*(2.8900)},{\sy*(0.0000)})
	--({\sx*(2.9000)},{\sy*(0.0000)})
	--({\sx*(2.9100)},{\sy*(0.0000)})
	--({\sx*(2.9200)},{\sy*(-0.0000)})
	--({\sx*(2.9300)},{\sy*(-0.0000)})
	--({\sx*(2.9400)},{\sy*(-0.0000)})
	--({\sx*(2.9500)},{\sy*(-0.0000)})
	--({\sx*(2.9600)},{\sy*(-0.0000)})
	--({\sx*(2.9700)},{\sy*(-0.0000)})
	--({\sx*(2.9800)},{\sy*(-0.0000)})
	--({\sx*(2.9900)},{\sy*(-0.0000)})
	--({\sx*(3.0000)},{\sy*(-0.0001)})
	--({\sx*(3.0100)},{\sy*(-0.0001)})
	--({\sx*(3.0200)},{\sy*(-0.0001)})
	--({\sx*(3.0300)},{\sy*(-0.0001)})
	--({\sx*(3.0400)},{\sy*(-0.0001)})
	--({\sx*(3.0500)},{\sy*(-0.0001)})
	--({\sx*(3.0600)},{\sy*(-0.0001)})
	--({\sx*(3.0700)},{\sy*(-0.0001)})
	--({\sx*(3.0800)},{\sy*(-0.0001)})
	--({\sx*(3.0900)},{\sy*(-0.0001)})
	--({\sx*(3.1000)},{\sy*(-0.0001)})
	--({\sx*(3.1100)},{\sy*(-0.0001)})
	--({\sx*(3.1200)},{\sy*(-0.0001)})
	--({\sx*(3.1300)},{\sy*(-0.0001)})
	--({\sx*(3.1400)},{\sy*(-0.0002)})
	--({\sx*(3.1500)},{\sy*(-0.0002)})
	--({\sx*(3.1600)},{\sy*(-0.0002)})
	--({\sx*(3.1700)},{\sy*(-0.0002)})
	--({\sx*(3.1800)},{\sy*(-0.0002)})
	--({\sx*(3.1900)},{\sy*(-0.0002)})
	--({\sx*(3.2000)},{\sy*(-0.0002)})
	--({\sx*(3.2100)},{\sy*(-0.0002)})
	--({\sx*(3.2200)},{\sy*(-0.0002)})
	--({\sx*(3.2300)},{\sy*(-0.0002)})
	--({\sx*(3.2400)},{\sy*(-0.0002)})
	--({\sx*(3.2500)},{\sy*(-0.0001)})
	--({\sx*(3.2600)},{\sy*(-0.0001)})
	--({\sx*(3.2700)},{\sy*(-0.0001)})
	--({\sx*(3.2800)},{\sy*(-0.0001)})
	--({\sx*(3.2900)},{\sy*(-0.0001)})
	--({\sx*(3.3000)},{\sy*(-0.0001)})
	--({\sx*(3.3100)},{\sy*(-0.0001)})
	--({\sx*(3.3200)},{\sy*(-0.0000)})
	--({\sx*(3.3300)},{\sy*(-0.0000)})
	--({\sx*(3.3400)},{\sy*(0.0000)})
	--({\sx*(3.3500)},{\sy*(0.0000)})
	--({\sx*(3.3600)},{\sy*(0.0001)})
	--({\sx*(3.3700)},{\sy*(0.0001)})
	--({\sx*(3.3800)},{\sy*(0.0001)})
	--({\sx*(3.3900)},{\sy*(0.0002)})
	--({\sx*(3.4000)},{\sy*(0.0002)})
	--({\sx*(3.4100)},{\sy*(0.0003)})
	--({\sx*(3.4200)},{\sy*(0.0003)})
	--({\sx*(3.4300)},{\sy*(0.0004)})
	--({\sx*(3.4400)},{\sy*(0.0004)})
	--({\sx*(3.4500)},{\sy*(0.0005)})
	--({\sx*(3.4600)},{\sy*(0.0005)})
	--({\sx*(3.4700)},{\sy*(0.0006)})
	--({\sx*(3.4800)},{\sy*(0.0006)})
	--({\sx*(3.4900)},{\sy*(0.0007)})
	--({\sx*(3.5000)},{\sy*(0.0007)})
	--({\sx*(3.5100)},{\sy*(0.0008)})
	--({\sx*(3.5200)},{\sy*(0.0008)})
	--({\sx*(3.5300)},{\sy*(0.0009)})
	--({\sx*(3.5400)},{\sy*(0.0009)})
	--({\sx*(3.5500)},{\sy*(0.0010)})
	--({\sx*(3.5600)},{\sy*(0.0010)})
	--({\sx*(3.5700)},{\sy*(0.0011)})
	--({\sx*(3.5800)},{\sy*(0.0011)})
	--({\sx*(3.5900)},{\sy*(0.0011)})
	--({\sx*(3.6000)},{\sy*(0.0012)})
	--({\sx*(3.6100)},{\sy*(0.0012)})
	--({\sx*(3.6200)},{\sy*(0.0012)})
	--({\sx*(3.6300)},{\sy*(0.0012)})
	--({\sx*(3.6400)},{\sy*(0.0012)})
	--({\sx*(3.6500)},{\sy*(0.0012)})
	--({\sx*(3.6600)},{\sy*(0.0011)})
	--({\sx*(3.6700)},{\sy*(0.0011)})
	--({\sx*(3.6800)},{\sy*(0.0010)})
	--({\sx*(3.6900)},{\sy*(0.0009)})
	--({\sx*(3.7000)},{\sy*(0.0008)})
	--({\sx*(3.7100)},{\sy*(0.0007)})
	--({\sx*(3.7200)},{\sy*(0.0006)})
	--({\sx*(3.7300)},{\sy*(0.0004)})
	--({\sx*(3.7400)},{\sy*(0.0002)})
	--({\sx*(3.7500)},{\sy*(0.0000)})
	--({\sx*(3.7600)},{\sy*(-0.0002)})
	--({\sx*(3.7700)},{\sy*(-0.0005)})
	--({\sx*(3.7800)},{\sy*(-0.0008)})
	--({\sx*(3.7900)},{\sy*(-0.0011)})
	--({\sx*(3.8000)},{\sy*(-0.0015)})
	--({\sx*(3.8100)},{\sy*(-0.0019)})
	--({\sx*(3.8200)},{\sy*(-0.0023)})
	--({\sx*(3.8300)},{\sy*(-0.0027)})
	--({\sx*(3.8400)},{\sy*(-0.0032)})
	--({\sx*(3.8500)},{\sy*(-0.0037)})
	--({\sx*(3.8600)},{\sy*(-0.0043)})
	--({\sx*(3.8700)},{\sy*(-0.0048)})
	--({\sx*(3.8800)},{\sy*(-0.0054)})
	--({\sx*(3.8900)},{\sy*(-0.0061)})
	--({\sx*(3.9000)},{\sy*(-0.0067)})
	--({\sx*(3.9100)},{\sy*(-0.0074)})
	--({\sx*(3.9200)},{\sy*(-0.0081)})
	--({\sx*(3.9300)},{\sy*(-0.0088)})
	--({\sx*(3.9400)},{\sy*(-0.0095)})
	--({\sx*(3.9500)},{\sy*(-0.0102)})
	--({\sx*(3.9600)},{\sy*(-0.0109)})
	--({\sx*(3.9700)},{\sy*(-0.0116)})
	--({\sx*(3.9800)},{\sy*(-0.0123)})
	--({\sx*(3.9900)},{\sy*(-0.0130)})
	--({\sx*(4.0000)},{\sy*(-0.0135)})
	--({\sx*(4.0100)},{\sy*(-0.0141)})
	--({\sx*(4.0200)},{\sy*(-0.0146)})
	--({\sx*(4.0300)},{\sy*(-0.0149)})
	--({\sx*(4.0400)},{\sy*(-0.0152)})
	--({\sx*(4.0500)},{\sy*(-0.0154)})
	--({\sx*(4.0600)},{\sy*(-0.0154)})
	--({\sx*(4.0700)},{\sy*(-0.0152)})
	--({\sx*(4.0800)},{\sy*(-0.0149)})
	--({\sx*(4.0900)},{\sy*(-0.0143)})
	--({\sx*(4.1000)},{\sy*(-0.0136)})
	--({\sx*(4.1100)},{\sy*(-0.0125)})
	--({\sx*(4.1200)},{\sy*(-0.0111)})
	--({\sx*(4.1300)},{\sy*(-0.0094)})
	--({\sx*(4.1400)},{\sy*(-0.0074)})
	--({\sx*(4.1500)},{\sy*(-0.0050)})
	--({\sx*(4.1600)},{\sy*(-0.0021)})
	--({\sx*(4.1700)},{\sy*(0.0011)})
	--({\sx*(4.1800)},{\sy*(0.0049)})
	--({\sx*(4.1900)},{\sy*(0.0092)})
	--({\sx*(4.2000)},{\sy*(0.0139)})
	--({\sx*(4.2100)},{\sy*(0.0193)})
	--({\sx*(4.2200)},{\sy*(0.0252)})
	--({\sx*(4.2300)},{\sy*(0.0316)})
	--({\sx*(4.2400)},{\sy*(0.0387)})
	--({\sx*(4.2500)},{\sy*(0.0463)})
	--({\sx*(4.2600)},{\sy*(0.0545)})
	--({\sx*(4.2700)},{\sy*(0.0633)})
	--({\sx*(4.2800)},{\sy*(0.0726)})
	--({\sx*(4.2900)},{\sy*(0.0824)})
	--({\sx*(4.3000)},{\sy*(0.0928)})
	--({\sx*(4.3100)},{\sy*(0.1035)})
	--({\sx*(4.3200)},{\sy*(0.1147)})
	--({\sx*(4.3300)},{\sy*(0.1262)})
	--({\sx*(4.3400)},{\sy*(0.1379)})
	--({\sx*(4.3500)},{\sy*(0.1499)})
	--({\sx*(4.3600)},{\sy*(0.1619)})
	--({\sx*(4.3700)},{\sy*(0.1739)})
	--({\sx*(4.3800)},{\sy*(0.1858)})
	--({\sx*(4.3900)},{\sy*(0.1975)})
	--({\sx*(4.4000)},{\sy*(0.2089)})
	--({\sx*(4.4100)},{\sy*(0.2198)})
	--({\sx*(4.4200)},{\sy*(0.2301)})
	--({\sx*(4.4300)},{\sy*(0.2396)})
	--({\sx*(4.4400)},{\sy*(0.2483)})
	--({\sx*(4.4500)},{\sy*(0.2558)})
	--({\sx*(4.4600)},{\sy*(0.2620)})
	--({\sx*(4.4700)},{\sy*(0.2667)})
	--({\sx*(4.4800)},{\sy*(0.2695)})
	--({\sx*(4.4900)},{\sy*(0.2701)})
	--({\sx*(4.5000)},{\sy*(0.2682)})
	--({\sx*(4.5100)},{\sy*(0.2632)})
	--({\sx*(4.5200)},{\sy*(0.2543)})
	--({\sx*(4.5300)},{\sy*(0.2407)})
	--({\sx*(4.5400)},{\sy*(0.2211)})
	--({\sx*(4.5500)},{\sy*(0.1938)})
	--({\sx*(4.5600)},{\sy*(0.1561)})
	--({\sx*(4.5700)},{\sy*(0.1040)})
	--({\sx*(4.5800)},{\sy*(0.0308)})
	--({\sx*(4.5900)},{\sy*(-0.0748)})
	--({\sx*(4.6000)},{\sy*(-0.2349)})
	--({\sx*(4.6100)},{\sy*(-0.4976)})
	--({\sx*(4.6200)},{\sy*(-0.9922)})
	--({\sx*(4.6300)},{\sy*(-2.2263)})
	--({\sx*(4.6400)},{\sy*(-10.2251)})
	--({\sx*(4.6500)},{\sy*(7.1209)})
	--({\sx*(4.6600)},{\sy*(3.2208)})
	--({\sx*(4.6700)},{\sy*(2.2909)})
	--({\sx*(4.6800)},{\sy*(1.8777)})
	--({\sx*(4.6900)},{\sy*(1.6462)})
	--({\sx*(4.7000)},{\sy*(1.4994)})
	--({\sx*(4.7100)},{\sy*(1.3990)})
	--({\sx*(4.7200)},{\sy*(1.3265)})
	--({\sx*(4.7300)},{\sy*(1.2722)})
	--({\sx*(4.7400)},{\sy*(1.2303)})
	--({\sx*(4.7500)},{\sy*(1.1972)})
	--({\sx*(4.7600)},{\sy*(1.1707)})
	--({\sx*(4.7700)},{\sy*(1.1491)})
	--({\sx*(4.7800)},{\sy*(1.1313)})
	--({\sx*(4.7900)},{\sy*(1.1166)})
	--({\sx*(4.8000)},{\sy*(1.1041)})
	--({\sx*(4.8100)},{\sy*(1.0937)})
	--({\sx*(4.8200)},{\sy*(1.0848)})
	--({\sx*(4.8300)},{\sy*(1.0773)})
	--({\sx*(4.8400)},{\sy*(1.0709)})
	--({\sx*(4.8500)},{\sy*(1.0654)})
	--({\sx*(4.8600)},{\sy*(1.0608)})
	--({\sx*(4.8700)},{\sy*(1.0570)})
	--({\sx*(4.8800)},{\sy*(1.0538)})
	--({\sx*(4.8900)},{\sy*(1.0513)})
	--({\sx*(4.9000)},{\sy*(1.0494)})
	--({\sx*(4.9100)},{\sy*(1.0481)})
	--({\sx*(4.9200)},{\sy*(1.0475)})
	--({\sx*(4.9300)},{\sy*(1.0478)})
	--({\sx*(4.9400)},{\sy*(1.0492)})
	--({\sx*(4.9500)},{\sy*(1.0522)})
	--({\sx*(4.9600)},{\sy*(1.0579)})
	--({\sx*(4.9700)},{\sy*(1.0689)})
	--({\sx*(4.9800)},{\sy*(1.0935)})
	--({\sx*(4.9900)},{\sy*(1.1784)})
	--({\sx*(5.0000)},{\sy*(0.0000)});
}
\def\xwerteg{
\fill[color=red] (0.0000,0) circle[radius={0.07/\skala}];
\fill[color=red] (0.3571,0) circle[radius={0.07/\skala}];
\fill[color=red] (0.7143,0) circle[radius={0.07/\skala}];
\fill[color=red] (1.0714,0) circle[radius={0.07/\skala}];
\fill[color=red] (1.4286,0) circle[radius={0.07/\skala}];
\fill[color=red] (1.7857,0) circle[radius={0.07/\skala}];
\fill[color=red] (2.1429,0) circle[radius={0.07/\skala}];
\fill[color=red] (2.5000,0) circle[radius={0.07/\skala}];
\fill[color=red] (2.8571,0) circle[radius={0.07/\skala}];
\fill[color=red] (3.2143,0) circle[radius={0.07/\skala}];
\fill[color=red] (3.5714,0) circle[radius={0.07/\skala}];
\fill[color=red] (3.9286,0) circle[radius={0.07/\skala}];
\fill[color=red] (4.2857,0) circle[radius={0.07/\skala}];
\fill[color=red] (4.6429,0) circle[radius={0.07/\skala}];
\fill[color=red] (5.0000,0) circle[radius={0.07/\skala}];
}
\def\punkteg{14}
\def\maxfehlerg{2.281\cdot 10^{-5}}
\def\fehlerg{
\draw[color=red,line width=1.4pt,line join=round] ({\sx*(0.000)},{\sy*(0.0000)})
	--({\sx*(0.0100)},{\sy*(-0.2461)})
	--({\sx*(0.0200)},{\sy*(-0.4480)})
	--({\sx*(0.0300)},{\sy*(-0.6108)})
	--({\sx*(0.0400)},{\sy*(-0.7392)})
	--({\sx*(0.0500)},{\sy*(-0.8374)})
	--({\sx*(0.0600)},{\sy*(-0.9092)})
	--({\sx*(0.0700)},{\sy*(-0.9582)})
	--({\sx*(0.0800)},{\sy*(-0.9875)})
	--({\sx*(0.0900)},{\sy*(-1.0000)})
	--({\sx*(0.1000)},{\sy*(-0.9981)})
	--({\sx*(0.1100)},{\sy*(-0.9843)})
	--({\sx*(0.1200)},{\sy*(-0.9604)})
	--({\sx*(0.1300)},{\sy*(-0.9284)})
	--({\sx*(0.1400)},{\sy*(-0.8899)})
	--({\sx*(0.1500)},{\sy*(-0.8463)})
	--({\sx*(0.1600)},{\sy*(-0.7989)})
	--({\sx*(0.1700)},{\sy*(-0.7487)})
	--({\sx*(0.1800)},{\sy*(-0.6968)})
	--({\sx*(0.1900)},{\sy*(-0.6440)})
	--({\sx*(0.2000)},{\sy*(-0.5910)})
	--({\sx*(0.2100)},{\sy*(-0.5384)})
	--({\sx*(0.2200)},{\sy*(-0.4868)})
	--({\sx*(0.2300)},{\sy*(-0.4365)})
	--({\sx*(0.2400)},{\sy*(-0.3879)})
	--({\sx*(0.2500)},{\sy*(-0.3414)})
	--({\sx*(0.2600)},{\sy*(-0.2971)})
	--({\sx*(0.2700)},{\sy*(-0.2552)})
	--({\sx*(0.2800)},{\sy*(-0.2158)})
	--({\sx*(0.2900)},{\sy*(-0.1791)})
	--({\sx*(0.3000)},{\sy*(-0.1450)})
	--({\sx*(0.3100)},{\sy*(-0.1135)})
	--({\sx*(0.3200)},{\sy*(-0.0847)})
	--({\sx*(0.3300)},{\sy*(-0.0585)})
	--({\sx*(0.3400)},{\sy*(-0.0349)})
	--({\sx*(0.3500)},{\sy*(-0.0137)})
	--({\sx*(0.3600)},{\sy*(0.0051)})
	--({\sx*(0.3700)},{\sy*(0.0217)})
	--({\sx*(0.3800)},{\sy*(0.0361)})
	--({\sx*(0.3900)},{\sy*(0.0484)})
	--({\sx*(0.4000)},{\sy*(0.0588)})
	--({\sx*(0.4100)},{\sy*(0.0675)})
	--({\sx*(0.4200)},{\sy*(0.0744)})
	--({\sx*(0.4300)},{\sy*(0.0799)})
	--({\sx*(0.4400)},{\sy*(0.0839)})
	--({\sx*(0.4500)},{\sy*(0.0866)})
	--({\sx*(0.4600)},{\sy*(0.0882)})
	--({\sx*(0.4700)},{\sy*(0.0887)})
	--({\sx*(0.4800)},{\sy*(0.0882)})
	--({\sx*(0.4900)},{\sy*(0.0870)})
	--({\sx*(0.5000)},{\sy*(0.0850)})
	--({\sx*(0.5100)},{\sy*(0.0824)})
	--({\sx*(0.5200)},{\sy*(0.0792)})
	--({\sx*(0.5300)},{\sy*(0.0756)})
	--({\sx*(0.5400)},{\sy*(0.0717)})
	--({\sx*(0.5500)},{\sy*(0.0674)})
	--({\sx*(0.5600)},{\sy*(0.0630)})
	--({\sx*(0.5700)},{\sy*(0.0583)})
	--({\sx*(0.5800)},{\sy*(0.0536)})
	--({\sx*(0.5900)},{\sy*(0.0488)})
	--({\sx*(0.6000)},{\sy*(0.0441)})
	--({\sx*(0.6100)},{\sy*(0.0394)})
	--({\sx*(0.6200)},{\sy*(0.0348)})
	--({\sx*(0.6300)},{\sy*(0.0303)})
	--({\sx*(0.6400)},{\sy*(0.0259)})
	--({\sx*(0.6500)},{\sy*(0.0217)})
	--({\sx*(0.6600)},{\sy*(0.0177)})
	--({\sx*(0.6700)},{\sy*(0.0139)})
	--({\sx*(0.6800)},{\sy*(0.0104)})
	--({\sx*(0.6900)},{\sy*(0.0071)})
	--({\sx*(0.7000)},{\sy*(0.0040)})
	--({\sx*(0.7100)},{\sy*(0.0011)})
	--({\sx*(0.7200)},{\sy*(-0.0015)})
	--({\sx*(0.7300)},{\sy*(-0.0038)})
	--({\sx*(0.7400)},{\sy*(-0.0059)})
	--({\sx*(0.7500)},{\sy*(-0.0078)})
	--({\sx*(0.7600)},{\sy*(-0.0094)})
	--({\sx*(0.7700)},{\sy*(-0.0108)})
	--({\sx*(0.7800)},{\sy*(-0.0120)})
	--({\sx*(0.7900)},{\sy*(-0.0129)})
	--({\sx*(0.8000)},{\sy*(-0.0137)})
	--({\sx*(0.8100)},{\sy*(-0.0143)})
	--({\sx*(0.8200)},{\sy*(-0.0147)})
	--({\sx*(0.8300)},{\sy*(-0.0149)})
	--({\sx*(0.8400)},{\sy*(-0.0150)})
	--({\sx*(0.8500)},{\sy*(-0.0149)})
	--({\sx*(0.8600)},{\sy*(-0.0148)})
	--({\sx*(0.8700)},{\sy*(-0.0145)})
	--({\sx*(0.8800)},{\sy*(-0.0141)})
	--({\sx*(0.8900)},{\sy*(-0.0136)})
	--({\sx*(0.9000)},{\sy*(-0.0130)})
	--({\sx*(0.9100)},{\sy*(-0.0123)})
	--({\sx*(0.9200)},{\sy*(-0.0116)})
	--({\sx*(0.9300)},{\sy*(-0.0109)})
	--({\sx*(0.9400)},{\sy*(-0.0101)})
	--({\sx*(0.9500)},{\sy*(-0.0093)})
	--({\sx*(0.9600)},{\sy*(-0.0085)})
	--({\sx*(0.9700)},{\sy*(-0.0076)})
	--({\sx*(0.9800)},{\sy*(-0.0068)})
	--({\sx*(0.9900)},{\sy*(-0.0060)})
	--({\sx*(1.0000)},{\sy*(-0.0051)})
	--({\sx*(1.0100)},{\sy*(-0.0043)})
	--({\sx*(1.0200)},{\sy*(-0.0035)})
	--({\sx*(1.0300)},{\sy*(-0.0028)})
	--({\sx*(1.0400)},{\sy*(-0.0020)})
	--({\sx*(1.0500)},{\sy*(-0.0014)})
	--({\sx*(1.0600)},{\sy*(-0.0007)})
	--({\sx*(1.0700)},{\sy*(-0.0001)})
	--({\sx*(1.0800)},{\sy*(0.0005)})
	--({\sx*(1.0900)},{\sy*(0.0010)})
	--({\sx*(1.1000)},{\sy*(0.0015)})
	--({\sx*(1.1100)},{\sy*(0.0019)})
	--({\sx*(1.1200)},{\sy*(0.0023)})
	--({\sx*(1.1300)},{\sy*(0.0027)})
	--({\sx*(1.1400)},{\sy*(0.0030)})
	--({\sx*(1.1500)},{\sy*(0.0032)})
	--({\sx*(1.1600)},{\sy*(0.0034)})
	--({\sx*(1.1700)},{\sy*(0.0036)})
	--({\sx*(1.1800)},{\sy*(0.0037)})
	--({\sx*(1.1900)},{\sy*(0.0038)})
	--({\sx*(1.2000)},{\sy*(0.0039)})
	--({\sx*(1.2100)},{\sy*(0.0039)})
	--({\sx*(1.2200)},{\sy*(0.0038)})
	--({\sx*(1.2300)},{\sy*(0.0038)})
	--({\sx*(1.2400)},{\sy*(0.0037)})
	--({\sx*(1.2500)},{\sy*(0.0036)})
	--({\sx*(1.2600)},{\sy*(0.0035)})
	--({\sx*(1.2700)},{\sy*(0.0033)})
	--({\sx*(1.2800)},{\sy*(0.0032)})
	--({\sx*(1.2900)},{\sy*(0.0030)})
	--({\sx*(1.3000)},{\sy*(0.0028)})
	--({\sx*(1.3100)},{\sy*(0.0026)})
	--({\sx*(1.3200)},{\sy*(0.0024)})
	--({\sx*(1.3300)},{\sy*(0.0021)})
	--({\sx*(1.3400)},{\sy*(0.0019)})
	--({\sx*(1.3500)},{\sy*(0.0017)})
	--({\sx*(1.3600)},{\sy*(0.0015)})
	--({\sx*(1.3700)},{\sy*(0.0012)})
	--({\sx*(1.3800)},{\sy*(0.0010)})
	--({\sx*(1.3900)},{\sy*(0.0008)})
	--({\sx*(1.4000)},{\sy*(0.0006)})
	--({\sx*(1.4100)},{\sy*(0.0004)})
	--({\sx*(1.4200)},{\sy*(0.0002)})
	--({\sx*(1.4300)},{\sy*(-0.0000)})
	--({\sx*(1.4400)},{\sy*(-0.0002)})
	--({\sx*(1.4500)},{\sy*(-0.0004)})
	--({\sx*(1.4600)},{\sy*(-0.0005)})
	--({\sx*(1.4700)},{\sy*(-0.0007)})
	--({\sx*(1.4800)},{\sy*(-0.0008)})
	--({\sx*(1.4900)},{\sy*(-0.0009)})
	--({\sx*(1.5000)},{\sy*(-0.0010)})
	--({\sx*(1.5100)},{\sy*(-0.0011)})
	--({\sx*(1.5200)},{\sy*(-0.0012)})
	--({\sx*(1.5300)},{\sy*(-0.0013)})
	--({\sx*(1.5400)},{\sy*(-0.0013)})
	--({\sx*(1.5500)},{\sy*(-0.0013)})
	--({\sx*(1.5600)},{\sy*(-0.0014)})
	--({\sx*(1.5700)},{\sy*(-0.0014)})
	--({\sx*(1.5800)},{\sy*(-0.0014)})
	--({\sx*(1.5900)},{\sy*(-0.0014)})
	--({\sx*(1.6000)},{\sy*(-0.0013)})
	--({\sx*(1.6100)},{\sy*(-0.0013)})
	--({\sx*(1.6200)},{\sy*(-0.0013)})
	--({\sx*(1.6300)},{\sy*(-0.0012)})
	--({\sx*(1.6400)},{\sy*(-0.0012)})
	--({\sx*(1.6500)},{\sy*(-0.0011)})
	--({\sx*(1.6600)},{\sy*(-0.0010)})
	--({\sx*(1.6700)},{\sy*(-0.0010)})
	--({\sx*(1.6800)},{\sy*(-0.0009)})
	--({\sx*(1.6900)},{\sy*(-0.0008)})
	--({\sx*(1.7000)},{\sy*(-0.0007)})
	--({\sx*(1.7100)},{\sy*(-0.0006)})
	--({\sx*(1.7200)},{\sy*(-0.0006)})
	--({\sx*(1.7300)},{\sy*(-0.0005)})
	--({\sx*(1.7400)},{\sy*(-0.0004)})
	--({\sx*(1.7500)},{\sy*(-0.0003)})
	--({\sx*(1.7600)},{\sy*(-0.0002)})
	--({\sx*(1.7700)},{\sy*(-0.0001)})
	--({\sx*(1.7800)},{\sy*(-0.0000)})
	--({\sx*(1.7900)},{\sy*(0.0000)})
	--({\sx*(1.8000)},{\sy*(0.0001)})
	--({\sx*(1.8100)},{\sy*(0.0002)})
	--({\sx*(1.8200)},{\sy*(0.0002)})
	--({\sx*(1.8300)},{\sy*(0.0003)})
	--({\sx*(1.8400)},{\sy*(0.0004)})
	--({\sx*(1.8500)},{\sy*(0.0004)})
	--({\sx*(1.8600)},{\sy*(0.0005)})
	--({\sx*(1.8700)},{\sy*(0.0005)})
	--({\sx*(1.8800)},{\sy*(0.0005)})
	--({\sx*(1.8900)},{\sy*(0.0006)})
	--({\sx*(1.9000)},{\sy*(0.0006)})
	--({\sx*(1.9100)},{\sy*(0.0006)})
	--({\sx*(1.9200)},{\sy*(0.0006)})
	--({\sx*(1.9300)},{\sy*(0.0006)})
	--({\sx*(1.9400)},{\sy*(0.0006)})
	--({\sx*(1.9500)},{\sy*(0.0006)})
	--({\sx*(1.9600)},{\sy*(0.0006)})
	--({\sx*(1.9700)},{\sy*(0.0006)})
	--({\sx*(1.9800)},{\sy*(0.0006)})
	--({\sx*(1.9900)},{\sy*(0.0006)})
	--({\sx*(2.0000)},{\sy*(0.0006)})
	--({\sx*(2.0100)},{\sy*(0.0005)})
	--({\sx*(2.0200)},{\sy*(0.0005)})
	--({\sx*(2.0300)},{\sy*(0.0005)})
	--({\sx*(2.0400)},{\sy*(0.0004)})
	--({\sx*(2.0500)},{\sy*(0.0004)})
	--({\sx*(2.0600)},{\sy*(0.0004)})
	--({\sx*(2.0700)},{\sy*(0.0003)})
	--({\sx*(2.0800)},{\sy*(0.0003)})
	--({\sx*(2.0900)},{\sy*(0.0002)})
	--({\sx*(2.1000)},{\sy*(0.0002)})
	--({\sx*(2.1100)},{\sy*(0.0001)})
	--({\sx*(2.1200)},{\sy*(0.0001)})
	--({\sx*(2.1300)},{\sy*(0.0001)})
	--({\sx*(2.1400)},{\sy*(0.0000)})
	--({\sx*(2.1500)},{\sy*(-0.0000)})
	--({\sx*(2.1600)},{\sy*(-0.0001)})
	--({\sx*(2.1700)},{\sy*(-0.0001)})
	--({\sx*(2.1800)},{\sy*(-0.0001)})
	--({\sx*(2.1900)},{\sy*(-0.0002)})
	--({\sx*(2.2000)},{\sy*(-0.0002)})
	--({\sx*(2.2100)},{\sy*(-0.0002)})
	--({\sx*(2.2200)},{\sy*(-0.0003)})
	--({\sx*(2.2300)},{\sy*(-0.0003)})
	--({\sx*(2.2400)},{\sy*(-0.0003)})
	--({\sx*(2.2500)},{\sy*(-0.0003)})
	--({\sx*(2.2600)},{\sy*(-0.0004)})
	--({\sx*(2.2700)},{\sy*(-0.0004)})
	--({\sx*(2.2800)},{\sy*(-0.0004)})
	--({\sx*(2.2900)},{\sy*(-0.0004)})
	--({\sx*(2.3000)},{\sy*(-0.0004)})
	--({\sx*(2.3100)},{\sy*(-0.0004)})
	--({\sx*(2.3200)},{\sy*(-0.0004)})
	--({\sx*(2.3300)},{\sy*(-0.0004)})
	--({\sx*(2.3400)},{\sy*(-0.0004)})
	--({\sx*(2.3500)},{\sy*(-0.0004)})
	--({\sx*(2.3600)},{\sy*(-0.0003)})
	--({\sx*(2.3700)},{\sy*(-0.0003)})
	--({\sx*(2.3800)},{\sy*(-0.0003)})
	--({\sx*(2.3900)},{\sy*(-0.0003)})
	--({\sx*(2.4000)},{\sy*(-0.0003)})
	--({\sx*(2.4100)},{\sy*(-0.0002)})
	--({\sx*(2.4200)},{\sy*(-0.0002)})
	--({\sx*(2.4300)},{\sy*(-0.0002)})
	--({\sx*(2.4400)},{\sy*(-0.0002)})
	--({\sx*(2.4500)},{\sy*(-0.0001)})
	--({\sx*(2.4600)},{\sy*(-0.0001)})
	--({\sx*(2.4700)},{\sy*(-0.0001)})
	--({\sx*(2.4800)},{\sy*(-0.0001)})
	--({\sx*(2.4900)},{\sy*(-0.0000)})
	--({\sx*(2.5000)},{\sy*(0.0000)})
	--({\sx*(2.5100)},{\sy*(0.0000)})
	--({\sx*(2.5200)},{\sy*(0.0001)})
	--({\sx*(2.5300)},{\sy*(0.0001)})
	--({\sx*(2.5400)},{\sy*(0.0001)})
	--({\sx*(2.5500)},{\sy*(0.0001)})
	--({\sx*(2.5600)},{\sy*(0.0002)})
	--({\sx*(2.5700)},{\sy*(0.0002)})
	--({\sx*(2.5800)},{\sy*(0.0002)})
	--({\sx*(2.5900)},{\sy*(0.0002)})
	--({\sx*(2.6000)},{\sy*(0.0002)})
	--({\sx*(2.6100)},{\sy*(0.0002)})
	--({\sx*(2.6200)},{\sy*(0.0003)})
	--({\sx*(2.6300)},{\sy*(0.0003)})
	--({\sx*(2.6400)},{\sy*(0.0003)})
	--({\sx*(2.6500)},{\sy*(0.0003)})
	--({\sx*(2.6600)},{\sy*(0.0003)})
	--({\sx*(2.6700)},{\sy*(0.0003)})
	--({\sx*(2.6800)},{\sy*(0.0003)})
	--({\sx*(2.6900)},{\sy*(0.0003)})
	--({\sx*(2.7000)},{\sy*(0.0003)})
	--({\sx*(2.7100)},{\sy*(0.0003)})
	--({\sx*(2.7200)},{\sy*(0.0003)})
	--({\sx*(2.7300)},{\sy*(0.0002)})
	--({\sx*(2.7400)},{\sy*(0.0002)})
	--({\sx*(2.7500)},{\sy*(0.0002)})
	--({\sx*(2.7600)},{\sy*(0.0002)})
	--({\sx*(2.7700)},{\sy*(0.0002)})
	--({\sx*(2.7800)},{\sy*(0.0002)})
	--({\sx*(2.7900)},{\sy*(0.0002)})
	--({\sx*(2.8000)},{\sy*(0.0001)})
	--({\sx*(2.8100)},{\sy*(0.0001)})
	--({\sx*(2.8200)},{\sy*(0.0001)})
	--({\sx*(2.8300)},{\sy*(0.0001)})
	--({\sx*(2.8400)},{\sy*(0.0000)})
	--({\sx*(2.8500)},{\sy*(0.0000)})
	--({\sx*(2.8600)},{\sy*(-0.0000)})
	--({\sx*(2.8700)},{\sy*(-0.0000)})
	--({\sx*(2.8800)},{\sy*(-0.0001)})
	--({\sx*(2.8900)},{\sy*(-0.0001)})
	--({\sx*(2.9000)},{\sy*(-0.0001)})
	--({\sx*(2.9100)},{\sy*(-0.0001)})
	--({\sx*(2.9200)},{\sy*(-0.0001)})
	--({\sx*(2.9300)},{\sy*(-0.0002)})
	--({\sx*(2.9400)},{\sy*(-0.0002)})
	--({\sx*(2.9500)},{\sy*(-0.0002)})
	--({\sx*(2.9600)},{\sy*(-0.0002)})
	--({\sx*(2.9700)},{\sy*(-0.0002)})
	--({\sx*(2.9800)},{\sy*(-0.0002)})
	--({\sx*(2.9900)},{\sy*(-0.0002)})
	--({\sx*(3.0000)},{\sy*(-0.0002)})
	--({\sx*(3.0100)},{\sy*(-0.0002)})
	--({\sx*(3.0200)},{\sy*(-0.0003)})
	--({\sx*(3.0300)},{\sy*(-0.0003)})
	--({\sx*(3.0400)},{\sy*(-0.0003)})
	--({\sx*(3.0500)},{\sy*(-0.0003)})
	--({\sx*(3.0600)},{\sy*(-0.0002)})
	--({\sx*(3.0700)},{\sy*(-0.0002)})
	--({\sx*(3.0800)},{\sy*(-0.0002)})
	--({\sx*(3.0900)},{\sy*(-0.0002)})
	--({\sx*(3.1000)},{\sy*(-0.0002)})
	--({\sx*(3.1100)},{\sy*(-0.0002)})
	--({\sx*(3.1200)},{\sy*(-0.0002)})
	--({\sx*(3.1300)},{\sy*(-0.0002)})
	--({\sx*(3.1400)},{\sy*(-0.0002)})
	--({\sx*(3.1500)},{\sy*(-0.0001)})
	--({\sx*(3.1600)},{\sy*(-0.0001)})
	--({\sx*(3.1700)},{\sy*(-0.0001)})
	--({\sx*(3.1800)},{\sy*(-0.0001)})
	--({\sx*(3.1900)},{\sy*(-0.0001)})
	--({\sx*(3.2000)},{\sy*(-0.0000)})
	--({\sx*(3.2100)},{\sy*(-0.0000)})
	--({\sx*(3.2200)},{\sy*(0.0000)})
	--({\sx*(3.2300)},{\sy*(0.0000)})
	--({\sx*(3.2400)},{\sy*(0.0001)})
	--({\sx*(3.2500)},{\sy*(0.0001)})
	--({\sx*(3.2600)},{\sy*(0.0001)})
	--({\sx*(3.2700)},{\sy*(0.0001)})
	--({\sx*(3.2800)},{\sy*(0.0001)})
	--({\sx*(3.2900)},{\sy*(0.0002)})
	--({\sx*(3.3000)},{\sy*(0.0002)})
	--({\sx*(3.3100)},{\sy*(0.0002)})
	--({\sx*(3.3200)},{\sy*(0.0002)})
	--({\sx*(3.3300)},{\sy*(0.0002)})
	--({\sx*(3.3400)},{\sy*(0.0002)})
	--({\sx*(3.3500)},{\sy*(0.0002)})
	--({\sx*(3.3600)},{\sy*(0.0002)})
	--({\sx*(3.3700)},{\sy*(0.0003)})
	--({\sx*(3.3800)},{\sy*(0.0003)})
	--({\sx*(3.3900)},{\sy*(0.0003)})
	--({\sx*(3.4000)},{\sy*(0.0003)})
	--({\sx*(3.4100)},{\sy*(0.0003)})
	--({\sx*(3.4200)},{\sy*(0.0003)})
	--({\sx*(3.4300)},{\sy*(0.0002)})
	--({\sx*(3.4400)},{\sy*(0.0002)})
	--({\sx*(3.4500)},{\sy*(0.0002)})
	--({\sx*(3.4600)},{\sy*(0.0002)})
	--({\sx*(3.4700)},{\sy*(0.0002)})
	--({\sx*(3.4800)},{\sy*(0.0002)})
	--({\sx*(3.4900)},{\sy*(0.0002)})
	--({\sx*(3.5000)},{\sy*(0.0002)})
	--({\sx*(3.5100)},{\sy*(0.0001)})
	--({\sx*(3.5200)},{\sy*(0.0001)})
	--({\sx*(3.5300)},{\sy*(0.0001)})
	--({\sx*(3.5400)},{\sy*(0.0001)})
	--({\sx*(3.5500)},{\sy*(0.0000)})
	--({\sx*(3.5600)},{\sy*(0.0000)})
	--({\sx*(3.5700)},{\sy*(0.0000)})
	--({\sx*(3.5800)},{\sy*(-0.0000)})
	--({\sx*(3.5900)},{\sy*(-0.0000)})
	--({\sx*(3.6000)},{\sy*(-0.0001)})
	--({\sx*(3.6100)},{\sy*(-0.0001)})
	--({\sx*(3.6200)},{\sy*(-0.0001)})
	--({\sx*(3.6300)},{\sy*(-0.0001)})
	--({\sx*(3.6400)},{\sy*(-0.0001)})
	--({\sx*(3.6500)},{\sy*(-0.0002)})
	--({\sx*(3.6600)},{\sy*(-0.0002)})
	--({\sx*(3.6700)},{\sy*(-0.0002)})
	--({\sx*(3.6800)},{\sy*(-0.0002)})
	--({\sx*(3.6900)},{\sy*(-0.0002)})
	--({\sx*(3.7000)},{\sy*(-0.0002)})
	--({\sx*(3.7100)},{\sy*(-0.0002)})
	--({\sx*(3.7200)},{\sy*(-0.0002)})
	--({\sx*(3.7300)},{\sy*(-0.0002)})
	--({\sx*(3.7400)},{\sy*(-0.0002)})
	--({\sx*(3.7500)},{\sy*(-0.0002)})
	--({\sx*(3.7600)},{\sy*(-0.0002)})
	--({\sx*(3.7700)},{\sy*(-0.0002)})
	--({\sx*(3.7800)},{\sy*(-0.0002)})
	--({\sx*(3.7900)},{\sy*(-0.0002)})
	--({\sx*(3.8000)},{\sy*(-0.0002)})
	--({\sx*(3.8100)},{\sy*(-0.0002)})
	--({\sx*(3.8200)},{\sy*(-0.0002)})
	--({\sx*(3.8300)},{\sy*(-0.0002)})
	--({\sx*(3.8400)},{\sy*(-0.0002)})
	--({\sx*(3.8500)},{\sy*(-0.0001)})
	--({\sx*(3.8600)},{\sy*(-0.0001)})
	--({\sx*(3.8700)},{\sy*(-0.0001)})
	--({\sx*(3.8800)},{\sy*(-0.0001)})
	--({\sx*(3.8900)},{\sy*(-0.0001)})
	--({\sx*(3.9000)},{\sy*(-0.0000)})
	--({\sx*(3.9100)},{\sy*(-0.0000)})
	--({\sx*(3.9200)},{\sy*(-0.0000)})
	--({\sx*(3.9300)},{\sy*(0.0000)})
	--({\sx*(3.9400)},{\sy*(0.0000)})
	--({\sx*(3.9500)},{\sy*(0.0000)})
	--({\sx*(3.9600)},{\sy*(0.0000)})
	--({\sx*(3.9700)},{\sy*(0.0000)})
	--({\sx*(3.9800)},{\sy*(0.0000)})
	--({\sx*(3.9900)},{\sy*(0.0000)})
	--({\sx*(4.0000)},{\sy*(0.0000)})
	--({\sx*(4.0100)},{\sy*(0.0000)})
	--({\sx*(4.0200)},{\sy*(0.0000)})
	--({\sx*(4.0300)},{\sy*(0.0000)})
	--({\sx*(4.0400)},{\sy*(-0.0000)})
	--({\sx*(4.0500)},{\sy*(-0.0000)})
	--({\sx*(4.0600)},{\sy*(-0.0000)})
	--({\sx*(4.0700)},{\sy*(-0.0001)})
	--({\sx*(4.0800)},{\sy*(-0.0001)})
	--({\sx*(4.0900)},{\sy*(-0.0001)})
	--({\sx*(4.1000)},{\sy*(-0.0002)})
	--({\sx*(4.1100)},{\sy*(-0.0002)})
	--({\sx*(4.1200)},{\sy*(-0.0002)})
	--({\sx*(4.1300)},{\sy*(-0.0003)})
	--({\sx*(4.1400)},{\sy*(-0.0003)})
	--({\sx*(4.1500)},{\sy*(-0.0003)})
	--({\sx*(4.1600)},{\sy*(-0.0004)})
	--({\sx*(4.1700)},{\sy*(-0.0004)})
	--({\sx*(4.1800)},{\sy*(-0.0004)})
	--({\sx*(4.1900)},{\sy*(-0.0004)})
	--({\sx*(4.2000)},{\sy*(-0.0004)})
	--({\sx*(4.2100)},{\sy*(-0.0004)})
	--({\sx*(4.2200)},{\sy*(-0.0004)})
	--({\sx*(4.2300)},{\sy*(-0.0004)})
	--({\sx*(4.2400)},{\sy*(-0.0003)})
	--({\sx*(4.2500)},{\sy*(-0.0003)})
	--({\sx*(4.2600)},{\sy*(-0.0002)})
	--({\sx*(4.2700)},{\sy*(-0.0002)})
	--({\sx*(4.2800)},{\sy*(-0.0001)})
	--({\sx*(4.2900)},{\sy*(0.0001)})
	--({\sx*(4.3000)},{\sy*(0.0002)})
	--({\sx*(4.3100)},{\sy*(0.0003)})
	--({\sx*(4.3200)},{\sy*(0.0005)})
	--({\sx*(4.3300)},{\sy*(0.0007)})
	--({\sx*(4.3400)},{\sy*(0.0009)})
	--({\sx*(4.3500)},{\sy*(0.0012)})
	--({\sx*(4.3600)},{\sy*(0.0014)})
	--({\sx*(4.3700)},{\sy*(0.0017)})
	--({\sx*(4.3800)},{\sy*(0.0020)})
	--({\sx*(4.3900)},{\sy*(0.0023)})
	--({\sx*(4.4000)},{\sy*(0.0027)})
	--({\sx*(4.4100)},{\sy*(0.0030)})
	--({\sx*(4.4200)},{\sy*(0.0034)})
	--({\sx*(4.4300)},{\sy*(0.0038)})
	--({\sx*(4.4400)},{\sy*(0.0041)})
	--({\sx*(4.4500)},{\sy*(0.0045)})
	--({\sx*(4.4600)},{\sy*(0.0049)})
	--({\sx*(4.4700)},{\sy*(0.0053)})
	--({\sx*(4.4800)},{\sy*(0.0056)})
	--({\sx*(4.4900)},{\sy*(0.0060)})
	--({\sx*(4.5000)},{\sy*(0.0062)})
	--({\sx*(4.5100)},{\sy*(0.0065)})
	--({\sx*(4.5200)},{\sy*(0.0067)})
	--({\sx*(4.5300)},{\sy*(0.0068)})
	--({\sx*(4.5400)},{\sy*(0.0069)})
	--({\sx*(4.5500)},{\sy*(0.0069)})
	--({\sx*(4.5600)},{\sy*(0.0068)})
	--({\sx*(4.5700)},{\sy*(0.0065)})
	--({\sx*(4.5800)},{\sy*(0.0062)})
	--({\sx*(4.5900)},{\sy*(0.0057)})
	--({\sx*(4.6000)},{\sy*(0.0050)})
	--({\sx*(4.6100)},{\sy*(0.0042)})
	--({\sx*(4.6200)},{\sy*(0.0032)})
	--({\sx*(4.6300)},{\sy*(0.0019)})
	--({\sx*(4.6400)},{\sy*(0.0005)})
	--({\sx*(4.6500)},{\sy*(-0.0012)})
	--({\sx*(4.6600)},{\sy*(-0.0032)})
	--({\sx*(4.6700)},{\sy*(-0.0054)})
	--({\sx*(4.6800)},{\sy*(-0.0079)})
	--({\sx*(4.6900)},{\sy*(-0.0108)})
	--({\sx*(4.7000)},{\sy*(-0.0139)})
	--({\sx*(4.7100)},{\sy*(-0.0173)})
	--({\sx*(4.7200)},{\sy*(-0.0211)})
	--({\sx*(4.7300)},{\sy*(-0.0252)})
	--({\sx*(4.7400)},{\sy*(-0.0297)})
	--({\sx*(4.7500)},{\sy*(-0.0344)})
	--({\sx*(4.7600)},{\sy*(-0.0395)})
	--({\sx*(4.7700)},{\sy*(-0.0449)})
	--({\sx*(4.7800)},{\sy*(-0.0505)})
	--({\sx*(4.7900)},{\sy*(-0.0563)})
	--({\sx*(4.8000)},{\sy*(-0.0624)})
	--({\sx*(4.8100)},{\sy*(-0.0686)})
	--({\sx*(4.8200)},{\sy*(-0.0749)})
	--({\sx*(4.8300)},{\sy*(-0.0811)})
	--({\sx*(4.8400)},{\sy*(-0.0873)})
	--({\sx*(4.8500)},{\sy*(-0.0932)})
	--({\sx*(4.8600)},{\sy*(-0.0988)})
	--({\sx*(4.8700)},{\sy*(-0.1039)})
	--({\sx*(4.8800)},{\sy*(-0.1083)})
	--({\sx*(4.8900)},{\sy*(-0.1119)})
	--({\sx*(4.9000)},{\sy*(-0.1143)})
	--({\sx*(4.9100)},{\sy*(-0.1154)})
	--({\sx*(4.9200)},{\sy*(-0.1148)})
	--({\sx*(4.9300)},{\sy*(-0.1122)})
	--({\sx*(4.9400)},{\sy*(-0.1072)})
	--({\sx*(4.9500)},{\sy*(-0.0994)})
	--({\sx*(4.9600)},{\sy*(-0.0884)})
	--({\sx*(4.9700)},{\sy*(-0.0735)})
	--({\sx*(4.9800)},{\sy*(-0.0543)})
	--({\sx*(4.9900)},{\sy*(-0.0300)})
	--({\sx*(5.0000)},{\sy*(0.0000)});
}
\def\relfehlerg{
\draw[color=blue,line width=1.4pt,line join=round] ({\sx*(0.000)},{\sy*(0.0000)})
	--({\sx*(0.0100)},{\sy*(-0.0000)})
	--({\sx*(0.0200)},{\sy*(-0.0000)})
	--({\sx*(0.0300)},{\sy*(-0.0000)})
	--({\sx*(0.0400)},{\sy*(-0.0000)})
	--({\sx*(0.0500)},{\sy*(-0.0000)})
	--({\sx*(0.0600)},{\sy*(-0.0001)})
	--({\sx*(0.0700)},{\sy*(-0.0001)})
	--({\sx*(0.0800)},{\sy*(-0.0001)})
	--({\sx*(0.0900)},{\sy*(-0.0001)})
	--({\sx*(0.1000)},{\sy*(-0.0001)})
	--({\sx*(0.1100)},{\sy*(-0.0001)})
	--({\sx*(0.1200)},{\sy*(-0.0001)})
	--({\sx*(0.1300)},{\sy*(-0.0001)})
	--({\sx*(0.1400)},{\sy*(-0.0001)})
	--({\sx*(0.1500)},{\sy*(-0.0000)})
	--({\sx*(0.1600)},{\sy*(-0.0000)})
	--({\sx*(0.1700)},{\sy*(-0.0000)})
	--({\sx*(0.1800)},{\sy*(-0.0000)})
	--({\sx*(0.1900)},{\sy*(-0.0000)})
	--({\sx*(0.2000)},{\sy*(-0.0000)})
	--({\sx*(0.2100)},{\sy*(-0.0000)})
	--({\sx*(0.2200)},{\sy*(-0.0000)})
	--({\sx*(0.2300)},{\sy*(-0.0000)})
	--({\sx*(0.2400)},{\sy*(-0.0000)})
	--({\sx*(0.2500)},{\sy*(-0.0000)})
	--({\sx*(0.2600)},{\sy*(-0.0000)})
	--({\sx*(0.2700)},{\sy*(-0.0000)})
	--({\sx*(0.2800)},{\sy*(-0.0000)})
	--({\sx*(0.2900)},{\sy*(-0.0000)})
	--({\sx*(0.3000)},{\sy*(-0.0000)})
	--({\sx*(0.3100)},{\sy*(-0.0000)})
	--({\sx*(0.3200)},{\sy*(-0.0000)})
	--({\sx*(0.3300)},{\sy*(-0.0000)})
	--({\sx*(0.3400)},{\sy*(-0.0000)})
	--({\sx*(0.3500)},{\sy*(-0.0000)})
	--({\sx*(0.3600)},{\sy*(0.0000)})
	--({\sx*(0.3700)},{\sy*(0.0000)})
	--({\sx*(0.3800)},{\sy*(0.0000)})
	--({\sx*(0.3900)},{\sy*(0.0000)})
	--({\sx*(0.4000)},{\sy*(0.0000)})
	--({\sx*(0.4100)},{\sy*(0.0000)})
	--({\sx*(0.4200)},{\sy*(0.0000)})
	--({\sx*(0.4300)},{\sy*(0.0000)})
	--({\sx*(0.4400)},{\sy*(0.0000)})
	--({\sx*(0.4500)},{\sy*(0.0000)})
	--({\sx*(0.4600)},{\sy*(0.0000)})
	--({\sx*(0.4700)},{\sy*(0.0000)})
	--({\sx*(0.4800)},{\sy*(0.0000)})
	--({\sx*(0.4900)},{\sy*(0.0000)})
	--({\sx*(0.5000)},{\sy*(0.0000)})
	--({\sx*(0.5100)},{\sy*(0.0000)})
	--({\sx*(0.5200)},{\sy*(0.0000)})
	--({\sx*(0.5300)},{\sy*(0.0000)})
	--({\sx*(0.5400)},{\sy*(0.0000)})
	--({\sx*(0.5500)},{\sy*(0.0000)})
	--({\sx*(0.5600)},{\sy*(0.0000)})
	--({\sx*(0.5700)},{\sy*(0.0000)})
	--({\sx*(0.5800)},{\sy*(0.0000)})
	--({\sx*(0.5900)},{\sy*(0.0000)})
	--({\sx*(0.6000)},{\sy*(0.0000)})
	--({\sx*(0.6100)},{\sy*(0.0000)})
	--({\sx*(0.6200)},{\sy*(0.0000)})
	--({\sx*(0.6300)},{\sy*(0.0000)})
	--({\sx*(0.6400)},{\sy*(0.0000)})
	--({\sx*(0.6500)},{\sy*(0.0000)})
	--({\sx*(0.6600)},{\sy*(0.0000)})
	--({\sx*(0.6700)},{\sy*(0.0000)})
	--({\sx*(0.6800)},{\sy*(0.0000)})
	--({\sx*(0.6900)},{\sy*(0.0000)})
	--({\sx*(0.7000)},{\sy*(0.0000)})
	--({\sx*(0.7100)},{\sy*(0.0000)})
	--({\sx*(0.7200)},{\sy*(-0.0000)})
	--({\sx*(0.7300)},{\sy*(-0.0000)})
	--({\sx*(0.7400)},{\sy*(-0.0000)})
	--({\sx*(0.7500)},{\sy*(-0.0000)})
	--({\sx*(0.7600)},{\sy*(-0.0000)})
	--({\sx*(0.7700)},{\sy*(-0.0000)})
	--({\sx*(0.7800)},{\sy*(-0.0000)})
	--({\sx*(0.7900)},{\sy*(-0.0000)})
	--({\sx*(0.8000)},{\sy*(-0.0000)})
	--({\sx*(0.8100)},{\sy*(-0.0000)})
	--({\sx*(0.8200)},{\sy*(-0.0000)})
	--({\sx*(0.8300)},{\sy*(-0.0000)})
	--({\sx*(0.8400)},{\sy*(-0.0000)})
	--({\sx*(0.8500)},{\sy*(-0.0000)})
	--({\sx*(0.8600)},{\sy*(-0.0000)})
	--({\sx*(0.8700)},{\sy*(-0.0000)})
	--({\sx*(0.8800)},{\sy*(-0.0000)})
	--({\sx*(0.8900)},{\sy*(-0.0000)})
	--({\sx*(0.9000)},{\sy*(-0.0000)})
	--({\sx*(0.9100)},{\sy*(-0.0000)})
	--({\sx*(0.9200)},{\sy*(-0.0000)})
	--({\sx*(0.9300)},{\sy*(-0.0000)})
	--({\sx*(0.9400)},{\sy*(-0.0000)})
	--({\sx*(0.9500)},{\sy*(-0.0000)})
	--({\sx*(0.9600)},{\sy*(-0.0000)})
	--({\sx*(0.9700)},{\sy*(-0.0000)})
	--({\sx*(0.9800)},{\sy*(-0.0000)})
	--({\sx*(0.9900)},{\sy*(-0.0000)})
	--({\sx*(1.0000)},{\sy*(-0.0000)})
	--({\sx*(1.0100)},{\sy*(-0.0000)})
	--({\sx*(1.0200)},{\sy*(-0.0000)})
	--({\sx*(1.0300)},{\sy*(-0.0000)})
	--({\sx*(1.0400)},{\sy*(-0.0000)})
	--({\sx*(1.0500)},{\sy*(-0.0000)})
	--({\sx*(1.0600)},{\sy*(-0.0000)})
	--({\sx*(1.0700)},{\sy*(-0.0000)})
	--({\sx*(1.0800)},{\sy*(0.0000)})
	--({\sx*(1.0900)},{\sy*(0.0000)})
	--({\sx*(1.1000)},{\sy*(0.0000)})
	--({\sx*(1.1100)},{\sy*(0.0000)})
	--({\sx*(1.1200)},{\sy*(0.0000)})
	--({\sx*(1.1300)},{\sy*(0.0000)})
	--({\sx*(1.1400)},{\sy*(0.0000)})
	--({\sx*(1.1500)},{\sy*(0.0000)})
	--({\sx*(1.1600)},{\sy*(0.0000)})
	--({\sx*(1.1700)},{\sy*(0.0000)})
	--({\sx*(1.1800)},{\sy*(0.0000)})
	--({\sx*(1.1900)},{\sy*(0.0000)})
	--({\sx*(1.2000)},{\sy*(0.0000)})
	--({\sx*(1.2100)},{\sy*(0.0000)})
	--({\sx*(1.2200)},{\sy*(0.0000)})
	--({\sx*(1.2300)},{\sy*(0.0000)})
	--({\sx*(1.2400)},{\sy*(0.0000)})
	--({\sx*(1.2500)},{\sy*(0.0000)})
	--({\sx*(1.2600)},{\sy*(0.0000)})
	--({\sx*(1.2700)},{\sy*(0.0000)})
	--({\sx*(1.2800)},{\sy*(0.0000)})
	--({\sx*(1.2900)},{\sy*(0.0000)})
	--({\sx*(1.3000)},{\sy*(0.0000)})
	--({\sx*(1.3100)},{\sy*(0.0000)})
	--({\sx*(1.3200)},{\sy*(0.0000)})
	--({\sx*(1.3300)},{\sy*(0.0000)})
	--({\sx*(1.3400)},{\sy*(0.0000)})
	--({\sx*(1.3500)},{\sy*(0.0000)})
	--({\sx*(1.3600)},{\sy*(0.0000)})
	--({\sx*(1.3700)},{\sy*(0.0000)})
	--({\sx*(1.3800)},{\sy*(0.0000)})
	--({\sx*(1.3900)},{\sy*(0.0000)})
	--({\sx*(1.4000)},{\sy*(0.0000)})
	--({\sx*(1.4100)},{\sy*(0.0000)})
	--({\sx*(1.4200)},{\sy*(0.0000)})
	--({\sx*(1.4300)},{\sy*(-0.0000)})
	--({\sx*(1.4400)},{\sy*(-0.0000)})
	--({\sx*(1.4500)},{\sy*(-0.0000)})
	--({\sx*(1.4600)},{\sy*(-0.0000)})
	--({\sx*(1.4700)},{\sy*(-0.0000)})
	--({\sx*(1.4800)},{\sy*(-0.0000)})
	--({\sx*(1.4900)},{\sy*(-0.0000)})
	--({\sx*(1.5000)},{\sy*(-0.0000)})
	--({\sx*(1.5100)},{\sy*(-0.0000)})
	--({\sx*(1.5200)},{\sy*(-0.0000)})
	--({\sx*(1.5300)},{\sy*(-0.0000)})
	--({\sx*(1.5400)},{\sy*(-0.0000)})
	--({\sx*(1.5500)},{\sy*(-0.0000)})
	--({\sx*(1.5600)},{\sy*(-0.0000)})
	--({\sx*(1.5700)},{\sy*(-0.0000)})
	--({\sx*(1.5800)},{\sy*(-0.0000)})
	--({\sx*(1.5900)},{\sy*(-0.0000)})
	--({\sx*(1.6000)},{\sy*(-0.0000)})
	--({\sx*(1.6100)},{\sy*(-0.0000)})
	--({\sx*(1.6200)},{\sy*(-0.0000)})
	--({\sx*(1.6300)},{\sy*(-0.0000)})
	--({\sx*(1.6400)},{\sy*(-0.0000)})
	--({\sx*(1.6500)},{\sy*(-0.0000)})
	--({\sx*(1.6600)},{\sy*(-0.0000)})
	--({\sx*(1.6700)},{\sy*(-0.0000)})
	--({\sx*(1.6800)},{\sy*(-0.0000)})
	--({\sx*(1.6900)},{\sy*(-0.0000)})
	--({\sx*(1.7000)},{\sy*(-0.0000)})
	--({\sx*(1.7100)},{\sy*(-0.0000)})
	--({\sx*(1.7200)},{\sy*(-0.0000)})
	--({\sx*(1.7300)},{\sy*(-0.0000)})
	--({\sx*(1.7400)},{\sy*(-0.0000)})
	--({\sx*(1.7500)},{\sy*(-0.0000)})
	--({\sx*(1.7600)},{\sy*(-0.0000)})
	--({\sx*(1.7700)},{\sy*(-0.0000)})
	--({\sx*(1.7800)},{\sy*(-0.0000)})
	--({\sx*(1.7900)},{\sy*(0.0000)})
	--({\sx*(1.8000)},{\sy*(0.0000)})
	--({\sx*(1.8100)},{\sy*(0.0000)})
	--({\sx*(1.8200)},{\sy*(0.0000)})
	--({\sx*(1.8300)},{\sy*(0.0000)})
	--({\sx*(1.8400)},{\sy*(0.0000)})
	--({\sx*(1.8500)},{\sy*(0.0000)})
	--({\sx*(1.8600)},{\sy*(0.0000)})
	--({\sx*(1.8700)},{\sy*(0.0000)})
	--({\sx*(1.8800)},{\sy*(0.0000)})
	--({\sx*(1.8900)},{\sy*(0.0000)})
	--({\sx*(1.9000)},{\sy*(0.0000)})
	--({\sx*(1.9100)},{\sy*(0.0000)})
	--({\sx*(1.9200)},{\sy*(0.0000)})
	--({\sx*(1.9300)},{\sy*(0.0000)})
	--({\sx*(1.9400)},{\sy*(0.0000)})
	--({\sx*(1.9500)},{\sy*(0.0000)})
	--({\sx*(1.9600)},{\sy*(0.0000)})
	--({\sx*(1.9700)},{\sy*(0.0000)})
	--({\sx*(1.9800)},{\sy*(0.0000)})
	--({\sx*(1.9900)},{\sy*(0.0000)})
	--({\sx*(2.0000)},{\sy*(0.0000)})
	--({\sx*(2.0100)},{\sy*(0.0000)})
	--({\sx*(2.0200)},{\sy*(0.0000)})
	--({\sx*(2.0300)},{\sy*(0.0000)})
	--({\sx*(2.0400)},{\sy*(0.0000)})
	--({\sx*(2.0500)},{\sy*(0.0000)})
	--({\sx*(2.0600)},{\sy*(0.0000)})
	--({\sx*(2.0700)},{\sy*(0.0000)})
	--({\sx*(2.0800)},{\sy*(0.0000)})
	--({\sx*(2.0900)},{\sy*(0.0000)})
	--({\sx*(2.1000)},{\sy*(0.0000)})
	--({\sx*(2.1100)},{\sy*(0.0000)})
	--({\sx*(2.1200)},{\sy*(0.0000)})
	--({\sx*(2.1300)},{\sy*(0.0000)})
	--({\sx*(2.1400)},{\sy*(0.0000)})
	--({\sx*(2.1500)},{\sy*(-0.0000)})
	--({\sx*(2.1600)},{\sy*(-0.0000)})
	--({\sx*(2.1700)},{\sy*(-0.0000)})
	--({\sx*(2.1800)},{\sy*(-0.0000)})
	--({\sx*(2.1900)},{\sy*(-0.0000)})
	--({\sx*(2.2000)},{\sy*(-0.0000)})
	--({\sx*(2.2100)},{\sy*(-0.0000)})
	--({\sx*(2.2200)},{\sy*(-0.0000)})
	--({\sx*(2.2300)},{\sy*(-0.0000)})
	--({\sx*(2.2400)},{\sy*(-0.0000)})
	--({\sx*(2.2500)},{\sy*(-0.0000)})
	--({\sx*(2.2600)},{\sy*(-0.0000)})
	--({\sx*(2.2700)},{\sy*(-0.0000)})
	--({\sx*(2.2800)},{\sy*(-0.0000)})
	--({\sx*(2.2900)},{\sy*(-0.0000)})
	--({\sx*(2.3000)},{\sy*(-0.0000)})
	--({\sx*(2.3100)},{\sy*(-0.0000)})
	--({\sx*(2.3200)},{\sy*(-0.0000)})
	--({\sx*(2.3300)},{\sy*(-0.0000)})
	--({\sx*(2.3400)},{\sy*(-0.0000)})
	--({\sx*(2.3500)},{\sy*(-0.0000)})
	--({\sx*(2.3600)},{\sy*(-0.0000)})
	--({\sx*(2.3700)},{\sy*(-0.0000)})
	--({\sx*(2.3800)},{\sy*(-0.0000)})
	--({\sx*(2.3900)},{\sy*(-0.0000)})
	--({\sx*(2.4000)},{\sy*(-0.0000)})
	--({\sx*(2.4100)},{\sy*(-0.0000)})
	--({\sx*(2.4200)},{\sy*(-0.0000)})
	--({\sx*(2.4300)},{\sy*(-0.0000)})
	--({\sx*(2.4400)},{\sy*(-0.0000)})
	--({\sx*(2.4500)},{\sy*(-0.0000)})
	--({\sx*(2.4600)},{\sy*(-0.0000)})
	--({\sx*(2.4700)},{\sy*(-0.0000)})
	--({\sx*(2.4800)},{\sy*(-0.0000)})
	--({\sx*(2.4900)},{\sy*(-0.0000)})
	--({\sx*(2.5000)},{\sy*(0.0000)})
	--({\sx*(2.5100)},{\sy*(0.0000)})
	--({\sx*(2.5200)},{\sy*(0.0000)})
	--({\sx*(2.5300)},{\sy*(0.0000)})
	--({\sx*(2.5400)},{\sy*(0.0000)})
	--({\sx*(2.5500)},{\sy*(0.0000)})
	--({\sx*(2.5600)},{\sy*(0.0000)})
	--({\sx*(2.5700)},{\sy*(0.0000)})
	--({\sx*(2.5800)},{\sy*(0.0000)})
	--({\sx*(2.5900)},{\sy*(0.0000)})
	--({\sx*(2.6000)},{\sy*(0.0000)})
	--({\sx*(2.6100)},{\sy*(0.0000)})
	--({\sx*(2.6200)},{\sy*(0.0000)})
	--({\sx*(2.6300)},{\sy*(0.0000)})
	--({\sx*(2.6400)},{\sy*(0.0000)})
	--({\sx*(2.6500)},{\sy*(0.0000)})
	--({\sx*(2.6600)},{\sy*(0.0000)})
	--({\sx*(2.6700)},{\sy*(0.0000)})
	--({\sx*(2.6800)},{\sy*(0.0000)})
	--({\sx*(2.6900)},{\sy*(0.0000)})
	--({\sx*(2.7000)},{\sy*(0.0000)})
	--({\sx*(2.7100)},{\sy*(0.0000)})
	--({\sx*(2.7200)},{\sy*(0.0000)})
	--({\sx*(2.7300)},{\sy*(0.0000)})
	--({\sx*(2.7400)},{\sy*(0.0000)})
	--({\sx*(2.7500)},{\sy*(0.0000)})
	--({\sx*(2.7600)},{\sy*(0.0000)})
	--({\sx*(2.7700)},{\sy*(0.0000)})
	--({\sx*(2.7800)},{\sy*(0.0000)})
	--({\sx*(2.7900)},{\sy*(0.0000)})
	--({\sx*(2.8000)},{\sy*(0.0000)})
	--({\sx*(2.8100)},{\sy*(0.0000)})
	--({\sx*(2.8200)},{\sy*(0.0000)})
	--({\sx*(2.8300)},{\sy*(0.0000)})
	--({\sx*(2.8400)},{\sy*(0.0000)})
	--({\sx*(2.8500)},{\sy*(0.0000)})
	--({\sx*(2.8600)},{\sy*(-0.0000)})
	--({\sx*(2.8700)},{\sy*(-0.0000)})
	--({\sx*(2.8800)},{\sy*(-0.0000)})
	--({\sx*(2.8900)},{\sy*(-0.0000)})
	--({\sx*(2.9000)},{\sy*(-0.0000)})
	--({\sx*(2.9100)},{\sy*(-0.0000)})
	--({\sx*(2.9200)},{\sy*(-0.0000)})
	--({\sx*(2.9300)},{\sy*(-0.0000)})
	--({\sx*(2.9400)},{\sy*(-0.0000)})
	--({\sx*(2.9500)},{\sy*(-0.0000)})
	--({\sx*(2.9600)},{\sy*(-0.0000)})
	--({\sx*(2.9700)},{\sy*(-0.0000)})
	--({\sx*(2.9800)},{\sy*(-0.0000)})
	--({\sx*(2.9900)},{\sy*(-0.0000)})
	--({\sx*(3.0000)},{\sy*(-0.0000)})
	--({\sx*(3.0100)},{\sy*(-0.0000)})
	--({\sx*(3.0200)},{\sy*(-0.0000)})
	--({\sx*(3.0300)},{\sy*(-0.0000)})
	--({\sx*(3.0400)},{\sy*(-0.0000)})
	--({\sx*(3.0500)},{\sy*(-0.0000)})
	--({\sx*(3.0600)},{\sy*(-0.0000)})
	--({\sx*(3.0700)},{\sy*(-0.0000)})
	--({\sx*(3.0800)},{\sy*(-0.0000)})
	--({\sx*(3.0900)},{\sy*(-0.0000)})
	--({\sx*(3.1000)},{\sy*(-0.0000)})
	--({\sx*(3.1100)},{\sy*(-0.0000)})
	--({\sx*(3.1200)},{\sy*(-0.0000)})
	--({\sx*(3.1300)},{\sy*(-0.0000)})
	--({\sx*(3.1400)},{\sy*(-0.0000)})
	--({\sx*(3.1500)},{\sy*(-0.0000)})
	--({\sx*(3.1600)},{\sy*(-0.0000)})
	--({\sx*(3.1700)},{\sy*(-0.0000)})
	--({\sx*(3.1800)},{\sy*(-0.0000)})
	--({\sx*(3.1900)},{\sy*(-0.0000)})
	--({\sx*(3.2000)},{\sy*(-0.0000)})
	--({\sx*(3.2100)},{\sy*(-0.0000)})
	--({\sx*(3.2200)},{\sy*(0.0000)})
	--({\sx*(3.2300)},{\sy*(0.0000)})
	--({\sx*(3.2400)},{\sy*(0.0000)})
	--({\sx*(3.2500)},{\sy*(0.0000)})
	--({\sx*(3.2600)},{\sy*(0.0000)})
	--({\sx*(3.2700)},{\sy*(0.0000)})
	--({\sx*(3.2800)},{\sy*(0.0000)})
	--({\sx*(3.2900)},{\sy*(0.0000)})
	--({\sx*(3.3000)},{\sy*(0.0000)})
	--({\sx*(3.3100)},{\sy*(0.0000)})
	--({\sx*(3.3200)},{\sy*(0.0000)})
	--({\sx*(3.3300)},{\sy*(0.0000)})
	--({\sx*(3.3400)},{\sy*(0.0000)})
	--({\sx*(3.3500)},{\sy*(0.0000)})
	--({\sx*(3.3600)},{\sy*(0.0000)})
	--({\sx*(3.3700)},{\sy*(0.0000)})
	--({\sx*(3.3800)},{\sy*(0.0000)})
	--({\sx*(3.3900)},{\sy*(0.0000)})
	--({\sx*(3.4000)},{\sy*(0.0000)})
	--({\sx*(3.4100)},{\sy*(0.0000)})
	--({\sx*(3.4200)},{\sy*(0.0000)})
	--({\sx*(3.4300)},{\sy*(0.0000)})
	--({\sx*(3.4400)},{\sy*(0.0000)})
	--({\sx*(3.4500)},{\sy*(0.0000)})
	--({\sx*(3.4600)},{\sy*(0.0000)})
	--({\sx*(3.4700)},{\sy*(0.0000)})
	--({\sx*(3.4800)},{\sy*(0.0000)})
	--({\sx*(3.4900)},{\sy*(0.0000)})
	--({\sx*(3.5000)},{\sy*(0.0000)})
	--({\sx*(3.5100)},{\sy*(0.0000)})
	--({\sx*(3.5200)},{\sy*(0.0000)})
	--({\sx*(3.5300)},{\sy*(0.0000)})
	--({\sx*(3.5400)},{\sy*(0.0000)})
	--({\sx*(3.5500)},{\sy*(0.0000)})
	--({\sx*(3.5600)},{\sy*(0.0000)})
	--({\sx*(3.5700)},{\sy*(0.0000)})
	--({\sx*(3.5800)},{\sy*(-0.0000)})
	--({\sx*(3.5900)},{\sy*(-0.0000)})
	--({\sx*(3.6000)},{\sy*(-0.0000)})
	--({\sx*(3.6100)},{\sy*(-0.0000)})
	--({\sx*(3.6200)},{\sy*(-0.0000)})
	--({\sx*(3.6300)},{\sy*(-0.0000)})
	--({\sx*(3.6400)},{\sy*(-0.0000)})
	--({\sx*(3.6500)},{\sy*(-0.0000)})
	--({\sx*(3.6600)},{\sy*(-0.0000)})
	--({\sx*(3.6700)},{\sy*(-0.0000)})
	--({\sx*(3.6800)},{\sy*(-0.0000)})
	--({\sx*(3.6900)},{\sy*(-0.0000)})
	--({\sx*(3.7000)},{\sy*(-0.0000)})
	--({\sx*(3.7100)},{\sy*(-0.0000)})
	--({\sx*(3.7200)},{\sy*(-0.0000)})
	--({\sx*(3.7300)},{\sy*(-0.0000)})
	--({\sx*(3.7400)},{\sy*(-0.0000)})
	--({\sx*(3.7500)},{\sy*(-0.0000)})
	--({\sx*(3.7600)},{\sy*(-0.0000)})
	--({\sx*(3.7700)},{\sy*(-0.0000)})
	--({\sx*(3.7800)},{\sy*(-0.0000)})
	--({\sx*(3.7900)},{\sy*(-0.0000)})
	--({\sx*(3.8000)},{\sy*(-0.0000)})
	--({\sx*(3.8100)},{\sy*(-0.0000)})
	--({\sx*(3.8200)},{\sy*(-0.0000)})
	--({\sx*(3.8300)},{\sy*(-0.0000)})
	--({\sx*(3.8400)},{\sy*(-0.0000)})
	--({\sx*(3.8500)},{\sy*(-0.0000)})
	--({\sx*(3.8600)},{\sy*(-0.0000)})
	--({\sx*(3.8700)},{\sy*(-0.0000)})
	--({\sx*(3.8800)},{\sy*(-0.0000)})
	--({\sx*(3.8900)},{\sy*(-0.0000)})
	--({\sx*(3.9000)},{\sy*(-0.0000)})
	--({\sx*(3.9100)},{\sy*(-0.0000)})
	--({\sx*(3.9200)},{\sy*(-0.0000)})
	--({\sx*(3.9300)},{\sy*(0.0000)})
	--({\sx*(3.9400)},{\sy*(0.0000)})
	--({\sx*(3.9500)},{\sy*(0.0000)})
	--({\sx*(3.9600)},{\sy*(0.0000)})
	--({\sx*(3.9700)},{\sy*(0.0000)})
	--({\sx*(3.9800)},{\sy*(0.0000)})
	--({\sx*(3.9900)},{\sy*(0.0000)})
	--({\sx*(4.0000)},{\sy*(0.0000)})
	--({\sx*(4.0100)},{\sy*(0.0000)})
	--({\sx*(4.0200)},{\sy*(0.0000)})
	--({\sx*(4.0300)},{\sy*(0.0000)})
	--({\sx*(4.0400)},{\sy*(-0.0000)})
	--({\sx*(4.0500)},{\sy*(-0.0000)})
	--({\sx*(4.0600)},{\sy*(-0.0000)})
	--({\sx*(4.0700)},{\sy*(-0.0000)})
	--({\sx*(4.0800)},{\sy*(-0.0000)})
	--({\sx*(4.0900)},{\sy*(-0.0000)})
	--({\sx*(4.1000)},{\sy*(-0.0000)})
	--({\sx*(4.1100)},{\sy*(-0.0001)})
	--({\sx*(4.1200)},{\sy*(-0.0001)})
	--({\sx*(4.1300)},{\sy*(-0.0001)})
	--({\sx*(4.1400)},{\sy*(-0.0001)})
	--({\sx*(4.1500)},{\sy*(-0.0001)})
	--({\sx*(4.1600)},{\sy*(-0.0001)})
	--({\sx*(4.1700)},{\sy*(-0.0001)})
	--({\sx*(4.1800)},{\sy*(-0.0001)})
	--({\sx*(4.1900)},{\sy*(-0.0002)})
	--({\sx*(4.2000)},{\sy*(-0.0002)})
	--({\sx*(4.2100)},{\sy*(-0.0002)})
	--({\sx*(4.2200)},{\sy*(-0.0002)})
	--({\sx*(4.2300)},{\sy*(-0.0002)})
	--({\sx*(4.2400)},{\sy*(-0.0002)})
	--({\sx*(4.2500)},{\sy*(-0.0001)})
	--({\sx*(4.2600)},{\sy*(-0.0001)})
	--({\sx*(4.2700)},{\sy*(-0.0001)})
	--({\sx*(4.2800)},{\sy*(-0.0000)})
	--({\sx*(4.2900)},{\sy*(0.0000)})
	--({\sx*(4.3000)},{\sy*(0.0001)})
	--({\sx*(4.3100)},{\sy*(0.0002)})
	--({\sx*(4.3200)},{\sy*(0.0003)})
	--({\sx*(4.3300)},{\sy*(0.0005)})
	--({\sx*(4.3400)},{\sy*(0.0007)})
	--({\sx*(4.3500)},{\sy*(0.0009)})
	--({\sx*(4.3600)},{\sy*(0.0011)})
	--({\sx*(4.3700)},{\sy*(0.0014)})
	--({\sx*(4.3800)},{\sy*(0.0017)})
	--({\sx*(4.3900)},{\sy*(0.0020)})
	--({\sx*(4.4000)},{\sy*(0.0024)})
	--({\sx*(4.4100)},{\sy*(0.0029)})
	--({\sx*(4.4200)},{\sy*(0.0034)})
	--({\sx*(4.4300)},{\sy*(0.0039)})
	--({\sx*(4.4400)},{\sy*(0.0045)})
	--({\sx*(4.4500)},{\sy*(0.0051)})
	--({\sx*(4.4600)},{\sy*(0.0058)})
	--({\sx*(4.4700)},{\sy*(0.0065)})
	--({\sx*(4.4800)},{\sy*(0.0073)})
	--({\sx*(4.4900)},{\sy*(0.0081)})
	--({\sx*(4.5000)},{\sy*(0.0088)})
	--({\sx*(4.5100)},{\sy*(0.0096)})
	--({\sx*(4.5200)},{\sy*(0.0104)})
	--({\sx*(4.5300)},{\sy*(0.0110)})
	--({\sx*(4.5400)},{\sy*(0.0117)})
	--({\sx*(4.5500)},{\sy*(0.0122)})
	--({\sx*(4.5600)},{\sy*(0.0125)})
	--({\sx*(4.5700)},{\sy*(0.0126)})
	--({\sx*(4.5800)},{\sy*(0.0125)})
	--({\sx*(4.5900)},{\sy*(0.0120)})
	--({\sx*(4.6000)},{\sy*(0.0111)})
	--({\sx*(4.6100)},{\sy*(0.0097)})
	--({\sx*(4.6200)},{\sy*(0.0077)})
	--({\sx*(4.6300)},{\sy*(0.0049)})
	--({\sx*(4.6400)},{\sy*(0.0012)})
	--({\sx*(4.6500)},{\sy*(-0.0035)})
	--({\sx*(4.6600)},{\sy*(-0.0096)})
	--({\sx*(4.6700)},{\sy*(-0.0172)})
	--({\sx*(4.6800)},{\sy*(-0.0266)})
	--({\sx*(4.6900)},{\sy*(-0.0382)})
	--({\sx*(4.7000)},{\sy*(-0.0524)})
	--({\sx*(4.7100)},{\sy*(-0.0696)})
	--({\sx*(4.7200)},{\sy*(-0.0906)})
	--({\sx*(4.7300)},{\sy*(-0.1162)})
	--({\sx*(4.7400)},{\sy*(-0.1472)})
	--({\sx*(4.7500)},{\sy*(-0.1850)})
	--({\sx*(4.7600)},{\sy*(-0.2312)})
	--({\sx*(4.7700)},{\sy*(-0.2882)})
	--({\sx*(4.7800)},{\sy*(-0.3589)})
	--({\sx*(4.7900)},{\sy*(-0.4477)})
	--({\sx*(4.8000)},{\sy*(-0.5609)})
	--({\sx*(4.8100)},{\sy*(-0.7078)})
	--({\sx*(4.8200)},{\sy*(-0.9034)})
	--({\sx*(4.8300)},{\sy*(-1.1726)})
	--({\sx*(4.8400)},{\sy*(-1.5605)})
	--({\sx*(4.8500)},{\sy*(-2.1571)})
	--({\sx*(4.8600)},{\sy*(-3.1719)})
	--({\sx*(4.8700)},{\sy*(-5.2259)})
	--({\sx*(4.8800)},{\sy*(-11.3186)})
	--({\sx*(4.8900)},{\sy*(-276.7900)})
	--({\sx*(4.9000)},{\sy*(15.4522)})
	--({\sx*(4.9100)},{\sy*(8.4938)})
	--({\sx*(4.9200)},{\sy*(6.4237)})
	--({\sx*(4.9300)},{\sy*(5.6298)})
	--({\sx*(4.9400)},{\sy*(5.5286)})
	--({\sx*(4.9500)},{\sy*(6.2732)})
	--({\sx*(4.9600)},{\sy*(9.9989)})
	--({\sx*(4.9700)},{\sy*(-34.0545)})
	--({\sx*(4.9800)},{\sy*(-3.0637)})
	--({\sx*(4.9900)},{\sy*(-0.7800)})
	--({\sx*(5.0000)},{\sy*(0.0000)});
}
\def\xwerteh{
\fill[color=red] (0.0000,0) circle[radius={0.07/\skala}];
\fill[color=red] (0.3125,0) circle[radius={0.07/\skala}];
\fill[color=red] (0.6250,0) circle[radius={0.07/\skala}];
\fill[color=red] (0.9375,0) circle[radius={0.07/\skala}];
\fill[color=red] (1.2500,0) circle[radius={0.07/\skala}];
\fill[color=red] (1.5625,0) circle[radius={0.07/\skala}];
\fill[color=red] (1.8750,0) circle[radius={0.07/\skala}];
\fill[color=red] (2.1875,0) circle[radius={0.07/\skala}];
\fill[color=red] (2.5000,0) circle[radius={0.07/\skala}];
\fill[color=red] (2.8125,0) circle[radius={0.07/\skala}];
\fill[color=red] (3.1250,0) circle[radius={0.07/\skala}];
\fill[color=red] (3.4375,0) circle[radius={0.07/\skala}];
\fill[color=red] (3.7500,0) circle[radius={0.07/\skala}];
\fill[color=red] (4.0625,0) circle[radius={0.07/\skala}];
\fill[color=red] (4.3750,0) circle[radius={0.07/\skala}];
\fill[color=red] (4.6875,0) circle[radius={0.07/\skala}];
\fill[color=red] (5.0000,0) circle[radius={0.07/\skala}];
}
\def\punkteh{16}
\def\maxfehlerh{2.503\cdot 10^{-6}}
\def\fehlerh{
\draw[color=red,line width=1.4pt,line join=round] ({\sx*(0.000)},{\sy*(0.0000)})
	--({\sx*(0.0100)},{\sy*(0.1507)})
	--({\sx*(0.0200)},{\sy*(0.2676)})
	--({\sx*(0.0300)},{\sy*(0.3559)})
	--({\sx*(0.0400)},{\sy*(0.4198)})
	--({\sx*(0.0500)},{\sy*(0.4632)})
	--({\sx*(0.0600)},{\sy*(0.4897)})
	--({\sx*(0.0700)},{\sy*(0.5020)})
	--({\sx*(0.0800)},{\sy*(0.5029)})
	--({\sx*(0.0900)},{\sy*(0.4947)})
	--({\sx*(0.1000)},{\sy*(0.4791)})
	--({\sx*(0.1100)},{\sy*(0.4580)})
	--({\sx*(0.1200)},{\sy*(0.4328)})
	--({\sx*(0.1300)},{\sy*(0.4046)})
	--({\sx*(0.1400)},{\sy*(0.3746)})
	--({\sx*(0.1500)},{\sy*(0.3435)})
	--({\sx*(0.1600)},{\sy*(0.3121)})
	--({\sx*(0.1700)},{\sy*(0.2810)})
	--({\sx*(0.1800)},{\sy*(0.2506)})
	--({\sx*(0.1900)},{\sy*(0.2213)})
	--({\sx*(0.2000)},{\sy*(0.1933)})
	--({\sx*(0.2100)},{\sy*(0.1670)})
	--({\sx*(0.2200)},{\sy*(0.1423)})
	--({\sx*(0.2300)},{\sy*(0.1195)})
	--({\sx*(0.2400)},{\sy*(0.0986)})
	--({\sx*(0.2500)},{\sy*(0.0795)})
	--({\sx*(0.2600)},{\sy*(0.0623)})
	--({\sx*(0.2700)},{\sy*(0.0469)})
	--({\sx*(0.2800)},{\sy*(0.0333)})
	--({\sx*(0.2900)},{\sy*(0.0213)})
	--({\sx*(0.3000)},{\sy*(0.0109)})
	--({\sx*(0.3100)},{\sy*(0.0020)})
	--({\sx*(0.3200)},{\sy*(-0.0055)})
	--({\sx*(0.3300)},{\sy*(-0.0118)})
	--({\sx*(0.3400)},{\sy*(-0.0169)})
	--({\sx*(0.3500)},{\sy*(-0.0210)})
	--({\sx*(0.3600)},{\sy*(-0.0241)})
	--({\sx*(0.3700)},{\sy*(-0.0263)})
	--({\sx*(0.3800)},{\sy*(-0.0278)})
	--({\sx*(0.3900)},{\sy*(-0.0287)})
	--({\sx*(0.4000)},{\sy*(-0.0290)})
	--({\sx*(0.4100)},{\sy*(-0.0288)})
	--({\sx*(0.4200)},{\sy*(-0.0283)})
	--({\sx*(0.4300)},{\sy*(-0.0273)})
	--({\sx*(0.4400)},{\sy*(-0.0261)})
	--({\sx*(0.4500)},{\sy*(-0.0247)})
	--({\sx*(0.4600)},{\sy*(-0.0232)})
	--({\sx*(0.4700)},{\sy*(-0.0215)})
	--({\sx*(0.4800)},{\sy*(-0.0197)})
	--({\sx*(0.4900)},{\sy*(-0.0180)})
	--({\sx*(0.5000)},{\sy*(-0.0162)})
	--({\sx*(0.5100)},{\sy*(-0.0144)})
	--({\sx*(0.5200)},{\sy*(-0.0127)})
	--({\sx*(0.5300)},{\sy*(-0.0110)})
	--({\sx*(0.5400)},{\sy*(-0.0094)})
	--({\sx*(0.5500)},{\sy*(-0.0079)})
	--({\sx*(0.5600)},{\sy*(-0.0065)})
	--({\sx*(0.5700)},{\sy*(-0.0052)})
	--({\sx*(0.5800)},{\sy*(-0.0040)})
	--({\sx*(0.5900)},{\sy*(-0.0029)})
	--({\sx*(0.6000)},{\sy*(-0.0020)})
	--({\sx*(0.6100)},{\sy*(-0.0011)})
	--({\sx*(0.6200)},{\sy*(-0.0003)})
	--({\sx*(0.6300)},{\sy*(0.0003)})
	--({\sx*(0.6400)},{\sy*(0.0009)})
	--({\sx*(0.6500)},{\sy*(0.0013)})
	--({\sx*(0.6600)},{\sy*(0.0017)})
	--({\sx*(0.6700)},{\sy*(0.0020)})
	--({\sx*(0.6800)},{\sy*(0.0023)})
	--({\sx*(0.6900)},{\sy*(0.0024)})
	--({\sx*(0.7000)},{\sy*(0.0025)})
	--({\sx*(0.7100)},{\sy*(0.0026)})
	--({\sx*(0.7200)},{\sy*(0.0026)})
	--({\sx*(0.7300)},{\sy*(0.0026)})
	--({\sx*(0.7400)},{\sy*(0.0025)})
	--({\sx*(0.7500)},{\sy*(0.0024)})
	--({\sx*(0.7600)},{\sy*(0.0023)})
	--({\sx*(0.7700)},{\sy*(0.0021)})
	--({\sx*(0.7800)},{\sy*(0.0020)})
	--({\sx*(0.7900)},{\sy*(0.0018)})
	--({\sx*(0.8000)},{\sy*(0.0017)})
	--({\sx*(0.8100)},{\sy*(0.0015)})
	--({\sx*(0.8200)},{\sy*(0.0013)})
	--({\sx*(0.8300)},{\sy*(0.0012)})
	--({\sx*(0.8400)},{\sy*(0.0010)})
	--({\sx*(0.8500)},{\sy*(0.0009)})
	--({\sx*(0.8600)},{\sy*(0.0007)})
	--({\sx*(0.8700)},{\sy*(0.0006)})
	--({\sx*(0.8800)},{\sy*(0.0005)})
	--({\sx*(0.8900)},{\sy*(0.0004)})
	--({\sx*(0.9000)},{\sy*(0.0003)})
	--({\sx*(0.9100)},{\sy*(0.0002)})
	--({\sx*(0.9200)},{\sy*(0.0001)})
	--({\sx*(0.9300)},{\sy*(0.0000)})
	--({\sx*(0.9400)},{\sy*(-0.0000)})
	--({\sx*(0.9500)},{\sy*(-0.0001)})
	--({\sx*(0.9600)},{\sy*(-0.0001)})
	--({\sx*(0.9700)},{\sy*(-0.0001)})
	--({\sx*(0.9800)},{\sy*(-0.0001)})
	--({\sx*(0.9900)},{\sy*(-0.0001)})
	--({\sx*(1.0000)},{\sy*(-0.0002)})
	--({\sx*(1.0100)},{\sy*(-0.0002)})
	--({\sx*(1.0200)},{\sy*(-0.0001)})
	--({\sx*(1.0300)},{\sy*(-0.0001)})
	--({\sx*(1.0400)},{\sy*(-0.0001)})
	--({\sx*(1.0500)},{\sy*(-0.0001)})
	--({\sx*(1.0600)},{\sy*(-0.0001)})
	--({\sx*(1.0700)},{\sy*(-0.0001)})
	--({\sx*(1.0800)},{\sy*(-0.0001)})
	--({\sx*(1.0900)},{\sy*(-0.0000)})
	--({\sx*(1.1000)},{\sy*(-0.0000)})
	--({\sx*(1.1100)},{\sy*(-0.0000)})
	--({\sx*(1.1200)},{\sy*(0.0000)})
	--({\sx*(1.1300)},{\sy*(0.0000)})
	--({\sx*(1.1400)},{\sy*(0.0000)})
	--({\sx*(1.1500)},{\sy*(0.0000)})
	--({\sx*(1.1600)},{\sy*(0.0000)})
	--({\sx*(1.1700)},{\sy*(0.0000)})
	--({\sx*(1.1800)},{\sy*(0.0000)})
	--({\sx*(1.1900)},{\sy*(0.0000)})
	--({\sx*(1.2000)},{\sy*(0.0000)})
	--({\sx*(1.2100)},{\sy*(0.0000)})
	--({\sx*(1.2200)},{\sy*(0.0000)})
	--({\sx*(1.2300)},{\sy*(0.0000)})
	--({\sx*(1.2400)},{\sy*(0.0000)})
	--({\sx*(1.2500)},{\sy*(0.0000)})
	--({\sx*(1.2600)},{\sy*(-0.0000)})
	--({\sx*(1.2700)},{\sy*(-0.0000)})
	--({\sx*(1.2800)},{\sy*(-0.0000)})
	--({\sx*(1.2900)},{\sy*(-0.0001)})
	--({\sx*(1.3000)},{\sy*(-0.0001)})
	--({\sx*(1.3100)},{\sy*(-0.0001)})
	--({\sx*(1.3200)},{\sy*(-0.0001)})
	--({\sx*(1.3300)},{\sy*(-0.0001)})
	--({\sx*(1.3400)},{\sy*(-0.0001)})
	--({\sx*(1.3500)},{\sy*(-0.0001)})
	--({\sx*(1.3600)},{\sy*(-0.0001)})
	--({\sx*(1.3700)},{\sy*(-0.0001)})
	--({\sx*(1.3800)},{\sy*(-0.0001)})
	--({\sx*(1.3900)},{\sy*(-0.0002)})
	--({\sx*(1.4000)},{\sy*(-0.0002)})
	--({\sx*(1.4100)},{\sy*(-0.0002)})
	--({\sx*(1.4200)},{\sy*(-0.0002)})
	--({\sx*(1.4300)},{\sy*(-0.0002)})
	--({\sx*(1.4400)},{\sy*(-0.0001)})
	--({\sx*(1.4500)},{\sy*(-0.0001)})
	--({\sx*(1.4600)},{\sy*(-0.0001)})
	--({\sx*(1.4700)},{\sy*(-0.0001)})
	--({\sx*(1.4800)},{\sy*(-0.0001)})
	--({\sx*(1.4900)},{\sy*(-0.0001)})
	--({\sx*(1.5000)},{\sy*(-0.0001)})
	--({\sx*(1.5100)},{\sy*(-0.0001)})
	--({\sx*(1.5200)},{\sy*(-0.0001)})
	--({\sx*(1.5300)},{\sy*(-0.0001)})
	--({\sx*(1.5400)},{\sy*(-0.0000)})
	--({\sx*(1.5500)},{\sy*(-0.0000)})
	--({\sx*(1.5600)},{\sy*(-0.0000)})
	--({\sx*(1.5700)},{\sy*(0.0000)})
	--({\sx*(1.5800)},{\sy*(0.0000)})
	--({\sx*(1.5900)},{\sy*(0.0000)})
	--({\sx*(1.6000)},{\sy*(0.0001)})
	--({\sx*(1.6100)},{\sy*(0.0001)})
	--({\sx*(1.6200)},{\sy*(0.0001)})
	--({\sx*(1.6300)},{\sy*(0.0001)})
	--({\sx*(1.6400)},{\sy*(0.0001)})
	--({\sx*(1.6500)},{\sy*(0.0001)})
	--({\sx*(1.6600)},{\sy*(0.0001)})
	--({\sx*(1.6700)},{\sy*(0.0001)})
	--({\sx*(1.6800)},{\sy*(0.0001)})
	--({\sx*(1.6900)},{\sy*(0.0001)})
	--({\sx*(1.7000)},{\sy*(0.0002)})
	--({\sx*(1.7100)},{\sy*(0.0002)})
	--({\sx*(1.7200)},{\sy*(0.0002)})
	--({\sx*(1.7300)},{\sy*(0.0002)})
	--({\sx*(1.7400)},{\sy*(0.0001)})
	--({\sx*(1.7500)},{\sy*(0.0001)})
	--({\sx*(1.7600)},{\sy*(0.0001)})
	--({\sx*(1.7700)},{\sy*(0.0001)})
	--({\sx*(1.7800)},{\sy*(0.0001)})
	--({\sx*(1.7900)},{\sy*(0.0001)})
	--({\sx*(1.8000)},{\sy*(0.0001)})
	--({\sx*(1.8100)},{\sy*(0.0001)})
	--({\sx*(1.8200)},{\sy*(0.0001)})
	--({\sx*(1.8300)},{\sy*(0.0001)})
	--({\sx*(1.8400)},{\sy*(0.0001)})
	--({\sx*(1.8500)},{\sy*(0.0000)})
	--({\sx*(1.8600)},{\sy*(0.0000)})
	--({\sx*(1.8700)},{\sy*(0.0000)})
	--({\sx*(1.8800)},{\sy*(-0.0000)})
	--({\sx*(1.8900)},{\sy*(-0.0000)})
	--({\sx*(1.9000)},{\sy*(-0.0000)})
	--({\sx*(1.9100)},{\sy*(-0.0000)})
	--({\sx*(1.9200)},{\sy*(-0.0001)})
	--({\sx*(1.9300)},{\sy*(-0.0001)})
	--({\sx*(1.9400)},{\sy*(-0.0001)})
	--({\sx*(1.9500)},{\sy*(-0.0001)})
	--({\sx*(1.9600)},{\sy*(-0.0001)})
	--({\sx*(1.9700)},{\sy*(-0.0001)})
	--({\sx*(1.9800)},{\sy*(-0.0001)})
	--({\sx*(1.9900)},{\sy*(-0.0001)})
	--({\sx*(2.0000)},{\sy*(-0.0001)})
	--({\sx*(2.0100)},{\sy*(-0.0001)})
	--({\sx*(2.0200)},{\sy*(-0.0001)})
	--({\sx*(2.0300)},{\sy*(-0.0001)})
	--({\sx*(2.0400)},{\sy*(-0.0001)})
	--({\sx*(2.0500)},{\sy*(-0.0001)})
	--({\sx*(2.0600)},{\sy*(-0.0001)})
	--({\sx*(2.0700)},{\sy*(-0.0001)})
	--({\sx*(2.0800)},{\sy*(-0.0001)})
	--({\sx*(2.0900)},{\sy*(-0.0001)})
	--({\sx*(2.1000)},{\sy*(-0.0001)})
	--({\sx*(2.1100)},{\sy*(-0.0001)})
	--({\sx*(2.1200)},{\sy*(-0.0001)})
	--({\sx*(2.1300)},{\sy*(-0.0001)})
	--({\sx*(2.1400)},{\sy*(-0.0001)})
	--({\sx*(2.1500)},{\sy*(-0.0001)})
	--({\sx*(2.1600)},{\sy*(-0.0000)})
	--({\sx*(2.1700)},{\sy*(-0.0000)})
	--({\sx*(2.1800)},{\sy*(-0.0000)})
	--({\sx*(2.1900)},{\sy*(0.0000)})
	--({\sx*(2.2000)},{\sy*(0.0000)})
	--({\sx*(2.2100)},{\sy*(0.0000)})
	--({\sx*(2.2200)},{\sy*(0.0000)})
	--({\sx*(2.2300)},{\sy*(0.0001)})
	--({\sx*(2.2400)},{\sy*(0.0001)})
	--({\sx*(2.2500)},{\sy*(0.0001)})
	--({\sx*(2.2600)},{\sy*(0.0001)})
	--({\sx*(2.2700)},{\sy*(0.0001)})
	--({\sx*(2.2800)},{\sy*(0.0001)})
	--({\sx*(2.2900)},{\sy*(0.0001)})
	--({\sx*(2.3000)},{\sy*(0.0001)})
	--({\sx*(2.3100)},{\sy*(0.0001)})
	--({\sx*(2.3200)},{\sy*(0.0001)})
	--({\sx*(2.3300)},{\sy*(0.0001)})
	--({\sx*(2.3400)},{\sy*(0.0001)})
	--({\sx*(2.3500)},{\sy*(0.0001)})
	--({\sx*(2.3600)},{\sy*(0.0001)})
	--({\sx*(2.3700)},{\sy*(0.0001)})
	--({\sx*(2.3800)},{\sy*(0.0001)})
	--({\sx*(2.3900)},{\sy*(0.0001)})
	--({\sx*(2.4000)},{\sy*(0.0001)})
	--({\sx*(2.4100)},{\sy*(0.0001)})
	--({\sx*(2.4200)},{\sy*(0.0001)})
	--({\sx*(2.4300)},{\sy*(0.0001)})
	--({\sx*(2.4400)},{\sy*(0.0001)})
	--({\sx*(2.4500)},{\sy*(0.0001)})
	--({\sx*(2.4600)},{\sy*(0.0001)})
	--({\sx*(2.4700)},{\sy*(0.0000)})
	--({\sx*(2.4800)},{\sy*(0.0000)})
	--({\sx*(2.4900)},{\sy*(0.0000)})
	--({\sx*(2.5000)},{\sy*(0.0000)})
	--({\sx*(2.5100)},{\sy*(-0.0000)})
	--({\sx*(2.5200)},{\sy*(-0.0000)})
	--({\sx*(2.5300)},{\sy*(-0.0000)})
	--({\sx*(2.5400)},{\sy*(-0.0001)})
	--({\sx*(2.5500)},{\sy*(-0.0001)})
	--({\sx*(2.5600)},{\sy*(-0.0001)})
	--({\sx*(2.5700)},{\sy*(-0.0001)})
	--({\sx*(2.5800)},{\sy*(-0.0001)})
	--({\sx*(2.5900)},{\sy*(-0.0001)})
	--({\sx*(2.6000)},{\sy*(-0.0001)})
	--({\sx*(2.6100)},{\sy*(-0.0001)})
	--({\sx*(2.6200)},{\sy*(-0.0002)})
	--({\sx*(2.6300)},{\sy*(-0.0002)})
	--({\sx*(2.6400)},{\sy*(-0.0002)})
	--({\sx*(2.6500)},{\sy*(-0.0002)})
	--({\sx*(2.6600)},{\sy*(-0.0002)})
	--({\sx*(2.6700)},{\sy*(-0.0002)})
	--({\sx*(2.6800)},{\sy*(-0.0002)})
	--({\sx*(2.6900)},{\sy*(-0.0002)})
	--({\sx*(2.7000)},{\sy*(-0.0002)})
	--({\sx*(2.7100)},{\sy*(-0.0002)})
	--({\sx*(2.7200)},{\sy*(-0.0001)})
	--({\sx*(2.7300)},{\sy*(-0.0001)})
	--({\sx*(2.7400)},{\sy*(-0.0001)})
	--({\sx*(2.7500)},{\sy*(-0.0001)})
	--({\sx*(2.7600)},{\sy*(-0.0001)})
	--({\sx*(2.7700)},{\sy*(-0.0001)})
	--({\sx*(2.7800)},{\sy*(-0.0001)})
	--({\sx*(2.7900)},{\sy*(-0.0000)})
	--({\sx*(2.8000)},{\sy*(-0.0000)})
	--({\sx*(2.8100)},{\sy*(-0.0000)})
	--({\sx*(2.8200)},{\sy*(0.0000)})
	--({\sx*(2.8300)},{\sy*(0.0000)})
	--({\sx*(2.8400)},{\sy*(0.0001)})
	--({\sx*(2.8500)},{\sy*(0.0001)})
	--({\sx*(2.8600)},{\sy*(0.0001)})
	--({\sx*(2.8700)},{\sy*(0.0001)})
	--({\sx*(2.8800)},{\sy*(0.0001)})
	--({\sx*(2.8900)},{\sy*(0.0002)})
	--({\sx*(2.9000)},{\sy*(0.0002)})
	--({\sx*(2.9100)},{\sy*(0.0002)})
	--({\sx*(2.9200)},{\sy*(0.0002)})
	--({\sx*(2.9300)},{\sy*(0.0002)})
	--({\sx*(2.9400)},{\sy*(0.0002)})
	--({\sx*(2.9500)},{\sy*(0.0002)})
	--({\sx*(2.9600)},{\sy*(0.0002)})
	--({\sx*(2.9700)},{\sy*(0.0002)})
	--({\sx*(2.9800)},{\sy*(0.0002)})
	--({\sx*(2.9900)},{\sy*(0.0002)})
	--({\sx*(3.0000)},{\sy*(0.0002)})
	--({\sx*(3.0100)},{\sy*(0.0002)})
	--({\sx*(3.0200)},{\sy*(0.0002)})
	--({\sx*(3.0300)},{\sy*(0.0002)})
	--({\sx*(3.0400)},{\sy*(0.0002)})
	--({\sx*(3.0500)},{\sy*(0.0002)})
	--({\sx*(3.0600)},{\sy*(0.0002)})
	--({\sx*(3.0700)},{\sy*(0.0002)})
	--({\sx*(3.0800)},{\sy*(0.0001)})
	--({\sx*(3.0900)},{\sy*(0.0001)})
	--({\sx*(3.1000)},{\sy*(0.0001)})
	--({\sx*(3.1100)},{\sy*(0.0000)})
	--({\sx*(3.1200)},{\sy*(0.0000)})
	--({\sx*(3.1300)},{\sy*(-0.0000)})
	--({\sx*(3.1400)},{\sy*(-0.0000)})
	--({\sx*(3.1500)},{\sy*(-0.0001)})
	--({\sx*(3.1600)},{\sy*(-0.0001)})
	--({\sx*(3.1700)},{\sy*(-0.0002)})
	--({\sx*(3.1800)},{\sy*(-0.0002)})
	--({\sx*(3.1900)},{\sy*(-0.0002)})
	--({\sx*(3.2000)},{\sy*(-0.0002)})
	--({\sx*(3.2100)},{\sy*(-0.0003)})
	--({\sx*(3.2200)},{\sy*(-0.0003)})
	--({\sx*(3.2300)},{\sy*(-0.0003)})
	--({\sx*(3.2400)},{\sy*(-0.0004)})
	--({\sx*(3.2500)},{\sy*(-0.0004)})
	--({\sx*(3.2600)},{\sy*(-0.0004)})
	--({\sx*(3.2700)},{\sy*(-0.0004)})
	--({\sx*(3.2800)},{\sy*(-0.0004)})
	--({\sx*(3.2900)},{\sy*(-0.0004)})
	--({\sx*(3.3000)},{\sy*(-0.0004)})
	--({\sx*(3.3100)},{\sy*(-0.0004)})
	--({\sx*(3.3200)},{\sy*(-0.0004)})
	--({\sx*(3.3300)},{\sy*(-0.0004)})
	--({\sx*(3.3400)},{\sy*(-0.0004)})
	--({\sx*(3.3500)},{\sy*(-0.0004)})
	--({\sx*(3.3600)},{\sy*(-0.0004)})
	--({\sx*(3.3700)},{\sy*(-0.0003)})
	--({\sx*(3.3800)},{\sy*(-0.0003)})
	--({\sx*(3.3900)},{\sy*(-0.0003)})
	--({\sx*(3.4000)},{\sy*(-0.0002)})
	--({\sx*(3.4100)},{\sy*(-0.0002)})
	--({\sx*(3.4200)},{\sy*(-0.0001)})
	--({\sx*(3.4300)},{\sy*(-0.0000)})
	--({\sx*(3.4400)},{\sy*(0.0000)})
	--({\sx*(3.4500)},{\sy*(0.0001)})
	--({\sx*(3.4600)},{\sy*(0.0001)})
	--({\sx*(3.4700)},{\sy*(0.0002)})
	--({\sx*(3.4800)},{\sy*(0.0003)})
	--({\sx*(3.4900)},{\sy*(0.0004)})
	--({\sx*(3.5000)},{\sy*(0.0004)})
	--({\sx*(3.5100)},{\sy*(0.0005)})
	--({\sx*(3.5200)},{\sy*(0.0006)})
	--({\sx*(3.5300)},{\sy*(0.0006)})
	--({\sx*(3.5400)},{\sy*(0.0007)})
	--({\sx*(3.5500)},{\sy*(0.0008)})
	--({\sx*(3.5600)},{\sy*(0.0008)})
	--({\sx*(3.5700)},{\sy*(0.0009)})
	--({\sx*(3.5800)},{\sy*(0.0009)})
	--({\sx*(3.5900)},{\sy*(0.0010)})
	--({\sx*(3.6000)},{\sy*(0.0010)})
	--({\sx*(3.6100)},{\sy*(0.0010)})
	--({\sx*(3.6200)},{\sy*(0.0010)})
	--({\sx*(3.6300)},{\sy*(0.0010)})
	--({\sx*(3.6400)},{\sy*(0.0010)})
	--({\sx*(3.6500)},{\sy*(0.0010)})
	--({\sx*(3.6600)},{\sy*(0.0009)})
	--({\sx*(3.6700)},{\sy*(0.0009)})
	--({\sx*(3.6800)},{\sy*(0.0008)})
	--({\sx*(3.6900)},{\sy*(0.0007)})
	--({\sx*(3.7000)},{\sy*(0.0006)})
	--({\sx*(3.7100)},{\sy*(0.0005)})
	--({\sx*(3.7200)},{\sy*(0.0004)})
	--({\sx*(3.7300)},{\sy*(0.0003)})
	--({\sx*(3.7400)},{\sy*(0.0002)})
	--({\sx*(3.7500)},{\sy*(0.0000)})
	--({\sx*(3.7600)},{\sy*(-0.0002)})
	--({\sx*(3.7700)},{\sy*(-0.0003)})
	--({\sx*(3.7800)},{\sy*(-0.0005)})
	--({\sx*(3.7900)},{\sy*(-0.0007)})
	--({\sx*(3.8000)},{\sy*(-0.0009)})
	--({\sx*(3.8100)},{\sy*(-0.0011)})
	--({\sx*(3.8200)},{\sy*(-0.0013)})
	--({\sx*(3.8300)},{\sy*(-0.0015)})
	--({\sx*(3.8400)},{\sy*(-0.0017)})
	--({\sx*(3.8500)},{\sy*(-0.0019)})
	--({\sx*(3.8600)},{\sy*(-0.0021)})
	--({\sx*(3.8700)},{\sy*(-0.0023)})
	--({\sx*(3.8800)},{\sy*(-0.0025)})
	--({\sx*(3.8900)},{\sy*(-0.0026)})
	--({\sx*(3.9000)},{\sy*(-0.0027)})
	--({\sx*(3.9100)},{\sy*(-0.0029)})
	--({\sx*(3.9200)},{\sy*(-0.0029)})
	--({\sx*(3.9300)},{\sy*(-0.0030)})
	--({\sx*(3.9400)},{\sy*(-0.0030)})
	--({\sx*(3.9500)},{\sy*(-0.0030)})
	--({\sx*(3.9600)},{\sy*(-0.0030)})
	--({\sx*(3.9700)},{\sy*(-0.0029)})
	--({\sx*(3.9800)},{\sy*(-0.0028)})
	--({\sx*(3.9900)},{\sy*(-0.0026)})
	--({\sx*(4.0000)},{\sy*(-0.0024)})
	--({\sx*(4.0100)},{\sy*(-0.0022)})
	--({\sx*(4.0200)},{\sy*(-0.0019)})
	--({\sx*(4.0300)},{\sy*(-0.0015)})
	--({\sx*(4.0400)},{\sy*(-0.0011)})
	--({\sx*(4.0500)},{\sy*(-0.0006)})
	--({\sx*(4.0600)},{\sy*(-0.0001)})
	--({\sx*(4.0700)},{\sy*(0.0004)})
	--({\sx*(4.0800)},{\sy*(0.0010)})
	--({\sx*(4.0900)},{\sy*(0.0017)})
	--({\sx*(4.1000)},{\sy*(0.0023)})
	--({\sx*(4.1100)},{\sy*(0.0031)})
	--({\sx*(4.1200)},{\sy*(0.0038)})
	--({\sx*(4.1300)},{\sy*(0.0046)})
	--({\sx*(4.1400)},{\sy*(0.0054)})
	--({\sx*(4.1500)},{\sy*(0.0062)})
	--({\sx*(4.1600)},{\sy*(0.0070)})
	--({\sx*(4.1700)},{\sy*(0.0078)})
	--({\sx*(4.1800)},{\sy*(0.0086)})
	--({\sx*(4.1900)},{\sy*(0.0093)})
	--({\sx*(4.2000)},{\sy*(0.0100)})
	--({\sx*(4.2100)},{\sy*(0.0107)})
	--({\sx*(4.2200)},{\sy*(0.0113)})
	--({\sx*(4.2300)},{\sy*(0.0118)})
	--({\sx*(4.2400)},{\sy*(0.0122)})
	--({\sx*(4.2500)},{\sy*(0.0125)})
	--({\sx*(4.2600)},{\sy*(0.0126)})
	--({\sx*(4.2700)},{\sy*(0.0126)})
	--({\sx*(4.2800)},{\sy*(0.0125)})
	--({\sx*(4.2900)},{\sy*(0.0122)})
	--({\sx*(4.3000)},{\sy*(0.0117)})
	--({\sx*(4.3100)},{\sy*(0.0109)})
	--({\sx*(4.3200)},{\sy*(0.0100)})
	--({\sx*(4.3300)},{\sy*(0.0087)})
	--({\sx*(4.3400)},{\sy*(0.0073)})
	--({\sx*(4.3500)},{\sy*(0.0056)})
	--({\sx*(4.3600)},{\sy*(0.0035)})
	--({\sx*(4.3700)},{\sy*(0.0013)})
	--({\sx*(4.3800)},{\sy*(-0.0013)})
	--({\sx*(4.3900)},{\sy*(-0.0042)})
	--({\sx*(4.4000)},{\sy*(-0.0074)})
	--({\sx*(4.4100)},{\sy*(-0.0108)})
	--({\sx*(4.4200)},{\sy*(-0.0146)})
	--({\sx*(4.4300)},{\sy*(-0.0186)})
	--({\sx*(4.4400)},{\sy*(-0.0228)})
	--({\sx*(4.4500)},{\sy*(-0.0273)})
	--({\sx*(4.4600)},{\sy*(-0.0319)})
	--({\sx*(4.4700)},{\sy*(-0.0367)})
	--({\sx*(4.4800)},{\sy*(-0.0416)})
	--({\sx*(4.4900)},{\sy*(-0.0466)})
	--({\sx*(4.5000)},{\sy*(-0.0515)})
	--({\sx*(4.5100)},{\sy*(-0.0564)})
	--({\sx*(4.5200)},{\sy*(-0.0611)})
	--({\sx*(4.5300)},{\sy*(-0.0656)})
	--({\sx*(4.5400)},{\sy*(-0.0697)})
	--({\sx*(4.5500)},{\sy*(-0.0734)})
	--({\sx*(4.5600)},{\sy*(-0.0765)})
	--({\sx*(4.5700)},{\sy*(-0.0789)})
	--({\sx*(4.5800)},{\sy*(-0.0805)})
	--({\sx*(4.5900)},{\sy*(-0.0811)})
	--({\sx*(4.6000)},{\sy*(-0.0805)})
	--({\sx*(4.6100)},{\sy*(-0.0787)})
	--({\sx*(4.6200)},{\sy*(-0.0753)})
	--({\sx*(4.6300)},{\sy*(-0.0704)})
	--({\sx*(4.6400)},{\sy*(-0.0635)})
	--({\sx*(4.6500)},{\sy*(-0.0547)})
	--({\sx*(4.6600)},{\sy*(-0.0436)})
	--({\sx*(4.6700)},{\sy*(-0.0300)})
	--({\sx*(4.6800)},{\sy*(-0.0139)})
	--({\sx*(4.6900)},{\sy*(0.0050)})
	--({\sx*(4.7000)},{\sy*(0.0269)})
	--({\sx*(4.7100)},{\sy*(0.0519)})
	--({\sx*(4.7200)},{\sy*(0.0802)})
	--({\sx*(4.7300)},{\sy*(0.1118)})
	--({\sx*(4.7400)},{\sy*(0.1469)})
	--({\sx*(4.7500)},{\sy*(0.1855)})
	--({\sx*(4.7600)},{\sy*(0.2276)})
	--({\sx*(4.7700)},{\sy*(0.2732)})
	--({\sx*(4.7800)},{\sy*(0.3220)})
	--({\sx*(4.7900)},{\sy*(0.3740)})
	--({\sx*(4.8000)},{\sy*(0.4288)})
	--({\sx*(4.8100)},{\sy*(0.4861)})
	--({\sx*(4.8200)},{\sy*(0.5452)})
	--({\sx*(4.8300)},{\sy*(0.6056)})
	--({\sx*(4.8400)},{\sy*(0.6664)})
	--({\sx*(4.8500)},{\sy*(0.7267)})
	--({\sx*(4.8600)},{\sy*(0.7852)})
	--({\sx*(4.8700)},{\sy*(0.8405)})
	--({\sx*(4.8800)},{\sy*(0.8910)})
	--({\sx*(4.8900)},{\sy*(0.9346)})
	--({\sx*(4.9000)},{\sy*(0.9692)})
	--({\sx*(4.9100)},{\sy*(0.9920)})
	--({\sx*(4.9200)},{\sy*(1.0000)})
	--({\sx*(4.9300)},{\sy*(0.9898)})
	--({\sx*(4.9400)},{\sy*(0.9574)})
	--({\sx*(4.9500)},{\sy*(0.8983)})
	--({\sx*(4.9600)},{\sy*(0.8074)})
	--({\sx*(4.9700)},{\sy*(0.6789)})
	--({\sx*(4.9800)},{\sy*(0.5066)})
	--({\sx*(4.9900)},{\sy*(0.2829)})
	--({\sx*(5.0000)},{\sy*(0.0000)});
}
\def\relfehlerh{
\draw[color=blue,line width=1.4pt,line join=round] ({\sx*(0.000)},{\sy*(0.0000)})
	--({\sx*(0.0100)},{\sy*(0.0000)})
	--({\sx*(0.0200)},{\sy*(0.0000)})
	--({\sx*(0.0300)},{\sy*(0.0000)})
	--({\sx*(0.0400)},{\sy*(0.0000)})
	--({\sx*(0.0500)},{\sy*(0.0000)})
	--({\sx*(0.0600)},{\sy*(0.0000)})
	--({\sx*(0.0700)},{\sy*(0.0000)})
	--({\sx*(0.0800)},{\sy*(0.0000)})
	--({\sx*(0.0900)},{\sy*(0.0000)})
	--({\sx*(0.1000)},{\sy*(0.0000)})
	--({\sx*(0.1100)},{\sy*(0.0000)})
	--({\sx*(0.1200)},{\sy*(0.0000)})
	--({\sx*(0.1300)},{\sy*(0.0000)})
	--({\sx*(0.1400)},{\sy*(0.0000)})
	--({\sx*(0.1500)},{\sy*(0.0000)})
	--({\sx*(0.1600)},{\sy*(0.0000)})
	--({\sx*(0.1700)},{\sy*(0.0000)})
	--({\sx*(0.1800)},{\sy*(0.0000)})
	--({\sx*(0.1900)},{\sy*(0.0000)})
	--({\sx*(0.2000)},{\sy*(0.0000)})
	--({\sx*(0.2100)},{\sy*(0.0000)})
	--({\sx*(0.2200)},{\sy*(0.0000)})
	--({\sx*(0.2300)},{\sy*(0.0000)})
	--({\sx*(0.2400)},{\sy*(0.0000)})
	--({\sx*(0.2500)},{\sy*(0.0000)})
	--({\sx*(0.2600)},{\sy*(0.0000)})
	--({\sx*(0.2700)},{\sy*(0.0000)})
	--({\sx*(0.2800)},{\sy*(0.0000)})
	--({\sx*(0.2900)},{\sy*(0.0000)})
	--({\sx*(0.3000)},{\sy*(0.0000)})
	--({\sx*(0.3100)},{\sy*(0.0000)})
	--({\sx*(0.3200)},{\sy*(-0.0000)})
	--({\sx*(0.3300)},{\sy*(-0.0000)})
	--({\sx*(0.3400)},{\sy*(-0.0000)})
	--({\sx*(0.3500)},{\sy*(-0.0000)})
	--({\sx*(0.3600)},{\sy*(-0.0000)})
	--({\sx*(0.3700)},{\sy*(-0.0000)})
	--({\sx*(0.3800)},{\sy*(-0.0000)})
	--({\sx*(0.3900)},{\sy*(-0.0000)})
	--({\sx*(0.4000)},{\sy*(-0.0000)})
	--({\sx*(0.4100)},{\sy*(-0.0000)})
	--({\sx*(0.4200)},{\sy*(-0.0000)})
	--({\sx*(0.4300)},{\sy*(-0.0000)})
	--({\sx*(0.4400)},{\sy*(-0.0000)})
	--({\sx*(0.4500)},{\sy*(-0.0000)})
	--({\sx*(0.4600)},{\sy*(-0.0000)})
	--({\sx*(0.4700)},{\sy*(-0.0000)})
	--({\sx*(0.4800)},{\sy*(-0.0000)})
	--({\sx*(0.4900)},{\sy*(-0.0000)})
	--({\sx*(0.5000)},{\sy*(-0.0000)})
	--({\sx*(0.5100)},{\sy*(-0.0000)})
	--({\sx*(0.5200)},{\sy*(-0.0000)})
	--({\sx*(0.5300)},{\sy*(-0.0000)})
	--({\sx*(0.5400)},{\sy*(-0.0000)})
	--({\sx*(0.5500)},{\sy*(-0.0000)})
	--({\sx*(0.5600)},{\sy*(-0.0000)})
	--({\sx*(0.5700)},{\sy*(-0.0000)})
	--({\sx*(0.5800)},{\sy*(-0.0000)})
	--({\sx*(0.5900)},{\sy*(-0.0000)})
	--({\sx*(0.6000)},{\sy*(-0.0000)})
	--({\sx*(0.6100)},{\sy*(-0.0000)})
	--({\sx*(0.6200)},{\sy*(-0.0000)})
	--({\sx*(0.6300)},{\sy*(0.0000)})
	--({\sx*(0.6400)},{\sy*(0.0000)})
	--({\sx*(0.6500)},{\sy*(0.0000)})
	--({\sx*(0.6600)},{\sy*(0.0000)})
	--({\sx*(0.6700)},{\sy*(0.0000)})
	--({\sx*(0.6800)},{\sy*(0.0000)})
	--({\sx*(0.6900)},{\sy*(0.0000)})
	--({\sx*(0.7000)},{\sy*(0.0000)})
	--({\sx*(0.7100)},{\sy*(0.0000)})
	--({\sx*(0.7200)},{\sy*(0.0000)})
	--({\sx*(0.7300)},{\sy*(0.0000)})
	--({\sx*(0.7400)},{\sy*(0.0000)})
	--({\sx*(0.7500)},{\sy*(0.0000)})
	--({\sx*(0.7600)},{\sy*(0.0000)})
	--({\sx*(0.7700)},{\sy*(0.0000)})
	--({\sx*(0.7800)},{\sy*(0.0000)})
	--({\sx*(0.7900)},{\sy*(0.0000)})
	--({\sx*(0.8000)},{\sy*(0.0000)})
	--({\sx*(0.8100)},{\sy*(0.0000)})
	--({\sx*(0.8200)},{\sy*(0.0000)})
	--({\sx*(0.8300)},{\sy*(0.0000)})
	--({\sx*(0.8400)},{\sy*(0.0000)})
	--({\sx*(0.8500)},{\sy*(0.0000)})
	--({\sx*(0.8600)},{\sy*(0.0000)})
	--({\sx*(0.8700)},{\sy*(0.0000)})
	--({\sx*(0.8800)},{\sy*(0.0000)})
	--({\sx*(0.8900)},{\sy*(0.0000)})
	--({\sx*(0.9000)},{\sy*(0.0000)})
	--({\sx*(0.9100)},{\sy*(0.0000)})
	--({\sx*(0.9200)},{\sy*(0.0000)})
	--({\sx*(0.9300)},{\sy*(0.0000)})
	--({\sx*(0.9400)},{\sy*(-0.0000)})
	--({\sx*(0.9500)},{\sy*(-0.0000)})
	--({\sx*(0.9600)},{\sy*(-0.0000)})
	--({\sx*(0.9700)},{\sy*(-0.0000)})
	--({\sx*(0.9800)},{\sy*(-0.0000)})
	--({\sx*(0.9900)},{\sy*(-0.0000)})
	--({\sx*(1.0000)},{\sy*(-0.0000)})
	--({\sx*(1.0100)},{\sy*(-0.0000)})
	--({\sx*(1.0200)},{\sy*(-0.0000)})
	--({\sx*(1.0300)},{\sy*(-0.0000)})
	--({\sx*(1.0400)},{\sy*(-0.0000)})
	--({\sx*(1.0500)},{\sy*(-0.0000)})
	--({\sx*(1.0600)},{\sy*(-0.0000)})
	--({\sx*(1.0700)},{\sy*(-0.0000)})
	--({\sx*(1.0800)},{\sy*(-0.0000)})
	--({\sx*(1.0900)},{\sy*(-0.0000)})
	--({\sx*(1.1000)},{\sy*(-0.0000)})
	--({\sx*(1.1100)},{\sy*(-0.0000)})
	--({\sx*(1.1200)},{\sy*(0.0000)})
	--({\sx*(1.1300)},{\sy*(0.0000)})
	--({\sx*(1.1400)},{\sy*(0.0000)})
	--({\sx*(1.1500)},{\sy*(0.0000)})
	--({\sx*(1.1600)},{\sy*(0.0000)})
	--({\sx*(1.1700)},{\sy*(0.0000)})
	--({\sx*(1.1800)},{\sy*(0.0000)})
	--({\sx*(1.1900)},{\sy*(0.0000)})
	--({\sx*(1.2000)},{\sy*(0.0000)})
	--({\sx*(1.2100)},{\sy*(0.0000)})
	--({\sx*(1.2200)},{\sy*(0.0000)})
	--({\sx*(1.2300)},{\sy*(0.0000)})
	--({\sx*(1.2400)},{\sy*(0.0000)})
	--({\sx*(1.2500)},{\sy*(0.0000)})
	--({\sx*(1.2600)},{\sy*(-0.0000)})
	--({\sx*(1.2700)},{\sy*(-0.0000)})
	--({\sx*(1.2800)},{\sy*(-0.0000)})
	--({\sx*(1.2900)},{\sy*(-0.0000)})
	--({\sx*(1.3000)},{\sy*(-0.0000)})
	--({\sx*(1.3100)},{\sy*(-0.0000)})
	--({\sx*(1.3200)},{\sy*(-0.0000)})
	--({\sx*(1.3300)},{\sy*(-0.0000)})
	--({\sx*(1.3400)},{\sy*(-0.0000)})
	--({\sx*(1.3500)},{\sy*(-0.0000)})
	--({\sx*(1.3600)},{\sy*(-0.0000)})
	--({\sx*(1.3700)},{\sy*(-0.0000)})
	--({\sx*(1.3800)},{\sy*(-0.0000)})
	--({\sx*(1.3900)},{\sy*(-0.0000)})
	--({\sx*(1.4000)},{\sy*(-0.0000)})
	--({\sx*(1.4100)},{\sy*(-0.0000)})
	--({\sx*(1.4200)},{\sy*(-0.0000)})
	--({\sx*(1.4300)},{\sy*(-0.0000)})
	--({\sx*(1.4400)},{\sy*(-0.0000)})
	--({\sx*(1.4500)},{\sy*(-0.0000)})
	--({\sx*(1.4600)},{\sy*(-0.0000)})
	--({\sx*(1.4700)},{\sy*(-0.0000)})
	--({\sx*(1.4800)},{\sy*(-0.0000)})
	--({\sx*(1.4900)},{\sy*(-0.0000)})
	--({\sx*(1.5000)},{\sy*(-0.0000)})
	--({\sx*(1.5100)},{\sy*(-0.0000)})
	--({\sx*(1.5200)},{\sy*(-0.0000)})
	--({\sx*(1.5300)},{\sy*(-0.0000)})
	--({\sx*(1.5400)},{\sy*(-0.0000)})
	--({\sx*(1.5500)},{\sy*(-0.0000)})
	--({\sx*(1.5600)},{\sy*(-0.0000)})
	--({\sx*(1.5700)},{\sy*(0.0000)})
	--({\sx*(1.5800)},{\sy*(0.0000)})
	--({\sx*(1.5900)},{\sy*(0.0000)})
	--({\sx*(1.6000)},{\sy*(0.0000)})
	--({\sx*(1.6100)},{\sy*(0.0000)})
	--({\sx*(1.6200)},{\sy*(0.0000)})
	--({\sx*(1.6300)},{\sy*(0.0000)})
	--({\sx*(1.6400)},{\sy*(0.0000)})
	--({\sx*(1.6500)},{\sy*(0.0000)})
	--({\sx*(1.6600)},{\sy*(0.0000)})
	--({\sx*(1.6700)},{\sy*(0.0000)})
	--({\sx*(1.6800)},{\sy*(0.0000)})
	--({\sx*(1.6900)},{\sy*(0.0000)})
	--({\sx*(1.7000)},{\sy*(0.0000)})
	--({\sx*(1.7100)},{\sy*(0.0000)})
	--({\sx*(1.7200)},{\sy*(0.0000)})
	--({\sx*(1.7300)},{\sy*(0.0000)})
	--({\sx*(1.7400)},{\sy*(0.0000)})
	--({\sx*(1.7500)},{\sy*(0.0000)})
	--({\sx*(1.7600)},{\sy*(0.0000)})
	--({\sx*(1.7700)},{\sy*(0.0000)})
	--({\sx*(1.7800)},{\sy*(0.0000)})
	--({\sx*(1.7900)},{\sy*(0.0000)})
	--({\sx*(1.8000)},{\sy*(0.0000)})
	--({\sx*(1.8100)},{\sy*(0.0000)})
	--({\sx*(1.8200)},{\sy*(0.0000)})
	--({\sx*(1.8300)},{\sy*(0.0000)})
	--({\sx*(1.8400)},{\sy*(0.0000)})
	--({\sx*(1.8500)},{\sy*(0.0000)})
	--({\sx*(1.8600)},{\sy*(0.0000)})
	--({\sx*(1.8700)},{\sy*(0.0000)})
	--({\sx*(1.8800)},{\sy*(-0.0000)})
	--({\sx*(1.8900)},{\sy*(-0.0000)})
	--({\sx*(1.9000)},{\sy*(-0.0000)})
	--({\sx*(1.9100)},{\sy*(-0.0000)})
	--({\sx*(1.9200)},{\sy*(-0.0000)})
	--({\sx*(1.9300)},{\sy*(-0.0000)})
	--({\sx*(1.9400)},{\sy*(-0.0000)})
	--({\sx*(1.9500)},{\sy*(-0.0000)})
	--({\sx*(1.9600)},{\sy*(-0.0000)})
	--({\sx*(1.9700)},{\sy*(-0.0000)})
	--({\sx*(1.9800)},{\sy*(-0.0000)})
	--({\sx*(1.9900)},{\sy*(-0.0000)})
	--({\sx*(2.0000)},{\sy*(-0.0000)})
	--({\sx*(2.0100)},{\sy*(-0.0000)})
	--({\sx*(2.0200)},{\sy*(-0.0000)})
	--({\sx*(2.0300)},{\sy*(-0.0000)})
	--({\sx*(2.0400)},{\sy*(-0.0000)})
	--({\sx*(2.0500)},{\sy*(-0.0000)})
	--({\sx*(2.0600)},{\sy*(-0.0000)})
	--({\sx*(2.0700)},{\sy*(-0.0000)})
	--({\sx*(2.0800)},{\sy*(-0.0000)})
	--({\sx*(2.0900)},{\sy*(-0.0000)})
	--({\sx*(2.1000)},{\sy*(-0.0000)})
	--({\sx*(2.1100)},{\sy*(-0.0000)})
	--({\sx*(2.1200)},{\sy*(-0.0000)})
	--({\sx*(2.1300)},{\sy*(-0.0000)})
	--({\sx*(2.1400)},{\sy*(-0.0000)})
	--({\sx*(2.1500)},{\sy*(-0.0000)})
	--({\sx*(2.1600)},{\sy*(-0.0000)})
	--({\sx*(2.1700)},{\sy*(-0.0000)})
	--({\sx*(2.1800)},{\sy*(-0.0000)})
	--({\sx*(2.1900)},{\sy*(0.0000)})
	--({\sx*(2.2000)},{\sy*(0.0000)})
	--({\sx*(2.2100)},{\sy*(0.0000)})
	--({\sx*(2.2200)},{\sy*(0.0000)})
	--({\sx*(2.2300)},{\sy*(0.0000)})
	--({\sx*(2.2400)},{\sy*(0.0000)})
	--({\sx*(2.2500)},{\sy*(0.0000)})
	--({\sx*(2.2600)},{\sy*(0.0000)})
	--({\sx*(2.2700)},{\sy*(0.0000)})
	--({\sx*(2.2800)},{\sy*(0.0000)})
	--({\sx*(2.2900)},{\sy*(0.0000)})
	--({\sx*(2.3000)},{\sy*(0.0000)})
	--({\sx*(2.3100)},{\sy*(0.0000)})
	--({\sx*(2.3200)},{\sy*(0.0000)})
	--({\sx*(2.3300)},{\sy*(0.0000)})
	--({\sx*(2.3400)},{\sy*(0.0000)})
	--({\sx*(2.3500)},{\sy*(0.0000)})
	--({\sx*(2.3600)},{\sy*(0.0000)})
	--({\sx*(2.3700)},{\sy*(0.0000)})
	--({\sx*(2.3800)},{\sy*(0.0000)})
	--({\sx*(2.3900)},{\sy*(0.0000)})
	--({\sx*(2.4000)},{\sy*(0.0000)})
	--({\sx*(2.4100)},{\sy*(0.0000)})
	--({\sx*(2.4200)},{\sy*(0.0000)})
	--({\sx*(2.4300)},{\sy*(0.0000)})
	--({\sx*(2.4400)},{\sy*(0.0000)})
	--({\sx*(2.4500)},{\sy*(0.0000)})
	--({\sx*(2.4600)},{\sy*(0.0000)})
	--({\sx*(2.4700)},{\sy*(0.0000)})
	--({\sx*(2.4800)},{\sy*(0.0000)})
	--({\sx*(2.4900)},{\sy*(0.0000)})
	--({\sx*(2.5000)},{\sy*(0.0000)})
	--({\sx*(2.5100)},{\sy*(-0.0000)})
	--({\sx*(2.5200)},{\sy*(-0.0000)})
	--({\sx*(2.5300)},{\sy*(-0.0000)})
	--({\sx*(2.5400)},{\sy*(-0.0000)})
	--({\sx*(2.5500)},{\sy*(-0.0000)})
	--({\sx*(2.5600)},{\sy*(-0.0000)})
	--({\sx*(2.5700)},{\sy*(-0.0000)})
	--({\sx*(2.5800)},{\sy*(-0.0000)})
	--({\sx*(2.5900)},{\sy*(-0.0000)})
	--({\sx*(2.6000)},{\sy*(-0.0000)})
	--({\sx*(2.6100)},{\sy*(-0.0000)})
	--({\sx*(2.6200)},{\sy*(-0.0000)})
	--({\sx*(2.6300)},{\sy*(-0.0000)})
	--({\sx*(2.6400)},{\sy*(-0.0000)})
	--({\sx*(2.6500)},{\sy*(-0.0000)})
	--({\sx*(2.6600)},{\sy*(-0.0000)})
	--({\sx*(2.6700)},{\sy*(-0.0000)})
	--({\sx*(2.6800)},{\sy*(-0.0000)})
	--({\sx*(2.6900)},{\sy*(-0.0000)})
	--({\sx*(2.7000)},{\sy*(-0.0000)})
	--({\sx*(2.7100)},{\sy*(-0.0000)})
	--({\sx*(2.7200)},{\sy*(-0.0000)})
	--({\sx*(2.7300)},{\sy*(-0.0000)})
	--({\sx*(2.7400)},{\sy*(-0.0000)})
	--({\sx*(2.7500)},{\sy*(-0.0000)})
	--({\sx*(2.7600)},{\sy*(-0.0000)})
	--({\sx*(2.7700)},{\sy*(-0.0000)})
	--({\sx*(2.7800)},{\sy*(-0.0000)})
	--({\sx*(2.7900)},{\sy*(-0.0000)})
	--({\sx*(2.8000)},{\sy*(-0.0000)})
	--({\sx*(2.8100)},{\sy*(-0.0000)})
	--({\sx*(2.8200)},{\sy*(0.0000)})
	--({\sx*(2.8300)},{\sy*(0.0000)})
	--({\sx*(2.8400)},{\sy*(0.0000)})
	--({\sx*(2.8500)},{\sy*(0.0000)})
	--({\sx*(2.8600)},{\sy*(0.0000)})
	--({\sx*(2.8700)},{\sy*(0.0000)})
	--({\sx*(2.8800)},{\sy*(0.0000)})
	--({\sx*(2.8900)},{\sy*(0.0000)})
	--({\sx*(2.9000)},{\sy*(0.0000)})
	--({\sx*(2.9100)},{\sy*(0.0000)})
	--({\sx*(2.9200)},{\sy*(0.0000)})
	--({\sx*(2.9300)},{\sy*(0.0000)})
	--({\sx*(2.9400)},{\sy*(0.0000)})
	--({\sx*(2.9500)},{\sy*(0.0000)})
	--({\sx*(2.9600)},{\sy*(0.0000)})
	--({\sx*(2.9700)},{\sy*(0.0000)})
	--({\sx*(2.9800)},{\sy*(0.0000)})
	--({\sx*(2.9900)},{\sy*(0.0000)})
	--({\sx*(3.0000)},{\sy*(0.0000)})
	--({\sx*(3.0100)},{\sy*(0.0000)})
	--({\sx*(3.0200)},{\sy*(0.0000)})
	--({\sx*(3.0300)},{\sy*(0.0000)})
	--({\sx*(3.0400)},{\sy*(0.0000)})
	--({\sx*(3.0500)},{\sy*(0.0000)})
	--({\sx*(3.0600)},{\sy*(0.0000)})
	--({\sx*(3.0700)},{\sy*(0.0000)})
	--({\sx*(3.0800)},{\sy*(0.0000)})
	--({\sx*(3.0900)},{\sy*(0.0000)})
	--({\sx*(3.1000)},{\sy*(0.0000)})
	--({\sx*(3.1100)},{\sy*(0.0000)})
	--({\sx*(3.1200)},{\sy*(0.0000)})
	--({\sx*(3.1300)},{\sy*(-0.0000)})
	--({\sx*(3.1400)},{\sy*(-0.0000)})
	--({\sx*(3.1500)},{\sy*(-0.0000)})
	--({\sx*(3.1600)},{\sy*(-0.0000)})
	--({\sx*(3.1700)},{\sy*(-0.0000)})
	--({\sx*(3.1800)},{\sy*(-0.0000)})
	--({\sx*(3.1900)},{\sy*(-0.0000)})
	--({\sx*(3.2000)},{\sy*(-0.0000)})
	--({\sx*(3.2100)},{\sy*(-0.0000)})
	--({\sx*(3.2200)},{\sy*(-0.0000)})
	--({\sx*(3.2300)},{\sy*(-0.0000)})
	--({\sx*(3.2400)},{\sy*(-0.0000)})
	--({\sx*(3.2500)},{\sy*(-0.0000)})
	--({\sx*(3.2600)},{\sy*(-0.0000)})
	--({\sx*(3.2700)},{\sy*(-0.0000)})
	--({\sx*(3.2800)},{\sy*(-0.0000)})
	--({\sx*(3.2900)},{\sy*(-0.0000)})
	--({\sx*(3.3000)},{\sy*(-0.0000)})
	--({\sx*(3.3100)},{\sy*(-0.0000)})
	--({\sx*(3.3200)},{\sy*(-0.0000)})
	--({\sx*(3.3300)},{\sy*(-0.0000)})
	--({\sx*(3.3400)},{\sy*(-0.0000)})
	--({\sx*(3.3500)},{\sy*(-0.0000)})
	--({\sx*(3.3600)},{\sy*(-0.0000)})
	--({\sx*(3.3700)},{\sy*(-0.0000)})
	--({\sx*(3.3800)},{\sy*(-0.0000)})
	--({\sx*(3.3900)},{\sy*(-0.0000)})
	--({\sx*(3.4000)},{\sy*(-0.0000)})
	--({\sx*(3.4100)},{\sy*(-0.0000)})
	--({\sx*(3.4200)},{\sy*(-0.0000)})
	--({\sx*(3.4300)},{\sy*(-0.0000)})
	--({\sx*(3.4400)},{\sy*(0.0000)})
	--({\sx*(3.4500)},{\sy*(0.0000)})
	--({\sx*(3.4600)},{\sy*(0.0000)})
	--({\sx*(3.4700)},{\sy*(0.0000)})
	--({\sx*(3.4800)},{\sy*(0.0000)})
	--({\sx*(3.4900)},{\sy*(0.0000)})
	--({\sx*(3.5000)},{\sy*(0.0000)})
	--({\sx*(3.5100)},{\sy*(0.0000)})
	--({\sx*(3.5200)},{\sy*(0.0000)})
	--({\sx*(3.5300)},{\sy*(0.0000)})
	--({\sx*(3.5400)},{\sy*(0.0000)})
	--({\sx*(3.5500)},{\sy*(0.0000)})
	--({\sx*(3.5600)},{\sy*(0.0000)})
	--({\sx*(3.5700)},{\sy*(0.0000)})
	--({\sx*(3.5800)},{\sy*(0.0000)})
	--({\sx*(3.5900)},{\sy*(0.0000)})
	--({\sx*(3.6000)},{\sy*(0.0000)})
	--({\sx*(3.6100)},{\sy*(0.0000)})
	--({\sx*(3.6200)},{\sy*(0.0000)})
	--({\sx*(3.6300)},{\sy*(0.0000)})
	--({\sx*(3.6400)},{\sy*(0.0000)})
	--({\sx*(3.6500)},{\sy*(0.0000)})
	--({\sx*(3.6600)},{\sy*(0.0000)})
	--({\sx*(3.6700)},{\sy*(0.0000)})
	--({\sx*(3.6800)},{\sy*(0.0000)})
	--({\sx*(3.6900)},{\sy*(0.0000)})
	--({\sx*(3.7000)},{\sy*(0.0000)})
	--({\sx*(3.7100)},{\sy*(0.0000)})
	--({\sx*(3.7200)},{\sy*(0.0000)})
	--({\sx*(3.7300)},{\sy*(0.0000)})
	--({\sx*(3.7400)},{\sy*(0.0000)})
	--({\sx*(3.7500)},{\sy*(0.0000)})
	--({\sx*(3.7600)},{\sy*(-0.0000)})
	--({\sx*(3.7700)},{\sy*(-0.0000)})
	--({\sx*(3.7800)},{\sy*(-0.0000)})
	--({\sx*(3.7900)},{\sy*(-0.0000)})
	--({\sx*(3.8000)},{\sy*(-0.0000)})
	--({\sx*(3.8100)},{\sy*(-0.0000)})
	--({\sx*(3.8200)},{\sy*(-0.0000)})
	--({\sx*(3.8300)},{\sy*(-0.0000)})
	--({\sx*(3.8400)},{\sy*(-0.0000)})
	--({\sx*(3.8500)},{\sy*(-0.0000)})
	--({\sx*(3.8600)},{\sy*(-0.0000)})
	--({\sx*(3.8700)},{\sy*(-0.0000)})
	--({\sx*(3.8800)},{\sy*(-0.0000)})
	--({\sx*(3.8900)},{\sy*(-0.0000)})
	--({\sx*(3.9000)},{\sy*(-0.0000)})
	--({\sx*(3.9100)},{\sy*(-0.0000)})
	--({\sx*(3.9200)},{\sy*(-0.0000)})
	--({\sx*(3.9300)},{\sy*(-0.0000)})
	--({\sx*(3.9400)},{\sy*(-0.0000)})
	--({\sx*(3.9500)},{\sy*(-0.0000)})
	--({\sx*(3.9600)},{\sy*(-0.0000)})
	--({\sx*(3.9700)},{\sy*(-0.0000)})
	--({\sx*(3.9800)},{\sy*(-0.0000)})
	--({\sx*(3.9900)},{\sy*(-0.0000)})
	--({\sx*(4.0000)},{\sy*(-0.0000)})
	--({\sx*(4.0100)},{\sy*(-0.0000)})
	--({\sx*(4.0200)},{\sy*(-0.0000)})
	--({\sx*(4.0300)},{\sy*(-0.0000)})
	--({\sx*(4.0400)},{\sy*(-0.0000)})
	--({\sx*(4.0500)},{\sy*(-0.0000)})
	--({\sx*(4.0600)},{\sy*(-0.0000)})
	--({\sx*(4.0700)},{\sy*(0.0000)})
	--({\sx*(4.0800)},{\sy*(0.0000)})
	--({\sx*(4.0900)},{\sy*(0.0000)})
	--({\sx*(4.1000)},{\sy*(0.0001)})
	--({\sx*(4.1100)},{\sy*(0.0001)})
	--({\sx*(4.1200)},{\sy*(0.0001)})
	--({\sx*(4.1300)},{\sy*(0.0001)})
	--({\sx*(4.1400)},{\sy*(0.0002)})
	--({\sx*(4.1500)},{\sy*(0.0002)})
	--({\sx*(4.1600)},{\sy*(0.0003)})
	--({\sx*(4.1700)},{\sy*(0.0003)})
	--({\sx*(4.1800)},{\sy*(0.0003)})
	--({\sx*(4.1900)},{\sy*(0.0004)})
	--({\sx*(4.2000)},{\sy*(0.0004)})
	--({\sx*(4.2100)},{\sy*(0.0005)})
	--({\sx*(4.2200)},{\sy*(0.0005)})
	--({\sx*(4.2300)},{\sy*(0.0006)})
	--({\sx*(4.2400)},{\sy*(0.0006)})
	--({\sx*(4.2500)},{\sy*(0.0007)})
	--({\sx*(4.2600)},{\sy*(0.0007)})
	--({\sx*(4.2700)},{\sy*(0.0007)})
	--({\sx*(4.2800)},{\sy*(0.0007)})
	--({\sx*(4.2900)},{\sy*(0.0008)})
	--({\sx*(4.3000)},{\sy*(0.0008)})
	--({\sx*(4.3100)},{\sy*(0.0007)})
	--({\sx*(4.3200)},{\sy*(0.0007)})
	--({\sx*(4.3300)},{\sy*(0.0006)})
	--({\sx*(4.3400)},{\sy*(0.0006)})
	--({\sx*(4.3500)},{\sy*(0.0004)})
	--({\sx*(4.3600)},{\sy*(0.0003)})
	--({\sx*(4.3700)},{\sy*(0.0001)})
	--({\sx*(4.3800)},{\sy*(-0.0001)})
	--({\sx*(4.3900)},{\sy*(-0.0004)})
	--({\sx*(4.4000)},{\sy*(-0.0007)})
	--({\sx*(4.4100)},{\sy*(-0.0011)})
	--({\sx*(4.4200)},{\sy*(-0.0016)})
	--({\sx*(4.4300)},{\sy*(-0.0021)})
	--({\sx*(4.4400)},{\sy*(-0.0027)})
	--({\sx*(4.4500)},{\sy*(-0.0034)})
	--({\sx*(4.4600)},{\sy*(-0.0042)})
	--({\sx*(4.4700)},{\sy*(-0.0051)})
	--({\sx*(4.4800)},{\sy*(-0.0060)})
	--({\sx*(4.4900)},{\sy*(-0.0070)})
	--({\sx*(4.5000)},{\sy*(-0.0081)})
	--({\sx*(4.5100)},{\sy*(-0.0093)})
	--({\sx*(4.5200)},{\sy*(-0.0106)})
	--({\sx*(4.5300)},{\sy*(-0.0119)})
	--({\sx*(4.5400)},{\sy*(-0.0133)})
	--({\sx*(4.5500)},{\sy*(-0.0146)})
	--({\sx*(4.5600)},{\sy*(-0.0160)})
	--({\sx*(4.5700)},{\sy*(-0.0173)})
	--({\sx*(4.5800)},{\sy*(-0.0185)})
	--({\sx*(4.5900)},{\sy*(-0.0195)})
	--({\sx*(4.6000)},{\sy*(-0.0203)})
	--({\sx*(4.6100)},{\sy*(-0.0208)})
	--({\sx*(4.6200)},{\sy*(-0.0208)})
	--({\sx*(4.6300)},{\sy*(-0.0204)})
	--({\sx*(4.6400)},{\sy*(-0.0192)})
	--({\sx*(4.6500)},{\sy*(-0.0173)})
	--({\sx*(4.6600)},{\sy*(-0.0144)})
	--({\sx*(4.6700)},{\sy*(-0.0104)})
	--({\sx*(4.6800)},{\sy*(-0.0050)})
	--({\sx*(4.6900)},{\sy*(0.0019)})
	--({\sx*(4.7000)},{\sy*(0.0105)})
	--({\sx*(4.7100)},{\sy*(0.0209)})
	--({\sx*(4.7200)},{\sy*(0.0335)})
	--({\sx*(4.7300)},{\sy*(0.0482)})
	--({\sx*(4.7400)},{\sy*(0.0652)})
	--({\sx*(4.7500)},{\sy*(0.0845)})
	--({\sx*(4.7600)},{\sy*(0.1062)})
	--({\sx*(4.7700)},{\sy*(0.1301)})
	--({\sx*(4.7800)},{\sy*(0.1561)})
	--({\sx*(4.7900)},{\sy*(0.1839)})
	--({\sx*(4.8000)},{\sy*(0.2132)})
	--({\sx*(4.8100)},{\sy*(0.2437)})
	--({\sx*(4.8200)},{\sy*(0.2750)})
	--({\sx*(4.8300)},{\sy*(0.3066)})
	--({\sx*(4.8400)},{\sy*(0.3380)})
	--({\sx*(4.8500)},{\sy*(0.3689)})
	--({\sx*(4.8600)},{\sy*(0.3987)})
	--({\sx*(4.8700)},{\sy*(0.4269)})
	--({\sx*(4.8800)},{\sy*(0.4533)})
	--({\sx*(4.8900)},{\sy*(0.4774)})
	--({\sx*(4.9000)},{\sy*(0.4987)})
	--({\sx*(4.9100)},{\sy*(0.5167)})
	--({\sx*(4.9200)},{\sy*(0.5310)})
	--({\sx*(4.9300)},{\sy*(0.5407)})
	--({\sx*(4.9400)},{\sy*(0.5447)})
	--({\sx*(4.9500)},{\sy*(0.5411)})
	--({\sx*(4.9600)},{\sy*(0.5269)})
	--({\sx*(4.9700)},{\sy*(0.4961)})
	--({\sx*(4.9800)},{\sy*(0.4356)})
	--({\sx*(4.9900)},{\sy*(0.3118)})
	--({\sx*(5.0000)},{\sy*(0.0000)});
}
\def\xwertei{
\fill[color=red] (0.0000,0) circle[radius={0.07/\skala}];
\fill[color=red] (0.2778,0) circle[radius={0.07/\skala}];
\fill[color=red] (0.5556,0) circle[radius={0.07/\skala}];
\fill[color=red] (0.8333,0) circle[radius={0.07/\skala}];
\fill[color=red] (1.1111,0) circle[radius={0.07/\skala}];
\fill[color=red] (1.3889,0) circle[radius={0.07/\skala}];
\fill[color=red] (1.6667,0) circle[radius={0.07/\skala}];
\fill[color=red] (1.9444,0) circle[radius={0.07/\skala}];
\fill[color=red] (2.2222,0) circle[radius={0.07/\skala}];
\fill[color=red] (2.5000,0) circle[radius={0.07/\skala}];
\fill[color=red] (2.7778,0) circle[radius={0.07/\skala}];
\fill[color=red] (3.0556,0) circle[radius={0.07/\skala}];
\fill[color=red] (3.3333,0) circle[radius={0.07/\skala}];
\fill[color=red] (3.6111,0) circle[radius={0.07/\skala}];
\fill[color=red] (3.8889,0) circle[radius={0.07/\skala}];
\fill[color=red] (4.1667,0) circle[radius={0.07/\skala}];
\fill[color=red] (4.4444,0) circle[radius={0.07/\skala}];
\fill[color=red] (4.7222,0) circle[radius={0.07/\skala}];
\fill[color=red] (5.0000,0) circle[radius={0.07/\skala}];
}
\def\punktei{18}
\def\maxfehleri{5.738\cdot 10^{-7}}
\def\fehleri{
\draw[color=red,line width=1.4pt,line join=round] ({\sx*(0.000)},{\sy*(0.0000)})
	--({\sx*(0.0100)},{\sy*(0.1556)})
	--({\sx*(0.0200)},{\sy*(0.2753)})
	--({\sx*(0.0300)},{\sy*(0.3644)})
	--({\sx*(0.0400)},{\sy*(0.4277)})
	--({\sx*(0.0500)},{\sy*(0.4694)})
	--({\sx*(0.0600)},{\sy*(0.4932)})
	--({\sx*(0.0700)},{\sy*(0.5022)})
	--({\sx*(0.0800)},{\sy*(0.4993)})
	--({\sx*(0.0900)},{\sy*(0.4870)})
	--({\sx*(0.1000)},{\sy*(0.4673)})
	--({\sx*(0.1100)},{\sy*(0.4420)})
	--({\sx*(0.1200)},{\sy*(0.4128)})
	--({\sx*(0.1300)},{\sy*(0.3808)})
	--({\sx*(0.1400)},{\sy*(0.3472)})
	--({\sx*(0.1500)},{\sy*(0.3130)})
	--({\sx*(0.1600)},{\sy*(0.2788)})
	--({\sx*(0.1700)},{\sy*(0.2453)})
	--({\sx*(0.1800)},{\sy*(0.2129)})
	--({\sx*(0.1900)},{\sy*(0.1821)})
	--({\sx*(0.2000)},{\sy*(0.1530)})
	--({\sx*(0.2100)},{\sy*(0.1260)})
	--({\sx*(0.2200)},{\sy*(0.1011)})
	--({\sx*(0.2300)},{\sy*(0.0784)})
	--({\sx*(0.2400)},{\sy*(0.0579)})
	--({\sx*(0.2500)},{\sy*(0.0397)})
	--({\sx*(0.2600)},{\sy*(0.0236)})
	--({\sx*(0.2700)},{\sy*(0.0095)})
	--({\sx*(0.2800)},{\sy*(-0.0025)})
	--({\sx*(0.2900)},{\sy*(-0.0127)})
	--({\sx*(0.3000)},{\sy*(-0.0211)})
	--({\sx*(0.3100)},{\sy*(-0.0280)})
	--({\sx*(0.3200)},{\sy*(-0.0333)})
	--({\sx*(0.3300)},{\sy*(-0.0373)})
	--({\sx*(0.3400)},{\sy*(-0.0401)})
	--({\sx*(0.3500)},{\sy*(-0.0419)})
	--({\sx*(0.3600)},{\sy*(-0.0426)})
	--({\sx*(0.3700)},{\sy*(-0.0426)})
	--({\sx*(0.3800)},{\sy*(-0.0419)})
	--({\sx*(0.3900)},{\sy*(-0.0406)})
	--({\sx*(0.4000)},{\sy*(-0.0388)})
	--({\sx*(0.4100)},{\sy*(-0.0367)})
	--({\sx*(0.4200)},{\sy*(-0.0342)})
	--({\sx*(0.4300)},{\sy*(-0.0315)})
	--({\sx*(0.4400)},{\sy*(-0.0287)})
	--({\sx*(0.4500)},{\sy*(-0.0258)})
	--({\sx*(0.4600)},{\sy*(-0.0228)})
	--({\sx*(0.4700)},{\sy*(-0.0199)})
	--({\sx*(0.4800)},{\sy*(-0.0170)})
	--({\sx*(0.4900)},{\sy*(-0.0143)})
	--({\sx*(0.5000)},{\sy*(-0.0116)})
	--({\sx*(0.5100)},{\sy*(-0.0091)})
	--({\sx*(0.5200)},{\sy*(-0.0068)})
	--({\sx*(0.5300)},{\sy*(-0.0046)})
	--({\sx*(0.5400)},{\sy*(-0.0027)})
	--({\sx*(0.5500)},{\sy*(-0.0009)})
	--({\sx*(0.5600)},{\sy*(0.0007)})
	--({\sx*(0.5700)},{\sy*(0.0020)})
	--({\sx*(0.5800)},{\sy*(0.0032)})
	--({\sx*(0.5900)},{\sy*(0.0042)})
	--({\sx*(0.6000)},{\sy*(0.0050)})
	--({\sx*(0.6100)},{\sy*(0.0057)})
	--({\sx*(0.6200)},{\sy*(0.0062)})
	--({\sx*(0.6300)},{\sy*(0.0065)})
	--({\sx*(0.6400)},{\sy*(0.0067)})
	--({\sx*(0.6500)},{\sy*(0.0068)})
	--({\sx*(0.6600)},{\sy*(0.0068)})
	--({\sx*(0.6700)},{\sy*(0.0067)})
	--({\sx*(0.6800)},{\sy*(0.0065)})
	--({\sx*(0.6900)},{\sy*(0.0062)})
	--({\sx*(0.7000)},{\sy*(0.0059)})
	--({\sx*(0.7100)},{\sy*(0.0055)})
	--({\sx*(0.7200)},{\sy*(0.0050)})
	--({\sx*(0.7300)},{\sy*(0.0046)})
	--({\sx*(0.7400)},{\sy*(0.0041)})
	--({\sx*(0.7500)},{\sy*(0.0036)})
	--({\sx*(0.7600)},{\sy*(0.0031)})
	--({\sx*(0.7700)},{\sy*(0.0026)})
	--({\sx*(0.7800)},{\sy*(0.0022)})
	--({\sx*(0.7900)},{\sy*(0.0017)})
	--({\sx*(0.8000)},{\sy*(0.0013)})
	--({\sx*(0.8100)},{\sy*(0.0009)})
	--({\sx*(0.8200)},{\sy*(0.0005)})
	--({\sx*(0.8300)},{\sy*(0.0001)})
	--({\sx*(0.8400)},{\sy*(-0.0002)})
	--({\sx*(0.8500)},{\sy*(-0.0005)})
	--({\sx*(0.8600)},{\sy*(-0.0008)})
	--({\sx*(0.8700)},{\sy*(-0.0010)})
	--({\sx*(0.8800)},{\sy*(-0.0012)})
	--({\sx*(0.8900)},{\sy*(-0.0013)})
	--({\sx*(0.9000)},{\sy*(-0.0014)})
	--({\sx*(0.9100)},{\sy*(-0.0015)})
	--({\sx*(0.9200)},{\sy*(-0.0016)})
	--({\sx*(0.9300)},{\sy*(-0.0016)})
	--({\sx*(0.9400)},{\sy*(-0.0016)})
	--({\sx*(0.9500)},{\sy*(-0.0016)})
	--({\sx*(0.9600)},{\sy*(-0.0016)})
	--({\sx*(0.9700)},{\sy*(-0.0015)})
	--({\sx*(0.9800)},{\sy*(-0.0015)})
	--({\sx*(0.9900)},{\sy*(-0.0014)})
	--({\sx*(1.0000)},{\sy*(-0.0013)})
	--({\sx*(1.0100)},{\sy*(-0.0012)})
	--({\sx*(1.0200)},{\sy*(-0.0011)})
	--({\sx*(1.0300)},{\sy*(-0.0009)})
	--({\sx*(1.0400)},{\sy*(-0.0008)})
	--({\sx*(1.0500)},{\sy*(-0.0007)})
	--({\sx*(1.0600)},{\sy*(-0.0006)})
	--({\sx*(1.0700)},{\sy*(-0.0005)})
	--({\sx*(1.0800)},{\sy*(-0.0003)})
	--({\sx*(1.0900)},{\sy*(-0.0002)})
	--({\sx*(1.1000)},{\sy*(-0.0001)})
	--({\sx*(1.1100)},{\sy*(-0.0000)})
	--({\sx*(1.1200)},{\sy*(0.0001)})
	--({\sx*(1.1300)},{\sy*(0.0002)})
	--({\sx*(1.1400)},{\sy*(0.0002)})
	--({\sx*(1.1500)},{\sy*(0.0003)})
	--({\sx*(1.1600)},{\sy*(0.0004)})
	--({\sx*(1.1700)},{\sy*(0.0004)})
	--({\sx*(1.1800)},{\sy*(0.0005)})
	--({\sx*(1.1900)},{\sy*(0.0005)})
	--({\sx*(1.2000)},{\sy*(0.0005)})
	--({\sx*(1.2100)},{\sy*(0.0005)})
	--({\sx*(1.2200)},{\sy*(0.0005)})
	--({\sx*(1.2300)},{\sy*(0.0005)})
	--({\sx*(1.2400)},{\sy*(0.0005)})
	--({\sx*(1.2500)},{\sy*(0.0005)})
	--({\sx*(1.2600)},{\sy*(0.0005)})
	--({\sx*(1.2700)},{\sy*(0.0005)})
	--({\sx*(1.2800)},{\sy*(0.0004)})
	--({\sx*(1.2900)},{\sy*(0.0004)})
	--({\sx*(1.3000)},{\sy*(0.0004)})
	--({\sx*(1.3100)},{\sy*(0.0003)})
	--({\sx*(1.3200)},{\sy*(0.0003)})
	--({\sx*(1.3300)},{\sy*(0.0002)})
	--({\sx*(1.3400)},{\sy*(0.0002)})
	--({\sx*(1.3500)},{\sy*(0.0002)})
	--({\sx*(1.3600)},{\sy*(0.0001)})
	--({\sx*(1.3700)},{\sy*(0.0001)})
	--({\sx*(1.3800)},{\sy*(0.0000)})
	--({\sx*(1.3900)},{\sy*(-0.0000)})
	--({\sx*(1.4000)},{\sy*(-0.0000)})
	--({\sx*(1.4100)},{\sy*(-0.0001)})
	--({\sx*(1.4200)},{\sy*(-0.0001)})
	--({\sx*(1.4300)},{\sy*(-0.0001)})
	--({\sx*(1.4400)},{\sy*(-0.0002)})
	--({\sx*(1.4500)},{\sy*(-0.0002)})
	--({\sx*(1.4600)},{\sy*(-0.0002)})
	--({\sx*(1.4700)},{\sy*(-0.0002)})
	--({\sx*(1.4800)},{\sy*(-0.0002)})
	--({\sx*(1.4900)},{\sy*(-0.0002)})
	--({\sx*(1.5000)},{\sy*(-0.0002)})
	--({\sx*(1.5100)},{\sy*(-0.0002)})
	--({\sx*(1.5200)},{\sy*(-0.0002)})
	--({\sx*(1.5300)},{\sy*(-0.0002)})
	--({\sx*(1.5400)},{\sy*(-0.0002)})
	--({\sx*(1.5500)},{\sy*(-0.0002)})
	--({\sx*(1.5600)},{\sy*(-0.0002)})
	--({\sx*(1.5700)},{\sy*(-0.0002)})
	--({\sx*(1.5800)},{\sy*(-0.0002)})
	--({\sx*(1.5900)},{\sy*(-0.0001)})
	--({\sx*(1.6000)},{\sy*(-0.0001)})
	--({\sx*(1.6100)},{\sy*(-0.0001)})
	--({\sx*(1.6200)},{\sy*(-0.0001)})
	--({\sx*(1.6300)},{\sy*(-0.0001)})
	--({\sx*(1.6400)},{\sy*(-0.0001)})
	--({\sx*(1.6500)},{\sy*(-0.0000)})
	--({\sx*(1.6600)},{\sy*(-0.0000)})
	--({\sx*(1.6700)},{\sy*(0.0000)})
	--({\sx*(1.6800)},{\sy*(0.0000)})
	--({\sx*(1.6900)},{\sy*(0.0000)})
	--({\sx*(1.7000)},{\sy*(0.0001)})
	--({\sx*(1.7100)},{\sy*(0.0001)})
	--({\sx*(1.7200)},{\sy*(0.0001)})
	--({\sx*(1.7300)},{\sy*(0.0001)})
	--({\sx*(1.7400)},{\sy*(0.0001)})
	--({\sx*(1.7500)},{\sy*(0.0001)})
	--({\sx*(1.7600)},{\sy*(0.0001)})
	--({\sx*(1.7700)},{\sy*(0.0001)})
	--({\sx*(1.7800)},{\sy*(0.0001)})
	--({\sx*(1.7900)},{\sy*(0.0001)})
	--({\sx*(1.8000)},{\sy*(0.0001)})
	--({\sx*(1.8100)},{\sy*(0.0001)})
	--({\sx*(1.8200)},{\sy*(0.0001)})
	--({\sx*(1.8300)},{\sy*(0.0001)})
	--({\sx*(1.8400)},{\sy*(0.0001)})
	--({\sx*(1.8500)},{\sy*(0.0001)})
	--({\sx*(1.8600)},{\sy*(0.0001)})
	--({\sx*(1.8700)},{\sy*(0.0001)})
	--({\sx*(1.8800)},{\sy*(0.0001)})
	--({\sx*(1.8900)},{\sy*(0.0001)})
	--({\sx*(1.9000)},{\sy*(0.0001)})
	--({\sx*(1.9100)},{\sy*(0.0000)})
	--({\sx*(1.9200)},{\sy*(0.0000)})
	--({\sx*(1.9300)},{\sy*(0.0000)})
	--({\sx*(1.9400)},{\sy*(0.0000)})
	--({\sx*(1.9500)},{\sy*(-0.0000)})
	--({\sx*(1.9600)},{\sy*(-0.0000)})
	--({\sx*(1.9700)},{\sy*(-0.0000)})
	--({\sx*(1.9800)},{\sy*(-0.0000)})
	--({\sx*(1.9900)},{\sy*(-0.0000)})
	--({\sx*(2.0000)},{\sy*(-0.0001)})
	--({\sx*(2.0100)},{\sy*(-0.0001)})
	--({\sx*(2.0200)},{\sy*(-0.0001)})
	--({\sx*(2.0300)},{\sy*(-0.0001)})
	--({\sx*(2.0400)},{\sy*(-0.0001)})
	--({\sx*(2.0500)},{\sy*(-0.0001)})
	--({\sx*(2.0600)},{\sy*(-0.0001)})
	--({\sx*(2.0700)},{\sy*(-0.0001)})
	--({\sx*(2.0800)},{\sy*(-0.0001)})
	--({\sx*(2.0900)},{\sy*(-0.0001)})
	--({\sx*(2.1000)},{\sy*(-0.0001)})
	--({\sx*(2.1100)},{\sy*(-0.0001)})
	--({\sx*(2.1200)},{\sy*(-0.0001)})
	--({\sx*(2.1300)},{\sy*(-0.0001)})
	--({\sx*(2.1400)},{\sy*(-0.0001)})
	--({\sx*(2.1500)},{\sy*(-0.0001)})
	--({\sx*(2.1600)},{\sy*(-0.0001)})
	--({\sx*(2.1700)},{\sy*(-0.0000)})
	--({\sx*(2.1800)},{\sy*(-0.0000)})
	--({\sx*(2.1900)},{\sy*(-0.0000)})
	--({\sx*(2.2000)},{\sy*(-0.0000)})
	--({\sx*(2.2100)},{\sy*(-0.0000)})
	--({\sx*(2.2200)},{\sy*(-0.0000)})
	--({\sx*(2.2300)},{\sy*(0.0000)})
	--({\sx*(2.2400)},{\sy*(0.0000)})
	--({\sx*(2.2500)},{\sy*(0.0000)})
	--({\sx*(2.2600)},{\sy*(0.0000)})
	--({\sx*(2.2700)},{\sy*(0.0000)})
	--({\sx*(2.2800)},{\sy*(0.0000)})
	--({\sx*(2.2900)},{\sy*(0.0001)})
	--({\sx*(2.3000)},{\sy*(0.0001)})
	--({\sx*(2.3100)},{\sy*(0.0001)})
	--({\sx*(2.3200)},{\sy*(0.0001)})
	--({\sx*(2.3300)},{\sy*(0.0001)})
	--({\sx*(2.3400)},{\sy*(0.0001)})
	--({\sx*(2.3500)},{\sy*(0.0001)})
	--({\sx*(2.3600)},{\sy*(0.0001)})
	--({\sx*(2.3700)},{\sy*(0.0001)})
	--({\sx*(2.3800)},{\sy*(0.0001)})
	--({\sx*(2.3900)},{\sy*(0.0001)})
	--({\sx*(2.4000)},{\sy*(0.0001)})
	--({\sx*(2.4100)},{\sy*(0.0001)})
	--({\sx*(2.4200)},{\sy*(0.0001)})
	--({\sx*(2.4300)},{\sy*(0.0001)})
	--({\sx*(2.4400)},{\sy*(0.0000)})
	--({\sx*(2.4500)},{\sy*(0.0000)})
	--({\sx*(2.4600)},{\sy*(0.0000)})
	--({\sx*(2.4700)},{\sy*(0.0000)})
	--({\sx*(2.4800)},{\sy*(0.0000)})
	--({\sx*(2.4900)},{\sy*(0.0000)})
	--({\sx*(2.5000)},{\sy*(0.0000)})
	--({\sx*(2.5100)},{\sy*(-0.0000)})
	--({\sx*(2.5200)},{\sy*(-0.0000)})
	--({\sx*(2.5300)},{\sy*(-0.0000)})
	--({\sx*(2.5400)},{\sy*(-0.0000)})
	--({\sx*(2.5500)},{\sy*(-0.0000)})
	--({\sx*(2.5600)},{\sy*(-0.0000)})
	--({\sx*(2.5700)},{\sy*(-0.0001)})
	--({\sx*(2.5800)},{\sy*(-0.0001)})
	--({\sx*(2.5900)},{\sy*(-0.0001)})
	--({\sx*(2.6000)},{\sy*(-0.0001)})
	--({\sx*(2.6100)},{\sy*(-0.0001)})
	--({\sx*(2.6200)},{\sy*(-0.0001)})
	--({\sx*(2.6300)},{\sy*(-0.0001)})
	--({\sx*(2.6400)},{\sy*(-0.0001)})
	--({\sx*(2.6500)},{\sy*(-0.0001)})
	--({\sx*(2.6600)},{\sy*(-0.0001)})
	--({\sx*(2.6700)},{\sy*(-0.0001)})
	--({\sx*(2.6800)},{\sy*(-0.0001)})
	--({\sx*(2.6900)},{\sy*(-0.0001)})
	--({\sx*(2.7000)},{\sy*(-0.0001)})
	--({\sx*(2.7100)},{\sy*(-0.0001)})
	--({\sx*(2.7200)},{\sy*(-0.0000)})
	--({\sx*(2.7300)},{\sy*(-0.0000)})
	--({\sx*(2.7400)},{\sy*(-0.0000)})
	--({\sx*(2.7500)},{\sy*(-0.0000)})
	--({\sx*(2.7600)},{\sy*(-0.0000)})
	--({\sx*(2.7700)},{\sy*(-0.0000)})
	--({\sx*(2.7800)},{\sy*(0.0000)})
	--({\sx*(2.7900)},{\sy*(0.0000)})
	--({\sx*(2.8000)},{\sy*(0.0000)})
	--({\sx*(2.8100)},{\sy*(0.0000)})
	--({\sx*(2.8200)},{\sy*(0.0000)})
	--({\sx*(2.8300)},{\sy*(0.0000)})
	--({\sx*(2.8400)},{\sy*(0.0001)})
	--({\sx*(2.8500)},{\sy*(0.0001)})
	--({\sx*(2.8600)},{\sy*(0.0001)})
	--({\sx*(2.8700)},{\sy*(0.0001)})
	--({\sx*(2.8800)},{\sy*(0.0001)})
	--({\sx*(2.8900)},{\sy*(0.0001)})
	--({\sx*(2.9000)},{\sy*(0.0001)})
	--({\sx*(2.9100)},{\sy*(0.0001)})
	--({\sx*(2.9200)},{\sy*(0.0001)})
	--({\sx*(2.9300)},{\sy*(0.0001)})
	--({\sx*(2.9400)},{\sy*(0.0001)})
	--({\sx*(2.9500)},{\sy*(0.0001)})
	--({\sx*(2.9600)},{\sy*(0.0001)})
	--({\sx*(2.9700)},{\sy*(0.0001)})
	--({\sx*(2.9800)},{\sy*(0.0001)})
	--({\sx*(2.9900)},{\sy*(0.0001)})
	--({\sx*(3.0000)},{\sy*(0.0001)})
	--({\sx*(3.0100)},{\sy*(0.0001)})
	--({\sx*(3.0200)},{\sy*(0.0000)})
	--({\sx*(3.0300)},{\sy*(0.0000)})
	--({\sx*(3.0400)},{\sy*(0.0000)})
	--({\sx*(3.0500)},{\sy*(0.0000)})
	--({\sx*(3.0600)},{\sy*(-0.0000)})
	--({\sx*(3.0700)},{\sy*(-0.0000)})
	--({\sx*(3.0800)},{\sy*(-0.0000)})
	--({\sx*(3.0900)},{\sy*(-0.0000)})
	--({\sx*(3.1000)},{\sy*(-0.0001)})
	--({\sx*(3.1100)},{\sy*(-0.0001)})
	--({\sx*(3.1200)},{\sy*(-0.0001)})
	--({\sx*(3.1300)},{\sy*(-0.0001)})
	--({\sx*(3.1400)},{\sy*(-0.0001)})
	--({\sx*(3.1500)},{\sy*(-0.0001)})
	--({\sx*(3.1600)},{\sy*(-0.0001)})
	--({\sx*(3.1700)},{\sy*(-0.0001)})
	--({\sx*(3.1800)},{\sy*(-0.0001)})
	--({\sx*(3.1900)},{\sy*(-0.0001)})
	--({\sx*(3.2000)},{\sy*(-0.0001)})
	--({\sx*(3.2100)},{\sy*(-0.0001)})
	--({\sx*(3.2200)},{\sy*(-0.0001)})
	--({\sx*(3.2300)},{\sy*(-0.0001)})
	--({\sx*(3.2400)},{\sy*(-0.0001)})
	--({\sx*(3.2500)},{\sy*(-0.0001)})
	--({\sx*(3.2600)},{\sy*(-0.0001)})
	--({\sx*(3.2700)},{\sy*(-0.0001)})
	--({\sx*(3.2800)},{\sy*(-0.0001)})
	--({\sx*(3.2900)},{\sy*(-0.0001)})
	--({\sx*(3.3000)},{\sy*(-0.0001)})
	--({\sx*(3.3100)},{\sy*(-0.0000)})
	--({\sx*(3.3200)},{\sy*(-0.0000)})
	--({\sx*(3.3300)},{\sy*(-0.0000)})
	--({\sx*(3.3400)},{\sy*(0.0000)})
	--({\sx*(3.3500)},{\sy*(0.0000)})
	--({\sx*(3.3600)},{\sy*(0.0001)})
	--({\sx*(3.3700)},{\sy*(0.0001)})
	--({\sx*(3.3800)},{\sy*(0.0001)})
	--({\sx*(3.3900)},{\sy*(0.0001)})
	--({\sx*(3.4000)},{\sy*(0.0002)})
	--({\sx*(3.4100)},{\sy*(0.0002)})
	--({\sx*(3.4200)},{\sy*(0.0002)})
	--({\sx*(3.4300)},{\sy*(0.0002)})
	--({\sx*(3.4400)},{\sy*(0.0002)})
	--({\sx*(3.4500)},{\sy*(0.0002)})
	--({\sx*(3.4600)},{\sy*(0.0003)})
	--({\sx*(3.4700)},{\sy*(0.0003)})
	--({\sx*(3.4800)},{\sy*(0.0003)})
	--({\sx*(3.4900)},{\sy*(0.0003)})
	--({\sx*(3.5000)},{\sy*(0.0003)})
	--({\sx*(3.5100)},{\sy*(0.0003)})
	--({\sx*(3.5200)},{\sy*(0.0003)})
	--({\sx*(3.5300)},{\sy*(0.0003)})
	--({\sx*(3.5400)},{\sy*(0.0002)})
	--({\sx*(3.5500)},{\sy*(0.0002)})
	--({\sx*(3.5600)},{\sy*(0.0002)})
	--({\sx*(3.5700)},{\sy*(0.0002)})
	--({\sx*(3.5800)},{\sy*(0.0001)})
	--({\sx*(3.5900)},{\sy*(0.0001)})
	--({\sx*(3.6000)},{\sy*(0.0000)})
	--({\sx*(3.6100)},{\sy*(0.0000)})
	--({\sx*(3.6200)},{\sy*(-0.0000)})
	--({\sx*(3.6300)},{\sy*(-0.0001)})
	--({\sx*(3.6400)},{\sy*(-0.0001)})
	--({\sx*(3.6500)},{\sy*(-0.0002)})
	--({\sx*(3.6600)},{\sy*(-0.0003)})
	--({\sx*(3.6700)},{\sy*(-0.0003)})
	--({\sx*(3.6800)},{\sy*(-0.0004)})
	--({\sx*(3.6900)},{\sy*(-0.0004)})
	--({\sx*(3.7000)},{\sy*(-0.0005)})
	--({\sx*(3.7100)},{\sy*(-0.0005)})
	--({\sx*(3.7200)},{\sy*(-0.0006)})
	--({\sx*(3.7300)},{\sy*(-0.0006)})
	--({\sx*(3.7400)},{\sy*(-0.0006)})
	--({\sx*(3.7500)},{\sy*(-0.0007)})
	--({\sx*(3.7600)},{\sy*(-0.0007)})
	--({\sx*(3.7700)},{\sy*(-0.0007)})
	--({\sx*(3.7800)},{\sy*(-0.0007)})
	--({\sx*(3.7900)},{\sy*(-0.0007)})
	--({\sx*(3.8000)},{\sy*(-0.0007)})
	--({\sx*(3.8100)},{\sy*(-0.0006)})
	--({\sx*(3.8200)},{\sy*(-0.0006)})
	--({\sx*(3.8300)},{\sy*(-0.0006)})
	--({\sx*(3.8400)},{\sy*(-0.0005)})
	--({\sx*(3.8500)},{\sy*(-0.0004)})
	--({\sx*(3.8600)},{\sy*(-0.0003)})
	--({\sx*(3.8700)},{\sy*(-0.0002)})
	--({\sx*(3.8800)},{\sy*(-0.0001)})
	--({\sx*(3.8900)},{\sy*(0.0000)})
	--({\sx*(3.9000)},{\sy*(0.0001)})
	--({\sx*(3.9100)},{\sy*(0.0003)})
	--({\sx*(3.9200)},{\sy*(0.0004)})
	--({\sx*(3.9300)},{\sy*(0.0006)})
	--({\sx*(3.9400)},{\sy*(0.0008)})
	--({\sx*(3.9500)},{\sy*(0.0009)})
	--({\sx*(3.9600)},{\sy*(0.0011)})
	--({\sx*(3.9700)},{\sy*(0.0013)})
	--({\sx*(3.9800)},{\sy*(0.0015)})
	--({\sx*(3.9900)},{\sy*(0.0016)})
	--({\sx*(4.0000)},{\sy*(0.0018)})
	--({\sx*(4.0100)},{\sy*(0.0019)})
	--({\sx*(4.0200)},{\sy*(0.0020)})
	--({\sx*(4.0300)},{\sy*(0.0021)})
	--({\sx*(4.0400)},{\sy*(0.0022)})
	--({\sx*(4.0500)},{\sy*(0.0023)})
	--({\sx*(4.0600)},{\sy*(0.0023)})
	--({\sx*(4.0700)},{\sy*(0.0023)})
	--({\sx*(4.0800)},{\sy*(0.0022)})
	--({\sx*(4.0900)},{\sy*(0.0022)})
	--({\sx*(4.1000)},{\sy*(0.0020)})
	--({\sx*(4.1100)},{\sy*(0.0019)})
	--({\sx*(4.1200)},{\sy*(0.0017)})
	--({\sx*(4.1300)},{\sy*(0.0014)})
	--({\sx*(4.1400)},{\sy*(0.0011)})
	--({\sx*(4.1500)},{\sy*(0.0007)})
	--({\sx*(4.1600)},{\sy*(0.0003)})
	--({\sx*(4.1700)},{\sy*(-0.0002)})
	--({\sx*(4.1800)},{\sy*(-0.0007)})
	--({\sx*(4.1900)},{\sy*(-0.0012)})
	--({\sx*(4.2000)},{\sy*(-0.0019)})
	--({\sx*(4.2100)},{\sy*(-0.0025)})
	--({\sx*(4.2200)},{\sy*(-0.0032)})
	--({\sx*(4.2300)},{\sy*(-0.0039)})
	--({\sx*(4.2400)},{\sy*(-0.0046)})
	--({\sx*(4.2500)},{\sy*(-0.0054)})
	--({\sx*(4.2600)},{\sy*(-0.0061)})
	--({\sx*(4.2700)},{\sy*(-0.0069)})
	--({\sx*(4.2800)},{\sy*(-0.0076)})
	--({\sx*(4.2900)},{\sy*(-0.0082)})
	--({\sx*(4.3000)},{\sy*(-0.0089)})
	--({\sx*(4.3100)},{\sy*(-0.0094)})
	--({\sx*(4.3200)},{\sy*(-0.0099)})
	--({\sx*(4.3300)},{\sy*(-0.0102)})
	--({\sx*(4.3400)},{\sy*(-0.0104)})
	--({\sx*(4.3500)},{\sy*(-0.0105)})
	--({\sx*(4.3600)},{\sy*(-0.0104)})
	--({\sx*(4.3700)},{\sy*(-0.0101)})
	--({\sx*(4.3800)},{\sy*(-0.0096)})
	--({\sx*(4.3900)},{\sy*(-0.0089)})
	--({\sx*(4.4000)},{\sy*(-0.0079)})
	--({\sx*(4.4100)},{\sy*(-0.0067)})
	--({\sx*(4.4200)},{\sy*(-0.0051)})
	--({\sx*(4.4300)},{\sy*(-0.0033)})
	--({\sx*(4.4400)},{\sy*(-0.0011)})
	--({\sx*(4.4500)},{\sy*(0.0014)})
	--({\sx*(4.4600)},{\sy*(0.0043)})
	--({\sx*(4.4700)},{\sy*(0.0075)})
	--({\sx*(4.4800)},{\sy*(0.0110)})
	--({\sx*(4.4900)},{\sy*(0.0148)})
	--({\sx*(4.5000)},{\sy*(0.0189)})
	--({\sx*(4.5100)},{\sy*(0.0234)})
	--({\sx*(4.5200)},{\sy*(0.0280)})
	--({\sx*(4.5300)},{\sy*(0.0329)})
	--({\sx*(4.5400)},{\sy*(0.0379)})
	--({\sx*(4.5500)},{\sy*(0.0429)})
	--({\sx*(4.5600)},{\sy*(0.0480)})
	--({\sx*(4.5700)},{\sy*(0.0530)})
	--({\sx*(4.5800)},{\sy*(0.0577)})
	--({\sx*(4.5900)},{\sy*(0.0622)})
	--({\sx*(4.6000)},{\sy*(0.0661)})
	--({\sx*(4.6100)},{\sy*(0.0694)})
	--({\sx*(4.6200)},{\sy*(0.0720)})
	--({\sx*(4.6300)},{\sy*(0.0735)})
	--({\sx*(4.6400)},{\sy*(0.0739)})
	--({\sx*(4.6500)},{\sy*(0.0728)})
	--({\sx*(4.6600)},{\sy*(0.0701)})
	--({\sx*(4.6700)},{\sy*(0.0655)})
	--({\sx*(4.6800)},{\sy*(0.0587)})
	--({\sx*(4.6900)},{\sy*(0.0495)})
	--({\sx*(4.7000)},{\sy*(0.0376)})
	--({\sx*(4.7100)},{\sy*(0.0227)})
	--({\sx*(4.7200)},{\sy*(0.0045)})
	--({\sx*(4.7300)},{\sy*(-0.0172)})
	--({\sx*(4.7400)},{\sy*(-0.0427)})
	--({\sx*(4.7500)},{\sy*(-0.0722)})
	--({\sx*(4.7600)},{\sy*(-0.1060)})
	--({\sx*(4.7700)},{\sy*(-0.1441)})
	--({\sx*(4.7800)},{\sy*(-0.1867)})
	--({\sx*(4.7900)},{\sy*(-0.2338)})
	--({\sx*(4.8000)},{\sy*(-0.2853)})
	--({\sx*(4.8100)},{\sy*(-0.3411)})
	--({\sx*(4.8200)},{\sy*(-0.4008)})
	--({\sx*(4.8300)},{\sy*(-0.4640)})
	--({\sx*(4.8400)},{\sy*(-0.5301)})
	--({\sx*(4.8500)},{\sy*(-0.5981)})
	--({\sx*(4.8600)},{\sy*(-0.6668)})
	--({\sx*(4.8700)},{\sy*(-0.7350)})
	--({\sx*(4.8800)},{\sy*(-0.8008)})
	--({\sx*(4.8900)},{\sy*(-0.8620)})
	--({\sx*(4.9000)},{\sy*(-0.9160)})
	--({\sx*(4.9100)},{\sy*(-0.9596)})
	--({\sx*(4.9200)},{\sy*(-0.9891)})
	--({\sx*(4.9300)},{\sy*(-1.0000)})
	--({\sx*(4.9400)},{\sy*(-0.9872)})
	--({\sx*(4.9500)},{\sy*(-0.9447)})
	--({\sx*(4.9600)},{\sy*(-0.8654)})
	--({\sx*(4.9700)},{\sy*(-0.7413)})
	--({\sx*(4.9800)},{\sy*(-0.5630)})
	--({\sx*(4.9900)},{\sy*(-0.3200)})
	--({\sx*(5.0000)},{\sy*(0.0000)});
}
\def\relfehleri{
\draw[color=blue,line width=1.4pt,line join=round] ({\sx*(0.000)},{\sy*(0.0000)})
	--({\sx*(0.0100)},{\sy*(0.0000)})
	--({\sx*(0.0200)},{\sy*(0.0000)})
	--({\sx*(0.0300)},{\sy*(0.0000)})
	--({\sx*(0.0400)},{\sy*(0.0000)})
	--({\sx*(0.0500)},{\sy*(0.0000)})
	--({\sx*(0.0600)},{\sy*(0.0000)})
	--({\sx*(0.0700)},{\sy*(0.0000)})
	--({\sx*(0.0800)},{\sy*(0.0000)})
	--({\sx*(0.0900)},{\sy*(0.0000)})
	--({\sx*(0.1000)},{\sy*(0.0000)})
	--({\sx*(0.1100)},{\sy*(0.0000)})
	--({\sx*(0.1200)},{\sy*(0.0000)})
	--({\sx*(0.1300)},{\sy*(0.0000)})
	--({\sx*(0.1400)},{\sy*(0.0000)})
	--({\sx*(0.1500)},{\sy*(0.0000)})
	--({\sx*(0.1600)},{\sy*(0.0000)})
	--({\sx*(0.1700)},{\sy*(0.0000)})
	--({\sx*(0.1800)},{\sy*(0.0000)})
	--({\sx*(0.1900)},{\sy*(0.0000)})
	--({\sx*(0.2000)},{\sy*(0.0000)})
	--({\sx*(0.2100)},{\sy*(0.0000)})
	--({\sx*(0.2200)},{\sy*(0.0000)})
	--({\sx*(0.2300)},{\sy*(0.0000)})
	--({\sx*(0.2400)},{\sy*(0.0000)})
	--({\sx*(0.2500)},{\sy*(0.0000)})
	--({\sx*(0.2600)},{\sy*(0.0000)})
	--({\sx*(0.2700)},{\sy*(0.0000)})
	--({\sx*(0.2800)},{\sy*(-0.0000)})
	--({\sx*(0.2900)},{\sy*(-0.0000)})
	--({\sx*(0.3000)},{\sy*(-0.0000)})
	--({\sx*(0.3100)},{\sy*(-0.0000)})
	--({\sx*(0.3200)},{\sy*(-0.0000)})
	--({\sx*(0.3300)},{\sy*(-0.0000)})
	--({\sx*(0.3400)},{\sy*(-0.0000)})
	--({\sx*(0.3500)},{\sy*(-0.0000)})
	--({\sx*(0.3600)},{\sy*(-0.0000)})
	--({\sx*(0.3700)},{\sy*(-0.0000)})
	--({\sx*(0.3800)},{\sy*(-0.0000)})
	--({\sx*(0.3900)},{\sy*(-0.0000)})
	--({\sx*(0.4000)},{\sy*(-0.0000)})
	--({\sx*(0.4100)},{\sy*(-0.0000)})
	--({\sx*(0.4200)},{\sy*(-0.0000)})
	--({\sx*(0.4300)},{\sy*(-0.0000)})
	--({\sx*(0.4400)},{\sy*(-0.0000)})
	--({\sx*(0.4500)},{\sy*(-0.0000)})
	--({\sx*(0.4600)},{\sy*(-0.0000)})
	--({\sx*(0.4700)},{\sy*(-0.0000)})
	--({\sx*(0.4800)},{\sy*(-0.0000)})
	--({\sx*(0.4900)},{\sy*(-0.0000)})
	--({\sx*(0.5000)},{\sy*(-0.0000)})
	--({\sx*(0.5100)},{\sy*(-0.0000)})
	--({\sx*(0.5200)},{\sy*(-0.0000)})
	--({\sx*(0.5300)},{\sy*(-0.0000)})
	--({\sx*(0.5400)},{\sy*(-0.0000)})
	--({\sx*(0.5500)},{\sy*(-0.0000)})
	--({\sx*(0.5600)},{\sy*(0.0000)})
	--({\sx*(0.5700)},{\sy*(0.0000)})
	--({\sx*(0.5800)},{\sy*(0.0000)})
	--({\sx*(0.5900)},{\sy*(0.0000)})
	--({\sx*(0.6000)},{\sy*(0.0000)})
	--({\sx*(0.6100)},{\sy*(0.0000)})
	--({\sx*(0.6200)},{\sy*(0.0000)})
	--({\sx*(0.6300)},{\sy*(0.0000)})
	--({\sx*(0.6400)},{\sy*(0.0000)})
	--({\sx*(0.6500)},{\sy*(0.0000)})
	--({\sx*(0.6600)},{\sy*(0.0000)})
	--({\sx*(0.6700)},{\sy*(0.0000)})
	--({\sx*(0.6800)},{\sy*(0.0000)})
	--({\sx*(0.6900)},{\sy*(0.0000)})
	--({\sx*(0.7000)},{\sy*(0.0000)})
	--({\sx*(0.7100)},{\sy*(0.0000)})
	--({\sx*(0.7200)},{\sy*(0.0000)})
	--({\sx*(0.7300)},{\sy*(0.0000)})
	--({\sx*(0.7400)},{\sy*(0.0000)})
	--({\sx*(0.7500)},{\sy*(0.0000)})
	--({\sx*(0.7600)},{\sy*(0.0000)})
	--({\sx*(0.7700)},{\sy*(0.0000)})
	--({\sx*(0.7800)},{\sy*(0.0000)})
	--({\sx*(0.7900)},{\sy*(0.0000)})
	--({\sx*(0.8000)},{\sy*(0.0000)})
	--({\sx*(0.8100)},{\sy*(0.0000)})
	--({\sx*(0.8200)},{\sy*(0.0000)})
	--({\sx*(0.8300)},{\sy*(0.0000)})
	--({\sx*(0.8400)},{\sy*(-0.0000)})
	--({\sx*(0.8500)},{\sy*(-0.0000)})
	--({\sx*(0.8600)},{\sy*(-0.0000)})
	--({\sx*(0.8700)},{\sy*(-0.0000)})
	--({\sx*(0.8800)},{\sy*(-0.0000)})
	--({\sx*(0.8900)},{\sy*(-0.0000)})
	--({\sx*(0.9000)},{\sy*(-0.0000)})
	--({\sx*(0.9100)},{\sy*(-0.0000)})
	--({\sx*(0.9200)},{\sy*(-0.0000)})
	--({\sx*(0.9300)},{\sy*(-0.0000)})
	--({\sx*(0.9400)},{\sy*(-0.0000)})
	--({\sx*(0.9500)},{\sy*(-0.0000)})
	--({\sx*(0.9600)},{\sy*(-0.0000)})
	--({\sx*(0.9700)},{\sy*(-0.0000)})
	--({\sx*(0.9800)},{\sy*(-0.0000)})
	--({\sx*(0.9900)},{\sy*(-0.0000)})
	--({\sx*(1.0000)},{\sy*(-0.0000)})
	--({\sx*(1.0100)},{\sy*(-0.0000)})
	--({\sx*(1.0200)},{\sy*(-0.0000)})
	--({\sx*(1.0300)},{\sy*(-0.0000)})
	--({\sx*(1.0400)},{\sy*(-0.0000)})
	--({\sx*(1.0500)},{\sy*(-0.0000)})
	--({\sx*(1.0600)},{\sy*(-0.0000)})
	--({\sx*(1.0700)},{\sy*(-0.0000)})
	--({\sx*(1.0800)},{\sy*(-0.0000)})
	--({\sx*(1.0900)},{\sy*(-0.0000)})
	--({\sx*(1.1000)},{\sy*(-0.0000)})
	--({\sx*(1.1100)},{\sy*(-0.0000)})
	--({\sx*(1.1200)},{\sy*(0.0000)})
	--({\sx*(1.1300)},{\sy*(0.0000)})
	--({\sx*(1.1400)},{\sy*(0.0000)})
	--({\sx*(1.1500)},{\sy*(0.0000)})
	--({\sx*(1.1600)},{\sy*(0.0000)})
	--({\sx*(1.1700)},{\sy*(0.0000)})
	--({\sx*(1.1800)},{\sy*(0.0000)})
	--({\sx*(1.1900)},{\sy*(0.0000)})
	--({\sx*(1.2000)},{\sy*(0.0000)})
	--({\sx*(1.2100)},{\sy*(0.0000)})
	--({\sx*(1.2200)},{\sy*(0.0000)})
	--({\sx*(1.2300)},{\sy*(0.0000)})
	--({\sx*(1.2400)},{\sy*(0.0000)})
	--({\sx*(1.2500)},{\sy*(0.0000)})
	--({\sx*(1.2600)},{\sy*(0.0000)})
	--({\sx*(1.2700)},{\sy*(0.0000)})
	--({\sx*(1.2800)},{\sy*(0.0000)})
	--({\sx*(1.2900)},{\sy*(0.0000)})
	--({\sx*(1.3000)},{\sy*(0.0000)})
	--({\sx*(1.3100)},{\sy*(0.0000)})
	--({\sx*(1.3200)},{\sy*(0.0000)})
	--({\sx*(1.3300)},{\sy*(0.0000)})
	--({\sx*(1.3400)},{\sy*(0.0000)})
	--({\sx*(1.3500)},{\sy*(0.0000)})
	--({\sx*(1.3600)},{\sy*(0.0000)})
	--({\sx*(1.3700)},{\sy*(0.0000)})
	--({\sx*(1.3800)},{\sy*(0.0000)})
	--({\sx*(1.3900)},{\sy*(-0.0000)})
	--({\sx*(1.4000)},{\sy*(-0.0000)})
	--({\sx*(1.4100)},{\sy*(-0.0000)})
	--({\sx*(1.4200)},{\sy*(-0.0000)})
	--({\sx*(1.4300)},{\sy*(-0.0000)})
	--({\sx*(1.4400)},{\sy*(-0.0000)})
	--({\sx*(1.4500)},{\sy*(-0.0000)})
	--({\sx*(1.4600)},{\sy*(-0.0000)})
	--({\sx*(1.4700)},{\sy*(-0.0000)})
	--({\sx*(1.4800)},{\sy*(-0.0000)})
	--({\sx*(1.4900)},{\sy*(-0.0000)})
	--({\sx*(1.5000)},{\sy*(-0.0000)})
	--({\sx*(1.5100)},{\sy*(-0.0000)})
	--({\sx*(1.5200)},{\sy*(-0.0000)})
	--({\sx*(1.5300)},{\sy*(-0.0000)})
	--({\sx*(1.5400)},{\sy*(-0.0000)})
	--({\sx*(1.5500)},{\sy*(-0.0000)})
	--({\sx*(1.5600)},{\sy*(-0.0000)})
	--({\sx*(1.5700)},{\sy*(-0.0000)})
	--({\sx*(1.5800)},{\sy*(-0.0000)})
	--({\sx*(1.5900)},{\sy*(-0.0000)})
	--({\sx*(1.6000)},{\sy*(-0.0000)})
	--({\sx*(1.6100)},{\sy*(-0.0000)})
	--({\sx*(1.6200)},{\sy*(-0.0000)})
	--({\sx*(1.6300)},{\sy*(-0.0000)})
	--({\sx*(1.6400)},{\sy*(-0.0000)})
	--({\sx*(1.6500)},{\sy*(-0.0000)})
	--({\sx*(1.6600)},{\sy*(-0.0000)})
	--({\sx*(1.6700)},{\sy*(0.0000)})
	--({\sx*(1.6800)},{\sy*(0.0000)})
	--({\sx*(1.6900)},{\sy*(0.0000)})
	--({\sx*(1.7000)},{\sy*(0.0000)})
	--({\sx*(1.7100)},{\sy*(0.0000)})
	--({\sx*(1.7200)},{\sy*(0.0000)})
	--({\sx*(1.7300)},{\sy*(0.0000)})
	--({\sx*(1.7400)},{\sy*(0.0000)})
	--({\sx*(1.7500)},{\sy*(0.0000)})
	--({\sx*(1.7600)},{\sy*(0.0000)})
	--({\sx*(1.7700)},{\sy*(0.0000)})
	--({\sx*(1.7800)},{\sy*(0.0000)})
	--({\sx*(1.7900)},{\sy*(0.0000)})
	--({\sx*(1.8000)},{\sy*(0.0000)})
	--({\sx*(1.8100)},{\sy*(0.0000)})
	--({\sx*(1.8200)},{\sy*(0.0000)})
	--({\sx*(1.8300)},{\sy*(0.0000)})
	--({\sx*(1.8400)},{\sy*(0.0000)})
	--({\sx*(1.8500)},{\sy*(0.0000)})
	--({\sx*(1.8600)},{\sy*(0.0000)})
	--({\sx*(1.8700)},{\sy*(0.0000)})
	--({\sx*(1.8800)},{\sy*(0.0000)})
	--({\sx*(1.8900)},{\sy*(0.0000)})
	--({\sx*(1.9000)},{\sy*(0.0000)})
	--({\sx*(1.9100)},{\sy*(0.0000)})
	--({\sx*(1.9200)},{\sy*(0.0000)})
	--({\sx*(1.9300)},{\sy*(0.0000)})
	--({\sx*(1.9400)},{\sy*(0.0000)})
	--({\sx*(1.9500)},{\sy*(-0.0000)})
	--({\sx*(1.9600)},{\sy*(-0.0000)})
	--({\sx*(1.9700)},{\sy*(-0.0000)})
	--({\sx*(1.9800)},{\sy*(-0.0000)})
	--({\sx*(1.9900)},{\sy*(-0.0000)})
	--({\sx*(2.0000)},{\sy*(-0.0000)})
	--({\sx*(2.0100)},{\sy*(-0.0000)})
	--({\sx*(2.0200)},{\sy*(-0.0000)})
	--({\sx*(2.0300)},{\sy*(-0.0000)})
	--({\sx*(2.0400)},{\sy*(-0.0000)})
	--({\sx*(2.0500)},{\sy*(-0.0000)})
	--({\sx*(2.0600)},{\sy*(-0.0000)})
	--({\sx*(2.0700)},{\sy*(-0.0000)})
	--({\sx*(2.0800)},{\sy*(-0.0000)})
	--({\sx*(2.0900)},{\sy*(-0.0000)})
	--({\sx*(2.1000)},{\sy*(-0.0000)})
	--({\sx*(2.1100)},{\sy*(-0.0000)})
	--({\sx*(2.1200)},{\sy*(-0.0000)})
	--({\sx*(2.1300)},{\sy*(-0.0000)})
	--({\sx*(2.1400)},{\sy*(-0.0000)})
	--({\sx*(2.1500)},{\sy*(-0.0000)})
	--({\sx*(2.1600)},{\sy*(-0.0000)})
	--({\sx*(2.1700)},{\sy*(-0.0000)})
	--({\sx*(2.1800)},{\sy*(-0.0000)})
	--({\sx*(2.1900)},{\sy*(-0.0000)})
	--({\sx*(2.2000)},{\sy*(-0.0000)})
	--({\sx*(2.2100)},{\sy*(-0.0000)})
	--({\sx*(2.2200)},{\sy*(-0.0000)})
	--({\sx*(2.2300)},{\sy*(0.0000)})
	--({\sx*(2.2400)},{\sy*(0.0000)})
	--({\sx*(2.2500)},{\sy*(0.0000)})
	--({\sx*(2.2600)},{\sy*(0.0000)})
	--({\sx*(2.2700)},{\sy*(0.0000)})
	--({\sx*(2.2800)},{\sy*(0.0000)})
	--({\sx*(2.2900)},{\sy*(0.0000)})
	--({\sx*(2.3000)},{\sy*(0.0000)})
	--({\sx*(2.3100)},{\sy*(0.0000)})
	--({\sx*(2.3200)},{\sy*(0.0000)})
	--({\sx*(2.3300)},{\sy*(0.0000)})
	--({\sx*(2.3400)},{\sy*(0.0000)})
	--({\sx*(2.3500)},{\sy*(0.0000)})
	--({\sx*(2.3600)},{\sy*(0.0000)})
	--({\sx*(2.3700)},{\sy*(0.0000)})
	--({\sx*(2.3800)},{\sy*(0.0000)})
	--({\sx*(2.3900)},{\sy*(0.0000)})
	--({\sx*(2.4000)},{\sy*(0.0000)})
	--({\sx*(2.4100)},{\sy*(0.0000)})
	--({\sx*(2.4200)},{\sy*(0.0000)})
	--({\sx*(2.4300)},{\sy*(0.0000)})
	--({\sx*(2.4400)},{\sy*(0.0000)})
	--({\sx*(2.4500)},{\sy*(0.0000)})
	--({\sx*(2.4600)},{\sy*(0.0000)})
	--({\sx*(2.4700)},{\sy*(0.0000)})
	--({\sx*(2.4800)},{\sy*(0.0000)})
	--({\sx*(2.4900)},{\sy*(0.0000)})
	--({\sx*(2.5000)},{\sy*(0.0000)})
	--({\sx*(2.5100)},{\sy*(-0.0000)})
	--({\sx*(2.5200)},{\sy*(-0.0000)})
	--({\sx*(2.5300)},{\sy*(-0.0000)})
	--({\sx*(2.5400)},{\sy*(-0.0000)})
	--({\sx*(2.5500)},{\sy*(-0.0000)})
	--({\sx*(2.5600)},{\sy*(-0.0000)})
	--({\sx*(2.5700)},{\sy*(-0.0000)})
	--({\sx*(2.5800)},{\sy*(-0.0000)})
	--({\sx*(2.5900)},{\sy*(-0.0000)})
	--({\sx*(2.6000)},{\sy*(-0.0000)})
	--({\sx*(2.6100)},{\sy*(-0.0000)})
	--({\sx*(2.6200)},{\sy*(-0.0000)})
	--({\sx*(2.6300)},{\sy*(-0.0000)})
	--({\sx*(2.6400)},{\sy*(-0.0000)})
	--({\sx*(2.6500)},{\sy*(-0.0000)})
	--({\sx*(2.6600)},{\sy*(-0.0000)})
	--({\sx*(2.6700)},{\sy*(-0.0000)})
	--({\sx*(2.6800)},{\sy*(-0.0000)})
	--({\sx*(2.6900)},{\sy*(-0.0000)})
	--({\sx*(2.7000)},{\sy*(-0.0000)})
	--({\sx*(2.7100)},{\sy*(-0.0000)})
	--({\sx*(2.7200)},{\sy*(-0.0000)})
	--({\sx*(2.7300)},{\sy*(-0.0000)})
	--({\sx*(2.7400)},{\sy*(-0.0000)})
	--({\sx*(2.7500)},{\sy*(-0.0000)})
	--({\sx*(2.7600)},{\sy*(-0.0000)})
	--({\sx*(2.7700)},{\sy*(-0.0000)})
	--({\sx*(2.7800)},{\sy*(0.0000)})
	--({\sx*(2.7900)},{\sy*(0.0000)})
	--({\sx*(2.8000)},{\sy*(0.0000)})
	--({\sx*(2.8100)},{\sy*(0.0000)})
	--({\sx*(2.8200)},{\sy*(0.0000)})
	--({\sx*(2.8300)},{\sy*(0.0000)})
	--({\sx*(2.8400)},{\sy*(0.0000)})
	--({\sx*(2.8500)},{\sy*(0.0000)})
	--({\sx*(2.8600)},{\sy*(0.0000)})
	--({\sx*(2.8700)},{\sy*(0.0000)})
	--({\sx*(2.8800)},{\sy*(0.0000)})
	--({\sx*(2.8900)},{\sy*(0.0000)})
	--({\sx*(2.9000)},{\sy*(0.0000)})
	--({\sx*(2.9100)},{\sy*(0.0000)})
	--({\sx*(2.9200)},{\sy*(0.0000)})
	--({\sx*(2.9300)},{\sy*(0.0000)})
	--({\sx*(2.9400)},{\sy*(0.0000)})
	--({\sx*(2.9500)},{\sy*(0.0000)})
	--({\sx*(2.9600)},{\sy*(0.0000)})
	--({\sx*(2.9700)},{\sy*(0.0000)})
	--({\sx*(2.9800)},{\sy*(0.0000)})
	--({\sx*(2.9900)},{\sy*(0.0000)})
	--({\sx*(3.0000)},{\sy*(0.0000)})
	--({\sx*(3.0100)},{\sy*(0.0000)})
	--({\sx*(3.0200)},{\sy*(0.0000)})
	--({\sx*(3.0300)},{\sy*(0.0000)})
	--({\sx*(3.0400)},{\sy*(0.0000)})
	--({\sx*(3.0500)},{\sy*(0.0000)})
	--({\sx*(3.0600)},{\sy*(-0.0000)})
	--({\sx*(3.0700)},{\sy*(-0.0000)})
	--({\sx*(3.0800)},{\sy*(-0.0000)})
	--({\sx*(3.0900)},{\sy*(-0.0000)})
	--({\sx*(3.1000)},{\sy*(-0.0000)})
	--({\sx*(3.1100)},{\sy*(-0.0000)})
	--({\sx*(3.1200)},{\sy*(-0.0000)})
	--({\sx*(3.1300)},{\sy*(-0.0000)})
	--({\sx*(3.1400)},{\sy*(-0.0000)})
	--({\sx*(3.1500)},{\sy*(-0.0000)})
	--({\sx*(3.1600)},{\sy*(-0.0000)})
	--({\sx*(3.1700)},{\sy*(-0.0000)})
	--({\sx*(3.1800)},{\sy*(-0.0000)})
	--({\sx*(3.1900)},{\sy*(-0.0000)})
	--({\sx*(3.2000)},{\sy*(-0.0000)})
	--({\sx*(3.2100)},{\sy*(-0.0000)})
	--({\sx*(3.2200)},{\sy*(-0.0000)})
	--({\sx*(3.2300)},{\sy*(-0.0000)})
	--({\sx*(3.2400)},{\sy*(-0.0000)})
	--({\sx*(3.2500)},{\sy*(-0.0000)})
	--({\sx*(3.2600)},{\sy*(-0.0000)})
	--({\sx*(3.2700)},{\sy*(-0.0000)})
	--({\sx*(3.2800)},{\sy*(-0.0000)})
	--({\sx*(3.2900)},{\sy*(-0.0000)})
	--({\sx*(3.3000)},{\sy*(-0.0000)})
	--({\sx*(3.3100)},{\sy*(-0.0000)})
	--({\sx*(3.3200)},{\sy*(-0.0000)})
	--({\sx*(3.3300)},{\sy*(-0.0000)})
	--({\sx*(3.3400)},{\sy*(0.0000)})
	--({\sx*(3.3500)},{\sy*(0.0000)})
	--({\sx*(3.3600)},{\sy*(0.0000)})
	--({\sx*(3.3700)},{\sy*(0.0000)})
	--({\sx*(3.3800)},{\sy*(0.0000)})
	--({\sx*(3.3900)},{\sy*(0.0000)})
	--({\sx*(3.4000)},{\sy*(0.0000)})
	--({\sx*(3.4100)},{\sy*(0.0000)})
	--({\sx*(3.4200)},{\sy*(0.0000)})
	--({\sx*(3.4300)},{\sy*(0.0000)})
	--({\sx*(3.4400)},{\sy*(0.0000)})
	--({\sx*(3.4500)},{\sy*(0.0000)})
	--({\sx*(3.4600)},{\sy*(0.0000)})
	--({\sx*(3.4700)},{\sy*(0.0000)})
	--({\sx*(3.4800)},{\sy*(0.0000)})
	--({\sx*(3.4900)},{\sy*(0.0000)})
	--({\sx*(3.5000)},{\sy*(0.0000)})
	--({\sx*(3.5100)},{\sy*(0.0000)})
	--({\sx*(3.5200)},{\sy*(0.0000)})
	--({\sx*(3.5300)},{\sy*(0.0000)})
	--({\sx*(3.5400)},{\sy*(0.0000)})
	--({\sx*(3.5500)},{\sy*(0.0000)})
	--({\sx*(3.5600)},{\sy*(0.0000)})
	--({\sx*(3.5700)},{\sy*(0.0000)})
	--({\sx*(3.5800)},{\sy*(0.0000)})
	--({\sx*(3.5900)},{\sy*(0.0000)})
	--({\sx*(3.6000)},{\sy*(0.0000)})
	--({\sx*(3.6100)},{\sy*(0.0000)})
	--({\sx*(3.6200)},{\sy*(-0.0000)})
	--({\sx*(3.6300)},{\sy*(-0.0000)})
	--({\sx*(3.6400)},{\sy*(-0.0000)})
	--({\sx*(3.6500)},{\sy*(-0.0000)})
	--({\sx*(3.6600)},{\sy*(-0.0000)})
	--({\sx*(3.6700)},{\sy*(-0.0000)})
	--({\sx*(3.6800)},{\sy*(-0.0000)})
	--({\sx*(3.6900)},{\sy*(-0.0000)})
	--({\sx*(3.7000)},{\sy*(-0.0000)})
	--({\sx*(3.7100)},{\sy*(-0.0000)})
	--({\sx*(3.7200)},{\sy*(-0.0000)})
	--({\sx*(3.7300)},{\sy*(-0.0000)})
	--({\sx*(3.7400)},{\sy*(-0.0000)})
	--({\sx*(3.7500)},{\sy*(-0.0000)})
	--({\sx*(3.7600)},{\sy*(-0.0000)})
	--({\sx*(3.7700)},{\sy*(-0.0000)})
	--({\sx*(3.7800)},{\sy*(-0.0000)})
	--({\sx*(3.7900)},{\sy*(-0.0000)})
	--({\sx*(3.8000)},{\sy*(-0.0000)})
	--({\sx*(3.8100)},{\sy*(-0.0000)})
	--({\sx*(3.8200)},{\sy*(-0.0000)})
	--({\sx*(3.8300)},{\sy*(-0.0000)})
	--({\sx*(3.8400)},{\sy*(-0.0000)})
	--({\sx*(3.8500)},{\sy*(-0.0000)})
	--({\sx*(3.8600)},{\sy*(-0.0000)})
	--({\sx*(3.8700)},{\sy*(-0.0000)})
	--({\sx*(3.8800)},{\sy*(-0.0000)})
	--({\sx*(3.8900)},{\sy*(0.0000)})
	--({\sx*(3.9000)},{\sy*(0.0000)})
	--({\sx*(3.9100)},{\sy*(0.0000)})
	--({\sx*(3.9200)},{\sy*(0.0000)})
	--({\sx*(3.9300)},{\sy*(0.0000)})
	--({\sx*(3.9400)},{\sy*(0.0000)})
	--({\sx*(3.9500)},{\sy*(0.0000)})
	--({\sx*(3.9600)},{\sy*(0.0000)})
	--({\sx*(3.9700)},{\sy*(0.0000)})
	--({\sx*(3.9800)},{\sy*(0.0000)})
	--({\sx*(3.9900)},{\sy*(0.0000)})
	--({\sx*(4.0000)},{\sy*(0.0000)})
	--({\sx*(4.0100)},{\sy*(0.0000)})
	--({\sx*(4.0200)},{\sy*(0.0000)})
	--({\sx*(4.0300)},{\sy*(0.0000)})
	--({\sx*(4.0400)},{\sy*(0.0000)})
	--({\sx*(4.0500)},{\sy*(0.0000)})
	--({\sx*(4.0600)},{\sy*(0.0000)})
	--({\sx*(4.0700)},{\sy*(0.0000)})
	--({\sx*(4.0800)},{\sy*(0.0000)})
	--({\sx*(4.0900)},{\sy*(0.0000)})
	--({\sx*(4.1000)},{\sy*(0.0000)})
	--({\sx*(4.1100)},{\sy*(0.0000)})
	--({\sx*(4.1200)},{\sy*(0.0000)})
	--({\sx*(4.1300)},{\sy*(0.0000)})
	--({\sx*(4.1400)},{\sy*(0.0000)})
	--({\sx*(4.1500)},{\sy*(0.0000)})
	--({\sx*(4.1600)},{\sy*(0.0000)})
	--({\sx*(4.1700)},{\sy*(-0.0000)})
	--({\sx*(4.1800)},{\sy*(-0.0000)})
	--({\sx*(4.1900)},{\sy*(-0.0000)})
	--({\sx*(4.2000)},{\sy*(-0.0000)})
	--({\sx*(4.2100)},{\sy*(-0.0000)})
	--({\sx*(4.2200)},{\sy*(-0.0000)})
	--({\sx*(4.2300)},{\sy*(-0.0000)})
	--({\sx*(4.2400)},{\sy*(-0.0001)})
	--({\sx*(4.2500)},{\sy*(-0.0001)})
	--({\sx*(4.2600)},{\sy*(-0.0001)})
	--({\sx*(4.2700)},{\sy*(-0.0001)})
	--({\sx*(4.2800)},{\sy*(-0.0001)})
	--({\sx*(4.2900)},{\sy*(-0.0001)})
	--({\sx*(4.3000)},{\sy*(-0.0001)})
	--({\sx*(4.3100)},{\sy*(-0.0001)})
	--({\sx*(4.3200)},{\sy*(-0.0002)})
	--({\sx*(4.3300)},{\sy*(-0.0002)})
	--({\sx*(4.3400)},{\sy*(-0.0002)})
	--({\sx*(4.3500)},{\sy*(-0.0002)})
	--({\sx*(4.3600)},{\sy*(-0.0002)})
	--({\sx*(4.3700)},{\sy*(-0.0002)})
	--({\sx*(4.3800)},{\sy*(-0.0002)})
	--({\sx*(4.3900)},{\sy*(-0.0002)})
	--({\sx*(4.4000)},{\sy*(-0.0002)})
	--({\sx*(4.4100)},{\sy*(-0.0002)})
	--({\sx*(4.4200)},{\sy*(-0.0001)})
	--({\sx*(4.4300)},{\sy*(-0.0001)})
	--({\sx*(4.4400)},{\sy*(-0.0000)})
	--({\sx*(4.4500)},{\sy*(0.0000)})
	--({\sx*(4.4600)},{\sy*(0.0001)})
	--({\sx*(4.4700)},{\sy*(0.0002)})
	--({\sx*(4.4800)},{\sy*(0.0004)})
	--({\sx*(4.4900)},{\sy*(0.0005)})
	--({\sx*(4.5000)},{\sy*(0.0007)})
	--({\sx*(4.5100)},{\sy*(0.0009)})
	--({\sx*(4.5200)},{\sy*(0.0011)})
	--({\sx*(4.5300)},{\sy*(0.0013)})
	--({\sx*(4.5400)},{\sy*(0.0016)})
	--({\sx*(4.5500)},{\sy*(0.0019)})
	--({\sx*(4.5600)},{\sy*(0.0023)})
	--({\sx*(4.5700)},{\sy*(0.0026)})
	--({\sx*(4.5800)},{\sy*(0.0030)})
	--({\sx*(4.5900)},{\sy*(0.0033)})
	--({\sx*(4.6000)},{\sy*(0.0037)})
	--({\sx*(4.6100)},{\sy*(0.0041)})
	--({\sx*(4.6200)},{\sy*(0.0044)})
	--({\sx*(4.6300)},{\sy*(0.0048)})
	--({\sx*(4.6400)},{\sy*(0.0050)})
	--({\sx*(4.6500)},{\sy*(0.0052)})
	--({\sx*(4.6600)},{\sy*(0.0052)})
	--({\sx*(4.6700)},{\sy*(0.0051)})
	--({\sx*(4.6800)},{\sy*(0.0048)})
	--({\sx*(4.6900)},{\sy*(0.0042)})
	--({\sx*(4.7000)},{\sy*(0.0034)})
	--({\sx*(4.7100)},{\sy*(0.0021)})
	--({\sx*(4.7200)},{\sy*(0.0004)})
	--({\sx*(4.7300)},{\sy*(-0.0018)})
	--({\sx*(4.7400)},{\sy*(-0.0047)})
	--({\sx*(4.7500)},{\sy*(-0.0083)})
	--({\sx*(4.7600)},{\sy*(-0.0128)})
	--({\sx*(4.7700)},{\sy*(-0.0184)})
	--({\sx*(4.7800)},{\sy*(-0.0252)})
	--({\sx*(4.7900)},{\sy*(-0.0334)})
	--({\sx*(4.8000)},{\sy*(-0.0431)})
	--({\sx*(4.8100)},{\sy*(-0.0547)})
	--({\sx*(4.8200)},{\sy*(-0.0683)})
	--({\sx*(4.8300)},{\sy*(-0.0842)})
	--({\sx*(4.8400)},{\sy*(-0.1027)})
	--({\sx*(4.8500)},{\sy*(-0.1239)})
	--({\sx*(4.8600)},{\sy*(-0.1482)})
	--({\sx*(4.8700)},{\sy*(-0.1756)})
	--({\sx*(4.8800)},{\sy*(-0.2060)})
	--({\sx*(4.8900)},{\sy*(-0.2393)})
	--({\sx*(4.9000)},{\sy*(-0.2747)})
	--({\sx*(4.9100)},{\sy*(-0.3108)})
	--({\sx*(4.9200)},{\sy*(-0.3453)})
	--({\sx*(4.9300)},{\sy*(-0.3748)})
	--({\sx*(4.9400)},{\sy*(-0.3942)})
	--({\sx*(4.9500)},{\sy*(-0.3972)})
	--({\sx*(4.9600)},{\sy*(-0.3768)})
	--({\sx*(4.9700)},{\sy*(-0.3269)})
	--({\sx*(4.9800)},{\sy*(-0.2448)})
	--({\sx*(4.9900)},{\sy*(-0.1331)})
	--({\sx*(5.0000)},{\sy*(0.0000)});
}
\def\xwertej{
\fill[color=red] (0.0000,0) circle[radius={0.07/\skala}];
\fill[color=red] (0.2500,0) circle[radius={0.07/\skala}];
\fill[color=red] (0.5000,0) circle[radius={0.07/\skala}];
\fill[color=red] (0.7500,0) circle[radius={0.07/\skala}];
\fill[color=red] (1.0000,0) circle[radius={0.07/\skala}];
\fill[color=red] (1.2500,0) circle[radius={0.07/\skala}];
\fill[color=red] (1.5000,0) circle[radius={0.07/\skala}];
\fill[color=red] (1.7500,0) circle[radius={0.07/\skala}];
\fill[color=red] (2.0000,0) circle[radius={0.07/\skala}];
\fill[color=red] (2.2500,0) circle[radius={0.07/\skala}];
\fill[color=red] (2.5000,0) circle[radius={0.07/\skala}];
\fill[color=red] (2.7500,0) circle[radius={0.07/\skala}];
\fill[color=red] (3.0000,0) circle[radius={0.07/\skala}];
\fill[color=red] (3.2500,0) circle[radius={0.07/\skala}];
\fill[color=red] (3.5000,0) circle[radius={0.07/\skala}];
\fill[color=red] (3.7500,0) circle[radius={0.07/\skala}];
\fill[color=red] (4.0000,0) circle[radius={0.07/\skala}];
\fill[color=red] (4.2500,0) circle[radius={0.07/\skala}];
\fill[color=red] (4.5000,0) circle[radius={0.07/\skala}];
\fill[color=red] (4.7500,0) circle[radius={0.07/\skala}];
\fill[color=red] (5.0000,0) circle[radius={0.07/\skala}];
}
\def\punktej{20}
\def\maxfehlerj{9.403\cdot 10^{-8}}
\def\fehlerj{
\draw[color=red,line width=1.4pt,line join=round] ({\sx*(0.000)},{\sy*(0.0000)})
	--({\sx*(0.0100)},{\sy*(-0.3568)})
	--({\sx*(0.0200)},{\sy*(-0.6167)})
	--({\sx*(0.0300)},{\sy*(-0.7970)})
	--({\sx*(0.0400)},{\sy*(-0.9129)})
	--({\sx*(0.0500)},{\sy*(-0.9769)})
	--({\sx*(0.0600)},{\sy*(-1.0000)})
	--({\sx*(0.0700)},{\sy*(-0.9913)})
	--({\sx*(0.0800)},{\sy*(-0.9585)})
	--({\sx*(0.0900)},{\sy*(-0.9081)})
	--({\sx*(0.1000)},{\sy*(-0.8453)})
	--({\sx*(0.1100)},{\sy*(-0.7745)})
	--({\sx*(0.1200)},{\sy*(-0.6993)})
	--({\sx*(0.1300)},{\sy*(-0.6224)})
	--({\sx*(0.1400)},{\sy*(-0.5460)})
	--({\sx*(0.1500)},{\sy*(-0.4719)})
	--({\sx*(0.1600)},{\sy*(-0.4014)})
	--({\sx*(0.1700)},{\sy*(-0.3353)})
	--({\sx*(0.1800)},{\sy*(-0.2743)})
	--({\sx*(0.1900)},{\sy*(-0.2188)})
	--({\sx*(0.2000)},{\sy*(-0.1688)})
	--({\sx*(0.2100)},{\sy*(-0.1245)})
	--({\sx*(0.2200)},{\sy*(-0.0857)})
	--({\sx*(0.2300)},{\sy*(-0.0523)})
	--({\sx*(0.2400)},{\sy*(-0.0238)})
	--({\sx*(0.2500)},{\sy*(0.0000)})
	--({\sx*(0.2600)},{\sy*(0.0195)})
	--({\sx*(0.2700)},{\sy*(0.0350)})
	--({\sx*(0.2800)},{\sy*(0.0470)})
	--({\sx*(0.2900)},{\sy*(0.0558)})
	--({\sx*(0.3000)},{\sy*(0.0619)})
	--({\sx*(0.3100)},{\sy*(0.0656)})
	--({\sx*(0.3200)},{\sy*(0.0673)})
	--({\sx*(0.3300)},{\sy*(0.0672)})
	--({\sx*(0.3400)},{\sy*(0.0657)})
	--({\sx*(0.3500)},{\sy*(0.0631)})
	--({\sx*(0.3600)},{\sy*(0.0596)})
	--({\sx*(0.3700)},{\sy*(0.0554)})
	--({\sx*(0.3800)},{\sy*(0.0507)})
	--({\sx*(0.3900)},{\sy*(0.0457)})
	--({\sx*(0.4000)},{\sy*(0.0406)})
	--({\sx*(0.4100)},{\sy*(0.0355)})
	--({\sx*(0.4200)},{\sy*(0.0304)})
	--({\sx*(0.4300)},{\sy*(0.0255)})
	--({\sx*(0.4400)},{\sy*(0.0209)})
	--({\sx*(0.4500)},{\sy*(0.0165)})
	--({\sx*(0.4600)},{\sy*(0.0125)})
	--({\sx*(0.4700)},{\sy*(0.0088)})
	--({\sx*(0.4800)},{\sy*(0.0055)})
	--({\sx*(0.4900)},{\sy*(0.0025)})
	--({\sx*(0.5000)},{\sy*(0.0000)})
	--({\sx*(0.5100)},{\sy*(-0.0022)})
	--({\sx*(0.5200)},{\sy*(-0.0040)})
	--({\sx*(0.5300)},{\sy*(-0.0055)})
	--({\sx*(0.5400)},{\sy*(-0.0067)})
	--({\sx*(0.5500)},{\sy*(-0.0075)})
	--({\sx*(0.5600)},{\sy*(-0.0081)})
	--({\sx*(0.5700)},{\sy*(-0.0085)})
	--({\sx*(0.5800)},{\sy*(-0.0087)})
	--({\sx*(0.5900)},{\sy*(-0.0086)})
	--({\sx*(0.6000)},{\sy*(-0.0084)})
	--({\sx*(0.6100)},{\sy*(-0.0081)})
	--({\sx*(0.6200)},{\sy*(-0.0077)})
	--({\sx*(0.6300)},{\sy*(-0.0072)})
	--({\sx*(0.6400)},{\sy*(-0.0066)})
	--({\sx*(0.6500)},{\sy*(-0.0059)})
	--({\sx*(0.6600)},{\sy*(-0.0053)})
	--({\sx*(0.6700)},{\sy*(-0.0046)})
	--({\sx*(0.6800)},{\sy*(-0.0039)})
	--({\sx*(0.6900)},{\sy*(-0.0033)})
	--({\sx*(0.7000)},{\sy*(-0.0026)})
	--({\sx*(0.7100)},{\sy*(-0.0020)})
	--({\sx*(0.7200)},{\sy*(-0.0014)})
	--({\sx*(0.7300)},{\sy*(-0.0009)})
	--({\sx*(0.7400)},{\sy*(-0.0004)})
	--({\sx*(0.7500)},{\sy*(0.0000)})
	--({\sx*(0.7600)},{\sy*(0.0004)})
	--({\sx*(0.7700)},{\sy*(0.0007)})
	--({\sx*(0.7800)},{\sy*(0.0010)})
	--({\sx*(0.7900)},{\sy*(0.0012)})
	--({\sx*(0.8000)},{\sy*(0.0014)})
	--({\sx*(0.8100)},{\sy*(0.0015)})
	--({\sx*(0.8200)},{\sy*(0.0016)})
	--({\sx*(0.8300)},{\sy*(0.0017)})
	--({\sx*(0.8400)},{\sy*(0.0017)})
	--({\sx*(0.8500)},{\sy*(0.0017)})
	--({\sx*(0.8600)},{\sy*(0.0016)})
	--({\sx*(0.8700)},{\sy*(0.0016)})
	--({\sx*(0.8800)},{\sy*(0.0015)})
	--({\sx*(0.8900)},{\sy*(0.0014)})
	--({\sx*(0.9000)},{\sy*(0.0013)})
	--({\sx*(0.9100)},{\sy*(0.0011)})
	--({\sx*(0.9200)},{\sy*(0.0010)})
	--({\sx*(0.9300)},{\sy*(0.0009)})
	--({\sx*(0.9400)},{\sy*(0.0007)})
	--({\sx*(0.9500)},{\sy*(0.0006)})
	--({\sx*(0.9600)},{\sy*(0.0005)})
	--({\sx*(0.9700)},{\sy*(0.0003)})
	--({\sx*(0.9800)},{\sy*(0.0002)})
	--({\sx*(0.9900)},{\sy*(0.0001)})
	--({\sx*(1.0000)},{\sy*(0.0000)})
	--({\sx*(1.0100)},{\sy*(-0.0001)})
	--({\sx*(1.0200)},{\sy*(-0.0002)})
	--({\sx*(1.0300)},{\sy*(-0.0002)})
	--({\sx*(1.0400)},{\sy*(-0.0003)})
	--({\sx*(1.0500)},{\sy*(-0.0004)})
	--({\sx*(1.0600)},{\sy*(-0.0004)})
	--({\sx*(1.0700)},{\sy*(-0.0004)})
	--({\sx*(1.0800)},{\sy*(-0.0004)})
	--({\sx*(1.0900)},{\sy*(-0.0005)})
	--({\sx*(1.1000)},{\sy*(-0.0005)})
	--({\sx*(1.1100)},{\sy*(-0.0005)})
	--({\sx*(1.1200)},{\sy*(-0.0004)})
	--({\sx*(1.1300)},{\sy*(-0.0004)})
	--({\sx*(1.1400)},{\sy*(-0.0004)})
	--({\sx*(1.1500)},{\sy*(-0.0004)})
	--({\sx*(1.1600)},{\sy*(-0.0003)})
	--({\sx*(1.1700)},{\sy*(-0.0003)})
	--({\sx*(1.1800)},{\sy*(-0.0003)})
	--({\sx*(1.1900)},{\sy*(-0.0002)})
	--({\sx*(1.2000)},{\sy*(-0.0002)})
	--({\sx*(1.2100)},{\sy*(-0.0001)})
	--({\sx*(1.2200)},{\sy*(-0.0001)})
	--({\sx*(1.2300)},{\sy*(-0.0001)})
	--({\sx*(1.2400)},{\sy*(-0.0000)})
	--({\sx*(1.2500)},{\sy*(0.0000)})
	--({\sx*(1.2600)},{\sy*(0.0000)})
	--({\sx*(1.2700)},{\sy*(0.0001)})
	--({\sx*(1.2800)},{\sy*(0.0001)})
	--({\sx*(1.2900)},{\sy*(0.0001)})
	--({\sx*(1.3000)},{\sy*(0.0001)})
	--({\sx*(1.3100)},{\sy*(0.0001)})
	--({\sx*(1.3200)},{\sy*(0.0001)})
	--({\sx*(1.3300)},{\sy*(0.0002)})
	--({\sx*(1.3400)},{\sy*(0.0002)})
	--({\sx*(1.3500)},{\sy*(0.0002)})
	--({\sx*(1.3600)},{\sy*(0.0002)})
	--({\sx*(1.3700)},{\sy*(0.0002)})
	--({\sx*(1.3800)},{\sy*(0.0002)})
	--({\sx*(1.3900)},{\sy*(0.0001)})
	--({\sx*(1.4000)},{\sy*(0.0001)})
	--({\sx*(1.4100)},{\sy*(0.0001)})
	--({\sx*(1.4200)},{\sy*(0.0001)})
	--({\sx*(1.4300)},{\sy*(0.0001)})
	--({\sx*(1.4400)},{\sy*(0.0001)})
	--({\sx*(1.4500)},{\sy*(0.0001)})
	--({\sx*(1.4600)},{\sy*(0.0001)})
	--({\sx*(1.4700)},{\sy*(0.0000)})
	--({\sx*(1.4800)},{\sy*(0.0000)})
	--({\sx*(1.4900)},{\sy*(0.0000)})
	--({\sx*(1.5000)},{\sy*(0.0000)})
	--({\sx*(1.5100)},{\sy*(-0.0000)})
	--({\sx*(1.5200)},{\sy*(-0.0000)})
	--({\sx*(1.5300)},{\sy*(-0.0000)})
	--({\sx*(1.5400)},{\sy*(-0.0000)})
	--({\sx*(1.5500)},{\sy*(-0.0001)})
	--({\sx*(1.5600)},{\sy*(-0.0001)})
	--({\sx*(1.5700)},{\sy*(-0.0001)})
	--({\sx*(1.5800)},{\sy*(-0.0001)})
	--({\sx*(1.5900)},{\sy*(-0.0001)})
	--({\sx*(1.6000)},{\sy*(-0.0001)})
	--({\sx*(1.6100)},{\sy*(-0.0001)})
	--({\sx*(1.6200)},{\sy*(-0.0001)})
	--({\sx*(1.6300)},{\sy*(-0.0001)})
	--({\sx*(1.6400)},{\sy*(-0.0001)})
	--({\sx*(1.6500)},{\sy*(-0.0001)})
	--({\sx*(1.6600)},{\sy*(-0.0001)})
	--({\sx*(1.6700)},{\sy*(-0.0001)})
	--({\sx*(1.6800)},{\sy*(-0.0000)})
	--({\sx*(1.6900)},{\sy*(-0.0000)})
	--({\sx*(1.7000)},{\sy*(-0.0000)})
	--({\sx*(1.7100)},{\sy*(-0.0000)})
	--({\sx*(1.7200)},{\sy*(-0.0000)})
	--({\sx*(1.7300)},{\sy*(-0.0000)})
	--({\sx*(1.7400)},{\sy*(-0.0000)})
	--({\sx*(1.7500)},{\sy*(0.0000)})
	--({\sx*(1.7600)},{\sy*(0.0000)})
	--({\sx*(1.7700)},{\sy*(0.0000)})
	--({\sx*(1.7800)},{\sy*(0.0000)})
	--({\sx*(1.7900)},{\sy*(0.0000)})
	--({\sx*(1.8000)},{\sy*(0.0000)})
	--({\sx*(1.8100)},{\sy*(0.0000)})
	--({\sx*(1.8200)},{\sy*(0.0000)})
	--({\sx*(1.8300)},{\sy*(0.0000)})
	--({\sx*(1.8400)},{\sy*(0.0000)})
	--({\sx*(1.8500)},{\sy*(0.0000)})
	--({\sx*(1.8600)},{\sy*(0.0000)})
	--({\sx*(1.8700)},{\sy*(0.0000)})
	--({\sx*(1.8800)},{\sy*(0.0000)})
	--({\sx*(1.8900)},{\sy*(0.0000)})
	--({\sx*(1.9000)},{\sy*(0.0000)})
	--({\sx*(1.9100)},{\sy*(0.0000)})
	--({\sx*(1.9200)},{\sy*(0.0000)})
	--({\sx*(1.9300)},{\sy*(0.0000)})
	--({\sx*(1.9400)},{\sy*(0.0000)})
	--({\sx*(1.9500)},{\sy*(0.0000)})
	--({\sx*(1.9600)},{\sy*(0.0000)})
	--({\sx*(1.9700)},{\sy*(0.0000)})
	--({\sx*(1.9800)},{\sy*(0.0000)})
	--({\sx*(1.9900)},{\sy*(0.0000)})
	--({\sx*(2.0000)},{\sy*(0.0000)})
	--({\sx*(2.0100)},{\sy*(-0.0000)})
	--({\sx*(2.0200)},{\sy*(-0.0000)})
	--({\sx*(2.0300)},{\sy*(-0.0000)})
	--({\sx*(2.0400)},{\sy*(-0.0000)})
	--({\sx*(2.0500)},{\sy*(-0.0000)})
	--({\sx*(2.0600)},{\sy*(-0.0000)})
	--({\sx*(2.0700)},{\sy*(-0.0000)})
	--({\sx*(2.0800)},{\sy*(-0.0000)})
	--({\sx*(2.0900)},{\sy*(-0.0000)})
	--({\sx*(2.1000)},{\sy*(-0.0000)})
	--({\sx*(2.1100)},{\sy*(-0.0000)})
	--({\sx*(2.1200)},{\sy*(-0.0000)})
	--({\sx*(2.1300)},{\sy*(-0.0000)})
	--({\sx*(2.1400)},{\sy*(-0.0000)})
	--({\sx*(2.1500)},{\sy*(-0.0000)})
	--({\sx*(2.1600)},{\sy*(-0.0000)})
	--({\sx*(2.1700)},{\sy*(-0.0000)})
	--({\sx*(2.1800)},{\sy*(-0.0000)})
	--({\sx*(2.1900)},{\sy*(-0.0000)})
	--({\sx*(2.2000)},{\sy*(-0.0000)})
	--({\sx*(2.2100)},{\sy*(-0.0000)})
	--({\sx*(2.2200)},{\sy*(-0.0000)})
	--({\sx*(2.2300)},{\sy*(-0.0000)})
	--({\sx*(2.2400)},{\sy*(-0.0000)})
	--({\sx*(2.2500)},{\sy*(0.0000)})
	--({\sx*(2.2600)},{\sy*(0.0000)})
	--({\sx*(2.2700)},{\sy*(0.0000)})
	--({\sx*(2.2800)},{\sy*(0.0000)})
	--({\sx*(2.2900)},{\sy*(0.0000)})
	--({\sx*(2.3000)},{\sy*(0.0000)})
	--({\sx*(2.3100)},{\sy*(0.0000)})
	--({\sx*(2.3200)},{\sy*(0.0000)})
	--({\sx*(2.3300)},{\sy*(0.0000)})
	--({\sx*(2.3400)},{\sy*(0.0000)})
	--({\sx*(2.3500)},{\sy*(0.0000)})
	--({\sx*(2.3600)},{\sy*(0.0000)})
	--({\sx*(2.3700)},{\sy*(0.0000)})
	--({\sx*(2.3800)},{\sy*(0.0000)})
	--({\sx*(2.3900)},{\sy*(0.0000)})
	--({\sx*(2.4000)},{\sy*(0.0000)})
	--({\sx*(2.4100)},{\sy*(0.0000)})
	--({\sx*(2.4200)},{\sy*(0.0000)})
	--({\sx*(2.4300)},{\sy*(0.0000)})
	--({\sx*(2.4400)},{\sy*(0.0000)})
	--({\sx*(2.4500)},{\sy*(0.0000)})
	--({\sx*(2.4600)},{\sy*(0.0000)})
	--({\sx*(2.4700)},{\sy*(0.0000)})
	--({\sx*(2.4800)},{\sy*(0.0000)})
	--({\sx*(2.4900)},{\sy*(0.0000)})
	--({\sx*(2.5000)},{\sy*(0.0000)})
	--({\sx*(2.5100)},{\sy*(-0.0000)})
	--({\sx*(2.5200)},{\sy*(-0.0000)})
	--({\sx*(2.5300)},{\sy*(-0.0000)})
	--({\sx*(2.5400)},{\sy*(-0.0000)})
	--({\sx*(2.5500)},{\sy*(-0.0000)})
	--({\sx*(2.5600)},{\sy*(-0.0000)})
	--({\sx*(2.5700)},{\sy*(-0.0000)})
	--({\sx*(2.5800)},{\sy*(-0.0000)})
	--({\sx*(2.5900)},{\sy*(-0.0000)})
	--({\sx*(2.6000)},{\sy*(-0.0000)})
	--({\sx*(2.6100)},{\sy*(-0.0000)})
	--({\sx*(2.6200)},{\sy*(-0.0000)})
	--({\sx*(2.6300)},{\sy*(-0.0000)})
	--({\sx*(2.6400)},{\sy*(-0.0000)})
	--({\sx*(2.6500)},{\sy*(-0.0000)})
	--({\sx*(2.6600)},{\sy*(-0.0000)})
	--({\sx*(2.6700)},{\sy*(-0.0000)})
	--({\sx*(2.6800)},{\sy*(-0.0000)})
	--({\sx*(2.6900)},{\sy*(-0.0000)})
	--({\sx*(2.7000)},{\sy*(-0.0000)})
	--({\sx*(2.7100)},{\sy*(-0.0000)})
	--({\sx*(2.7200)},{\sy*(-0.0000)})
	--({\sx*(2.7300)},{\sy*(-0.0000)})
	--({\sx*(2.7400)},{\sy*(-0.0000)})
	--({\sx*(2.7500)},{\sy*(0.0000)})
	--({\sx*(2.7600)},{\sy*(0.0000)})
	--({\sx*(2.7700)},{\sy*(0.0000)})
	--({\sx*(2.7800)},{\sy*(0.0000)})
	--({\sx*(2.7900)},{\sy*(0.0000)})
	--({\sx*(2.8000)},{\sy*(0.0000)})
	--({\sx*(2.8100)},{\sy*(0.0000)})
	--({\sx*(2.8200)},{\sy*(0.0000)})
	--({\sx*(2.8300)},{\sy*(0.0000)})
	--({\sx*(2.8400)},{\sy*(0.0000)})
	--({\sx*(2.8500)},{\sy*(0.0000)})
	--({\sx*(2.8600)},{\sy*(0.0000)})
	--({\sx*(2.8700)},{\sy*(0.0000)})
	--({\sx*(2.8800)},{\sy*(0.0000)})
	--({\sx*(2.8900)},{\sy*(0.0000)})
	--({\sx*(2.9000)},{\sy*(0.0000)})
	--({\sx*(2.9100)},{\sy*(0.0000)})
	--({\sx*(2.9200)},{\sy*(0.0000)})
	--({\sx*(2.9300)},{\sy*(0.0000)})
	--({\sx*(2.9400)},{\sy*(0.0000)})
	--({\sx*(2.9500)},{\sy*(0.0000)})
	--({\sx*(2.9600)},{\sy*(0.0000)})
	--({\sx*(2.9700)},{\sy*(0.0000)})
	--({\sx*(2.9800)},{\sy*(0.0000)})
	--({\sx*(2.9900)},{\sy*(0.0000)})
	--({\sx*(3.0000)},{\sy*(0.0000)})
	--({\sx*(3.0100)},{\sy*(-0.0000)})
	--({\sx*(3.0200)},{\sy*(-0.0000)})
	--({\sx*(3.0300)},{\sy*(-0.0000)})
	--({\sx*(3.0400)},{\sy*(-0.0000)})
	--({\sx*(3.0500)},{\sy*(-0.0000)})
	--({\sx*(3.0600)},{\sy*(-0.0000)})
	--({\sx*(3.0700)},{\sy*(-0.0000)})
	--({\sx*(3.0800)},{\sy*(-0.0000)})
	--({\sx*(3.0900)},{\sy*(-0.0000)})
	--({\sx*(3.1000)},{\sy*(-0.0000)})
	--({\sx*(3.1100)},{\sy*(-0.0000)})
	--({\sx*(3.1200)},{\sy*(-0.0000)})
	--({\sx*(3.1300)},{\sy*(-0.0000)})
	--({\sx*(3.1400)},{\sy*(-0.0000)})
	--({\sx*(3.1500)},{\sy*(-0.0000)})
	--({\sx*(3.1600)},{\sy*(-0.0000)})
	--({\sx*(3.1700)},{\sy*(-0.0000)})
	--({\sx*(3.1800)},{\sy*(-0.0000)})
	--({\sx*(3.1900)},{\sy*(-0.0000)})
	--({\sx*(3.2000)},{\sy*(-0.0000)})
	--({\sx*(3.2100)},{\sy*(-0.0000)})
	--({\sx*(3.2200)},{\sy*(-0.0000)})
	--({\sx*(3.2300)},{\sy*(-0.0000)})
	--({\sx*(3.2400)},{\sy*(-0.0000)})
	--({\sx*(3.2500)},{\sy*(0.0000)})
	--({\sx*(3.2600)},{\sy*(0.0000)})
	--({\sx*(3.2700)},{\sy*(0.0000)})
	--({\sx*(3.2800)},{\sy*(0.0000)})
	--({\sx*(3.2900)},{\sy*(0.0000)})
	--({\sx*(3.3000)},{\sy*(0.0000)})
	--({\sx*(3.3100)},{\sy*(0.0000)})
	--({\sx*(3.3200)},{\sy*(0.0000)})
	--({\sx*(3.3300)},{\sy*(0.0000)})
	--({\sx*(3.3400)},{\sy*(0.0001)})
	--({\sx*(3.3500)},{\sy*(0.0001)})
	--({\sx*(3.3600)},{\sy*(0.0001)})
	--({\sx*(3.3700)},{\sy*(0.0001)})
	--({\sx*(3.3800)},{\sy*(0.0001)})
	--({\sx*(3.3900)},{\sy*(0.0001)})
	--({\sx*(3.4000)},{\sy*(0.0001)})
	--({\sx*(3.4100)},{\sy*(0.0001)})
	--({\sx*(3.4200)},{\sy*(0.0001)})
	--({\sx*(3.4300)},{\sy*(0.0001)})
	--({\sx*(3.4400)},{\sy*(0.0001)})
	--({\sx*(3.4500)},{\sy*(0.0000)})
	--({\sx*(3.4600)},{\sy*(0.0000)})
	--({\sx*(3.4700)},{\sy*(0.0000)})
	--({\sx*(3.4800)},{\sy*(0.0000)})
	--({\sx*(3.4900)},{\sy*(0.0000)})
	--({\sx*(3.5000)},{\sy*(0.0000)})
	--({\sx*(3.5100)},{\sy*(-0.0000)})
	--({\sx*(3.5200)},{\sy*(-0.0000)})
	--({\sx*(3.5300)},{\sy*(-0.0000)})
	--({\sx*(3.5400)},{\sy*(-0.0000)})
	--({\sx*(3.5500)},{\sy*(-0.0001)})
	--({\sx*(3.5600)},{\sy*(-0.0001)})
	--({\sx*(3.5700)},{\sy*(-0.0001)})
	--({\sx*(3.5800)},{\sy*(-0.0001)})
	--({\sx*(3.5900)},{\sy*(-0.0001)})
	--({\sx*(3.6000)},{\sy*(-0.0001)})
	--({\sx*(3.6100)},{\sy*(-0.0001)})
	--({\sx*(3.6200)},{\sy*(-0.0001)})
	--({\sx*(3.6300)},{\sy*(-0.0001)})
	--({\sx*(3.6400)},{\sy*(-0.0001)})
	--({\sx*(3.6500)},{\sy*(-0.0001)})
	--({\sx*(3.6600)},{\sy*(-0.0001)})
	--({\sx*(3.6700)},{\sy*(-0.0001)})
	--({\sx*(3.6800)},{\sy*(-0.0001)})
	--({\sx*(3.6900)},{\sy*(-0.0001)})
	--({\sx*(3.7000)},{\sy*(-0.0001)})
	--({\sx*(3.7100)},{\sy*(-0.0001)})
	--({\sx*(3.7200)},{\sy*(-0.0001)})
	--({\sx*(3.7300)},{\sy*(-0.0001)})
	--({\sx*(3.7400)},{\sy*(-0.0000)})
	--({\sx*(3.7500)},{\sy*(0.0000)})
	--({\sx*(3.7600)},{\sy*(0.0000)})
	--({\sx*(3.7700)},{\sy*(0.0001)})
	--({\sx*(3.7800)},{\sy*(0.0001)})
	--({\sx*(3.7900)},{\sy*(0.0001)})
	--({\sx*(3.8000)},{\sy*(0.0002)})
	--({\sx*(3.8100)},{\sy*(0.0002)})
	--({\sx*(3.8200)},{\sy*(0.0002)})
	--({\sx*(3.8300)},{\sy*(0.0003)})
	--({\sx*(3.8400)},{\sy*(0.0003)})
	--({\sx*(3.8500)},{\sy*(0.0003)})
	--({\sx*(3.8600)},{\sy*(0.0003)})
	--({\sx*(3.8700)},{\sy*(0.0004)})
	--({\sx*(3.8800)},{\sy*(0.0004)})
	--({\sx*(3.8900)},{\sy*(0.0004)})
	--({\sx*(3.9000)},{\sy*(0.0004)})
	--({\sx*(3.9100)},{\sy*(0.0004)})
	--({\sx*(3.9200)},{\sy*(0.0004)})
	--({\sx*(3.9300)},{\sy*(0.0004)})
	--({\sx*(3.9400)},{\sy*(0.0003)})
	--({\sx*(3.9500)},{\sy*(0.0003)})
	--({\sx*(3.9600)},{\sy*(0.0003)})
	--({\sx*(3.9700)},{\sy*(0.0002)})
	--({\sx*(3.9800)},{\sy*(0.0002)})
	--({\sx*(3.9900)},{\sy*(0.0001)})
	--({\sx*(4.0000)},{\sy*(0.0000)})
	--({\sx*(4.0100)},{\sy*(-0.0001)})
	--({\sx*(4.0200)},{\sy*(-0.0002)})
	--({\sx*(4.0300)},{\sy*(-0.0003)})
	--({\sx*(4.0400)},{\sy*(-0.0004)})
	--({\sx*(4.0500)},{\sy*(-0.0005)})
	--({\sx*(4.0600)},{\sy*(-0.0006)})
	--({\sx*(4.0700)},{\sy*(-0.0008)})
	--({\sx*(4.0800)},{\sy*(-0.0009)})
	--({\sx*(4.0900)},{\sy*(-0.0010)})
	--({\sx*(4.1000)},{\sy*(-0.0011)})
	--({\sx*(4.1100)},{\sy*(-0.0012)})
	--({\sx*(4.1200)},{\sy*(-0.0013)})
	--({\sx*(4.1300)},{\sy*(-0.0014)})
	--({\sx*(4.1400)},{\sy*(-0.0014)})
	--({\sx*(4.1500)},{\sy*(-0.0015)})
	--({\sx*(4.1600)},{\sy*(-0.0015)})
	--({\sx*(4.1700)},{\sy*(-0.0015)})
	--({\sx*(4.1800)},{\sy*(-0.0014)})
	--({\sx*(4.1900)},{\sy*(-0.0013)})
	--({\sx*(4.2000)},{\sy*(-0.0012)})
	--({\sx*(4.2100)},{\sy*(-0.0011)})
	--({\sx*(4.2200)},{\sy*(-0.0009)})
	--({\sx*(4.2300)},{\sy*(-0.0006)})
	--({\sx*(4.2400)},{\sy*(-0.0003)})
	--({\sx*(4.2500)},{\sy*(0.0000)})
	--({\sx*(4.2600)},{\sy*(0.0004)})
	--({\sx*(4.2700)},{\sy*(0.0008)})
	--({\sx*(4.2800)},{\sy*(0.0012)})
	--({\sx*(4.2900)},{\sy*(0.0017)})
	--({\sx*(4.3000)},{\sy*(0.0023)})
	--({\sx*(4.3100)},{\sy*(0.0028)})
	--({\sx*(4.3200)},{\sy*(0.0034)})
	--({\sx*(4.3300)},{\sy*(0.0040)})
	--({\sx*(4.3400)},{\sy*(0.0046)})
	--({\sx*(4.3500)},{\sy*(0.0051)})
	--({\sx*(4.3600)},{\sy*(0.0057)})
	--({\sx*(4.3700)},{\sy*(0.0062)})
	--({\sx*(4.3800)},{\sy*(0.0066)})
	--({\sx*(4.3900)},{\sy*(0.0070)})
	--({\sx*(4.4000)},{\sy*(0.0073)})
	--({\sx*(4.4100)},{\sy*(0.0074)})
	--({\sx*(4.4200)},{\sy*(0.0075)})
	--({\sx*(4.4300)},{\sy*(0.0073)})
	--({\sx*(4.4400)},{\sy*(0.0070)})
	--({\sx*(4.4500)},{\sy*(0.0065)})
	--({\sx*(4.4600)},{\sy*(0.0057)})
	--({\sx*(4.4700)},{\sy*(0.0047)})
	--({\sx*(4.4800)},{\sy*(0.0034)})
	--({\sx*(4.4900)},{\sy*(0.0019)})
	--({\sx*(4.5000)},{\sy*(0.0000)})
	--({\sx*(4.5100)},{\sy*(-0.0022)})
	--({\sx*(4.5200)},{\sy*(-0.0047)})
	--({\sx*(4.5300)},{\sy*(-0.0076)})
	--({\sx*(4.5400)},{\sy*(-0.0107)})
	--({\sx*(4.5500)},{\sy*(-0.0142)})
	--({\sx*(4.5600)},{\sy*(-0.0180)})
	--({\sx*(4.5700)},{\sy*(-0.0220)})
	--({\sx*(4.5800)},{\sy*(-0.0262)})
	--({\sx*(4.5900)},{\sy*(-0.0305)})
	--({\sx*(4.6000)},{\sy*(-0.0350)})
	--({\sx*(4.6100)},{\sy*(-0.0394)})
	--({\sx*(4.6200)},{\sy*(-0.0436)})
	--({\sx*(4.6300)},{\sy*(-0.0476)})
	--({\sx*(4.6400)},{\sy*(-0.0513)})
	--({\sx*(4.6500)},{\sy*(-0.0543)})
	--({\sx*(4.6600)},{\sy*(-0.0566)})
	--({\sx*(4.6700)},{\sy*(-0.0578)})
	--({\sx*(4.6800)},{\sy*(-0.0579)})
	--({\sx*(4.6900)},{\sy*(-0.0565)})
	--({\sx*(4.7000)},{\sy*(-0.0533)})
	--({\sx*(4.7100)},{\sy*(-0.0481)})
	--({\sx*(4.7200)},{\sy*(-0.0405)})
	--({\sx*(4.7300)},{\sy*(-0.0301)})
	--({\sx*(4.7400)},{\sy*(-0.0168)})
	--({\sx*(4.7500)},{\sy*(0.0000)})
	--({\sx*(4.7600)},{\sy*(0.0205)})
	--({\sx*(4.7700)},{\sy*(0.0450)})
	--({\sx*(4.7800)},{\sy*(0.0739)})
	--({\sx*(4.7900)},{\sy*(0.1073)})
	--({\sx*(4.8000)},{\sy*(0.1455)})
	--({\sx*(4.8100)},{\sy*(0.1885)})
	--({\sx*(4.8200)},{\sy*(0.2365)})
	--({\sx*(4.8300)},{\sy*(0.2891)})
	--({\sx*(4.8400)},{\sy*(0.3461)})
	--({\sx*(4.8500)},{\sy*(0.4070)})
	--({\sx*(4.8600)},{\sy*(0.4710)})
	--({\sx*(4.8700)},{\sy*(0.5369)})
	--({\sx*(4.8800)},{\sy*(0.6034)})
	--({\sx*(4.8900)},{\sy*(0.6685)})
	--({\sx*(4.9000)},{\sy*(0.7297)})
	--({\sx*(4.9100)},{\sy*(0.7841)})
	--({\sx*(4.9200)},{\sy*(0.8279)})
	--({\sx*(4.9300)},{\sy*(0.8564)})
	--({\sx*(4.9400)},{\sy*(0.8641)})
	--({\sx*(4.9500)},{\sy*(0.8444)})
	--({\sx*(4.9600)},{\sy*(0.7892)})
	--({\sx*(4.9700)},{\sy*(0.6893)})
	--({\sx*(4.9800)},{\sy*(0.5334)})
	--({\sx*(4.9900)},{\sy*(0.3087)})
	--({\sx*(5.0000)},{\sy*(0.0000)});
}
\def\relfehlerj{
\draw[color=blue,line width=1.4pt,line join=round] ({\sx*(0.000)},{\sy*(0.0000)})
	--({\sx*(0.0100)},{\sy*(-0.0000)})
	--({\sx*(0.0200)},{\sy*(-0.0000)})
	--({\sx*(0.0300)},{\sy*(-0.0000)})
	--({\sx*(0.0400)},{\sy*(-0.0000)})
	--({\sx*(0.0500)},{\sy*(-0.0000)})
	--({\sx*(0.0600)},{\sy*(-0.0000)})
	--({\sx*(0.0700)},{\sy*(-0.0000)})
	--({\sx*(0.0800)},{\sy*(-0.0000)})
	--({\sx*(0.0900)},{\sy*(-0.0000)})
	--({\sx*(0.1000)},{\sy*(-0.0000)})
	--({\sx*(0.1100)},{\sy*(-0.0000)})
	--({\sx*(0.1200)},{\sy*(-0.0000)})
	--({\sx*(0.1300)},{\sy*(-0.0000)})
	--({\sx*(0.1400)},{\sy*(-0.0000)})
	--({\sx*(0.1500)},{\sy*(-0.0000)})
	--({\sx*(0.1600)},{\sy*(-0.0000)})
	--({\sx*(0.1700)},{\sy*(-0.0000)})
	--({\sx*(0.1800)},{\sy*(-0.0000)})
	--({\sx*(0.1900)},{\sy*(-0.0000)})
	--({\sx*(0.2000)},{\sy*(-0.0000)})
	--({\sx*(0.2100)},{\sy*(-0.0000)})
	--({\sx*(0.2200)},{\sy*(-0.0000)})
	--({\sx*(0.2300)},{\sy*(-0.0000)})
	--({\sx*(0.2400)},{\sy*(-0.0000)})
	--({\sx*(0.2500)},{\sy*(0.0000)})
	--({\sx*(0.2600)},{\sy*(0.0000)})
	--({\sx*(0.2700)},{\sy*(0.0000)})
	--({\sx*(0.2800)},{\sy*(0.0000)})
	--({\sx*(0.2900)},{\sy*(0.0000)})
	--({\sx*(0.3000)},{\sy*(0.0000)})
	--({\sx*(0.3100)},{\sy*(0.0000)})
	--({\sx*(0.3200)},{\sy*(0.0000)})
	--({\sx*(0.3300)},{\sy*(0.0000)})
	--({\sx*(0.3400)},{\sy*(0.0000)})
	--({\sx*(0.3500)},{\sy*(0.0000)})
	--({\sx*(0.3600)},{\sy*(0.0000)})
	--({\sx*(0.3700)},{\sy*(0.0000)})
	--({\sx*(0.3800)},{\sy*(0.0000)})
	--({\sx*(0.3900)},{\sy*(0.0000)})
	--({\sx*(0.4000)},{\sy*(0.0000)})
	--({\sx*(0.4100)},{\sy*(0.0000)})
	--({\sx*(0.4200)},{\sy*(0.0000)})
	--({\sx*(0.4300)},{\sy*(0.0000)})
	--({\sx*(0.4400)},{\sy*(0.0000)})
	--({\sx*(0.4500)},{\sy*(0.0000)})
	--({\sx*(0.4600)},{\sy*(0.0000)})
	--({\sx*(0.4700)},{\sy*(0.0000)})
	--({\sx*(0.4800)},{\sy*(0.0000)})
	--({\sx*(0.4900)},{\sy*(0.0000)})
	--({\sx*(0.5000)},{\sy*(0.0000)})
	--({\sx*(0.5100)},{\sy*(-0.0000)})
	--({\sx*(0.5200)},{\sy*(-0.0000)})
	--({\sx*(0.5300)},{\sy*(-0.0000)})
	--({\sx*(0.5400)},{\sy*(-0.0000)})
	--({\sx*(0.5500)},{\sy*(-0.0000)})
	--({\sx*(0.5600)},{\sy*(-0.0000)})
	--({\sx*(0.5700)},{\sy*(-0.0000)})
	--({\sx*(0.5800)},{\sy*(-0.0000)})
	--({\sx*(0.5900)},{\sy*(-0.0000)})
	--({\sx*(0.6000)},{\sy*(-0.0000)})
	--({\sx*(0.6100)},{\sy*(-0.0000)})
	--({\sx*(0.6200)},{\sy*(-0.0000)})
	--({\sx*(0.6300)},{\sy*(-0.0000)})
	--({\sx*(0.6400)},{\sy*(-0.0000)})
	--({\sx*(0.6500)},{\sy*(-0.0000)})
	--({\sx*(0.6600)},{\sy*(-0.0000)})
	--({\sx*(0.6700)},{\sy*(-0.0000)})
	--({\sx*(0.6800)},{\sy*(-0.0000)})
	--({\sx*(0.6900)},{\sy*(-0.0000)})
	--({\sx*(0.7000)},{\sy*(-0.0000)})
	--({\sx*(0.7100)},{\sy*(-0.0000)})
	--({\sx*(0.7200)},{\sy*(-0.0000)})
	--({\sx*(0.7300)},{\sy*(-0.0000)})
	--({\sx*(0.7400)},{\sy*(-0.0000)})
	--({\sx*(0.7500)},{\sy*(0.0000)})
	--({\sx*(0.7600)},{\sy*(0.0000)})
	--({\sx*(0.7700)},{\sy*(0.0000)})
	--({\sx*(0.7800)},{\sy*(0.0000)})
	--({\sx*(0.7900)},{\sy*(0.0000)})
	--({\sx*(0.8000)},{\sy*(0.0000)})
	--({\sx*(0.8100)},{\sy*(0.0000)})
	--({\sx*(0.8200)},{\sy*(0.0000)})
	--({\sx*(0.8300)},{\sy*(0.0000)})
	--({\sx*(0.8400)},{\sy*(0.0000)})
	--({\sx*(0.8500)},{\sy*(0.0000)})
	--({\sx*(0.8600)},{\sy*(0.0000)})
	--({\sx*(0.8700)},{\sy*(0.0000)})
	--({\sx*(0.8800)},{\sy*(0.0000)})
	--({\sx*(0.8900)},{\sy*(0.0000)})
	--({\sx*(0.9000)},{\sy*(0.0000)})
	--({\sx*(0.9100)},{\sy*(0.0000)})
	--({\sx*(0.9200)},{\sy*(0.0000)})
	--({\sx*(0.9300)},{\sy*(0.0000)})
	--({\sx*(0.9400)},{\sy*(0.0000)})
	--({\sx*(0.9500)},{\sy*(0.0000)})
	--({\sx*(0.9600)},{\sy*(0.0000)})
	--({\sx*(0.9700)},{\sy*(0.0000)})
	--({\sx*(0.9800)},{\sy*(0.0000)})
	--({\sx*(0.9900)},{\sy*(0.0000)})
	--({\sx*(1.0000)},{\sy*(0.0000)})
	--({\sx*(1.0100)},{\sy*(-0.0000)})
	--({\sx*(1.0200)},{\sy*(-0.0000)})
	--({\sx*(1.0300)},{\sy*(-0.0000)})
	--({\sx*(1.0400)},{\sy*(-0.0000)})
	--({\sx*(1.0500)},{\sy*(-0.0000)})
	--({\sx*(1.0600)},{\sy*(-0.0000)})
	--({\sx*(1.0700)},{\sy*(-0.0000)})
	--({\sx*(1.0800)},{\sy*(-0.0000)})
	--({\sx*(1.0900)},{\sy*(-0.0000)})
	--({\sx*(1.1000)},{\sy*(-0.0000)})
	--({\sx*(1.1100)},{\sy*(-0.0000)})
	--({\sx*(1.1200)},{\sy*(-0.0000)})
	--({\sx*(1.1300)},{\sy*(-0.0000)})
	--({\sx*(1.1400)},{\sy*(-0.0000)})
	--({\sx*(1.1500)},{\sy*(-0.0000)})
	--({\sx*(1.1600)},{\sy*(-0.0000)})
	--({\sx*(1.1700)},{\sy*(-0.0000)})
	--({\sx*(1.1800)},{\sy*(-0.0000)})
	--({\sx*(1.1900)},{\sy*(-0.0000)})
	--({\sx*(1.2000)},{\sy*(-0.0000)})
	--({\sx*(1.2100)},{\sy*(-0.0000)})
	--({\sx*(1.2200)},{\sy*(-0.0000)})
	--({\sx*(1.2300)},{\sy*(-0.0000)})
	--({\sx*(1.2400)},{\sy*(-0.0000)})
	--({\sx*(1.2500)},{\sy*(0.0000)})
	--({\sx*(1.2600)},{\sy*(0.0000)})
	--({\sx*(1.2700)},{\sy*(0.0000)})
	--({\sx*(1.2800)},{\sy*(0.0000)})
	--({\sx*(1.2900)},{\sy*(0.0000)})
	--({\sx*(1.3000)},{\sy*(0.0000)})
	--({\sx*(1.3100)},{\sy*(0.0000)})
	--({\sx*(1.3200)},{\sy*(0.0000)})
	--({\sx*(1.3300)},{\sy*(0.0000)})
	--({\sx*(1.3400)},{\sy*(0.0000)})
	--({\sx*(1.3500)},{\sy*(0.0000)})
	--({\sx*(1.3600)},{\sy*(0.0000)})
	--({\sx*(1.3700)},{\sy*(0.0000)})
	--({\sx*(1.3800)},{\sy*(0.0000)})
	--({\sx*(1.3900)},{\sy*(0.0000)})
	--({\sx*(1.4000)},{\sy*(0.0000)})
	--({\sx*(1.4100)},{\sy*(0.0000)})
	--({\sx*(1.4200)},{\sy*(0.0000)})
	--({\sx*(1.4300)},{\sy*(0.0000)})
	--({\sx*(1.4400)},{\sy*(0.0000)})
	--({\sx*(1.4500)},{\sy*(0.0000)})
	--({\sx*(1.4600)},{\sy*(0.0000)})
	--({\sx*(1.4700)},{\sy*(0.0000)})
	--({\sx*(1.4800)},{\sy*(0.0000)})
	--({\sx*(1.4900)},{\sy*(0.0000)})
	--({\sx*(1.5000)},{\sy*(0.0000)})
	--({\sx*(1.5100)},{\sy*(-0.0000)})
	--({\sx*(1.5200)},{\sy*(-0.0000)})
	--({\sx*(1.5300)},{\sy*(-0.0000)})
	--({\sx*(1.5400)},{\sy*(-0.0000)})
	--({\sx*(1.5500)},{\sy*(-0.0000)})
	--({\sx*(1.5600)},{\sy*(-0.0000)})
	--({\sx*(1.5700)},{\sy*(-0.0000)})
	--({\sx*(1.5800)},{\sy*(-0.0000)})
	--({\sx*(1.5900)},{\sy*(-0.0000)})
	--({\sx*(1.6000)},{\sy*(-0.0000)})
	--({\sx*(1.6100)},{\sy*(-0.0000)})
	--({\sx*(1.6200)},{\sy*(-0.0000)})
	--({\sx*(1.6300)},{\sy*(-0.0000)})
	--({\sx*(1.6400)},{\sy*(-0.0000)})
	--({\sx*(1.6500)},{\sy*(-0.0000)})
	--({\sx*(1.6600)},{\sy*(-0.0000)})
	--({\sx*(1.6700)},{\sy*(-0.0000)})
	--({\sx*(1.6800)},{\sy*(-0.0000)})
	--({\sx*(1.6900)},{\sy*(-0.0000)})
	--({\sx*(1.7000)},{\sy*(-0.0000)})
	--({\sx*(1.7100)},{\sy*(-0.0000)})
	--({\sx*(1.7200)},{\sy*(-0.0000)})
	--({\sx*(1.7300)},{\sy*(-0.0000)})
	--({\sx*(1.7400)},{\sy*(-0.0000)})
	--({\sx*(1.7500)},{\sy*(0.0000)})
	--({\sx*(1.7600)},{\sy*(0.0000)})
	--({\sx*(1.7700)},{\sy*(0.0000)})
	--({\sx*(1.7800)},{\sy*(0.0000)})
	--({\sx*(1.7900)},{\sy*(0.0000)})
	--({\sx*(1.8000)},{\sy*(0.0000)})
	--({\sx*(1.8100)},{\sy*(0.0000)})
	--({\sx*(1.8200)},{\sy*(0.0000)})
	--({\sx*(1.8300)},{\sy*(0.0000)})
	--({\sx*(1.8400)},{\sy*(0.0000)})
	--({\sx*(1.8500)},{\sy*(0.0000)})
	--({\sx*(1.8600)},{\sy*(0.0000)})
	--({\sx*(1.8700)},{\sy*(0.0000)})
	--({\sx*(1.8800)},{\sy*(0.0000)})
	--({\sx*(1.8900)},{\sy*(0.0000)})
	--({\sx*(1.9000)},{\sy*(0.0000)})
	--({\sx*(1.9100)},{\sy*(0.0000)})
	--({\sx*(1.9200)},{\sy*(0.0000)})
	--({\sx*(1.9300)},{\sy*(0.0000)})
	--({\sx*(1.9400)},{\sy*(0.0000)})
	--({\sx*(1.9500)},{\sy*(0.0000)})
	--({\sx*(1.9600)},{\sy*(0.0000)})
	--({\sx*(1.9700)},{\sy*(0.0000)})
	--({\sx*(1.9800)},{\sy*(0.0000)})
	--({\sx*(1.9900)},{\sy*(0.0000)})
	--({\sx*(2.0000)},{\sy*(0.0000)})
	--({\sx*(2.0100)},{\sy*(-0.0000)})
	--({\sx*(2.0200)},{\sy*(-0.0000)})
	--({\sx*(2.0300)},{\sy*(-0.0000)})
	--({\sx*(2.0400)},{\sy*(-0.0000)})
	--({\sx*(2.0500)},{\sy*(-0.0000)})
	--({\sx*(2.0600)},{\sy*(-0.0000)})
	--({\sx*(2.0700)},{\sy*(-0.0000)})
	--({\sx*(2.0800)},{\sy*(-0.0000)})
	--({\sx*(2.0900)},{\sy*(-0.0000)})
	--({\sx*(2.1000)},{\sy*(-0.0000)})
	--({\sx*(2.1100)},{\sy*(-0.0000)})
	--({\sx*(2.1200)},{\sy*(-0.0000)})
	--({\sx*(2.1300)},{\sy*(-0.0000)})
	--({\sx*(2.1400)},{\sy*(-0.0000)})
	--({\sx*(2.1500)},{\sy*(-0.0000)})
	--({\sx*(2.1600)},{\sy*(-0.0000)})
	--({\sx*(2.1700)},{\sy*(-0.0000)})
	--({\sx*(2.1800)},{\sy*(-0.0000)})
	--({\sx*(2.1900)},{\sy*(-0.0000)})
	--({\sx*(2.2000)},{\sy*(-0.0000)})
	--({\sx*(2.2100)},{\sy*(-0.0000)})
	--({\sx*(2.2200)},{\sy*(-0.0000)})
	--({\sx*(2.2300)},{\sy*(-0.0000)})
	--({\sx*(2.2400)},{\sy*(-0.0000)})
	--({\sx*(2.2500)},{\sy*(0.0000)})
	--({\sx*(2.2600)},{\sy*(0.0000)})
	--({\sx*(2.2700)},{\sy*(0.0000)})
	--({\sx*(2.2800)},{\sy*(0.0000)})
	--({\sx*(2.2900)},{\sy*(0.0000)})
	--({\sx*(2.3000)},{\sy*(0.0000)})
	--({\sx*(2.3100)},{\sy*(0.0000)})
	--({\sx*(2.3200)},{\sy*(0.0000)})
	--({\sx*(2.3300)},{\sy*(0.0000)})
	--({\sx*(2.3400)},{\sy*(0.0000)})
	--({\sx*(2.3500)},{\sy*(0.0000)})
	--({\sx*(2.3600)},{\sy*(0.0000)})
	--({\sx*(2.3700)},{\sy*(0.0000)})
	--({\sx*(2.3800)},{\sy*(0.0000)})
	--({\sx*(2.3900)},{\sy*(0.0000)})
	--({\sx*(2.4000)},{\sy*(0.0000)})
	--({\sx*(2.4100)},{\sy*(0.0000)})
	--({\sx*(2.4200)},{\sy*(0.0000)})
	--({\sx*(2.4300)},{\sy*(0.0000)})
	--({\sx*(2.4400)},{\sy*(0.0000)})
	--({\sx*(2.4500)},{\sy*(0.0000)})
	--({\sx*(2.4600)},{\sy*(0.0000)})
	--({\sx*(2.4700)},{\sy*(0.0000)})
	--({\sx*(2.4800)},{\sy*(0.0000)})
	--({\sx*(2.4900)},{\sy*(0.0000)})
	--({\sx*(2.5000)},{\sy*(0.0000)})
	--({\sx*(2.5100)},{\sy*(-0.0000)})
	--({\sx*(2.5200)},{\sy*(-0.0000)})
	--({\sx*(2.5300)},{\sy*(-0.0000)})
	--({\sx*(2.5400)},{\sy*(-0.0000)})
	--({\sx*(2.5500)},{\sy*(-0.0000)})
	--({\sx*(2.5600)},{\sy*(-0.0000)})
	--({\sx*(2.5700)},{\sy*(-0.0000)})
	--({\sx*(2.5800)},{\sy*(-0.0000)})
	--({\sx*(2.5900)},{\sy*(-0.0000)})
	--({\sx*(2.6000)},{\sy*(-0.0000)})
	--({\sx*(2.6100)},{\sy*(-0.0000)})
	--({\sx*(2.6200)},{\sy*(-0.0000)})
	--({\sx*(2.6300)},{\sy*(-0.0000)})
	--({\sx*(2.6400)},{\sy*(-0.0000)})
	--({\sx*(2.6500)},{\sy*(-0.0000)})
	--({\sx*(2.6600)},{\sy*(-0.0000)})
	--({\sx*(2.6700)},{\sy*(-0.0000)})
	--({\sx*(2.6800)},{\sy*(-0.0000)})
	--({\sx*(2.6900)},{\sy*(-0.0000)})
	--({\sx*(2.7000)},{\sy*(-0.0000)})
	--({\sx*(2.7100)},{\sy*(-0.0000)})
	--({\sx*(2.7200)},{\sy*(-0.0000)})
	--({\sx*(2.7300)},{\sy*(-0.0000)})
	--({\sx*(2.7400)},{\sy*(-0.0000)})
	--({\sx*(2.7500)},{\sy*(0.0000)})
	--({\sx*(2.7600)},{\sy*(0.0000)})
	--({\sx*(2.7700)},{\sy*(0.0000)})
	--({\sx*(2.7800)},{\sy*(0.0000)})
	--({\sx*(2.7900)},{\sy*(0.0000)})
	--({\sx*(2.8000)},{\sy*(0.0000)})
	--({\sx*(2.8100)},{\sy*(0.0000)})
	--({\sx*(2.8200)},{\sy*(0.0000)})
	--({\sx*(2.8300)},{\sy*(0.0000)})
	--({\sx*(2.8400)},{\sy*(0.0000)})
	--({\sx*(2.8500)},{\sy*(0.0000)})
	--({\sx*(2.8600)},{\sy*(0.0000)})
	--({\sx*(2.8700)},{\sy*(0.0000)})
	--({\sx*(2.8800)},{\sy*(0.0000)})
	--({\sx*(2.8900)},{\sy*(0.0000)})
	--({\sx*(2.9000)},{\sy*(0.0000)})
	--({\sx*(2.9100)},{\sy*(0.0000)})
	--({\sx*(2.9200)},{\sy*(0.0000)})
	--({\sx*(2.9300)},{\sy*(0.0000)})
	--({\sx*(2.9400)},{\sy*(0.0000)})
	--({\sx*(2.9500)},{\sy*(0.0000)})
	--({\sx*(2.9600)},{\sy*(0.0000)})
	--({\sx*(2.9700)},{\sy*(0.0000)})
	--({\sx*(2.9800)},{\sy*(0.0000)})
	--({\sx*(2.9900)},{\sy*(0.0000)})
	--({\sx*(3.0000)},{\sy*(0.0000)})
	--({\sx*(3.0100)},{\sy*(-0.0000)})
	--({\sx*(3.0200)},{\sy*(-0.0000)})
	--({\sx*(3.0300)},{\sy*(-0.0000)})
	--({\sx*(3.0400)},{\sy*(-0.0000)})
	--({\sx*(3.0500)},{\sy*(-0.0000)})
	--({\sx*(3.0600)},{\sy*(-0.0000)})
	--({\sx*(3.0700)},{\sy*(-0.0000)})
	--({\sx*(3.0800)},{\sy*(-0.0000)})
	--({\sx*(3.0900)},{\sy*(-0.0000)})
	--({\sx*(3.1000)},{\sy*(-0.0000)})
	--({\sx*(3.1100)},{\sy*(-0.0000)})
	--({\sx*(3.1200)},{\sy*(-0.0000)})
	--({\sx*(3.1300)},{\sy*(-0.0000)})
	--({\sx*(3.1400)},{\sy*(-0.0000)})
	--({\sx*(3.1500)},{\sy*(-0.0000)})
	--({\sx*(3.1600)},{\sy*(-0.0000)})
	--({\sx*(3.1700)},{\sy*(-0.0000)})
	--({\sx*(3.1800)},{\sy*(-0.0000)})
	--({\sx*(3.1900)},{\sy*(-0.0000)})
	--({\sx*(3.2000)},{\sy*(-0.0000)})
	--({\sx*(3.2100)},{\sy*(-0.0000)})
	--({\sx*(3.2200)},{\sy*(-0.0000)})
	--({\sx*(3.2300)},{\sy*(-0.0000)})
	--({\sx*(3.2400)},{\sy*(-0.0000)})
	--({\sx*(3.2500)},{\sy*(0.0000)})
	--({\sx*(3.2600)},{\sy*(0.0000)})
	--({\sx*(3.2700)},{\sy*(0.0000)})
	--({\sx*(3.2800)},{\sy*(0.0000)})
	--({\sx*(3.2900)},{\sy*(0.0000)})
	--({\sx*(3.3000)},{\sy*(0.0000)})
	--({\sx*(3.3100)},{\sy*(0.0000)})
	--({\sx*(3.3200)},{\sy*(0.0000)})
	--({\sx*(3.3300)},{\sy*(0.0000)})
	--({\sx*(3.3400)},{\sy*(0.0000)})
	--({\sx*(3.3500)},{\sy*(0.0000)})
	--({\sx*(3.3600)},{\sy*(0.0000)})
	--({\sx*(3.3700)},{\sy*(0.0000)})
	--({\sx*(3.3800)},{\sy*(0.0000)})
	--({\sx*(3.3900)},{\sy*(0.0000)})
	--({\sx*(3.4000)},{\sy*(0.0000)})
	--({\sx*(3.4100)},{\sy*(0.0000)})
	--({\sx*(3.4200)},{\sy*(0.0000)})
	--({\sx*(3.4300)},{\sy*(0.0000)})
	--({\sx*(3.4400)},{\sy*(0.0000)})
	--({\sx*(3.4500)},{\sy*(0.0000)})
	--({\sx*(3.4600)},{\sy*(0.0000)})
	--({\sx*(3.4700)},{\sy*(0.0000)})
	--({\sx*(3.4800)},{\sy*(0.0000)})
	--({\sx*(3.4900)},{\sy*(0.0000)})
	--({\sx*(3.5000)},{\sy*(0.0000)})
	--({\sx*(3.5100)},{\sy*(-0.0000)})
	--({\sx*(3.5200)},{\sy*(-0.0000)})
	--({\sx*(3.5300)},{\sy*(-0.0000)})
	--({\sx*(3.5400)},{\sy*(-0.0000)})
	--({\sx*(3.5500)},{\sy*(-0.0000)})
	--({\sx*(3.5600)},{\sy*(-0.0000)})
	--({\sx*(3.5700)},{\sy*(-0.0000)})
	--({\sx*(3.5800)},{\sy*(-0.0000)})
	--({\sx*(3.5900)},{\sy*(-0.0000)})
	--({\sx*(3.6000)},{\sy*(-0.0000)})
	--({\sx*(3.6100)},{\sy*(-0.0000)})
	--({\sx*(3.6200)},{\sy*(-0.0000)})
	--({\sx*(3.6300)},{\sy*(-0.0000)})
	--({\sx*(3.6400)},{\sy*(-0.0000)})
	--({\sx*(3.6500)},{\sy*(-0.0000)})
	--({\sx*(3.6600)},{\sy*(-0.0000)})
	--({\sx*(3.6700)},{\sy*(-0.0000)})
	--({\sx*(3.6800)},{\sy*(-0.0000)})
	--({\sx*(3.6900)},{\sy*(-0.0000)})
	--({\sx*(3.7000)},{\sy*(-0.0000)})
	--({\sx*(3.7100)},{\sy*(-0.0000)})
	--({\sx*(3.7200)},{\sy*(-0.0000)})
	--({\sx*(3.7300)},{\sy*(-0.0000)})
	--({\sx*(3.7400)},{\sy*(-0.0000)})
	--({\sx*(3.7500)},{\sy*(0.0000)})
	--({\sx*(3.7600)},{\sy*(0.0000)})
	--({\sx*(3.7700)},{\sy*(0.0000)})
	--({\sx*(3.7800)},{\sy*(0.0000)})
	--({\sx*(3.7900)},{\sy*(0.0000)})
	--({\sx*(3.8000)},{\sy*(0.0000)})
	--({\sx*(3.8100)},{\sy*(0.0000)})
	--({\sx*(3.8200)},{\sy*(0.0000)})
	--({\sx*(3.8300)},{\sy*(0.0000)})
	--({\sx*(3.8400)},{\sy*(0.0000)})
	--({\sx*(3.8500)},{\sy*(0.0000)})
	--({\sx*(3.8600)},{\sy*(0.0000)})
	--({\sx*(3.8700)},{\sy*(0.0000)})
	--({\sx*(3.8800)},{\sy*(0.0000)})
	--({\sx*(3.8900)},{\sy*(0.0000)})
	--({\sx*(3.9000)},{\sy*(0.0000)})
	--({\sx*(3.9100)},{\sy*(0.0000)})
	--({\sx*(3.9200)},{\sy*(0.0000)})
	--({\sx*(3.9300)},{\sy*(0.0000)})
	--({\sx*(3.9400)},{\sy*(0.0000)})
	--({\sx*(3.9500)},{\sy*(0.0000)})
	--({\sx*(3.9600)},{\sy*(0.0000)})
	--({\sx*(3.9700)},{\sy*(0.0000)})
	--({\sx*(3.9800)},{\sy*(0.0000)})
	--({\sx*(3.9900)},{\sy*(0.0000)})
	--({\sx*(4.0000)},{\sy*(0.0000)})
	--({\sx*(4.0100)},{\sy*(-0.0000)})
	--({\sx*(4.0200)},{\sy*(-0.0000)})
	--({\sx*(4.0300)},{\sy*(-0.0000)})
	--({\sx*(4.0400)},{\sy*(-0.0000)})
	--({\sx*(4.0500)},{\sy*(-0.0000)})
	--({\sx*(4.0600)},{\sy*(-0.0000)})
	--({\sx*(4.0700)},{\sy*(-0.0000)})
	--({\sx*(4.0800)},{\sy*(-0.0000)})
	--({\sx*(4.0900)},{\sy*(-0.0000)})
	--({\sx*(4.1000)},{\sy*(-0.0000)})
	--({\sx*(4.1100)},{\sy*(-0.0000)})
	--({\sx*(4.1200)},{\sy*(-0.0000)})
	--({\sx*(4.1300)},{\sy*(-0.0000)})
	--({\sx*(4.1400)},{\sy*(-0.0000)})
	--({\sx*(4.1500)},{\sy*(-0.0000)})
	--({\sx*(4.1600)},{\sy*(-0.0000)})
	--({\sx*(4.1700)},{\sy*(-0.0000)})
	--({\sx*(4.1800)},{\sy*(-0.0000)})
	--({\sx*(4.1900)},{\sy*(-0.0000)})
	--({\sx*(4.2000)},{\sy*(-0.0000)})
	--({\sx*(4.2100)},{\sy*(-0.0000)})
	--({\sx*(4.2200)},{\sy*(-0.0000)})
	--({\sx*(4.2300)},{\sy*(-0.0000)})
	--({\sx*(4.2400)},{\sy*(-0.0000)})
	--({\sx*(4.2500)},{\sy*(0.0000)})
	--({\sx*(4.2600)},{\sy*(0.0000)})
	--({\sx*(4.2700)},{\sy*(0.0000)})
	--({\sx*(4.2800)},{\sy*(0.0000)})
	--({\sx*(4.2900)},{\sy*(0.0000)})
	--({\sx*(4.3000)},{\sy*(0.0000)})
	--({\sx*(4.3100)},{\sy*(0.0000)})
	--({\sx*(4.3200)},{\sy*(0.0000)})
	--({\sx*(4.3300)},{\sy*(0.0000)})
	--({\sx*(4.3400)},{\sy*(0.0000)})
	--({\sx*(4.3500)},{\sy*(0.0000)})
	--({\sx*(4.3600)},{\sy*(0.0000)})
	--({\sx*(4.3700)},{\sy*(0.0000)})
	--({\sx*(4.3800)},{\sy*(0.0000)})
	--({\sx*(4.3900)},{\sy*(0.0000)})
	--({\sx*(4.4000)},{\sy*(0.0000)})
	--({\sx*(4.4100)},{\sy*(0.0000)})
	--({\sx*(4.4200)},{\sy*(0.0000)})
	--({\sx*(4.4300)},{\sy*(0.0000)})
	--({\sx*(4.4400)},{\sy*(0.0000)})
	--({\sx*(4.4500)},{\sy*(0.0000)})
	--({\sx*(4.4600)},{\sy*(0.0000)})
	--({\sx*(4.4700)},{\sy*(0.0000)})
	--({\sx*(4.4800)},{\sy*(0.0000)})
	--({\sx*(4.4900)},{\sy*(0.0000)})
	--({\sx*(4.5000)},{\sy*(0.0000)})
	--({\sx*(4.5100)},{\sy*(-0.0000)})
	--({\sx*(4.5200)},{\sy*(-0.0000)})
	--({\sx*(4.5300)},{\sy*(-0.0001)})
	--({\sx*(4.5400)},{\sy*(-0.0001)})
	--({\sx*(4.5500)},{\sy*(-0.0001)})
	--({\sx*(4.5600)},{\sy*(-0.0001)})
	--({\sx*(4.5700)},{\sy*(-0.0002)})
	--({\sx*(4.5800)},{\sy*(-0.0002)})
	--({\sx*(4.5900)},{\sy*(-0.0003)})
	--({\sx*(4.6000)},{\sy*(-0.0003)})
	--({\sx*(4.6100)},{\sy*(-0.0004)})
	--({\sx*(4.6200)},{\sy*(-0.0004)})
	--({\sx*(4.6300)},{\sy*(-0.0005)})
	--({\sx*(4.6400)},{\sy*(-0.0006)})
	--({\sx*(4.6500)},{\sy*(-0.0006)})
	--({\sx*(4.6600)},{\sy*(-0.0007)})
	--({\sx*(4.6700)},{\sy*(-0.0007)})
	--({\sx*(4.6800)},{\sy*(-0.0008)})
	--({\sx*(4.6900)},{\sy*(-0.0008)})
	--({\sx*(4.7000)},{\sy*(-0.0008)})
	--({\sx*(4.7100)},{\sy*(-0.0007)})
	--({\sx*(4.7200)},{\sy*(-0.0007)})
	--({\sx*(4.7300)},{\sy*(-0.0005)})
	--({\sx*(4.7400)},{\sy*(-0.0003)})
	--({\sx*(4.7500)},{\sy*(0.0000)})
	--({\sx*(4.7600)},{\sy*(0.0004)})
	--({\sx*(4.7700)},{\sy*(0.0009)})
	--({\sx*(4.7800)},{\sy*(0.0016)})
	--({\sx*(4.7900)},{\sy*(0.0024)})
	--({\sx*(4.8000)},{\sy*(0.0034)})
	--({\sx*(4.8100)},{\sy*(0.0047)})
	--({\sx*(4.8200)},{\sy*(0.0061)})
	--({\sx*(4.8300)},{\sy*(0.0079)})
	--({\sx*(4.8400)},{\sy*(0.0099)})
	--({\sx*(4.8500)},{\sy*(0.0121)})
	--({\sx*(4.8600)},{\sy*(0.0147)})
	--({\sx*(4.8700)},{\sy*(0.0176)})
	--({\sx*(4.8800)},{\sy*(0.0207)})
	--({\sx*(4.8900)},{\sy*(0.0240)})
	--({\sx*(4.9000)},{\sy*(0.0274)})
	--({\sx*(4.9100)},{\sy*(0.0308)})
	--({\sx*(4.9200)},{\sy*(0.0340)})
	--({\sx*(4.9300)},{\sy*(0.0369)})
	--({\sx*(4.9400)},{\sy*(0.0390)})
	--({\sx*(4.9500)},{\sy*(0.0400)})
	--({\sx*(4.9600)},{\sy*(0.0393)})
	--({\sx*(4.9700)},{\sy*(0.0362)})
	--({\sx*(4.9800)},{\sy*(0.0296)})
	--({\sx*(4.9900)},{\sy*(0.0182)})
	--({\sx*(5.0000)},{\sy*(0.0000)});
}
\def\xwertek{
\fill[color=red] (0.0000,0) circle[radius={0.07/\skala}];
\fill[color=red] (0.2273,0) circle[radius={0.07/\skala}];
\fill[color=red] (0.4545,0) circle[radius={0.07/\skala}];
\fill[color=red] (0.6818,0) circle[radius={0.07/\skala}];
\fill[color=red] (0.9091,0) circle[radius={0.07/\skala}];
\fill[color=red] (1.1364,0) circle[radius={0.07/\skala}];
\fill[color=red] (1.3636,0) circle[radius={0.07/\skala}];
\fill[color=red] (1.5909,0) circle[radius={0.07/\skala}];
\fill[color=red] (1.8182,0) circle[radius={0.07/\skala}];
\fill[color=red] (2.0455,0) circle[radius={0.07/\skala}];
\fill[color=red] (2.2727,0) circle[radius={0.07/\skala}];
\fill[color=red] (2.5000,0) circle[radius={0.07/\skala}];
\fill[color=red] (2.7273,0) circle[radius={0.07/\skala}];
\fill[color=red] (2.9545,0) circle[radius={0.07/\skala}];
\fill[color=red] (3.1818,0) circle[radius={0.07/\skala}];
\fill[color=red] (3.4091,0) circle[radius={0.07/\skala}];
\fill[color=red] (3.6364,0) circle[radius={0.07/\skala}];
\fill[color=red] (3.8636,0) circle[radius={0.07/\skala}];
\fill[color=red] (4.0909,0) circle[radius={0.07/\skala}];
\fill[color=red] (4.3182,0) circle[radius={0.07/\skala}];
\fill[color=red] (4.5455,0) circle[radius={0.07/\skala}];
\fill[color=red] (4.7727,0) circle[radius={0.07/\skala}];
\fill[color=red] (5.0000,0) circle[radius={0.07/\skala}];
}
\def\punktek{22}
\def\maxfehlerk{1.512\cdot 10^{-8}}
\def\fehlerk{
\draw[color=red,line width=1.4pt,line join=round] ({\sx*(0.000)},{\sy*(0.0000)})
	--({\sx*(0.0100)},{\sy*(0.3987)})
	--({\sx*(0.0200)},{\sy*(0.6748)})
	--({\sx*(0.0300)},{\sy*(0.8537)})
	--({\sx*(0.0400)},{\sy*(0.9563)})
	--({\sx*(0.0500)},{\sy*(1.0000)})
	--({\sx*(0.0600)},{\sy*(0.9994)})
	--({\sx*(0.0700)},{\sy*(0.9661)})
	--({\sx*(0.0800)},{\sy*(0.9099)})
	--({\sx*(0.0900)},{\sy*(0.8384)})
	--({\sx*(0.1000)},{\sy*(0.7578)})
	--({\sx*(0.1100)},{\sy*(0.6727)})
	--({\sx*(0.1200)},{\sy*(0.5870)})
	--({\sx*(0.1300)},{\sy*(0.5033)})
	--({\sx*(0.1400)},{\sy*(0.4236)})
	--({\sx*(0.1500)},{\sy*(0.3495)})
	--({\sx*(0.1600)},{\sy*(0.2816)})
	--({\sx*(0.1700)},{\sy*(0.2206)})
	--({\sx*(0.1800)},{\sy*(0.1666)})
	--({\sx*(0.1900)},{\sy*(0.1195)})
	--({\sx*(0.2000)},{\sy*(0.0791)})
	--({\sx*(0.2100)},{\sy*(0.0451)})
	--({\sx*(0.2200)},{\sy*(0.0170)})
	--({\sx*(0.2300)},{\sy*(-0.0057)})
	--({\sx*(0.2400)},{\sy*(-0.0236)})
	--({\sx*(0.2500)},{\sy*(-0.0371)})
	--({\sx*(0.2600)},{\sy*(-0.0469)})
	--({\sx*(0.2700)},{\sy*(-0.0535)})
	--({\sx*(0.2800)},{\sy*(-0.0573)})
	--({\sx*(0.2900)},{\sy*(-0.0588)})
	--({\sx*(0.3000)},{\sy*(-0.0585)})
	--({\sx*(0.3100)},{\sy*(-0.0567)})
	--({\sx*(0.3200)},{\sy*(-0.0537)})
	--({\sx*(0.3300)},{\sy*(-0.0499)})
	--({\sx*(0.3400)},{\sy*(-0.0455)})
	--({\sx*(0.3500)},{\sy*(-0.0408)})
	--({\sx*(0.3600)},{\sy*(-0.0358)})
	--({\sx*(0.3700)},{\sy*(-0.0309)})
	--({\sx*(0.3800)},{\sy*(-0.0261)})
	--({\sx*(0.3900)},{\sy*(-0.0214)})
	--({\sx*(0.4000)},{\sy*(-0.0171)})
	--({\sx*(0.4100)},{\sy*(-0.0131)})
	--({\sx*(0.4200)},{\sy*(-0.0095)})
	--({\sx*(0.4300)},{\sy*(-0.0062)})
	--({\sx*(0.4400)},{\sy*(-0.0034)})
	--({\sx*(0.4500)},{\sy*(-0.0010)})
	--({\sx*(0.4600)},{\sy*(0.0011)})
	--({\sx*(0.4700)},{\sy*(0.0027)})
	--({\sx*(0.4800)},{\sy*(0.0041)})
	--({\sx*(0.4900)},{\sy*(0.0051)})
	--({\sx*(0.5000)},{\sy*(0.0058)})
	--({\sx*(0.5100)},{\sy*(0.0063)})
	--({\sx*(0.5200)},{\sy*(0.0065)})
	--({\sx*(0.5300)},{\sy*(0.0066)})
	--({\sx*(0.5400)},{\sy*(0.0065)})
	--({\sx*(0.5500)},{\sy*(0.0062)})
	--({\sx*(0.5600)},{\sy*(0.0059)})
	--({\sx*(0.5700)},{\sy*(0.0054)})
	--({\sx*(0.5800)},{\sy*(0.0049)})
	--({\sx*(0.5900)},{\sy*(0.0044)})
	--({\sx*(0.6000)},{\sy*(0.0038)})
	--({\sx*(0.6100)},{\sy*(0.0033)})
	--({\sx*(0.6200)},{\sy*(0.0027)})
	--({\sx*(0.6300)},{\sy*(0.0022)})
	--({\sx*(0.6400)},{\sy*(0.0017)})
	--({\sx*(0.6500)},{\sy*(0.0012)})
	--({\sx*(0.6600)},{\sy*(0.0008)})
	--({\sx*(0.6700)},{\sy*(0.0004)})
	--({\sx*(0.6800)},{\sy*(0.0001)})
	--({\sx*(0.6900)},{\sy*(-0.0002)})
	--({\sx*(0.7000)},{\sy*(-0.0005)})
	--({\sx*(0.7100)},{\sy*(-0.0007)})
	--({\sx*(0.7200)},{\sy*(-0.0008)})
	--({\sx*(0.7300)},{\sy*(-0.0010)})
	--({\sx*(0.7400)},{\sy*(-0.0011)})
	--({\sx*(0.7500)},{\sy*(-0.0011)})
	--({\sx*(0.7600)},{\sy*(-0.0011)})
	--({\sx*(0.7700)},{\sy*(-0.0011)})
	--({\sx*(0.7800)},{\sy*(-0.0011)})
	--({\sx*(0.7900)},{\sy*(-0.0010)})
	--({\sx*(0.8000)},{\sy*(-0.0010)})
	--({\sx*(0.8100)},{\sy*(-0.0009)})
	--({\sx*(0.8200)},{\sy*(-0.0008)})
	--({\sx*(0.8300)},{\sy*(-0.0007)})
	--({\sx*(0.8400)},{\sy*(-0.0006)})
	--({\sx*(0.8500)},{\sy*(-0.0005)})
	--({\sx*(0.8600)},{\sy*(-0.0004)})
	--({\sx*(0.8700)},{\sy*(-0.0003)})
	--({\sx*(0.8800)},{\sy*(-0.0002)})
	--({\sx*(0.8900)},{\sy*(-0.0001)})
	--({\sx*(0.9000)},{\sy*(-0.0001)})
	--({\sx*(0.9100)},{\sy*(0.0000)})
	--({\sx*(0.9200)},{\sy*(0.0001)})
	--({\sx*(0.9300)},{\sy*(0.0001)})
	--({\sx*(0.9400)},{\sy*(0.0002)})
	--({\sx*(0.9500)},{\sy*(0.0002)})
	--({\sx*(0.9600)},{\sy*(0.0002)})
	--({\sx*(0.9700)},{\sy*(0.0002)})
	--({\sx*(0.9800)},{\sy*(0.0003)})
	--({\sx*(0.9900)},{\sy*(0.0003)})
	--({\sx*(1.0000)},{\sy*(0.0003)})
	--({\sx*(1.0100)},{\sy*(0.0003)})
	--({\sx*(1.0200)},{\sy*(0.0002)})
	--({\sx*(1.0300)},{\sy*(0.0002)})
	--({\sx*(1.0400)},{\sy*(0.0002)})
	--({\sx*(1.0500)},{\sy*(0.0002)})
	--({\sx*(1.0600)},{\sy*(0.0002)})
	--({\sx*(1.0700)},{\sy*(0.0001)})
	--({\sx*(1.0800)},{\sy*(0.0001)})
	--({\sx*(1.0900)},{\sy*(0.0001)})
	--({\sx*(1.1000)},{\sy*(0.0001)})
	--({\sx*(1.1100)},{\sy*(0.0001)})
	--({\sx*(1.1200)},{\sy*(0.0000)})
	--({\sx*(1.1300)},{\sy*(0.0000)})
	--({\sx*(1.1400)},{\sy*(-0.0000)})
	--({\sx*(1.1500)},{\sy*(-0.0000)})
	--({\sx*(1.1600)},{\sy*(-0.0000)})
	--({\sx*(1.1700)},{\sy*(-0.0000)})
	--({\sx*(1.1800)},{\sy*(-0.0001)})
	--({\sx*(1.1900)},{\sy*(-0.0001)})
	--({\sx*(1.2000)},{\sy*(-0.0001)})
	--({\sx*(1.2100)},{\sy*(-0.0001)})
	--({\sx*(1.2200)},{\sy*(-0.0001)})
	--({\sx*(1.2300)},{\sy*(-0.0001)})
	--({\sx*(1.2400)},{\sy*(-0.0001)})
	--({\sx*(1.2500)},{\sy*(-0.0001)})
	--({\sx*(1.2600)},{\sy*(-0.0001)})
	--({\sx*(1.2700)},{\sy*(-0.0001)})
	--({\sx*(1.2800)},{\sy*(-0.0001)})
	--({\sx*(1.2900)},{\sy*(-0.0001)})
	--({\sx*(1.3000)},{\sy*(-0.0000)})
	--({\sx*(1.3100)},{\sy*(-0.0000)})
	--({\sx*(1.3200)},{\sy*(-0.0000)})
	--({\sx*(1.3300)},{\sy*(-0.0000)})
	--({\sx*(1.3400)},{\sy*(-0.0000)})
	--({\sx*(1.3500)},{\sy*(-0.0000)})
	--({\sx*(1.3600)},{\sy*(-0.0000)})
	--({\sx*(1.3700)},{\sy*(0.0000)})
	--({\sx*(1.3800)},{\sy*(0.0000)})
	--({\sx*(1.3900)},{\sy*(0.0000)})
	--({\sx*(1.4000)},{\sy*(0.0000)})
	--({\sx*(1.4100)},{\sy*(0.0000)})
	--({\sx*(1.4200)},{\sy*(0.0000)})
	--({\sx*(1.4300)},{\sy*(0.0000)})
	--({\sx*(1.4400)},{\sy*(0.0000)})
	--({\sx*(1.4500)},{\sy*(0.0000)})
	--({\sx*(1.4600)},{\sy*(0.0000)})
	--({\sx*(1.4700)},{\sy*(0.0000)})
	--({\sx*(1.4800)},{\sy*(0.0000)})
	--({\sx*(1.4900)},{\sy*(0.0000)})
	--({\sx*(1.5000)},{\sy*(0.0000)})
	--({\sx*(1.5100)},{\sy*(0.0000)})
	--({\sx*(1.5200)},{\sy*(0.0000)})
	--({\sx*(1.5300)},{\sy*(0.0000)})
	--({\sx*(1.5400)},{\sy*(0.0000)})
	--({\sx*(1.5500)},{\sy*(0.0000)})
	--({\sx*(1.5600)},{\sy*(0.0000)})
	--({\sx*(1.5700)},{\sy*(0.0000)})
	--({\sx*(1.5800)},{\sy*(0.0000)})
	--({\sx*(1.5900)},{\sy*(0.0000)})
	--({\sx*(1.6000)},{\sy*(-0.0000)})
	--({\sx*(1.6100)},{\sy*(-0.0000)})
	--({\sx*(1.6200)},{\sy*(-0.0000)})
	--({\sx*(1.6300)},{\sy*(-0.0000)})
	--({\sx*(1.6400)},{\sy*(-0.0000)})
	--({\sx*(1.6500)},{\sy*(-0.0000)})
	--({\sx*(1.6600)},{\sy*(-0.0000)})
	--({\sx*(1.6700)},{\sy*(-0.0000)})
	--({\sx*(1.6800)},{\sy*(-0.0000)})
	--({\sx*(1.6900)},{\sy*(-0.0000)})
	--({\sx*(1.7000)},{\sy*(-0.0000)})
	--({\sx*(1.7100)},{\sy*(-0.0000)})
	--({\sx*(1.7200)},{\sy*(-0.0000)})
	--({\sx*(1.7300)},{\sy*(-0.0000)})
	--({\sx*(1.7400)},{\sy*(-0.0000)})
	--({\sx*(1.7500)},{\sy*(-0.0000)})
	--({\sx*(1.7600)},{\sy*(-0.0000)})
	--({\sx*(1.7700)},{\sy*(-0.0000)})
	--({\sx*(1.7800)},{\sy*(-0.0000)})
	--({\sx*(1.7900)},{\sy*(-0.0000)})
	--({\sx*(1.8000)},{\sy*(-0.0000)})
	--({\sx*(1.8100)},{\sy*(-0.0000)})
	--({\sx*(1.8200)},{\sy*(0.0000)})
	--({\sx*(1.8300)},{\sy*(0.0000)})
	--({\sx*(1.8400)},{\sy*(0.0000)})
	--({\sx*(1.8500)},{\sy*(0.0000)})
	--({\sx*(1.8600)},{\sy*(0.0000)})
	--({\sx*(1.8700)},{\sy*(0.0000)})
	--({\sx*(1.8800)},{\sy*(0.0000)})
	--({\sx*(1.8900)},{\sy*(0.0000)})
	--({\sx*(1.9000)},{\sy*(0.0000)})
	--({\sx*(1.9100)},{\sy*(0.0000)})
	--({\sx*(1.9200)},{\sy*(0.0000)})
	--({\sx*(1.9300)},{\sy*(0.0000)})
	--({\sx*(1.9400)},{\sy*(0.0000)})
	--({\sx*(1.9500)},{\sy*(0.0000)})
	--({\sx*(1.9600)},{\sy*(0.0000)})
	--({\sx*(1.9700)},{\sy*(0.0000)})
	--({\sx*(1.9800)},{\sy*(0.0000)})
	--({\sx*(1.9900)},{\sy*(0.0000)})
	--({\sx*(2.0000)},{\sy*(0.0000)})
	--({\sx*(2.0100)},{\sy*(0.0000)})
	--({\sx*(2.0200)},{\sy*(0.0000)})
	--({\sx*(2.0300)},{\sy*(0.0000)})
	--({\sx*(2.0400)},{\sy*(0.0000)})
	--({\sx*(2.0500)},{\sy*(-0.0000)})
	--({\sx*(2.0600)},{\sy*(-0.0000)})
	--({\sx*(2.0700)},{\sy*(-0.0000)})
	--({\sx*(2.0800)},{\sy*(-0.0000)})
	--({\sx*(2.0900)},{\sy*(-0.0000)})
	--({\sx*(2.1000)},{\sy*(-0.0000)})
	--({\sx*(2.1100)},{\sy*(-0.0000)})
	--({\sx*(2.1200)},{\sy*(-0.0000)})
	--({\sx*(2.1300)},{\sy*(-0.0000)})
	--({\sx*(2.1400)},{\sy*(-0.0000)})
	--({\sx*(2.1500)},{\sy*(-0.0000)})
	--({\sx*(2.1600)},{\sy*(-0.0000)})
	--({\sx*(2.1700)},{\sy*(-0.0000)})
	--({\sx*(2.1800)},{\sy*(-0.0000)})
	--({\sx*(2.1900)},{\sy*(-0.0000)})
	--({\sx*(2.2000)},{\sy*(-0.0000)})
	--({\sx*(2.2100)},{\sy*(-0.0000)})
	--({\sx*(2.2200)},{\sy*(-0.0000)})
	--({\sx*(2.2300)},{\sy*(-0.0000)})
	--({\sx*(2.2400)},{\sy*(-0.0000)})
	--({\sx*(2.2500)},{\sy*(-0.0000)})
	--({\sx*(2.2600)},{\sy*(-0.0000)})
	--({\sx*(2.2700)},{\sy*(-0.0000)})
	--({\sx*(2.2800)},{\sy*(0.0000)})
	--({\sx*(2.2900)},{\sy*(0.0000)})
	--({\sx*(2.3000)},{\sy*(0.0000)})
	--({\sx*(2.3100)},{\sy*(0.0000)})
	--({\sx*(2.3200)},{\sy*(0.0000)})
	--({\sx*(2.3300)},{\sy*(0.0000)})
	--({\sx*(2.3400)},{\sy*(0.0000)})
	--({\sx*(2.3500)},{\sy*(0.0000)})
	--({\sx*(2.3600)},{\sy*(0.0000)})
	--({\sx*(2.3700)},{\sy*(0.0000)})
	--({\sx*(2.3800)},{\sy*(0.0000)})
	--({\sx*(2.3900)},{\sy*(0.0000)})
	--({\sx*(2.4000)},{\sy*(0.0000)})
	--({\sx*(2.4100)},{\sy*(0.0000)})
	--({\sx*(2.4200)},{\sy*(0.0000)})
	--({\sx*(2.4300)},{\sy*(0.0000)})
	--({\sx*(2.4400)},{\sy*(0.0000)})
	--({\sx*(2.4500)},{\sy*(0.0000)})
	--({\sx*(2.4600)},{\sy*(0.0000)})
	--({\sx*(2.4700)},{\sy*(0.0000)})
	--({\sx*(2.4800)},{\sy*(0.0000)})
	--({\sx*(2.4900)},{\sy*(0.0000)})
	--({\sx*(2.5000)},{\sy*(0.0000)})
	--({\sx*(2.5100)},{\sy*(-0.0000)})
	--({\sx*(2.5200)},{\sy*(-0.0000)})
	--({\sx*(2.5300)},{\sy*(-0.0000)})
	--({\sx*(2.5400)},{\sy*(-0.0000)})
	--({\sx*(2.5500)},{\sy*(-0.0000)})
	--({\sx*(2.5600)},{\sy*(-0.0000)})
	--({\sx*(2.5700)},{\sy*(-0.0000)})
	--({\sx*(2.5800)},{\sy*(-0.0000)})
	--({\sx*(2.5900)},{\sy*(-0.0000)})
	--({\sx*(2.6000)},{\sy*(-0.0000)})
	--({\sx*(2.6100)},{\sy*(-0.0000)})
	--({\sx*(2.6200)},{\sy*(-0.0000)})
	--({\sx*(2.6300)},{\sy*(-0.0000)})
	--({\sx*(2.6400)},{\sy*(-0.0000)})
	--({\sx*(2.6500)},{\sy*(-0.0000)})
	--({\sx*(2.6600)},{\sy*(-0.0000)})
	--({\sx*(2.6700)},{\sy*(-0.0000)})
	--({\sx*(2.6800)},{\sy*(-0.0000)})
	--({\sx*(2.6900)},{\sy*(-0.0000)})
	--({\sx*(2.7000)},{\sy*(-0.0000)})
	--({\sx*(2.7100)},{\sy*(-0.0000)})
	--({\sx*(2.7200)},{\sy*(-0.0000)})
	--({\sx*(2.7300)},{\sy*(0.0000)})
	--({\sx*(2.7400)},{\sy*(0.0000)})
	--({\sx*(2.7500)},{\sy*(0.0000)})
	--({\sx*(2.7600)},{\sy*(0.0000)})
	--({\sx*(2.7700)},{\sy*(0.0000)})
	--({\sx*(2.7800)},{\sy*(0.0000)})
	--({\sx*(2.7900)},{\sy*(0.0000)})
	--({\sx*(2.8000)},{\sy*(0.0000)})
	--({\sx*(2.8100)},{\sy*(0.0000)})
	--({\sx*(2.8200)},{\sy*(0.0000)})
	--({\sx*(2.8300)},{\sy*(0.0000)})
	--({\sx*(2.8400)},{\sy*(0.0000)})
	--({\sx*(2.8500)},{\sy*(0.0000)})
	--({\sx*(2.8600)},{\sy*(0.0000)})
	--({\sx*(2.8700)},{\sy*(0.0000)})
	--({\sx*(2.8800)},{\sy*(0.0000)})
	--({\sx*(2.8900)},{\sy*(0.0000)})
	--({\sx*(2.9000)},{\sy*(0.0000)})
	--({\sx*(2.9100)},{\sy*(0.0000)})
	--({\sx*(2.9200)},{\sy*(0.0000)})
	--({\sx*(2.9300)},{\sy*(0.0000)})
	--({\sx*(2.9400)},{\sy*(0.0000)})
	--({\sx*(2.9500)},{\sy*(0.0000)})
	--({\sx*(2.9600)},{\sy*(-0.0000)})
	--({\sx*(2.9700)},{\sy*(-0.0000)})
	--({\sx*(2.9800)},{\sy*(-0.0000)})
	--({\sx*(2.9900)},{\sy*(-0.0000)})
	--({\sx*(3.0000)},{\sy*(-0.0000)})
	--({\sx*(3.0100)},{\sy*(-0.0000)})
	--({\sx*(3.0200)},{\sy*(-0.0000)})
	--({\sx*(3.0300)},{\sy*(-0.0000)})
	--({\sx*(3.0400)},{\sy*(-0.0000)})
	--({\sx*(3.0500)},{\sy*(-0.0000)})
	--({\sx*(3.0600)},{\sy*(-0.0000)})
	--({\sx*(3.0700)},{\sy*(-0.0000)})
	--({\sx*(3.0800)},{\sy*(-0.0000)})
	--({\sx*(3.0900)},{\sy*(-0.0000)})
	--({\sx*(3.1000)},{\sy*(-0.0000)})
	--({\sx*(3.1100)},{\sy*(-0.0000)})
	--({\sx*(3.1200)},{\sy*(-0.0000)})
	--({\sx*(3.1300)},{\sy*(-0.0000)})
	--({\sx*(3.1400)},{\sy*(-0.0000)})
	--({\sx*(3.1500)},{\sy*(-0.0000)})
	--({\sx*(3.1600)},{\sy*(-0.0000)})
	--({\sx*(3.1700)},{\sy*(-0.0000)})
	--({\sx*(3.1800)},{\sy*(-0.0000)})
	--({\sx*(3.1900)},{\sy*(0.0000)})
	--({\sx*(3.2000)},{\sy*(0.0000)})
	--({\sx*(3.2100)},{\sy*(0.0000)})
	--({\sx*(3.2200)},{\sy*(0.0000)})
	--({\sx*(3.2300)},{\sy*(0.0000)})
	--({\sx*(3.2400)},{\sy*(0.0000)})
	--({\sx*(3.2500)},{\sy*(0.0000)})
	--({\sx*(3.2600)},{\sy*(0.0000)})
	--({\sx*(3.2700)},{\sy*(0.0000)})
	--({\sx*(3.2800)},{\sy*(0.0000)})
	--({\sx*(3.2900)},{\sy*(0.0000)})
	--({\sx*(3.3000)},{\sy*(0.0000)})
	--({\sx*(3.3100)},{\sy*(0.0000)})
	--({\sx*(3.3200)},{\sy*(0.0000)})
	--({\sx*(3.3300)},{\sy*(0.0000)})
	--({\sx*(3.3400)},{\sy*(0.0000)})
	--({\sx*(3.3500)},{\sy*(0.0000)})
	--({\sx*(3.3600)},{\sy*(0.0000)})
	--({\sx*(3.3700)},{\sy*(0.0000)})
	--({\sx*(3.3800)},{\sy*(0.0000)})
	--({\sx*(3.3900)},{\sy*(0.0000)})
	--({\sx*(3.4000)},{\sy*(0.0000)})
	--({\sx*(3.4100)},{\sy*(-0.0000)})
	--({\sx*(3.4200)},{\sy*(-0.0000)})
	--({\sx*(3.4300)},{\sy*(-0.0000)})
	--({\sx*(3.4400)},{\sy*(-0.0000)})
	--({\sx*(3.4500)},{\sy*(-0.0000)})
	--({\sx*(3.4600)},{\sy*(-0.0000)})
	--({\sx*(3.4700)},{\sy*(-0.0000)})
	--({\sx*(3.4800)},{\sy*(-0.0000)})
	--({\sx*(3.4900)},{\sy*(-0.0000)})
	--({\sx*(3.5000)},{\sy*(-0.0000)})
	--({\sx*(3.5100)},{\sy*(-0.0000)})
	--({\sx*(3.5200)},{\sy*(-0.0000)})
	--({\sx*(3.5300)},{\sy*(-0.0000)})
	--({\sx*(3.5400)},{\sy*(-0.0000)})
	--({\sx*(3.5500)},{\sy*(-0.0000)})
	--({\sx*(3.5600)},{\sy*(-0.0000)})
	--({\sx*(3.5700)},{\sy*(-0.0000)})
	--({\sx*(3.5800)},{\sy*(-0.0000)})
	--({\sx*(3.5900)},{\sy*(-0.0000)})
	--({\sx*(3.6000)},{\sy*(-0.0000)})
	--({\sx*(3.6100)},{\sy*(-0.0000)})
	--({\sx*(3.6200)},{\sy*(-0.0000)})
	--({\sx*(3.6300)},{\sy*(-0.0000)})
	--({\sx*(3.6400)},{\sy*(0.0000)})
	--({\sx*(3.6500)},{\sy*(0.0000)})
	--({\sx*(3.6600)},{\sy*(0.0000)})
	--({\sx*(3.6700)},{\sy*(0.0000)})
	--({\sx*(3.6800)},{\sy*(0.0000)})
	--({\sx*(3.6900)},{\sy*(0.0000)})
	--({\sx*(3.7000)},{\sy*(0.0000)})
	--({\sx*(3.7100)},{\sy*(0.0000)})
	--({\sx*(3.7200)},{\sy*(0.0000)})
	--({\sx*(3.7300)},{\sy*(0.0000)})
	--({\sx*(3.7400)},{\sy*(0.0000)})
	--({\sx*(3.7500)},{\sy*(0.0000)})
	--({\sx*(3.7600)},{\sy*(0.0001)})
	--({\sx*(3.7700)},{\sy*(0.0001)})
	--({\sx*(3.7800)},{\sy*(0.0001)})
	--({\sx*(3.7900)},{\sy*(0.0001)})
	--({\sx*(3.8000)},{\sy*(0.0000)})
	--({\sx*(3.8100)},{\sy*(0.0000)})
	--({\sx*(3.8200)},{\sy*(0.0000)})
	--({\sx*(3.8300)},{\sy*(0.0000)})
	--({\sx*(3.8400)},{\sy*(0.0000)})
	--({\sx*(3.8500)},{\sy*(0.0000)})
	--({\sx*(3.8600)},{\sy*(0.0000)})
	--({\sx*(3.8700)},{\sy*(-0.0000)})
	--({\sx*(3.8800)},{\sy*(-0.0000)})
	--({\sx*(3.8900)},{\sy*(-0.0000)})
	--({\sx*(3.9000)},{\sy*(-0.0000)})
	--({\sx*(3.9100)},{\sy*(-0.0001)})
	--({\sx*(3.9200)},{\sy*(-0.0001)})
	--({\sx*(3.9300)},{\sy*(-0.0001)})
	--({\sx*(3.9400)},{\sy*(-0.0001)})
	--({\sx*(3.9500)},{\sy*(-0.0001)})
	--({\sx*(3.9600)},{\sy*(-0.0001)})
	--({\sx*(3.9700)},{\sy*(-0.0001)})
	--({\sx*(3.9800)},{\sy*(-0.0002)})
	--({\sx*(3.9900)},{\sy*(-0.0002)})
	--({\sx*(4.0000)},{\sy*(-0.0002)})
	--({\sx*(4.0100)},{\sy*(-0.0002)})
	--({\sx*(4.0200)},{\sy*(-0.0002)})
	--({\sx*(4.0300)},{\sy*(-0.0001)})
	--({\sx*(4.0400)},{\sy*(-0.0001)})
	--({\sx*(4.0500)},{\sy*(-0.0001)})
	--({\sx*(4.0600)},{\sy*(-0.0001)})
	--({\sx*(4.0700)},{\sy*(-0.0001)})
	--({\sx*(4.0800)},{\sy*(-0.0000)})
	--({\sx*(4.0900)},{\sy*(-0.0000)})
	--({\sx*(4.1000)},{\sy*(0.0000)})
	--({\sx*(4.1100)},{\sy*(0.0001)})
	--({\sx*(4.1200)},{\sy*(0.0001)})
	--({\sx*(4.1300)},{\sy*(0.0002)})
	--({\sx*(4.1400)},{\sy*(0.0002)})
	--({\sx*(4.1500)},{\sy*(0.0003)})
	--({\sx*(4.1600)},{\sy*(0.0004)})
	--({\sx*(4.1700)},{\sy*(0.0004)})
	--({\sx*(4.1800)},{\sy*(0.0005)})
	--({\sx*(4.1900)},{\sy*(0.0005)})
	--({\sx*(4.2000)},{\sy*(0.0006)})
	--({\sx*(4.2100)},{\sy*(0.0006)})
	--({\sx*(4.2200)},{\sy*(0.0006)})
	--({\sx*(4.2300)},{\sy*(0.0006)})
	--({\sx*(4.2400)},{\sy*(0.0006)})
	--({\sx*(4.2500)},{\sy*(0.0006)})
	--({\sx*(4.2600)},{\sy*(0.0006)})
	--({\sx*(4.2700)},{\sy*(0.0006)})
	--({\sx*(4.2800)},{\sy*(0.0005)})
	--({\sx*(4.2900)},{\sy*(0.0004)})
	--({\sx*(4.3000)},{\sy*(0.0003)})
	--({\sx*(4.3100)},{\sy*(0.0001)})
	--({\sx*(4.3200)},{\sy*(-0.0000)})
	--({\sx*(4.3300)},{\sy*(-0.0002)})
	--({\sx*(4.3400)},{\sy*(-0.0004)})
	--({\sx*(4.3500)},{\sy*(-0.0007)})
	--({\sx*(4.3600)},{\sy*(-0.0009)})
	--({\sx*(4.3700)},{\sy*(-0.0012)})
	--({\sx*(4.3800)},{\sy*(-0.0015)})
	--({\sx*(4.3900)},{\sy*(-0.0018)})
	--({\sx*(4.4000)},{\sy*(-0.0021)})
	--({\sx*(4.4100)},{\sy*(-0.0024)})
	--({\sx*(4.4200)},{\sy*(-0.0027)})
	--({\sx*(4.4300)},{\sy*(-0.0030)})
	--({\sx*(4.4400)},{\sy*(-0.0032)})
	--({\sx*(4.4500)},{\sy*(-0.0034)})
	--({\sx*(4.4600)},{\sy*(-0.0035)})
	--({\sx*(4.4700)},{\sy*(-0.0036)})
	--({\sx*(4.4800)},{\sy*(-0.0035)})
	--({\sx*(4.4900)},{\sy*(-0.0034)})
	--({\sx*(4.5000)},{\sy*(-0.0031)})
	--({\sx*(4.5100)},{\sy*(-0.0027)})
	--({\sx*(4.5200)},{\sy*(-0.0022)})
	--({\sx*(4.5300)},{\sy*(-0.0015)})
	--({\sx*(4.5400)},{\sy*(-0.0006)})
	--({\sx*(4.5500)},{\sy*(0.0005)})
	--({\sx*(4.5600)},{\sy*(0.0018)})
	--({\sx*(4.5700)},{\sy*(0.0033)})
	--({\sx*(4.5800)},{\sy*(0.0050)})
	--({\sx*(4.5900)},{\sy*(0.0069)})
	--({\sx*(4.6000)},{\sy*(0.0090)})
	--({\sx*(4.6100)},{\sy*(0.0112)})
	--({\sx*(4.6200)},{\sy*(0.0136)})
	--({\sx*(4.6300)},{\sy*(0.0161)})
	--({\sx*(4.6400)},{\sy*(0.0186)})
	--({\sx*(4.6500)},{\sy*(0.0211)})
	--({\sx*(4.6600)},{\sy*(0.0235)})
	--({\sx*(4.6700)},{\sy*(0.0257)})
	--({\sx*(4.6800)},{\sy*(0.0276)})
	--({\sx*(4.6900)},{\sy*(0.0291)})
	--({\sx*(4.7000)},{\sy*(0.0299)})
	--({\sx*(4.7100)},{\sy*(0.0300)})
	--({\sx*(4.7200)},{\sy*(0.0292)})
	--({\sx*(4.7300)},{\sy*(0.0272)})
	--({\sx*(4.7400)},{\sy*(0.0238)})
	--({\sx*(4.7500)},{\sy*(0.0188)})
	--({\sx*(4.7600)},{\sy*(0.0119)})
	--({\sx*(4.7700)},{\sy*(0.0029)})
	--({\sx*(4.7800)},{\sy*(-0.0085)})
	--({\sx*(4.7900)},{\sy*(-0.0226)})
	--({\sx*(4.8000)},{\sy*(-0.0395)})
	--({\sx*(4.8100)},{\sy*(-0.0595)})
	--({\sx*(4.8200)},{\sy*(-0.0828)})
	--({\sx*(4.8300)},{\sy*(-0.1094)})
	--({\sx*(4.8400)},{\sy*(-0.1393)})
	--({\sx*(4.8500)},{\sy*(-0.1724)})
	--({\sx*(4.8600)},{\sy*(-0.2086)})
	--({\sx*(4.8700)},{\sy*(-0.2472)})
	--({\sx*(4.8800)},{\sy*(-0.2876)})
	--({\sx*(4.8900)},{\sy*(-0.3288)})
	--({\sx*(4.9000)},{\sy*(-0.3695)})
	--({\sx*(4.9100)},{\sy*(-0.4079)})
	--({\sx*(4.9200)},{\sy*(-0.4417)})
	--({\sx*(4.9300)},{\sy*(-0.4678)})
	--({\sx*(4.9400)},{\sy*(-0.4828)})
	--({\sx*(4.9500)},{\sy*(-0.4820)})
	--({\sx*(4.9600)},{\sy*(-0.4599)})
	--({\sx*(4.9700)},{\sy*(-0.4096)})
	--({\sx*(4.9800)},{\sy*(-0.3231)})
	--({\sx*(4.9900)},{\sy*(-0.1904)})
	--({\sx*(5.0000)},{\sy*(0.0000)});
}
\def\relfehlerk{
\draw[color=blue,line width=1.4pt,line join=round] ({\sx*(0.000)},{\sy*(0.0000)})
	--({\sx*(0.0100)},{\sy*(0.0000)})
	--({\sx*(0.0200)},{\sy*(0.0000)})
	--({\sx*(0.0300)},{\sy*(0.0000)})
	--({\sx*(0.0400)},{\sy*(0.0000)})
	--({\sx*(0.0500)},{\sy*(0.0000)})
	--({\sx*(0.0600)},{\sy*(0.0000)})
	--({\sx*(0.0700)},{\sy*(0.0000)})
	--({\sx*(0.0800)},{\sy*(0.0000)})
	--({\sx*(0.0900)},{\sy*(0.0000)})
	--({\sx*(0.1000)},{\sy*(0.0000)})
	--({\sx*(0.1100)},{\sy*(0.0000)})
	--({\sx*(0.1200)},{\sy*(0.0000)})
	--({\sx*(0.1300)},{\sy*(0.0000)})
	--({\sx*(0.1400)},{\sy*(0.0000)})
	--({\sx*(0.1500)},{\sy*(0.0000)})
	--({\sx*(0.1600)},{\sy*(0.0000)})
	--({\sx*(0.1700)},{\sy*(0.0000)})
	--({\sx*(0.1800)},{\sy*(0.0000)})
	--({\sx*(0.1900)},{\sy*(0.0000)})
	--({\sx*(0.2000)},{\sy*(0.0000)})
	--({\sx*(0.2100)},{\sy*(0.0000)})
	--({\sx*(0.2200)},{\sy*(0.0000)})
	--({\sx*(0.2300)},{\sy*(-0.0000)})
	--({\sx*(0.2400)},{\sy*(-0.0000)})
	--({\sx*(0.2500)},{\sy*(-0.0000)})
	--({\sx*(0.2600)},{\sy*(-0.0000)})
	--({\sx*(0.2700)},{\sy*(-0.0000)})
	--({\sx*(0.2800)},{\sy*(-0.0000)})
	--({\sx*(0.2900)},{\sy*(-0.0000)})
	--({\sx*(0.3000)},{\sy*(-0.0000)})
	--({\sx*(0.3100)},{\sy*(-0.0000)})
	--({\sx*(0.3200)},{\sy*(-0.0000)})
	--({\sx*(0.3300)},{\sy*(-0.0000)})
	--({\sx*(0.3400)},{\sy*(-0.0000)})
	--({\sx*(0.3500)},{\sy*(-0.0000)})
	--({\sx*(0.3600)},{\sy*(-0.0000)})
	--({\sx*(0.3700)},{\sy*(-0.0000)})
	--({\sx*(0.3800)},{\sy*(-0.0000)})
	--({\sx*(0.3900)},{\sy*(-0.0000)})
	--({\sx*(0.4000)},{\sy*(-0.0000)})
	--({\sx*(0.4100)},{\sy*(-0.0000)})
	--({\sx*(0.4200)},{\sy*(-0.0000)})
	--({\sx*(0.4300)},{\sy*(-0.0000)})
	--({\sx*(0.4400)},{\sy*(-0.0000)})
	--({\sx*(0.4500)},{\sy*(-0.0000)})
	--({\sx*(0.4600)},{\sy*(0.0000)})
	--({\sx*(0.4700)},{\sy*(0.0000)})
	--({\sx*(0.4800)},{\sy*(0.0000)})
	--({\sx*(0.4900)},{\sy*(0.0000)})
	--({\sx*(0.5000)},{\sy*(0.0000)})
	--({\sx*(0.5100)},{\sy*(0.0000)})
	--({\sx*(0.5200)},{\sy*(0.0000)})
	--({\sx*(0.5300)},{\sy*(0.0000)})
	--({\sx*(0.5400)},{\sy*(0.0000)})
	--({\sx*(0.5500)},{\sy*(0.0000)})
	--({\sx*(0.5600)},{\sy*(0.0000)})
	--({\sx*(0.5700)},{\sy*(0.0000)})
	--({\sx*(0.5800)},{\sy*(0.0000)})
	--({\sx*(0.5900)},{\sy*(0.0000)})
	--({\sx*(0.6000)},{\sy*(0.0000)})
	--({\sx*(0.6100)},{\sy*(0.0000)})
	--({\sx*(0.6200)},{\sy*(0.0000)})
	--({\sx*(0.6300)},{\sy*(0.0000)})
	--({\sx*(0.6400)},{\sy*(0.0000)})
	--({\sx*(0.6500)},{\sy*(0.0000)})
	--({\sx*(0.6600)},{\sy*(0.0000)})
	--({\sx*(0.6700)},{\sy*(0.0000)})
	--({\sx*(0.6800)},{\sy*(0.0000)})
	--({\sx*(0.6900)},{\sy*(-0.0000)})
	--({\sx*(0.7000)},{\sy*(-0.0000)})
	--({\sx*(0.7100)},{\sy*(-0.0000)})
	--({\sx*(0.7200)},{\sy*(-0.0000)})
	--({\sx*(0.7300)},{\sy*(-0.0000)})
	--({\sx*(0.7400)},{\sy*(-0.0000)})
	--({\sx*(0.7500)},{\sy*(-0.0000)})
	--({\sx*(0.7600)},{\sy*(-0.0000)})
	--({\sx*(0.7700)},{\sy*(-0.0000)})
	--({\sx*(0.7800)},{\sy*(-0.0000)})
	--({\sx*(0.7900)},{\sy*(-0.0000)})
	--({\sx*(0.8000)},{\sy*(-0.0000)})
	--({\sx*(0.8100)},{\sy*(-0.0000)})
	--({\sx*(0.8200)},{\sy*(-0.0000)})
	--({\sx*(0.8300)},{\sy*(-0.0000)})
	--({\sx*(0.8400)},{\sy*(-0.0000)})
	--({\sx*(0.8500)},{\sy*(-0.0000)})
	--({\sx*(0.8600)},{\sy*(-0.0000)})
	--({\sx*(0.8700)},{\sy*(-0.0000)})
	--({\sx*(0.8800)},{\sy*(-0.0000)})
	--({\sx*(0.8900)},{\sy*(-0.0000)})
	--({\sx*(0.9000)},{\sy*(-0.0000)})
	--({\sx*(0.9100)},{\sy*(0.0000)})
	--({\sx*(0.9200)},{\sy*(0.0000)})
	--({\sx*(0.9300)},{\sy*(0.0000)})
	--({\sx*(0.9400)},{\sy*(0.0000)})
	--({\sx*(0.9500)},{\sy*(0.0000)})
	--({\sx*(0.9600)},{\sy*(0.0000)})
	--({\sx*(0.9700)},{\sy*(0.0000)})
	--({\sx*(0.9800)},{\sy*(0.0000)})
	--({\sx*(0.9900)},{\sy*(0.0000)})
	--({\sx*(1.0000)},{\sy*(0.0000)})
	--({\sx*(1.0100)},{\sy*(0.0000)})
	--({\sx*(1.0200)},{\sy*(0.0000)})
	--({\sx*(1.0300)},{\sy*(0.0000)})
	--({\sx*(1.0400)},{\sy*(0.0000)})
	--({\sx*(1.0500)},{\sy*(0.0000)})
	--({\sx*(1.0600)},{\sy*(0.0000)})
	--({\sx*(1.0700)},{\sy*(0.0000)})
	--({\sx*(1.0800)},{\sy*(0.0000)})
	--({\sx*(1.0900)},{\sy*(0.0000)})
	--({\sx*(1.1000)},{\sy*(0.0000)})
	--({\sx*(1.1100)},{\sy*(0.0000)})
	--({\sx*(1.1200)},{\sy*(0.0000)})
	--({\sx*(1.1300)},{\sy*(0.0000)})
	--({\sx*(1.1400)},{\sy*(-0.0000)})
	--({\sx*(1.1500)},{\sy*(-0.0000)})
	--({\sx*(1.1600)},{\sy*(-0.0000)})
	--({\sx*(1.1700)},{\sy*(-0.0000)})
	--({\sx*(1.1800)},{\sy*(-0.0000)})
	--({\sx*(1.1900)},{\sy*(-0.0000)})
	--({\sx*(1.2000)},{\sy*(-0.0000)})
	--({\sx*(1.2100)},{\sy*(-0.0000)})
	--({\sx*(1.2200)},{\sy*(-0.0000)})
	--({\sx*(1.2300)},{\sy*(-0.0000)})
	--({\sx*(1.2400)},{\sy*(-0.0000)})
	--({\sx*(1.2500)},{\sy*(-0.0000)})
	--({\sx*(1.2600)},{\sy*(-0.0000)})
	--({\sx*(1.2700)},{\sy*(-0.0000)})
	--({\sx*(1.2800)},{\sy*(-0.0000)})
	--({\sx*(1.2900)},{\sy*(-0.0000)})
	--({\sx*(1.3000)},{\sy*(-0.0000)})
	--({\sx*(1.3100)},{\sy*(-0.0000)})
	--({\sx*(1.3200)},{\sy*(-0.0000)})
	--({\sx*(1.3300)},{\sy*(-0.0000)})
	--({\sx*(1.3400)},{\sy*(-0.0000)})
	--({\sx*(1.3500)},{\sy*(-0.0000)})
	--({\sx*(1.3600)},{\sy*(-0.0000)})
	--({\sx*(1.3700)},{\sy*(0.0000)})
	--({\sx*(1.3800)},{\sy*(0.0000)})
	--({\sx*(1.3900)},{\sy*(0.0000)})
	--({\sx*(1.4000)},{\sy*(0.0000)})
	--({\sx*(1.4100)},{\sy*(0.0000)})
	--({\sx*(1.4200)},{\sy*(0.0000)})
	--({\sx*(1.4300)},{\sy*(0.0000)})
	--({\sx*(1.4400)},{\sy*(0.0000)})
	--({\sx*(1.4500)},{\sy*(0.0000)})
	--({\sx*(1.4600)},{\sy*(0.0000)})
	--({\sx*(1.4700)},{\sy*(0.0000)})
	--({\sx*(1.4800)},{\sy*(0.0000)})
	--({\sx*(1.4900)},{\sy*(0.0000)})
	--({\sx*(1.5000)},{\sy*(0.0000)})
	--({\sx*(1.5100)},{\sy*(0.0000)})
	--({\sx*(1.5200)},{\sy*(0.0000)})
	--({\sx*(1.5300)},{\sy*(0.0000)})
	--({\sx*(1.5400)},{\sy*(0.0000)})
	--({\sx*(1.5500)},{\sy*(0.0000)})
	--({\sx*(1.5600)},{\sy*(0.0000)})
	--({\sx*(1.5700)},{\sy*(0.0000)})
	--({\sx*(1.5800)},{\sy*(0.0000)})
	--({\sx*(1.5900)},{\sy*(0.0000)})
	--({\sx*(1.6000)},{\sy*(-0.0000)})
	--({\sx*(1.6100)},{\sy*(-0.0000)})
	--({\sx*(1.6200)},{\sy*(-0.0000)})
	--({\sx*(1.6300)},{\sy*(-0.0000)})
	--({\sx*(1.6400)},{\sy*(-0.0000)})
	--({\sx*(1.6500)},{\sy*(-0.0000)})
	--({\sx*(1.6600)},{\sy*(-0.0000)})
	--({\sx*(1.6700)},{\sy*(-0.0000)})
	--({\sx*(1.6800)},{\sy*(-0.0000)})
	--({\sx*(1.6900)},{\sy*(-0.0000)})
	--({\sx*(1.7000)},{\sy*(-0.0000)})
	--({\sx*(1.7100)},{\sy*(-0.0000)})
	--({\sx*(1.7200)},{\sy*(-0.0000)})
	--({\sx*(1.7300)},{\sy*(-0.0000)})
	--({\sx*(1.7400)},{\sy*(-0.0000)})
	--({\sx*(1.7500)},{\sy*(-0.0000)})
	--({\sx*(1.7600)},{\sy*(-0.0000)})
	--({\sx*(1.7700)},{\sy*(-0.0000)})
	--({\sx*(1.7800)},{\sy*(-0.0000)})
	--({\sx*(1.7900)},{\sy*(-0.0000)})
	--({\sx*(1.8000)},{\sy*(-0.0000)})
	--({\sx*(1.8100)},{\sy*(-0.0000)})
	--({\sx*(1.8200)},{\sy*(0.0000)})
	--({\sx*(1.8300)},{\sy*(0.0000)})
	--({\sx*(1.8400)},{\sy*(0.0000)})
	--({\sx*(1.8500)},{\sy*(0.0000)})
	--({\sx*(1.8600)},{\sy*(0.0000)})
	--({\sx*(1.8700)},{\sy*(0.0000)})
	--({\sx*(1.8800)},{\sy*(0.0000)})
	--({\sx*(1.8900)},{\sy*(0.0000)})
	--({\sx*(1.9000)},{\sy*(0.0000)})
	--({\sx*(1.9100)},{\sy*(0.0000)})
	--({\sx*(1.9200)},{\sy*(0.0000)})
	--({\sx*(1.9300)},{\sy*(0.0000)})
	--({\sx*(1.9400)},{\sy*(0.0000)})
	--({\sx*(1.9500)},{\sy*(0.0000)})
	--({\sx*(1.9600)},{\sy*(0.0000)})
	--({\sx*(1.9700)},{\sy*(0.0000)})
	--({\sx*(1.9800)},{\sy*(0.0000)})
	--({\sx*(1.9900)},{\sy*(0.0000)})
	--({\sx*(2.0000)},{\sy*(0.0000)})
	--({\sx*(2.0100)},{\sy*(0.0000)})
	--({\sx*(2.0200)},{\sy*(0.0000)})
	--({\sx*(2.0300)},{\sy*(0.0000)})
	--({\sx*(2.0400)},{\sy*(0.0000)})
	--({\sx*(2.0500)},{\sy*(-0.0000)})
	--({\sx*(2.0600)},{\sy*(-0.0000)})
	--({\sx*(2.0700)},{\sy*(-0.0000)})
	--({\sx*(2.0800)},{\sy*(-0.0000)})
	--({\sx*(2.0900)},{\sy*(-0.0000)})
	--({\sx*(2.1000)},{\sy*(-0.0000)})
	--({\sx*(2.1100)},{\sy*(-0.0000)})
	--({\sx*(2.1200)},{\sy*(-0.0000)})
	--({\sx*(2.1300)},{\sy*(-0.0000)})
	--({\sx*(2.1400)},{\sy*(-0.0000)})
	--({\sx*(2.1500)},{\sy*(-0.0000)})
	--({\sx*(2.1600)},{\sy*(-0.0000)})
	--({\sx*(2.1700)},{\sy*(-0.0000)})
	--({\sx*(2.1800)},{\sy*(-0.0000)})
	--({\sx*(2.1900)},{\sy*(-0.0000)})
	--({\sx*(2.2000)},{\sy*(-0.0000)})
	--({\sx*(2.2100)},{\sy*(-0.0000)})
	--({\sx*(2.2200)},{\sy*(-0.0000)})
	--({\sx*(2.2300)},{\sy*(-0.0000)})
	--({\sx*(2.2400)},{\sy*(-0.0000)})
	--({\sx*(2.2500)},{\sy*(-0.0000)})
	--({\sx*(2.2600)},{\sy*(-0.0000)})
	--({\sx*(2.2700)},{\sy*(-0.0000)})
	--({\sx*(2.2800)},{\sy*(0.0000)})
	--({\sx*(2.2900)},{\sy*(0.0000)})
	--({\sx*(2.3000)},{\sy*(0.0000)})
	--({\sx*(2.3100)},{\sy*(0.0000)})
	--({\sx*(2.3200)},{\sy*(0.0000)})
	--({\sx*(2.3300)},{\sy*(0.0000)})
	--({\sx*(2.3400)},{\sy*(0.0000)})
	--({\sx*(2.3500)},{\sy*(0.0000)})
	--({\sx*(2.3600)},{\sy*(0.0000)})
	--({\sx*(2.3700)},{\sy*(0.0000)})
	--({\sx*(2.3800)},{\sy*(0.0000)})
	--({\sx*(2.3900)},{\sy*(0.0000)})
	--({\sx*(2.4000)},{\sy*(0.0000)})
	--({\sx*(2.4100)},{\sy*(0.0000)})
	--({\sx*(2.4200)},{\sy*(0.0000)})
	--({\sx*(2.4300)},{\sy*(0.0000)})
	--({\sx*(2.4400)},{\sy*(0.0000)})
	--({\sx*(2.4500)},{\sy*(0.0000)})
	--({\sx*(2.4600)},{\sy*(0.0000)})
	--({\sx*(2.4700)},{\sy*(0.0000)})
	--({\sx*(2.4800)},{\sy*(0.0000)})
	--({\sx*(2.4900)},{\sy*(0.0000)})
	--({\sx*(2.5000)},{\sy*(0.0000)})
	--({\sx*(2.5100)},{\sy*(-0.0000)})
	--({\sx*(2.5200)},{\sy*(-0.0000)})
	--({\sx*(2.5300)},{\sy*(-0.0000)})
	--({\sx*(2.5400)},{\sy*(-0.0000)})
	--({\sx*(2.5500)},{\sy*(-0.0000)})
	--({\sx*(2.5600)},{\sy*(-0.0000)})
	--({\sx*(2.5700)},{\sy*(-0.0000)})
	--({\sx*(2.5800)},{\sy*(-0.0000)})
	--({\sx*(2.5900)},{\sy*(-0.0000)})
	--({\sx*(2.6000)},{\sy*(-0.0000)})
	--({\sx*(2.6100)},{\sy*(-0.0000)})
	--({\sx*(2.6200)},{\sy*(-0.0000)})
	--({\sx*(2.6300)},{\sy*(-0.0000)})
	--({\sx*(2.6400)},{\sy*(-0.0000)})
	--({\sx*(2.6500)},{\sy*(-0.0000)})
	--({\sx*(2.6600)},{\sy*(-0.0000)})
	--({\sx*(2.6700)},{\sy*(-0.0000)})
	--({\sx*(2.6800)},{\sy*(-0.0000)})
	--({\sx*(2.6900)},{\sy*(-0.0000)})
	--({\sx*(2.7000)},{\sy*(-0.0000)})
	--({\sx*(2.7100)},{\sy*(-0.0000)})
	--({\sx*(2.7200)},{\sy*(-0.0000)})
	--({\sx*(2.7300)},{\sy*(0.0000)})
	--({\sx*(2.7400)},{\sy*(0.0000)})
	--({\sx*(2.7500)},{\sy*(0.0000)})
	--({\sx*(2.7600)},{\sy*(0.0000)})
	--({\sx*(2.7700)},{\sy*(0.0000)})
	--({\sx*(2.7800)},{\sy*(0.0000)})
	--({\sx*(2.7900)},{\sy*(0.0000)})
	--({\sx*(2.8000)},{\sy*(0.0000)})
	--({\sx*(2.8100)},{\sy*(0.0000)})
	--({\sx*(2.8200)},{\sy*(0.0000)})
	--({\sx*(2.8300)},{\sy*(0.0000)})
	--({\sx*(2.8400)},{\sy*(0.0000)})
	--({\sx*(2.8500)},{\sy*(0.0000)})
	--({\sx*(2.8600)},{\sy*(0.0000)})
	--({\sx*(2.8700)},{\sy*(0.0000)})
	--({\sx*(2.8800)},{\sy*(0.0000)})
	--({\sx*(2.8900)},{\sy*(0.0000)})
	--({\sx*(2.9000)},{\sy*(0.0000)})
	--({\sx*(2.9100)},{\sy*(0.0000)})
	--({\sx*(2.9200)},{\sy*(0.0000)})
	--({\sx*(2.9300)},{\sy*(0.0000)})
	--({\sx*(2.9400)},{\sy*(0.0000)})
	--({\sx*(2.9500)},{\sy*(0.0000)})
	--({\sx*(2.9600)},{\sy*(-0.0000)})
	--({\sx*(2.9700)},{\sy*(-0.0000)})
	--({\sx*(2.9800)},{\sy*(-0.0000)})
	--({\sx*(2.9900)},{\sy*(-0.0000)})
	--({\sx*(3.0000)},{\sy*(-0.0000)})
	--({\sx*(3.0100)},{\sy*(-0.0000)})
	--({\sx*(3.0200)},{\sy*(-0.0000)})
	--({\sx*(3.0300)},{\sy*(-0.0000)})
	--({\sx*(3.0400)},{\sy*(-0.0000)})
	--({\sx*(3.0500)},{\sy*(-0.0000)})
	--({\sx*(3.0600)},{\sy*(-0.0000)})
	--({\sx*(3.0700)},{\sy*(-0.0000)})
	--({\sx*(3.0800)},{\sy*(-0.0000)})
	--({\sx*(3.0900)},{\sy*(-0.0000)})
	--({\sx*(3.1000)},{\sy*(-0.0000)})
	--({\sx*(3.1100)},{\sy*(-0.0000)})
	--({\sx*(3.1200)},{\sy*(-0.0000)})
	--({\sx*(3.1300)},{\sy*(-0.0000)})
	--({\sx*(3.1400)},{\sy*(-0.0000)})
	--({\sx*(3.1500)},{\sy*(-0.0000)})
	--({\sx*(3.1600)},{\sy*(-0.0000)})
	--({\sx*(3.1700)},{\sy*(-0.0000)})
	--({\sx*(3.1800)},{\sy*(-0.0000)})
	--({\sx*(3.1900)},{\sy*(0.0000)})
	--({\sx*(3.2000)},{\sy*(0.0000)})
	--({\sx*(3.2100)},{\sy*(0.0000)})
	--({\sx*(3.2200)},{\sy*(0.0000)})
	--({\sx*(3.2300)},{\sy*(0.0000)})
	--({\sx*(3.2400)},{\sy*(0.0000)})
	--({\sx*(3.2500)},{\sy*(0.0000)})
	--({\sx*(3.2600)},{\sy*(0.0000)})
	--({\sx*(3.2700)},{\sy*(0.0000)})
	--({\sx*(3.2800)},{\sy*(0.0000)})
	--({\sx*(3.2900)},{\sy*(0.0000)})
	--({\sx*(3.3000)},{\sy*(0.0000)})
	--({\sx*(3.3100)},{\sy*(0.0000)})
	--({\sx*(3.3200)},{\sy*(0.0000)})
	--({\sx*(3.3300)},{\sy*(0.0000)})
	--({\sx*(3.3400)},{\sy*(0.0000)})
	--({\sx*(3.3500)},{\sy*(0.0000)})
	--({\sx*(3.3600)},{\sy*(0.0000)})
	--({\sx*(3.3700)},{\sy*(0.0000)})
	--({\sx*(3.3800)},{\sy*(0.0000)})
	--({\sx*(3.3900)},{\sy*(0.0000)})
	--({\sx*(3.4000)},{\sy*(0.0000)})
	--({\sx*(3.4100)},{\sy*(-0.0000)})
	--({\sx*(3.4200)},{\sy*(-0.0000)})
	--({\sx*(3.4300)},{\sy*(-0.0000)})
	--({\sx*(3.4400)},{\sy*(-0.0000)})
	--({\sx*(3.4500)},{\sy*(-0.0000)})
	--({\sx*(3.4600)},{\sy*(-0.0000)})
	--({\sx*(3.4700)},{\sy*(-0.0000)})
	--({\sx*(3.4800)},{\sy*(-0.0000)})
	--({\sx*(3.4900)},{\sy*(-0.0000)})
	--({\sx*(3.5000)},{\sy*(-0.0000)})
	--({\sx*(3.5100)},{\sy*(-0.0000)})
	--({\sx*(3.5200)},{\sy*(-0.0000)})
	--({\sx*(3.5300)},{\sy*(-0.0000)})
	--({\sx*(3.5400)},{\sy*(-0.0000)})
	--({\sx*(3.5500)},{\sy*(-0.0000)})
	--({\sx*(3.5600)},{\sy*(-0.0000)})
	--({\sx*(3.5700)},{\sy*(-0.0000)})
	--({\sx*(3.5800)},{\sy*(-0.0000)})
	--({\sx*(3.5900)},{\sy*(-0.0000)})
	--({\sx*(3.6000)},{\sy*(-0.0000)})
	--({\sx*(3.6100)},{\sy*(-0.0000)})
	--({\sx*(3.6200)},{\sy*(-0.0000)})
	--({\sx*(3.6300)},{\sy*(-0.0000)})
	--({\sx*(3.6400)},{\sy*(0.0000)})
	--({\sx*(3.6500)},{\sy*(0.0000)})
	--({\sx*(3.6600)},{\sy*(0.0000)})
	--({\sx*(3.6700)},{\sy*(0.0000)})
	--({\sx*(3.6800)},{\sy*(0.0000)})
	--({\sx*(3.6900)},{\sy*(0.0000)})
	--({\sx*(3.7000)},{\sy*(0.0000)})
	--({\sx*(3.7100)},{\sy*(0.0000)})
	--({\sx*(3.7200)},{\sy*(0.0000)})
	--({\sx*(3.7300)},{\sy*(0.0000)})
	--({\sx*(3.7400)},{\sy*(0.0000)})
	--({\sx*(3.7500)},{\sy*(0.0000)})
	--({\sx*(3.7600)},{\sy*(0.0000)})
	--({\sx*(3.7700)},{\sy*(0.0000)})
	--({\sx*(3.7800)},{\sy*(0.0000)})
	--({\sx*(3.7900)},{\sy*(0.0000)})
	--({\sx*(3.8000)},{\sy*(0.0000)})
	--({\sx*(3.8100)},{\sy*(0.0000)})
	--({\sx*(3.8200)},{\sy*(0.0000)})
	--({\sx*(3.8300)},{\sy*(0.0000)})
	--({\sx*(3.8400)},{\sy*(0.0000)})
	--({\sx*(3.8500)},{\sy*(0.0000)})
	--({\sx*(3.8600)},{\sy*(0.0000)})
	--({\sx*(3.8700)},{\sy*(-0.0000)})
	--({\sx*(3.8800)},{\sy*(-0.0000)})
	--({\sx*(3.8900)},{\sy*(-0.0000)})
	--({\sx*(3.9000)},{\sy*(-0.0000)})
	--({\sx*(3.9100)},{\sy*(-0.0000)})
	--({\sx*(3.9200)},{\sy*(-0.0000)})
	--({\sx*(3.9300)},{\sy*(-0.0000)})
	--({\sx*(3.9400)},{\sy*(-0.0000)})
	--({\sx*(3.9500)},{\sy*(-0.0000)})
	--({\sx*(3.9600)},{\sy*(-0.0000)})
	--({\sx*(3.9700)},{\sy*(-0.0000)})
	--({\sx*(3.9800)},{\sy*(-0.0000)})
	--({\sx*(3.9900)},{\sy*(-0.0000)})
	--({\sx*(4.0000)},{\sy*(-0.0000)})
	--({\sx*(4.0100)},{\sy*(-0.0000)})
	--({\sx*(4.0200)},{\sy*(-0.0000)})
	--({\sx*(4.0300)},{\sy*(-0.0000)})
	--({\sx*(4.0400)},{\sy*(-0.0000)})
	--({\sx*(4.0500)},{\sy*(-0.0000)})
	--({\sx*(4.0600)},{\sy*(-0.0000)})
	--({\sx*(4.0700)},{\sy*(-0.0000)})
	--({\sx*(4.0800)},{\sy*(-0.0000)})
	--({\sx*(4.0900)},{\sy*(-0.0000)})
	--({\sx*(4.1000)},{\sy*(0.0000)})
	--({\sx*(4.1100)},{\sy*(0.0000)})
	--({\sx*(4.1200)},{\sy*(0.0000)})
	--({\sx*(4.1300)},{\sy*(0.0000)})
	--({\sx*(4.1400)},{\sy*(0.0000)})
	--({\sx*(4.1500)},{\sy*(0.0000)})
	--({\sx*(4.1600)},{\sy*(0.0000)})
	--({\sx*(4.1700)},{\sy*(0.0000)})
	--({\sx*(4.1800)},{\sy*(0.0000)})
	--({\sx*(4.1900)},{\sy*(0.0000)})
	--({\sx*(4.2000)},{\sy*(0.0000)})
	--({\sx*(4.2100)},{\sy*(0.0000)})
	--({\sx*(4.2200)},{\sy*(0.0000)})
	--({\sx*(4.2300)},{\sy*(0.0000)})
	--({\sx*(4.2400)},{\sy*(0.0000)})
	--({\sx*(4.2500)},{\sy*(0.0000)})
	--({\sx*(4.2600)},{\sy*(0.0000)})
	--({\sx*(4.2700)},{\sy*(0.0000)})
	--({\sx*(4.2800)},{\sy*(0.0000)})
	--({\sx*(4.2900)},{\sy*(0.0000)})
	--({\sx*(4.3000)},{\sy*(0.0000)})
	--({\sx*(4.3100)},{\sy*(0.0000)})
	--({\sx*(4.3200)},{\sy*(-0.0000)})
	--({\sx*(4.3300)},{\sy*(-0.0000)})
	--({\sx*(4.3400)},{\sy*(-0.0000)})
	--({\sx*(4.3500)},{\sy*(-0.0000)})
	--({\sx*(4.3600)},{\sy*(-0.0000)})
	--({\sx*(4.3700)},{\sy*(-0.0000)})
	--({\sx*(4.3800)},{\sy*(-0.0000)})
	--({\sx*(4.3900)},{\sy*(-0.0000)})
	--({\sx*(4.4000)},{\sy*(-0.0000)})
	--({\sx*(4.4100)},{\sy*(-0.0000)})
	--({\sx*(4.4200)},{\sy*(-0.0000)})
	--({\sx*(4.4300)},{\sy*(-0.0000)})
	--({\sx*(4.4400)},{\sy*(-0.0000)})
	--({\sx*(4.4500)},{\sy*(-0.0000)})
	--({\sx*(4.4600)},{\sy*(-0.0000)})
	--({\sx*(4.4700)},{\sy*(-0.0000)})
	--({\sx*(4.4800)},{\sy*(-0.0000)})
	--({\sx*(4.4900)},{\sy*(-0.0000)})
	--({\sx*(4.5000)},{\sy*(-0.0000)})
	--({\sx*(4.5100)},{\sy*(-0.0000)})
	--({\sx*(4.5200)},{\sy*(-0.0000)})
	--({\sx*(4.5300)},{\sy*(-0.0000)})
	--({\sx*(4.5400)},{\sy*(-0.0000)})
	--({\sx*(4.5500)},{\sy*(0.0000)})
	--({\sx*(4.5600)},{\sy*(0.0000)})
	--({\sx*(4.5700)},{\sy*(0.0000)})
	--({\sx*(4.5800)},{\sy*(0.0000)})
	--({\sx*(4.5900)},{\sy*(0.0000)})
	--({\sx*(4.6000)},{\sy*(0.0000)})
	--({\sx*(4.6100)},{\sy*(0.0000)})
	--({\sx*(4.6200)},{\sy*(0.0000)})
	--({\sx*(4.6300)},{\sy*(0.0000)})
	--({\sx*(4.6400)},{\sy*(0.0000)})
	--({\sx*(4.6500)},{\sy*(0.0000)})
	--({\sx*(4.6600)},{\sy*(0.0000)})
	--({\sx*(4.6700)},{\sy*(0.0001)})
	--({\sx*(4.6800)},{\sy*(0.0001)})
	--({\sx*(4.6900)},{\sy*(0.0001)})
	--({\sx*(4.7000)},{\sy*(0.0001)})
	--({\sx*(4.7100)},{\sy*(0.0001)})
	--({\sx*(4.7200)},{\sy*(0.0001)})
	--({\sx*(4.7300)},{\sy*(0.0001)})
	--({\sx*(4.7400)},{\sy*(0.0001)})
	--({\sx*(4.7500)},{\sy*(0.0001)})
	--({\sx*(4.7600)},{\sy*(0.0000)})
	--({\sx*(4.7700)},{\sy*(0.0000)})
	--({\sx*(4.7800)},{\sy*(-0.0000)})
	--({\sx*(4.7900)},{\sy*(-0.0001)})
	--({\sx*(4.8000)},{\sy*(-0.0002)})
	--({\sx*(4.8100)},{\sy*(-0.0002)})
	--({\sx*(4.8200)},{\sy*(-0.0003)})
	--({\sx*(4.8300)},{\sy*(-0.0005)})
	--({\sx*(4.8400)},{\sy*(-0.0006)})
	--({\sx*(4.8500)},{\sy*(-0.0008)})
	--({\sx*(4.8600)},{\sy*(-0.0011)})
	--({\sx*(4.8700)},{\sy*(-0.0013)})
	--({\sx*(4.8800)},{\sy*(-0.0016)})
	--({\sx*(4.8900)},{\sy*(-0.0019)})
	--({\sx*(4.9000)},{\sy*(-0.0023)})
	--({\sx*(4.9100)},{\sy*(-0.0027)})
	--({\sx*(4.9200)},{\sy*(-0.0030)})
	--({\sx*(4.9300)},{\sy*(-0.0034)})
	--({\sx*(4.9400)},{\sy*(-0.0037)})
	--({\sx*(4.9500)},{\sy*(-0.0038)})
	--({\sx*(4.9600)},{\sy*(-0.0038)})
	--({\sx*(4.9700)},{\sy*(-0.0036)})
	--({\sx*(4.9800)},{\sy*(-0.0030)})
	--({\sx*(4.9900)},{\sy*(-0.0018)})
	--({\sx*(5.0000)},{\sy*(0.0000)});
}
\def\xwertel{
\fill[color=red] (0.0000,0) circle[radius={0.07/\skala}];
\fill[color=red] (0.2083,0) circle[radius={0.07/\skala}];
\fill[color=red] (0.4167,0) circle[radius={0.07/\skala}];
\fill[color=red] (0.6250,0) circle[radius={0.07/\skala}];
\fill[color=red] (0.8333,0) circle[radius={0.07/\skala}];
\fill[color=red] (1.0417,0) circle[radius={0.07/\skala}];
\fill[color=red] (1.2500,0) circle[radius={0.07/\skala}];
\fill[color=red] (1.4583,0) circle[radius={0.07/\skala}];
\fill[color=red] (1.6667,0) circle[radius={0.07/\skala}];
\fill[color=red] (1.8750,0) circle[radius={0.07/\skala}];
\fill[color=red] (2.0833,0) circle[radius={0.07/\skala}];
\fill[color=red] (2.2917,0) circle[radius={0.07/\skala}];
\fill[color=red] (2.5000,0) circle[radius={0.07/\skala}];
\fill[color=red] (2.7083,0) circle[radius={0.07/\skala}];
\fill[color=red] (2.9167,0) circle[radius={0.07/\skala}];
\fill[color=red] (3.1250,0) circle[radius={0.07/\skala}];
\fill[color=red] (3.3333,0) circle[radius={0.07/\skala}];
\fill[color=red] (3.5417,0) circle[radius={0.07/\skala}];
\fill[color=red] (3.7500,0) circle[radius={0.07/\skala}];
\fill[color=red] (3.9583,0) circle[radius={0.07/\skala}];
\fill[color=red] (4.1667,0) circle[radius={0.07/\skala}];
\fill[color=red] (4.3750,0) circle[radius={0.07/\skala}];
\fill[color=red] (4.5833,0) circle[radius={0.07/\skala}];
\fill[color=red] (4.7917,0) circle[radius={0.07/\skala}];
\fill[color=red] (5.0000,0) circle[radius={0.07/\skala}];
}
\def\punktel{24}
\def\maxfehlerl{1.633\cdot 10^{-9}}
\def\fehlerl{
\draw[color=red,line width=1.4pt,line join=round] ({\sx*(0.000)},{\sy*(0.0000)})
	--({\sx*(0.0100)},{\sy*(-0.4359)})
	--({\sx*(0.0200)},{\sy*(-0.7224)})
	--({\sx*(0.0300)},{\sy*(-0.8938)})
	--({\sx*(0.0400)},{\sy*(-0.9789)})
	--({\sx*(0.0500)},{\sy*(-1.0000)})
	--({\sx*(0.0600)},{\sy*(-0.9748)})
	--({\sx*(0.0700)},{\sy*(-0.9184)})
	--({\sx*(0.0800)},{\sy*(-0.8414)})
	--({\sx*(0.0900)},{\sy*(-0.7527)})
	--({\sx*(0.1000)},{\sy*(-0.6592)})
	--({\sx*(0.1100)},{\sy*(-0.5662)})
	--({\sx*(0.1200)},{\sy*(-0.4747)})
	--({\sx*(0.1300)},{\sy*(-0.3908)})
	--({\sx*(0.1400)},{\sy*(-0.3130)})
	--({\sx*(0.1500)},{\sy*(-0.2438)})
	--({\sx*(0.1600)},{\sy*(-0.1829)})
	--({\sx*(0.1700)},{\sy*(-0.1306)})
	--({\sx*(0.1800)},{\sy*(-0.0864)})
	--({\sx*(0.1900)},{\sy*(-0.0496)})
	--({\sx*(0.2000)},{\sy*(-0.0199)})
	--({\sx*(0.2100)},{\sy*(0.0035)})
	--({\sx*(0.2200)},{\sy*(0.0214)})
	--({\sx*(0.2300)},{\sy*(0.0344)})
	--({\sx*(0.2400)},{\sy*(0.0434)})
	--({\sx*(0.2500)},{\sy*(0.0489)})
	--({\sx*(0.2600)},{\sy*(0.0515)})
	--({\sx*(0.2700)},{\sy*(0.0519)})
	--({\sx*(0.2800)},{\sy*(0.0505)})
	--({\sx*(0.2900)},{\sy*(0.0479)})
	--({\sx*(0.3000)},{\sy*(0.0442)})
	--({\sx*(0.3100)},{\sy*(0.0399)})
	--({\sx*(0.3200)},{\sy*(0.0353)})
	--({\sx*(0.3300)},{\sy*(0.0305)})
	--({\sx*(0.3400)},{\sy*(0.0258)})
	--({\sx*(0.3500)},{\sy*(0.0212)})
	--({\sx*(0.3600)},{\sy*(0.0170)})
	--({\sx*(0.3700)},{\sy*(0.0130)})
	--({\sx*(0.3800)},{\sy*(0.0095)})
	--({\sx*(0.3900)},{\sy*(0.0063)})
	--({\sx*(0.4000)},{\sy*(0.0036)})
	--({\sx*(0.4100)},{\sy*(0.0013)})
	--({\sx*(0.4200)},{\sy*(-0.0006)})
	--({\sx*(0.4300)},{\sy*(-0.0021)})
	--({\sx*(0.4400)},{\sy*(-0.0033)})
	--({\sx*(0.4500)},{\sy*(-0.0041)})
	--({\sx*(0.4600)},{\sy*(-0.0047)})
	--({\sx*(0.4700)},{\sy*(-0.0050)})
	--({\sx*(0.4800)},{\sy*(-0.0051)})
	--({\sx*(0.4900)},{\sy*(-0.0051)})
	--({\sx*(0.5000)},{\sy*(-0.0049)})
	--({\sx*(0.5100)},{\sy*(-0.0046)})
	--({\sx*(0.5200)},{\sy*(-0.0042)})
	--({\sx*(0.5300)},{\sy*(-0.0038)})
	--({\sx*(0.5400)},{\sy*(-0.0034)})
	--({\sx*(0.5500)},{\sy*(-0.0029)})
	--({\sx*(0.5600)},{\sy*(-0.0024)})
	--({\sx*(0.5700)},{\sy*(-0.0020)})
	--({\sx*(0.5800)},{\sy*(-0.0015)})
	--({\sx*(0.5900)},{\sy*(-0.0011)})
	--({\sx*(0.6000)},{\sy*(-0.0007)})
	--({\sx*(0.6100)},{\sy*(-0.0004)})
	--({\sx*(0.6200)},{\sy*(-0.0001)})
	--({\sx*(0.6300)},{\sy*(0.0001)})
	--({\sx*(0.6400)},{\sy*(0.0003)})
	--({\sx*(0.6500)},{\sy*(0.0005)})
	--({\sx*(0.6600)},{\sy*(0.0006)})
	--({\sx*(0.6700)},{\sy*(0.0007)})
	--({\sx*(0.6800)},{\sy*(0.0007)})
	--({\sx*(0.6900)},{\sy*(0.0008)})
	--({\sx*(0.7000)},{\sy*(0.0008)})
	--({\sx*(0.7100)},{\sy*(0.0007)})
	--({\sx*(0.7200)},{\sy*(0.0007)})
	--({\sx*(0.7300)},{\sy*(0.0007)})
	--({\sx*(0.7400)},{\sy*(0.0006)})
	--({\sx*(0.7500)},{\sy*(0.0005)})
	--({\sx*(0.7600)},{\sy*(0.0005)})
	--({\sx*(0.7700)},{\sy*(0.0004)})
	--({\sx*(0.7800)},{\sy*(0.0003)})
	--({\sx*(0.7900)},{\sy*(0.0003)})
	--({\sx*(0.8000)},{\sy*(0.0002)})
	--({\sx*(0.8100)},{\sy*(0.0001)})
	--({\sx*(0.8200)},{\sy*(0.0001)})
	--({\sx*(0.8300)},{\sy*(0.0000)})
	--({\sx*(0.8400)},{\sy*(-0.0000)})
	--({\sx*(0.8500)},{\sy*(-0.0001)})
	--({\sx*(0.8600)},{\sy*(-0.0001)})
	--({\sx*(0.8700)},{\sy*(-0.0001)})
	--({\sx*(0.8800)},{\sy*(-0.0001)})
	--({\sx*(0.8900)},{\sy*(-0.0001)})
	--({\sx*(0.9000)},{\sy*(-0.0002)})
	--({\sx*(0.9100)},{\sy*(-0.0002)})
	--({\sx*(0.9200)},{\sy*(-0.0002)})
	--({\sx*(0.9300)},{\sy*(-0.0001)})
	--({\sx*(0.9400)},{\sy*(-0.0001)})
	--({\sx*(0.9500)},{\sy*(-0.0001)})
	--({\sx*(0.9600)},{\sy*(-0.0001)})
	--({\sx*(0.9700)},{\sy*(-0.0001)})
	--({\sx*(0.9800)},{\sy*(-0.0001)})
	--({\sx*(0.9900)},{\sy*(-0.0001)})
	--({\sx*(1.0000)},{\sy*(-0.0001)})
	--({\sx*(1.0100)},{\sy*(-0.0000)})
	--({\sx*(1.0200)},{\sy*(-0.0000)})
	--({\sx*(1.0300)},{\sy*(-0.0000)})
	--({\sx*(1.0400)},{\sy*(-0.0000)})
	--({\sx*(1.0500)},{\sy*(0.0000)})
	--({\sx*(1.0600)},{\sy*(0.0000)})
	--({\sx*(1.0700)},{\sy*(0.0000)})
	--({\sx*(1.0800)},{\sy*(0.0000)})
	--({\sx*(1.0900)},{\sy*(0.0000)})
	--({\sx*(1.1000)},{\sy*(0.0000)})
	--({\sx*(1.1100)},{\sy*(0.0000)})
	--({\sx*(1.1200)},{\sy*(0.0000)})
	--({\sx*(1.1300)},{\sy*(0.0000)})
	--({\sx*(1.1400)},{\sy*(0.0000)})
	--({\sx*(1.1500)},{\sy*(0.0000)})
	--({\sx*(1.1600)},{\sy*(0.0000)})
	--({\sx*(1.1700)},{\sy*(0.0000)})
	--({\sx*(1.1800)},{\sy*(0.0000)})
	--({\sx*(1.1900)},{\sy*(0.0000)})
	--({\sx*(1.2000)},{\sy*(0.0000)})
	--({\sx*(1.2100)},{\sy*(0.0000)})
	--({\sx*(1.2200)},{\sy*(0.0000)})
	--({\sx*(1.2300)},{\sy*(0.0000)})
	--({\sx*(1.2400)},{\sy*(0.0000)})
	--({\sx*(1.2500)},{\sy*(0.0000)})
	--({\sx*(1.2600)},{\sy*(-0.0000)})
	--({\sx*(1.2700)},{\sy*(-0.0000)})
	--({\sx*(1.2800)},{\sy*(-0.0000)})
	--({\sx*(1.2900)},{\sy*(-0.0000)})
	--({\sx*(1.3000)},{\sy*(-0.0000)})
	--({\sx*(1.3100)},{\sy*(-0.0000)})
	--({\sx*(1.3200)},{\sy*(-0.0000)})
	--({\sx*(1.3300)},{\sy*(-0.0000)})
	--({\sx*(1.3400)},{\sy*(-0.0000)})
	--({\sx*(1.3500)},{\sy*(-0.0000)})
	--({\sx*(1.3600)},{\sy*(-0.0000)})
	--({\sx*(1.3700)},{\sy*(-0.0000)})
	--({\sx*(1.3800)},{\sy*(-0.0000)})
	--({\sx*(1.3900)},{\sy*(-0.0000)})
	--({\sx*(1.4000)},{\sy*(-0.0000)})
	--({\sx*(1.4100)},{\sy*(-0.0000)})
	--({\sx*(1.4200)},{\sy*(-0.0000)})
	--({\sx*(1.4300)},{\sy*(-0.0000)})
	--({\sx*(1.4400)},{\sy*(-0.0000)})
	--({\sx*(1.4500)},{\sy*(-0.0000)})
	--({\sx*(1.4600)},{\sy*(0.0000)})
	--({\sx*(1.4700)},{\sy*(0.0000)})
	--({\sx*(1.4800)},{\sy*(0.0000)})
	--({\sx*(1.4900)},{\sy*(0.0000)})
	--({\sx*(1.5000)},{\sy*(0.0000)})
	--({\sx*(1.5100)},{\sy*(0.0000)})
	--({\sx*(1.5200)},{\sy*(0.0000)})
	--({\sx*(1.5300)},{\sy*(0.0000)})
	--({\sx*(1.5400)},{\sy*(0.0000)})
	--({\sx*(1.5500)},{\sy*(0.0000)})
	--({\sx*(1.5600)},{\sy*(0.0000)})
	--({\sx*(1.5700)},{\sy*(0.0000)})
	--({\sx*(1.5800)},{\sy*(0.0000)})
	--({\sx*(1.5900)},{\sy*(0.0000)})
	--({\sx*(1.6000)},{\sy*(0.0000)})
	--({\sx*(1.6100)},{\sy*(0.0000)})
	--({\sx*(1.6200)},{\sy*(0.0000)})
	--({\sx*(1.6300)},{\sy*(0.0000)})
	--({\sx*(1.6400)},{\sy*(0.0000)})
	--({\sx*(1.6500)},{\sy*(0.0000)})
	--({\sx*(1.6600)},{\sy*(0.0000)})
	--({\sx*(1.6700)},{\sy*(-0.0000)})
	--({\sx*(1.6800)},{\sy*(-0.0000)})
	--({\sx*(1.6900)},{\sy*(-0.0000)})
	--({\sx*(1.7000)},{\sy*(-0.0000)})
	--({\sx*(1.7100)},{\sy*(-0.0000)})
	--({\sx*(1.7200)},{\sy*(-0.0000)})
	--({\sx*(1.7300)},{\sy*(-0.0000)})
	--({\sx*(1.7400)},{\sy*(-0.0000)})
	--({\sx*(1.7500)},{\sy*(-0.0000)})
	--({\sx*(1.7600)},{\sy*(-0.0000)})
	--({\sx*(1.7700)},{\sy*(-0.0000)})
	--({\sx*(1.7800)},{\sy*(-0.0000)})
	--({\sx*(1.7900)},{\sy*(-0.0000)})
	--({\sx*(1.8000)},{\sy*(-0.0000)})
	--({\sx*(1.8100)},{\sy*(-0.0000)})
	--({\sx*(1.8200)},{\sy*(-0.0000)})
	--({\sx*(1.8300)},{\sy*(-0.0000)})
	--({\sx*(1.8400)},{\sy*(-0.0000)})
	--({\sx*(1.8500)},{\sy*(-0.0000)})
	--({\sx*(1.8600)},{\sy*(-0.0000)})
	--({\sx*(1.8700)},{\sy*(-0.0000)})
	--({\sx*(1.8800)},{\sy*(0.0000)})
	--({\sx*(1.8900)},{\sy*(0.0000)})
	--({\sx*(1.9000)},{\sy*(0.0000)})
	--({\sx*(1.9100)},{\sy*(0.0000)})
	--({\sx*(1.9200)},{\sy*(0.0000)})
	--({\sx*(1.9300)},{\sy*(0.0000)})
	--({\sx*(1.9400)},{\sy*(0.0000)})
	--({\sx*(1.9500)},{\sy*(0.0000)})
	--({\sx*(1.9600)},{\sy*(0.0000)})
	--({\sx*(1.9700)},{\sy*(0.0000)})
	--({\sx*(1.9800)},{\sy*(0.0000)})
	--({\sx*(1.9900)},{\sy*(0.0000)})
	--({\sx*(2.0000)},{\sy*(0.0000)})
	--({\sx*(2.0100)},{\sy*(0.0000)})
	--({\sx*(2.0200)},{\sy*(0.0000)})
	--({\sx*(2.0300)},{\sy*(0.0000)})
	--({\sx*(2.0400)},{\sy*(0.0000)})
	--({\sx*(2.0500)},{\sy*(0.0000)})
	--({\sx*(2.0600)},{\sy*(0.0000)})
	--({\sx*(2.0700)},{\sy*(0.0000)})
	--({\sx*(2.0800)},{\sy*(0.0000)})
	--({\sx*(2.0900)},{\sy*(-0.0000)})
	--({\sx*(2.1000)},{\sy*(-0.0000)})
	--({\sx*(2.1100)},{\sy*(-0.0000)})
	--({\sx*(2.1200)},{\sy*(-0.0000)})
	--({\sx*(2.1300)},{\sy*(-0.0000)})
	--({\sx*(2.1400)},{\sy*(-0.0000)})
	--({\sx*(2.1500)},{\sy*(-0.0000)})
	--({\sx*(2.1600)},{\sy*(-0.0000)})
	--({\sx*(2.1700)},{\sy*(-0.0000)})
	--({\sx*(2.1800)},{\sy*(-0.0000)})
	--({\sx*(2.1900)},{\sy*(-0.0000)})
	--({\sx*(2.2000)},{\sy*(-0.0000)})
	--({\sx*(2.2100)},{\sy*(-0.0000)})
	--({\sx*(2.2200)},{\sy*(-0.0000)})
	--({\sx*(2.2300)},{\sy*(-0.0000)})
	--({\sx*(2.2400)},{\sy*(-0.0000)})
	--({\sx*(2.2500)},{\sy*(-0.0000)})
	--({\sx*(2.2600)},{\sy*(-0.0000)})
	--({\sx*(2.2700)},{\sy*(-0.0000)})
	--({\sx*(2.2800)},{\sy*(-0.0000)})
	--({\sx*(2.2900)},{\sy*(-0.0000)})
	--({\sx*(2.3000)},{\sy*(0.0000)})
	--({\sx*(2.3100)},{\sy*(0.0000)})
	--({\sx*(2.3200)},{\sy*(0.0000)})
	--({\sx*(2.3300)},{\sy*(0.0000)})
	--({\sx*(2.3400)},{\sy*(0.0000)})
	--({\sx*(2.3500)},{\sy*(0.0000)})
	--({\sx*(2.3600)},{\sy*(0.0000)})
	--({\sx*(2.3700)},{\sy*(0.0000)})
	--({\sx*(2.3800)},{\sy*(0.0000)})
	--({\sx*(2.3900)},{\sy*(0.0000)})
	--({\sx*(2.4000)},{\sy*(0.0000)})
	--({\sx*(2.4100)},{\sy*(0.0000)})
	--({\sx*(2.4200)},{\sy*(0.0000)})
	--({\sx*(2.4300)},{\sy*(0.0000)})
	--({\sx*(2.4400)},{\sy*(0.0000)})
	--({\sx*(2.4500)},{\sy*(0.0000)})
	--({\sx*(2.4600)},{\sy*(0.0000)})
	--({\sx*(2.4700)},{\sy*(0.0000)})
	--({\sx*(2.4800)},{\sy*(0.0000)})
	--({\sx*(2.4900)},{\sy*(0.0000)})
	--({\sx*(2.5000)},{\sy*(0.0000)})
	--({\sx*(2.5100)},{\sy*(-0.0000)})
	--({\sx*(2.5200)},{\sy*(-0.0000)})
	--({\sx*(2.5300)},{\sy*(-0.0000)})
	--({\sx*(2.5400)},{\sy*(-0.0000)})
	--({\sx*(2.5500)},{\sy*(-0.0000)})
	--({\sx*(2.5600)},{\sy*(-0.0000)})
	--({\sx*(2.5700)},{\sy*(-0.0000)})
	--({\sx*(2.5800)},{\sy*(-0.0000)})
	--({\sx*(2.5900)},{\sy*(-0.0000)})
	--({\sx*(2.6000)},{\sy*(-0.0000)})
	--({\sx*(2.6100)},{\sy*(-0.0000)})
	--({\sx*(2.6200)},{\sy*(-0.0000)})
	--({\sx*(2.6300)},{\sy*(-0.0000)})
	--({\sx*(2.6400)},{\sy*(-0.0000)})
	--({\sx*(2.6500)},{\sy*(-0.0000)})
	--({\sx*(2.6600)},{\sy*(-0.0000)})
	--({\sx*(2.6700)},{\sy*(-0.0000)})
	--({\sx*(2.6800)},{\sy*(-0.0000)})
	--({\sx*(2.6900)},{\sy*(-0.0000)})
	--({\sx*(2.7000)},{\sy*(-0.0000)})
	--({\sx*(2.7100)},{\sy*(0.0000)})
	--({\sx*(2.7200)},{\sy*(0.0000)})
	--({\sx*(2.7300)},{\sy*(0.0000)})
	--({\sx*(2.7400)},{\sy*(0.0000)})
	--({\sx*(2.7500)},{\sy*(0.0000)})
	--({\sx*(2.7600)},{\sy*(0.0000)})
	--({\sx*(2.7700)},{\sy*(0.0000)})
	--({\sx*(2.7800)},{\sy*(0.0000)})
	--({\sx*(2.7900)},{\sy*(0.0000)})
	--({\sx*(2.8000)},{\sy*(0.0000)})
	--({\sx*(2.8100)},{\sy*(0.0000)})
	--({\sx*(2.8200)},{\sy*(0.0000)})
	--({\sx*(2.8300)},{\sy*(0.0000)})
	--({\sx*(2.8400)},{\sy*(0.0000)})
	--({\sx*(2.8500)},{\sy*(0.0000)})
	--({\sx*(2.8600)},{\sy*(0.0000)})
	--({\sx*(2.8700)},{\sy*(0.0000)})
	--({\sx*(2.8800)},{\sy*(0.0000)})
	--({\sx*(2.8900)},{\sy*(0.0000)})
	--({\sx*(2.9000)},{\sy*(0.0000)})
	--({\sx*(2.9100)},{\sy*(0.0000)})
	--({\sx*(2.9200)},{\sy*(-0.0000)})
	--({\sx*(2.9300)},{\sy*(-0.0000)})
	--({\sx*(2.9400)},{\sy*(-0.0000)})
	--({\sx*(2.9500)},{\sy*(-0.0000)})
	--({\sx*(2.9600)},{\sy*(-0.0000)})
	--({\sx*(2.9700)},{\sy*(-0.0000)})
	--({\sx*(2.9800)},{\sy*(-0.0000)})
	--({\sx*(2.9900)},{\sy*(-0.0000)})
	--({\sx*(3.0000)},{\sy*(-0.0000)})
	--({\sx*(3.0100)},{\sy*(-0.0000)})
	--({\sx*(3.0200)},{\sy*(-0.0000)})
	--({\sx*(3.0300)},{\sy*(-0.0000)})
	--({\sx*(3.0400)},{\sy*(-0.0000)})
	--({\sx*(3.0500)},{\sy*(-0.0000)})
	--({\sx*(3.0600)},{\sy*(-0.0000)})
	--({\sx*(3.0700)},{\sy*(-0.0000)})
	--({\sx*(3.0800)},{\sy*(-0.0000)})
	--({\sx*(3.0900)},{\sy*(-0.0000)})
	--({\sx*(3.1000)},{\sy*(-0.0000)})
	--({\sx*(3.1100)},{\sy*(-0.0000)})
	--({\sx*(3.1200)},{\sy*(-0.0000)})
	--({\sx*(3.1300)},{\sy*(0.0000)})
	--({\sx*(3.1400)},{\sy*(0.0000)})
	--({\sx*(3.1500)},{\sy*(0.0000)})
	--({\sx*(3.1600)},{\sy*(0.0000)})
	--({\sx*(3.1700)},{\sy*(0.0000)})
	--({\sx*(3.1800)},{\sy*(0.0000)})
	--({\sx*(3.1900)},{\sy*(0.0000)})
	--({\sx*(3.2000)},{\sy*(0.0000)})
	--({\sx*(3.2100)},{\sy*(0.0000)})
	--({\sx*(3.2200)},{\sy*(0.0000)})
	--({\sx*(3.2300)},{\sy*(0.0000)})
	--({\sx*(3.2400)},{\sy*(0.0000)})
	--({\sx*(3.2500)},{\sy*(0.0000)})
	--({\sx*(3.2600)},{\sy*(0.0000)})
	--({\sx*(3.2700)},{\sy*(0.0000)})
	--({\sx*(3.2800)},{\sy*(0.0000)})
	--({\sx*(3.2900)},{\sy*(0.0000)})
	--({\sx*(3.3000)},{\sy*(0.0000)})
	--({\sx*(3.3100)},{\sy*(0.0000)})
	--({\sx*(3.3200)},{\sy*(0.0000)})
	--({\sx*(3.3300)},{\sy*(0.0000)})
	--({\sx*(3.3400)},{\sy*(-0.0000)})
	--({\sx*(3.3500)},{\sy*(-0.0000)})
	--({\sx*(3.3600)},{\sy*(-0.0000)})
	--({\sx*(3.3700)},{\sy*(-0.0000)})
	--({\sx*(3.3800)},{\sy*(-0.0000)})
	--({\sx*(3.3900)},{\sy*(-0.0000)})
	--({\sx*(3.4000)},{\sy*(-0.0000)})
	--({\sx*(3.4100)},{\sy*(-0.0000)})
	--({\sx*(3.4200)},{\sy*(-0.0000)})
	--({\sx*(3.4300)},{\sy*(-0.0000)})
	--({\sx*(3.4400)},{\sy*(-0.0000)})
	--({\sx*(3.4500)},{\sy*(-0.0000)})
	--({\sx*(3.4600)},{\sy*(-0.0000)})
	--({\sx*(3.4700)},{\sy*(-0.0000)})
	--({\sx*(3.4800)},{\sy*(-0.0000)})
	--({\sx*(3.4900)},{\sy*(-0.0000)})
	--({\sx*(3.5000)},{\sy*(-0.0000)})
	--({\sx*(3.5100)},{\sy*(-0.0000)})
	--({\sx*(3.5200)},{\sy*(-0.0000)})
	--({\sx*(3.5300)},{\sy*(-0.0000)})
	--({\sx*(3.5400)},{\sy*(-0.0000)})
	--({\sx*(3.5500)},{\sy*(0.0000)})
	--({\sx*(3.5600)},{\sy*(0.0000)})
	--({\sx*(3.5700)},{\sy*(0.0000)})
	--({\sx*(3.5800)},{\sy*(0.0000)})
	--({\sx*(3.5900)},{\sy*(0.0000)})
	--({\sx*(3.6000)},{\sy*(0.0000)})
	--({\sx*(3.6100)},{\sy*(0.0000)})
	--({\sx*(3.6200)},{\sy*(0.0000)})
	--({\sx*(3.6300)},{\sy*(0.0000)})
	--({\sx*(3.6400)},{\sy*(0.0000)})
	--({\sx*(3.6500)},{\sy*(0.0000)})
	--({\sx*(3.6600)},{\sy*(0.0000)})
	--({\sx*(3.6700)},{\sy*(0.0000)})
	--({\sx*(3.6800)},{\sy*(0.0000)})
	--({\sx*(3.6900)},{\sy*(0.0000)})
	--({\sx*(3.7000)},{\sy*(0.0000)})
	--({\sx*(3.7100)},{\sy*(0.0000)})
	--({\sx*(3.7200)},{\sy*(0.0000)})
	--({\sx*(3.7300)},{\sy*(0.0000)})
	--({\sx*(3.7400)},{\sy*(0.0000)})
	--({\sx*(3.7500)},{\sy*(0.0000)})
	--({\sx*(3.7600)},{\sy*(-0.0000)})
	--({\sx*(3.7700)},{\sy*(-0.0000)})
	--({\sx*(3.7800)},{\sy*(-0.0000)})
	--({\sx*(3.7900)},{\sy*(-0.0000)})
	--({\sx*(3.8000)},{\sy*(-0.0000)})
	--({\sx*(3.8100)},{\sy*(-0.0000)})
	--({\sx*(3.8200)},{\sy*(-0.0000)})
	--({\sx*(3.8300)},{\sy*(-0.0000)})
	--({\sx*(3.8400)},{\sy*(-0.0000)})
	--({\sx*(3.8500)},{\sy*(-0.0000)})
	--({\sx*(3.8600)},{\sy*(-0.0000)})
	--({\sx*(3.8700)},{\sy*(-0.0000)})
	--({\sx*(3.8800)},{\sy*(-0.0000)})
	--({\sx*(3.8900)},{\sy*(-0.0000)})
	--({\sx*(3.9000)},{\sy*(-0.0000)})
	--({\sx*(3.9100)},{\sy*(-0.0000)})
	--({\sx*(3.9200)},{\sy*(-0.0000)})
	--({\sx*(3.9300)},{\sy*(-0.0000)})
	--({\sx*(3.9400)},{\sy*(-0.0000)})
	--({\sx*(3.9500)},{\sy*(-0.0000)})
	--({\sx*(3.9600)},{\sy*(0.0000)})
	--({\sx*(3.9700)},{\sy*(0.0000)})
	--({\sx*(3.9800)},{\sy*(0.0000)})
	--({\sx*(3.9900)},{\sy*(0.0000)})
	--({\sx*(4.0000)},{\sy*(0.0000)})
	--({\sx*(4.0100)},{\sy*(0.0000)})
	--({\sx*(4.0200)},{\sy*(0.0000)})
	--({\sx*(4.0300)},{\sy*(0.0000)})
	--({\sx*(4.0400)},{\sy*(0.0000)})
	--({\sx*(4.0500)},{\sy*(0.0000)})
	--({\sx*(4.0600)},{\sy*(0.0000)})
	--({\sx*(4.0700)},{\sy*(0.0000)})
	--({\sx*(4.0800)},{\sy*(0.0000)})
	--({\sx*(4.0900)},{\sy*(0.0000)})
	--({\sx*(4.1000)},{\sy*(0.0000)})
	--({\sx*(4.1100)},{\sy*(0.0000)})
	--({\sx*(4.1200)},{\sy*(0.0000)})
	--({\sx*(4.1300)},{\sy*(0.0000)})
	--({\sx*(4.1400)},{\sy*(0.0000)})
	--({\sx*(4.1500)},{\sy*(0.0000)})
	--({\sx*(4.1600)},{\sy*(0.0000)})
	--({\sx*(4.1700)},{\sy*(-0.0000)})
	--({\sx*(4.1800)},{\sy*(-0.0000)})
	--({\sx*(4.1900)},{\sy*(-0.0000)})
	--({\sx*(4.2000)},{\sy*(-0.0000)})
	--({\sx*(4.2100)},{\sy*(-0.0001)})
	--({\sx*(4.2200)},{\sy*(-0.0001)})
	--({\sx*(4.2300)},{\sy*(-0.0001)})
	--({\sx*(4.2400)},{\sy*(-0.0001)})
	--({\sx*(4.2500)},{\sy*(-0.0001)})
	--({\sx*(4.2600)},{\sy*(-0.0002)})
	--({\sx*(4.2700)},{\sy*(-0.0002)})
	--({\sx*(4.2800)},{\sy*(-0.0002)})
	--({\sx*(4.2900)},{\sy*(-0.0002)})
	--({\sx*(4.3000)},{\sy*(-0.0002)})
	--({\sx*(4.3100)},{\sy*(-0.0002)})
	--({\sx*(4.3200)},{\sy*(-0.0002)})
	--({\sx*(4.3300)},{\sy*(-0.0002)})
	--({\sx*(4.3400)},{\sy*(-0.0001)})
	--({\sx*(4.3500)},{\sy*(-0.0001)})
	--({\sx*(4.3600)},{\sy*(-0.0001)})
	--({\sx*(4.3700)},{\sy*(-0.0000)})
	--({\sx*(4.3800)},{\sy*(0.0000)})
	--({\sx*(4.3900)},{\sy*(0.0001)})
	--({\sx*(4.4000)},{\sy*(0.0002)})
	--({\sx*(4.4100)},{\sy*(0.0002)})
	--({\sx*(4.4200)},{\sy*(0.0003)})
	--({\sx*(4.4300)},{\sy*(0.0004)})
	--({\sx*(4.4400)},{\sy*(0.0005)})
	--({\sx*(4.4500)},{\sy*(0.0006)})
	--({\sx*(4.4600)},{\sy*(0.0007)})
	--({\sx*(4.4700)},{\sy*(0.0008)})
	--({\sx*(4.4800)},{\sy*(0.0009)})
	--({\sx*(4.4900)},{\sy*(0.0010)})
	--({\sx*(4.5000)},{\sy*(0.0010)})
	--({\sx*(4.5100)},{\sy*(0.0010)})
	--({\sx*(4.5200)},{\sy*(0.0010)})
	--({\sx*(4.5300)},{\sy*(0.0010)})
	--({\sx*(4.5400)},{\sy*(0.0009)})
	--({\sx*(4.5500)},{\sy*(0.0008)})
	--({\sx*(4.5600)},{\sy*(0.0006)})
	--({\sx*(4.5700)},{\sy*(0.0004)})
	--({\sx*(4.5800)},{\sy*(0.0001)})
	--({\sx*(4.5900)},{\sy*(-0.0002)})
	--({\sx*(4.6000)},{\sy*(-0.0007)})
	--({\sx*(4.6100)},{\sy*(-0.0012)})
	--({\sx*(4.6200)},{\sy*(-0.0017)})
	--({\sx*(4.6300)},{\sy*(-0.0024)})
	--({\sx*(4.6400)},{\sy*(-0.0031)})
	--({\sx*(4.6500)},{\sy*(-0.0038)})
	--({\sx*(4.6600)},{\sy*(-0.0046)})
	--({\sx*(4.6700)},{\sy*(-0.0054)})
	--({\sx*(4.6800)},{\sy*(-0.0061)})
	--({\sx*(4.6900)},{\sy*(-0.0069)})
	--({\sx*(4.7000)},{\sy*(-0.0075)})
	--({\sx*(4.7100)},{\sy*(-0.0081)})
	--({\sx*(4.7200)},{\sy*(-0.0085)})
	--({\sx*(4.7300)},{\sy*(-0.0086)})
	--({\sx*(4.7400)},{\sy*(-0.0084)})
	--({\sx*(4.7500)},{\sy*(-0.0079)})
	--({\sx*(4.7600)},{\sy*(-0.0070)})
	--({\sx*(4.7700)},{\sy*(-0.0055)})
	--({\sx*(4.7800)},{\sy*(-0.0034)})
	--({\sx*(4.7900)},{\sy*(-0.0005)})
	--({\sx*(4.8000)},{\sy*(0.0031)})
	--({\sx*(4.8100)},{\sy*(0.0075)})
	--({\sx*(4.8200)},{\sy*(0.0131)})
	--({\sx*(4.8300)},{\sy*(0.0195)})
	--({\sx*(4.8400)},{\sy*(0.0272)})
	--({\sx*(4.8500)},{\sy*(0.0357)})
	--({\sx*(4.8600)},{\sy*(0.0455)})
	--({\sx*(4.8700)},{\sy*(0.0560)})
	--({\sx*(4.8800)},{\sy*(0.0675)})
	--({\sx*(4.8900)},{\sy*(0.0795)})
	--({\sx*(4.9000)},{\sy*(0.0920)})
	--({\sx*(4.9100)},{\sy*(0.1035)})
	--({\sx*(4.9200)},{\sy*(0.1151)})
	--({\sx*(4.9300)},{\sy*(0.1235)})
	--({\sx*(4.9400)},{\sy*(0.1300)})
	--({\sx*(4.9500)},{\sy*(0.1317)})
	--({\sx*(4.9600)},{\sy*(0.1278)})
	--({\sx*(4.9700)},{\sy*(0.1148)})
	--({\sx*(4.9800)},{\sy*(0.0920)})
	--({\sx*(4.9900)},{\sy*(0.0548)})
	--({\sx*(5.0000)},{\sy*(0.0000)});
}
\def\relfehlerl{
\draw[color=blue,line width=1.4pt,line join=round] ({\sx*(0.000)},{\sy*(0.0000)})
	--({\sx*(0.0100)},{\sy*(-0.0000)})
	--({\sx*(0.0200)},{\sy*(-0.0000)})
	--({\sx*(0.0300)},{\sy*(-0.0000)})
	--({\sx*(0.0400)},{\sy*(-0.0000)})
	--({\sx*(0.0500)},{\sy*(-0.0000)})
	--({\sx*(0.0600)},{\sy*(-0.0000)})
	--({\sx*(0.0700)},{\sy*(-0.0000)})
	--({\sx*(0.0800)},{\sy*(-0.0000)})
	--({\sx*(0.0900)},{\sy*(-0.0000)})
	--({\sx*(0.1000)},{\sy*(-0.0000)})
	--({\sx*(0.1100)},{\sy*(-0.0000)})
	--({\sx*(0.1200)},{\sy*(-0.0000)})
	--({\sx*(0.1300)},{\sy*(-0.0000)})
	--({\sx*(0.1400)},{\sy*(-0.0000)})
	--({\sx*(0.1500)},{\sy*(-0.0000)})
	--({\sx*(0.1600)},{\sy*(-0.0000)})
	--({\sx*(0.1700)},{\sy*(-0.0000)})
	--({\sx*(0.1800)},{\sy*(-0.0000)})
	--({\sx*(0.1900)},{\sy*(-0.0000)})
	--({\sx*(0.2000)},{\sy*(-0.0000)})
	--({\sx*(0.2100)},{\sy*(0.0000)})
	--({\sx*(0.2200)},{\sy*(0.0000)})
	--({\sx*(0.2300)},{\sy*(0.0000)})
	--({\sx*(0.2400)},{\sy*(0.0000)})
	--({\sx*(0.2500)},{\sy*(0.0000)})
	--({\sx*(0.2600)},{\sy*(0.0000)})
	--({\sx*(0.2700)},{\sy*(0.0000)})
	--({\sx*(0.2800)},{\sy*(0.0000)})
	--({\sx*(0.2900)},{\sy*(0.0000)})
	--({\sx*(0.3000)},{\sy*(0.0000)})
	--({\sx*(0.3100)},{\sy*(0.0000)})
	--({\sx*(0.3200)},{\sy*(0.0000)})
	--({\sx*(0.3300)},{\sy*(0.0000)})
	--({\sx*(0.3400)},{\sy*(0.0000)})
	--({\sx*(0.3500)},{\sy*(0.0000)})
	--({\sx*(0.3600)},{\sy*(0.0000)})
	--({\sx*(0.3700)},{\sy*(0.0000)})
	--({\sx*(0.3800)},{\sy*(0.0000)})
	--({\sx*(0.3900)},{\sy*(0.0000)})
	--({\sx*(0.4000)},{\sy*(0.0000)})
	--({\sx*(0.4100)},{\sy*(0.0000)})
	--({\sx*(0.4200)},{\sy*(-0.0000)})
	--({\sx*(0.4300)},{\sy*(-0.0000)})
	--({\sx*(0.4400)},{\sy*(-0.0000)})
	--({\sx*(0.4500)},{\sy*(-0.0000)})
	--({\sx*(0.4600)},{\sy*(-0.0000)})
	--({\sx*(0.4700)},{\sy*(-0.0000)})
	--({\sx*(0.4800)},{\sy*(-0.0000)})
	--({\sx*(0.4900)},{\sy*(-0.0000)})
	--({\sx*(0.5000)},{\sy*(-0.0000)})
	--({\sx*(0.5100)},{\sy*(-0.0000)})
	--({\sx*(0.5200)},{\sy*(-0.0000)})
	--({\sx*(0.5300)},{\sy*(-0.0000)})
	--({\sx*(0.5400)},{\sy*(-0.0000)})
	--({\sx*(0.5500)},{\sy*(-0.0000)})
	--({\sx*(0.5600)},{\sy*(-0.0000)})
	--({\sx*(0.5700)},{\sy*(-0.0000)})
	--({\sx*(0.5800)},{\sy*(-0.0000)})
	--({\sx*(0.5900)},{\sy*(-0.0000)})
	--({\sx*(0.6000)},{\sy*(-0.0000)})
	--({\sx*(0.6100)},{\sy*(-0.0000)})
	--({\sx*(0.6200)},{\sy*(-0.0000)})
	--({\sx*(0.6300)},{\sy*(0.0000)})
	--({\sx*(0.6400)},{\sy*(0.0000)})
	--({\sx*(0.6500)},{\sy*(0.0000)})
	--({\sx*(0.6600)},{\sy*(0.0000)})
	--({\sx*(0.6700)},{\sy*(0.0000)})
	--({\sx*(0.6800)},{\sy*(0.0000)})
	--({\sx*(0.6900)},{\sy*(0.0000)})
	--({\sx*(0.7000)},{\sy*(0.0000)})
	--({\sx*(0.7100)},{\sy*(0.0000)})
	--({\sx*(0.7200)},{\sy*(0.0000)})
	--({\sx*(0.7300)},{\sy*(0.0000)})
	--({\sx*(0.7400)},{\sy*(0.0000)})
	--({\sx*(0.7500)},{\sy*(0.0000)})
	--({\sx*(0.7600)},{\sy*(0.0000)})
	--({\sx*(0.7700)},{\sy*(0.0000)})
	--({\sx*(0.7800)},{\sy*(0.0000)})
	--({\sx*(0.7900)},{\sy*(0.0000)})
	--({\sx*(0.8000)},{\sy*(0.0000)})
	--({\sx*(0.8100)},{\sy*(0.0000)})
	--({\sx*(0.8200)},{\sy*(0.0000)})
	--({\sx*(0.8300)},{\sy*(0.0000)})
	--({\sx*(0.8400)},{\sy*(-0.0000)})
	--({\sx*(0.8500)},{\sy*(-0.0000)})
	--({\sx*(0.8600)},{\sy*(-0.0000)})
	--({\sx*(0.8700)},{\sy*(-0.0000)})
	--({\sx*(0.8800)},{\sy*(-0.0000)})
	--({\sx*(0.8900)},{\sy*(-0.0000)})
	--({\sx*(0.9000)},{\sy*(-0.0000)})
	--({\sx*(0.9100)},{\sy*(-0.0000)})
	--({\sx*(0.9200)},{\sy*(-0.0000)})
	--({\sx*(0.9300)},{\sy*(-0.0000)})
	--({\sx*(0.9400)},{\sy*(-0.0000)})
	--({\sx*(0.9500)},{\sy*(-0.0000)})
	--({\sx*(0.9600)},{\sy*(-0.0000)})
	--({\sx*(0.9700)},{\sy*(-0.0000)})
	--({\sx*(0.9800)},{\sy*(-0.0000)})
	--({\sx*(0.9900)},{\sy*(-0.0000)})
	--({\sx*(1.0000)},{\sy*(-0.0000)})
	--({\sx*(1.0100)},{\sy*(-0.0000)})
	--({\sx*(1.0200)},{\sy*(-0.0000)})
	--({\sx*(1.0300)},{\sy*(-0.0000)})
	--({\sx*(1.0400)},{\sy*(-0.0000)})
	--({\sx*(1.0500)},{\sy*(0.0000)})
	--({\sx*(1.0600)},{\sy*(0.0000)})
	--({\sx*(1.0700)},{\sy*(0.0000)})
	--({\sx*(1.0800)},{\sy*(0.0000)})
	--({\sx*(1.0900)},{\sy*(0.0000)})
	--({\sx*(1.1000)},{\sy*(0.0000)})
	--({\sx*(1.1100)},{\sy*(0.0000)})
	--({\sx*(1.1200)},{\sy*(0.0000)})
	--({\sx*(1.1300)},{\sy*(0.0000)})
	--({\sx*(1.1400)},{\sy*(0.0000)})
	--({\sx*(1.1500)},{\sy*(0.0000)})
	--({\sx*(1.1600)},{\sy*(0.0000)})
	--({\sx*(1.1700)},{\sy*(0.0000)})
	--({\sx*(1.1800)},{\sy*(0.0000)})
	--({\sx*(1.1900)},{\sy*(0.0000)})
	--({\sx*(1.2000)},{\sy*(0.0000)})
	--({\sx*(1.2100)},{\sy*(0.0000)})
	--({\sx*(1.2200)},{\sy*(0.0000)})
	--({\sx*(1.2300)},{\sy*(0.0000)})
	--({\sx*(1.2400)},{\sy*(0.0000)})
	--({\sx*(1.2500)},{\sy*(0.0000)})
	--({\sx*(1.2600)},{\sy*(-0.0000)})
	--({\sx*(1.2700)},{\sy*(-0.0000)})
	--({\sx*(1.2800)},{\sy*(-0.0000)})
	--({\sx*(1.2900)},{\sy*(-0.0000)})
	--({\sx*(1.3000)},{\sy*(-0.0000)})
	--({\sx*(1.3100)},{\sy*(-0.0000)})
	--({\sx*(1.3200)},{\sy*(-0.0000)})
	--({\sx*(1.3300)},{\sy*(-0.0000)})
	--({\sx*(1.3400)},{\sy*(-0.0000)})
	--({\sx*(1.3500)},{\sy*(-0.0000)})
	--({\sx*(1.3600)},{\sy*(-0.0000)})
	--({\sx*(1.3700)},{\sy*(-0.0000)})
	--({\sx*(1.3800)},{\sy*(-0.0000)})
	--({\sx*(1.3900)},{\sy*(-0.0000)})
	--({\sx*(1.4000)},{\sy*(-0.0000)})
	--({\sx*(1.4100)},{\sy*(-0.0000)})
	--({\sx*(1.4200)},{\sy*(-0.0000)})
	--({\sx*(1.4300)},{\sy*(-0.0000)})
	--({\sx*(1.4400)},{\sy*(-0.0000)})
	--({\sx*(1.4500)},{\sy*(-0.0000)})
	--({\sx*(1.4600)},{\sy*(0.0000)})
	--({\sx*(1.4700)},{\sy*(0.0000)})
	--({\sx*(1.4800)},{\sy*(0.0000)})
	--({\sx*(1.4900)},{\sy*(0.0000)})
	--({\sx*(1.5000)},{\sy*(0.0000)})
	--({\sx*(1.5100)},{\sy*(0.0000)})
	--({\sx*(1.5200)},{\sy*(0.0000)})
	--({\sx*(1.5300)},{\sy*(0.0000)})
	--({\sx*(1.5400)},{\sy*(0.0000)})
	--({\sx*(1.5500)},{\sy*(0.0000)})
	--({\sx*(1.5600)},{\sy*(0.0000)})
	--({\sx*(1.5700)},{\sy*(0.0000)})
	--({\sx*(1.5800)},{\sy*(0.0000)})
	--({\sx*(1.5900)},{\sy*(0.0000)})
	--({\sx*(1.6000)},{\sy*(0.0000)})
	--({\sx*(1.6100)},{\sy*(0.0000)})
	--({\sx*(1.6200)},{\sy*(0.0000)})
	--({\sx*(1.6300)},{\sy*(0.0000)})
	--({\sx*(1.6400)},{\sy*(0.0000)})
	--({\sx*(1.6500)},{\sy*(0.0000)})
	--({\sx*(1.6600)},{\sy*(0.0000)})
	--({\sx*(1.6700)},{\sy*(-0.0000)})
	--({\sx*(1.6800)},{\sy*(-0.0000)})
	--({\sx*(1.6900)},{\sy*(-0.0000)})
	--({\sx*(1.7000)},{\sy*(-0.0000)})
	--({\sx*(1.7100)},{\sy*(-0.0000)})
	--({\sx*(1.7200)},{\sy*(-0.0000)})
	--({\sx*(1.7300)},{\sy*(-0.0000)})
	--({\sx*(1.7400)},{\sy*(-0.0000)})
	--({\sx*(1.7500)},{\sy*(-0.0000)})
	--({\sx*(1.7600)},{\sy*(-0.0000)})
	--({\sx*(1.7700)},{\sy*(-0.0000)})
	--({\sx*(1.7800)},{\sy*(-0.0000)})
	--({\sx*(1.7900)},{\sy*(-0.0000)})
	--({\sx*(1.8000)},{\sy*(-0.0000)})
	--({\sx*(1.8100)},{\sy*(-0.0000)})
	--({\sx*(1.8200)},{\sy*(-0.0000)})
	--({\sx*(1.8300)},{\sy*(-0.0000)})
	--({\sx*(1.8400)},{\sy*(-0.0000)})
	--({\sx*(1.8500)},{\sy*(-0.0000)})
	--({\sx*(1.8600)},{\sy*(-0.0000)})
	--({\sx*(1.8700)},{\sy*(-0.0000)})
	--({\sx*(1.8800)},{\sy*(0.0000)})
	--({\sx*(1.8900)},{\sy*(0.0000)})
	--({\sx*(1.9000)},{\sy*(0.0000)})
	--({\sx*(1.9100)},{\sy*(0.0000)})
	--({\sx*(1.9200)},{\sy*(0.0000)})
	--({\sx*(1.9300)},{\sy*(0.0000)})
	--({\sx*(1.9400)},{\sy*(0.0000)})
	--({\sx*(1.9500)},{\sy*(0.0000)})
	--({\sx*(1.9600)},{\sy*(0.0000)})
	--({\sx*(1.9700)},{\sy*(0.0000)})
	--({\sx*(1.9800)},{\sy*(0.0000)})
	--({\sx*(1.9900)},{\sy*(0.0000)})
	--({\sx*(2.0000)},{\sy*(0.0000)})
	--({\sx*(2.0100)},{\sy*(0.0000)})
	--({\sx*(2.0200)},{\sy*(0.0000)})
	--({\sx*(2.0300)},{\sy*(0.0000)})
	--({\sx*(2.0400)},{\sy*(0.0000)})
	--({\sx*(2.0500)},{\sy*(0.0000)})
	--({\sx*(2.0600)},{\sy*(0.0000)})
	--({\sx*(2.0700)},{\sy*(0.0000)})
	--({\sx*(2.0800)},{\sy*(0.0000)})
	--({\sx*(2.0900)},{\sy*(-0.0000)})
	--({\sx*(2.1000)},{\sy*(-0.0000)})
	--({\sx*(2.1100)},{\sy*(-0.0000)})
	--({\sx*(2.1200)},{\sy*(-0.0000)})
	--({\sx*(2.1300)},{\sy*(-0.0000)})
	--({\sx*(2.1400)},{\sy*(-0.0000)})
	--({\sx*(2.1500)},{\sy*(-0.0000)})
	--({\sx*(2.1600)},{\sy*(-0.0000)})
	--({\sx*(2.1700)},{\sy*(-0.0000)})
	--({\sx*(2.1800)},{\sy*(-0.0000)})
	--({\sx*(2.1900)},{\sy*(-0.0000)})
	--({\sx*(2.2000)},{\sy*(-0.0000)})
	--({\sx*(2.2100)},{\sy*(-0.0000)})
	--({\sx*(2.2200)},{\sy*(-0.0000)})
	--({\sx*(2.2300)},{\sy*(-0.0000)})
	--({\sx*(2.2400)},{\sy*(-0.0000)})
	--({\sx*(2.2500)},{\sy*(-0.0000)})
	--({\sx*(2.2600)},{\sy*(-0.0000)})
	--({\sx*(2.2700)},{\sy*(-0.0000)})
	--({\sx*(2.2800)},{\sy*(-0.0000)})
	--({\sx*(2.2900)},{\sy*(-0.0000)})
	--({\sx*(2.3000)},{\sy*(0.0000)})
	--({\sx*(2.3100)},{\sy*(0.0000)})
	--({\sx*(2.3200)},{\sy*(0.0000)})
	--({\sx*(2.3300)},{\sy*(0.0000)})
	--({\sx*(2.3400)},{\sy*(0.0000)})
	--({\sx*(2.3500)},{\sy*(0.0000)})
	--({\sx*(2.3600)},{\sy*(0.0000)})
	--({\sx*(2.3700)},{\sy*(0.0000)})
	--({\sx*(2.3800)},{\sy*(0.0000)})
	--({\sx*(2.3900)},{\sy*(0.0000)})
	--({\sx*(2.4000)},{\sy*(0.0000)})
	--({\sx*(2.4100)},{\sy*(0.0000)})
	--({\sx*(2.4200)},{\sy*(0.0000)})
	--({\sx*(2.4300)},{\sy*(0.0000)})
	--({\sx*(2.4400)},{\sy*(0.0000)})
	--({\sx*(2.4500)},{\sy*(0.0000)})
	--({\sx*(2.4600)},{\sy*(0.0000)})
	--({\sx*(2.4700)},{\sy*(0.0000)})
	--({\sx*(2.4800)},{\sy*(0.0000)})
	--({\sx*(2.4900)},{\sy*(0.0000)})
	--({\sx*(2.5000)},{\sy*(0.0000)})
	--({\sx*(2.5100)},{\sy*(-0.0000)})
	--({\sx*(2.5200)},{\sy*(-0.0000)})
	--({\sx*(2.5300)},{\sy*(-0.0000)})
	--({\sx*(2.5400)},{\sy*(-0.0000)})
	--({\sx*(2.5500)},{\sy*(-0.0000)})
	--({\sx*(2.5600)},{\sy*(-0.0000)})
	--({\sx*(2.5700)},{\sy*(-0.0000)})
	--({\sx*(2.5800)},{\sy*(-0.0000)})
	--({\sx*(2.5900)},{\sy*(-0.0000)})
	--({\sx*(2.6000)},{\sy*(-0.0000)})
	--({\sx*(2.6100)},{\sy*(-0.0000)})
	--({\sx*(2.6200)},{\sy*(-0.0000)})
	--({\sx*(2.6300)},{\sy*(-0.0000)})
	--({\sx*(2.6400)},{\sy*(-0.0000)})
	--({\sx*(2.6500)},{\sy*(-0.0000)})
	--({\sx*(2.6600)},{\sy*(-0.0000)})
	--({\sx*(2.6700)},{\sy*(-0.0000)})
	--({\sx*(2.6800)},{\sy*(-0.0000)})
	--({\sx*(2.6900)},{\sy*(-0.0000)})
	--({\sx*(2.7000)},{\sy*(-0.0000)})
	--({\sx*(2.7100)},{\sy*(0.0000)})
	--({\sx*(2.7200)},{\sy*(0.0000)})
	--({\sx*(2.7300)},{\sy*(0.0000)})
	--({\sx*(2.7400)},{\sy*(0.0000)})
	--({\sx*(2.7500)},{\sy*(0.0000)})
	--({\sx*(2.7600)},{\sy*(0.0000)})
	--({\sx*(2.7700)},{\sy*(0.0000)})
	--({\sx*(2.7800)},{\sy*(0.0000)})
	--({\sx*(2.7900)},{\sy*(0.0000)})
	--({\sx*(2.8000)},{\sy*(0.0000)})
	--({\sx*(2.8100)},{\sy*(0.0000)})
	--({\sx*(2.8200)},{\sy*(0.0000)})
	--({\sx*(2.8300)},{\sy*(0.0000)})
	--({\sx*(2.8400)},{\sy*(0.0000)})
	--({\sx*(2.8500)},{\sy*(0.0000)})
	--({\sx*(2.8600)},{\sy*(0.0000)})
	--({\sx*(2.8700)},{\sy*(0.0000)})
	--({\sx*(2.8800)},{\sy*(0.0000)})
	--({\sx*(2.8900)},{\sy*(0.0000)})
	--({\sx*(2.9000)},{\sy*(0.0000)})
	--({\sx*(2.9100)},{\sy*(0.0000)})
	--({\sx*(2.9200)},{\sy*(-0.0000)})
	--({\sx*(2.9300)},{\sy*(-0.0000)})
	--({\sx*(2.9400)},{\sy*(-0.0000)})
	--({\sx*(2.9500)},{\sy*(-0.0000)})
	--({\sx*(2.9600)},{\sy*(-0.0000)})
	--({\sx*(2.9700)},{\sy*(-0.0000)})
	--({\sx*(2.9800)},{\sy*(-0.0000)})
	--({\sx*(2.9900)},{\sy*(-0.0000)})
	--({\sx*(3.0000)},{\sy*(-0.0000)})
	--({\sx*(3.0100)},{\sy*(-0.0000)})
	--({\sx*(3.0200)},{\sy*(-0.0000)})
	--({\sx*(3.0300)},{\sy*(-0.0000)})
	--({\sx*(3.0400)},{\sy*(-0.0000)})
	--({\sx*(3.0500)},{\sy*(-0.0000)})
	--({\sx*(3.0600)},{\sy*(-0.0000)})
	--({\sx*(3.0700)},{\sy*(-0.0000)})
	--({\sx*(3.0800)},{\sy*(-0.0000)})
	--({\sx*(3.0900)},{\sy*(-0.0000)})
	--({\sx*(3.1000)},{\sy*(-0.0000)})
	--({\sx*(3.1100)},{\sy*(-0.0000)})
	--({\sx*(3.1200)},{\sy*(-0.0000)})
	--({\sx*(3.1300)},{\sy*(0.0000)})
	--({\sx*(3.1400)},{\sy*(0.0000)})
	--({\sx*(3.1500)},{\sy*(0.0000)})
	--({\sx*(3.1600)},{\sy*(0.0000)})
	--({\sx*(3.1700)},{\sy*(0.0000)})
	--({\sx*(3.1800)},{\sy*(0.0000)})
	--({\sx*(3.1900)},{\sy*(0.0000)})
	--({\sx*(3.2000)},{\sy*(0.0000)})
	--({\sx*(3.2100)},{\sy*(0.0000)})
	--({\sx*(3.2200)},{\sy*(0.0000)})
	--({\sx*(3.2300)},{\sy*(0.0000)})
	--({\sx*(3.2400)},{\sy*(0.0000)})
	--({\sx*(3.2500)},{\sy*(0.0000)})
	--({\sx*(3.2600)},{\sy*(0.0000)})
	--({\sx*(3.2700)},{\sy*(0.0000)})
	--({\sx*(3.2800)},{\sy*(0.0000)})
	--({\sx*(3.2900)},{\sy*(0.0000)})
	--({\sx*(3.3000)},{\sy*(0.0000)})
	--({\sx*(3.3100)},{\sy*(0.0000)})
	--({\sx*(3.3200)},{\sy*(0.0000)})
	--({\sx*(3.3300)},{\sy*(0.0000)})
	--({\sx*(3.3400)},{\sy*(-0.0000)})
	--({\sx*(3.3500)},{\sy*(-0.0000)})
	--({\sx*(3.3600)},{\sy*(-0.0000)})
	--({\sx*(3.3700)},{\sy*(-0.0000)})
	--({\sx*(3.3800)},{\sy*(-0.0000)})
	--({\sx*(3.3900)},{\sy*(-0.0000)})
	--({\sx*(3.4000)},{\sy*(-0.0000)})
	--({\sx*(3.4100)},{\sy*(-0.0000)})
	--({\sx*(3.4200)},{\sy*(-0.0000)})
	--({\sx*(3.4300)},{\sy*(-0.0000)})
	--({\sx*(3.4400)},{\sy*(-0.0000)})
	--({\sx*(3.4500)},{\sy*(-0.0000)})
	--({\sx*(3.4600)},{\sy*(-0.0000)})
	--({\sx*(3.4700)},{\sy*(-0.0000)})
	--({\sx*(3.4800)},{\sy*(-0.0000)})
	--({\sx*(3.4900)},{\sy*(-0.0000)})
	--({\sx*(3.5000)},{\sy*(-0.0000)})
	--({\sx*(3.5100)},{\sy*(-0.0000)})
	--({\sx*(3.5200)},{\sy*(-0.0000)})
	--({\sx*(3.5300)},{\sy*(-0.0000)})
	--({\sx*(3.5400)},{\sy*(-0.0000)})
	--({\sx*(3.5500)},{\sy*(0.0000)})
	--({\sx*(3.5600)},{\sy*(0.0000)})
	--({\sx*(3.5700)},{\sy*(0.0000)})
	--({\sx*(3.5800)},{\sy*(0.0000)})
	--({\sx*(3.5900)},{\sy*(0.0000)})
	--({\sx*(3.6000)},{\sy*(0.0000)})
	--({\sx*(3.6100)},{\sy*(0.0000)})
	--({\sx*(3.6200)},{\sy*(0.0000)})
	--({\sx*(3.6300)},{\sy*(0.0000)})
	--({\sx*(3.6400)},{\sy*(0.0000)})
	--({\sx*(3.6500)},{\sy*(0.0000)})
	--({\sx*(3.6600)},{\sy*(0.0000)})
	--({\sx*(3.6700)},{\sy*(0.0000)})
	--({\sx*(3.6800)},{\sy*(0.0000)})
	--({\sx*(3.6900)},{\sy*(0.0000)})
	--({\sx*(3.7000)},{\sy*(0.0000)})
	--({\sx*(3.7100)},{\sy*(0.0000)})
	--({\sx*(3.7200)},{\sy*(0.0000)})
	--({\sx*(3.7300)},{\sy*(0.0000)})
	--({\sx*(3.7400)},{\sy*(0.0000)})
	--({\sx*(3.7500)},{\sy*(0.0000)})
	--({\sx*(3.7600)},{\sy*(-0.0000)})
	--({\sx*(3.7700)},{\sy*(-0.0000)})
	--({\sx*(3.7800)},{\sy*(-0.0000)})
	--({\sx*(3.7900)},{\sy*(-0.0000)})
	--({\sx*(3.8000)},{\sy*(-0.0000)})
	--({\sx*(3.8100)},{\sy*(-0.0000)})
	--({\sx*(3.8200)},{\sy*(-0.0000)})
	--({\sx*(3.8300)},{\sy*(-0.0000)})
	--({\sx*(3.8400)},{\sy*(-0.0000)})
	--({\sx*(3.8500)},{\sy*(-0.0000)})
	--({\sx*(3.8600)},{\sy*(-0.0000)})
	--({\sx*(3.8700)},{\sy*(-0.0000)})
	--({\sx*(3.8800)},{\sy*(-0.0000)})
	--({\sx*(3.8900)},{\sy*(-0.0000)})
	--({\sx*(3.9000)},{\sy*(-0.0000)})
	--({\sx*(3.9100)},{\sy*(-0.0000)})
	--({\sx*(3.9200)},{\sy*(-0.0000)})
	--({\sx*(3.9300)},{\sy*(-0.0000)})
	--({\sx*(3.9400)},{\sy*(-0.0000)})
	--({\sx*(3.9500)},{\sy*(-0.0000)})
	--({\sx*(3.9600)},{\sy*(0.0000)})
	--({\sx*(3.9700)},{\sy*(0.0000)})
	--({\sx*(3.9800)},{\sy*(0.0000)})
	--({\sx*(3.9900)},{\sy*(0.0000)})
	--({\sx*(4.0000)},{\sy*(0.0000)})
	--({\sx*(4.0100)},{\sy*(0.0000)})
	--({\sx*(4.0200)},{\sy*(0.0000)})
	--({\sx*(4.0300)},{\sy*(0.0000)})
	--({\sx*(4.0400)},{\sy*(0.0000)})
	--({\sx*(4.0500)},{\sy*(0.0000)})
	--({\sx*(4.0600)},{\sy*(0.0000)})
	--({\sx*(4.0700)},{\sy*(0.0000)})
	--({\sx*(4.0800)},{\sy*(0.0000)})
	--({\sx*(4.0900)},{\sy*(0.0000)})
	--({\sx*(4.1000)},{\sy*(0.0000)})
	--({\sx*(4.1100)},{\sy*(0.0000)})
	--({\sx*(4.1200)},{\sy*(0.0000)})
	--({\sx*(4.1300)},{\sy*(0.0000)})
	--({\sx*(4.1400)},{\sy*(0.0000)})
	--({\sx*(4.1500)},{\sy*(0.0000)})
	--({\sx*(4.1600)},{\sy*(0.0000)})
	--({\sx*(4.1700)},{\sy*(-0.0000)})
	--({\sx*(4.1800)},{\sy*(-0.0000)})
	--({\sx*(4.1900)},{\sy*(-0.0000)})
	--({\sx*(4.2000)},{\sy*(-0.0000)})
	--({\sx*(4.2100)},{\sy*(-0.0000)})
	--({\sx*(4.2200)},{\sy*(-0.0000)})
	--({\sx*(4.2300)},{\sy*(-0.0000)})
	--({\sx*(4.2400)},{\sy*(-0.0000)})
	--({\sx*(4.2500)},{\sy*(-0.0000)})
	--({\sx*(4.2600)},{\sy*(-0.0000)})
	--({\sx*(4.2700)},{\sy*(-0.0000)})
	--({\sx*(4.2800)},{\sy*(-0.0000)})
	--({\sx*(4.2900)},{\sy*(-0.0000)})
	--({\sx*(4.3000)},{\sy*(-0.0000)})
	--({\sx*(4.3100)},{\sy*(-0.0000)})
	--({\sx*(4.3200)},{\sy*(-0.0000)})
	--({\sx*(4.3300)},{\sy*(-0.0000)})
	--({\sx*(4.3400)},{\sy*(-0.0000)})
	--({\sx*(4.3500)},{\sy*(-0.0000)})
	--({\sx*(4.3600)},{\sy*(-0.0000)})
	--({\sx*(4.3700)},{\sy*(-0.0000)})
	--({\sx*(4.3800)},{\sy*(0.0000)})
	--({\sx*(4.3900)},{\sy*(0.0000)})
	--({\sx*(4.4000)},{\sy*(0.0000)})
	--({\sx*(4.4100)},{\sy*(0.0000)})
	--({\sx*(4.4200)},{\sy*(0.0000)})
	--({\sx*(4.4300)},{\sy*(0.0000)})
	--({\sx*(4.4400)},{\sy*(0.0000)})
	--({\sx*(4.4500)},{\sy*(0.0000)})
	--({\sx*(4.4600)},{\sy*(0.0000)})
	--({\sx*(4.4700)},{\sy*(0.0000)})
	--({\sx*(4.4800)},{\sy*(0.0000)})
	--({\sx*(4.4900)},{\sy*(0.0000)})
	--({\sx*(4.5000)},{\sy*(0.0000)})
	--({\sx*(4.5100)},{\sy*(0.0000)})
	--({\sx*(4.5200)},{\sy*(0.0000)})
	--({\sx*(4.5300)},{\sy*(0.0000)})
	--({\sx*(4.5400)},{\sy*(0.0000)})
	--({\sx*(4.5500)},{\sy*(0.0000)})
	--({\sx*(4.5600)},{\sy*(0.0000)})
	--({\sx*(4.5700)},{\sy*(0.0000)})
	--({\sx*(4.5800)},{\sy*(0.0000)})
	--({\sx*(4.5900)},{\sy*(-0.0000)})
	--({\sx*(4.6000)},{\sy*(-0.0000)})
	--({\sx*(4.6100)},{\sy*(-0.0000)})
	--({\sx*(4.6200)},{\sy*(-0.0000)})
	--({\sx*(4.6300)},{\sy*(-0.0000)})
	--({\sx*(4.6400)},{\sy*(-0.0000)})
	--({\sx*(4.6500)},{\sy*(-0.0000)})
	--({\sx*(4.6600)},{\sy*(-0.0000)})
	--({\sx*(4.6700)},{\sy*(-0.0000)})
	--({\sx*(4.6800)},{\sy*(-0.0000)})
	--({\sx*(4.6900)},{\sy*(-0.0000)})
	--({\sx*(4.7000)},{\sy*(-0.0000)})
	--({\sx*(4.7100)},{\sy*(-0.0000)})
	--({\sx*(4.7200)},{\sy*(-0.0000)})
	--({\sx*(4.7300)},{\sy*(-0.0000)})
	--({\sx*(4.7400)},{\sy*(-0.0000)})
	--({\sx*(4.7500)},{\sy*(-0.0000)})
	--({\sx*(4.7600)},{\sy*(-0.0000)})
	--({\sx*(4.7700)},{\sy*(-0.0000)})
	--({\sx*(4.7800)},{\sy*(-0.0000)})
	--({\sx*(4.7900)},{\sy*(-0.0000)})
	--({\sx*(4.8000)},{\sy*(0.0000)})
	--({\sx*(4.8100)},{\sy*(0.0000)})
	--({\sx*(4.8200)},{\sy*(0.0000)})
	--({\sx*(4.8300)},{\sy*(0.0000)})
	--({\sx*(4.8400)},{\sy*(0.0000)})
	--({\sx*(4.8500)},{\sy*(0.0000)})
	--({\sx*(4.8600)},{\sy*(0.0000)})
	--({\sx*(4.8700)},{\sy*(0.0000)})
	--({\sx*(4.8800)},{\sy*(0.0000)})
	--({\sx*(4.8900)},{\sy*(0.0001)})
	--({\sx*(4.9000)},{\sy*(0.0001)})
	--({\sx*(4.9100)},{\sy*(0.0001)})
	--({\sx*(4.9200)},{\sy*(0.0001)})
	--({\sx*(4.9300)},{\sy*(0.0001)})
	--({\sx*(4.9400)},{\sy*(0.0001)})
	--({\sx*(4.9500)},{\sy*(0.0001)})
	--({\sx*(4.9600)},{\sy*(0.0001)})
	--({\sx*(4.9700)},{\sy*(0.0001)})
	--({\sx*(4.9800)},{\sy*(0.0001)})
	--({\sx*(4.9900)},{\sy*(0.0001)})
	--({\sx*(5.0000)},{\sy*(0.0000)});
}
\def\xwertem{
\fill[color=red] (0.0000,0) circle[radius={0.07/\skala}];
\fill[color=red] (0.1923,0) circle[radius={0.07/\skala}];
\fill[color=red] (0.3846,0) circle[radius={0.07/\skala}];
\fill[color=red] (0.5769,0) circle[radius={0.07/\skala}];
\fill[color=red] (0.7692,0) circle[radius={0.07/\skala}];
\fill[color=red] (0.9615,0) circle[radius={0.07/\skala}];
\fill[color=red] (1.1538,0) circle[radius={0.07/\skala}];
\fill[color=red] (1.3462,0) circle[radius={0.07/\skala}];
\fill[color=red] (1.5385,0) circle[radius={0.07/\skala}];
\fill[color=red] (1.7308,0) circle[radius={0.07/\skala}];
\fill[color=red] (1.9231,0) circle[radius={0.07/\skala}];
\fill[color=red] (2.1154,0) circle[radius={0.07/\skala}];
\fill[color=red] (2.3077,0) circle[radius={0.07/\skala}];
\fill[color=red] (2.5000,0) circle[radius={0.07/\skala}];
\fill[color=red] (2.6923,0) circle[radius={0.07/\skala}];
\fill[color=red] (2.8846,0) circle[radius={0.07/\skala}];
\fill[color=red] (3.0769,0) circle[radius={0.07/\skala}];
\fill[color=red] (3.2692,0) circle[radius={0.07/\skala}];
\fill[color=red] (3.4615,0) circle[radius={0.07/\skala}];
\fill[color=red] (3.6538,0) circle[radius={0.07/\skala}];
\fill[color=red] (3.8462,0) circle[radius={0.07/\skala}];
\fill[color=red] (4.0385,0) circle[radius={0.07/\skala}];
\fill[color=red] (4.2308,0) circle[radius={0.07/\skala}];
\fill[color=red] (4.4231,0) circle[radius={0.07/\skala}];
\fill[color=red] (4.6154,0) circle[radius={0.07/\skala}];
\fill[color=red] (4.8077,0) circle[radius={0.07/\skala}];
\fill[color=red] (5.0000,0) circle[radius={0.07/\skala}];
}
\def\punktem{26}
\def\maxfehlerm{1.088\cdot 10^{-10}}
\def\fehlerm{
\draw[color=red,line width=1.4pt,line join=round] ({\sx*(0.000)},{\sy*(0.0000)})
	--({\sx*(0.0100)},{\sy*(0.4814)})
	--({\sx*(0.0200)},{\sy*(0.7506)})
	--({\sx*(0.0300)},{\sy*(0.9426)})
	--({\sx*(0.0400)},{\sy*(1.0000)})
	--({\sx*(0.0500)},{\sy*(0.9911)})
	--({\sx*(0.0600)},{\sy*(0.9289)})
	--({\sx*(0.0700)},{\sy*(0.9157)})
	--({\sx*(0.0800)},{\sy*(0.7768)})
	--({\sx*(0.0900)},{\sy*(0.6540)})
	--({\sx*(0.1000)},{\sy*(0.5605)})
	--({\sx*(0.1100)},{\sy*(0.4618)})
	--({\sx*(0.1200)},{\sy*(0.3902)})
	--({\sx*(0.1300)},{\sy*(0.2943)})
	--({\sx*(0.1400)},{\sy*(0.2197)})
	--({\sx*(0.1500)},{\sy*(0.1588)})
	--({\sx*(0.1600)},{\sy*(0.1076)})
	--({\sx*(0.1700)},{\sy*(0.0629)})
	--({\sx*(0.1800)},{\sy*(0.0306)})
	--({\sx*(0.1900)},{\sy*(0.0053)})
	--({\sx*(0.2000)},{\sy*(-0.0147)})
	--({\sx*(0.2100)},{\sy*(-0.0288)})
	--({\sx*(0.2200)},{\sy*(-0.0387)})
	--({\sx*(0.2300)},{\sy*(-0.0426)})
	--({\sx*(0.2400)},{\sy*(-0.0454)})
	--({\sx*(0.2500)},{\sy*(-0.0461)})
	--({\sx*(0.2600)},{\sy*(-0.0439)})
	--({\sx*(0.2700)},{\sy*(-0.0395)})
	--({\sx*(0.2800)},{\sy*(-0.0379)})
	--({\sx*(0.2900)},{\sy*(-0.0320)})
	--({\sx*(0.3000)},{\sy*(-0.0286)})
	--({\sx*(0.3100)},{\sy*(-0.0228)})
	--({\sx*(0.3200)},{\sy*(-0.0179)})
	--({\sx*(0.3300)},{\sy*(-0.0150)})
	--({\sx*(0.3400)},{\sy*(-0.0115)})
	--({\sx*(0.3500)},{\sy*(-0.0080)})
	--({\sx*(0.3600)},{\sy*(-0.0053)})
	--({\sx*(0.3700)},{\sy*(-0.0028)})
	--({\sx*(0.3800)},{\sy*(-0.0008)})
	--({\sx*(0.3900)},{\sy*(0.0008)})
	--({\sx*(0.4000)},{\sy*(0.0021)})
	--({\sx*(0.4100)},{\sy*(0.0028)})
	--({\sx*(0.4200)},{\sy*(0.0033)})
	--({\sx*(0.4300)},{\sy*(0.0036)})
	--({\sx*(0.4400)},{\sy*(0.0041)})
	--({\sx*(0.4500)},{\sy*(0.0040)})
	--({\sx*(0.4600)},{\sy*(0.0040)})
	--({\sx*(0.4700)},{\sy*(0.0034)})
	--({\sx*(0.4800)},{\sy*(0.0033)})
	--({\sx*(0.4900)},{\sy*(0.0029)})
	--({\sx*(0.5000)},{\sy*(0.0024)})
	--({\sx*(0.5100)},{\sy*(0.0021)})
	--({\sx*(0.5200)},{\sy*(0.0017)})
	--({\sx*(0.5300)},{\sy*(0.0013)})
	--({\sx*(0.5400)},{\sy*(0.0010)})
	--({\sx*(0.5500)},{\sy*(0.0006)})
	--({\sx*(0.5600)},{\sy*(0.0004)})
	--({\sx*(0.5700)},{\sy*(0.0001)})
	--({\sx*(0.5800)},{\sy*(-0.0001)})
	--({\sx*(0.5900)},{\sy*(-0.0002)})
	--({\sx*(0.6000)},{\sy*(-0.0003)})
	--({\sx*(0.6100)},{\sy*(-0.0004)})
	--({\sx*(0.6200)},{\sy*(-0.0005)})
	--({\sx*(0.6300)},{\sy*(-0.0005)})
	--({\sx*(0.6400)},{\sy*(-0.0006)})
	--({\sx*(0.6500)},{\sy*(-0.0005)})
	--({\sx*(0.6600)},{\sy*(-0.0005)})
	--({\sx*(0.6700)},{\sy*(-0.0004)})
	--({\sx*(0.6800)},{\sy*(-0.0004)})
	--({\sx*(0.6900)},{\sy*(-0.0003)})
	--({\sx*(0.7000)},{\sy*(-0.0003)})
	--({\sx*(0.7100)},{\sy*(-0.0002)})
	--({\sx*(0.7200)},{\sy*(-0.0002)})
	--({\sx*(0.7300)},{\sy*(-0.0001)})
	--({\sx*(0.7400)},{\sy*(-0.0001)})
	--({\sx*(0.7500)},{\sy*(-0.0001)})
	--({\sx*(0.7600)},{\sy*(-0.0000)})
	--({\sx*(0.7700)},{\sy*(0.0000)})
	--({\sx*(0.7800)},{\sy*(0.0000)})
	--({\sx*(0.7900)},{\sy*(0.0000)})
	--({\sx*(0.8000)},{\sy*(0.0001)})
	--({\sx*(0.8100)},{\sy*(0.0001)})
	--({\sx*(0.8200)},{\sy*(0.0001)})
	--({\sx*(0.8300)},{\sy*(0.0001)})
	--({\sx*(0.8400)},{\sy*(0.0001)})
	--({\sx*(0.8500)},{\sy*(0.0001)})
	--({\sx*(0.8600)},{\sy*(0.0001)})
	--({\sx*(0.8700)},{\sy*(0.0001)})
	--({\sx*(0.8800)},{\sy*(0.0001)})
	--({\sx*(0.8900)},{\sy*(0.0001)})
	--({\sx*(0.9000)},{\sy*(0.0001)})
	--({\sx*(0.9100)},{\sy*(0.0000)})
	--({\sx*(0.9200)},{\sy*(0.0000)})
	--({\sx*(0.9300)},{\sy*(0.0000)})
	--({\sx*(0.9400)},{\sy*(0.0000)})
	--({\sx*(0.9500)},{\sy*(0.0000)})
	--({\sx*(0.9600)},{\sy*(0.0000)})
	--({\sx*(0.9700)},{\sy*(-0.0000)})
	--({\sx*(0.9800)},{\sy*(-0.0000)})
	--({\sx*(0.9900)},{\sy*(-0.0000)})
	--({\sx*(1.0000)},{\sy*(-0.0000)})
	--({\sx*(1.0100)},{\sy*(-0.0000)})
	--({\sx*(1.0200)},{\sy*(-0.0000)})
	--({\sx*(1.0300)},{\sy*(-0.0000)})
	--({\sx*(1.0400)},{\sy*(-0.0000)})
	--({\sx*(1.0500)},{\sy*(-0.0000)})
	--({\sx*(1.0600)},{\sy*(-0.0000)})
	--({\sx*(1.0700)},{\sy*(-0.0000)})
	--({\sx*(1.0800)},{\sy*(-0.0000)})
	--({\sx*(1.0900)},{\sy*(-0.0000)})
	--({\sx*(1.1000)},{\sy*(-0.0000)})
	--({\sx*(1.1100)},{\sy*(-0.0000)})
	--({\sx*(1.1200)},{\sy*(-0.0000)})
	--({\sx*(1.1300)},{\sy*(-0.0000)})
	--({\sx*(1.1400)},{\sy*(-0.0000)})
	--({\sx*(1.1500)},{\sy*(0.0000)})
	--({\sx*(1.1600)},{\sy*(0.0000)})
	--({\sx*(1.1700)},{\sy*(0.0000)})
	--({\sx*(1.1800)},{\sy*(0.0000)})
	--({\sx*(1.1900)},{\sy*(0.0000)})
	--({\sx*(1.2000)},{\sy*(0.0000)})
	--({\sx*(1.2100)},{\sy*(0.0000)})
	--({\sx*(1.2200)},{\sy*(0.0000)})
	--({\sx*(1.2300)},{\sy*(0.0000)})
	--({\sx*(1.2400)},{\sy*(0.0000)})
	--({\sx*(1.2500)},{\sy*(0.0000)})
	--({\sx*(1.2600)},{\sy*(0.0000)})
	--({\sx*(1.2700)},{\sy*(0.0000)})
	--({\sx*(1.2800)},{\sy*(0.0000)})
	--({\sx*(1.2900)},{\sy*(0.0000)})
	--({\sx*(1.3000)},{\sy*(0.0000)})
	--({\sx*(1.3100)},{\sy*(0.0000)})
	--({\sx*(1.3200)},{\sy*(0.0000)})
	--({\sx*(1.3300)},{\sy*(0.0000)})
	--({\sx*(1.3400)},{\sy*(0.0000)})
	--({\sx*(1.3500)},{\sy*(0.0000)})
	--({\sx*(1.3600)},{\sy*(-0.0000)})
	--({\sx*(1.3700)},{\sy*(-0.0000)})
	--({\sx*(1.3800)},{\sy*(-0.0000)})
	--({\sx*(1.3900)},{\sy*(-0.0000)})
	--({\sx*(1.4000)},{\sy*(-0.0000)})
	--({\sx*(1.4100)},{\sy*(-0.0000)})
	--({\sx*(1.4200)},{\sy*(-0.0000)})
	--({\sx*(1.4300)},{\sy*(-0.0000)})
	--({\sx*(1.4400)},{\sy*(-0.0000)})
	--({\sx*(1.4500)},{\sy*(-0.0000)})
	--({\sx*(1.4600)},{\sy*(-0.0000)})
	--({\sx*(1.4700)},{\sy*(-0.0000)})
	--({\sx*(1.4800)},{\sy*(-0.0000)})
	--({\sx*(1.4900)},{\sy*(-0.0000)})
	--({\sx*(1.5000)},{\sy*(-0.0000)})
	--({\sx*(1.5100)},{\sy*(0.0000)})
	--({\sx*(1.5200)},{\sy*(-0.0000)})
	--({\sx*(1.5300)},{\sy*(-0.0000)})
	--({\sx*(1.5400)},{\sy*(-0.0000)})
	--({\sx*(1.5500)},{\sy*(0.0000)})
	--({\sx*(1.5600)},{\sy*(-0.0000)})
	--({\sx*(1.5700)},{\sy*(-0.0000)})
	--({\sx*(1.5800)},{\sy*(0.0000)})
	--({\sx*(1.5900)},{\sy*(0.0000)})
	--({\sx*(1.6000)},{\sy*(0.0000)})
	--({\sx*(1.6100)},{\sy*(-0.0000)})
	--({\sx*(1.6200)},{\sy*(0.0000)})
	--({\sx*(1.6300)},{\sy*(0.0000)})
	--({\sx*(1.6400)},{\sy*(-0.0000)})
	--({\sx*(1.6500)},{\sy*(-0.0000)})
	--({\sx*(1.6600)},{\sy*(-0.0000)})
	--({\sx*(1.6700)},{\sy*(0.0000)})
	--({\sx*(1.6800)},{\sy*(0.0000)})
	--({\sx*(1.6900)},{\sy*(0.0000)})
	--({\sx*(1.7000)},{\sy*(-0.0000)})
	--({\sx*(1.7100)},{\sy*(0.0000)})
	--({\sx*(1.7200)},{\sy*(0.0000)})
	--({\sx*(1.7300)},{\sy*(-0.0000)})
	--({\sx*(1.7400)},{\sy*(0.0000)})
	--({\sx*(1.7500)},{\sy*(-0.0000)})
	--({\sx*(1.7600)},{\sy*(-0.0000)})
	--({\sx*(1.7700)},{\sy*(0.0000)})
	--({\sx*(1.7800)},{\sy*(-0.0000)})
	--({\sx*(1.7900)},{\sy*(0.0000)})
	--({\sx*(1.8000)},{\sy*(0.0000)})
	--({\sx*(1.8100)},{\sy*(0.0000)})
	--({\sx*(1.8200)},{\sy*(-0.0000)})
	--({\sx*(1.8300)},{\sy*(0.0000)})
	--({\sx*(1.8400)},{\sy*(0.0000)})
	--({\sx*(1.8500)},{\sy*(-0.0000)})
	--({\sx*(1.8600)},{\sy*(-0.0000)})
	--({\sx*(1.8700)},{\sy*(0.0000)})
	--({\sx*(1.8800)},{\sy*(0.0000)})
	--({\sx*(1.8900)},{\sy*(0.0000)})
	--({\sx*(1.9000)},{\sy*(0.0000)})
	--({\sx*(1.9100)},{\sy*(0.0000)})
	--({\sx*(1.9200)},{\sy*(0.0000)})
	--({\sx*(1.9300)},{\sy*(0.0000)})
	--({\sx*(1.9400)},{\sy*(0.0000)})
	--({\sx*(1.9500)},{\sy*(-0.0000)})
	--({\sx*(1.9600)},{\sy*(0.0000)})
	--({\sx*(1.9700)},{\sy*(0.0000)})
	--({\sx*(1.9800)},{\sy*(0.0000)})
	--({\sx*(1.9900)},{\sy*(0.0000)})
	--({\sx*(2.0000)},{\sy*(0.0000)})
	--({\sx*(2.0100)},{\sy*(0.0000)})
	--({\sx*(2.0200)},{\sy*(0.0000)})
	--({\sx*(2.0300)},{\sy*(0.0000)})
	--({\sx*(2.0400)},{\sy*(0.0000)})
	--({\sx*(2.0500)},{\sy*(0.0000)})
	--({\sx*(2.0600)},{\sy*(0.0000)})
	--({\sx*(2.0700)},{\sy*(-0.0000)})
	--({\sx*(2.0800)},{\sy*(0.0000)})
	--({\sx*(2.0900)},{\sy*(0.0000)})
	--({\sx*(2.1000)},{\sy*(0.0000)})
	--({\sx*(2.1100)},{\sy*(0.0000)})
	--({\sx*(2.1200)},{\sy*(0.0000)})
	--({\sx*(2.1300)},{\sy*(-0.0000)})
	--({\sx*(2.1400)},{\sy*(0.0000)})
	--({\sx*(2.1500)},{\sy*(0.0000)})
	--({\sx*(2.1600)},{\sy*(-0.0000)})
	--({\sx*(2.1700)},{\sy*(-0.0000)})
	--({\sx*(2.1800)},{\sy*(0.0000)})
	--({\sx*(2.1900)},{\sy*(0.0000)})
	--({\sx*(2.2000)},{\sy*(-0.0000)})
	--({\sx*(2.2100)},{\sy*(-0.0000)})
	--({\sx*(2.2200)},{\sy*(-0.0000)})
	--({\sx*(2.2300)},{\sy*(-0.0000)})
	--({\sx*(2.2400)},{\sy*(-0.0000)})
	--({\sx*(2.2500)},{\sy*(-0.0000)})
	--({\sx*(2.2600)},{\sy*(0.0000)})
	--({\sx*(2.2700)},{\sy*(-0.0000)})
	--({\sx*(2.2800)},{\sy*(-0.0000)})
	--({\sx*(2.2900)},{\sy*(-0.0000)})
	--({\sx*(2.3000)},{\sy*(-0.0000)})
	--({\sx*(2.3100)},{\sy*(-0.0000)})
	--({\sx*(2.3200)},{\sy*(-0.0000)})
	--({\sx*(2.3300)},{\sy*(-0.0000)})
	--({\sx*(2.3400)},{\sy*(-0.0000)})
	--({\sx*(2.3500)},{\sy*(0.0000)})
	--({\sx*(2.3600)},{\sy*(-0.0000)})
	--({\sx*(2.3700)},{\sy*(-0.0000)})
	--({\sx*(2.3800)},{\sy*(0.0000)})
	--({\sx*(2.3900)},{\sy*(-0.0000)})
	--({\sx*(2.4000)},{\sy*(0.0000)})
	--({\sx*(2.4100)},{\sy*(-0.0000)})
	--({\sx*(2.4200)},{\sy*(-0.0000)})
	--({\sx*(2.4300)},{\sy*(-0.0000)})
	--({\sx*(2.4400)},{\sy*(0.0000)})
	--({\sx*(2.4500)},{\sy*(-0.0000)})
	--({\sx*(2.4600)},{\sy*(0.0000)})
	--({\sx*(2.4700)},{\sy*(0.0000)})
	--({\sx*(2.4800)},{\sy*(0.0000)})
	--({\sx*(2.4900)},{\sy*(0.0000)})
	--({\sx*(2.5000)},{\sy*(0.0000)})
	--({\sx*(2.5100)},{\sy*(0.0000)})
	--({\sx*(2.5200)},{\sy*(0.0000)})
	--({\sx*(2.5300)},{\sy*(-0.0000)})
	--({\sx*(2.5400)},{\sy*(0.0000)})
	--({\sx*(2.5500)},{\sy*(0.0000)})
	--({\sx*(2.5600)},{\sy*(0.0000)})
	--({\sx*(2.5700)},{\sy*(0.0000)})
	--({\sx*(2.5800)},{\sy*(-0.0000)})
	--({\sx*(2.5900)},{\sy*(-0.0000)})
	--({\sx*(2.6000)},{\sy*(0.0000)})
	--({\sx*(2.6100)},{\sy*(0.0000)})
	--({\sx*(2.6200)},{\sy*(0.0000)})
	--({\sx*(2.6300)},{\sy*(0.0000)})
	--({\sx*(2.6400)},{\sy*(0.0000)})
	--({\sx*(2.6500)},{\sy*(0.0000)})
	--({\sx*(2.6600)},{\sy*(0.0000)})
	--({\sx*(2.6700)},{\sy*(0.0000)})
	--({\sx*(2.6800)},{\sy*(0.0000)})
	--({\sx*(2.6900)},{\sy*(-0.0000)})
	--({\sx*(2.7000)},{\sy*(-0.0000)})
	--({\sx*(2.7100)},{\sy*(-0.0000)})
	--({\sx*(2.7200)},{\sy*(0.0000)})
	--({\sx*(2.7300)},{\sy*(-0.0000)})
	--({\sx*(2.7400)},{\sy*(0.0000)})
	--({\sx*(2.7500)},{\sy*(0.0000)})
	--({\sx*(2.7600)},{\sy*(0.0000)})
	--({\sx*(2.7700)},{\sy*(0.0000)})
	--({\sx*(2.7800)},{\sy*(0.0000)})
	--({\sx*(2.7900)},{\sy*(0.0000)})
	--({\sx*(2.8000)},{\sy*(0.0000)})
	--({\sx*(2.8100)},{\sy*(0.0000)})
	--({\sx*(2.8200)},{\sy*(-0.0000)})
	--({\sx*(2.8300)},{\sy*(0.0000)})
	--({\sx*(2.8400)},{\sy*(-0.0000)})
	--({\sx*(2.8500)},{\sy*(0.0000)})
	--({\sx*(2.8600)},{\sy*(-0.0000)})
	--({\sx*(2.8700)},{\sy*(-0.0000)})
	--({\sx*(2.8800)},{\sy*(0.0000)})
	--({\sx*(2.8900)},{\sy*(-0.0000)})
	--({\sx*(2.9000)},{\sy*(0.0000)})
	--({\sx*(2.9100)},{\sy*(0.0000)})
	--({\sx*(2.9200)},{\sy*(0.0000)})
	--({\sx*(2.9300)},{\sy*(0.0000)})
	--({\sx*(2.9400)},{\sy*(0.0000)})
	--({\sx*(2.9500)},{\sy*(0.0000)})
	--({\sx*(2.9600)},{\sy*(0.0000)})
	--({\sx*(2.9700)},{\sy*(0.0000)})
	--({\sx*(2.9800)},{\sy*(0.0000)})
	--({\sx*(2.9900)},{\sy*(0.0000)})
	--({\sx*(3.0000)},{\sy*(0.0000)})
	--({\sx*(3.0100)},{\sy*(0.0000)})
	--({\sx*(3.0200)},{\sy*(0.0000)})
	--({\sx*(3.0300)},{\sy*(0.0000)})
	--({\sx*(3.0400)},{\sy*(0.0000)})
	--({\sx*(3.0500)},{\sy*(0.0000)})
	--({\sx*(3.0600)},{\sy*(0.0000)})
	--({\sx*(3.0700)},{\sy*(0.0000)})
	--({\sx*(3.0800)},{\sy*(0.0000)})
	--({\sx*(3.0900)},{\sy*(0.0000)})
	--({\sx*(3.1000)},{\sy*(0.0000)})
	--({\sx*(3.1100)},{\sy*(-0.0000)})
	--({\sx*(3.1200)},{\sy*(-0.0000)})
	--({\sx*(3.1300)},{\sy*(-0.0000)})
	--({\sx*(3.1400)},{\sy*(-0.0000)})
	--({\sx*(3.1500)},{\sy*(-0.0000)})
	--({\sx*(3.1600)},{\sy*(-0.0000)})
	--({\sx*(3.1700)},{\sy*(-0.0000)})
	--({\sx*(3.1800)},{\sy*(-0.0000)})
	--({\sx*(3.1900)},{\sy*(-0.0000)})
	--({\sx*(3.2000)},{\sy*(-0.0000)})
	--({\sx*(3.2100)},{\sy*(-0.0000)})
	--({\sx*(3.2200)},{\sy*(-0.0000)})
	--({\sx*(3.2300)},{\sy*(-0.0000)})
	--({\sx*(3.2400)},{\sy*(-0.0000)})
	--({\sx*(3.2500)},{\sy*(-0.0000)})
	--({\sx*(3.2600)},{\sy*(-0.0000)})
	--({\sx*(3.2700)},{\sy*(-0.0000)})
	--({\sx*(3.2800)},{\sy*(0.0000)})
	--({\sx*(3.2900)},{\sy*(0.0000)})
	--({\sx*(3.3000)},{\sy*(0.0000)})
	--({\sx*(3.3100)},{\sy*(0.0000)})
	--({\sx*(3.3200)},{\sy*(0.0000)})
	--({\sx*(3.3300)},{\sy*(0.0000)})
	--({\sx*(3.3400)},{\sy*(0.0000)})
	--({\sx*(3.3500)},{\sy*(0.0000)})
	--({\sx*(3.3600)},{\sy*(0.0000)})
	--({\sx*(3.3700)},{\sy*(0.0000)})
	--({\sx*(3.3800)},{\sy*(0.0000)})
	--({\sx*(3.3900)},{\sy*(0.0000)})
	--({\sx*(3.4000)},{\sy*(0.0000)})
	--({\sx*(3.4100)},{\sy*(0.0000)})
	--({\sx*(3.4200)},{\sy*(0.0000)})
	--({\sx*(3.4300)},{\sy*(0.0000)})
	--({\sx*(3.4400)},{\sy*(0.0000)})
	--({\sx*(3.4500)},{\sy*(0.0000)})
	--({\sx*(3.4600)},{\sy*(0.0000)})
	--({\sx*(3.4700)},{\sy*(-0.0000)})
	--({\sx*(3.4800)},{\sy*(-0.0000)})
	--({\sx*(3.4900)},{\sy*(-0.0000)})
	--({\sx*(3.5000)},{\sy*(-0.0000)})
	--({\sx*(3.5100)},{\sy*(-0.0000)})
	--({\sx*(3.5200)},{\sy*(-0.0000)})
	--({\sx*(3.5300)},{\sy*(-0.0000)})
	--({\sx*(3.5400)},{\sy*(-0.0000)})
	--({\sx*(3.5500)},{\sy*(-0.0000)})
	--({\sx*(3.5600)},{\sy*(-0.0000)})
	--({\sx*(3.5700)},{\sy*(-0.0000)})
	--({\sx*(3.5800)},{\sy*(-0.0000)})
	--({\sx*(3.5900)},{\sy*(-0.0000)})
	--({\sx*(3.6000)},{\sy*(-0.0000)})
	--({\sx*(3.6100)},{\sy*(-0.0000)})
	--({\sx*(3.6200)},{\sy*(-0.0000)})
	--({\sx*(3.6300)},{\sy*(-0.0000)})
	--({\sx*(3.6400)},{\sy*(-0.0000)})
	--({\sx*(3.6500)},{\sy*(-0.0000)})
	--({\sx*(3.6600)},{\sy*(0.0000)})
	--({\sx*(3.6700)},{\sy*(0.0000)})
	--({\sx*(3.6800)},{\sy*(0.0000)})
	--({\sx*(3.6900)},{\sy*(0.0000)})
	--({\sx*(3.7000)},{\sy*(0.0000)})
	--({\sx*(3.7100)},{\sy*(0.0000)})
	--({\sx*(3.7200)},{\sy*(0.0000)})
	--({\sx*(3.7300)},{\sy*(0.0000)})
	--({\sx*(3.7400)},{\sy*(0.0000)})
	--({\sx*(3.7500)},{\sy*(0.0000)})
	--({\sx*(3.7600)},{\sy*(0.0000)})
	--({\sx*(3.7700)},{\sy*(0.0000)})
	--({\sx*(3.7800)},{\sy*(0.0000)})
	--({\sx*(3.7900)},{\sy*(0.0000)})
	--({\sx*(3.8000)},{\sy*(0.0000)})
	--({\sx*(3.8100)},{\sy*(0.0000)})
	--({\sx*(3.8200)},{\sy*(0.0000)})
	--({\sx*(3.8300)},{\sy*(0.0000)})
	--({\sx*(3.8400)},{\sy*(0.0000)})
	--({\sx*(3.8500)},{\sy*(-0.0000)})
	--({\sx*(3.8600)},{\sy*(-0.0000)})
	--({\sx*(3.8700)},{\sy*(-0.0000)})
	--({\sx*(3.8800)},{\sy*(-0.0000)})
	--({\sx*(3.8900)},{\sy*(-0.0000)})
	--({\sx*(3.9000)},{\sy*(-0.0000)})
	--({\sx*(3.9100)},{\sy*(-0.0000)})
	--({\sx*(3.9200)},{\sy*(-0.0000)})
	--({\sx*(3.9300)},{\sy*(-0.0000)})
	--({\sx*(3.9400)},{\sy*(-0.0000)})
	--({\sx*(3.9500)},{\sy*(-0.0000)})
	--({\sx*(3.9600)},{\sy*(-0.0000)})
	--({\sx*(3.9700)},{\sy*(-0.0000)})
	--({\sx*(3.9800)},{\sy*(-0.0000)})
	--({\sx*(3.9900)},{\sy*(-0.0000)})
	--({\sx*(4.0000)},{\sy*(-0.0000)})
	--({\sx*(4.0100)},{\sy*(-0.0000)})
	--({\sx*(4.0200)},{\sy*(-0.0000)})
	--({\sx*(4.0300)},{\sy*(-0.0000)})
	--({\sx*(4.0400)},{\sy*(0.0000)})
	--({\sx*(4.0500)},{\sy*(0.0000)})
	--({\sx*(4.0600)},{\sy*(0.0000)})
	--({\sx*(4.0700)},{\sy*(0.0000)})
	--({\sx*(4.0800)},{\sy*(0.0000)})
	--({\sx*(4.0900)},{\sy*(0.0000)})
	--({\sx*(4.1000)},{\sy*(0.0000)})
	--({\sx*(4.1100)},{\sy*(0.0000)})
	--({\sx*(4.1200)},{\sy*(0.0000)})
	--({\sx*(4.1300)},{\sy*(0.0000)})
	--({\sx*(4.1400)},{\sy*(0.0000)})
	--({\sx*(4.1500)},{\sy*(0.0000)})
	--({\sx*(4.1600)},{\sy*(0.0000)})
	--({\sx*(4.1700)},{\sy*(0.0001)})
	--({\sx*(4.1800)},{\sy*(0.0000)})
	--({\sx*(4.1900)},{\sy*(0.0000)})
	--({\sx*(4.2000)},{\sy*(0.0000)})
	--({\sx*(4.2100)},{\sy*(0.0000)})
	--({\sx*(4.2200)},{\sy*(0.0000)})
	--({\sx*(4.2300)},{\sy*(0.0000)})
	--({\sx*(4.2400)},{\sy*(-0.0000)})
	--({\sx*(4.2500)},{\sy*(-0.0000)})
	--({\sx*(4.2600)},{\sy*(-0.0001)})
	--({\sx*(4.2700)},{\sy*(-0.0001)})
	--({\sx*(4.2800)},{\sy*(-0.0001)})
	--({\sx*(4.2900)},{\sy*(-0.0001)})
	--({\sx*(4.3000)},{\sy*(-0.0002)})
	--({\sx*(4.3100)},{\sy*(-0.0002)})
	--({\sx*(4.3200)},{\sy*(-0.0002)})
	--({\sx*(4.3300)},{\sy*(-0.0003)})
	--({\sx*(4.3400)},{\sy*(-0.0003)})
	--({\sx*(4.3500)},{\sy*(-0.0003)})
	--({\sx*(4.3600)},{\sy*(-0.0003)})
	--({\sx*(4.3700)},{\sy*(-0.0003)})
	--({\sx*(4.3800)},{\sy*(-0.0002)})
	--({\sx*(4.3900)},{\sy*(-0.0002)})
	--({\sx*(4.4000)},{\sy*(-0.0002)})
	--({\sx*(4.4100)},{\sy*(-0.0001)})
	--({\sx*(4.4200)},{\sy*(-0.0000)})
	--({\sx*(4.4300)},{\sy*(0.0001)})
	--({\sx*(4.4400)},{\sy*(0.0002)})
	--({\sx*(4.4500)},{\sy*(0.0004)})
	--({\sx*(4.4600)},{\sy*(0.0005)})
	--({\sx*(4.4700)},{\sy*(0.0008)})
	--({\sx*(4.4800)},{\sy*(0.0009)})
	--({\sx*(4.4900)},{\sy*(0.0011)})
	--({\sx*(4.5000)},{\sy*(0.0015)})
	--({\sx*(4.5100)},{\sy*(0.0017)})
	--({\sx*(4.5200)},{\sy*(0.0018)})
	--({\sx*(4.5300)},{\sy*(0.0021)})
	--({\sx*(4.5400)},{\sy*(0.0021)})
	--({\sx*(4.5500)},{\sy*(0.0023)})
	--({\sx*(4.5600)},{\sy*(0.0023)})
	--({\sx*(4.5700)},{\sy*(0.0022)})
	--({\sx*(4.5800)},{\sy*(0.0020)})
	--({\sx*(4.5900)},{\sy*(0.0016)})
	--({\sx*(4.6000)},{\sy*(0.0011)})
	--({\sx*(4.6100)},{\sy*(0.0005)})
	--({\sx*(4.6200)},{\sy*(-0.0005)})
	--({\sx*(4.6300)},{\sy*(-0.0016)})
	--({\sx*(4.6400)},{\sy*(-0.0031)})
	--({\sx*(4.6500)},{\sy*(-0.0045)})
	--({\sx*(4.6600)},{\sy*(-0.0062)})
	--({\sx*(4.6700)},{\sy*(-0.0088)})
	--({\sx*(4.6800)},{\sy*(-0.0112)})
	--({\sx*(4.6900)},{\sy*(-0.0147)})
	--({\sx*(4.7000)},{\sy*(-0.0162)})
	--({\sx*(4.7100)},{\sy*(-0.0189)})
	--({\sx*(4.7200)},{\sy*(-0.0209)})
	--({\sx*(4.7300)},{\sy*(-0.0230)})
	--({\sx*(4.7400)},{\sy*(-0.0254)})
	--({\sx*(4.7500)},{\sy*(-0.0263)})
	--({\sx*(4.7600)},{\sy*(-0.0269)})
	--({\sx*(4.7700)},{\sy*(-0.0245)})
	--({\sx*(4.7800)},{\sy*(-0.0209)})
	--({\sx*(4.7900)},{\sy*(-0.0158)})
	--({\sx*(4.8000)},{\sy*(-0.0085)})
	--({\sx*(4.8100)},{\sy*(0.0030)})
	--({\sx*(4.8200)},{\sy*(0.0178)})
	--({\sx*(4.8300)},{\sy*(0.0376)})
	--({\sx*(4.8400)},{\sy*(0.0609)})
	--({\sx*(4.8500)},{\sy*(0.0911)})
	--({\sx*(4.8600)},{\sy*(0.1231)})
	--({\sx*(4.8700)},{\sy*(0.1708)})
	--({\sx*(4.8800)},{\sy*(0.2237)})
	--({\sx*(4.8900)},{\sy*(0.2662)})
	--({\sx*(4.9000)},{\sy*(0.3217)})
	--({\sx*(4.9100)},{\sy*(0.3876)})
	--({\sx*(4.9200)},{\sy*(0.4440)})
	--({\sx*(4.9300)},{\sy*(0.4927)})
	--({\sx*(4.9400)},{\sy*(0.5404)})
	--({\sx*(4.9500)},{\sy*(0.5512)})
	--({\sx*(4.9600)},{\sy*(0.5563)})
	--({\sx*(4.9700)},{\sy*(0.5423)})
	--({\sx*(4.9800)},{\sy*(0.4355)})
	--({\sx*(4.9900)},{\sy*(0.2746)})
	--({\sx*(5.0000)},{\sy*(0.0000)});
}
\def\relfehlerm{
\draw[color=blue,line width=1.4pt,line join=round] ({\sx*(0.000)},{\sy*(0.0000)})
	--({\sx*(0.0100)},{\sy*(0.0000)})
	--({\sx*(0.0200)},{\sy*(0.0000)})
	--({\sx*(0.0300)},{\sy*(0.0000)})
	--({\sx*(0.0400)},{\sy*(0.0000)})
	--({\sx*(0.0500)},{\sy*(0.0000)})
	--({\sx*(0.0600)},{\sy*(0.0000)})
	--({\sx*(0.0700)},{\sy*(0.0000)})
	--({\sx*(0.0800)},{\sy*(0.0000)})
	--({\sx*(0.0900)},{\sy*(0.0000)})
	--({\sx*(0.1000)},{\sy*(0.0000)})
	--({\sx*(0.1100)},{\sy*(0.0000)})
	--({\sx*(0.1200)},{\sy*(0.0000)})
	--({\sx*(0.1300)},{\sy*(0.0000)})
	--({\sx*(0.1400)},{\sy*(0.0000)})
	--({\sx*(0.1500)},{\sy*(0.0000)})
	--({\sx*(0.1600)},{\sy*(0.0000)})
	--({\sx*(0.1700)},{\sy*(0.0000)})
	--({\sx*(0.1800)},{\sy*(0.0000)})
	--({\sx*(0.1900)},{\sy*(0.0000)})
	--({\sx*(0.2000)},{\sy*(-0.0000)})
	--({\sx*(0.2100)},{\sy*(-0.0000)})
	--({\sx*(0.2200)},{\sy*(-0.0000)})
	--({\sx*(0.2300)},{\sy*(-0.0000)})
	--({\sx*(0.2400)},{\sy*(-0.0000)})
	--({\sx*(0.2500)},{\sy*(-0.0000)})
	--({\sx*(0.2600)},{\sy*(-0.0000)})
	--({\sx*(0.2700)},{\sy*(-0.0000)})
	--({\sx*(0.2800)},{\sy*(-0.0000)})
	--({\sx*(0.2900)},{\sy*(-0.0000)})
	--({\sx*(0.3000)},{\sy*(-0.0000)})
	--({\sx*(0.3100)},{\sy*(-0.0000)})
	--({\sx*(0.3200)},{\sy*(-0.0000)})
	--({\sx*(0.3300)},{\sy*(-0.0000)})
	--({\sx*(0.3400)},{\sy*(-0.0000)})
	--({\sx*(0.3500)},{\sy*(-0.0000)})
	--({\sx*(0.3600)},{\sy*(-0.0000)})
	--({\sx*(0.3700)},{\sy*(-0.0000)})
	--({\sx*(0.3800)},{\sy*(-0.0000)})
	--({\sx*(0.3900)},{\sy*(0.0000)})
	--({\sx*(0.4000)},{\sy*(0.0000)})
	--({\sx*(0.4100)},{\sy*(0.0000)})
	--({\sx*(0.4200)},{\sy*(0.0000)})
	--({\sx*(0.4300)},{\sy*(0.0000)})
	--({\sx*(0.4400)},{\sy*(0.0000)})
	--({\sx*(0.4500)},{\sy*(0.0000)})
	--({\sx*(0.4600)},{\sy*(0.0000)})
	--({\sx*(0.4700)},{\sy*(0.0000)})
	--({\sx*(0.4800)},{\sy*(0.0000)})
	--({\sx*(0.4900)},{\sy*(0.0000)})
	--({\sx*(0.5000)},{\sy*(0.0000)})
	--({\sx*(0.5100)},{\sy*(0.0000)})
	--({\sx*(0.5200)},{\sy*(0.0000)})
	--({\sx*(0.5300)},{\sy*(0.0000)})
	--({\sx*(0.5400)},{\sy*(0.0000)})
	--({\sx*(0.5500)},{\sy*(0.0000)})
	--({\sx*(0.5600)},{\sy*(0.0000)})
	--({\sx*(0.5700)},{\sy*(0.0000)})
	--({\sx*(0.5800)},{\sy*(-0.0000)})
	--({\sx*(0.5900)},{\sy*(-0.0000)})
	--({\sx*(0.6000)},{\sy*(-0.0000)})
	--({\sx*(0.6100)},{\sy*(-0.0000)})
	--({\sx*(0.6200)},{\sy*(-0.0000)})
	--({\sx*(0.6300)},{\sy*(-0.0000)})
	--({\sx*(0.6400)},{\sy*(-0.0000)})
	--({\sx*(0.6500)},{\sy*(-0.0000)})
	--({\sx*(0.6600)},{\sy*(-0.0000)})
	--({\sx*(0.6700)},{\sy*(-0.0000)})
	--({\sx*(0.6800)},{\sy*(-0.0000)})
	--({\sx*(0.6900)},{\sy*(-0.0000)})
	--({\sx*(0.7000)},{\sy*(-0.0000)})
	--({\sx*(0.7100)},{\sy*(-0.0000)})
	--({\sx*(0.7200)},{\sy*(-0.0000)})
	--({\sx*(0.7300)},{\sy*(-0.0000)})
	--({\sx*(0.7400)},{\sy*(-0.0000)})
	--({\sx*(0.7500)},{\sy*(-0.0000)})
	--({\sx*(0.7600)},{\sy*(-0.0000)})
	--({\sx*(0.7700)},{\sy*(0.0000)})
	--({\sx*(0.7800)},{\sy*(0.0000)})
	--({\sx*(0.7900)},{\sy*(0.0000)})
	--({\sx*(0.8000)},{\sy*(0.0000)})
	--({\sx*(0.8100)},{\sy*(0.0000)})
	--({\sx*(0.8200)},{\sy*(0.0000)})
	--({\sx*(0.8300)},{\sy*(0.0000)})
	--({\sx*(0.8400)},{\sy*(0.0000)})
	--({\sx*(0.8500)},{\sy*(0.0000)})
	--({\sx*(0.8600)},{\sy*(0.0000)})
	--({\sx*(0.8700)},{\sy*(0.0000)})
	--({\sx*(0.8800)},{\sy*(0.0000)})
	--({\sx*(0.8900)},{\sy*(0.0000)})
	--({\sx*(0.9000)},{\sy*(0.0000)})
	--({\sx*(0.9100)},{\sy*(0.0000)})
	--({\sx*(0.9200)},{\sy*(0.0000)})
	--({\sx*(0.9300)},{\sy*(0.0000)})
	--({\sx*(0.9400)},{\sy*(0.0000)})
	--({\sx*(0.9500)},{\sy*(0.0000)})
	--({\sx*(0.9600)},{\sy*(0.0000)})
	--({\sx*(0.9700)},{\sy*(-0.0000)})
	--({\sx*(0.9800)},{\sy*(-0.0000)})
	--({\sx*(0.9900)},{\sy*(-0.0000)})
	--({\sx*(1.0000)},{\sy*(-0.0000)})
	--({\sx*(1.0100)},{\sy*(-0.0000)})
	--({\sx*(1.0200)},{\sy*(-0.0000)})
	--({\sx*(1.0300)},{\sy*(-0.0000)})
	--({\sx*(1.0400)},{\sy*(-0.0000)})
	--({\sx*(1.0500)},{\sy*(-0.0000)})
	--({\sx*(1.0600)},{\sy*(-0.0000)})
	--({\sx*(1.0700)},{\sy*(-0.0000)})
	--({\sx*(1.0800)},{\sy*(-0.0000)})
	--({\sx*(1.0900)},{\sy*(-0.0000)})
	--({\sx*(1.1000)},{\sy*(-0.0000)})
	--({\sx*(1.1100)},{\sy*(-0.0000)})
	--({\sx*(1.1200)},{\sy*(-0.0000)})
	--({\sx*(1.1300)},{\sy*(-0.0000)})
	--({\sx*(1.1400)},{\sy*(-0.0000)})
	--({\sx*(1.1500)},{\sy*(0.0000)})
	--({\sx*(1.1600)},{\sy*(0.0000)})
	--({\sx*(1.1700)},{\sy*(0.0000)})
	--({\sx*(1.1800)},{\sy*(0.0000)})
	--({\sx*(1.1900)},{\sy*(0.0000)})
	--({\sx*(1.2000)},{\sy*(0.0000)})
	--({\sx*(1.2100)},{\sy*(0.0000)})
	--({\sx*(1.2200)},{\sy*(0.0000)})
	--({\sx*(1.2300)},{\sy*(0.0000)})
	--({\sx*(1.2400)},{\sy*(0.0000)})
	--({\sx*(1.2500)},{\sy*(0.0000)})
	--({\sx*(1.2600)},{\sy*(0.0000)})
	--({\sx*(1.2700)},{\sy*(0.0000)})
	--({\sx*(1.2800)},{\sy*(0.0000)})
	--({\sx*(1.2900)},{\sy*(0.0000)})
	--({\sx*(1.3000)},{\sy*(0.0000)})
	--({\sx*(1.3100)},{\sy*(0.0000)})
	--({\sx*(1.3200)},{\sy*(0.0000)})
	--({\sx*(1.3300)},{\sy*(0.0000)})
	--({\sx*(1.3400)},{\sy*(0.0000)})
	--({\sx*(1.3500)},{\sy*(0.0000)})
	--({\sx*(1.3600)},{\sy*(-0.0000)})
	--({\sx*(1.3700)},{\sy*(-0.0000)})
	--({\sx*(1.3800)},{\sy*(-0.0000)})
	--({\sx*(1.3900)},{\sy*(-0.0000)})
	--({\sx*(1.4000)},{\sy*(-0.0000)})
	--({\sx*(1.4100)},{\sy*(-0.0000)})
	--({\sx*(1.4200)},{\sy*(-0.0000)})
	--({\sx*(1.4300)},{\sy*(-0.0000)})
	--({\sx*(1.4400)},{\sy*(-0.0000)})
	--({\sx*(1.4500)},{\sy*(-0.0000)})
	--({\sx*(1.4600)},{\sy*(-0.0000)})
	--({\sx*(1.4700)},{\sy*(-0.0000)})
	--({\sx*(1.4800)},{\sy*(-0.0000)})
	--({\sx*(1.4900)},{\sy*(-0.0000)})
	--({\sx*(1.5000)},{\sy*(-0.0000)})
	--({\sx*(1.5100)},{\sy*(0.0000)})
	--({\sx*(1.5200)},{\sy*(-0.0000)})
	--({\sx*(1.5300)},{\sy*(-0.0000)})
	--({\sx*(1.5400)},{\sy*(-0.0000)})
	--({\sx*(1.5500)},{\sy*(0.0000)})
	--({\sx*(1.5600)},{\sy*(-0.0000)})
	--({\sx*(1.5700)},{\sy*(-0.0000)})
	--({\sx*(1.5800)},{\sy*(0.0000)})
	--({\sx*(1.5900)},{\sy*(0.0000)})
	--({\sx*(1.6000)},{\sy*(0.0000)})
	--({\sx*(1.6100)},{\sy*(-0.0000)})
	--({\sx*(1.6200)},{\sy*(0.0000)})
	--({\sx*(1.6300)},{\sy*(0.0000)})
	--({\sx*(1.6400)},{\sy*(-0.0000)})
	--({\sx*(1.6500)},{\sy*(-0.0000)})
	--({\sx*(1.6600)},{\sy*(-0.0000)})
	--({\sx*(1.6700)},{\sy*(0.0000)})
	--({\sx*(1.6800)},{\sy*(0.0000)})
	--({\sx*(1.6900)},{\sy*(0.0000)})
	--({\sx*(1.7000)},{\sy*(-0.0000)})
	--({\sx*(1.7100)},{\sy*(0.0000)})
	--({\sx*(1.7200)},{\sy*(0.0000)})
	--({\sx*(1.7300)},{\sy*(-0.0000)})
	--({\sx*(1.7400)},{\sy*(0.0000)})
	--({\sx*(1.7500)},{\sy*(-0.0000)})
	--({\sx*(1.7600)},{\sy*(-0.0000)})
	--({\sx*(1.7700)},{\sy*(0.0000)})
	--({\sx*(1.7800)},{\sy*(-0.0000)})
	--({\sx*(1.7900)},{\sy*(0.0000)})
	--({\sx*(1.8000)},{\sy*(0.0000)})
	--({\sx*(1.8100)},{\sy*(0.0000)})
	--({\sx*(1.8200)},{\sy*(-0.0000)})
	--({\sx*(1.8300)},{\sy*(0.0000)})
	--({\sx*(1.8400)},{\sy*(0.0000)})
	--({\sx*(1.8500)},{\sy*(-0.0000)})
	--({\sx*(1.8600)},{\sy*(-0.0000)})
	--({\sx*(1.8700)},{\sy*(0.0000)})
	--({\sx*(1.8800)},{\sy*(0.0000)})
	--({\sx*(1.8900)},{\sy*(0.0000)})
	--({\sx*(1.9000)},{\sy*(0.0000)})
	--({\sx*(1.9100)},{\sy*(0.0000)})
	--({\sx*(1.9200)},{\sy*(0.0000)})
	--({\sx*(1.9300)},{\sy*(0.0000)})
	--({\sx*(1.9400)},{\sy*(0.0000)})
	--({\sx*(1.9500)},{\sy*(-0.0000)})
	--({\sx*(1.9600)},{\sy*(0.0000)})
	--({\sx*(1.9700)},{\sy*(0.0000)})
	--({\sx*(1.9800)},{\sy*(0.0000)})
	--({\sx*(1.9900)},{\sy*(0.0000)})
	--({\sx*(2.0000)},{\sy*(0.0000)})
	--({\sx*(2.0100)},{\sy*(0.0000)})
	--({\sx*(2.0200)},{\sy*(0.0000)})
	--({\sx*(2.0300)},{\sy*(0.0000)})
	--({\sx*(2.0400)},{\sy*(0.0000)})
	--({\sx*(2.0500)},{\sy*(0.0000)})
	--({\sx*(2.0600)},{\sy*(0.0000)})
	--({\sx*(2.0700)},{\sy*(-0.0000)})
	--({\sx*(2.0800)},{\sy*(0.0000)})
	--({\sx*(2.0900)},{\sy*(0.0000)})
	--({\sx*(2.1000)},{\sy*(0.0000)})
	--({\sx*(2.1100)},{\sy*(0.0000)})
	--({\sx*(2.1200)},{\sy*(0.0000)})
	--({\sx*(2.1300)},{\sy*(-0.0000)})
	--({\sx*(2.1400)},{\sy*(0.0000)})
	--({\sx*(2.1500)},{\sy*(0.0000)})
	--({\sx*(2.1600)},{\sy*(-0.0000)})
	--({\sx*(2.1700)},{\sy*(-0.0000)})
	--({\sx*(2.1800)},{\sy*(0.0000)})
	--({\sx*(2.1900)},{\sy*(0.0000)})
	--({\sx*(2.2000)},{\sy*(-0.0000)})
	--({\sx*(2.2100)},{\sy*(-0.0000)})
	--({\sx*(2.2200)},{\sy*(-0.0000)})
	--({\sx*(2.2300)},{\sy*(-0.0000)})
	--({\sx*(2.2400)},{\sy*(-0.0000)})
	--({\sx*(2.2500)},{\sy*(-0.0000)})
	--({\sx*(2.2600)},{\sy*(0.0000)})
	--({\sx*(2.2700)},{\sy*(-0.0000)})
	--({\sx*(2.2800)},{\sy*(-0.0000)})
	--({\sx*(2.2900)},{\sy*(-0.0000)})
	--({\sx*(2.3000)},{\sy*(-0.0000)})
	--({\sx*(2.3100)},{\sy*(-0.0000)})
	--({\sx*(2.3200)},{\sy*(-0.0000)})
	--({\sx*(2.3300)},{\sy*(-0.0000)})
	--({\sx*(2.3400)},{\sy*(-0.0000)})
	--({\sx*(2.3500)},{\sy*(0.0000)})
	--({\sx*(2.3600)},{\sy*(-0.0000)})
	--({\sx*(2.3700)},{\sy*(-0.0000)})
	--({\sx*(2.3800)},{\sy*(0.0000)})
	--({\sx*(2.3900)},{\sy*(-0.0000)})
	--({\sx*(2.4000)},{\sy*(0.0000)})
	--({\sx*(2.4100)},{\sy*(-0.0000)})
	--({\sx*(2.4200)},{\sy*(-0.0000)})
	--({\sx*(2.4300)},{\sy*(-0.0000)})
	--({\sx*(2.4400)},{\sy*(0.0000)})
	--({\sx*(2.4500)},{\sy*(-0.0000)})
	--({\sx*(2.4600)},{\sy*(0.0000)})
	--({\sx*(2.4700)},{\sy*(0.0000)})
	--({\sx*(2.4800)},{\sy*(0.0000)})
	--({\sx*(2.4900)},{\sy*(0.0000)})
	--({\sx*(2.5000)},{\sy*(0.0000)})
	--({\sx*(2.5100)},{\sy*(0.0000)})
	--({\sx*(2.5200)},{\sy*(0.0000)})
	--({\sx*(2.5300)},{\sy*(-0.0000)})
	--({\sx*(2.5400)},{\sy*(0.0000)})
	--({\sx*(2.5500)},{\sy*(0.0000)})
	--({\sx*(2.5600)},{\sy*(0.0000)})
	--({\sx*(2.5700)},{\sy*(0.0000)})
	--({\sx*(2.5800)},{\sy*(-0.0000)})
	--({\sx*(2.5900)},{\sy*(-0.0000)})
	--({\sx*(2.6000)},{\sy*(0.0000)})
	--({\sx*(2.6100)},{\sy*(0.0000)})
	--({\sx*(2.6200)},{\sy*(0.0000)})
	--({\sx*(2.6300)},{\sy*(0.0000)})
	--({\sx*(2.6400)},{\sy*(0.0000)})
	--({\sx*(2.6500)},{\sy*(0.0000)})
	--({\sx*(2.6600)},{\sy*(0.0000)})
	--({\sx*(2.6700)},{\sy*(0.0000)})
	--({\sx*(2.6800)},{\sy*(0.0000)})
	--({\sx*(2.6900)},{\sy*(-0.0000)})
	--({\sx*(2.7000)},{\sy*(-0.0000)})
	--({\sx*(2.7100)},{\sy*(-0.0000)})
	--({\sx*(2.7200)},{\sy*(0.0000)})
	--({\sx*(2.7300)},{\sy*(-0.0000)})
	--({\sx*(2.7400)},{\sy*(0.0000)})
	--({\sx*(2.7500)},{\sy*(0.0000)})
	--({\sx*(2.7600)},{\sy*(0.0000)})
	--({\sx*(2.7700)},{\sy*(0.0000)})
	--({\sx*(2.7800)},{\sy*(0.0000)})
	--({\sx*(2.7900)},{\sy*(0.0000)})
	--({\sx*(2.8000)},{\sy*(0.0000)})
	--({\sx*(2.8100)},{\sy*(0.0000)})
	--({\sx*(2.8200)},{\sy*(-0.0000)})
	--({\sx*(2.8300)},{\sy*(0.0000)})
	--({\sx*(2.8400)},{\sy*(-0.0000)})
	--({\sx*(2.8500)},{\sy*(0.0000)})
	--({\sx*(2.8600)},{\sy*(-0.0000)})
	--({\sx*(2.8700)},{\sy*(-0.0000)})
	--({\sx*(2.8800)},{\sy*(0.0000)})
	--({\sx*(2.8900)},{\sy*(-0.0000)})
	--({\sx*(2.9000)},{\sy*(0.0000)})
	--({\sx*(2.9100)},{\sy*(0.0000)})
	--({\sx*(2.9200)},{\sy*(0.0000)})
	--({\sx*(2.9300)},{\sy*(0.0000)})
	--({\sx*(2.9400)},{\sy*(0.0000)})
	--({\sx*(2.9500)},{\sy*(0.0000)})
	--({\sx*(2.9600)},{\sy*(0.0000)})
	--({\sx*(2.9700)},{\sy*(0.0000)})
	--({\sx*(2.9800)},{\sy*(0.0000)})
	--({\sx*(2.9900)},{\sy*(0.0000)})
	--({\sx*(3.0000)},{\sy*(0.0000)})
	--({\sx*(3.0100)},{\sy*(0.0000)})
	--({\sx*(3.0200)},{\sy*(0.0000)})
	--({\sx*(3.0300)},{\sy*(0.0000)})
	--({\sx*(3.0400)},{\sy*(0.0000)})
	--({\sx*(3.0500)},{\sy*(0.0000)})
	--({\sx*(3.0600)},{\sy*(0.0000)})
	--({\sx*(3.0700)},{\sy*(0.0000)})
	--({\sx*(3.0800)},{\sy*(0.0000)})
	--({\sx*(3.0900)},{\sy*(0.0000)})
	--({\sx*(3.1000)},{\sy*(0.0000)})
	--({\sx*(3.1100)},{\sy*(-0.0000)})
	--({\sx*(3.1200)},{\sy*(-0.0000)})
	--({\sx*(3.1300)},{\sy*(-0.0000)})
	--({\sx*(3.1400)},{\sy*(-0.0000)})
	--({\sx*(3.1500)},{\sy*(-0.0000)})
	--({\sx*(3.1600)},{\sy*(-0.0000)})
	--({\sx*(3.1700)},{\sy*(-0.0000)})
	--({\sx*(3.1800)},{\sy*(-0.0000)})
	--({\sx*(3.1900)},{\sy*(-0.0000)})
	--({\sx*(3.2000)},{\sy*(-0.0000)})
	--({\sx*(3.2100)},{\sy*(-0.0000)})
	--({\sx*(3.2200)},{\sy*(-0.0000)})
	--({\sx*(3.2300)},{\sy*(-0.0000)})
	--({\sx*(3.2400)},{\sy*(-0.0000)})
	--({\sx*(3.2500)},{\sy*(-0.0000)})
	--({\sx*(3.2600)},{\sy*(-0.0000)})
	--({\sx*(3.2700)},{\sy*(-0.0000)})
	--({\sx*(3.2800)},{\sy*(0.0000)})
	--({\sx*(3.2900)},{\sy*(0.0000)})
	--({\sx*(3.3000)},{\sy*(0.0000)})
	--({\sx*(3.3100)},{\sy*(0.0000)})
	--({\sx*(3.3200)},{\sy*(0.0000)})
	--({\sx*(3.3300)},{\sy*(0.0000)})
	--({\sx*(3.3400)},{\sy*(0.0000)})
	--({\sx*(3.3500)},{\sy*(0.0000)})
	--({\sx*(3.3600)},{\sy*(0.0000)})
	--({\sx*(3.3700)},{\sy*(0.0000)})
	--({\sx*(3.3800)},{\sy*(0.0000)})
	--({\sx*(3.3900)},{\sy*(0.0000)})
	--({\sx*(3.4000)},{\sy*(0.0000)})
	--({\sx*(3.4100)},{\sy*(0.0000)})
	--({\sx*(3.4200)},{\sy*(0.0000)})
	--({\sx*(3.4300)},{\sy*(0.0000)})
	--({\sx*(3.4400)},{\sy*(0.0000)})
	--({\sx*(3.4500)},{\sy*(0.0000)})
	--({\sx*(3.4600)},{\sy*(0.0000)})
	--({\sx*(3.4700)},{\sy*(-0.0000)})
	--({\sx*(3.4800)},{\sy*(-0.0000)})
	--({\sx*(3.4900)},{\sy*(-0.0000)})
	--({\sx*(3.5000)},{\sy*(-0.0000)})
	--({\sx*(3.5100)},{\sy*(-0.0000)})
	--({\sx*(3.5200)},{\sy*(-0.0000)})
	--({\sx*(3.5300)},{\sy*(-0.0000)})
	--({\sx*(3.5400)},{\sy*(-0.0000)})
	--({\sx*(3.5500)},{\sy*(-0.0000)})
	--({\sx*(3.5600)},{\sy*(-0.0000)})
	--({\sx*(3.5700)},{\sy*(-0.0000)})
	--({\sx*(3.5800)},{\sy*(-0.0000)})
	--({\sx*(3.5900)},{\sy*(-0.0000)})
	--({\sx*(3.6000)},{\sy*(-0.0000)})
	--({\sx*(3.6100)},{\sy*(-0.0000)})
	--({\sx*(3.6200)},{\sy*(-0.0000)})
	--({\sx*(3.6300)},{\sy*(-0.0000)})
	--({\sx*(3.6400)},{\sy*(-0.0000)})
	--({\sx*(3.6500)},{\sy*(-0.0000)})
	--({\sx*(3.6600)},{\sy*(0.0000)})
	--({\sx*(3.6700)},{\sy*(0.0000)})
	--({\sx*(3.6800)},{\sy*(0.0000)})
	--({\sx*(3.6900)},{\sy*(0.0000)})
	--({\sx*(3.7000)},{\sy*(0.0000)})
	--({\sx*(3.7100)},{\sy*(0.0000)})
	--({\sx*(3.7200)},{\sy*(0.0000)})
	--({\sx*(3.7300)},{\sy*(0.0000)})
	--({\sx*(3.7400)},{\sy*(0.0000)})
	--({\sx*(3.7500)},{\sy*(0.0000)})
	--({\sx*(3.7600)},{\sy*(0.0000)})
	--({\sx*(3.7700)},{\sy*(0.0000)})
	--({\sx*(3.7800)},{\sy*(0.0000)})
	--({\sx*(3.7900)},{\sy*(0.0000)})
	--({\sx*(3.8000)},{\sy*(0.0000)})
	--({\sx*(3.8100)},{\sy*(0.0000)})
	--({\sx*(3.8200)},{\sy*(0.0000)})
	--({\sx*(3.8300)},{\sy*(0.0000)})
	--({\sx*(3.8400)},{\sy*(0.0000)})
	--({\sx*(3.8500)},{\sy*(-0.0000)})
	--({\sx*(3.8600)},{\sy*(-0.0000)})
	--({\sx*(3.8700)},{\sy*(-0.0000)})
	--({\sx*(3.8800)},{\sy*(-0.0000)})
	--({\sx*(3.8900)},{\sy*(-0.0000)})
	--({\sx*(3.9000)},{\sy*(-0.0000)})
	--({\sx*(3.9100)},{\sy*(-0.0000)})
	--({\sx*(3.9200)},{\sy*(-0.0000)})
	--({\sx*(3.9300)},{\sy*(-0.0000)})
	--({\sx*(3.9400)},{\sy*(-0.0000)})
	--({\sx*(3.9500)},{\sy*(-0.0000)})
	--({\sx*(3.9600)},{\sy*(-0.0000)})
	--({\sx*(3.9700)},{\sy*(-0.0000)})
	--({\sx*(3.9800)},{\sy*(-0.0000)})
	--({\sx*(3.9900)},{\sy*(-0.0000)})
	--({\sx*(4.0000)},{\sy*(-0.0000)})
	--({\sx*(4.0100)},{\sy*(-0.0000)})
	--({\sx*(4.0200)},{\sy*(-0.0000)})
	--({\sx*(4.0300)},{\sy*(-0.0000)})
	--({\sx*(4.0400)},{\sy*(0.0000)})
	--({\sx*(4.0500)},{\sy*(0.0000)})
	--({\sx*(4.0600)},{\sy*(0.0000)})
	--({\sx*(4.0700)},{\sy*(0.0000)})
	--({\sx*(4.0800)},{\sy*(0.0000)})
	--({\sx*(4.0900)},{\sy*(0.0000)})
	--({\sx*(4.1000)},{\sy*(0.0000)})
	--({\sx*(4.1100)},{\sy*(0.0000)})
	--({\sx*(4.1200)},{\sy*(0.0000)})
	--({\sx*(4.1300)},{\sy*(0.0000)})
	--({\sx*(4.1400)},{\sy*(0.0000)})
	--({\sx*(4.1500)},{\sy*(0.0000)})
	--({\sx*(4.1600)},{\sy*(0.0000)})
	--({\sx*(4.1700)},{\sy*(0.0000)})
	--({\sx*(4.1800)},{\sy*(0.0000)})
	--({\sx*(4.1900)},{\sy*(0.0000)})
	--({\sx*(4.2000)},{\sy*(0.0000)})
	--({\sx*(4.2100)},{\sy*(0.0000)})
	--({\sx*(4.2200)},{\sy*(0.0000)})
	--({\sx*(4.2300)},{\sy*(0.0000)})
	--({\sx*(4.2400)},{\sy*(-0.0000)})
	--({\sx*(4.2500)},{\sy*(-0.0000)})
	--({\sx*(4.2600)},{\sy*(-0.0000)})
	--({\sx*(4.2700)},{\sy*(-0.0000)})
	--({\sx*(4.2800)},{\sy*(-0.0000)})
	--({\sx*(4.2900)},{\sy*(-0.0000)})
	--({\sx*(4.3000)},{\sy*(-0.0000)})
	--({\sx*(4.3100)},{\sy*(-0.0000)})
	--({\sx*(4.3200)},{\sy*(-0.0000)})
	--({\sx*(4.3300)},{\sy*(-0.0000)})
	--({\sx*(4.3400)},{\sy*(-0.0000)})
	--({\sx*(4.3500)},{\sy*(-0.0000)})
	--({\sx*(4.3600)},{\sy*(-0.0000)})
	--({\sx*(4.3700)},{\sy*(-0.0000)})
	--({\sx*(4.3800)},{\sy*(-0.0000)})
	--({\sx*(4.3900)},{\sy*(-0.0000)})
	--({\sx*(4.4000)},{\sy*(-0.0000)})
	--({\sx*(4.4100)},{\sy*(-0.0000)})
	--({\sx*(4.4200)},{\sy*(-0.0000)})
	--({\sx*(4.4300)},{\sy*(0.0000)})
	--({\sx*(4.4400)},{\sy*(0.0000)})
	--({\sx*(4.4500)},{\sy*(0.0000)})
	--({\sx*(4.4600)},{\sy*(0.0000)})
	--({\sx*(4.4700)},{\sy*(0.0000)})
	--({\sx*(4.4800)},{\sy*(0.0000)})
	--({\sx*(4.4900)},{\sy*(0.0000)})
	--({\sx*(4.5000)},{\sy*(0.0000)})
	--({\sx*(4.5100)},{\sy*(0.0000)})
	--({\sx*(4.5200)},{\sy*(0.0000)})
	--({\sx*(4.5300)},{\sy*(0.0000)})
	--({\sx*(4.5400)},{\sy*(0.0000)})
	--({\sx*(4.5500)},{\sy*(0.0000)})
	--({\sx*(4.5600)},{\sy*(0.0000)})
	--({\sx*(4.5700)},{\sy*(0.0000)})
	--({\sx*(4.5800)},{\sy*(0.0000)})
	--({\sx*(4.5900)},{\sy*(0.0000)})
	--({\sx*(4.6000)},{\sy*(0.0000)})
	--({\sx*(4.6100)},{\sy*(0.0000)})
	--({\sx*(4.6200)},{\sy*(-0.0000)})
	--({\sx*(4.6300)},{\sy*(-0.0000)})
	--({\sx*(4.6400)},{\sy*(-0.0000)})
	--({\sx*(4.6500)},{\sy*(-0.0000)})
	--({\sx*(4.6600)},{\sy*(-0.0000)})
	--({\sx*(4.6700)},{\sy*(-0.0000)})
	--({\sx*(4.6800)},{\sy*(-0.0000)})
	--({\sx*(4.6900)},{\sy*(-0.0000)})
	--({\sx*(4.7000)},{\sy*(-0.0000)})
	--({\sx*(4.7100)},{\sy*(-0.0000)})
	--({\sx*(4.7200)},{\sy*(-0.0000)})
	--({\sx*(4.7300)},{\sy*(-0.0000)})
	--({\sx*(4.7400)},{\sy*(-0.0000)})
	--({\sx*(4.7500)},{\sy*(-0.0000)})
	--({\sx*(4.7600)},{\sy*(-0.0000)})
	--({\sx*(4.7700)},{\sy*(-0.0000)})
	--({\sx*(4.7800)},{\sy*(-0.0000)})
	--({\sx*(4.7900)},{\sy*(-0.0000)})
	--({\sx*(4.8000)},{\sy*(-0.0000)})
	--({\sx*(4.8100)},{\sy*(0.0000)})
	--({\sx*(4.8200)},{\sy*(0.0000)})
	--({\sx*(4.8300)},{\sy*(0.0000)})
	--({\sx*(4.8400)},{\sy*(0.0000)})
	--({\sx*(4.8500)},{\sy*(0.0000)})
	--({\sx*(4.8600)},{\sy*(0.0000)})
	--({\sx*(4.8700)},{\sy*(0.0000)})
	--({\sx*(4.8800)},{\sy*(0.0000)})
	--({\sx*(4.8900)},{\sy*(0.0000)})
	--({\sx*(4.9000)},{\sy*(0.0000)})
	--({\sx*(4.9100)},{\sy*(0.0000)})
	--({\sx*(4.9200)},{\sy*(0.0000)})
	--({\sx*(4.9300)},{\sy*(0.0000)})
	--({\sx*(4.9400)},{\sy*(0.0000)})
	--({\sx*(4.9500)},{\sy*(0.0000)})
	--({\sx*(4.9600)},{\sy*(0.0000)})
	--({\sx*(4.9700)},{\sy*(0.0000)})
	--({\sx*(4.9800)},{\sy*(0.0000)})
	--({\sx*(4.9900)},{\sy*(0.0000)})
	--({\sx*(5.0000)},{\sy*(0.0000)});
}
\def\xwerten{
\fill[color=red] (0.0000,0) circle[radius={0.07/\skala}];
\fill[color=red] (0.1786,0) circle[radius={0.07/\skala}];
\fill[color=red] (0.3571,0) circle[radius={0.07/\skala}];
\fill[color=red] (0.5357,0) circle[radius={0.07/\skala}];
\fill[color=red] (0.7143,0) circle[radius={0.07/\skala}];
\fill[color=red] (0.8929,0) circle[radius={0.07/\skala}];
\fill[color=red] (1.0714,0) circle[radius={0.07/\skala}];
\fill[color=red] (1.2500,0) circle[radius={0.07/\skala}];
\fill[color=red] (1.4286,0) circle[radius={0.07/\skala}];
\fill[color=red] (1.6071,0) circle[radius={0.07/\skala}];
\fill[color=red] (1.7857,0) circle[radius={0.07/\skala}];
\fill[color=red] (1.9643,0) circle[radius={0.07/\skala}];
\fill[color=red] (2.1429,0) circle[radius={0.07/\skala}];
\fill[color=red] (2.3214,0) circle[radius={0.07/\skala}];
\fill[color=red] (2.5000,0) circle[radius={0.07/\skala}];
\fill[color=red] (2.6786,0) circle[radius={0.07/\skala}];
\fill[color=red] (2.8571,0) circle[radius={0.07/\skala}];
\fill[color=red] (3.0357,0) circle[radius={0.07/\skala}];
\fill[color=red] (3.2143,0) circle[radius={0.07/\skala}];
\fill[color=red] (3.3929,0) circle[radius={0.07/\skala}];
\fill[color=red] (3.5714,0) circle[radius={0.07/\skala}];
\fill[color=red] (3.7500,0) circle[radius={0.07/\skala}];
\fill[color=red] (3.9286,0) circle[radius={0.07/\skala}];
\fill[color=red] (4.1071,0) circle[radius={0.07/\skala}];
\fill[color=red] (4.2857,0) circle[radius={0.07/\skala}];
\fill[color=red] (4.4643,0) circle[radius={0.07/\skala}];
\fill[color=red] (4.6429,0) circle[radius={0.07/\skala}];
\fill[color=red] (4.8214,0) circle[radius={0.07/\skala}];
\fill[color=red] (5.0000,0) circle[radius={0.07/\skala}];
}
\def\punkten{28}
\def\maxfehlern{2.140\cdot 10^{-11}}
\def\fehlern{
\draw[color=red,line width=1.4pt,line join=round] ({\sx*(0.000)},{\sy*(0.0000)})
	--({\sx*(0.0100)},{\sy*(-0.0093)})
	--({\sx*(0.0200)},{\sy*(-0.5798)})
	--({\sx*(0.0300)},{\sy*(0.1435)})
	--({\sx*(0.0400)},{\sy*(0.2096)})
	--({\sx*(0.0500)},{\sy*(-0.0054)})
	--({\sx*(0.0600)},{\sy*(-0.1091)})
	--({\sx*(0.0700)},{\sy*(0.3946)})
	--({\sx*(0.0800)},{\sy*(0.0255)})
	--({\sx*(0.0900)},{\sy*(-0.2630)})
	--({\sx*(0.1000)},{\sy*(-0.0127)})
	--({\sx*(0.1100)},{\sy*(0.0100)})
	--({\sx*(0.1200)},{\sy*(-0.1647)})
	--({\sx*(0.1300)},{\sy*(-0.0950)})
	--({\sx*(0.1400)},{\sy*(-0.0232)})
	--({\sx*(0.1500)},{\sy*(0.0587)})
	--({\sx*(0.1600)},{\sy*(-0.0238)})
	--({\sx*(0.1700)},{\sy*(-0.0078)})
	--({\sx*(0.1800)},{\sy*(0.0007)})
	--({\sx*(0.1900)},{\sy*(-0.0084)})
	--({\sx*(0.2000)},{\sy*(-0.0080)})
	--({\sx*(0.2100)},{\sy*(-0.0090)})
	--({\sx*(0.2200)},{\sy*(0.0305)})
	--({\sx*(0.2300)},{\sy*(0.0174)})
	--({\sx*(0.2400)},{\sy*(-0.0147)})
	--({\sx*(0.2500)},{\sy*(-0.0084)})
	--({\sx*(0.2600)},{\sy*(-0.0135)})
	--({\sx*(0.2700)},{\sy*(-0.0136)})
	--({\sx*(0.2800)},{\sy*(0.0099)})
	--({\sx*(0.2900)},{\sy*(-0.0159)})
	--({\sx*(0.3000)},{\sy*(-0.0115)})
	--({\sx*(0.3100)},{\sy*(-0.0034)})
	--({\sx*(0.3200)},{\sy*(0.0003)})
	--({\sx*(0.3300)},{\sy*(-0.0031)})
	--({\sx*(0.3400)},{\sy*(-0.0007)})
	--({\sx*(0.3500)},{\sy*(-0.0006)})
	--({\sx*(0.3600)},{\sy*(0.0001)})
	--({\sx*(0.3700)},{\sy*(0.0009)})
	--({\sx*(0.3800)},{\sy*(0.0012)})
	--({\sx*(0.3900)},{\sy*(0.0009)})
	--({\sx*(0.4000)},{\sy*(0.0006)})
	--({\sx*(0.4100)},{\sy*(0.0033)})
	--({\sx*(0.4200)},{\sy*(-0.0016)})
	--({\sx*(0.4300)},{\sy*(0.0010)})
	--({\sx*(0.4400)},{\sy*(0.0036)})
	--({\sx*(0.4500)},{\sy*(0.0008)})
	--({\sx*(0.4600)},{\sy*(-0.0003)})
	--({\sx*(0.4700)},{\sy*(-0.0002)})
	--({\sx*(0.4800)},{\sy*(-0.0011)})
	--({\sx*(0.4900)},{\sy*(0.0002)})
	--({\sx*(0.5000)},{\sy*(0.0008)})
	--({\sx*(0.5100)},{\sy*(-0.0003)})
	--({\sx*(0.5200)},{\sy*(0.0000)})
	--({\sx*(0.5300)},{\sy*(-0.0000)})
	--({\sx*(0.5400)},{\sy*(-0.0000)})
	--({\sx*(0.5500)},{\sy*(-0.0001)})
	--({\sx*(0.5600)},{\sy*(0.0000)})
	--({\sx*(0.5700)},{\sy*(0.0001)})
	--({\sx*(0.5800)},{\sy*(-0.0002)})
	--({\sx*(0.5900)},{\sy*(0.0000)})
	--({\sx*(0.6000)},{\sy*(0.0002)})
	--({\sx*(0.6100)},{\sy*(0.0002)})
	--({\sx*(0.6200)},{\sy*(-0.0001)})
	--({\sx*(0.6300)},{\sy*(0.0001)})
	--({\sx*(0.6400)},{\sy*(0.0000)})
	--({\sx*(0.6500)},{\sy*(0.0000)})
	--({\sx*(0.6600)},{\sy*(0.0000)})
	--({\sx*(0.6700)},{\sy*(0.0000)})
	--({\sx*(0.6800)},{\sy*(-0.0000)})
	--({\sx*(0.6900)},{\sy*(0.0001)})
	--({\sx*(0.7000)},{\sy*(-0.0000)})
	--({\sx*(0.7100)},{\sy*(0.0000)})
	--({\sx*(0.7200)},{\sy*(0.0000)})
	--({\sx*(0.7300)},{\sy*(-0.0000)})
	--({\sx*(0.7400)},{\sy*(-0.0000)})
	--({\sx*(0.7500)},{\sy*(0.0001)})
	--({\sx*(0.7600)},{\sy*(-0.0000)})
	--({\sx*(0.7700)},{\sy*(-0.0000)})
	--({\sx*(0.7800)},{\sy*(-0.0000)})
	--({\sx*(0.7900)},{\sy*(0.0000)})
	--({\sx*(0.8000)},{\sy*(0.0001)})
	--({\sx*(0.8100)},{\sy*(-0.0000)})
	--({\sx*(0.8200)},{\sy*(-0.0000)})
	--({\sx*(0.8300)},{\sy*(-0.0000)})
	--({\sx*(0.8400)},{\sy*(-0.0000)})
	--({\sx*(0.8500)},{\sy*(-0.0000)})
	--({\sx*(0.8600)},{\sy*(-0.0000)})
	--({\sx*(0.8700)},{\sy*(-0.0000)})
	--({\sx*(0.8800)},{\sy*(0.0000)})
	--({\sx*(0.8900)},{\sy*(-0.0000)})
	--({\sx*(0.9000)},{\sy*(0.0000)})
	--({\sx*(0.9100)},{\sy*(-0.0000)})
	--({\sx*(0.9200)},{\sy*(-0.0000)})
	--({\sx*(0.9300)},{\sy*(-0.0000)})
	--({\sx*(0.9400)},{\sy*(0.0000)})
	--({\sx*(0.9500)},{\sy*(-0.0000)})
	--({\sx*(0.9600)},{\sy*(-0.0000)})
	--({\sx*(0.9700)},{\sy*(-0.0000)})
	--({\sx*(0.9800)},{\sy*(-0.0000)})
	--({\sx*(0.9900)},{\sy*(-0.0000)})
	--({\sx*(1.0000)},{\sy*(-0.0000)})
	--({\sx*(1.0100)},{\sy*(0.0000)})
	--({\sx*(1.0200)},{\sy*(-0.0000)})
	--({\sx*(1.0300)},{\sy*(-0.0000)})
	--({\sx*(1.0400)},{\sy*(0.0000)})
	--({\sx*(1.0500)},{\sy*(0.0000)})
	--({\sx*(1.0600)},{\sy*(-0.0000)})
	--({\sx*(1.0700)},{\sy*(-0.0000)})
	--({\sx*(1.0800)},{\sy*(-0.0000)})
	--({\sx*(1.0900)},{\sy*(0.0000)})
	--({\sx*(1.1000)},{\sy*(-0.0000)})
	--({\sx*(1.1100)},{\sy*(-0.0000)})
	--({\sx*(1.1200)},{\sy*(0.0000)})
	--({\sx*(1.1300)},{\sy*(-0.0000)})
	--({\sx*(1.1400)},{\sy*(-0.0000)})
	--({\sx*(1.1500)},{\sy*(-0.0000)})
	--({\sx*(1.1600)},{\sy*(-0.0000)})
	--({\sx*(1.1700)},{\sy*(0.0000)})
	--({\sx*(1.1800)},{\sy*(0.0000)})
	--({\sx*(1.1900)},{\sy*(-0.0000)})
	--({\sx*(1.2000)},{\sy*(-0.0000)})
	--({\sx*(1.2100)},{\sy*(0.0000)})
	--({\sx*(1.2200)},{\sy*(-0.0000)})
	--({\sx*(1.2300)},{\sy*(0.0000)})
	--({\sx*(1.2400)},{\sy*(-0.0000)})
	--({\sx*(1.2500)},{\sy*(0.0000)})
	--({\sx*(1.2600)},{\sy*(0.0000)})
	--({\sx*(1.2700)},{\sy*(-0.0000)})
	--({\sx*(1.2800)},{\sy*(-0.0000)})
	--({\sx*(1.2900)},{\sy*(0.0000)})
	--({\sx*(1.3000)},{\sy*(0.0000)})
	--({\sx*(1.3100)},{\sy*(0.0000)})
	--({\sx*(1.3200)},{\sy*(-0.0000)})
	--({\sx*(1.3300)},{\sy*(-0.0000)})
	--({\sx*(1.3400)},{\sy*(-0.0000)})
	--({\sx*(1.3500)},{\sy*(0.0000)})
	--({\sx*(1.3600)},{\sy*(-0.0000)})
	--({\sx*(1.3700)},{\sy*(0.0000)})
	--({\sx*(1.3800)},{\sy*(0.0000)})
	--({\sx*(1.3900)},{\sy*(0.0000)})
	--({\sx*(1.4000)},{\sy*(0.0000)})
	--({\sx*(1.4100)},{\sy*(0.0000)})
	--({\sx*(1.4200)},{\sy*(0.0000)})
	--({\sx*(1.4300)},{\sy*(0.0000)})
	--({\sx*(1.4400)},{\sy*(0.0000)})
	--({\sx*(1.4500)},{\sy*(0.0000)})
	--({\sx*(1.4600)},{\sy*(0.0000)})
	--({\sx*(1.4700)},{\sy*(0.0000)})
	--({\sx*(1.4800)},{\sy*(0.0000)})
	--({\sx*(1.4900)},{\sy*(-0.0000)})
	--({\sx*(1.5000)},{\sy*(0.0000)})
	--({\sx*(1.5100)},{\sy*(0.0000)})
	--({\sx*(1.5200)},{\sy*(-0.0000)})
	--({\sx*(1.5300)},{\sy*(-0.0000)})
	--({\sx*(1.5400)},{\sy*(0.0000)})
	--({\sx*(1.5500)},{\sy*(0.0000)})
	--({\sx*(1.5600)},{\sy*(0.0000)})
	--({\sx*(1.5700)},{\sy*(-0.0000)})
	--({\sx*(1.5800)},{\sy*(0.0000)})
	--({\sx*(1.5900)},{\sy*(0.0000)})
	--({\sx*(1.6000)},{\sy*(0.0000)})
	--({\sx*(1.6100)},{\sy*(-0.0000)})
	--({\sx*(1.6200)},{\sy*(-0.0000)})
	--({\sx*(1.6300)},{\sy*(0.0000)})
	--({\sx*(1.6400)},{\sy*(0.0000)})
	--({\sx*(1.6500)},{\sy*(-0.0000)})
	--({\sx*(1.6600)},{\sy*(-0.0000)})
	--({\sx*(1.6700)},{\sy*(-0.0000)})
	--({\sx*(1.6800)},{\sy*(-0.0000)})
	--({\sx*(1.6900)},{\sy*(-0.0000)})
	--({\sx*(1.7000)},{\sy*(-0.0000)})
	--({\sx*(1.7100)},{\sy*(-0.0000)})
	--({\sx*(1.7200)},{\sy*(0.0000)})
	--({\sx*(1.7300)},{\sy*(-0.0000)})
	--({\sx*(1.7400)},{\sy*(-0.0000)})
	--({\sx*(1.7500)},{\sy*(-0.0000)})
	--({\sx*(1.7600)},{\sy*(-0.0000)})
	--({\sx*(1.7700)},{\sy*(0.0000)})
	--({\sx*(1.7800)},{\sy*(-0.0000)})
	--({\sx*(1.7900)},{\sy*(-0.0000)})
	--({\sx*(1.8000)},{\sy*(0.0000)})
	--({\sx*(1.8100)},{\sy*(-0.0000)})
	--({\sx*(1.8200)},{\sy*(-0.0000)})
	--({\sx*(1.8300)},{\sy*(-0.0000)})
	--({\sx*(1.8400)},{\sy*(-0.0000)})
	--({\sx*(1.8500)},{\sy*(-0.0000)})
	--({\sx*(1.8600)},{\sy*(-0.0000)})
	--({\sx*(1.8700)},{\sy*(-0.0000)})
	--({\sx*(1.8800)},{\sy*(-0.0000)})
	--({\sx*(1.8900)},{\sy*(-0.0000)})
	--({\sx*(1.9000)},{\sy*(-0.0000)})
	--({\sx*(1.9100)},{\sy*(-0.0000)})
	--({\sx*(1.9200)},{\sy*(0.0000)})
	--({\sx*(1.9300)},{\sy*(0.0000)})
	--({\sx*(1.9400)},{\sy*(-0.0000)})
	--({\sx*(1.9500)},{\sy*(-0.0000)})
	--({\sx*(1.9600)},{\sy*(0.0000)})
	--({\sx*(1.9700)},{\sy*(0.0000)})
	--({\sx*(1.9800)},{\sy*(0.0000)})
	--({\sx*(1.9900)},{\sy*(-0.0000)})
	--({\sx*(2.0000)},{\sy*(0.0000)})
	--({\sx*(2.0100)},{\sy*(-0.0000)})
	--({\sx*(2.0200)},{\sy*(-0.0000)})
	--({\sx*(2.0300)},{\sy*(0.0000)})
	--({\sx*(2.0400)},{\sy*(0.0000)})
	--({\sx*(2.0500)},{\sy*(-0.0000)})
	--({\sx*(2.0600)},{\sy*(-0.0000)})
	--({\sx*(2.0700)},{\sy*(0.0000)})
	--({\sx*(2.0800)},{\sy*(0.0000)})
	--({\sx*(2.0900)},{\sy*(-0.0000)})
	--({\sx*(2.1000)},{\sy*(-0.0000)})
	--({\sx*(2.1100)},{\sy*(0.0000)})
	--({\sx*(2.1200)},{\sy*(-0.0000)})
	--({\sx*(2.1300)},{\sy*(-0.0000)})
	--({\sx*(2.1400)},{\sy*(0.0000)})
	--({\sx*(2.1500)},{\sy*(-0.0000)})
	--({\sx*(2.1600)},{\sy*(0.0000)})
	--({\sx*(2.1700)},{\sy*(-0.0000)})
	--({\sx*(2.1800)},{\sy*(0.0000)})
	--({\sx*(2.1900)},{\sy*(0.0000)})
	--({\sx*(2.2000)},{\sy*(0.0000)})
	--({\sx*(2.2100)},{\sy*(0.0000)})
	--({\sx*(2.2200)},{\sy*(0.0000)})
	--({\sx*(2.2300)},{\sy*(0.0000)})
	--({\sx*(2.2400)},{\sy*(0.0000)})
	--({\sx*(2.2500)},{\sy*(0.0000)})
	--({\sx*(2.2600)},{\sy*(0.0000)})
	--({\sx*(2.2700)},{\sy*(-0.0000)})
	--({\sx*(2.2800)},{\sy*(0.0000)})
	--({\sx*(2.2900)},{\sy*(-0.0000)})
	--({\sx*(2.3000)},{\sy*(-0.0000)})
	--({\sx*(2.3100)},{\sy*(0.0000)})
	--({\sx*(2.3200)},{\sy*(-0.0000)})
	--({\sx*(2.3300)},{\sy*(-0.0000)})
	--({\sx*(2.3400)},{\sy*(0.0000)})
	--({\sx*(2.3500)},{\sy*(-0.0000)})
	--({\sx*(2.3600)},{\sy*(-0.0000)})
	--({\sx*(2.3700)},{\sy*(-0.0000)})
	--({\sx*(2.3800)},{\sy*(-0.0000)})
	--({\sx*(2.3900)},{\sy*(-0.0000)})
	--({\sx*(2.4000)},{\sy*(-0.0000)})
	--({\sx*(2.4100)},{\sy*(-0.0000)})
	--({\sx*(2.4200)},{\sy*(-0.0000)})
	--({\sx*(2.4300)},{\sy*(0.0000)})
	--({\sx*(2.4400)},{\sy*(0.0000)})
	--({\sx*(2.4500)},{\sy*(0.0000)})
	--({\sx*(2.4600)},{\sy*(-0.0000)})
	--({\sx*(2.4700)},{\sy*(0.0000)})
	--({\sx*(2.4800)},{\sy*(0.0000)})
	--({\sx*(2.4900)},{\sy*(0.0000)})
	--({\sx*(2.5000)},{\sy*(0.0000)})
	--({\sx*(2.5100)},{\sy*(0.0000)})
	--({\sx*(2.5200)},{\sy*(0.0000)})
	--({\sx*(2.5300)},{\sy*(-0.0000)})
	--({\sx*(2.5400)},{\sy*(0.0000)})
	--({\sx*(2.5500)},{\sy*(0.0000)})
	--({\sx*(2.5600)},{\sy*(0.0000)})
	--({\sx*(2.5700)},{\sy*(0.0000)})
	--({\sx*(2.5800)},{\sy*(-0.0000)})
	--({\sx*(2.5900)},{\sy*(-0.0000)})
	--({\sx*(2.6000)},{\sy*(0.0000)})
	--({\sx*(2.6100)},{\sy*(0.0000)})
	--({\sx*(2.6200)},{\sy*(0.0000)})
	--({\sx*(2.6300)},{\sy*(-0.0000)})
	--({\sx*(2.6400)},{\sy*(0.0000)})
	--({\sx*(2.6500)},{\sy*(-0.0000)})
	--({\sx*(2.6600)},{\sy*(0.0000)})
	--({\sx*(2.6700)},{\sy*(0.0000)})
	--({\sx*(2.6800)},{\sy*(0.0000)})
	--({\sx*(2.6900)},{\sy*(-0.0000)})
	--({\sx*(2.7000)},{\sy*(0.0000)})
	--({\sx*(2.7100)},{\sy*(-0.0000)})
	--({\sx*(2.7200)},{\sy*(0.0000)})
	--({\sx*(2.7300)},{\sy*(0.0000)})
	--({\sx*(2.7400)},{\sy*(0.0000)})
	--({\sx*(2.7500)},{\sy*(0.0000)})
	--({\sx*(2.7600)},{\sy*(-0.0000)})
	--({\sx*(2.7700)},{\sy*(0.0000)})
	--({\sx*(2.7800)},{\sy*(-0.0000)})
	--({\sx*(2.7900)},{\sy*(0.0000)})
	--({\sx*(2.8000)},{\sy*(0.0000)})
	--({\sx*(2.8100)},{\sy*(0.0000)})
	--({\sx*(2.8200)},{\sy*(-0.0000)})
	--({\sx*(2.8300)},{\sy*(0.0000)})
	--({\sx*(2.8400)},{\sy*(0.0000)})
	--({\sx*(2.8500)},{\sy*(-0.0000)})
	--({\sx*(2.8600)},{\sy*(-0.0000)})
	--({\sx*(2.8700)},{\sy*(-0.0000)})
	--({\sx*(2.8800)},{\sy*(0.0000)})
	--({\sx*(2.8900)},{\sy*(-0.0000)})
	--({\sx*(2.9000)},{\sy*(0.0000)})
	--({\sx*(2.9100)},{\sy*(0.0000)})
	--({\sx*(2.9200)},{\sy*(-0.0000)})
	--({\sx*(2.9300)},{\sy*(0.0000)})
	--({\sx*(2.9400)},{\sy*(-0.0000)})
	--({\sx*(2.9500)},{\sy*(0.0000)})
	--({\sx*(2.9600)},{\sy*(-0.0000)})
	--({\sx*(2.9700)},{\sy*(-0.0000)})
	--({\sx*(2.9800)},{\sy*(0.0000)})
	--({\sx*(2.9900)},{\sy*(0.0000)})
	--({\sx*(3.0000)},{\sy*(0.0000)})
	--({\sx*(3.0100)},{\sy*(0.0000)})
	--({\sx*(3.0200)},{\sy*(0.0000)})
	--({\sx*(3.0300)},{\sy*(0.0000)})
	--({\sx*(3.0400)},{\sy*(-0.0000)})
	--({\sx*(3.0500)},{\sy*(-0.0000)})
	--({\sx*(3.0600)},{\sy*(-0.0000)})
	--({\sx*(3.0700)},{\sy*(-0.0000)})
	--({\sx*(3.0800)},{\sy*(0.0000)})
	--({\sx*(3.0900)},{\sy*(-0.0000)})
	--({\sx*(3.1000)},{\sy*(0.0000)})
	--({\sx*(3.1100)},{\sy*(-0.0000)})
	--({\sx*(3.1200)},{\sy*(-0.0000)})
	--({\sx*(3.1300)},{\sy*(-0.0000)})
	--({\sx*(3.1400)},{\sy*(-0.0000)})
	--({\sx*(3.1500)},{\sy*(-0.0000)})
	--({\sx*(3.1600)},{\sy*(-0.0000)})
	--({\sx*(3.1700)},{\sy*(-0.0000)})
	--({\sx*(3.1800)},{\sy*(-0.0000)})
	--({\sx*(3.1900)},{\sy*(-0.0000)})
	--({\sx*(3.2000)},{\sy*(0.0000)})
	--({\sx*(3.2100)},{\sy*(-0.0000)})
	--({\sx*(3.2200)},{\sy*(-0.0000)})
	--({\sx*(3.2300)},{\sy*(0.0000)})
	--({\sx*(3.2400)},{\sy*(0.0000)})
	--({\sx*(3.2500)},{\sy*(-0.0000)})
	--({\sx*(3.2600)},{\sy*(0.0000)})
	--({\sx*(3.2700)},{\sy*(0.0000)})
	--({\sx*(3.2800)},{\sy*(0.0000)})
	--({\sx*(3.2900)},{\sy*(0.0000)})
	--({\sx*(3.3000)},{\sy*(0.0000)})
	--({\sx*(3.3100)},{\sy*(0.0000)})
	--({\sx*(3.3200)},{\sy*(0.0000)})
	--({\sx*(3.3300)},{\sy*(0.0000)})
	--({\sx*(3.3400)},{\sy*(0.0000)})
	--({\sx*(3.3500)},{\sy*(0.0000)})
	--({\sx*(3.3600)},{\sy*(0.0000)})
	--({\sx*(3.3700)},{\sy*(0.0000)})
	--({\sx*(3.3800)},{\sy*(-0.0000)})
	--({\sx*(3.3900)},{\sy*(0.0000)})
	--({\sx*(3.4000)},{\sy*(-0.0000)})
	--({\sx*(3.4100)},{\sy*(-0.0000)})
	--({\sx*(3.4200)},{\sy*(-0.0000)})
	--({\sx*(3.4300)},{\sy*(-0.0000)})
	--({\sx*(3.4400)},{\sy*(-0.0000)})
	--({\sx*(3.4500)},{\sy*(-0.0000)})
	--({\sx*(3.4600)},{\sy*(0.0000)})
	--({\sx*(3.4700)},{\sy*(-0.0000)})
	--({\sx*(3.4800)},{\sy*(-0.0000)})
	--({\sx*(3.4900)},{\sy*(-0.0000)})
	--({\sx*(3.5000)},{\sy*(-0.0000)})
	--({\sx*(3.5100)},{\sy*(-0.0000)})
	--({\sx*(3.5200)},{\sy*(-0.0000)})
	--({\sx*(3.5300)},{\sy*(-0.0000)})
	--({\sx*(3.5400)},{\sy*(-0.0000)})
	--({\sx*(3.5500)},{\sy*(-0.0000)})
	--({\sx*(3.5600)},{\sy*(-0.0000)})
	--({\sx*(3.5700)},{\sy*(-0.0000)})
	--({\sx*(3.5800)},{\sy*(0.0000)})
	--({\sx*(3.5900)},{\sy*(0.0000)})
	--({\sx*(3.6000)},{\sy*(-0.0000)})
	--({\sx*(3.6100)},{\sy*(0.0000)})
	--({\sx*(3.6200)},{\sy*(0.0000)})
	--({\sx*(3.6300)},{\sy*(0.0000)})
	--({\sx*(3.6400)},{\sy*(0.0000)})
	--({\sx*(3.6500)},{\sy*(-0.0000)})
	--({\sx*(3.6600)},{\sy*(-0.0000)})
	--({\sx*(3.6700)},{\sy*(0.0000)})
	--({\sx*(3.6800)},{\sy*(0.0000)})
	--({\sx*(3.6900)},{\sy*(-0.0000)})
	--({\sx*(3.7000)},{\sy*(0.0000)})
	--({\sx*(3.7100)},{\sy*(0.0000)})
	--({\sx*(3.7200)},{\sy*(0.0000)})
	--({\sx*(3.7300)},{\sy*(0.0000)})
	--({\sx*(3.7400)},{\sy*(0.0000)})
	--({\sx*(3.7500)},{\sy*(0.0000)})
	--({\sx*(3.7600)},{\sy*(-0.0000)})
	--({\sx*(3.7700)},{\sy*(0.0000)})
	--({\sx*(3.7800)},{\sy*(-0.0000)})
	--({\sx*(3.7900)},{\sy*(-0.0000)})
	--({\sx*(3.8000)},{\sy*(-0.0000)})
	--({\sx*(3.8100)},{\sy*(-0.0000)})
	--({\sx*(3.8200)},{\sy*(-0.0000)})
	--({\sx*(3.8300)},{\sy*(-0.0000)})
	--({\sx*(3.8400)},{\sy*(-0.0000)})
	--({\sx*(3.8500)},{\sy*(-0.0000)})
	--({\sx*(3.8600)},{\sy*(-0.0000)})
	--({\sx*(3.8700)},{\sy*(-0.0000)})
	--({\sx*(3.8800)},{\sy*(-0.0000)})
	--({\sx*(3.8900)},{\sy*(-0.0000)})
	--({\sx*(3.9000)},{\sy*(-0.0000)})
	--({\sx*(3.9100)},{\sy*(0.0000)})
	--({\sx*(3.9200)},{\sy*(-0.0000)})
	--({\sx*(3.9300)},{\sy*(0.0000)})
	--({\sx*(3.9400)},{\sy*(0.0000)})
	--({\sx*(3.9500)},{\sy*(0.0000)})
	--({\sx*(3.9600)},{\sy*(0.0000)})
	--({\sx*(3.9700)},{\sy*(0.0000)})
	--({\sx*(3.9800)},{\sy*(0.0000)})
	--({\sx*(3.9900)},{\sy*(0.0000)})
	--({\sx*(4.0000)},{\sy*(0.0000)})
	--({\sx*(4.0100)},{\sy*(0.0000)})
	--({\sx*(4.0200)},{\sy*(0.0000)})
	--({\sx*(4.0300)},{\sy*(0.0000)})
	--({\sx*(4.0400)},{\sy*(0.0000)})
	--({\sx*(4.0500)},{\sy*(0.0000)})
	--({\sx*(4.0600)},{\sy*(0.0000)})
	--({\sx*(4.0700)},{\sy*(0.0000)})
	--({\sx*(4.0800)},{\sy*(0.0000)})
	--({\sx*(4.0900)},{\sy*(0.0000)})
	--({\sx*(4.1000)},{\sy*(0.0000)})
	--({\sx*(4.1100)},{\sy*(-0.0000)})
	--({\sx*(4.1200)},{\sy*(-0.0000)})
	--({\sx*(4.1300)},{\sy*(-0.0000)})
	--({\sx*(4.1400)},{\sy*(-0.0000)})
	--({\sx*(4.1500)},{\sy*(-0.0000)})
	--({\sx*(4.1600)},{\sy*(-0.0000)})
	--({\sx*(4.1700)},{\sy*(-0.0000)})
	--({\sx*(4.1800)},{\sy*(-0.0001)})
	--({\sx*(4.1900)},{\sy*(-0.0001)})
	--({\sx*(4.2000)},{\sy*(-0.0001)})
	--({\sx*(4.2100)},{\sy*(-0.0000)})
	--({\sx*(4.2200)},{\sy*(-0.0001)})
	--({\sx*(4.2300)},{\sy*(-0.0001)})
	--({\sx*(4.2400)},{\sy*(-0.0000)})
	--({\sx*(4.2500)},{\sy*(-0.0000)})
	--({\sx*(4.2600)},{\sy*(-0.0000)})
	--({\sx*(4.2700)},{\sy*(-0.0001)})
	--({\sx*(4.2800)},{\sy*(-0.0000)})
	--({\sx*(4.2900)},{\sy*(0.0000)})
	--({\sx*(4.3000)},{\sy*(0.0000)})
	--({\sx*(4.3100)},{\sy*(-0.0000)})
	--({\sx*(4.3200)},{\sy*(0.0002)})
	--({\sx*(4.3300)},{\sy*(0.0001)})
	--({\sx*(4.3400)},{\sy*(0.0000)})
	--({\sx*(4.3500)},{\sy*(0.0001)})
	--({\sx*(4.3600)},{\sy*(0.0003)})
	--({\sx*(4.3700)},{\sy*(0.0004)})
	--({\sx*(4.3800)},{\sy*(0.0002)})
	--({\sx*(4.3900)},{\sy*(0.0003)})
	--({\sx*(4.4000)},{\sy*(0.0003)})
	--({\sx*(4.4100)},{\sy*(0.0000)})
	--({\sx*(4.4200)},{\sy*(0.0001)})
	--({\sx*(4.4300)},{\sy*(0.0004)})
	--({\sx*(4.4400)},{\sy*(0.0003)})
	--({\sx*(4.4500)},{\sy*(0.0001)})
	--({\sx*(4.4600)},{\sy*(0.0000)})
	--({\sx*(4.4700)},{\sy*(-0.0001)})
	--({\sx*(4.4800)},{\sy*(-0.0003)})
	--({\sx*(4.4900)},{\sy*(-0.0002)})
	--({\sx*(4.5000)},{\sy*(-0.0008)})
	--({\sx*(4.5100)},{\sy*(-0.0006)})
	--({\sx*(4.5200)},{\sy*(-0.0017)})
	--({\sx*(4.5300)},{\sy*(-0.0010)})
	--({\sx*(4.5400)},{\sy*(-0.0024)})
	--({\sx*(4.5500)},{\sy*(-0.0023)})
	--({\sx*(4.5600)},{\sy*(-0.0040)})
	--({\sx*(4.5700)},{\sy*(-0.0008)})
	--({\sx*(4.5800)},{\sy*(-0.0041)})
	--({\sx*(4.5900)},{\sy*(-0.0047)})
	--({\sx*(4.6000)},{\sy*(-0.0028)})
	--({\sx*(4.6100)},{\sy*(-0.0016)})
	--({\sx*(4.6200)},{\sy*(-0.0016)})
	--({\sx*(4.6300)},{\sy*(-0.0003)})
	--({\sx*(4.6400)},{\sy*(-0.0004)})
	--({\sx*(4.6500)},{\sy*(0.0004)})
	--({\sx*(4.6600)},{\sy*(0.0018)})
	--({\sx*(4.6700)},{\sy*(0.0063)})
	--({\sx*(4.6800)},{\sy*(0.0097)})
	--({\sx*(4.6900)},{\sy*(0.0050)})
	--({\sx*(4.7000)},{\sy*(0.0143)})
	--({\sx*(4.7100)},{\sy*(0.0186)})
	--({\sx*(4.7200)},{\sy*(0.0233)})
	--({\sx*(4.7300)},{\sy*(0.0174)})
	--({\sx*(4.7400)},{\sy*(0.0225)})
	--({\sx*(4.7500)},{\sy*(0.0300)})
	--({\sx*(4.7600)},{\sy*(0.0450)})
	--({\sx*(4.7700)},{\sy*(0.0318)})
	--({\sx*(4.7800)},{\sy*(0.0239)})
	--({\sx*(4.7900)},{\sy*(0.0228)})
	--({\sx*(4.8000)},{\sy*(0.0210)})
	--({\sx*(4.8100)},{\sy*(0.0073)})
	--({\sx*(4.8200)},{\sy*(0.0019)})
	--({\sx*(4.8300)},{\sy*(-0.0150)})
	--({\sx*(4.8400)},{\sy*(-0.0440)})
	--({\sx*(4.8500)},{\sy*(-0.0013)})
	--({\sx*(4.8600)},{\sy*(-0.1712)})
	--({\sx*(4.8700)},{\sy*(-0.1488)})
	--({\sx*(4.8800)},{\sy*(-0.2815)})
	--({\sx*(4.8900)},{\sy*(-0.2955)})
	--({\sx*(4.9000)},{\sy*(-0.2139)})
	--({\sx*(4.9100)},{\sy*(-0.2219)})
	--({\sx*(4.9200)},{\sy*(-0.4330)})
	--({\sx*(4.9300)},{\sy*(-0.3861)})
	--({\sx*(4.9400)},{\sy*(-0.8739)})
	--({\sx*(4.9500)},{\sy*(-1.0000)})
	--({\sx*(4.9600)},{\sy*(-0.8373)})
	--({\sx*(4.9700)},{\sy*(-0.5337)})
	--({\sx*(4.9800)},{\sy*(-0.8500)})
	--({\sx*(4.9900)},{\sy*(-0.4042)})
	--({\sx*(5.0000)},{\sy*(0.0000)});
}
\def\relfehlern{
\draw[color=blue,line width=1.4pt,line join=round] ({\sx*(0.000)},{\sy*(0.0000)})
	--({\sx*(0.0100)},{\sy*(-0.0000)})
	--({\sx*(0.0200)},{\sy*(-0.0000)})
	--({\sx*(0.0300)},{\sy*(0.0000)})
	--({\sx*(0.0400)},{\sy*(0.0000)})
	--({\sx*(0.0500)},{\sy*(-0.0000)})
	--({\sx*(0.0600)},{\sy*(-0.0000)})
	--({\sx*(0.0700)},{\sy*(0.0000)})
	--({\sx*(0.0800)},{\sy*(0.0000)})
	--({\sx*(0.0900)},{\sy*(-0.0000)})
	--({\sx*(0.1000)},{\sy*(-0.0000)})
	--({\sx*(0.1100)},{\sy*(0.0000)})
	--({\sx*(0.1200)},{\sy*(-0.0000)})
	--({\sx*(0.1300)},{\sy*(-0.0000)})
	--({\sx*(0.1400)},{\sy*(-0.0000)})
	--({\sx*(0.1500)},{\sy*(0.0000)})
	--({\sx*(0.1600)},{\sy*(-0.0000)})
	--({\sx*(0.1700)},{\sy*(-0.0000)})
	--({\sx*(0.1800)},{\sy*(0.0000)})
	--({\sx*(0.1900)},{\sy*(-0.0000)})
	--({\sx*(0.2000)},{\sy*(-0.0000)})
	--({\sx*(0.2100)},{\sy*(-0.0000)})
	--({\sx*(0.2200)},{\sy*(0.0000)})
	--({\sx*(0.2300)},{\sy*(0.0000)})
	--({\sx*(0.2400)},{\sy*(-0.0000)})
	--({\sx*(0.2500)},{\sy*(-0.0000)})
	--({\sx*(0.2600)},{\sy*(-0.0000)})
	--({\sx*(0.2700)},{\sy*(-0.0000)})
	--({\sx*(0.2800)},{\sy*(0.0000)})
	--({\sx*(0.2900)},{\sy*(-0.0000)})
	--({\sx*(0.3000)},{\sy*(-0.0000)})
	--({\sx*(0.3100)},{\sy*(-0.0000)})
	--({\sx*(0.3200)},{\sy*(0.0000)})
	--({\sx*(0.3300)},{\sy*(-0.0000)})
	--({\sx*(0.3400)},{\sy*(-0.0000)})
	--({\sx*(0.3500)},{\sy*(-0.0000)})
	--({\sx*(0.3600)},{\sy*(0.0000)})
	--({\sx*(0.3700)},{\sy*(0.0000)})
	--({\sx*(0.3800)},{\sy*(0.0000)})
	--({\sx*(0.3900)},{\sy*(0.0000)})
	--({\sx*(0.4000)},{\sy*(0.0000)})
	--({\sx*(0.4100)},{\sy*(0.0000)})
	--({\sx*(0.4200)},{\sy*(-0.0000)})
	--({\sx*(0.4300)},{\sy*(0.0000)})
	--({\sx*(0.4400)},{\sy*(0.0000)})
	--({\sx*(0.4500)},{\sy*(0.0000)})
	--({\sx*(0.4600)},{\sy*(-0.0000)})
	--({\sx*(0.4700)},{\sy*(-0.0000)})
	--({\sx*(0.4800)},{\sy*(-0.0000)})
	--({\sx*(0.4900)},{\sy*(0.0000)})
	--({\sx*(0.5000)},{\sy*(0.0000)})
	--({\sx*(0.5100)},{\sy*(-0.0000)})
	--({\sx*(0.5200)},{\sy*(0.0000)})
	--({\sx*(0.5300)},{\sy*(-0.0000)})
	--({\sx*(0.5400)},{\sy*(-0.0000)})
	--({\sx*(0.5500)},{\sy*(-0.0000)})
	--({\sx*(0.5600)},{\sy*(0.0000)})
	--({\sx*(0.5700)},{\sy*(0.0000)})
	--({\sx*(0.5800)},{\sy*(-0.0000)})
	--({\sx*(0.5900)},{\sy*(0.0000)})
	--({\sx*(0.6000)},{\sy*(0.0000)})
	--({\sx*(0.6100)},{\sy*(0.0000)})
	--({\sx*(0.6200)},{\sy*(-0.0000)})
	--({\sx*(0.6300)},{\sy*(0.0000)})
	--({\sx*(0.6400)},{\sy*(0.0000)})
	--({\sx*(0.6500)},{\sy*(0.0000)})
	--({\sx*(0.6600)},{\sy*(0.0000)})
	--({\sx*(0.6700)},{\sy*(0.0000)})
	--({\sx*(0.6800)},{\sy*(-0.0000)})
	--({\sx*(0.6900)},{\sy*(0.0000)})
	--({\sx*(0.7000)},{\sy*(-0.0000)})
	--({\sx*(0.7100)},{\sy*(0.0000)})
	--({\sx*(0.7200)},{\sy*(0.0000)})
	--({\sx*(0.7300)},{\sy*(-0.0000)})
	--({\sx*(0.7400)},{\sy*(-0.0000)})
	--({\sx*(0.7500)},{\sy*(0.0000)})
	--({\sx*(0.7600)},{\sy*(-0.0000)})
	--({\sx*(0.7700)},{\sy*(-0.0000)})
	--({\sx*(0.7800)},{\sy*(-0.0000)})
	--({\sx*(0.7900)},{\sy*(0.0000)})
	--({\sx*(0.8000)},{\sy*(0.0000)})
	--({\sx*(0.8100)},{\sy*(-0.0000)})
	--({\sx*(0.8200)},{\sy*(-0.0000)})
	--({\sx*(0.8300)},{\sy*(-0.0000)})
	--({\sx*(0.8400)},{\sy*(-0.0000)})
	--({\sx*(0.8500)},{\sy*(-0.0000)})
	--({\sx*(0.8600)},{\sy*(-0.0000)})
	--({\sx*(0.8700)},{\sy*(-0.0000)})
	--({\sx*(0.8800)},{\sy*(0.0000)})
	--({\sx*(0.8900)},{\sy*(-0.0000)})
	--({\sx*(0.9000)},{\sy*(0.0000)})
	--({\sx*(0.9100)},{\sy*(-0.0000)})
	--({\sx*(0.9200)},{\sy*(-0.0000)})
	--({\sx*(0.9300)},{\sy*(-0.0000)})
	--({\sx*(0.9400)},{\sy*(0.0000)})
	--({\sx*(0.9500)},{\sy*(-0.0000)})
	--({\sx*(0.9600)},{\sy*(-0.0000)})
	--({\sx*(0.9700)},{\sy*(-0.0000)})
	--({\sx*(0.9800)},{\sy*(-0.0000)})
	--({\sx*(0.9900)},{\sy*(-0.0000)})
	--({\sx*(1.0000)},{\sy*(-0.0000)})
	--({\sx*(1.0100)},{\sy*(0.0000)})
	--({\sx*(1.0200)},{\sy*(-0.0000)})
	--({\sx*(1.0300)},{\sy*(-0.0000)})
	--({\sx*(1.0400)},{\sy*(0.0000)})
	--({\sx*(1.0500)},{\sy*(0.0000)})
	--({\sx*(1.0600)},{\sy*(-0.0000)})
	--({\sx*(1.0700)},{\sy*(-0.0000)})
	--({\sx*(1.0800)},{\sy*(-0.0000)})
	--({\sx*(1.0900)},{\sy*(0.0000)})
	--({\sx*(1.1000)},{\sy*(-0.0000)})
	--({\sx*(1.1100)},{\sy*(-0.0000)})
	--({\sx*(1.1200)},{\sy*(0.0000)})
	--({\sx*(1.1300)},{\sy*(-0.0000)})
	--({\sx*(1.1400)},{\sy*(-0.0000)})
	--({\sx*(1.1500)},{\sy*(-0.0000)})
	--({\sx*(1.1600)},{\sy*(-0.0000)})
	--({\sx*(1.1700)},{\sy*(0.0000)})
	--({\sx*(1.1800)},{\sy*(0.0000)})
	--({\sx*(1.1900)},{\sy*(-0.0000)})
	--({\sx*(1.2000)},{\sy*(-0.0000)})
	--({\sx*(1.2100)},{\sy*(0.0000)})
	--({\sx*(1.2200)},{\sy*(-0.0000)})
	--({\sx*(1.2300)},{\sy*(0.0000)})
	--({\sx*(1.2400)},{\sy*(-0.0000)})
	--({\sx*(1.2500)},{\sy*(0.0000)})
	--({\sx*(1.2600)},{\sy*(0.0000)})
	--({\sx*(1.2700)},{\sy*(-0.0000)})
	--({\sx*(1.2800)},{\sy*(-0.0000)})
	--({\sx*(1.2900)},{\sy*(0.0000)})
	--({\sx*(1.3000)},{\sy*(0.0000)})
	--({\sx*(1.3100)},{\sy*(0.0000)})
	--({\sx*(1.3200)},{\sy*(-0.0000)})
	--({\sx*(1.3300)},{\sy*(-0.0000)})
	--({\sx*(1.3400)},{\sy*(-0.0000)})
	--({\sx*(1.3500)},{\sy*(0.0000)})
	--({\sx*(1.3600)},{\sy*(-0.0000)})
	--({\sx*(1.3700)},{\sy*(0.0000)})
	--({\sx*(1.3800)},{\sy*(0.0000)})
	--({\sx*(1.3900)},{\sy*(0.0000)})
	--({\sx*(1.4000)},{\sy*(0.0000)})
	--({\sx*(1.4100)},{\sy*(0.0000)})
	--({\sx*(1.4200)},{\sy*(0.0000)})
	--({\sx*(1.4300)},{\sy*(0.0000)})
	--({\sx*(1.4400)},{\sy*(0.0000)})
	--({\sx*(1.4500)},{\sy*(0.0000)})
	--({\sx*(1.4600)},{\sy*(0.0000)})
	--({\sx*(1.4700)},{\sy*(0.0000)})
	--({\sx*(1.4800)},{\sy*(0.0000)})
	--({\sx*(1.4900)},{\sy*(-0.0000)})
	--({\sx*(1.5000)},{\sy*(0.0000)})
	--({\sx*(1.5100)},{\sy*(0.0000)})
	--({\sx*(1.5200)},{\sy*(-0.0000)})
	--({\sx*(1.5300)},{\sy*(-0.0000)})
	--({\sx*(1.5400)},{\sy*(0.0000)})
	--({\sx*(1.5500)},{\sy*(0.0000)})
	--({\sx*(1.5600)},{\sy*(0.0000)})
	--({\sx*(1.5700)},{\sy*(-0.0000)})
	--({\sx*(1.5800)},{\sy*(0.0000)})
	--({\sx*(1.5900)},{\sy*(0.0000)})
	--({\sx*(1.6000)},{\sy*(0.0000)})
	--({\sx*(1.6100)},{\sy*(-0.0000)})
	--({\sx*(1.6200)},{\sy*(-0.0000)})
	--({\sx*(1.6300)},{\sy*(0.0000)})
	--({\sx*(1.6400)},{\sy*(0.0000)})
	--({\sx*(1.6500)},{\sy*(-0.0000)})
	--({\sx*(1.6600)},{\sy*(-0.0000)})
	--({\sx*(1.6700)},{\sy*(-0.0000)})
	--({\sx*(1.6800)},{\sy*(-0.0000)})
	--({\sx*(1.6900)},{\sy*(-0.0000)})
	--({\sx*(1.7000)},{\sy*(-0.0000)})
	--({\sx*(1.7100)},{\sy*(-0.0000)})
	--({\sx*(1.7200)},{\sy*(0.0000)})
	--({\sx*(1.7300)},{\sy*(-0.0000)})
	--({\sx*(1.7400)},{\sy*(-0.0000)})
	--({\sx*(1.7500)},{\sy*(-0.0000)})
	--({\sx*(1.7600)},{\sy*(-0.0000)})
	--({\sx*(1.7700)},{\sy*(0.0000)})
	--({\sx*(1.7800)},{\sy*(-0.0000)})
	--({\sx*(1.7900)},{\sy*(-0.0000)})
	--({\sx*(1.8000)},{\sy*(0.0000)})
	--({\sx*(1.8100)},{\sy*(-0.0000)})
	--({\sx*(1.8200)},{\sy*(-0.0000)})
	--({\sx*(1.8300)},{\sy*(-0.0000)})
	--({\sx*(1.8400)},{\sy*(-0.0000)})
	--({\sx*(1.8500)},{\sy*(-0.0000)})
	--({\sx*(1.8600)},{\sy*(-0.0000)})
	--({\sx*(1.8700)},{\sy*(-0.0000)})
	--({\sx*(1.8800)},{\sy*(-0.0000)})
	--({\sx*(1.8900)},{\sy*(-0.0000)})
	--({\sx*(1.9000)},{\sy*(-0.0000)})
	--({\sx*(1.9100)},{\sy*(-0.0000)})
	--({\sx*(1.9200)},{\sy*(0.0000)})
	--({\sx*(1.9300)},{\sy*(0.0000)})
	--({\sx*(1.9400)},{\sy*(-0.0000)})
	--({\sx*(1.9500)},{\sy*(-0.0000)})
	--({\sx*(1.9600)},{\sy*(0.0000)})
	--({\sx*(1.9700)},{\sy*(0.0000)})
	--({\sx*(1.9800)},{\sy*(0.0000)})
	--({\sx*(1.9900)},{\sy*(-0.0000)})
	--({\sx*(2.0000)},{\sy*(0.0000)})
	--({\sx*(2.0100)},{\sy*(-0.0000)})
	--({\sx*(2.0200)},{\sy*(-0.0000)})
	--({\sx*(2.0300)},{\sy*(0.0000)})
	--({\sx*(2.0400)},{\sy*(0.0000)})
	--({\sx*(2.0500)},{\sy*(-0.0000)})
	--({\sx*(2.0600)},{\sy*(-0.0000)})
	--({\sx*(2.0700)},{\sy*(0.0000)})
	--({\sx*(2.0800)},{\sy*(0.0000)})
	--({\sx*(2.0900)},{\sy*(-0.0000)})
	--({\sx*(2.1000)},{\sy*(-0.0000)})
	--({\sx*(2.1100)},{\sy*(0.0000)})
	--({\sx*(2.1200)},{\sy*(-0.0000)})
	--({\sx*(2.1300)},{\sy*(-0.0000)})
	--({\sx*(2.1400)},{\sy*(0.0000)})
	--({\sx*(2.1500)},{\sy*(-0.0000)})
	--({\sx*(2.1600)},{\sy*(0.0000)})
	--({\sx*(2.1700)},{\sy*(-0.0000)})
	--({\sx*(2.1800)},{\sy*(0.0000)})
	--({\sx*(2.1900)},{\sy*(0.0000)})
	--({\sx*(2.2000)},{\sy*(0.0000)})
	--({\sx*(2.2100)},{\sy*(0.0000)})
	--({\sx*(2.2200)},{\sy*(0.0000)})
	--({\sx*(2.2300)},{\sy*(0.0000)})
	--({\sx*(2.2400)},{\sy*(0.0000)})
	--({\sx*(2.2500)},{\sy*(0.0000)})
	--({\sx*(2.2600)},{\sy*(0.0000)})
	--({\sx*(2.2700)},{\sy*(-0.0000)})
	--({\sx*(2.2800)},{\sy*(0.0000)})
	--({\sx*(2.2900)},{\sy*(-0.0000)})
	--({\sx*(2.3000)},{\sy*(-0.0000)})
	--({\sx*(2.3100)},{\sy*(0.0000)})
	--({\sx*(2.3200)},{\sy*(-0.0000)})
	--({\sx*(2.3300)},{\sy*(-0.0000)})
	--({\sx*(2.3400)},{\sy*(0.0000)})
	--({\sx*(2.3500)},{\sy*(-0.0000)})
	--({\sx*(2.3600)},{\sy*(-0.0000)})
	--({\sx*(2.3700)},{\sy*(-0.0000)})
	--({\sx*(2.3800)},{\sy*(-0.0000)})
	--({\sx*(2.3900)},{\sy*(-0.0000)})
	--({\sx*(2.4000)},{\sy*(-0.0000)})
	--({\sx*(2.4100)},{\sy*(-0.0000)})
	--({\sx*(2.4200)},{\sy*(-0.0000)})
	--({\sx*(2.4300)},{\sy*(0.0000)})
	--({\sx*(2.4400)},{\sy*(0.0000)})
	--({\sx*(2.4500)},{\sy*(0.0000)})
	--({\sx*(2.4600)},{\sy*(-0.0000)})
	--({\sx*(2.4700)},{\sy*(0.0000)})
	--({\sx*(2.4800)},{\sy*(0.0000)})
	--({\sx*(2.4900)},{\sy*(0.0000)})
	--({\sx*(2.5000)},{\sy*(0.0000)})
	--({\sx*(2.5100)},{\sy*(0.0000)})
	--({\sx*(2.5200)},{\sy*(0.0000)})
	--({\sx*(2.5300)},{\sy*(-0.0000)})
	--({\sx*(2.5400)},{\sy*(0.0000)})
	--({\sx*(2.5500)},{\sy*(0.0000)})
	--({\sx*(2.5600)},{\sy*(0.0000)})
	--({\sx*(2.5700)},{\sy*(0.0000)})
	--({\sx*(2.5800)},{\sy*(-0.0000)})
	--({\sx*(2.5900)},{\sy*(-0.0000)})
	--({\sx*(2.6000)},{\sy*(0.0000)})
	--({\sx*(2.6100)},{\sy*(0.0000)})
	--({\sx*(2.6200)},{\sy*(0.0000)})
	--({\sx*(2.6300)},{\sy*(-0.0000)})
	--({\sx*(2.6400)},{\sy*(0.0000)})
	--({\sx*(2.6500)},{\sy*(-0.0000)})
	--({\sx*(2.6600)},{\sy*(0.0000)})
	--({\sx*(2.6700)},{\sy*(0.0000)})
	--({\sx*(2.6800)},{\sy*(0.0000)})
	--({\sx*(2.6900)},{\sy*(-0.0000)})
	--({\sx*(2.7000)},{\sy*(0.0000)})
	--({\sx*(2.7100)},{\sy*(-0.0000)})
	--({\sx*(2.7200)},{\sy*(0.0000)})
	--({\sx*(2.7300)},{\sy*(0.0000)})
	--({\sx*(2.7400)},{\sy*(0.0000)})
	--({\sx*(2.7500)},{\sy*(0.0000)})
	--({\sx*(2.7600)},{\sy*(-0.0000)})
	--({\sx*(2.7700)},{\sy*(0.0000)})
	--({\sx*(2.7800)},{\sy*(-0.0000)})
	--({\sx*(2.7900)},{\sy*(0.0000)})
	--({\sx*(2.8000)},{\sy*(0.0000)})
	--({\sx*(2.8100)},{\sy*(0.0000)})
	--({\sx*(2.8200)},{\sy*(-0.0000)})
	--({\sx*(2.8300)},{\sy*(0.0000)})
	--({\sx*(2.8400)},{\sy*(0.0000)})
	--({\sx*(2.8500)},{\sy*(-0.0000)})
	--({\sx*(2.8600)},{\sy*(-0.0000)})
	--({\sx*(2.8700)},{\sy*(-0.0000)})
	--({\sx*(2.8800)},{\sy*(0.0000)})
	--({\sx*(2.8900)},{\sy*(-0.0000)})
	--({\sx*(2.9000)},{\sy*(0.0000)})
	--({\sx*(2.9100)},{\sy*(0.0000)})
	--({\sx*(2.9200)},{\sy*(-0.0000)})
	--({\sx*(2.9300)},{\sy*(0.0000)})
	--({\sx*(2.9400)},{\sy*(-0.0000)})
	--({\sx*(2.9500)},{\sy*(0.0000)})
	--({\sx*(2.9600)},{\sy*(-0.0000)})
	--({\sx*(2.9700)},{\sy*(-0.0000)})
	--({\sx*(2.9800)},{\sy*(0.0000)})
	--({\sx*(2.9900)},{\sy*(0.0000)})
	--({\sx*(3.0000)},{\sy*(0.0000)})
	--({\sx*(3.0100)},{\sy*(0.0000)})
	--({\sx*(3.0200)},{\sy*(0.0000)})
	--({\sx*(3.0300)},{\sy*(0.0000)})
	--({\sx*(3.0400)},{\sy*(-0.0000)})
	--({\sx*(3.0500)},{\sy*(-0.0000)})
	--({\sx*(3.0600)},{\sy*(-0.0000)})
	--({\sx*(3.0700)},{\sy*(-0.0000)})
	--({\sx*(3.0800)},{\sy*(0.0000)})
	--({\sx*(3.0900)},{\sy*(-0.0000)})
	--({\sx*(3.1000)},{\sy*(0.0000)})
	--({\sx*(3.1100)},{\sy*(-0.0000)})
	--({\sx*(3.1200)},{\sy*(-0.0000)})
	--({\sx*(3.1300)},{\sy*(-0.0000)})
	--({\sx*(3.1400)},{\sy*(-0.0000)})
	--({\sx*(3.1500)},{\sy*(-0.0000)})
	--({\sx*(3.1600)},{\sy*(-0.0000)})
	--({\sx*(3.1700)},{\sy*(-0.0000)})
	--({\sx*(3.1800)},{\sy*(-0.0000)})
	--({\sx*(3.1900)},{\sy*(-0.0000)})
	--({\sx*(3.2000)},{\sy*(0.0000)})
	--({\sx*(3.2100)},{\sy*(-0.0000)})
	--({\sx*(3.2200)},{\sy*(-0.0000)})
	--({\sx*(3.2300)},{\sy*(0.0000)})
	--({\sx*(3.2400)},{\sy*(0.0000)})
	--({\sx*(3.2500)},{\sy*(-0.0000)})
	--({\sx*(3.2600)},{\sy*(0.0000)})
	--({\sx*(3.2700)},{\sy*(0.0000)})
	--({\sx*(3.2800)},{\sy*(0.0000)})
	--({\sx*(3.2900)},{\sy*(0.0000)})
	--({\sx*(3.3000)},{\sy*(0.0000)})
	--({\sx*(3.3100)},{\sy*(0.0000)})
	--({\sx*(3.3200)},{\sy*(0.0000)})
	--({\sx*(3.3300)},{\sy*(0.0000)})
	--({\sx*(3.3400)},{\sy*(0.0000)})
	--({\sx*(3.3500)},{\sy*(0.0000)})
	--({\sx*(3.3600)},{\sy*(0.0000)})
	--({\sx*(3.3700)},{\sy*(0.0000)})
	--({\sx*(3.3800)},{\sy*(-0.0000)})
	--({\sx*(3.3900)},{\sy*(0.0000)})
	--({\sx*(3.4000)},{\sy*(-0.0000)})
	--({\sx*(3.4100)},{\sy*(-0.0000)})
	--({\sx*(3.4200)},{\sy*(-0.0000)})
	--({\sx*(3.4300)},{\sy*(-0.0000)})
	--({\sx*(3.4400)},{\sy*(-0.0000)})
	--({\sx*(3.4500)},{\sy*(-0.0000)})
	--({\sx*(3.4600)},{\sy*(0.0000)})
	--({\sx*(3.4700)},{\sy*(-0.0000)})
	--({\sx*(3.4800)},{\sy*(-0.0000)})
	--({\sx*(3.4900)},{\sy*(-0.0000)})
	--({\sx*(3.5000)},{\sy*(-0.0000)})
	--({\sx*(3.5100)},{\sy*(-0.0000)})
	--({\sx*(3.5200)},{\sy*(-0.0000)})
	--({\sx*(3.5300)},{\sy*(-0.0000)})
	--({\sx*(3.5400)},{\sy*(-0.0000)})
	--({\sx*(3.5500)},{\sy*(-0.0000)})
	--({\sx*(3.5600)},{\sy*(-0.0000)})
	--({\sx*(3.5700)},{\sy*(-0.0000)})
	--({\sx*(3.5800)},{\sy*(0.0000)})
	--({\sx*(3.5900)},{\sy*(0.0000)})
	--({\sx*(3.6000)},{\sy*(-0.0000)})
	--({\sx*(3.6100)},{\sy*(0.0000)})
	--({\sx*(3.6200)},{\sy*(0.0000)})
	--({\sx*(3.6300)},{\sy*(0.0000)})
	--({\sx*(3.6400)},{\sy*(0.0000)})
	--({\sx*(3.6500)},{\sy*(-0.0000)})
	--({\sx*(3.6600)},{\sy*(-0.0000)})
	--({\sx*(3.6700)},{\sy*(0.0000)})
	--({\sx*(3.6800)},{\sy*(0.0000)})
	--({\sx*(3.6900)},{\sy*(-0.0000)})
	--({\sx*(3.7000)},{\sy*(0.0000)})
	--({\sx*(3.7100)},{\sy*(0.0000)})
	--({\sx*(3.7200)},{\sy*(0.0000)})
	--({\sx*(3.7300)},{\sy*(0.0000)})
	--({\sx*(3.7400)},{\sy*(0.0000)})
	--({\sx*(3.7500)},{\sy*(0.0000)})
	--({\sx*(3.7600)},{\sy*(-0.0000)})
	--({\sx*(3.7700)},{\sy*(0.0000)})
	--({\sx*(3.7800)},{\sy*(-0.0000)})
	--({\sx*(3.7900)},{\sy*(-0.0000)})
	--({\sx*(3.8000)},{\sy*(-0.0000)})
	--({\sx*(3.8100)},{\sy*(-0.0000)})
	--({\sx*(3.8200)},{\sy*(-0.0000)})
	--({\sx*(3.8300)},{\sy*(-0.0000)})
	--({\sx*(3.8400)},{\sy*(-0.0000)})
	--({\sx*(3.8500)},{\sy*(-0.0000)})
	--({\sx*(3.8600)},{\sy*(-0.0000)})
	--({\sx*(3.8700)},{\sy*(-0.0000)})
	--({\sx*(3.8800)},{\sy*(-0.0000)})
	--({\sx*(3.8900)},{\sy*(-0.0000)})
	--({\sx*(3.9000)},{\sy*(-0.0000)})
	--({\sx*(3.9100)},{\sy*(0.0000)})
	--({\sx*(3.9200)},{\sy*(-0.0000)})
	--({\sx*(3.9300)},{\sy*(0.0000)})
	--({\sx*(3.9400)},{\sy*(0.0000)})
	--({\sx*(3.9500)},{\sy*(0.0000)})
	--({\sx*(3.9600)},{\sy*(0.0000)})
	--({\sx*(3.9700)},{\sy*(0.0000)})
	--({\sx*(3.9800)},{\sy*(0.0000)})
	--({\sx*(3.9900)},{\sy*(0.0000)})
	--({\sx*(4.0000)},{\sy*(0.0000)})
	--({\sx*(4.0100)},{\sy*(0.0000)})
	--({\sx*(4.0200)},{\sy*(0.0000)})
	--({\sx*(4.0300)},{\sy*(0.0000)})
	--({\sx*(4.0400)},{\sy*(0.0000)})
	--({\sx*(4.0500)},{\sy*(0.0000)})
	--({\sx*(4.0600)},{\sy*(0.0000)})
	--({\sx*(4.0700)},{\sy*(0.0000)})
	--({\sx*(4.0800)},{\sy*(0.0000)})
	--({\sx*(4.0900)},{\sy*(0.0000)})
	--({\sx*(4.1000)},{\sy*(0.0000)})
	--({\sx*(4.1100)},{\sy*(-0.0000)})
	--({\sx*(4.1200)},{\sy*(-0.0000)})
	--({\sx*(4.1300)},{\sy*(-0.0000)})
	--({\sx*(4.1400)},{\sy*(-0.0000)})
	--({\sx*(4.1500)},{\sy*(-0.0000)})
	--({\sx*(4.1600)},{\sy*(-0.0000)})
	--({\sx*(4.1700)},{\sy*(-0.0000)})
	--({\sx*(4.1800)},{\sy*(-0.0000)})
	--({\sx*(4.1900)},{\sy*(-0.0000)})
	--({\sx*(4.2000)},{\sy*(-0.0000)})
	--({\sx*(4.2100)},{\sy*(-0.0000)})
	--({\sx*(4.2200)},{\sy*(-0.0000)})
	--({\sx*(4.2300)},{\sy*(-0.0000)})
	--({\sx*(4.2400)},{\sy*(-0.0000)})
	--({\sx*(4.2500)},{\sy*(-0.0000)})
	--({\sx*(4.2600)},{\sy*(-0.0000)})
	--({\sx*(4.2700)},{\sy*(-0.0000)})
	--({\sx*(4.2800)},{\sy*(-0.0000)})
	--({\sx*(4.2900)},{\sy*(0.0000)})
	--({\sx*(4.3000)},{\sy*(0.0000)})
	--({\sx*(4.3100)},{\sy*(-0.0000)})
	--({\sx*(4.3200)},{\sy*(0.0000)})
	--({\sx*(4.3300)},{\sy*(0.0000)})
	--({\sx*(4.3400)},{\sy*(0.0000)})
	--({\sx*(4.3500)},{\sy*(0.0000)})
	--({\sx*(4.3600)},{\sy*(0.0000)})
	--({\sx*(4.3700)},{\sy*(0.0000)})
	--({\sx*(4.3800)},{\sy*(0.0000)})
	--({\sx*(4.3900)},{\sy*(0.0000)})
	--({\sx*(4.4000)},{\sy*(0.0000)})
	--({\sx*(4.4100)},{\sy*(0.0000)})
	--({\sx*(4.4200)},{\sy*(0.0000)})
	--({\sx*(4.4300)},{\sy*(0.0000)})
	--({\sx*(4.4400)},{\sy*(0.0000)})
	--({\sx*(4.4500)},{\sy*(0.0000)})
	--({\sx*(4.4600)},{\sy*(0.0000)})
	--({\sx*(4.4700)},{\sy*(-0.0000)})
	--({\sx*(4.4800)},{\sy*(-0.0000)})
	--({\sx*(4.4900)},{\sy*(-0.0000)})
	--({\sx*(4.5000)},{\sy*(-0.0000)})
	--({\sx*(4.5100)},{\sy*(-0.0000)})
	--({\sx*(4.5200)},{\sy*(-0.0000)})
	--({\sx*(4.5300)},{\sy*(-0.0000)})
	--({\sx*(4.5400)},{\sy*(-0.0000)})
	--({\sx*(4.5500)},{\sy*(-0.0000)})
	--({\sx*(4.5600)},{\sy*(-0.0000)})
	--({\sx*(4.5700)},{\sy*(-0.0000)})
	--({\sx*(4.5800)},{\sy*(-0.0000)})
	--({\sx*(4.5900)},{\sy*(-0.0000)})
	--({\sx*(4.6000)},{\sy*(-0.0000)})
	--({\sx*(4.6100)},{\sy*(-0.0000)})
	--({\sx*(4.6200)},{\sy*(-0.0000)})
	--({\sx*(4.6300)},{\sy*(-0.0000)})
	--({\sx*(4.6400)},{\sy*(-0.0000)})
	--({\sx*(4.6500)},{\sy*(0.0000)})
	--({\sx*(4.6600)},{\sy*(0.0000)})
	--({\sx*(4.6700)},{\sy*(0.0000)})
	--({\sx*(4.6800)},{\sy*(0.0000)})
	--({\sx*(4.6900)},{\sy*(0.0000)})
	--({\sx*(4.7000)},{\sy*(0.0000)})
	--({\sx*(4.7100)},{\sy*(0.0000)})
	--({\sx*(4.7200)},{\sy*(0.0000)})
	--({\sx*(4.7300)},{\sy*(0.0000)})
	--({\sx*(4.7400)},{\sy*(0.0000)})
	--({\sx*(4.7500)},{\sy*(0.0000)})
	--({\sx*(4.7600)},{\sy*(0.0000)})
	--({\sx*(4.7700)},{\sy*(0.0000)})
	--({\sx*(4.7800)},{\sy*(0.0000)})
	--({\sx*(4.7900)},{\sy*(0.0000)})
	--({\sx*(4.8000)},{\sy*(0.0000)})
	--({\sx*(4.8100)},{\sy*(0.0000)})
	--({\sx*(4.8200)},{\sy*(0.0000)})
	--({\sx*(4.8300)},{\sy*(-0.0000)})
	--({\sx*(4.8400)},{\sy*(-0.0000)})
	--({\sx*(4.8500)},{\sy*(-0.0000)})
	--({\sx*(4.8600)},{\sy*(-0.0000)})
	--({\sx*(4.8700)},{\sy*(-0.0000)})
	--({\sx*(4.8800)},{\sy*(-0.0000)})
	--({\sx*(4.8900)},{\sy*(-0.0000)})
	--({\sx*(4.9000)},{\sy*(-0.0000)})
	--({\sx*(4.9100)},{\sy*(-0.0000)})
	--({\sx*(4.9200)},{\sy*(-0.0000)})
	--({\sx*(4.9300)},{\sy*(-0.0000)})
	--({\sx*(4.9400)},{\sy*(-0.0000)})
	--({\sx*(4.9500)},{\sy*(-0.0000)})
	--({\sx*(4.9600)},{\sy*(-0.0000)})
	--({\sx*(4.9700)},{\sy*(-0.0000)})
	--({\sx*(4.9800)},{\sy*(-0.0000)})
	--({\sx*(4.9900)},{\sy*(-0.0000)})
	--({\sx*(5.0000)},{\sy*(0.0000)});
}
\def\xwerteo{
\fill[color=red] (0.0000,0) circle[radius={0.07/\skala}];
\fill[color=red] (0.1667,0) circle[radius={0.07/\skala}];
\fill[color=red] (0.3333,0) circle[radius={0.07/\skala}];
\fill[color=red] (0.5000,0) circle[radius={0.07/\skala}];
\fill[color=red] (0.6667,0) circle[radius={0.07/\skala}];
\fill[color=red] (0.8333,0) circle[radius={0.07/\skala}];
\fill[color=red] (1.0000,0) circle[radius={0.07/\skala}];
\fill[color=red] (1.1667,0) circle[radius={0.07/\skala}];
\fill[color=red] (1.3333,0) circle[radius={0.07/\skala}];
\fill[color=red] (1.5000,0) circle[radius={0.07/\skala}];
\fill[color=red] (1.6667,0) circle[radius={0.07/\skala}];
\fill[color=red] (1.8333,0) circle[radius={0.07/\skala}];
\fill[color=red] (2.0000,0) circle[radius={0.07/\skala}];
\fill[color=red] (2.1667,0) circle[radius={0.07/\skala}];
\fill[color=red] (2.3333,0) circle[radius={0.07/\skala}];
\fill[color=red] (2.5000,0) circle[radius={0.07/\skala}];
\fill[color=red] (2.6667,0) circle[radius={0.07/\skala}];
\fill[color=red] (2.8333,0) circle[radius={0.07/\skala}];
\fill[color=red] (3.0000,0) circle[radius={0.07/\skala}];
\fill[color=red] (3.1667,0) circle[radius={0.07/\skala}];
\fill[color=red] (3.3333,0) circle[radius={0.07/\skala}];
\fill[color=red] (3.5000,0) circle[radius={0.07/\skala}];
\fill[color=red] (3.6667,0) circle[radius={0.07/\skala}];
\fill[color=red] (3.8333,0) circle[radius={0.07/\skala}];
\fill[color=red] (4.0000,0) circle[radius={0.07/\skala}];
\fill[color=red] (4.1667,0) circle[radius={0.07/\skala}];
\fill[color=red] (4.3333,0) circle[radius={0.07/\skala}];
\fill[color=red] (4.5000,0) circle[radius={0.07/\skala}];
\fill[color=red] (4.6667,0) circle[radius={0.07/\skala}];
\fill[color=red] (4.8333,0) circle[radius={0.07/\skala}];
\fill[color=red] (5.0000,0) circle[radius={0.07/\skala}];
}
\def\punkteo{30}
\def\maxfehlero{5.478\cdot 10^{-11}}
\def\fehlero{
\draw[color=red,line width=1.4pt,line join=round] ({\sx*(0.000)},{\sy*(0.0000)})
	--({\sx*(0.0100)},{\sy*(-0.4336)})
	--({\sx*(0.0200)},{\sy*(0.5004)})
	--({\sx*(0.0300)},{\sy*(0.3769)})
	--({\sx*(0.0400)},{\sy*(-0.0595)})
	--({\sx*(0.0500)},{\sy*(-1.0000)})
	--({\sx*(0.0600)},{\sy*(-0.0624)})
	--({\sx*(0.0700)},{\sy*(-0.0463)})
	--({\sx*(0.0800)},{\sy*(-0.4349)})
	--({\sx*(0.0900)},{\sy*(-0.4246)})
	--({\sx*(0.1000)},{\sy*(-0.0605)})
	--({\sx*(0.1100)},{\sy*(0.0264)})
	--({\sx*(0.1200)},{\sy*(0.0497)})
	--({\sx*(0.1300)},{\sy*(-0.0634)})
	--({\sx*(0.1400)},{\sy*(0.0143)})
	--({\sx*(0.1500)},{\sy*(-0.0045)})
	--({\sx*(0.1600)},{\sy*(-0.0020)})
	--({\sx*(0.1700)},{\sy*(0.0026)})
	--({\sx*(0.1800)},{\sy*(0.0226)})
	--({\sx*(0.1900)},{\sy*(0.0184)})
	--({\sx*(0.2000)},{\sy*(-0.0261)})
	--({\sx*(0.2100)},{\sy*(0.0018)})
	--({\sx*(0.2200)},{\sy*(0.0202)})
	--({\sx*(0.2300)},{\sy*(0.0196)})
	--({\sx*(0.2400)},{\sy*(0.0164)})
	--({\sx*(0.2500)},{\sy*(-0.0053)})
	--({\sx*(0.2600)},{\sy*(0.0340)})
	--({\sx*(0.2700)},{\sy*(0.0157)})
	--({\sx*(0.2800)},{\sy*(-0.0096)})
	--({\sx*(0.2900)},{\sy*(0.0001)})
	--({\sx*(0.3000)},{\sy*(0.0022)})
	--({\sx*(0.3100)},{\sy*(0.0075)})
	--({\sx*(0.3200)},{\sy*(0.0011)})
	--({\sx*(0.3300)},{\sy*(-0.0001)})
	--({\sx*(0.3400)},{\sy*(-0.0009)})
	--({\sx*(0.3500)},{\sy*(-0.0005)})
	--({\sx*(0.3600)},{\sy*(-0.0005)})
	--({\sx*(0.3700)},{\sy*(0.0001)})
	--({\sx*(0.3800)},{\sy*(0.0039)})
	--({\sx*(0.3900)},{\sy*(-0.0004)})
	--({\sx*(0.4000)},{\sy*(0.0007)})
	--({\sx*(0.4100)},{\sy*(-0.0006)})
	--({\sx*(0.4200)},{\sy*(-0.0001)})
	--({\sx*(0.4300)},{\sy*(-0.0007)})
	--({\sx*(0.4400)},{\sy*(0.0008)})
	--({\sx*(0.4500)},{\sy*(0.0010)})
	--({\sx*(0.4600)},{\sy*(-0.0001)})
	--({\sx*(0.4700)},{\sy*(0.0000)})
	--({\sx*(0.4800)},{\sy*(-0.0001)})
	--({\sx*(0.4900)},{\sy*(0.0001)})
	--({\sx*(0.5000)},{\sy*(0.0000)})
	--({\sx*(0.5100)},{\sy*(0.0001)})
	--({\sx*(0.5200)},{\sy*(0.0003)})
	--({\sx*(0.5300)},{\sy*(0.0001)})
	--({\sx*(0.5400)},{\sy*(0.0004)})
	--({\sx*(0.5500)},{\sy*(-0.0001)})
	--({\sx*(0.5600)},{\sy*(0.0004)})
	--({\sx*(0.5700)},{\sy*(0.0002)})
	--({\sx*(0.5800)},{\sy*(0.0001)})
	--({\sx*(0.5900)},{\sy*(-0.0000)})
	--({\sx*(0.6000)},{\sy*(0.0003)})
	--({\sx*(0.6100)},{\sy*(0.0002)})
	--({\sx*(0.6200)},{\sy*(-0.0000)})
	--({\sx*(0.6300)},{\sy*(-0.0000)})
	--({\sx*(0.6400)},{\sy*(0.0002)})
	--({\sx*(0.6500)},{\sy*(0.0001)})
	--({\sx*(0.6600)},{\sy*(0.0000)})
	--({\sx*(0.6700)},{\sy*(0.0000)})
	--({\sx*(0.6800)},{\sy*(-0.0000)})
	--({\sx*(0.6900)},{\sy*(-0.0001)})
	--({\sx*(0.7000)},{\sy*(-0.0001)})
	--({\sx*(0.7100)},{\sy*(-0.0000)})
	--({\sx*(0.7200)},{\sy*(0.0001)})
	--({\sx*(0.7300)},{\sy*(0.0000)})
	--({\sx*(0.7400)},{\sy*(-0.0001)})
	--({\sx*(0.7500)},{\sy*(-0.0001)})
	--({\sx*(0.7600)},{\sy*(0.0000)})
	--({\sx*(0.7700)},{\sy*(0.0000)})
	--({\sx*(0.7800)},{\sy*(0.0000)})
	--({\sx*(0.7900)},{\sy*(-0.0000)})
	--({\sx*(0.8000)},{\sy*(0.0000)})
	--({\sx*(0.8100)},{\sy*(0.0000)})
	--({\sx*(0.8200)},{\sy*(-0.0000)})
	--({\sx*(0.8300)},{\sy*(-0.0000)})
	--({\sx*(0.8400)},{\sy*(0.0000)})
	--({\sx*(0.8500)},{\sy*(-0.0000)})
	--({\sx*(0.8600)},{\sy*(-0.0000)})
	--({\sx*(0.8700)},{\sy*(0.0000)})
	--({\sx*(0.8800)},{\sy*(-0.0000)})
	--({\sx*(0.8900)},{\sy*(0.0000)})
	--({\sx*(0.9000)},{\sy*(0.0000)})
	--({\sx*(0.9100)},{\sy*(-0.0000)})
	--({\sx*(0.9200)},{\sy*(0.0000)})
	--({\sx*(0.9300)},{\sy*(-0.0000)})
	--({\sx*(0.9400)},{\sy*(-0.0000)})
	--({\sx*(0.9500)},{\sy*(-0.0000)})
	--({\sx*(0.9600)},{\sy*(0.0000)})
	--({\sx*(0.9700)},{\sy*(0.0000)})
	--({\sx*(0.9800)},{\sy*(-0.0000)})
	--({\sx*(0.9900)},{\sy*(-0.0000)})
	--({\sx*(1.0000)},{\sy*(0.0000)})
	--({\sx*(1.0100)},{\sy*(0.0000)})
	--({\sx*(1.0200)},{\sy*(0.0000)})
	--({\sx*(1.0300)},{\sy*(-0.0000)})
	--({\sx*(1.0400)},{\sy*(0.0000)})
	--({\sx*(1.0500)},{\sy*(0.0000)})
	--({\sx*(1.0600)},{\sy*(0.0000)})
	--({\sx*(1.0700)},{\sy*(-0.0000)})
	--({\sx*(1.0800)},{\sy*(-0.0000)})
	--({\sx*(1.0900)},{\sy*(0.0000)})
	--({\sx*(1.1000)},{\sy*(0.0000)})
	--({\sx*(1.1100)},{\sy*(-0.0000)})
	--({\sx*(1.1200)},{\sy*(0.0000)})
	--({\sx*(1.1300)},{\sy*(-0.0000)})
	--({\sx*(1.1400)},{\sy*(0.0000)})
	--({\sx*(1.1500)},{\sy*(-0.0000)})
	--({\sx*(1.1600)},{\sy*(-0.0000)})
	--({\sx*(1.1700)},{\sy*(-0.0000)})
	--({\sx*(1.1800)},{\sy*(0.0000)})
	--({\sx*(1.1900)},{\sy*(-0.0000)})
	--({\sx*(1.2000)},{\sy*(0.0000)})
	--({\sx*(1.2100)},{\sy*(-0.0000)})
	--({\sx*(1.2200)},{\sy*(-0.0000)})
	--({\sx*(1.2300)},{\sy*(-0.0000)})
	--({\sx*(1.2400)},{\sy*(-0.0000)})
	--({\sx*(1.2500)},{\sy*(-0.0000)})
	--({\sx*(1.2600)},{\sy*(0.0000)})
	--({\sx*(1.2700)},{\sy*(-0.0000)})
	--({\sx*(1.2800)},{\sy*(-0.0000)})
	--({\sx*(1.2900)},{\sy*(-0.0000)})
	--({\sx*(1.3000)},{\sy*(0.0000)})
	--({\sx*(1.3100)},{\sy*(-0.0000)})
	--({\sx*(1.3200)},{\sy*(-0.0000)})
	--({\sx*(1.3300)},{\sy*(-0.0000)})
	--({\sx*(1.3400)},{\sy*(0.0000)})
	--({\sx*(1.3500)},{\sy*(0.0000)})
	--({\sx*(1.3600)},{\sy*(-0.0000)})
	--({\sx*(1.3700)},{\sy*(-0.0000)})
	--({\sx*(1.3800)},{\sy*(-0.0000)})
	--({\sx*(1.3900)},{\sy*(-0.0000)})
	--({\sx*(1.4000)},{\sy*(-0.0000)})
	--({\sx*(1.4100)},{\sy*(0.0000)})
	--({\sx*(1.4200)},{\sy*(-0.0000)})
	--({\sx*(1.4300)},{\sy*(0.0000)})
	--({\sx*(1.4400)},{\sy*(-0.0000)})
	--({\sx*(1.4500)},{\sy*(-0.0000)})
	--({\sx*(1.4600)},{\sy*(-0.0000)})
	--({\sx*(1.4700)},{\sy*(0.0000)})
	--({\sx*(1.4800)},{\sy*(0.0000)})
	--({\sx*(1.4900)},{\sy*(-0.0000)})
	--({\sx*(1.5000)},{\sy*(0.0000)})
	--({\sx*(1.5100)},{\sy*(0.0000)})
	--({\sx*(1.5200)},{\sy*(0.0000)})
	--({\sx*(1.5300)},{\sy*(0.0000)})
	--({\sx*(1.5400)},{\sy*(0.0000)})
	--({\sx*(1.5500)},{\sy*(0.0000)})
	--({\sx*(1.5600)},{\sy*(0.0000)})
	--({\sx*(1.5700)},{\sy*(-0.0000)})
	--({\sx*(1.5800)},{\sy*(0.0000)})
	--({\sx*(1.5900)},{\sy*(0.0000)})
	--({\sx*(1.6000)},{\sy*(0.0000)})
	--({\sx*(1.6100)},{\sy*(-0.0000)})
	--({\sx*(1.6200)},{\sy*(0.0000)})
	--({\sx*(1.6300)},{\sy*(0.0000)})
	--({\sx*(1.6400)},{\sy*(0.0000)})
	--({\sx*(1.6500)},{\sy*(-0.0000)})
	--({\sx*(1.6600)},{\sy*(-0.0000)})
	--({\sx*(1.6700)},{\sy*(0.0000)})
	--({\sx*(1.6800)},{\sy*(0.0000)})
	--({\sx*(1.6900)},{\sy*(-0.0000)})
	--({\sx*(1.7000)},{\sy*(-0.0000)})
	--({\sx*(1.7100)},{\sy*(0.0000)})
	--({\sx*(1.7200)},{\sy*(0.0000)})
	--({\sx*(1.7300)},{\sy*(0.0000)})
	--({\sx*(1.7400)},{\sy*(-0.0000)})
	--({\sx*(1.7500)},{\sy*(-0.0000)})
	--({\sx*(1.7600)},{\sy*(-0.0000)})
	--({\sx*(1.7700)},{\sy*(0.0000)})
	--({\sx*(1.7800)},{\sy*(-0.0000)})
	--({\sx*(1.7900)},{\sy*(-0.0000)})
	--({\sx*(1.8000)},{\sy*(0.0000)})
	--({\sx*(1.8100)},{\sy*(-0.0000)})
	--({\sx*(1.8200)},{\sy*(-0.0000)})
	--({\sx*(1.8300)},{\sy*(-0.0000)})
	--({\sx*(1.8400)},{\sy*(-0.0000)})
	--({\sx*(1.8500)},{\sy*(0.0000)})
	--({\sx*(1.8600)},{\sy*(-0.0000)})
	--({\sx*(1.8700)},{\sy*(-0.0000)})
	--({\sx*(1.8800)},{\sy*(-0.0000)})
	--({\sx*(1.8900)},{\sy*(-0.0000)})
	--({\sx*(1.9000)},{\sy*(-0.0000)})
	--({\sx*(1.9100)},{\sy*(-0.0000)})
	--({\sx*(1.9200)},{\sy*(0.0000)})
	--({\sx*(1.9300)},{\sy*(0.0000)})
	--({\sx*(1.9400)},{\sy*(0.0000)})
	--({\sx*(1.9500)},{\sy*(-0.0000)})
	--({\sx*(1.9600)},{\sy*(-0.0000)})
	--({\sx*(1.9700)},{\sy*(0.0000)})
	--({\sx*(1.9800)},{\sy*(0.0000)})
	--({\sx*(1.9900)},{\sy*(-0.0000)})
	--({\sx*(2.0000)},{\sy*(0.0000)})
	--({\sx*(2.0100)},{\sy*(0.0000)})
	--({\sx*(2.0200)},{\sy*(-0.0000)})
	--({\sx*(2.0300)},{\sy*(0.0000)})
	--({\sx*(2.0400)},{\sy*(-0.0000)})
	--({\sx*(2.0500)},{\sy*(0.0000)})
	--({\sx*(2.0600)},{\sy*(-0.0000)})
	--({\sx*(2.0700)},{\sy*(0.0000)})
	--({\sx*(2.0800)},{\sy*(0.0000)})
	--({\sx*(2.0900)},{\sy*(0.0000)})
	--({\sx*(2.1000)},{\sy*(0.0000)})
	--({\sx*(2.1100)},{\sy*(-0.0000)})
	--({\sx*(2.1200)},{\sy*(0.0000)})
	--({\sx*(2.1300)},{\sy*(0.0000)})
	--({\sx*(2.1400)},{\sy*(-0.0000)})
	--({\sx*(2.1500)},{\sy*(0.0000)})
	--({\sx*(2.1600)},{\sy*(0.0000)})
	--({\sx*(2.1700)},{\sy*(0.0000)})
	--({\sx*(2.1800)},{\sy*(0.0000)})
	--({\sx*(2.1900)},{\sy*(-0.0000)})
	--({\sx*(2.2000)},{\sy*(0.0000)})
	--({\sx*(2.2100)},{\sy*(0.0000)})
	--({\sx*(2.2200)},{\sy*(0.0000)})
	--({\sx*(2.2300)},{\sy*(0.0000)})
	--({\sx*(2.2400)},{\sy*(0.0000)})
	--({\sx*(2.2500)},{\sy*(0.0000)})
	--({\sx*(2.2600)},{\sy*(0.0000)})
	--({\sx*(2.2700)},{\sy*(-0.0000)})
	--({\sx*(2.2800)},{\sy*(0.0000)})
	--({\sx*(2.2900)},{\sy*(0.0000)})
	--({\sx*(2.3000)},{\sy*(-0.0000)})
	--({\sx*(2.3100)},{\sy*(-0.0000)})
	--({\sx*(2.3200)},{\sy*(-0.0000)})
	--({\sx*(2.3300)},{\sy*(0.0000)})
	--({\sx*(2.3400)},{\sy*(-0.0000)})
	--({\sx*(2.3500)},{\sy*(0.0000)})
	--({\sx*(2.3600)},{\sy*(0.0000)})
	--({\sx*(2.3700)},{\sy*(-0.0000)})
	--({\sx*(2.3800)},{\sy*(0.0000)})
	--({\sx*(2.3900)},{\sy*(-0.0000)})
	--({\sx*(2.4000)},{\sy*(0.0000)})
	--({\sx*(2.4100)},{\sy*(-0.0000)})
	--({\sx*(2.4200)},{\sy*(0.0000)})
	--({\sx*(2.4300)},{\sy*(0.0000)})
	--({\sx*(2.4400)},{\sy*(-0.0000)})
	--({\sx*(2.4500)},{\sy*(-0.0000)})
	--({\sx*(2.4600)},{\sy*(-0.0000)})
	--({\sx*(2.4700)},{\sy*(0.0000)})
	--({\sx*(2.4800)},{\sy*(0.0000)})
	--({\sx*(2.4900)},{\sy*(0.0000)})
	--({\sx*(2.5000)},{\sy*(0.0000)})
	--({\sx*(2.5100)},{\sy*(-0.0000)})
	--({\sx*(2.5200)},{\sy*(0.0000)})
	--({\sx*(2.5300)},{\sy*(0.0000)})
	--({\sx*(2.5400)},{\sy*(0.0000)})
	--({\sx*(2.5500)},{\sy*(-0.0000)})
	--({\sx*(2.5600)},{\sy*(0.0000)})
	--({\sx*(2.5700)},{\sy*(0.0000)})
	--({\sx*(2.5800)},{\sy*(0.0000)})
	--({\sx*(2.5900)},{\sy*(0.0000)})
	--({\sx*(2.6000)},{\sy*(0.0000)})
	--({\sx*(2.6100)},{\sy*(0.0000)})
	--({\sx*(2.6200)},{\sy*(-0.0000)})
	--({\sx*(2.6300)},{\sy*(0.0000)})
	--({\sx*(2.6400)},{\sy*(0.0000)})
	--({\sx*(2.6500)},{\sy*(0.0000)})
	--({\sx*(2.6600)},{\sy*(-0.0000)})
	--({\sx*(2.6700)},{\sy*(0.0000)})
	--({\sx*(2.6800)},{\sy*(0.0000)})
	--({\sx*(2.6900)},{\sy*(0.0000)})
	--({\sx*(2.7000)},{\sy*(0.0000)})
	--({\sx*(2.7100)},{\sy*(0.0000)})
	--({\sx*(2.7200)},{\sy*(0.0000)})
	--({\sx*(2.7300)},{\sy*(0.0000)})
	--({\sx*(2.7400)},{\sy*(-0.0000)})
	--({\sx*(2.7500)},{\sy*(-0.0000)})
	--({\sx*(2.7600)},{\sy*(0.0000)})
	--({\sx*(2.7700)},{\sy*(0.0000)})
	--({\sx*(2.7800)},{\sy*(-0.0000)})
	--({\sx*(2.7900)},{\sy*(-0.0000)})
	--({\sx*(2.8000)},{\sy*(0.0000)})
	--({\sx*(2.8100)},{\sy*(-0.0000)})
	--({\sx*(2.8200)},{\sy*(-0.0000)})
	--({\sx*(2.8300)},{\sy*(-0.0000)})
	--({\sx*(2.8400)},{\sy*(-0.0000)})
	--({\sx*(2.8500)},{\sy*(-0.0000)})
	--({\sx*(2.8600)},{\sy*(-0.0000)})
	--({\sx*(2.8700)},{\sy*(0.0000)})
	--({\sx*(2.8800)},{\sy*(-0.0000)})
	--({\sx*(2.8900)},{\sy*(-0.0000)})
	--({\sx*(2.9000)},{\sy*(0.0000)})
	--({\sx*(2.9100)},{\sy*(-0.0000)})
	--({\sx*(2.9200)},{\sy*(-0.0000)})
	--({\sx*(2.9300)},{\sy*(-0.0000)})
	--({\sx*(2.9400)},{\sy*(-0.0000)})
	--({\sx*(2.9500)},{\sy*(-0.0000)})
	--({\sx*(2.9600)},{\sy*(-0.0000)})
	--({\sx*(2.9700)},{\sy*(-0.0000)})
	--({\sx*(2.9800)},{\sy*(-0.0000)})
	--({\sx*(2.9900)},{\sy*(0.0000)})
	--({\sx*(3.0000)},{\sy*(0.0000)})
	--({\sx*(3.0100)},{\sy*(0.0000)})
	--({\sx*(3.0200)},{\sy*(0.0000)})
	--({\sx*(3.0300)},{\sy*(0.0000)})
	--({\sx*(3.0400)},{\sy*(0.0000)})
	--({\sx*(3.0500)},{\sy*(0.0000)})
	--({\sx*(3.0600)},{\sy*(0.0000)})
	--({\sx*(3.0700)},{\sy*(0.0000)})
	--({\sx*(3.0800)},{\sy*(0.0000)})
	--({\sx*(3.0900)},{\sy*(0.0000)})
	--({\sx*(3.1000)},{\sy*(0.0000)})
	--({\sx*(3.1100)},{\sy*(0.0000)})
	--({\sx*(3.1200)},{\sy*(0.0000)})
	--({\sx*(3.1300)},{\sy*(0.0000)})
	--({\sx*(3.1400)},{\sy*(0.0000)})
	--({\sx*(3.1500)},{\sy*(0.0000)})
	--({\sx*(3.1600)},{\sy*(0.0000)})
	--({\sx*(3.1700)},{\sy*(0.0000)})
	--({\sx*(3.1800)},{\sy*(0.0000)})
	--({\sx*(3.1900)},{\sy*(-0.0000)})
	--({\sx*(3.2000)},{\sy*(-0.0000)})
	--({\sx*(3.2100)},{\sy*(0.0000)})
	--({\sx*(3.2200)},{\sy*(-0.0000)})
	--({\sx*(3.2300)},{\sy*(0.0000)})
	--({\sx*(3.2400)},{\sy*(0.0000)})
	--({\sx*(3.2500)},{\sy*(-0.0000)})
	--({\sx*(3.2600)},{\sy*(0.0000)})
	--({\sx*(3.2700)},{\sy*(-0.0000)})
	--({\sx*(3.2800)},{\sy*(0.0000)})
	--({\sx*(3.2900)},{\sy*(-0.0000)})
	--({\sx*(3.3000)},{\sy*(0.0000)})
	--({\sx*(3.3100)},{\sy*(0.0000)})
	--({\sx*(3.3200)},{\sy*(-0.0000)})
	--({\sx*(3.3300)},{\sy*(-0.0000)})
	--({\sx*(3.3400)},{\sy*(0.0000)})
	--({\sx*(3.3500)},{\sy*(0.0000)})
	--({\sx*(3.3600)},{\sy*(0.0000)})
	--({\sx*(3.3700)},{\sy*(0.0000)})
	--({\sx*(3.3800)},{\sy*(0.0000)})
	--({\sx*(3.3900)},{\sy*(-0.0000)})
	--({\sx*(3.4000)},{\sy*(0.0000)})
	--({\sx*(3.4100)},{\sy*(0.0000)})
	--({\sx*(3.4200)},{\sy*(0.0000)})
	--({\sx*(3.4300)},{\sy*(0.0000)})
	--({\sx*(3.4400)},{\sy*(-0.0000)})
	--({\sx*(3.4500)},{\sy*(-0.0000)})
	--({\sx*(3.4600)},{\sy*(0.0000)})
	--({\sx*(3.4700)},{\sy*(0.0000)})
	--({\sx*(3.4800)},{\sy*(-0.0000)})
	--({\sx*(3.4900)},{\sy*(0.0000)})
	--({\sx*(3.5000)},{\sy*(0.0000)})
	--({\sx*(3.5100)},{\sy*(-0.0000)})
	--({\sx*(3.5200)},{\sy*(-0.0000)})
	--({\sx*(3.5300)},{\sy*(0.0000)})
	--({\sx*(3.5400)},{\sy*(0.0000)})
	--({\sx*(3.5500)},{\sy*(-0.0000)})
	--({\sx*(3.5600)},{\sy*(-0.0000)})
	--({\sx*(3.5700)},{\sy*(0.0000)})
	--({\sx*(3.5800)},{\sy*(-0.0000)})
	--({\sx*(3.5900)},{\sy*(-0.0000)})
	--({\sx*(3.6000)},{\sy*(0.0000)})
	--({\sx*(3.6100)},{\sy*(0.0000)})
	--({\sx*(3.6200)},{\sy*(0.0000)})
	--({\sx*(3.6300)},{\sy*(0.0000)})
	--({\sx*(3.6400)},{\sy*(-0.0000)})
	--({\sx*(3.6500)},{\sy*(0.0000)})
	--({\sx*(3.6600)},{\sy*(0.0000)})
	--({\sx*(3.6700)},{\sy*(-0.0000)})
	--({\sx*(3.6800)},{\sy*(0.0000)})
	--({\sx*(3.6900)},{\sy*(0.0000)})
	--({\sx*(3.7000)},{\sy*(-0.0000)})
	--({\sx*(3.7100)},{\sy*(0.0000)})
	--({\sx*(3.7200)},{\sy*(-0.0000)})
	--({\sx*(3.7300)},{\sy*(0.0000)})
	--({\sx*(3.7400)},{\sy*(0.0000)})
	--({\sx*(3.7500)},{\sy*(0.0000)})
	--({\sx*(3.7600)},{\sy*(-0.0000)})
	--({\sx*(3.7700)},{\sy*(0.0000)})
	--({\sx*(3.7800)},{\sy*(-0.0000)})
	--({\sx*(3.7900)},{\sy*(-0.0000)})
	--({\sx*(3.8000)},{\sy*(-0.0000)})
	--({\sx*(3.8100)},{\sy*(0.0000)})
	--({\sx*(3.8200)},{\sy*(-0.0000)})
	--({\sx*(3.8300)},{\sy*(-0.0000)})
	--({\sx*(3.8400)},{\sy*(0.0000)})
	--({\sx*(3.8500)},{\sy*(-0.0000)})
	--({\sx*(3.8600)},{\sy*(0.0000)})
	--({\sx*(3.8700)},{\sy*(-0.0000)})
	--({\sx*(3.8800)},{\sy*(-0.0000)})
	--({\sx*(3.8900)},{\sy*(0.0000)})
	--({\sx*(3.9000)},{\sy*(0.0000)})
	--({\sx*(3.9100)},{\sy*(0.0000)})
	--({\sx*(3.9200)},{\sy*(0.0000)})
	--({\sx*(3.9300)},{\sy*(0.0000)})
	--({\sx*(3.9400)},{\sy*(-0.0000)})
	--({\sx*(3.9500)},{\sy*(-0.0000)})
	--({\sx*(3.9600)},{\sy*(0.0000)})
	--({\sx*(3.9700)},{\sy*(0.0000)})
	--({\sx*(3.9800)},{\sy*(-0.0000)})
	--({\sx*(3.9900)},{\sy*(0.0000)})
	--({\sx*(4.0000)},{\sy*(0.0000)})
	--({\sx*(4.0100)},{\sy*(0.0000)})
	--({\sx*(4.0200)},{\sy*(0.0000)})
	--({\sx*(4.0300)},{\sy*(0.0000)})
	--({\sx*(4.0400)},{\sy*(0.0000)})
	--({\sx*(4.0500)},{\sy*(0.0000)})
	--({\sx*(4.0600)},{\sy*(-0.0000)})
	--({\sx*(4.0700)},{\sy*(0.0000)})
	--({\sx*(4.0800)},{\sy*(0.0000)})
	--({\sx*(4.0900)},{\sy*(-0.0000)})
	--({\sx*(4.1000)},{\sy*(-0.0000)})
	--({\sx*(4.1100)},{\sy*(0.0000)})
	--({\sx*(4.1200)},{\sy*(-0.0000)})
	--({\sx*(4.1300)},{\sy*(0.0000)})
	--({\sx*(4.1400)},{\sy*(-0.0000)})
	--({\sx*(4.1500)},{\sy*(-0.0000)})
	--({\sx*(4.1600)},{\sy*(0.0000)})
	--({\sx*(4.1700)},{\sy*(0.0000)})
	--({\sx*(4.1800)},{\sy*(0.0000)})
	--({\sx*(4.1900)},{\sy*(-0.0000)})
	--({\sx*(4.2000)},{\sy*(-0.0000)})
	--({\sx*(4.2100)},{\sy*(-0.0000)})
	--({\sx*(4.2200)},{\sy*(0.0000)})
	--({\sx*(4.2300)},{\sy*(-0.0000)})
	--({\sx*(4.2400)},{\sy*(-0.0000)})
	--({\sx*(4.2500)},{\sy*(-0.0000)})
	--({\sx*(4.2600)},{\sy*(0.0000)})
	--({\sx*(4.2700)},{\sy*(0.0000)})
	--({\sx*(4.2800)},{\sy*(-0.0000)})
	--({\sx*(4.2900)},{\sy*(0.0000)})
	--({\sx*(4.3000)},{\sy*(0.0000)})
	--({\sx*(4.3100)},{\sy*(-0.0000)})
	--({\sx*(4.3200)},{\sy*(-0.0000)})
	--({\sx*(4.3300)},{\sy*(0.0000)})
	--({\sx*(4.3400)},{\sy*(0.0000)})
	--({\sx*(4.3500)},{\sy*(-0.0000)})
	--({\sx*(4.3600)},{\sy*(0.0000)})
	--({\sx*(4.3700)},{\sy*(0.0000)})
	--({\sx*(4.3800)},{\sy*(0.0000)})
	--({\sx*(4.3900)},{\sy*(0.0000)})
	--({\sx*(4.4000)},{\sy*(0.0000)})
	--({\sx*(4.4100)},{\sy*(0.0001)})
	--({\sx*(4.4200)},{\sy*(0.0001)})
	--({\sx*(4.4300)},{\sy*(-0.0001)})
	--({\sx*(4.4400)},{\sy*(0.0001)})
	--({\sx*(4.4500)},{\sy*(0.0001)})
	--({\sx*(4.4600)},{\sy*(0.0002)})
	--({\sx*(4.4700)},{\sy*(0.0002)})
	--({\sx*(4.4800)},{\sy*(0.0002)})
	--({\sx*(4.4900)},{\sy*(0.0001)})
	--({\sx*(4.5000)},{\sy*(0.0000)})
	--({\sx*(4.5100)},{\sy*(-0.0000)})
	--({\sx*(4.5200)},{\sy*(0.0000)})
	--({\sx*(4.5300)},{\sy*(-0.0003)})
	--({\sx*(4.5400)},{\sy*(-0.0006)})
	--({\sx*(4.5500)},{\sy*(0.0003)})
	--({\sx*(4.5600)},{\sy*(-0.0016)})
	--({\sx*(4.5700)},{\sy*(-0.0002)})
	--({\sx*(4.5800)},{\sy*(-0.0003)})
	--({\sx*(4.5900)},{\sy*(-0.0028)})
	--({\sx*(4.6000)},{\sy*(-0.0025)})
	--({\sx*(4.6100)},{\sy*(-0.0032)})
	--({\sx*(4.6200)},{\sy*(0.0014)})
	--({\sx*(4.6300)},{\sy*(-0.0002)})
	--({\sx*(4.6400)},{\sy*(-0.0009)})
	--({\sx*(4.6500)},{\sy*(-0.0002)})
	--({\sx*(4.6600)},{\sy*(-0.0001)})
	--({\sx*(4.6700)},{\sy*(-0.0002)})
	--({\sx*(4.6800)},{\sy*(0.0010)})
	--({\sx*(4.6900)},{\sy*(0.0003)})
	--({\sx*(4.7000)},{\sy*(-0.0023)})
	--({\sx*(4.7100)},{\sy*(-0.0035)})
	--({\sx*(4.7200)},{\sy*(0.0007)})
	--({\sx*(4.7300)},{\sy*(-0.0009)})
	--({\sx*(4.7400)},{\sy*(-0.0095)})
	--({\sx*(4.7500)},{\sy*(0.0002)})
	--({\sx*(4.7600)},{\sy*(0.0250)})
	--({\sx*(4.7700)},{\sy*(-0.0203)})
	--({\sx*(4.7800)},{\sy*(-0.0219)})
	--({\sx*(4.7900)},{\sy*(0.0132)})
	--({\sx*(4.8000)},{\sy*(0.0226)})
	--({\sx*(4.8100)},{\sy*(0.0009)})
	--({\sx*(4.8200)},{\sy*(-0.0040)})
	--({\sx*(4.8300)},{\sy*(-0.0002)})
	--({\sx*(4.8400)},{\sy*(0.0018)})
	--({\sx*(4.8500)},{\sy*(-0.0082)})
	--({\sx*(4.8600)},{\sy*(-0.0045)})
	--({\sx*(4.8700)},{\sy*(-0.0160)})
	--({\sx*(4.8800)},{\sy*(-0.0454)})
	--({\sx*(4.8900)},{\sy*(-0.0144)})
	--({\sx*(4.9000)},{\sy*(-0.1627)})
	--({\sx*(4.9100)},{\sy*(0.1618)})
	--({\sx*(4.9200)},{\sy*(-0.0022)})
	--({\sx*(4.9300)},{\sy*(0.0702)})
	--({\sx*(4.9400)},{\sy*(0.5152)})
	--({\sx*(4.9500)},{\sy*(0.0305)})
	--({\sx*(4.9600)},{\sy*(-0.4471)})
	--({\sx*(4.9700)},{\sy*(-0.3661)})
	--({\sx*(4.9800)},{\sy*(0.4591)})
	--({\sx*(4.9900)},{\sy*(0.2828)})
	--({\sx*(5.0000)},{\sy*(0.0000)});
}
\def\relfehlero{
\draw[color=blue,line width=1.4pt,line join=round] ({\sx*(0.000)},{\sy*(0.0000)})
	--({\sx*(0.0100)},{\sy*(-0.0000)})
	--({\sx*(0.0200)},{\sy*(0.0000)})
	--({\sx*(0.0300)},{\sy*(0.0000)})
	--({\sx*(0.0400)},{\sy*(-0.0000)})
	--({\sx*(0.0500)},{\sy*(-0.0000)})
	--({\sx*(0.0600)},{\sy*(-0.0000)})
	--({\sx*(0.0700)},{\sy*(-0.0000)})
	--({\sx*(0.0800)},{\sy*(-0.0000)})
	--({\sx*(0.0900)},{\sy*(-0.0000)})
	--({\sx*(0.1000)},{\sy*(-0.0000)})
	--({\sx*(0.1100)},{\sy*(0.0000)})
	--({\sx*(0.1200)},{\sy*(0.0000)})
	--({\sx*(0.1300)},{\sy*(-0.0000)})
	--({\sx*(0.1400)},{\sy*(0.0000)})
	--({\sx*(0.1500)},{\sy*(-0.0000)})
	--({\sx*(0.1600)},{\sy*(-0.0000)})
	--({\sx*(0.1700)},{\sy*(0.0000)})
	--({\sx*(0.1800)},{\sy*(0.0000)})
	--({\sx*(0.1900)},{\sy*(0.0000)})
	--({\sx*(0.2000)},{\sy*(-0.0000)})
	--({\sx*(0.2100)},{\sy*(0.0000)})
	--({\sx*(0.2200)},{\sy*(0.0000)})
	--({\sx*(0.2300)},{\sy*(0.0000)})
	--({\sx*(0.2400)},{\sy*(0.0000)})
	--({\sx*(0.2500)},{\sy*(-0.0000)})
	--({\sx*(0.2600)},{\sy*(0.0000)})
	--({\sx*(0.2700)},{\sy*(0.0000)})
	--({\sx*(0.2800)},{\sy*(-0.0000)})
	--({\sx*(0.2900)},{\sy*(0.0000)})
	--({\sx*(0.3000)},{\sy*(0.0000)})
	--({\sx*(0.3100)},{\sy*(0.0000)})
	--({\sx*(0.3200)},{\sy*(0.0000)})
	--({\sx*(0.3300)},{\sy*(-0.0000)})
	--({\sx*(0.3400)},{\sy*(-0.0000)})
	--({\sx*(0.3500)},{\sy*(-0.0000)})
	--({\sx*(0.3600)},{\sy*(-0.0000)})
	--({\sx*(0.3700)},{\sy*(0.0000)})
	--({\sx*(0.3800)},{\sy*(0.0000)})
	--({\sx*(0.3900)},{\sy*(-0.0000)})
	--({\sx*(0.4000)},{\sy*(0.0000)})
	--({\sx*(0.4100)},{\sy*(-0.0000)})
	--({\sx*(0.4200)},{\sy*(-0.0000)})
	--({\sx*(0.4300)},{\sy*(-0.0000)})
	--({\sx*(0.4400)},{\sy*(0.0000)})
	--({\sx*(0.4500)},{\sy*(0.0000)})
	--({\sx*(0.4600)},{\sy*(-0.0000)})
	--({\sx*(0.4700)},{\sy*(0.0000)})
	--({\sx*(0.4800)},{\sy*(-0.0000)})
	--({\sx*(0.4900)},{\sy*(0.0000)})
	--({\sx*(0.5000)},{\sy*(0.0000)})
	--({\sx*(0.5100)},{\sy*(0.0000)})
	--({\sx*(0.5200)},{\sy*(0.0000)})
	--({\sx*(0.5300)},{\sy*(0.0000)})
	--({\sx*(0.5400)},{\sy*(0.0000)})
	--({\sx*(0.5500)},{\sy*(-0.0000)})
	--({\sx*(0.5600)},{\sy*(0.0000)})
	--({\sx*(0.5700)},{\sy*(0.0000)})
	--({\sx*(0.5800)},{\sy*(0.0000)})
	--({\sx*(0.5900)},{\sy*(-0.0000)})
	--({\sx*(0.6000)},{\sy*(0.0000)})
	--({\sx*(0.6100)},{\sy*(0.0000)})
	--({\sx*(0.6200)},{\sy*(-0.0000)})
	--({\sx*(0.6300)},{\sy*(-0.0000)})
	--({\sx*(0.6400)},{\sy*(0.0000)})
	--({\sx*(0.6500)},{\sy*(0.0000)})
	--({\sx*(0.6600)},{\sy*(0.0000)})
	--({\sx*(0.6700)},{\sy*(0.0000)})
	--({\sx*(0.6800)},{\sy*(-0.0000)})
	--({\sx*(0.6900)},{\sy*(-0.0000)})
	--({\sx*(0.7000)},{\sy*(-0.0000)})
	--({\sx*(0.7100)},{\sy*(-0.0000)})
	--({\sx*(0.7200)},{\sy*(0.0000)})
	--({\sx*(0.7300)},{\sy*(0.0000)})
	--({\sx*(0.7400)},{\sy*(-0.0000)})
	--({\sx*(0.7500)},{\sy*(-0.0000)})
	--({\sx*(0.7600)},{\sy*(0.0000)})
	--({\sx*(0.7700)},{\sy*(0.0000)})
	--({\sx*(0.7800)},{\sy*(0.0000)})
	--({\sx*(0.7900)},{\sy*(-0.0000)})
	--({\sx*(0.8000)},{\sy*(0.0000)})
	--({\sx*(0.8100)},{\sy*(0.0000)})
	--({\sx*(0.8200)},{\sy*(-0.0000)})
	--({\sx*(0.8300)},{\sy*(-0.0000)})
	--({\sx*(0.8400)},{\sy*(0.0000)})
	--({\sx*(0.8500)},{\sy*(-0.0000)})
	--({\sx*(0.8600)},{\sy*(-0.0000)})
	--({\sx*(0.8700)},{\sy*(0.0000)})
	--({\sx*(0.8800)},{\sy*(-0.0000)})
	--({\sx*(0.8900)},{\sy*(0.0000)})
	--({\sx*(0.9000)},{\sy*(0.0000)})
	--({\sx*(0.9100)},{\sy*(-0.0000)})
	--({\sx*(0.9200)},{\sy*(0.0000)})
	--({\sx*(0.9300)},{\sy*(-0.0000)})
	--({\sx*(0.9400)},{\sy*(-0.0000)})
	--({\sx*(0.9500)},{\sy*(-0.0000)})
	--({\sx*(0.9600)},{\sy*(0.0000)})
	--({\sx*(0.9700)},{\sy*(0.0000)})
	--({\sx*(0.9800)},{\sy*(-0.0000)})
	--({\sx*(0.9900)},{\sy*(-0.0000)})
	--({\sx*(1.0000)},{\sy*(0.0000)})
	--({\sx*(1.0100)},{\sy*(0.0000)})
	--({\sx*(1.0200)},{\sy*(0.0000)})
	--({\sx*(1.0300)},{\sy*(-0.0000)})
	--({\sx*(1.0400)},{\sy*(0.0000)})
	--({\sx*(1.0500)},{\sy*(0.0000)})
	--({\sx*(1.0600)},{\sy*(0.0000)})
	--({\sx*(1.0700)},{\sy*(-0.0000)})
	--({\sx*(1.0800)},{\sy*(-0.0000)})
	--({\sx*(1.0900)},{\sy*(0.0000)})
	--({\sx*(1.1000)},{\sy*(0.0000)})
	--({\sx*(1.1100)},{\sy*(-0.0000)})
	--({\sx*(1.1200)},{\sy*(0.0000)})
	--({\sx*(1.1300)},{\sy*(-0.0000)})
	--({\sx*(1.1400)},{\sy*(0.0000)})
	--({\sx*(1.1500)},{\sy*(-0.0000)})
	--({\sx*(1.1600)},{\sy*(-0.0000)})
	--({\sx*(1.1700)},{\sy*(-0.0000)})
	--({\sx*(1.1800)},{\sy*(0.0000)})
	--({\sx*(1.1900)},{\sy*(-0.0000)})
	--({\sx*(1.2000)},{\sy*(0.0000)})
	--({\sx*(1.2100)},{\sy*(-0.0000)})
	--({\sx*(1.2200)},{\sy*(-0.0000)})
	--({\sx*(1.2300)},{\sy*(-0.0000)})
	--({\sx*(1.2400)},{\sy*(-0.0000)})
	--({\sx*(1.2500)},{\sy*(-0.0000)})
	--({\sx*(1.2600)},{\sy*(0.0000)})
	--({\sx*(1.2700)},{\sy*(-0.0000)})
	--({\sx*(1.2800)},{\sy*(-0.0000)})
	--({\sx*(1.2900)},{\sy*(-0.0000)})
	--({\sx*(1.3000)},{\sy*(0.0000)})
	--({\sx*(1.3100)},{\sy*(-0.0000)})
	--({\sx*(1.3200)},{\sy*(-0.0000)})
	--({\sx*(1.3300)},{\sy*(-0.0000)})
	--({\sx*(1.3400)},{\sy*(0.0000)})
	--({\sx*(1.3500)},{\sy*(0.0000)})
	--({\sx*(1.3600)},{\sy*(-0.0000)})
	--({\sx*(1.3700)},{\sy*(-0.0000)})
	--({\sx*(1.3800)},{\sy*(-0.0000)})
	--({\sx*(1.3900)},{\sy*(-0.0000)})
	--({\sx*(1.4000)},{\sy*(-0.0000)})
	--({\sx*(1.4100)},{\sy*(0.0000)})
	--({\sx*(1.4200)},{\sy*(-0.0000)})
	--({\sx*(1.4300)},{\sy*(0.0000)})
	--({\sx*(1.4400)},{\sy*(-0.0000)})
	--({\sx*(1.4500)},{\sy*(-0.0000)})
	--({\sx*(1.4600)},{\sy*(-0.0000)})
	--({\sx*(1.4700)},{\sy*(0.0000)})
	--({\sx*(1.4800)},{\sy*(0.0000)})
	--({\sx*(1.4900)},{\sy*(-0.0000)})
	--({\sx*(1.5000)},{\sy*(0.0000)})
	--({\sx*(1.5100)},{\sy*(0.0000)})
	--({\sx*(1.5200)},{\sy*(0.0000)})
	--({\sx*(1.5300)},{\sy*(0.0000)})
	--({\sx*(1.5400)},{\sy*(0.0000)})
	--({\sx*(1.5500)},{\sy*(0.0000)})
	--({\sx*(1.5600)},{\sy*(0.0000)})
	--({\sx*(1.5700)},{\sy*(-0.0000)})
	--({\sx*(1.5800)},{\sy*(0.0000)})
	--({\sx*(1.5900)},{\sy*(0.0000)})
	--({\sx*(1.6000)},{\sy*(0.0000)})
	--({\sx*(1.6100)},{\sy*(-0.0000)})
	--({\sx*(1.6200)},{\sy*(0.0000)})
	--({\sx*(1.6300)},{\sy*(0.0000)})
	--({\sx*(1.6400)},{\sy*(0.0000)})
	--({\sx*(1.6500)},{\sy*(-0.0000)})
	--({\sx*(1.6600)},{\sy*(-0.0000)})
	--({\sx*(1.6700)},{\sy*(0.0000)})
	--({\sx*(1.6800)},{\sy*(0.0000)})
	--({\sx*(1.6900)},{\sy*(-0.0000)})
	--({\sx*(1.7000)},{\sy*(-0.0000)})
	--({\sx*(1.7100)},{\sy*(0.0000)})
	--({\sx*(1.7200)},{\sy*(0.0000)})
	--({\sx*(1.7300)},{\sy*(0.0000)})
	--({\sx*(1.7400)},{\sy*(-0.0000)})
	--({\sx*(1.7500)},{\sy*(-0.0000)})
	--({\sx*(1.7600)},{\sy*(-0.0000)})
	--({\sx*(1.7700)},{\sy*(0.0000)})
	--({\sx*(1.7800)},{\sy*(-0.0000)})
	--({\sx*(1.7900)},{\sy*(-0.0000)})
	--({\sx*(1.8000)},{\sy*(0.0000)})
	--({\sx*(1.8100)},{\sy*(-0.0000)})
	--({\sx*(1.8200)},{\sy*(-0.0000)})
	--({\sx*(1.8300)},{\sy*(-0.0000)})
	--({\sx*(1.8400)},{\sy*(-0.0000)})
	--({\sx*(1.8500)},{\sy*(0.0000)})
	--({\sx*(1.8600)},{\sy*(-0.0000)})
	--({\sx*(1.8700)},{\sy*(-0.0000)})
	--({\sx*(1.8800)},{\sy*(-0.0000)})
	--({\sx*(1.8900)},{\sy*(-0.0000)})
	--({\sx*(1.9000)},{\sy*(-0.0000)})
	--({\sx*(1.9100)},{\sy*(-0.0000)})
	--({\sx*(1.9200)},{\sy*(0.0000)})
	--({\sx*(1.9300)},{\sy*(0.0000)})
	--({\sx*(1.9400)},{\sy*(0.0000)})
	--({\sx*(1.9500)},{\sy*(-0.0000)})
	--({\sx*(1.9600)},{\sy*(-0.0000)})
	--({\sx*(1.9700)},{\sy*(0.0000)})
	--({\sx*(1.9800)},{\sy*(0.0000)})
	--({\sx*(1.9900)},{\sy*(-0.0000)})
	--({\sx*(2.0000)},{\sy*(0.0000)})
	--({\sx*(2.0100)},{\sy*(0.0000)})
	--({\sx*(2.0200)},{\sy*(-0.0000)})
	--({\sx*(2.0300)},{\sy*(0.0000)})
	--({\sx*(2.0400)},{\sy*(-0.0000)})
	--({\sx*(2.0500)},{\sy*(0.0000)})
	--({\sx*(2.0600)},{\sy*(-0.0000)})
	--({\sx*(2.0700)},{\sy*(0.0000)})
	--({\sx*(2.0800)},{\sy*(0.0000)})
	--({\sx*(2.0900)},{\sy*(0.0000)})
	--({\sx*(2.1000)},{\sy*(0.0000)})
	--({\sx*(2.1100)},{\sy*(-0.0000)})
	--({\sx*(2.1200)},{\sy*(0.0000)})
	--({\sx*(2.1300)},{\sy*(0.0000)})
	--({\sx*(2.1400)},{\sy*(-0.0000)})
	--({\sx*(2.1500)},{\sy*(0.0000)})
	--({\sx*(2.1600)},{\sy*(0.0000)})
	--({\sx*(2.1700)},{\sy*(0.0000)})
	--({\sx*(2.1800)},{\sy*(0.0000)})
	--({\sx*(2.1900)},{\sy*(-0.0000)})
	--({\sx*(2.2000)},{\sy*(0.0000)})
	--({\sx*(2.2100)},{\sy*(0.0000)})
	--({\sx*(2.2200)},{\sy*(0.0000)})
	--({\sx*(2.2300)},{\sy*(0.0000)})
	--({\sx*(2.2400)},{\sy*(0.0000)})
	--({\sx*(2.2500)},{\sy*(0.0000)})
	--({\sx*(2.2600)},{\sy*(0.0000)})
	--({\sx*(2.2700)},{\sy*(-0.0000)})
	--({\sx*(2.2800)},{\sy*(0.0000)})
	--({\sx*(2.2900)},{\sy*(0.0000)})
	--({\sx*(2.3000)},{\sy*(-0.0000)})
	--({\sx*(2.3100)},{\sy*(-0.0000)})
	--({\sx*(2.3200)},{\sy*(-0.0000)})
	--({\sx*(2.3300)},{\sy*(0.0000)})
	--({\sx*(2.3400)},{\sy*(-0.0000)})
	--({\sx*(2.3500)},{\sy*(0.0000)})
	--({\sx*(2.3600)},{\sy*(0.0000)})
	--({\sx*(2.3700)},{\sy*(-0.0000)})
	--({\sx*(2.3800)},{\sy*(0.0000)})
	--({\sx*(2.3900)},{\sy*(-0.0000)})
	--({\sx*(2.4000)},{\sy*(0.0000)})
	--({\sx*(2.4100)},{\sy*(-0.0000)})
	--({\sx*(2.4200)},{\sy*(0.0000)})
	--({\sx*(2.4300)},{\sy*(0.0000)})
	--({\sx*(2.4400)},{\sy*(-0.0000)})
	--({\sx*(2.4500)},{\sy*(-0.0000)})
	--({\sx*(2.4600)},{\sy*(-0.0000)})
	--({\sx*(2.4700)},{\sy*(0.0000)})
	--({\sx*(2.4800)},{\sy*(0.0000)})
	--({\sx*(2.4900)},{\sy*(0.0000)})
	--({\sx*(2.5000)},{\sy*(0.0000)})
	--({\sx*(2.5100)},{\sy*(-0.0000)})
	--({\sx*(2.5200)},{\sy*(0.0000)})
	--({\sx*(2.5300)},{\sy*(0.0000)})
	--({\sx*(2.5400)},{\sy*(0.0000)})
	--({\sx*(2.5500)},{\sy*(-0.0000)})
	--({\sx*(2.5600)},{\sy*(0.0000)})
	--({\sx*(2.5700)},{\sy*(0.0000)})
	--({\sx*(2.5800)},{\sy*(0.0000)})
	--({\sx*(2.5900)},{\sy*(0.0000)})
	--({\sx*(2.6000)},{\sy*(0.0000)})
	--({\sx*(2.6100)},{\sy*(0.0000)})
	--({\sx*(2.6200)},{\sy*(-0.0000)})
	--({\sx*(2.6300)},{\sy*(0.0000)})
	--({\sx*(2.6400)},{\sy*(0.0000)})
	--({\sx*(2.6500)},{\sy*(0.0000)})
	--({\sx*(2.6600)},{\sy*(-0.0000)})
	--({\sx*(2.6700)},{\sy*(0.0000)})
	--({\sx*(2.6800)},{\sy*(0.0000)})
	--({\sx*(2.6900)},{\sy*(0.0000)})
	--({\sx*(2.7000)},{\sy*(0.0000)})
	--({\sx*(2.7100)},{\sy*(0.0000)})
	--({\sx*(2.7200)},{\sy*(0.0000)})
	--({\sx*(2.7300)},{\sy*(0.0000)})
	--({\sx*(2.7400)},{\sy*(-0.0000)})
	--({\sx*(2.7500)},{\sy*(-0.0000)})
	--({\sx*(2.7600)},{\sy*(0.0000)})
	--({\sx*(2.7700)},{\sy*(0.0000)})
	--({\sx*(2.7800)},{\sy*(-0.0000)})
	--({\sx*(2.7900)},{\sy*(-0.0000)})
	--({\sx*(2.8000)},{\sy*(0.0000)})
	--({\sx*(2.8100)},{\sy*(-0.0000)})
	--({\sx*(2.8200)},{\sy*(-0.0000)})
	--({\sx*(2.8300)},{\sy*(-0.0000)})
	--({\sx*(2.8400)},{\sy*(-0.0000)})
	--({\sx*(2.8500)},{\sy*(-0.0000)})
	--({\sx*(2.8600)},{\sy*(-0.0000)})
	--({\sx*(2.8700)},{\sy*(0.0000)})
	--({\sx*(2.8800)},{\sy*(-0.0000)})
	--({\sx*(2.8900)},{\sy*(-0.0000)})
	--({\sx*(2.9000)},{\sy*(0.0000)})
	--({\sx*(2.9100)},{\sy*(-0.0000)})
	--({\sx*(2.9200)},{\sy*(-0.0000)})
	--({\sx*(2.9300)},{\sy*(-0.0000)})
	--({\sx*(2.9400)},{\sy*(-0.0000)})
	--({\sx*(2.9500)},{\sy*(-0.0000)})
	--({\sx*(2.9600)},{\sy*(-0.0000)})
	--({\sx*(2.9700)},{\sy*(-0.0000)})
	--({\sx*(2.9800)},{\sy*(-0.0000)})
	--({\sx*(2.9900)},{\sy*(0.0000)})
	--({\sx*(3.0000)},{\sy*(0.0000)})
	--({\sx*(3.0100)},{\sy*(0.0000)})
	--({\sx*(3.0200)},{\sy*(0.0000)})
	--({\sx*(3.0300)},{\sy*(0.0000)})
	--({\sx*(3.0400)},{\sy*(0.0000)})
	--({\sx*(3.0500)},{\sy*(0.0000)})
	--({\sx*(3.0600)},{\sy*(0.0000)})
	--({\sx*(3.0700)},{\sy*(0.0000)})
	--({\sx*(3.0800)},{\sy*(0.0000)})
	--({\sx*(3.0900)},{\sy*(0.0000)})
	--({\sx*(3.1000)},{\sy*(0.0000)})
	--({\sx*(3.1100)},{\sy*(0.0000)})
	--({\sx*(3.1200)},{\sy*(0.0000)})
	--({\sx*(3.1300)},{\sy*(0.0000)})
	--({\sx*(3.1400)},{\sy*(0.0000)})
	--({\sx*(3.1500)},{\sy*(0.0000)})
	--({\sx*(3.1600)},{\sy*(0.0000)})
	--({\sx*(3.1700)},{\sy*(0.0000)})
	--({\sx*(3.1800)},{\sy*(0.0000)})
	--({\sx*(3.1900)},{\sy*(-0.0000)})
	--({\sx*(3.2000)},{\sy*(-0.0000)})
	--({\sx*(3.2100)},{\sy*(0.0000)})
	--({\sx*(3.2200)},{\sy*(-0.0000)})
	--({\sx*(3.2300)},{\sy*(0.0000)})
	--({\sx*(3.2400)},{\sy*(0.0000)})
	--({\sx*(3.2500)},{\sy*(-0.0000)})
	--({\sx*(3.2600)},{\sy*(0.0000)})
	--({\sx*(3.2700)},{\sy*(-0.0000)})
	--({\sx*(3.2800)},{\sy*(0.0000)})
	--({\sx*(3.2900)},{\sy*(-0.0000)})
	--({\sx*(3.3000)},{\sy*(0.0000)})
	--({\sx*(3.3100)},{\sy*(0.0000)})
	--({\sx*(3.3200)},{\sy*(-0.0000)})
	--({\sx*(3.3300)},{\sy*(-0.0000)})
	--({\sx*(3.3400)},{\sy*(0.0000)})
	--({\sx*(3.3500)},{\sy*(0.0000)})
	--({\sx*(3.3600)},{\sy*(0.0000)})
	--({\sx*(3.3700)},{\sy*(0.0000)})
	--({\sx*(3.3800)},{\sy*(0.0000)})
	--({\sx*(3.3900)},{\sy*(-0.0000)})
	--({\sx*(3.4000)},{\sy*(0.0000)})
	--({\sx*(3.4100)},{\sy*(0.0000)})
	--({\sx*(3.4200)},{\sy*(0.0000)})
	--({\sx*(3.4300)},{\sy*(0.0000)})
	--({\sx*(3.4400)},{\sy*(-0.0000)})
	--({\sx*(3.4500)},{\sy*(-0.0000)})
	--({\sx*(3.4600)},{\sy*(0.0000)})
	--({\sx*(3.4700)},{\sy*(0.0000)})
	--({\sx*(3.4800)},{\sy*(-0.0000)})
	--({\sx*(3.4900)},{\sy*(0.0000)})
	--({\sx*(3.5000)},{\sy*(0.0000)})
	--({\sx*(3.5100)},{\sy*(-0.0000)})
	--({\sx*(3.5200)},{\sy*(-0.0000)})
	--({\sx*(3.5300)},{\sy*(0.0000)})
	--({\sx*(3.5400)},{\sy*(0.0000)})
	--({\sx*(3.5500)},{\sy*(-0.0000)})
	--({\sx*(3.5600)},{\sy*(-0.0000)})
	--({\sx*(3.5700)},{\sy*(0.0000)})
	--({\sx*(3.5800)},{\sy*(-0.0000)})
	--({\sx*(3.5900)},{\sy*(-0.0000)})
	--({\sx*(3.6000)},{\sy*(0.0000)})
	--({\sx*(3.6100)},{\sy*(0.0000)})
	--({\sx*(3.6200)},{\sy*(0.0000)})
	--({\sx*(3.6300)},{\sy*(0.0000)})
	--({\sx*(3.6400)},{\sy*(-0.0000)})
	--({\sx*(3.6500)},{\sy*(0.0000)})
	--({\sx*(3.6600)},{\sy*(0.0000)})
	--({\sx*(3.6700)},{\sy*(-0.0000)})
	--({\sx*(3.6800)},{\sy*(0.0000)})
	--({\sx*(3.6900)},{\sy*(0.0000)})
	--({\sx*(3.7000)},{\sy*(-0.0000)})
	--({\sx*(3.7100)},{\sy*(0.0000)})
	--({\sx*(3.7200)},{\sy*(-0.0000)})
	--({\sx*(3.7300)},{\sy*(0.0000)})
	--({\sx*(3.7400)},{\sy*(0.0000)})
	--({\sx*(3.7500)},{\sy*(0.0000)})
	--({\sx*(3.7600)},{\sy*(-0.0000)})
	--({\sx*(3.7700)},{\sy*(0.0000)})
	--({\sx*(3.7800)},{\sy*(-0.0000)})
	--({\sx*(3.7900)},{\sy*(-0.0000)})
	--({\sx*(3.8000)},{\sy*(-0.0000)})
	--({\sx*(3.8100)},{\sy*(0.0000)})
	--({\sx*(3.8200)},{\sy*(-0.0000)})
	--({\sx*(3.8300)},{\sy*(-0.0000)})
	--({\sx*(3.8400)},{\sy*(0.0000)})
	--({\sx*(3.8500)},{\sy*(-0.0000)})
	--({\sx*(3.8600)},{\sy*(0.0000)})
	--({\sx*(3.8700)},{\sy*(-0.0000)})
	--({\sx*(3.8800)},{\sy*(-0.0000)})
	--({\sx*(3.8900)},{\sy*(0.0000)})
	--({\sx*(3.9000)},{\sy*(0.0000)})
	--({\sx*(3.9100)},{\sy*(0.0000)})
	--({\sx*(3.9200)},{\sy*(0.0000)})
	--({\sx*(3.9300)},{\sy*(0.0000)})
	--({\sx*(3.9400)},{\sy*(-0.0000)})
	--({\sx*(3.9500)},{\sy*(-0.0000)})
	--({\sx*(3.9600)},{\sy*(0.0000)})
	--({\sx*(3.9700)},{\sy*(0.0000)})
	--({\sx*(3.9800)},{\sy*(-0.0000)})
	--({\sx*(3.9900)},{\sy*(0.0000)})
	--({\sx*(4.0000)},{\sy*(0.0000)})
	--({\sx*(4.0100)},{\sy*(0.0000)})
	--({\sx*(4.0200)},{\sy*(0.0000)})
	--({\sx*(4.0300)},{\sy*(0.0000)})
	--({\sx*(4.0400)},{\sy*(0.0000)})
	--({\sx*(4.0500)},{\sy*(0.0000)})
	--({\sx*(4.0600)},{\sy*(-0.0000)})
	--({\sx*(4.0700)},{\sy*(0.0000)})
	--({\sx*(4.0800)},{\sy*(0.0000)})
	--({\sx*(4.0900)},{\sy*(-0.0000)})
	--({\sx*(4.1000)},{\sy*(-0.0000)})
	--({\sx*(4.1100)},{\sy*(0.0000)})
	--({\sx*(4.1200)},{\sy*(-0.0000)})
	--({\sx*(4.1300)},{\sy*(0.0000)})
	--({\sx*(4.1400)},{\sy*(-0.0000)})
	--({\sx*(4.1500)},{\sy*(-0.0000)})
	--({\sx*(4.1600)},{\sy*(0.0000)})
	--({\sx*(4.1700)},{\sy*(0.0000)})
	--({\sx*(4.1800)},{\sy*(0.0000)})
	--({\sx*(4.1900)},{\sy*(-0.0000)})
	--({\sx*(4.2000)},{\sy*(-0.0000)})
	--({\sx*(4.2100)},{\sy*(-0.0000)})
	--({\sx*(4.2200)},{\sy*(0.0000)})
	--({\sx*(4.2300)},{\sy*(-0.0000)})
	--({\sx*(4.2400)},{\sy*(-0.0000)})
	--({\sx*(4.2500)},{\sy*(-0.0000)})
	--({\sx*(4.2600)},{\sy*(0.0000)})
	--({\sx*(4.2700)},{\sy*(0.0000)})
	--({\sx*(4.2800)},{\sy*(-0.0000)})
	--({\sx*(4.2900)},{\sy*(0.0000)})
	--({\sx*(4.3000)},{\sy*(0.0000)})
	--({\sx*(4.3100)},{\sy*(-0.0000)})
	--({\sx*(4.3200)},{\sy*(-0.0000)})
	--({\sx*(4.3300)},{\sy*(0.0000)})
	--({\sx*(4.3400)},{\sy*(0.0000)})
	--({\sx*(4.3500)},{\sy*(-0.0000)})
	--({\sx*(4.3600)},{\sy*(0.0000)})
	--({\sx*(4.3700)},{\sy*(0.0000)})
	--({\sx*(4.3800)},{\sy*(0.0000)})
	--({\sx*(4.3900)},{\sy*(0.0000)})
	--({\sx*(4.4000)},{\sy*(0.0000)})
	--({\sx*(4.4100)},{\sy*(0.0000)})
	--({\sx*(4.4200)},{\sy*(0.0000)})
	--({\sx*(4.4300)},{\sy*(-0.0000)})
	--({\sx*(4.4400)},{\sy*(0.0000)})
	--({\sx*(4.4500)},{\sy*(0.0000)})
	--({\sx*(4.4600)},{\sy*(0.0000)})
	--({\sx*(4.4700)},{\sy*(0.0000)})
	--({\sx*(4.4800)},{\sy*(0.0000)})
	--({\sx*(4.4900)},{\sy*(0.0000)})
	--({\sx*(4.5000)},{\sy*(0.0000)})
	--({\sx*(4.5100)},{\sy*(-0.0000)})
	--({\sx*(4.5200)},{\sy*(0.0000)})
	--({\sx*(4.5300)},{\sy*(-0.0000)})
	--({\sx*(4.5400)},{\sy*(-0.0000)})
	--({\sx*(4.5500)},{\sy*(0.0000)})
	--({\sx*(4.5600)},{\sy*(-0.0000)})
	--({\sx*(4.5700)},{\sy*(-0.0000)})
	--({\sx*(4.5800)},{\sy*(-0.0000)})
	--({\sx*(4.5900)},{\sy*(-0.0000)})
	--({\sx*(4.6000)},{\sy*(-0.0000)})
	--({\sx*(4.6100)},{\sy*(-0.0000)})
	--({\sx*(4.6200)},{\sy*(0.0000)})
	--({\sx*(4.6300)},{\sy*(-0.0000)})
	--({\sx*(4.6400)},{\sy*(-0.0000)})
	--({\sx*(4.6500)},{\sy*(-0.0000)})
	--({\sx*(4.6600)},{\sy*(-0.0000)})
	--({\sx*(4.6700)},{\sy*(-0.0000)})
	--({\sx*(4.6800)},{\sy*(0.0000)})
	--({\sx*(4.6900)},{\sy*(0.0000)})
	--({\sx*(4.7000)},{\sy*(-0.0000)})
	--({\sx*(4.7100)},{\sy*(-0.0000)})
	--({\sx*(4.7200)},{\sy*(0.0000)})
	--({\sx*(4.7300)},{\sy*(-0.0000)})
	--({\sx*(4.7400)},{\sy*(-0.0000)})
	--({\sx*(4.7500)},{\sy*(0.0000)})
	--({\sx*(4.7600)},{\sy*(0.0000)})
	--({\sx*(4.7700)},{\sy*(-0.0000)})
	--({\sx*(4.7800)},{\sy*(-0.0000)})
	--({\sx*(4.7900)},{\sy*(0.0000)})
	--({\sx*(4.8000)},{\sy*(0.0000)})
	--({\sx*(4.8100)},{\sy*(0.0000)})
	--({\sx*(4.8200)},{\sy*(-0.0000)})
	--({\sx*(4.8300)},{\sy*(-0.0000)})
	--({\sx*(4.8400)},{\sy*(0.0000)})
	--({\sx*(4.8500)},{\sy*(-0.0000)})
	--({\sx*(4.8600)},{\sy*(-0.0000)})
	--({\sx*(4.8700)},{\sy*(-0.0000)})
	--({\sx*(4.8800)},{\sy*(-0.0000)})
	--({\sx*(4.8900)},{\sy*(-0.0000)})
	--({\sx*(4.9000)},{\sy*(-0.0000)})
	--({\sx*(4.9100)},{\sy*(0.0000)})
	--({\sx*(4.9200)},{\sy*(-0.0000)})
	--({\sx*(4.9300)},{\sy*(0.0000)})
	--({\sx*(4.9400)},{\sy*(0.0000)})
	--({\sx*(4.9500)},{\sy*(0.0000)})
	--({\sx*(4.9600)},{\sy*(-0.0000)})
	--({\sx*(4.9700)},{\sy*(-0.0000)})
	--({\sx*(4.9800)},{\sy*(0.0000)})
	--({\sx*(4.9900)},{\sy*(0.0000)})
	--({\sx*(5.0000)},{\sy*(0.0000)});
}
\def\xwertep{
\fill[color=red] (0.0000,0) circle[radius={0.07/\skala}];
\fill[color=red] (0.1562,0) circle[radius={0.07/\skala}];
\fill[color=red] (0.3125,0) circle[radius={0.07/\skala}];
\fill[color=red] (0.4688,0) circle[radius={0.07/\skala}];
\fill[color=red] (0.6250,0) circle[radius={0.07/\skala}];
\fill[color=red] (0.7812,0) circle[radius={0.07/\skala}];
\fill[color=red] (0.9375,0) circle[radius={0.07/\skala}];
\fill[color=red] (1.0938,0) circle[radius={0.07/\skala}];
\fill[color=red] (1.2500,0) circle[radius={0.07/\skala}];
\fill[color=red] (1.4062,0) circle[radius={0.07/\skala}];
\fill[color=red] (1.5625,0) circle[radius={0.07/\skala}];
\fill[color=red] (1.7188,0) circle[radius={0.07/\skala}];
\fill[color=red] (1.8750,0) circle[radius={0.07/\skala}];
\fill[color=red] (2.0312,0) circle[radius={0.07/\skala}];
\fill[color=red] (2.1875,0) circle[radius={0.07/\skala}];
\fill[color=red] (2.3438,0) circle[radius={0.07/\skala}];
\fill[color=red] (2.5000,0) circle[radius={0.07/\skala}];
\fill[color=red] (2.6562,0) circle[radius={0.07/\skala}];
\fill[color=red] (2.8125,0) circle[radius={0.07/\skala}];
\fill[color=red] (2.9688,0) circle[radius={0.07/\skala}];
\fill[color=red] (3.1250,0) circle[radius={0.07/\skala}];
\fill[color=red] (3.2812,0) circle[radius={0.07/\skala}];
\fill[color=red] (3.4375,0) circle[radius={0.07/\skala}];
\fill[color=red] (3.5938,0) circle[radius={0.07/\skala}];
\fill[color=red] (3.7500,0) circle[radius={0.07/\skala}];
\fill[color=red] (3.9062,0) circle[radius={0.07/\skala}];
\fill[color=red] (4.0625,0) circle[radius={0.07/\skala}];
\fill[color=red] (4.2188,0) circle[radius={0.07/\skala}];
\fill[color=red] (4.3750,0) circle[radius={0.07/\skala}];
\fill[color=red] (4.5312,0) circle[radius={0.07/\skala}];
\fill[color=red] (4.6875,0) circle[radius={0.07/\skala}];
\fill[color=red] (4.8438,0) circle[radius={0.07/\skala}];
\fill[color=red] (5.0000,0) circle[radius={0.07/\skala}];
}
\def\punktep{32}
\def\maxfehlerp{1.172\cdot 10^{-10}}
\def\fehlerp{
\draw[color=red,line width=1.4pt,line join=round] ({\sx*(0.000)},{\sy*(0.0000)})
	--({\sx*(0.0100)},{\sy*(0.2068)})
	--({\sx*(0.0200)},{\sy*(-0.3349)})
	--({\sx*(0.0300)},{\sy*(0.5275)})
	--({\sx*(0.0400)},{\sy*(0.4372)})
	--({\sx*(0.0500)},{\sy*(0.6559)})
	--({\sx*(0.0600)},{\sy*(0.2362)})
	--({\sx*(0.0700)},{\sy*(1.0000)})
	--({\sx*(0.0800)},{\sy*(-0.1756)})
	--({\sx*(0.0900)},{\sy*(0.2299)})
	--({\sx*(0.1000)},{\sy*(0.1513)})
	--({\sx*(0.1100)},{\sy*(-0.2498)})
	--({\sx*(0.1200)},{\sy*(0.2818)})
	--({\sx*(0.1300)},{\sy*(-0.0630)})
	--({\sx*(0.1400)},{\sy*(-0.0053)})
	--({\sx*(0.1500)},{\sy*(-0.0158)})
	--({\sx*(0.1600)},{\sy*(-0.0019)})
	--({\sx*(0.1700)},{\sy*(-0.0031)})
	--({\sx*(0.1800)},{\sy*(-0.0129)})
	--({\sx*(0.1900)},{\sy*(-0.0042)})
	--({\sx*(0.2000)},{\sy*(0.0182)})
	--({\sx*(0.2100)},{\sy*(-0.0343)})
	--({\sx*(0.2200)},{\sy*(-0.0088)})
	--({\sx*(0.2300)},{\sy*(-0.0349)})
	--({\sx*(0.2400)},{\sy*(-0.0148)})
	--({\sx*(0.2500)},{\sy*(0.0016)})
	--({\sx*(0.2600)},{\sy*(-0.0045)})
	--({\sx*(0.2700)},{\sy*(-0.0108)})
	--({\sx*(0.2800)},{\sy*(-0.0054)})
	--({\sx*(0.2900)},{\sy*(-0.0028)})
	--({\sx*(0.3000)},{\sy*(-0.0006)})
	--({\sx*(0.3100)},{\sy*(0.0001)})
	--({\sx*(0.3200)},{\sy*(0.0001)})
	--({\sx*(0.3300)},{\sy*(-0.0002)})
	--({\sx*(0.3400)},{\sy*(0.0018)})
	--({\sx*(0.3500)},{\sy*(0.0003)})
	--({\sx*(0.3600)},{\sy*(0.0019)})
	--({\sx*(0.3700)},{\sy*(0.0029)})
	--({\sx*(0.3800)},{\sy*(0.0015)})
	--({\sx*(0.3900)},{\sy*(0.0018)})
	--({\sx*(0.4000)},{\sy*(0.0033)})
	--({\sx*(0.4100)},{\sy*(0.0000)})
	--({\sx*(0.4200)},{\sy*(-0.0007)})
	--({\sx*(0.4300)},{\sy*(0.0003)})
	--({\sx*(0.4400)},{\sy*(0.0010)})
	--({\sx*(0.4500)},{\sy*(-0.0000)})
	--({\sx*(0.4600)},{\sy*(0.0000)})
	--({\sx*(0.4700)},{\sy*(0.0000)})
	--({\sx*(0.4800)},{\sy*(0.0001)})
	--({\sx*(0.4900)},{\sy*(0.0000)})
	--({\sx*(0.5000)},{\sy*(-0.0002)})
	--({\sx*(0.5100)},{\sy*(0.0001)})
	--({\sx*(0.5200)},{\sy*(0.0001)})
	--({\sx*(0.5300)},{\sy*(0.0002)})
	--({\sx*(0.5400)},{\sy*(-0.0001)})
	--({\sx*(0.5500)},{\sy*(-0.0006)})
	--({\sx*(0.5600)},{\sy*(0.0000)})
	--({\sx*(0.5700)},{\sy*(-0.0003)})
	--({\sx*(0.5800)},{\sy*(-0.0001)})
	--({\sx*(0.5900)},{\sy*(-0.0003)})
	--({\sx*(0.6000)},{\sy*(-0.0002)})
	--({\sx*(0.6100)},{\sy*(0.0000)})
	--({\sx*(0.6200)},{\sy*(-0.0000)})
	--({\sx*(0.6300)},{\sy*(0.0000)})
	--({\sx*(0.6400)},{\sy*(0.0000)})
	--({\sx*(0.6500)},{\sy*(0.0001)})
	--({\sx*(0.6600)},{\sy*(-0.0001)})
	--({\sx*(0.6700)},{\sy*(0.0000)})
	--({\sx*(0.6800)},{\sy*(-0.0000)})
	--({\sx*(0.6900)},{\sy*(0.0000)})
	--({\sx*(0.7000)},{\sy*(0.0000)})
	--({\sx*(0.7100)},{\sy*(-0.0000)})
	--({\sx*(0.7200)},{\sy*(-0.0000)})
	--({\sx*(0.7300)},{\sy*(0.0001)})
	--({\sx*(0.7400)},{\sy*(0.0000)})
	--({\sx*(0.7500)},{\sy*(-0.0000)})
	--({\sx*(0.7600)},{\sy*(0.0000)})
	--({\sx*(0.7700)},{\sy*(-0.0000)})
	--({\sx*(0.7800)},{\sy*(-0.0000)})
	--({\sx*(0.7900)},{\sy*(-0.0000)})
	--({\sx*(0.8000)},{\sy*(-0.0000)})
	--({\sx*(0.8100)},{\sy*(-0.0000)})
	--({\sx*(0.8200)},{\sy*(0.0000)})
	--({\sx*(0.8300)},{\sy*(0.0000)})
	--({\sx*(0.8400)},{\sy*(0.0000)})
	--({\sx*(0.8500)},{\sy*(-0.0000)})
	--({\sx*(0.8600)},{\sy*(0.0000)})
	--({\sx*(0.8700)},{\sy*(0.0000)})
	--({\sx*(0.8800)},{\sy*(0.0000)})
	--({\sx*(0.8900)},{\sy*(0.0000)})
	--({\sx*(0.9000)},{\sy*(0.0000)})
	--({\sx*(0.9100)},{\sy*(0.0000)})
	--({\sx*(0.9200)},{\sy*(0.0000)})
	--({\sx*(0.9300)},{\sy*(-0.0000)})
	--({\sx*(0.9400)},{\sy*(0.0000)})
	--({\sx*(0.9500)},{\sy*(-0.0000)})
	--({\sx*(0.9600)},{\sy*(0.0000)})
	--({\sx*(0.9700)},{\sy*(0.0000)})
	--({\sx*(0.9800)},{\sy*(-0.0000)})
	--({\sx*(0.9900)},{\sy*(-0.0000)})
	--({\sx*(1.0000)},{\sy*(0.0000)})
	--({\sx*(1.0100)},{\sy*(0.0000)})
	--({\sx*(1.0200)},{\sy*(0.0000)})
	--({\sx*(1.0300)},{\sy*(-0.0000)})
	--({\sx*(1.0400)},{\sy*(-0.0000)})
	--({\sx*(1.0500)},{\sy*(0.0000)})
	--({\sx*(1.0600)},{\sy*(-0.0000)})
	--({\sx*(1.0700)},{\sy*(-0.0000)})
	--({\sx*(1.0800)},{\sy*(0.0000)})
	--({\sx*(1.0900)},{\sy*(0.0000)})
	--({\sx*(1.1000)},{\sy*(-0.0000)})
	--({\sx*(1.1100)},{\sy*(-0.0000)})
	--({\sx*(1.1200)},{\sy*(-0.0000)})
	--({\sx*(1.1300)},{\sy*(0.0000)})
	--({\sx*(1.1400)},{\sy*(-0.0000)})
	--({\sx*(1.1500)},{\sy*(-0.0000)})
	--({\sx*(1.1600)},{\sy*(-0.0000)})
	--({\sx*(1.1700)},{\sy*(-0.0000)})
	--({\sx*(1.1800)},{\sy*(0.0000)})
	--({\sx*(1.1900)},{\sy*(-0.0000)})
	--({\sx*(1.2000)},{\sy*(-0.0000)})
	--({\sx*(1.2100)},{\sy*(-0.0000)})
	--({\sx*(1.2200)},{\sy*(0.0000)})
	--({\sx*(1.2300)},{\sy*(-0.0000)})
	--({\sx*(1.2400)},{\sy*(-0.0000)})
	--({\sx*(1.2500)},{\sy*(0.0000)})
	--({\sx*(1.2600)},{\sy*(0.0000)})
	--({\sx*(1.2700)},{\sy*(-0.0000)})
	--({\sx*(1.2800)},{\sy*(-0.0000)})
	--({\sx*(1.2900)},{\sy*(0.0000)})
	--({\sx*(1.3000)},{\sy*(0.0000)})
	--({\sx*(1.3100)},{\sy*(-0.0000)})
	--({\sx*(1.3200)},{\sy*(-0.0000)})
	--({\sx*(1.3300)},{\sy*(0.0000)})
	--({\sx*(1.3400)},{\sy*(0.0000)})
	--({\sx*(1.3500)},{\sy*(0.0000)})
	--({\sx*(1.3600)},{\sy*(-0.0000)})
	--({\sx*(1.3700)},{\sy*(-0.0000)})
	--({\sx*(1.3800)},{\sy*(0.0000)})
	--({\sx*(1.3900)},{\sy*(0.0000)})
	--({\sx*(1.4000)},{\sy*(-0.0000)})
	--({\sx*(1.4100)},{\sy*(-0.0000)})
	--({\sx*(1.4200)},{\sy*(-0.0000)})
	--({\sx*(1.4300)},{\sy*(0.0000)})
	--({\sx*(1.4400)},{\sy*(0.0000)})
	--({\sx*(1.4500)},{\sy*(-0.0000)})
	--({\sx*(1.4600)},{\sy*(0.0000)})
	--({\sx*(1.4700)},{\sy*(0.0000)})
	--({\sx*(1.4800)},{\sy*(0.0000)})
	--({\sx*(1.4900)},{\sy*(-0.0000)})
	--({\sx*(1.5000)},{\sy*(-0.0000)})
	--({\sx*(1.5100)},{\sy*(0.0000)})
	--({\sx*(1.5200)},{\sy*(-0.0000)})
	--({\sx*(1.5300)},{\sy*(-0.0000)})
	--({\sx*(1.5400)},{\sy*(0.0000)})
	--({\sx*(1.5500)},{\sy*(0.0000)})
	--({\sx*(1.5600)},{\sy*(0.0000)})
	--({\sx*(1.5700)},{\sy*(-0.0000)})
	--({\sx*(1.5800)},{\sy*(0.0000)})
	--({\sx*(1.5900)},{\sy*(0.0000)})
	--({\sx*(1.6000)},{\sy*(0.0000)})
	--({\sx*(1.6100)},{\sy*(-0.0000)})
	--({\sx*(1.6200)},{\sy*(0.0000)})
	--({\sx*(1.6300)},{\sy*(0.0000)})
	--({\sx*(1.6400)},{\sy*(0.0000)})
	--({\sx*(1.6500)},{\sy*(-0.0000)})
	--({\sx*(1.6600)},{\sy*(-0.0000)})
	--({\sx*(1.6700)},{\sy*(-0.0000)})
	--({\sx*(1.6800)},{\sy*(0.0000)})
	--({\sx*(1.6900)},{\sy*(-0.0000)})
	--({\sx*(1.7000)},{\sy*(-0.0000)})
	--({\sx*(1.7100)},{\sy*(0.0000)})
	--({\sx*(1.7200)},{\sy*(0.0000)})
	--({\sx*(1.7300)},{\sy*(0.0000)})
	--({\sx*(1.7400)},{\sy*(-0.0000)})
	--({\sx*(1.7500)},{\sy*(-0.0000)})
	--({\sx*(1.7600)},{\sy*(0.0000)})
	--({\sx*(1.7700)},{\sy*(-0.0000)})
	--({\sx*(1.7800)},{\sy*(0.0000)})
	--({\sx*(1.7900)},{\sy*(-0.0000)})
	--({\sx*(1.8000)},{\sy*(0.0000)})
	--({\sx*(1.8100)},{\sy*(-0.0000)})
	--({\sx*(1.8200)},{\sy*(-0.0000)})
	--({\sx*(1.8300)},{\sy*(0.0000)})
	--({\sx*(1.8400)},{\sy*(0.0000)})
	--({\sx*(1.8500)},{\sy*(-0.0000)})
	--({\sx*(1.8600)},{\sy*(-0.0000)})
	--({\sx*(1.8700)},{\sy*(-0.0000)})
	--({\sx*(1.8800)},{\sy*(0.0000)})
	--({\sx*(1.8900)},{\sy*(0.0000)})
	--({\sx*(1.9000)},{\sy*(-0.0000)})
	--({\sx*(1.9100)},{\sy*(0.0000)})
	--({\sx*(1.9200)},{\sy*(0.0000)})
	--({\sx*(1.9300)},{\sy*(0.0000)})
	--({\sx*(1.9400)},{\sy*(-0.0000)})
	--({\sx*(1.9500)},{\sy*(-0.0000)})
	--({\sx*(1.9600)},{\sy*(-0.0000)})
	--({\sx*(1.9700)},{\sy*(0.0000)})
	--({\sx*(1.9800)},{\sy*(-0.0000)})
	--({\sx*(1.9900)},{\sy*(0.0000)})
	--({\sx*(2.0000)},{\sy*(-0.0000)})
	--({\sx*(2.0100)},{\sy*(-0.0000)})
	--({\sx*(2.0200)},{\sy*(0.0000)})
	--({\sx*(2.0300)},{\sy*(-0.0000)})
	--({\sx*(2.0400)},{\sy*(-0.0000)})
	--({\sx*(2.0500)},{\sy*(0.0000)})
	--({\sx*(2.0600)},{\sy*(-0.0000)})
	--({\sx*(2.0700)},{\sy*(0.0000)})
	--({\sx*(2.0800)},{\sy*(0.0000)})
	--({\sx*(2.0900)},{\sy*(0.0000)})
	--({\sx*(2.1000)},{\sy*(-0.0000)})
	--({\sx*(2.1100)},{\sy*(0.0000)})
	--({\sx*(2.1200)},{\sy*(-0.0000)})
	--({\sx*(2.1300)},{\sy*(-0.0000)})
	--({\sx*(2.1400)},{\sy*(-0.0000)})
	--({\sx*(2.1500)},{\sy*(0.0000)})
	--({\sx*(2.1600)},{\sy*(-0.0000)})
	--({\sx*(2.1700)},{\sy*(0.0000)})
	--({\sx*(2.1800)},{\sy*(0.0000)})
	--({\sx*(2.1900)},{\sy*(-0.0000)})
	--({\sx*(2.2000)},{\sy*(-0.0000)})
	--({\sx*(2.2100)},{\sy*(0.0000)})
	--({\sx*(2.2200)},{\sy*(-0.0000)})
	--({\sx*(2.2300)},{\sy*(-0.0000)})
	--({\sx*(2.2400)},{\sy*(-0.0000)})
	--({\sx*(2.2500)},{\sy*(0.0000)})
	--({\sx*(2.2600)},{\sy*(0.0000)})
	--({\sx*(2.2700)},{\sy*(0.0000)})
	--({\sx*(2.2800)},{\sy*(0.0000)})
	--({\sx*(2.2900)},{\sy*(0.0000)})
	--({\sx*(2.3000)},{\sy*(-0.0000)})
	--({\sx*(2.3100)},{\sy*(0.0000)})
	--({\sx*(2.3200)},{\sy*(0.0000)})
	--({\sx*(2.3300)},{\sy*(0.0000)})
	--({\sx*(2.3400)},{\sy*(-0.0000)})
	--({\sx*(2.3500)},{\sy*(0.0000)})
	--({\sx*(2.3600)},{\sy*(0.0000)})
	--({\sx*(2.3700)},{\sy*(0.0000)})
	--({\sx*(2.3800)},{\sy*(-0.0000)})
	--({\sx*(2.3900)},{\sy*(0.0000)})
	--({\sx*(2.4000)},{\sy*(0.0000)})
	--({\sx*(2.4100)},{\sy*(-0.0000)})
	--({\sx*(2.4200)},{\sy*(0.0000)})
	--({\sx*(2.4300)},{\sy*(0.0000)})
	--({\sx*(2.4400)},{\sy*(-0.0000)})
	--({\sx*(2.4500)},{\sy*(0.0000)})
	--({\sx*(2.4600)},{\sy*(0.0000)})
	--({\sx*(2.4700)},{\sy*(0.0000)})
	--({\sx*(2.4800)},{\sy*(-0.0000)})
	--({\sx*(2.4900)},{\sy*(0.0000)})
	--({\sx*(2.5000)},{\sy*(0.0000)})
	--({\sx*(2.5100)},{\sy*(0.0000)})
	--({\sx*(2.5200)},{\sy*(-0.0000)})
	--({\sx*(2.5300)},{\sy*(0.0000)})
	--({\sx*(2.5400)},{\sy*(0.0000)})
	--({\sx*(2.5500)},{\sy*(0.0000)})
	--({\sx*(2.5600)},{\sy*(-0.0000)})
	--({\sx*(2.5700)},{\sy*(0.0000)})
	--({\sx*(2.5800)},{\sy*(0.0000)})
	--({\sx*(2.5900)},{\sy*(-0.0000)})
	--({\sx*(2.6000)},{\sy*(0.0000)})
	--({\sx*(2.6100)},{\sy*(-0.0000)})
	--({\sx*(2.6200)},{\sy*(0.0000)})
	--({\sx*(2.6300)},{\sy*(0.0000)})
	--({\sx*(2.6400)},{\sy*(0.0000)})
	--({\sx*(2.6500)},{\sy*(0.0000)})
	--({\sx*(2.6600)},{\sy*(-0.0000)})
	--({\sx*(2.6700)},{\sy*(-0.0000)})
	--({\sx*(2.6800)},{\sy*(0.0000)})
	--({\sx*(2.6900)},{\sy*(0.0000)})
	--({\sx*(2.7000)},{\sy*(0.0000)})
	--({\sx*(2.7100)},{\sy*(0.0000)})
	--({\sx*(2.7200)},{\sy*(0.0000)})
	--({\sx*(2.7300)},{\sy*(0.0000)})
	--({\sx*(2.7400)},{\sy*(0.0000)})
	--({\sx*(2.7500)},{\sy*(0.0000)})
	--({\sx*(2.7600)},{\sy*(0.0000)})
	--({\sx*(2.7700)},{\sy*(0.0000)})
	--({\sx*(2.7800)},{\sy*(0.0000)})
	--({\sx*(2.7900)},{\sy*(0.0000)})
	--({\sx*(2.8000)},{\sy*(-0.0000)})
	--({\sx*(2.8100)},{\sy*(-0.0000)})
	--({\sx*(2.8200)},{\sy*(0.0000)})
	--({\sx*(2.8300)},{\sy*(0.0000)})
	--({\sx*(2.8400)},{\sy*(0.0000)})
	--({\sx*(2.8500)},{\sy*(0.0000)})
	--({\sx*(2.8600)},{\sy*(-0.0000)})
	--({\sx*(2.8700)},{\sy*(-0.0000)})
	--({\sx*(2.8800)},{\sy*(-0.0000)})
	--({\sx*(2.8900)},{\sy*(-0.0000)})
	--({\sx*(2.9000)},{\sy*(0.0000)})
	--({\sx*(2.9100)},{\sy*(0.0000)})
	--({\sx*(2.9200)},{\sy*(-0.0000)})
	--({\sx*(2.9300)},{\sy*(-0.0000)})
	--({\sx*(2.9400)},{\sy*(-0.0000)})
	--({\sx*(2.9500)},{\sy*(-0.0000)})
	--({\sx*(2.9600)},{\sy*(-0.0000)})
	--({\sx*(2.9700)},{\sy*(0.0000)})
	--({\sx*(2.9800)},{\sy*(0.0000)})
	--({\sx*(2.9900)},{\sy*(-0.0000)})
	--({\sx*(3.0000)},{\sy*(-0.0000)})
	--({\sx*(3.0100)},{\sy*(0.0000)})
	--({\sx*(3.0200)},{\sy*(0.0000)})
	--({\sx*(3.0300)},{\sy*(-0.0000)})
	--({\sx*(3.0400)},{\sy*(-0.0000)})
	--({\sx*(3.0500)},{\sy*(0.0000)})
	--({\sx*(3.0600)},{\sy*(0.0000)})
	--({\sx*(3.0700)},{\sy*(-0.0000)})
	--({\sx*(3.0800)},{\sy*(0.0000)})
	--({\sx*(3.0900)},{\sy*(0.0000)})
	--({\sx*(3.1000)},{\sy*(0.0000)})
	--({\sx*(3.1100)},{\sy*(-0.0000)})
	--({\sx*(3.1200)},{\sy*(0.0000)})
	--({\sx*(3.1300)},{\sy*(-0.0000)})
	--({\sx*(3.1400)},{\sy*(0.0000)})
	--({\sx*(3.1500)},{\sy*(0.0000)})
	--({\sx*(3.1600)},{\sy*(-0.0000)})
	--({\sx*(3.1700)},{\sy*(-0.0000)})
	--({\sx*(3.1800)},{\sy*(0.0000)})
	--({\sx*(3.1900)},{\sy*(0.0000)})
	--({\sx*(3.2000)},{\sy*(-0.0000)})
	--({\sx*(3.2100)},{\sy*(-0.0000)})
	--({\sx*(3.2200)},{\sy*(-0.0000)})
	--({\sx*(3.2300)},{\sy*(-0.0000)})
	--({\sx*(3.2400)},{\sy*(0.0000)})
	--({\sx*(3.2500)},{\sy*(-0.0000)})
	--({\sx*(3.2600)},{\sy*(-0.0000)})
	--({\sx*(3.2700)},{\sy*(-0.0000)})
	--({\sx*(3.2800)},{\sy*(0.0000)})
	--({\sx*(3.2900)},{\sy*(-0.0000)})
	--({\sx*(3.3000)},{\sy*(-0.0000)})
	--({\sx*(3.3100)},{\sy*(-0.0000)})
	--({\sx*(3.3200)},{\sy*(-0.0000)})
	--({\sx*(3.3300)},{\sy*(-0.0000)})
	--({\sx*(3.3400)},{\sy*(0.0000)})
	--({\sx*(3.3500)},{\sy*(0.0000)})
	--({\sx*(3.3600)},{\sy*(-0.0000)})
	--({\sx*(3.3700)},{\sy*(-0.0000)})
	--({\sx*(3.3800)},{\sy*(0.0000)})
	--({\sx*(3.3900)},{\sy*(-0.0000)})
	--({\sx*(3.4000)},{\sy*(0.0000)})
	--({\sx*(3.4100)},{\sy*(-0.0000)})
	--({\sx*(3.4200)},{\sy*(-0.0000)})
	--({\sx*(3.4300)},{\sy*(-0.0000)})
	--({\sx*(3.4400)},{\sy*(0.0000)})
	--({\sx*(3.4500)},{\sy*(-0.0000)})
	--({\sx*(3.4600)},{\sy*(0.0000)})
	--({\sx*(3.4700)},{\sy*(0.0000)})
	--({\sx*(3.4800)},{\sy*(-0.0000)})
	--({\sx*(3.4900)},{\sy*(-0.0000)})
	--({\sx*(3.5000)},{\sy*(-0.0000)})
	--({\sx*(3.5100)},{\sy*(-0.0000)})
	--({\sx*(3.5200)},{\sy*(-0.0000)})
	--({\sx*(3.5300)},{\sy*(0.0000)})
	--({\sx*(3.5400)},{\sy*(-0.0000)})
	--({\sx*(3.5500)},{\sy*(0.0000)})
	--({\sx*(3.5600)},{\sy*(0.0000)})
	--({\sx*(3.5700)},{\sy*(-0.0000)})
	--({\sx*(3.5800)},{\sy*(-0.0000)})
	--({\sx*(3.5900)},{\sy*(-0.0000)})
	--({\sx*(3.6000)},{\sy*(-0.0000)})
	--({\sx*(3.6100)},{\sy*(0.0000)})
	--({\sx*(3.6200)},{\sy*(0.0000)})
	--({\sx*(3.6300)},{\sy*(0.0000)})
	--({\sx*(3.6400)},{\sy*(0.0000)})
	--({\sx*(3.6500)},{\sy*(-0.0000)})
	--({\sx*(3.6600)},{\sy*(0.0000)})
	--({\sx*(3.6700)},{\sy*(-0.0000)})
	--({\sx*(3.6800)},{\sy*(0.0000)})
	--({\sx*(3.6900)},{\sy*(0.0000)})
	--({\sx*(3.7000)},{\sy*(-0.0000)})
	--({\sx*(3.7100)},{\sy*(-0.0000)})
	--({\sx*(3.7200)},{\sy*(0.0000)})
	--({\sx*(3.7300)},{\sy*(0.0000)})
	--({\sx*(3.7400)},{\sy*(0.0000)})
	--({\sx*(3.7500)},{\sy*(0.0000)})
	--({\sx*(3.7600)},{\sy*(-0.0000)})
	--({\sx*(3.7700)},{\sy*(-0.0000)})
	--({\sx*(3.7800)},{\sy*(-0.0000)})
	--({\sx*(3.7900)},{\sy*(-0.0000)})
	--({\sx*(3.8000)},{\sy*(-0.0000)})
	--({\sx*(3.8100)},{\sy*(0.0000)})
	--({\sx*(3.8200)},{\sy*(0.0000)})
	--({\sx*(3.8300)},{\sy*(0.0000)})
	--({\sx*(3.8400)},{\sy*(-0.0000)})
	--({\sx*(3.8500)},{\sy*(0.0000)})
	--({\sx*(3.8600)},{\sy*(-0.0000)})
	--({\sx*(3.8700)},{\sy*(-0.0000)})
	--({\sx*(3.8800)},{\sy*(0.0000)})
	--({\sx*(3.8900)},{\sy*(0.0000)})
	--({\sx*(3.9000)},{\sy*(-0.0000)})
	--({\sx*(3.9100)},{\sy*(0.0000)})
	--({\sx*(3.9200)},{\sy*(-0.0000)})
	--({\sx*(3.9300)},{\sy*(0.0000)})
	--({\sx*(3.9400)},{\sy*(0.0000)})
	--({\sx*(3.9500)},{\sy*(-0.0000)})
	--({\sx*(3.9600)},{\sy*(0.0000)})
	--({\sx*(3.9700)},{\sy*(0.0000)})
	--({\sx*(3.9800)},{\sy*(-0.0000)})
	--({\sx*(3.9900)},{\sy*(0.0000)})
	--({\sx*(4.0000)},{\sy*(0.0000)})
	--({\sx*(4.0100)},{\sy*(-0.0000)})
	--({\sx*(4.0200)},{\sy*(0.0000)})
	--({\sx*(4.0300)},{\sy*(0.0000)})
	--({\sx*(4.0400)},{\sy*(0.0000)})
	--({\sx*(4.0500)},{\sy*(0.0000)})
	--({\sx*(4.0600)},{\sy*(0.0000)})
	--({\sx*(4.0700)},{\sy*(-0.0000)})
	--({\sx*(4.0800)},{\sy*(0.0000)})
	--({\sx*(4.0900)},{\sy*(-0.0000)})
	--({\sx*(4.1000)},{\sy*(0.0000)})
	--({\sx*(4.1100)},{\sy*(0.0000)})
	--({\sx*(4.1200)},{\sy*(0.0000)})
	--({\sx*(4.1300)},{\sy*(0.0000)})
	--({\sx*(4.1400)},{\sy*(-0.0000)})
	--({\sx*(4.1500)},{\sy*(-0.0000)})
	--({\sx*(4.1600)},{\sy*(-0.0000)})
	--({\sx*(4.1700)},{\sy*(-0.0000)})
	--({\sx*(4.1800)},{\sy*(-0.0000)})
	--({\sx*(4.1900)},{\sy*(0.0000)})
	--({\sx*(4.2000)},{\sy*(0.0000)})
	--({\sx*(4.2100)},{\sy*(-0.0000)})
	--({\sx*(4.2200)},{\sy*(0.0000)})
	--({\sx*(4.2300)},{\sy*(0.0000)})
	--({\sx*(4.2400)},{\sy*(-0.0000)})
	--({\sx*(4.2500)},{\sy*(-0.0000)})
	--({\sx*(4.2600)},{\sy*(0.0000)})
	--({\sx*(4.2700)},{\sy*(0.0000)})
	--({\sx*(4.2800)},{\sy*(-0.0000)})
	--({\sx*(4.2900)},{\sy*(-0.0000)})
	--({\sx*(4.3000)},{\sy*(0.0000)})
	--({\sx*(4.3100)},{\sy*(0.0000)})
	--({\sx*(4.3200)},{\sy*(-0.0000)})
	--({\sx*(4.3300)},{\sy*(0.0001)})
	--({\sx*(4.3400)},{\sy*(0.0000)})
	--({\sx*(4.3500)},{\sy*(-0.0000)})
	--({\sx*(4.3600)},{\sy*(-0.0000)})
	--({\sx*(4.3700)},{\sy*(-0.0000)})
	--({\sx*(4.3800)},{\sy*(-0.0000)})
	--({\sx*(4.3900)},{\sy*(0.0000)})
	--({\sx*(4.4000)},{\sy*(-0.0001)})
	--({\sx*(4.4100)},{\sy*(0.0000)})
	--({\sx*(4.4200)},{\sy*(-0.0001)})
	--({\sx*(4.4300)},{\sy*(0.0001)})
	--({\sx*(4.4400)},{\sy*(0.0000)})
	--({\sx*(4.4500)},{\sy*(-0.0002)})
	--({\sx*(4.4600)},{\sy*(0.0001)})
	--({\sx*(4.4700)},{\sy*(0.0003)})
	--({\sx*(4.4800)},{\sy*(0.0001)})
	--({\sx*(4.4900)},{\sy*(-0.0002)})
	--({\sx*(4.5000)},{\sy*(0.0003)})
	--({\sx*(4.5100)},{\sy*(0.0002)})
	--({\sx*(4.5200)},{\sy*(0.0001)})
	--({\sx*(4.5300)},{\sy*(0.0000)})
	--({\sx*(4.5400)},{\sy*(0.0001)})
	--({\sx*(4.5500)},{\sy*(-0.0002)})
	--({\sx*(4.5600)},{\sy*(0.0003)})
	--({\sx*(4.5700)},{\sy*(-0.0008)})
	--({\sx*(4.5800)},{\sy*(-0.0000)})
	--({\sx*(4.5900)},{\sy*(0.0013)})
	--({\sx*(4.6000)},{\sy*(0.0001)})
	--({\sx*(4.6100)},{\sy*(0.0007)})
	--({\sx*(4.6200)},{\sy*(0.0015)})
	--({\sx*(4.6300)},{\sy*(-0.0022)})
	--({\sx*(4.6400)},{\sy*(0.0020)})
	--({\sx*(4.6500)},{\sy*(-0.0003)})
	--({\sx*(4.6600)},{\sy*(-0.0001)})
	--({\sx*(4.6700)},{\sy*(-0.0003)})
	--({\sx*(4.6800)},{\sy*(0.0001)})
	--({\sx*(4.6900)},{\sy*(-0.0001)})
	--({\sx*(4.7000)},{\sy*(-0.0004)})
	--({\sx*(4.7100)},{\sy*(-0.0013)})
	--({\sx*(4.7200)},{\sy*(-0.0066)})
	--({\sx*(4.7300)},{\sy*(-0.0009)})
	--({\sx*(4.7400)},{\sy*(-0.0014)})
	--({\sx*(4.7500)},{\sy*(-0.0159)})
	--({\sx*(4.7600)},{\sy*(-0.0082)})
	--({\sx*(4.7700)},{\sy*(-0.0417)})
	--({\sx*(4.7800)},{\sy*(-0.0046)})
	--({\sx*(4.7900)},{\sy*(-0.0098)})
	--({\sx*(4.8000)},{\sy*(-0.0279)})
	--({\sx*(4.8100)},{\sy*(0.0391)})
	--({\sx*(4.8200)},{\sy*(-0.0063)})
	--({\sx*(4.8300)},{\sy*(0.0228)})
	--({\sx*(4.8400)},{\sy*(0.0017)})
	--({\sx*(4.8500)},{\sy*(0.0188)})
	--({\sx*(4.8600)},{\sy*(-0.0495)})
	--({\sx*(4.8700)},{\sy*(0.0501)})
	--({\sx*(4.8800)},{\sy*(-0.0304)})
	--({\sx*(4.8900)},{\sy*(0.1887)})
	--({\sx*(4.9000)},{\sy*(-0.0682)})
	--({\sx*(4.9100)},{\sy*(0.0613)})
	--({\sx*(4.9200)},{\sy*(0.1055)})
	--({\sx*(4.9300)},{\sy*(-0.1549)})
	--({\sx*(4.9400)},{\sy*(0.3320)})
	--({\sx*(4.9500)},{\sy*(0.6483)})
	--({\sx*(4.9600)},{\sy*(-0.0869)})
	--({\sx*(4.9700)},{\sy*(-0.3256)})
	--({\sx*(4.9800)},{\sy*(0.4329)})
	--({\sx*(4.9900)},{\sy*(0.2573)})
	--({\sx*(5.0000)},{\sy*(0.0000)});
}
\def\relfehlerp{
\draw[color=blue,line width=1.4pt,line join=round] ({\sx*(0.000)},{\sy*(0.0000)})
	--({\sx*(0.0100)},{\sy*(0.0000)})
	--({\sx*(0.0200)},{\sy*(-0.0000)})
	--({\sx*(0.0300)},{\sy*(0.0000)})
	--({\sx*(0.0400)},{\sy*(0.0000)})
	--({\sx*(0.0500)},{\sy*(0.0000)})
	--({\sx*(0.0600)},{\sy*(0.0000)})
	--({\sx*(0.0700)},{\sy*(0.0000)})
	--({\sx*(0.0800)},{\sy*(-0.0000)})
	--({\sx*(0.0900)},{\sy*(0.0000)})
	--({\sx*(0.1000)},{\sy*(0.0000)})
	--({\sx*(0.1100)},{\sy*(-0.0000)})
	--({\sx*(0.1200)},{\sy*(0.0000)})
	--({\sx*(0.1300)},{\sy*(-0.0000)})
	--({\sx*(0.1400)},{\sy*(-0.0000)})
	--({\sx*(0.1500)},{\sy*(-0.0000)})
	--({\sx*(0.1600)},{\sy*(-0.0000)})
	--({\sx*(0.1700)},{\sy*(-0.0000)})
	--({\sx*(0.1800)},{\sy*(-0.0000)})
	--({\sx*(0.1900)},{\sy*(-0.0000)})
	--({\sx*(0.2000)},{\sy*(0.0000)})
	--({\sx*(0.2100)},{\sy*(-0.0000)})
	--({\sx*(0.2200)},{\sy*(-0.0000)})
	--({\sx*(0.2300)},{\sy*(-0.0000)})
	--({\sx*(0.2400)},{\sy*(-0.0000)})
	--({\sx*(0.2500)},{\sy*(0.0000)})
	--({\sx*(0.2600)},{\sy*(-0.0000)})
	--({\sx*(0.2700)},{\sy*(-0.0000)})
	--({\sx*(0.2800)},{\sy*(-0.0000)})
	--({\sx*(0.2900)},{\sy*(-0.0000)})
	--({\sx*(0.3000)},{\sy*(-0.0000)})
	--({\sx*(0.3100)},{\sy*(0.0000)})
	--({\sx*(0.3200)},{\sy*(0.0000)})
	--({\sx*(0.3300)},{\sy*(-0.0000)})
	--({\sx*(0.3400)},{\sy*(0.0000)})
	--({\sx*(0.3500)},{\sy*(0.0000)})
	--({\sx*(0.3600)},{\sy*(0.0000)})
	--({\sx*(0.3700)},{\sy*(0.0000)})
	--({\sx*(0.3800)},{\sy*(0.0000)})
	--({\sx*(0.3900)},{\sy*(0.0000)})
	--({\sx*(0.4000)},{\sy*(0.0000)})
	--({\sx*(0.4100)},{\sy*(0.0000)})
	--({\sx*(0.4200)},{\sy*(-0.0000)})
	--({\sx*(0.4300)},{\sy*(0.0000)})
	--({\sx*(0.4400)},{\sy*(0.0000)})
	--({\sx*(0.4500)},{\sy*(-0.0000)})
	--({\sx*(0.4600)},{\sy*(0.0000)})
	--({\sx*(0.4700)},{\sy*(0.0000)})
	--({\sx*(0.4800)},{\sy*(0.0000)})
	--({\sx*(0.4900)},{\sy*(0.0000)})
	--({\sx*(0.5000)},{\sy*(-0.0000)})
	--({\sx*(0.5100)},{\sy*(0.0000)})
	--({\sx*(0.5200)},{\sy*(0.0000)})
	--({\sx*(0.5300)},{\sy*(0.0000)})
	--({\sx*(0.5400)},{\sy*(-0.0000)})
	--({\sx*(0.5500)},{\sy*(-0.0000)})
	--({\sx*(0.5600)},{\sy*(0.0000)})
	--({\sx*(0.5700)},{\sy*(-0.0000)})
	--({\sx*(0.5800)},{\sy*(-0.0000)})
	--({\sx*(0.5900)},{\sy*(-0.0000)})
	--({\sx*(0.6000)},{\sy*(-0.0000)})
	--({\sx*(0.6100)},{\sy*(0.0000)})
	--({\sx*(0.6200)},{\sy*(-0.0000)})
	--({\sx*(0.6300)},{\sy*(0.0000)})
	--({\sx*(0.6400)},{\sy*(0.0000)})
	--({\sx*(0.6500)},{\sy*(0.0000)})
	--({\sx*(0.6600)},{\sy*(-0.0000)})
	--({\sx*(0.6700)},{\sy*(0.0000)})
	--({\sx*(0.6800)},{\sy*(-0.0000)})
	--({\sx*(0.6900)},{\sy*(0.0000)})
	--({\sx*(0.7000)},{\sy*(0.0000)})
	--({\sx*(0.7100)},{\sy*(-0.0000)})
	--({\sx*(0.7200)},{\sy*(-0.0000)})
	--({\sx*(0.7300)},{\sy*(0.0000)})
	--({\sx*(0.7400)},{\sy*(0.0000)})
	--({\sx*(0.7500)},{\sy*(-0.0000)})
	--({\sx*(0.7600)},{\sy*(0.0000)})
	--({\sx*(0.7700)},{\sy*(-0.0000)})
	--({\sx*(0.7800)},{\sy*(-0.0000)})
	--({\sx*(0.7900)},{\sy*(-0.0000)})
	--({\sx*(0.8000)},{\sy*(-0.0000)})
	--({\sx*(0.8100)},{\sy*(-0.0000)})
	--({\sx*(0.8200)},{\sy*(0.0000)})
	--({\sx*(0.8300)},{\sy*(0.0000)})
	--({\sx*(0.8400)},{\sy*(0.0000)})
	--({\sx*(0.8500)},{\sy*(-0.0000)})
	--({\sx*(0.8600)},{\sy*(0.0000)})
	--({\sx*(0.8700)},{\sy*(0.0000)})
	--({\sx*(0.8800)},{\sy*(0.0000)})
	--({\sx*(0.8900)},{\sy*(0.0000)})
	--({\sx*(0.9000)},{\sy*(0.0000)})
	--({\sx*(0.9100)},{\sy*(0.0000)})
	--({\sx*(0.9200)},{\sy*(0.0000)})
	--({\sx*(0.9300)},{\sy*(-0.0000)})
	--({\sx*(0.9400)},{\sy*(0.0000)})
	--({\sx*(0.9500)},{\sy*(-0.0000)})
	--({\sx*(0.9600)},{\sy*(0.0000)})
	--({\sx*(0.9700)},{\sy*(0.0000)})
	--({\sx*(0.9800)},{\sy*(-0.0000)})
	--({\sx*(0.9900)},{\sy*(-0.0000)})
	--({\sx*(1.0000)},{\sy*(0.0000)})
	--({\sx*(1.0100)},{\sy*(0.0000)})
	--({\sx*(1.0200)},{\sy*(0.0000)})
	--({\sx*(1.0300)},{\sy*(-0.0000)})
	--({\sx*(1.0400)},{\sy*(-0.0000)})
	--({\sx*(1.0500)},{\sy*(0.0000)})
	--({\sx*(1.0600)},{\sy*(-0.0000)})
	--({\sx*(1.0700)},{\sy*(-0.0000)})
	--({\sx*(1.0800)},{\sy*(0.0000)})
	--({\sx*(1.0900)},{\sy*(0.0000)})
	--({\sx*(1.1000)},{\sy*(-0.0000)})
	--({\sx*(1.1100)},{\sy*(-0.0000)})
	--({\sx*(1.1200)},{\sy*(-0.0000)})
	--({\sx*(1.1300)},{\sy*(0.0000)})
	--({\sx*(1.1400)},{\sy*(-0.0000)})
	--({\sx*(1.1500)},{\sy*(-0.0000)})
	--({\sx*(1.1600)},{\sy*(-0.0000)})
	--({\sx*(1.1700)},{\sy*(-0.0000)})
	--({\sx*(1.1800)},{\sy*(0.0000)})
	--({\sx*(1.1900)},{\sy*(-0.0000)})
	--({\sx*(1.2000)},{\sy*(-0.0000)})
	--({\sx*(1.2100)},{\sy*(-0.0000)})
	--({\sx*(1.2200)},{\sy*(0.0000)})
	--({\sx*(1.2300)},{\sy*(-0.0000)})
	--({\sx*(1.2400)},{\sy*(-0.0000)})
	--({\sx*(1.2500)},{\sy*(0.0000)})
	--({\sx*(1.2600)},{\sy*(0.0000)})
	--({\sx*(1.2700)},{\sy*(-0.0000)})
	--({\sx*(1.2800)},{\sy*(-0.0000)})
	--({\sx*(1.2900)},{\sy*(0.0000)})
	--({\sx*(1.3000)},{\sy*(0.0000)})
	--({\sx*(1.3100)},{\sy*(-0.0000)})
	--({\sx*(1.3200)},{\sy*(-0.0000)})
	--({\sx*(1.3300)},{\sy*(0.0000)})
	--({\sx*(1.3400)},{\sy*(0.0000)})
	--({\sx*(1.3500)},{\sy*(0.0000)})
	--({\sx*(1.3600)},{\sy*(-0.0000)})
	--({\sx*(1.3700)},{\sy*(-0.0000)})
	--({\sx*(1.3800)},{\sy*(0.0000)})
	--({\sx*(1.3900)},{\sy*(0.0000)})
	--({\sx*(1.4000)},{\sy*(-0.0000)})
	--({\sx*(1.4100)},{\sy*(-0.0000)})
	--({\sx*(1.4200)},{\sy*(-0.0000)})
	--({\sx*(1.4300)},{\sy*(0.0000)})
	--({\sx*(1.4400)},{\sy*(0.0000)})
	--({\sx*(1.4500)},{\sy*(-0.0000)})
	--({\sx*(1.4600)},{\sy*(0.0000)})
	--({\sx*(1.4700)},{\sy*(0.0000)})
	--({\sx*(1.4800)},{\sy*(0.0000)})
	--({\sx*(1.4900)},{\sy*(-0.0000)})
	--({\sx*(1.5000)},{\sy*(-0.0000)})
	--({\sx*(1.5100)},{\sy*(0.0000)})
	--({\sx*(1.5200)},{\sy*(-0.0000)})
	--({\sx*(1.5300)},{\sy*(-0.0000)})
	--({\sx*(1.5400)},{\sy*(0.0000)})
	--({\sx*(1.5500)},{\sy*(0.0000)})
	--({\sx*(1.5600)},{\sy*(0.0000)})
	--({\sx*(1.5700)},{\sy*(-0.0000)})
	--({\sx*(1.5800)},{\sy*(0.0000)})
	--({\sx*(1.5900)},{\sy*(0.0000)})
	--({\sx*(1.6000)},{\sy*(0.0000)})
	--({\sx*(1.6100)},{\sy*(-0.0000)})
	--({\sx*(1.6200)},{\sy*(0.0000)})
	--({\sx*(1.6300)},{\sy*(0.0000)})
	--({\sx*(1.6400)},{\sy*(0.0000)})
	--({\sx*(1.6500)},{\sy*(-0.0000)})
	--({\sx*(1.6600)},{\sy*(-0.0000)})
	--({\sx*(1.6700)},{\sy*(-0.0000)})
	--({\sx*(1.6800)},{\sy*(0.0000)})
	--({\sx*(1.6900)},{\sy*(-0.0000)})
	--({\sx*(1.7000)},{\sy*(-0.0000)})
	--({\sx*(1.7100)},{\sy*(0.0000)})
	--({\sx*(1.7200)},{\sy*(0.0000)})
	--({\sx*(1.7300)},{\sy*(0.0000)})
	--({\sx*(1.7400)},{\sy*(-0.0000)})
	--({\sx*(1.7500)},{\sy*(-0.0000)})
	--({\sx*(1.7600)},{\sy*(0.0000)})
	--({\sx*(1.7700)},{\sy*(-0.0000)})
	--({\sx*(1.7800)},{\sy*(0.0000)})
	--({\sx*(1.7900)},{\sy*(-0.0000)})
	--({\sx*(1.8000)},{\sy*(0.0000)})
	--({\sx*(1.8100)},{\sy*(-0.0000)})
	--({\sx*(1.8200)},{\sy*(-0.0000)})
	--({\sx*(1.8300)},{\sy*(0.0000)})
	--({\sx*(1.8400)},{\sy*(0.0000)})
	--({\sx*(1.8500)},{\sy*(-0.0000)})
	--({\sx*(1.8600)},{\sy*(-0.0000)})
	--({\sx*(1.8700)},{\sy*(-0.0000)})
	--({\sx*(1.8800)},{\sy*(0.0000)})
	--({\sx*(1.8900)},{\sy*(0.0000)})
	--({\sx*(1.9000)},{\sy*(-0.0000)})
	--({\sx*(1.9100)},{\sy*(0.0000)})
	--({\sx*(1.9200)},{\sy*(0.0000)})
	--({\sx*(1.9300)},{\sy*(0.0000)})
	--({\sx*(1.9400)},{\sy*(-0.0000)})
	--({\sx*(1.9500)},{\sy*(-0.0000)})
	--({\sx*(1.9600)},{\sy*(-0.0000)})
	--({\sx*(1.9700)},{\sy*(0.0000)})
	--({\sx*(1.9800)},{\sy*(-0.0000)})
	--({\sx*(1.9900)},{\sy*(0.0000)})
	--({\sx*(2.0000)},{\sy*(-0.0000)})
	--({\sx*(2.0100)},{\sy*(-0.0000)})
	--({\sx*(2.0200)},{\sy*(0.0000)})
	--({\sx*(2.0300)},{\sy*(-0.0000)})
	--({\sx*(2.0400)},{\sy*(-0.0000)})
	--({\sx*(2.0500)},{\sy*(0.0000)})
	--({\sx*(2.0600)},{\sy*(-0.0000)})
	--({\sx*(2.0700)},{\sy*(0.0000)})
	--({\sx*(2.0800)},{\sy*(0.0000)})
	--({\sx*(2.0900)},{\sy*(0.0000)})
	--({\sx*(2.1000)},{\sy*(-0.0000)})
	--({\sx*(2.1100)},{\sy*(0.0000)})
	--({\sx*(2.1200)},{\sy*(-0.0000)})
	--({\sx*(2.1300)},{\sy*(-0.0000)})
	--({\sx*(2.1400)},{\sy*(-0.0000)})
	--({\sx*(2.1500)},{\sy*(0.0000)})
	--({\sx*(2.1600)},{\sy*(-0.0000)})
	--({\sx*(2.1700)},{\sy*(0.0000)})
	--({\sx*(2.1800)},{\sy*(0.0000)})
	--({\sx*(2.1900)},{\sy*(-0.0000)})
	--({\sx*(2.2000)},{\sy*(-0.0000)})
	--({\sx*(2.2100)},{\sy*(0.0000)})
	--({\sx*(2.2200)},{\sy*(-0.0000)})
	--({\sx*(2.2300)},{\sy*(-0.0000)})
	--({\sx*(2.2400)},{\sy*(-0.0000)})
	--({\sx*(2.2500)},{\sy*(0.0000)})
	--({\sx*(2.2600)},{\sy*(0.0000)})
	--({\sx*(2.2700)},{\sy*(0.0000)})
	--({\sx*(2.2800)},{\sy*(0.0000)})
	--({\sx*(2.2900)},{\sy*(0.0000)})
	--({\sx*(2.3000)},{\sy*(-0.0000)})
	--({\sx*(2.3100)},{\sy*(0.0000)})
	--({\sx*(2.3200)},{\sy*(0.0000)})
	--({\sx*(2.3300)},{\sy*(0.0000)})
	--({\sx*(2.3400)},{\sy*(-0.0000)})
	--({\sx*(2.3500)},{\sy*(0.0000)})
	--({\sx*(2.3600)},{\sy*(0.0000)})
	--({\sx*(2.3700)},{\sy*(0.0000)})
	--({\sx*(2.3800)},{\sy*(-0.0000)})
	--({\sx*(2.3900)},{\sy*(0.0000)})
	--({\sx*(2.4000)},{\sy*(0.0000)})
	--({\sx*(2.4100)},{\sy*(-0.0000)})
	--({\sx*(2.4200)},{\sy*(0.0000)})
	--({\sx*(2.4300)},{\sy*(0.0000)})
	--({\sx*(2.4400)},{\sy*(-0.0000)})
	--({\sx*(2.4500)},{\sy*(0.0000)})
	--({\sx*(2.4600)},{\sy*(0.0000)})
	--({\sx*(2.4700)},{\sy*(0.0000)})
	--({\sx*(2.4800)},{\sy*(-0.0000)})
	--({\sx*(2.4900)},{\sy*(0.0000)})
	--({\sx*(2.5000)},{\sy*(0.0000)})
	--({\sx*(2.5100)},{\sy*(0.0000)})
	--({\sx*(2.5200)},{\sy*(-0.0000)})
	--({\sx*(2.5300)},{\sy*(0.0000)})
	--({\sx*(2.5400)},{\sy*(0.0000)})
	--({\sx*(2.5500)},{\sy*(0.0000)})
	--({\sx*(2.5600)},{\sy*(-0.0000)})
	--({\sx*(2.5700)},{\sy*(0.0000)})
	--({\sx*(2.5800)},{\sy*(0.0000)})
	--({\sx*(2.5900)},{\sy*(-0.0000)})
	--({\sx*(2.6000)},{\sy*(0.0000)})
	--({\sx*(2.6100)},{\sy*(-0.0000)})
	--({\sx*(2.6200)},{\sy*(0.0000)})
	--({\sx*(2.6300)},{\sy*(0.0000)})
	--({\sx*(2.6400)},{\sy*(0.0000)})
	--({\sx*(2.6500)},{\sy*(0.0000)})
	--({\sx*(2.6600)},{\sy*(-0.0000)})
	--({\sx*(2.6700)},{\sy*(-0.0000)})
	--({\sx*(2.6800)},{\sy*(0.0000)})
	--({\sx*(2.6900)},{\sy*(0.0000)})
	--({\sx*(2.7000)},{\sy*(0.0000)})
	--({\sx*(2.7100)},{\sy*(0.0000)})
	--({\sx*(2.7200)},{\sy*(0.0000)})
	--({\sx*(2.7300)},{\sy*(0.0000)})
	--({\sx*(2.7400)},{\sy*(0.0000)})
	--({\sx*(2.7500)},{\sy*(0.0000)})
	--({\sx*(2.7600)},{\sy*(0.0000)})
	--({\sx*(2.7700)},{\sy*(0.0000)})
	--({\sx*(2.7800)},{\sy*(0.0000)})
	--({\sx*(2.7900)},{\sy*(0.0000)})
	--({\sx*(2.8000)},{\sy*(-0.0000)})
	--({\sx*(2.8100)},{\sy*(-0.0000)})
	--({\sx*(2.8200)},{\sy*(0.0000)})
	--({\sx*(2.8300)},{\sy*(0.0000)})
	--({\sx*(2.8400)},{\sy*(0.0000)})
	--({\sx*(2.8500)},{\sy*(0.0000)})
	--({\sx*(2.8600)},{\sy*(-0.0000)})
	--({\sx*(2.8700)},{\sy*(-0.0000)})
	--({\sx*(2.8800)},{\sy*(-0.0000)})
	--({\sx*(2.8900)},{\sy*(-0.0000)})
	--({\sx*(2.9000)},{\sy*(0.0000)})
	--({\sx*(2.9100)},{\sy*(0.0000)})
	--({\sx*(2.9200)},{\sy*(-0.0000)})
	--({\sx*(2.9300)},{\sy*(-0.0000)})
	--({\sx*(2.9400)},{\sy*(-0.0000)})
	--({\sx*(2.9500)},{\sy*(-0.0000)})
	--({\sx*(2.9600)},{\sy*(-0.0000)})
	--({\sx*(2.9700)},{\sy*(0.0000)})
	--({\sx*(2.9800)},{\sy*(0.0000)})
	--({\sx*(2.9900)},{\sy*(-0.0000)})
	--({\sx*(3.0000)},{\sy*(-0.0000)})
	--({\sx*(3.0100)},{\sy*(0.0000)})
	--({\sx*(3.0200)},{\sy*(0.0000)})
	--({\sx*(3.0300)},{\sy*(-0.0000)})
	--({\sx*(3.0400)},{\sy*(-0.0000)})
	--({\sx*(3.0500)},{\sy*(0.0000)})
	--({\sx*(3.0600)},{\sy*(0.0000)})
	--({\sx*(3.0700)},{\sy*(-0.0000)})
	--({\sx*(3.0800)},{\sy*(0.0000)})
	--({\sx*(3.0900)},{\sy*(0.0000)})
	--({\sx*(3.1000)},{\sy*(0.0000)})
	--({\sx*(3.1100)},{\sy*(-0.0000)})
	--({\sx*(3.1200)},{\sy*(0.0000)})
	--({\sx*(3.1300)},{\sy*(-0.0000)})
	--({\sx*(3.1400)},{\sy*(0.0000)})
	--({\sx*(3.1500)},{\sy*(0.0000)})
	--({\sx*(3.1600)},{\sy*(-0.0000)})
	--({\sx*(3.1700)},{\sy*(-0.0000)})
	--({\sx*(3.1800)},{\sy*(0.0000)})
	--({\sx*(3.1900)},{\sy*(0.0000)})
	--({\sx*(3.2000)},{\sy*(-0.0000)})
	--({\sx*(3.2100)},{\sy*(-0.0000)})
	--({\sx*(3.2200)},{\sy*(-0.0000)})
	--({\sx*(3.2300)},{\sy*(-0.0000)})
	--({\sx*(3.2400)},{\sy*(0.0000)})
	--({\sx*(3.2500)},{\sy*(-0.0000)})
	--({\sx*(3.2600)},{\sy*(-0.0000)})
	--({\sx*(3.2700)},{\sy*(-0.0000)})
	--({\sx*(3.2800)},{\sy*(0.0000)})
	--({\sx*(3.2900)},{\sy*(-0.0000)})
	--({\sx*(3.3000)},{\sy*(-0.0000)})
	--({\sx*(3.3100)},{\sy*(-0.0000)})
	--({\sx*(3.3200)},{\sy*(-0.0000)})
	--({\sx*(3.3300)},{\sy*(-0.0000)})
	--({\sx*(3.3400)},{\sy*(0.0000)})
	--({\sx*(3.3500)},{\sy*(0.0000)})
	--({\sx*(3.3600)},{\sy*(-0.0000)})
	--({\sx*(3.3700)},{\sy*(-0.0000)})
	--({\sx*(3.3800)},{\sy*(0.0000)})
	--({\sx*(3.3900)},{\sy*(-0.0000)})
	--({\sx*(3.4000)},{\sy*(0.0000)})
	--({\sx*(3.4100)},{\sy*(-0.0000)})
	--({\sx*(3.4200)},{\sy*(-0.0000)})
	--({\sx*(3.4300)},{\sy*(-0.0000)})
	--({\sx*(3.4400)},{\sy*(0.0000)})
	--({\sx*(3.4500)},{\sy*(-0.0000)})
	--({\sx*(3.4600)},{\sy*(0.0000)})
	--({\sx*(3.4700)},{\sy*(0.0000)})
	--({\sx*(3.4800)},{\sy*(-0.0000)})
	--({\sx*(3.4900)},{\sy*(-0.0000)})
	--({\sx*(3.5000)},{\sy*(-0.0000)})
	--({\sx*(3.5100)},{\sy*(-0.0000)})
	--({\sx*(3.5200)},{\sy*(-0.0000)})
	--({\sx*(3.5300)},{\sy*(0.0000)})
	--({\sx*(3.5400)},{\sy*(-0.0000)})
	--({\sx*(3.5500)},{\sy*(0.0000)})
	--({\sx*(3.5600)},{\sy*(0.0000)})
	--({\sx*(3.5700)},{\sy*(-0.0000)})
	--({\sx*(3.5800)},{\sy*(-0.0000)})
	--({\sx*(3.5900)},{\sy*(-0.0000)})
	--({\sx*(3.6000)},{\sy*(-0.0000)})
	--({\sx*(3.6100)},{\sy*(0.0000)})
	--({\sx*(3.6200)},{\sy*(0.0000)})
	--({\sx*(3.6300)},{\sy*(0.0000)})
	--({\sx*(3.6400)},{\sy*(0.0000)})
	--({\sx*(3.6500)},{\sy*(-0.0000)})
	--({\sx*(3.6600)},{\sy*(0.0000)})
	--({\sx*(3.6700)},{\sy*(-0.0000)})
	--({\sx*(3.6800)},{\sy*(0.0000)})
	--({\sx*(3.6900)},{\sy*(0.0000)})
	--({\sx*(3.7000)},{\sy*(-0.0000)})
	--({\sx*(3.7100)},{\sy*(-0.0000)})
	--({\sx*(3.7200)},{\sy*(0.0000)})
	--({\sx*(3.7300)},{\sy*(0.0000)})
	--({\sx*(3.7400)},{\sy*(0.0000)})
	--({\sx*(3.7500)},{\sy*(0.0000)})
	--({\sx*(3.7600)},{\sy*(-0.0000)})
	--({\sx*(3.7700)},{\sy*(-0.0000)})
	--({\sx*(3.7800)},{\sy*(-0.0000)})
	--({\sx*(3.7900)},{\sy*(-0.0000)})
	--({\sx*(3.8000)},{\sy*(-0.0000)})
	--({\sx*(3.8100)},{\sy*(0.0000)})
	--({\sx*(3.8200)},{\sy*(0.0000)})
	--({\sx*(3.8300)},{\sy*(0.0000)})
	--({\sx*(3.8400)},{\sy*(-0.0000)})
	--({\sx*(3.8500)},{\sy*(0.0000)})
	--({\sx*(3.8600)},{\sy*(-0.0000)})
	--({\sx*(3.8700)},{\sy*(-0.0000)})
	--({\sx*(3.8800)},{\sy*(0.0000)})
	--({\sx*(3.8900)},{\sy*(0.0000)})
	--({\sx*(3.9000)},{\sy*(-0.0000)})
	--({\sx*(3.9100)},{\sy*(0.0000)})
	--({\sx*(3.9200)},{\sy*(-0.0000)})
	--({\sx*(3.9300)},{\sy*(0.0000)})
	--({\sx*(3.9400)},{\sy*(0.0000)})
	--({\sx*(3.9500)},{\sy*(-0.0000)})
	--({\sx*(3.9600)},{\sy*(0.0000)})
	--({\sx*(3.9700)},{\sy*(0.0000)})
	--({\sx*(3.9800)},{\sy*(-0.0000)})
	--({\sx*(3.9900)},{\sy*(0.0000)})
	--({\sx*(4.0000)},{\sy*(0.0000)})
	--({\sx*(4.0100)},{\sy*(-0.0000)})
	--({\sx*(4.0200)},{\sy*(0.0000)})
	--({\sx*(4.0300)},{\sy*(0.0000)})
	--({\sx*(4.0400)},{\sy*(0.0000)})
	--({\sx*(4.0500)},{\sy*(0.0000)})
	--({\sx*(4.0600)},{\sy*(0.0000)})
	--({\sx*(4.0700)},{\sy*(-0.0000)})
	--({\sx*(4.0800)},{\sy*(0.0000)})
	--({\sx*(4.0900)},{\sy*(-0.0000)})
	--({\sx*(4.1000)},{\sy*(0.0000)})
	--({\sx*(4.1100)},{\sy*(0.0000)})
	--({\sx*(4.1200)},{\sy*(0.0000)})
	--({\sx*(4.1300)},{\sy*(0.0000)})
	--({\sx*(4.1400)},{\sy*(-0.0000)})
	--({\sx*(4.1500)},{\sy*(-0.0000)})
	--({\sx*(4.1600)},{\sy*(-0.0000)})
	--({\sx*(4.1700)},{\sy*(-0.0000)})
	--({\sx*(4.1800)},{\sy*(-0.0000)})
	--({\sx*(4.1900)},{\sy*(0.0000)})
	--({\sx*(4.2000)},{\sy*(0.0000)})
	--({\sx*(4.2100)},{\sy*(-0.0000)})
	--({\sx*(4.2200)},{\sy*(0.0000)})
	--({\sx*(4.2300)},{\sy*(0.0000)})
	--({\sx*(4.2400)},{\sy*(-0.0000)})
	--({\sx*(4.2500)},{\sy*(-0.0000)})
	--({\sx*(4.2600)},{\sy*(0.0000)})
	--({\sx*(4.2700)},{\sy*(0.0000)})
	--({\sx*(4.2800)},{\sy*(-0.0000)})
	--({\sx*(4.2900)},{\sy*(-0.0000)})
	--({\sx*(4.3000)},{\sy*(0.0000)})
	--({\sx*(4.3100)},{\sy*(0.0000)})
	--({\sx*(4.3200)},{\sy*(-0.0000)})
	--({\sx*(4.3300)},{\sy*(0.0000)})
	--({\sx*(4.3400)},{\sy*(0.0000)})
	--({\sx*(4.3500)},{\sy*(-0.0000)})
	--({\sx*(4.3600)},{\sy*(-0.0000)})
	--({\sx*(4.3700)},{\sy*(-0.0000)})
	--({\sx*(4.3800)},{\sy*(-0.0000)})
	--({\sx*(4.3900)},{\sy*(0.0000)})
	--({\sx*(4.4000)},{\sy*(-0.0000)})
	--({\sx*(4.4100)},{\sy*(0.0000)})
	--({\sx*(4.4200)},{\sy*(-0.0000)})
	--({\sx*(4.4300)},{\sy*(0.0000)})
	--({\sx*(4.4400)},{\sy*(0.0000)})
	--({\sx*(4.4500)},{\sy*(-0.0000)})
	--({\sx*(4.4600)},{\sy*(0.0000)})
	--({\sx*(4.4700)},{\sy*(0.0000)})
	--({\sx*(4.4800)},{\sy*(0.0000)})
	--({\sx*(4.4900)},{\sy*(-0.0000)})
	--({\sx*(4.5000)},{\sy*(0.0000)})
	--({\sx*(4.5100)},{\sy*(0.0000)})
	--({\sx*(4.5200)},{\sy*(0.0000)})
	--({\sx*(4.5300)},{\sy*(0.0000)})
	--({\sx*(4.5400)},{\sy*(0.0000)})
	--({\sx*(4.5500)},{\sy*(-0.0000)})
	--({\sx*(4.5600)},{\sy*(0.0000)})
	--({\sx*(4.5700)},{\sy*(-0.0000)})
	--({\sx*(4.5800)},{\sy*(-0.0000)})
	--({\sx*(4.5900)},{\sy*(0.0000)})
	--({\sx*(4.6000)},{\sy*(0.0000)})
	--({\sx*(4.6100)},{\sy*(0.0000)})
	--({\sx*(4.6200)},{\sy*(0.0000)})
	--({\sx*(4.6300)},{\sy*(-0.0000)})
	--({\sx*(4.6400)},{\sy*(0.0000)})
	--({\sx*(4.6500)},{\sy*(-0.0000)})
	--({\sx*(4.6600)},{\sy*(-0.0000)})
	--({\sx*(4.6700)},{\sy*(-0.0000)})
	--({\sx*(4.6800)},{\sy*(0.0000)})
	--({\sx*(4.6900)},{\sy*(-0.0000)})
	--({\sx*(4.7000)},{\sy*(-0.0000)})
	--({\sx*(4.7100)},{\sy*(-0.0000)})
	--({\sx*(4.7200)},{\sy*(-0.0000)})
	--({\sx*(4.7300)},{\sy*(-0.0000)})
	--({\sx*(4.7400)},{\sy*(-0.0000)})
	--({\sx*(4.7500)},{\sy*(-0.0000)})
	--({\sx*(4.7600)},{\sy*(-0.0000)})
	--({\sx*(4.7700)},{\sy*(-0.0000)})
	--({\sx*(4.7800)},{\sy*(-0.0000)})
	--({\sx*(4.7900)},{\sy*(-0.0000)})
	--({\sx*(4.8000)},{\sy*(-0.0000)})
	--({\sx*(4.8100)},{\sy*(0.0000)})
	--({\sx*(4.8200)},{\sy*(-0.0000)})
	--({\sx*(4.8300)},{\sy*(0.0000)})
	--({\sx*(4.8400)},{\sy*(0.0000)})
	--({\sx*(4.8500)},{\sy*(0.0000)})
	--({\sx*(4.8600)},{\sy*(-0.0000)})
	--({\sx*(4.8700)},{\sy*(0.0000)})
	--({\sx*(4.8800)},{\sy*(-0.0000)})
	--({\sx*(4.8900)},{\sy*(0.0000)})
	--({\sx*(4.9000)},{\sy*(-0.0000)})
	--({\sx*(4.9100)},{\sy*(0.0000)})
	--({\sx*(4.9200)},{\sy*(0.0000)})
	--({\sx*(4.9300)},{\sy*(-0.0000)})
	--({\sx*(4.9400)},{\sy*(0.0000)})
	--({\sx*(4.9500)},{\sy*(0.0000)})
	--({\sx*(4.9600)},{\sy*(-0.0000)})
	--({\sx*(4.9700)},{\sy*(-0.0000)})
	--({\sx*(4.9800)},{\sy*(0.0000)})
	--({\sx*(4.9900)},{\sy*(0.0000)})
	--({\sx*(5.0000)},{\sy*(0.0000)});
}
\def\xwerteq{
\fill[color=red] (0.0000,0) circle[radius={0.07/\skala}];
\fill[color=red] (0.1471,0) circle[radius={0.07/\skala}];
\fill[color=red] (0.2941,0) circle[radius={0.07/\skala}];
\fill[color=red] (0.4412,0) circle[radius={0.07/\skala}];
\fill[color=red] (0.5882,0) circle[radius={0.07/\skala}];
\fill[color=red] (0.7353,0) circle[radius={0.07/\skala}];
\fill[color=red] (0.8824,0) circle[radius={0.07/\skala}];
\fill[color=red] (1.0294,0) circle[radius={0.07/\skala}];
\fill[color=red] (1.1765,0) circle[radius={0.07/\skala}];
\fill[color=red] (1.3235,0) circle[radius={0.07/\skala}];
\fill[color=red] (1.4706,0) circle[radius={0.07/\skala}];
\fill[color=red] (1.6176,0) circle[radius={0.07/\skala}];
\fill[color=red] (1.7647,0) circle[radius={0.07/\skala}];
\fill[color=red] (1.9118,0) circle[radius={0.07/\skala}];
\fill[color=red] (2.0588,0) circle[radius={0.07/\skala}];
\fill[color=red] (2.2059,0) circle[radius={0.07/\skala}];
\fill[color=red] (2.3529,0) circle[radius={0.07/\skala}];
\fill[color=red] (2.5000,0) circle[radius={0.07/\skala}];
\fill[color=red] (2.6471,0) circle[radius={0.07/\skala}];
\fill[color=red] (2.7941,0) circle[radius={0.07/\skala}];
\fill[color=red] (2.9412,0) circle[radius={0.07/\skala}];
\fill[color=red] (3.0882,0) circle[radius={0.07/\skala}];
\fill[color=red] (3.2353,0) circle[radius={0.07/\skala}];
\fill[color=red] (3.3824,0) circle[radius={0.07/\skala}];
\fill[color=red] (3.5294,0) circle[radius={0.07/\skala}];
\fill[color=red] (3.6765,0) circle[radius={0.07/\skala}];
\fill[color=red] (3.8235,0) circle[radius={0.07/\skala}];
\fill[color=red] (3.9706,0) circle[radius={0.07/\skala}];
\fill[color=red] (4.1176,0) circle[radius={0.07/\skala}];
\fill[color=red] (4.2647,0) circle[radius={0.07/\skala}];
\fill[color=red] (4.4118,0) circle[radius={0.07/\skala}];
\fill[color=red] (4.5588,0) circle[radius={0.07/\skala}];
\fill[color=red] (4.7059,0) circle[radius={0.07/\skala}];
\fill[color=red] (4.8529,0) circle[radius={0.07/\skala}];
\fill[color=red] (5.0000,0) circle[radius={0.07/\skala}];
}
\def\punkteq{34}
\def\maxfehlerq{4.516\cdot 10^{-10}}
\def\fehlerq{
\draw[color=red,line width=1.4pt,line join=round] ({\sx*(0.000)},{\sy*(0.0000)})
	--({\sx*(0.0100)},{\sy*(-0.4228)})
	--({\sx*(0.0200)},{\sy*(0.1368)})
	--({\sx*(0.0300)},{\sy*(-0.2996)})
	--({\sx*(0.0400)},{\sy*(1.0000)})
	--({\sx*(0.0500)},{\sy*(0.6637)})
	--({\sx*(0.0600)},{\sy*(0.9429)})
	--({\sx*(0.0700)},{\sy*(-0.8868)})
	--({\sx*(0.0800)},{\sy*(0.4891)})
	--({\sx*(0.0900)},{\sy*(0.0527)})
	--({\sx*(0.1000)},{\sy*(-0.1516)})
	--({\sx*(0.1100)},{\sy*(0.0486)})
	--({\sx*(0.1200)},{\sy*(0.0626)})
	--({\sx*(0.1300)},{\sy*(-0.0260)})
	--({\sx*(0.1400)},{\sy*(0.0191)})
	--({\sx*(0.1500)},{\sy*(0.0013)})
	--({\sx*(0.1600)},{\sy*(-0.0406)})
	--({\sx*(0.1700)},{\sy*(0.0003)})
	--({\sx*(0.1800)},{\sy*(-0.0119)})
	--({\sx*(0.1900)},{\sy*(0.0042)})
	--({\sx*(0.2000)},{\sy*(0.0017)})
	--({\sx*(0.2100)},{\sy*(0.0235)})
	--({\sx*(0.2200)},{\sy*(0.0164)})
	--({\sx*(0.2300)},{\sy*(-0.0133)})
	--({\sx*(0.2400)},{\sy*(-0.0196)})
	--({\sx*(0.2500)},{\sy*(-0.0147)})
	--({\sx*(0.2600)},{\sy*(0.0098)})
	--({\sx*(0.2700)},{\sy*(-0.0030)})
	--({\sx*(0.2800)},{\sy*(-0.0037)})
	--({\sx*(0.2900)},{\sy*(-0.0016)})
	--({\sx*(0.3000)},{\sy*(0.0003)})
	--({\sx*(0.3100)},{\sy*(0.0017)})
	--({\sx*(0.3200)},{\sy*(0.0009)})
	--({\sx*(0.3300)},{\sy*(0.0026)})
	--({\sx*(0.3400)},{\sy*(-0.0021)})
	--({\sx*(0.3500)},{\sy*(0.0006)})
	--({\sx*(0.3600)},{\sy*(0.0018)})
	--({\sx*(0.3700)},{\sy*(-0.0003)})
	--({\sx*(0.3800)},{\sy*(0.0006)})
	--({\sx*(0.3900)},{\sy*(-0.0005)})
	--({\sx*(0.4000)},{\sy*(0.0009)})
	--({\sx*(0.4100)},{\sy*(0.0008)})
	--({\sx*(0.4200)},{\sy*(0.0000)})
	--({\sx*(0.4300)},{\sy*(0.0002)})
	--({\sx*(0.4400)},{\sy*(0.0000)})
	--({\sx*(0.4500)},{\sy*(-0.0001)})
	--({\sx*(0.4600)},{\sy*(0.0001)})
	--({\sx*(0.4700)},{\sy*(-0.0001)})
	--({\sx*(0.4800)},{\sy*(-0.0002)})
	--({\sx*(0.4900)},{\sy*(0.0001)})
	--({\sx*(0.5000)},{\sy*(-0.0004)})
	--({\sx*(0.5100)},{\sy*(-0.0001)})
	--({\sx*(0.5200)},{\sy*(-0.0002)})
	--({\sx*(0.5300)},{\sy*(-0.0000)})
	--({\sx*(0.5400)},{\sy*(-0.0001)})
	--({\sx*(0.5500)},{\sy*(-0.0000)})
	--({\sx*(0.5600)},{\sy*(-0.0001)})
	--({\sx*(0.5700)},{\sy*(0.0001)})
	--({\sx*(0.5800)},{\sy*(-0.0000)})
	--({\sx*(0.5900)},{\sy*(0.0000)})
	--({\sx*(0.6000)},{\sy*(0.0000)})
	--({\sx*(0.6100)},{\sy*(0.0000)})
	--({\sx*(0.6200)},{\sy*(-0.0000)})
	--({\sx*(0.6300)},{\sy*(-0.0000)})
	--({\sx*(0.6400)},{\sy*(0.0000)})
	--({\sx*(0.6500)},{\sy*(-0.0000)})
	--({\sx*(0.6600)},{\sy*(-0.0000)})
	--({\sx*(0.6700)},{\sy*(-0.0000)})
	--({\sx*(0.6800)},{\sy*(0.0000)})
	--({\sx*(0.6900)},{\sy*(0.0000)})
	--({\sx*(0.7000)},{\sy*(0.0000)})
	--({\sx*(0.7100)},{\sy*(0.0000)})
	--({\sx*(0.7200)},{\sy*(0.0000)})
	--({\sx*(0.7300)},{\sy*(0.0000)})
	--({\sx*(0.7400)},{\sy*(-0.0000)})
	--({\sx*(0.7500)},{\sy*(-0.0000)})
	--({\sx*(0.7600)},{\sy*(-0.0000)})
	--({\sx*(0.7700)},{\sy*(0.0000)})
	--({\sx*(0.7800)},{\sy*(0.0000)})
	--({\sx*(0.7900)},{\sy*(-0.0000)})
	--({\sx*(0.8000)},{\sy*(0.0000)})
	--({\sx*(0.8100)},{\sy*(-0.0000)})
	--({\sx*(0.8200)},{\sy*(-0.0000)})
	--({\sx*(0.8300)},{\sy*(0.0000)})
	--({\sx*(0.8400)},{\sy*(-0.0000)})
	--({\sx*(0.8500)},{\sy*(-0.0000)})
	--({\sx*(0.8600)},{\sy*(0.0000)})
	--({\sx*(0.8700)},{\sy*(-0.0000)})
	--({\sx*(0.8800)},{\sy*(0.0000)})
	--({\sx*(0.8900)},{\sy*(-0.0000)})
	--({\sx*(0.9000)},{\sy*(0.0000)})
	--({\sx*(0.9100)},{\sy*(-0.0000)})
	--({\sx*(0.9200)},{\sy*(0.0000)})
	--({\sx*(0.9300)},{\sy*(-0.0000)})
	--({\sx*(0.9400)},{\sy*(0.0000)})
	--({\sx*(0.9500)},{\sy*(-0.0000)})
	--({\sx*(0.9600)},{\sy*(0.0000)})
	--({\sx*(0.9700)},{\sy*(-0.0000)})
	--({\sx*(0.9800)},{\sy*(0.0000)})
	--({\sx*(0.9900)},{\sy*(-0.0000)})
	--({\sx*(1.0000)},{\sy*(-0.0000)})
	--({\sx*(1.0100)},{\sy*(-0.0000)})
	--({\sx*(1.0200)},{\sy*(0.0000)})
	--({\sx*(1.0300)},{\sy*(-0.0000)})
	--({\sx*(1.0400)},{\sy*(0.0000)})
	--({\sx*(1.0500)},{\sy*(0.0000)})
	--({\sx*(1.0600)},{\sy*(0.0000)})
	--({\sx*(1.0700)},{\sy*(-0.0000)})
	--({\sx*(1.0800)},{\sy*(0.0000)})
	--({\sx*(1.0900)},{\sy*(-0.0000)})
	--({\sx*(1.1000)},{\sy*(0.0000)})
	--({\sx*(1.1100)},{\sy*(0.0000)})
	--({\sx*(1.1200)},{\sy*(0.0000)})
	--({\sx*(1.1300)},{\sy*(0.0000)})
	--({\sx*(1.1400)},{\sy*(0.0000)})
	--({\sx*(1.1500)},{\sy*(0.0000)})
	--({\sx*(1.1600)},{\sy*(0.0000)})
	--({\sx*(1.1700)},{\sy*(0.0000)})
	--({\sx*(1.1800)},{\sy*(0.0000)})
	--({\sx*(1.1900)},{\sy*(0.0000)})
	--({\sx*(1.2000)},{\sy*(0.0000)})
	--({\sx*(1.2100)},{\sy*(-0.0000)})
	--({\sx*(1.2200)},{\sy*(0.0000)})
	--({\sx*(1.2300)},{\sy*(0.0000)})
	--({\sx*(1.2400)},{\sy*(0.0000)})
	--({\sx*(1.2500)},{\sy*(0.0000)})
	--({\sx*(1.2600)},{\sy*(0.0000)})
	--({\sx*(1.2700)},{\sy*(-0.0000)})
	--({\sx*(1.2800)},{\sy*(-0.0000)})
	--({\sx*(1.2900)},{\sy*(0.0000)})
	--({\sx*(1.3000)},{\sy*(0.0000)})
	--({\sx*(1.3100)},{\sy*(-0.0000)})
	--({\sx*(1.3200)},{\sy*(-0.0000)})
	--({\sx*(1.3300)},{\sy*(-0.0000)})
	--({\sx*(1.3400)},{\sy*(0.0000)})
	--({\sx*(1.3500)},{\sy*(-0.0000)})
	--({\sx*(1.3600)},{\sy*(-0.0000)})
	--({\sx*(1.3700)},{\sy*(-0.0000)})
	--({\sx*(1.3800)},{\sy*(-0.0000)})
	--({\sx*(1.3900)},{\sy*(0.0000)})
	--({\sx*(1.4000)},{\sy*(-0.0000)})
	--({\sx*(1.4100)},{\sy*(-0.0000)})
	--({\sx*(1.4200)},{\sy*(-0.0000)})
	--({\sx*(1.4300)},{\sy*(-0.0000)})
	--({\sx*(1.4400)},{\sy*(0.0000)})
	--({\sx*(1.4500)},{\sy*(-0.0000)})
	--({\sx*(1.4600)},{\sy*(-0.0000)})
	--({\sx*(1.4700)},{\sy*(0.0000)})
	--({\sx*(1.4800)},{\sy*(0.0000)})
	--({\sx*(1.4900)},{\sy*(-0.0000)})
	--({\sx*(1.5000)},{\sy*(0.0000)})
	--({\sx*(1.5100)},{\sy*(0.0000)})
	--({\sx*(1.5200)},{\sy*(-0.0000)})
	--({\sx*(1.5300)},{\sy*(-0.0000)})
	--({\sx*(1.5400)},{\sy*(0.0000)})
	--({\sx*(1.5500)},{\sy*(0.0000)})
	--({\sx*(1.5600)},{\sy*(0.0000)})
	--({\sx*(1.5700)},{\sy*(-0.0000)})
	--({\sx*(1.5800)},{\sy*(0.0000)})
	--({\sx*(1.5900)},{\sy*(0.0000)})
	--({\sx*(1.6000)},{\sy*(0.0000)})
	--({\sx*(1.6100)},{\sy*(-0.0000)})
	--({\sx*(1.6200)},{\sy*(0.0000)})
	--({\sx*(1.6300)},{\sy*(0.0000)})
	--({\sx*(1.6400)},{\sy*(0.0000)})
	--({\sx*(1.6500)},{\sy*(-0.0000)})
	--({\sx*(1.6600)},{\sy*(0.0000)})
	--({\sx*(1.6700)},{\sy*(-0.0000)})
	--({\sx*(1.6800)},{\sy*(0.0000)})
	--({\sx*(1.6900)},{\sy*(-0.0000)})
	--({\sx*(1.7000)},{\sy*(-0.0000)})
	--({\sx*(1.7100)},{\sy*(-0.0000)})
	--({\sx*(1.7200)},{\sy*(0.0000)})
	--({\sx*(1.7300)},{\sy*(0.0000)})
	--({\sx*(1.7400)},{\sy*(0.0000)})
	--({\sx*(1.7500)},{\sy*(0.0000)})
	--({\sx*(1.7600)},{\sy*(0.0000)})
	--({\sx*(1.7700)},{\sy*(0.0000)})
	--({\sx*(1.7800)},{\sy*(-0.0000)})
	--({\sx*(1.7900)},{\sy*(0.0000)})
	--({\sx*(1.8000)},{\sy*(0.0000)})
	--({\sx*(1.8100)},{\sy*(0.0000)})
	--({\sx*(1.8200)},{\sy*(-0.0000)})
	--({\sx*(1.8300)},{\sy*(0.0000)})
	--({\sx*(1.8400)},{\sy*(0.0000)})
	--({\sx*(1.8500)},{\sy*(0.0000)})
	--({\sx*(1.8600)},{\sy*(0.0000)})
	--({\sx*(1.8700)},{\sy*(0.0000)})
	--({\sx*(1.8800)},{\sy*(0.0000)})
	--({\sx*(1.8900)},{\sy*(0.0000)})
	--({\sx*(1.9000)},{\sy*(0.0000)})
	--({\sx*(1.9100)},{\sy*(0.0000)})
	--({\sx*(1.9200)},{\sy*(0.0000)})
	--({\sx*(1.9300)},{\sy*(0.0000)})
	--({\sx*(1.9400)},{\sy*(0.0000)})
	--({\sx*(1.9500)},{\sy*(0.0000)})
	--({\sx*(1.9600)},{\sy*(-0.0000)})
	--({\sx*(1.9700)},{\sy*(0.0000)})
	--({\sx*(1.9800)},{\sy*(0.0000)})
	--({\sx*(1.9900)},{\sy*(-0.0000)})
	--({\sx*(2.0000)},{\sy*(0.0000)})
	--({\sx*(2.0100)},{\sy*(-0.0000)})
	--({\sx*(2.0200)},{\sy*(0.0000)})
	--({\sx*(2.0300)},{\sy*(-0.0000)})
	--({\sx*(2.0400)},{\sy*(0.0000)})
	--({\sx*(2.0500)},{\sy*(0.0000)})
	--({\sx*(2.0600)},{\sy*(0.0000)})
	--({\sx*(2.0700)},{\sy*(-0.0000)})
	--({\sx*(2.0800)},{\sy*(0.0000)})
	--({\sx*(2.0900)},{\sy*(0.0000)})
	--({\sx*(2.1000)},{\sy*(-0.0000)})
	--({\sx*(2.1100)},{\sy*(0.0000)})
	--({\sx*(2.1200)},{\sy*(0.0000)})
	--({\sx*(2.1300)},{\sy*(-0.0000)})
	--({\sx*(2.1400)},{\sy*(0.0000)})
	--({\sx*(2.1500)},{\sy*(0.0000)})
	--({\sx*(2.1600)},{\sy*(-0.0000)})
	--({\sx*(2.1700)},{\sy*(0.0000)})
	--({\sx*(2.1800)},{\sy*(0.0000)})
	--({\sx*(2.1900)},{\sy*(0.0000)})
	--({\sx*(2.2000)},{\sy*(-0.0000)})
	--({\sx*(2.2100)},{\sy*(0.0000)})
	--({\sx*(2.2200)},{\sy*(-0.0000)})
	--({\sx*(2.2300)},{\sy*(0.0000)})
	--({\sx*(2.2400)},{\sy*(0.0000)})
	--({\sx*(2.2500)},{\sy*(0.0000)})
	--({\sx*(2.2600)},{\sy*(0.0000)})
	--({\sx*(2.2700)},{\sy*(-0.0000)})
	--({\sx*(2.2800)},{\sy*(0.0000)})
	--({\sx*(2.2900)},{\sy*(0.0000)})
	--({\sx*(2.3000)},{\sy*(-0.0000)})
	--({\sx*(2.3100)},{\sy*(0.0000)})
	--({\sx*(2.3200)},{\sy*(0.0000)})
	--({\sx*(2.3300)},{\sy*(0.0000)})
	--({\sx*(2.3400)},{\sy*(0.0000)})
	--({\sx*(2.3500)},{\sy*(-0.0000)})
	--({\sx*(2.3600)},{\sy*(0.0000)})
	--({\sx*(2.3700)},{\sy*(0.0000)})
	--({\sx*(2.3800)},{\sy*(-0.0000)})
	--({\sx*(2.3900)},{\sy*(0.0000)})
	--({\sx*(2.4000)},{\sy*(0.0000)})
	--({\sx*(2.4100)},{\sy*(-0.0000)})
	--({\sx*(2.4200)},{\sy*(0.0000)})
	--({\sx*(2.4300)},{\sy*(-0.0000)})
	--({\sx*(2.4400)},{\sy*(-0.0000)})
	--({\sx*(2.4500)},{\sy*(0.0000)})
	--({\sx*(2.4600)},{\sy*(0.0000)})
	--({\sx*(2.4700)},{\sy*(0.0000)})
	--({\sx*(2.4800)},{\sy*(0.0000)})
	--({\sx*(2.4900)},{\sy*(-0.0000)})
	--({\sx*(2.5000)},{\sy*(0.0000)})
	--({\sx*(2.5100)},{\sy*(-0.0000)})
	--({\sx*(2.5200)},{\sy*(-0.0000)})
	--({\sx*(2.5300)},{\sy*(0.0000)})
	--({\sx*(2.5400)},{\sy*(0.0000)})
	--({\sx*(2.5500)},{\sy*(-0.0000)})
	--({\sx*(2.5600)},{\sy*(0.0000)})
	--({\sx*(2.5700)},{\sy*(-0.0000)})
	--({\sx*(2.5800)},{\sy*(-0.0000)})
	--({\sx*(2.5900)},{\sy*(0.0000)})
	--({\sx*(2.6000)},{\sy*(0.0000)})
	--({\sx*(2.6100)},{\sy*(0.0000)})
	--({\sx*(2.6200)},{\sy*(0.0000)})
	--({\sx*(2.6300)},{\sy*(0.0000)})
	--({\sx*(2.6400)},{\sy*(0.0000)})
	--({\sx*(2.6500)},{\sy*(-0.0000)})
	--({\sx*(2.6600)},{\sy*(-0.0000)})
	--({\sx*(2.6700)},{\sy*(-0.0000)})
	--({\sx*(2.6800)},{\sy*(0.0000)})
	--({\sx*(2.6900)},{\sy*(-0.0000)})
	--({\sx*(2.7000)},{\sy*(0.0000)})
	--({\sx*(2.7100)},{\sy*(0.0000)})
	--({\sx*(2.7200)},{\sy*(0.0000)})
	--({\sx*(2.7300)},{\sy*(-0.0000)})
	--({\sx*(2.7400)},{\sy*(-0.0000)})
	--({\sx*(2.7500)},{\sy*(0.0000)})
	--({\sx*(2.7600)},{\sy*(-0.0000)})
	--({\sx*(2.7700)},{\sy*(-0.0000)})
	--({\sx*(2.7800)},{\sy*(-0.0000)})
	--({\sx*(2.7900)},{\sy*(-0.0000)})
	--({\sx*(2.8000)},{\sy*(0.0000)})
	--({\sx*(2.8100)},{\sy*(0.0000)})
	--({\sx*(2.8200)},{\sy*(0.0000)})
	--({\sx*(2.8300)},{\sy*(0.0000)})
	--({\sx*(2.8400)},{\sy*(0.0000)})
	--({\sx*(2.8500)},{\sy*(0.0000)})
	--({\sx*(2.8600)},{\sy*(0.0000)})
	--({\sx*(2.8700)},{\sy*(-0.0000)})
	--({\sx*(2.8800)},{\sy*(0.0000)})
	--({\sx*(2.8900)},{\sy*(0.0000)})
	--({\sx*(2.9000)},{\sy*(-0.0000)})
	--({\sx*(2.9100)},{\sy*(0.0000)})
	--({\sx*(2.9200)},{\sy*(-0.0000)})
	--({\sx*(2.9300)},{\sy*(0.0000)})
	--({\sx*(2.9400)},{\sy*(-0.0000)})
	--({\sx*(2.9500)},{\sy*(-0.0000)})
	--({\sx*(2.9600)},{\sy*(-0.0000)})
	--({\sx*(2.9700)},{\sy*(-0.0000)})
	--({\sx*(2.9800)},{\sy*(0.0000)})
	--({\sx*(2.9900)},{\sy*(-0.0000)})
	--({\sx*(3.0000)},{\sy*(-0.0000)})
	--({\sx*(3.0100)},{\sy*(0.0000)})
	--({\sx*(3.0200)},{\sy*(-0.0000)})
	--({\sx*(3.0300)},{\sy*(-0.0000)})
	--({\sx*(3.0400)},{\sy*(-0.0000)})
	--({\sx*(3.0500)},{\sy*(-0.0000)})
	--({\sx*(3.0600)},{\sy*(-0.0000)})
	--({\sx*(3.0700)},{\sy*(-0.0000)})
	--({\sx*(3.0800)},{\sy*(-0.0000)})
	--({\sx*(3.0900)},{\sy*(0.0000)})
	--({\sx*(3.1000)},{\sy*(-0.0000)})
	--({\sx*(3.1100)},{\sy*(-0.0000)})
	--({\sx*(3.1200)},{\sy*(0.0000)})
	--({\sx*(3.1300)},{\sy*(-0.0000)})
	--({\sx*(3.1400)},{\sy*(0.0000)})
	--({\sx*(3.1500)},{\sy*(0.0000)})
	--({\sx*(3.1600)},{\sy*(0.0000)})
	--({\sx*(3.1700)},{\sy*(0.0000)})
	--({\sx*(3.1800)},{\sy*(0.0000)})
	--({\sx*(3.1900)},{\sy*(0.0000)})
	--({\sx*(3.2000)},{\sy*(0.0000)})
	--({\sx*(3.2100)},{\sy*(0.0000)})
	--({\sx*(3.2200)},{\sy*(0.0000)})
	--({\sx*(3.2300)},{\sy*(-0.0000)})
	--({\sx*(3.2400)},{\sy*(-0.0000)})
	--({\sx*(3.2500)},{\sy*(-0.0000)})
	--({\sx*(3.2600)},{\sy*(-0.0000)})
	--({\sx*(3.2700)},{\sy*(-0.0000)})
	--({\sx*(3.2800)},{\sy*(0.0000)})
	--({\sx*(3.2900)},{\sy*(-0.0000)})
	--({\sx*(3.3000)},{\sy*(-0.0000)})
	--({\sx*(3.3100)},{\sy*(0.0000)})
	--({\sx*(3.3200)},{\sy*(0.0000)})
	--({\sx*(3.3300)},{\sy*(-0.0000)})
	--({\sx*(3.3400)},{\sy*(-0.0000)})
	--({\sx*(3.3500)},{\sy*(-0.0000)})
	--({\sx*(3.3600)},{\sy*(-0.0000)})
	--({\sx*(3.3700)},{\sy*(-0.0000)})
	--({\sx*(3.3800)},{\sy*(-0.0000)})
	--({\sx*(3.3900)},{\sy*(0.0000)})
	--({\sx*(3.4000)},{\sy*(0.0000)})
	--({\sx*(3.4100)},{\sy*(0.0000)})
	--({\sx*(3.4200)},{\sy*(-0.0000)})
	--({\sx*(3.4300)},{\sy*(0.0000)})
	--({\sx*(3.4400)},{\sy*(-0.0000)})
	--({\sx*(3.4500)},{\sy*(0.0000)})
	--({\sx*(3.4600)},{\sy*(-0.0000)})
	--({\sx*(3.4700)},{\sy*(0.0000)})
	--({\sx*(3.4800)},{\sy*(0.0000)})
	--({\sx*(3.4900)},{\sy*(0.0000)})
	--({\sx*(3.5000)},{\sy*(0.0000)})
	--({\sx*(3.5100)},{\sy*(0.0000)})
	--({\sx*(3.5200)},{\sy*(0.0000)})
	--({\sx*(3.5300)},{\sy*(-0.0000)})
	--({\sx*(3.5400)},{\sy*(-0.0000)})
	--({\sx*(3.5500)},{\sy*(-0.0000)})
	--({\sx*(3.5600)},{\sy*(0.0000)})
	--({\sx*(3.5700)},{\sy*(0.0000)})
	--({\sx*(3.5800)},{\sy*(-0.0000)})
	--({\sx*(3.5900)},{\sy*(0.0000)})
	--({\sx*(3.6000)},{\sy*(-0.0000)})
	--({\sx*(3.6100)},{\sy*(-0.0000)})
	--({\sx*(3.6200)},{\sy*(-0.0000)})
	--({\sx*(3.6300)},{\sy*(0.0000)})
	--({\sx*(3.6400)},{\sy*(-0.0000)})
	--({\sx*(3.6500)},{\sy*(-0.0000)})
	--({\sx*(3.6600)},{\sy*(0.0000)})
	--({\sx*(3.6700)},{\sy*(-0.0000)})
	--({\sx*(3.6800)},{\sy*(-0.0000)})
	--({\sx*(3.6900)},{\sy*(0.0000)})
	--({\sx*(3.7000)},{\sy*(0.0000)})
	--({\sx*(3.7100)},{\sy*(0.0000)})
	--({\sx*(3.7200)},{\sy*(-0.0000)})
	--({\sx*(3.7300)},{\sy*(0.0000)})
	--({\sx*(3.7400)},{\sy*(0.0000)})
	--({\sx*(3.7500)},{\sy*(-0.0000)})
	--({\sx*(3.7600)},{\sy*(-0.0000)})
	--({\sx*(3.7700)},{\sy*(0.0000)})
	--({\sx*(3.7800)},{\sy*(-0.0000)})
	--({\sx*(3.7900)},{\sy*(0.0000)})
	--({\sx*(3.8000)},{\sy*(-0.0000)})
	--({\sx*(3.8100)},{\sy*(-0.0000)})
	--({\sx*(3.8200)},{\sy*(0.0000)})
	--({\sx*(3.8300)},{\sy*(-0.0000)})
	--({\sx*(3.8400)},{\sy*(0.0000)})
	--({\sx*(3.8500)},{\sy*(-0.0000)})
	--({\sx*(3.8600)},{\sy*(-0.0000)})
	--({\sx*(3.8700)},{\sy*(-0.0000)})
	--({\sx*(3.8800)},{\sy*(-0.0000)})
	--({\sx*(3.8900)},{\sy*(0.0000)})
	--({\sx*(3.9000)},{\sy*(-0.0000)})
	--({\sx*(3.9100)},{\sy*(-0.0000)})
	--({\sx*(3.9200)},{\sy*(-0.0000)})
	--({\sx*(3.9300)},{\sy*(-0.0000)})
	--({\sx*(3.9400)},{\sy*(0.0000)})
	--({\sx*(3.9500)},{\sy*(-0.0000)})
	--({\sx*(3.9600)},{\sy*(-0.0000)})
	--({\sx*(3.9700)},{\sy*(-0.0000)})
	--({\sx*(3.9800)},{\sy*(0.0000)})
	--({\sx*(3.9900)},{\sy*(-0.0000)})
	--({\sx*(4.0000)},{\sy*(0.0000)})
	--({\sx*(4.0100)},{\sy*(0.0000)})
	--({\sx*(4.0200)},{\sy*(-0.0000)})
	--({\sx*(4.0300)},{\sy*(0.0000)})
	--({\sx*(4.0400)},{\sy*(0.0000)})
	--({\sx*(4.0500)},{\sy*(0.0000)})
	--({\sx*(4.0600)},{\sy*(0.0000)})
	--({\sx*(4.0700)},{\sy*(0.0000)})
	--({\sx*(4.0800)},{\sy*(-0.0000)})
	--({\sx*(4.0900)},{\sy*(0.0000)})
	--({\sx*(4.1000)},{\sy*(-0.0000)})
	--({\sx*(4.1100)},{\sy*(0.0000)})
	--({\sx*(4.1200)},{\sy*(-0.0000)})
	--({\sx*(4.1300)},{\sy*(-0.0000)})
	--({\sx*(4.1400)},{\sy*(-0.0000)})
	--({\sx*(4.1500)},{\sy*(0.0000)})
	--({\sx*(4.1600)},{\sy*(0.0000)})
	--({\sx*(4.1700)},{\sy*(-0.0000)})
	--({\sx*(4.1800)},{\sy*(-0.0000)})
	--({\sx*(4.1900)},{\sy*(-0.0000)})
	--({\sx*(4.2000)},{\sy*(-0.0000)})
	--({\sx*(4.2100)},{\sy*(-0.0000)})
	--({\sx*(4.2200)},{\sy*(-0.0000)})
	--({\sx*(4.2300)},{\sy*(-0.0000)})
	--({\sx*(4.2400)},{\sy*(-0.0000)})
	--({\sx*(4.2500)},{\sy*(-0.0000)})
	--({\sx*(4.2600)},{\sy*(-0.0000)})
	--({\sx*(4.2700)},{\sy*(0.0000)})
	--({\sx*(4.2800)},{\sy*(0.0000)})
	--({\sx*(4.2900)},{\sy*(0.0000)})
	--({\sx*(4.3000)},{\sy*(0.0000)})
	--({\sx*(4.3100)},{\sy*(0.0000)})
	--({\sx*(4.3200)},{\sy*(0.0000)})
	--({\sx*(4.3300)},{\sy*(-0.0000)})
	--({\sx*(4.3400)},{\sy*(0.0000)})
	--({\sx*(4.3500)},{\sy*(0.0000)})
	--({\sx*(4.3600)},{\sy*(0.0000)})
	--({\sx*(4.3700)},{\sy*(0.0000)})
	--({\sx*(4.3800)},{\sy*(-0.0000)})
	--({\sx*(4.3900)},{\sy*(0.0000)})
	--({\sx*(4.4000)},{\sy*(0.0000)})
	--({\sx*(4.4100)},{\sy*(-0.0000)})
	--({\sx*(4.4200)},{\sy*(-0.0000)})
	--({\sx*(4.4300)},{\sy*(-0.0000)})
	--({\sx*(4.4400)},{\sy*(0.0000)})
	--({\sx*(4.4500)},{\sy*(-0.0001)})
	--({\sx*(4.4600)},{\sy*(-0.0001)})
	--({\sx*(4.4700)},{\sy*(0.0000)})
	--({\sx*(4.4800)},{\sy*(-0.0002)})
	--({\sx*(4.4900)},{\sy*(-0.0001)})
	--({\sx*(4.5000)},{\sy*(-0.0002)})
	--({\sx*(4.5100)},{\sy*(-0.0002)})
	--({\sx*(4.5200)},{\sy*(-0.0002)})
	--({\sx*(4.5300)},{\sy*(-0.0001)})
	--({\sx*(4.5400)},{\sy*(-0.0001)})
	--({\sx*(4.5500)},{\sy*(-0.0001)})
	--({\sx*(4.5600)},{\sy*(0.0000)})
	--({\sx*(4.5700)},{\sy*(-0.0000)})
	--({\sx*(4.5800)},{\sy*(0.0001)})
	--({\sx*(4.5900)},{\sy*(0.0006)})
	--({\sx*(4.6000)},{\sy*(0.0012)})
	--({\sx*(4.6100)},{\sy*(0.0002)})
	--({\sx*(4.6200)},{\sy*(0.0012)})
	--({\sx*(4.6300)},{\sy*(0.0023)})
	--({\sx*(4.6400)},{\sy*(-0.0003)})
	--({\sx*(4.6500)},{\sy*(0.0010)})
	--({\sx*(4.6600)},{\sy*(0.0004)})
	--({\sx*(4.6700)},{\sy*(0.0022)})
	--({\sx*(4.6800)},{\sy*(0.0035)})
	--({\sx*(4.6900)},{\sy*(0.0007)})
	--({\sx*(4.7000)},{\sy*(-0.0010)})
	--({\sx*(4.7100)},{\sy*(0.0001)})
	--({\sx*(4.7200)},{\sy*(0.0014)})
	--({\sx*(4.7300)},{\sy*(0.0001)})
	--({\sx*(4.7400)},{\sy*(-0.0086)})
	--({\sx*(4.7500)},{\sy*(-0.0101)})
	--({\sx*(4.7600)},{\sy*(-0.0046)})
	--({\sx*(4.7700)},{\sy*(-0.0065)})
	--({\sx*(4.7800)},{\sy*(-0.0130)})
	--({\sx*(4.7900)},{\sy*(-0.0405)})
	--({\sx*(4.8000)},{\sy*(-0.0318)})
	--({\sx*(4.8100)},{\sy*(0.0200)})
	--({\sx*(4.8200)},{\sy*(-0.0161)})
	--({\sx*(4.8300)},{\sy*(-0.0135)})
	--({\sx*(4.8400)},{\sy*(-0.0252)})
	--({\sx*(4.8500)},{\sy*(-0.0005)})
	--({\sx*(4.8600)},{\sy*(0.0208)})
	--({\sx*(4.8700)},{\sy*(0.0428)})
	--({\sx*(4.8800)},{\sy*(0.0002)})
	--({\sx*(4.8900)},{\sy*(-0.0710)})
	--({\sx*(4.9000)},{\sy*(0.0572)})
	--({\sx*(4.9100)},{\sy*(0.2024)})
	--({\sx*(4.9200)},{\sy*(0.2140)})
	--({\sx*(4.9300)},{\sy*(0.2935)})
	--({\sx*(4.9400)},{\sy*(0.6459)})
	--({\sx*(4.9500)},{\sy*(0.4815)})
	--({\sx*(4.9600)},{\sy*(0.7556)})
	--({\sx*(4.9700)},{\sy*(0.4482)})
	--({\sx*(4.9800)},{\sy*(0.5895)})
	--({\sx*(4.9900)},{\sy*(0.2270)})
	--({\sx*(5.0000)},{\sy*(0.0000)});
}
\def\relfehlerq{
\draw[color=blue,line width=1.4pt,line join=round] ({\sx*(0.000)},{\sy*(0.0000)})
	--({\sx*(0.0100)},{\sy*(-0.0000)})
	--({\sx*(0.0200)},{\sy*(0.0000)})
	--({\sx*(0.0300)},{\sy*(-0.0000)})
	--({\sx*(0.0400)},{\sy*(0.0000)})
	--({\sx*(0.0500)},{\sy*(0.0000)})
	--({\sx*(0.0600)},{\sy*(0.0000)})
	--({\sx*(0.0700)},{\sy*(-0.0000)})
	--({\sx*(0.0800)},{\sy*(0.0000)})
	--({\sx*(0.0900)},{\sy*(0.0000)})
	--({\sx*(0.1000)},{\sy*(-0.0000)})
	--({\sx*(0.1100)},{\sy*(0.0000)})
	--({\sx*(0.1200)},{\sy*(0.0000)})
	--({\sx*(0.1300)},{\sy*(-0.0000)})
	--({\sx*(0.1400)},{\sy*(0.0000)})
	--({\sx*(0.1500)},{\sy*(0.0000)})
	--({\sx*(0.1600)},{\sy*(-0.0000)})
	--({\sx*(0.1700)},{\sy*(0.0000)})
	--({\sx*(0.1800)},{\sy*(-0.0000)})
	--({\sx*(0.1900)},{\sy*(0.0000)})
	--({\sx*(0.2000)},{\sy*(0.0000)})
	--({\sx*(0.2100)},{\sy*(0.0000)})
	--({\sx*(0.2200)},{\sy*(0.0000)})
	--({\sx*(0.2300)},{\sy*(-0.0000)})
	--({\sx*(0.2400)},{\sy*(-0.0000)})
	--({\sx*(0.2500)},{\sy*(-0.0000)})
	--({\sx*(0.2600)},{\sy*(0.0000)})
	--({\sx*(0.2700)},{\sy*(-0.0000)})
	--({\sx*(0.2800)},{\sy*(-0.0000)})
	--({\sx*(0.2900)},{\sy*(-0.0000)})
	--({\sx*(0.3000)},{\sy*(0.0000)})
	--({\sx*(0.3100)},{\sy*(0.0000)})
	--({\sx*(0.3200)},{\sy*(0.0000)})
	--({\sx*(0.3300)},{\sy*(0.0000)})
	--({\sx*(0.3400)},{\sy*(-0.0000)})
	--({\sx*(0.3500)},{\sy*(0.0000)})
	--({\sx*(0.3600)},{\sy*(0.0000)})
	--({\sx*(0.3700)},{\sy*(-0.0000)})
	--({\sx*(0.3800)},{\sy*(0.0000)})
	--({\sx*(0.3900)},{\sy*(-0.0000)})
	--({\sx*(0.4000)},{\sy*(0.0000)})
	--({\sx*(0.4100)},{\sy*(0.0000)})
	--({\sx*(0.4200)},{\sy*(0.0000)})
	--({\sx*(0.4300)},{\sy*(0.0000)})
	--({\sx*(0.4400)},{\sy*(0.0000)})
	--({\sx*(0.4500)},{\sy*(-0.0000)})
	--({\sx*(0.4600)},{\sy*(0.0000)})
	--({\sx*(0.4700)},{\sy*(-0.0000)})
	--({\sx*(0.4800)},{\sy*(-0.0000)})
	--({\sx*(0.4900)},{\sy*(0.0000)})
	--({\sx*(0.5000)},{\sy*(-0.0000)})
	--({\sx*(0.5100)},{\sy*(-0.0000)})
	--({\sx*(0.5200)},{\sy*(-0.0000)})
	--({\sx*(0.5300)},{\sy*(-0.0000)})
	--({\sx*(0.5400)},{\sy*(-0.0000)})
	--({\sx*(0.5500)},{\sy*(-0.0000)})
	--({\sx*(0.5600)},{\sy*(-0.0000)})
	--({\sx*(0.5700)},{\sy*(0.0000)})
	--({\sx*(0.5800)},{\sy*(-0.0000)})
	--({\sx*(0.5900)},{\sy*(0.0000)})
	--({\sx*(0.6000)},{\sy*(0.0000)})
	--({\sx*(0.6100)},{\sy*(0.0000)})
	--({\sx*(0.6200)},{\sy*(-0.0000)})
	--({\sx*(0.6300)},{\sy*(-0.0000)})
	--({\sx*(0.6400)},{\sy*(0.0000)})
	--({\sx*(0.6500)},{\sy*(-0.0000)})
	--({\sx*(0.6600)},{\sy*(-0.0000)})
	--({\sx*(0.6700)},{\sy*(-0.0000)})
	--({\sx*(0.6800)},{\sy*(0.0000)})
	--({\sx*(0.6900)},{\sy*(0.0000)})
	--({\sx*(0.7000)},{\sy*(0.0000)})
	--({\sx*(0.7100)},{\sy*(0.0000)})
	--({\sx*(0.7200)},{\sy*(0.0000)})
	--({\sx*(0.7300)},{\sy*(0.0000)})
	--({\sx*(0.7400)},{\sy*(-0.0000)})
	--({\sx*(0.7500)},{\sy*(-0.0000)})
	--({\sx*(0.7600)},{\sy*(-0.0000)})
	--({\sx*(0.7700)},{\sy*(0.0000)})
	--({\sx*(0.7800)},{\sy*(0.0000)})
	--({\sx*(0.7900)},{\sy*(-0.0000)})
	--({\sx*(0.8000)},{\sy*(0.0000)})
	--({\sx*(0.8100)},{\sy*(-0.0000)})
	--({\sx*(0.8200)},{\sy*(-0.0000)})
	--({\sx*(0.8300)},{\sy*(0.0000)})
	--({\sx*(0.8400)},{\sy*(-0.0000)})
	--({\sx*(0.8500)},{\sy*(-0.0000)})
	--({\sx*(0.8600)},{\sy*(0.0000)})
	--({\sx*(0.8700)},{\sy*(-0.0000)})
	--({\sx*(0.8800)},{\sy*(0.0000)})
	--({\sx*(0.8900)},{\sy*(-0.0000)})
	--({\sx*(0.9000)},{\sy*(0.0000)})
	--({\sx*(0.9100)},{\sy*(-0.0000)})
	--({\sx*(0.9200)},{\sy*(0.0000)})
	--({\sx*(0.9300)},{\sy*(-0.0000)})
	--({\sx*(0.9400)},{\sy*(0.0000)})
	--({\sx*(0.9500)},{\sy*(-0.0000)})
	--({\sx*(0.9600)},{\sy*(0.0000)})
	--({\sx*(0.9700)},{\sy*(-0.0000)})
	--({\sx*(0.9800)},{\sy*(0.0000)})
	--({\sx*(0.9900)},{\sy*(-0.0000)})
	--({\sx*(1.0000)},{\sy*(-0.0000)})
	--({\sx*(1.0100)},{\sy*(-0.0000)})
	--({\sx*(1.0200)},{\sy*(0.0000)})
	--({\sx*(1.0300)},{\sy*(-0.0000)})
	--({\sx*(1.0400)},{\sy*(0.0000)})
	--({\sx*(1.0500)},{\sy*(0.0000)})
	--({\sx*(1.0600)},{\sy*(0.0000)})
	--({\sx*(1.0700)},{\sy*(-0.0000)})
	--({\sx*(1.0800)},{\sy*(0.0000)})
	--({\sx*(1.0900)},{\sy*(-0.0000)})
	--({\sx*(1.1000)},{\sy*(0.0000)})
	--({\sx*(1.1100)},{\sy*(0.0000)})
	--({\sx*(1.1200)},{\sy*(0.0000)})
	--({\sx*(1.1300)},{\sy*(0.0000)})
	--({\sx*(1.1400)},{\sy*(0.0000)})
	--({\sx*(1.1500)},{\sy*(0.0000)})
	--({\sx*(1.1600)},{\sy*(0.0000)})
	--({\sx*(1.1700)},{\sy*(0.0000)})
	--({\sx*(1.1800)},{\sy*(0.0000)})
	--({\sx*(1.1900)},{\sy*(0.0000)})
	--({\sx*(1.2000)},{\sy*(0.0000)})
	--({\sx*(1.2100)},{\sy*(-0.0000)})
	--({\sx*(1.2200)},{\sy*(0.0000)})
	--({\sx*(1.2300)},{\sy*(0.0000)})
	--({\sx*(1.2400)},{\sy*(0.0000)})
	--({\sx*(1.2500)},{\sy*(0.0000)})
	--({\sx*(1.2600)},{\sy*(0.0000)})
	--({\sx*(1.2700)},{\sy*(-0.0000)})
	--({\sx*(1.2800)},{\sy*(-0.0000)})
	--({\sx*(1.2900)},{\sy*(0.0000)})
	--({\sx*(1.3000)},{\sy*(0.0000)})
	--({\sx*(1.3100)},{\sy*(-0.0000)})
	--({\sx*(1.3200)},{\sy*(-0.0000)})
	--({\sx*(1.3300)},{\sy*(-0.0000)})
	--({\sx*(1.3400)},{\sy*(0.0000)})
	--({\sx*(1.3500)},{\sy*(-0.0000)})
	--({\sx*(1.3600)},{\sy*(-0.0000)})
	--({\sx*(1.3700)},{\sy*(-0.0000)})
	--({\sx*(1.3800)},{\sy*(-0.0000)})
	--({\sx*(1.3900)},{\sy*(0.0000)})
	--({\sx*(1.4000)},{\sy*(-0.0000)})
	--({\sx*(1.4100)},{\sy*(-0.0000)})
	--({\sx*(1.4200)},{\sy*(-0.0000)})
	--({\sx*(1.4300)},{\sy*(-0.0000)})
	--({\sx*(1.4400)},{\sy*(0.0000)})
	--({\sx*(1.4500)},{\sy*(-0.0000)})
	--({\sx*(1.4600)},{\sy*(-0.0000)})
	--({\sx*(1.4700)},{\sy*(0.0000)})
	--({\sx*(1.4800)},{\sy*(0.0000)})
	--({\sx*(1.4900)},{\sy*(-0.0000)})
	--({\sx*(1.5000)},{\sy*(0.0000)})
	--({\sx*(1.5100)},{\sy*(0.0000)})
	--({\sx*(1.5200)},{\sy*(-0.0000)})
	--({\sx*(1.5300)},{\sy*(-0.0000)})
	--({\sx*(1.5400)},{\sy*(0.0000)})
	--({\sx*(1.5500)},{\sy*(0.0000)})
	--({\sx*(1.5600)},{\sy*(0.0000)})
	--({\sx*(1.5700)},{\sy*(-0.0000)})
	--({\sx*(1.5800)},{\sy*(0.0000)})
	--({\sx*(1.5900)},{\sy*(0.0000)})
	--({\sx*(1.6000)},{\sy*(0.0000)})
	--({\sx*(1.6100)},{\sy*(-0.0000)})
	--({\sx*(1.6200)},{\sy*(0.0000)})
	--({\sx*(1.6300)},{\sy*(0.0000)})
	--({\sx*(1.6400)},{\sy*(0.0000)})
	--({\sx*(1.6500)},{\sy*(-0.0000)})
	--({\sx*(1.6600)},{\sy*(0.0000)})
	--({\sx*(1.6700)},{\sy*(-0.0000)})
	--({\sx*(1.6800)},{\sy*(0.0000)})
	--({\sx*(1.6900)},{\sy*(-0.0000)})
	--({\sx*(1.7000)},{\sy*(-0.0000)})
	--({\sx*(1.7100)},{\sy*(-0.0000)})
	--({\sx*(1.7200)},{\sy*(0.0000)})
	--({\sx*(1.7300)},{\sy*(0.0000)})
	--({\sx*(1.7400)},{\sy*(0.0000)})
	--({\sx*(1.7500)},{\sy*(0.0000)})
	--({\sx*(1.7600)},{\sy*(0.0000)})
	--({\sx*(1.7700)},{\sy*(0.0000)})
	--({\sx*(1.7800)},{\sy*(-0.0000)})
	--({\sx*(1.7900)},{\sy*(0.0000)})
	--({\sx*(1.8000)},{\sy*(0.0000)})
	--({\sx*(1.8100)},{\sy*(0.0000)})
	--({\sx*(1.8200)},{\sy*(-0.0000)})
	--({\sx*(1.8300)},{\sy*(0.0000)})
	--({\sx*(1.8400)},{\sy*(0.0000)})
	--({\sx*(1.8500)},{\sy*(0.0000)})
	--({\sx*(1.8600)},{\sy*(0.0000)})
	--({\sx*(1.8700)},{\sy*(0.0000)})
	--({\sx*(1.8800)},{\sy*(0.0000)})
	--({\sx*(1.8900)},{\sy*(0.0000)})
	--({\sx*(1.9000)},{\sy*(0.0000)})
	--({\sx*(1.9100)},{\sy*(0.0000)})
	--({\sx*(1.9200)},{\sy*(0.0000)})
	--({\sx*(1.9300)},{\sy*(0.0000)})
	--({\sx*(1.9400)},{\sy*(0.0000)})
	--({\sx*(1.9500)},{\sy*(0.0000)})
	--({\sx*(1.9600)},{\sy*(-0.0000)})
	--({\sx*(1.9700)},{\sy*(0.0000)})
	--({\sx*(1.9800)},{\sy*(0.0000)})
	--({\sx*(1.9900)},{\sy*(-0.0000)})
	--({\sx*(2.0000)},{\sy*(0.0000)})
	--({\sx*(2.0100)},{\sy*(-0.0000)})
	--({\sx*(2.0200)},{\sy*(0.0000)})
	--({\sx*(2.0300)},{\sy*(-0.0000)})
	--({\sx*(2.0400)},{\sy*(0.0000)})
	--({\sx*(2.0500)},{\sy*(0.0000)})
	--({\sx*(2.0600)},{\sy*(0.0000)})
	--({\sx*(2.0700)},{\sy*(-0.0000)})
	--({\sx*(2.0800)},{\sy*(0.0000)})
	--({\sx*(2.0900)},{\sy*(0.0000)})
	--({\sx*(2.1000)},{\sy*(-0.0000)})
	--({\sx*(2.1100)},{\sy*(0.0000)})
	--({\sx*(2.1200)},{\sy*(0.0000)})
	--({\sx*(2.1300)},{\sy*(-0.0000)})
	--({\sx*(2.1400)},{\sy*(0.0000)})
	--({\sx*(2.1500)},{\sy*(0.0000)})
	--({\sx*(2.1600)},{\sy*(-0.0000)})
	--({\sx*(2.1700)},{\sy*(0.0000)})
	--({\sx*(2.1800)},{\sy*(0.0000)})
	--({\sx*(2.1900)},{\sy*(0.0000)})
	--({\sx*(2.2000)},{\sy*(-0.0000)})
	--({\sx*(2.2100)},{\sy*(0.0000)})
	--({\sx*(2.2200)},{\sy*(-0.0000)})
	--({\sx*(2.2300)},{\sy*(0.0000)})
	--({\sx*(2.2400)},{\sy*(0.0000)})
	--({\sx*(2.2500)},{\sy*(0.0000)})
	--({\sx*(2.2600)},{\sy*(0.0000)})
	--({\sx*(2.2700)},{\sy*(-0.0000)})
	--({\sx*(2.2800)},{\sy*(0.0000)})
	--({\sx*(2.2900)},{\sy*(0.0000)})
	--({\sx*(2.3000)},{\sy*(-0.0000)})
	--({\sx*(2.3100)},{\sy*(0.0000)})
	--({\sx*(2.3200)},{\sy*(0.0000)})
	--({\sx*(2.3300)},{\sy*(0.0000)})
	--({\sx*(2.3400)},{\sy*(0.0000)})
	--({\sx*(2.3500)},{\sy*(-0.0000)})
	--({\sx*(2.3600)},{\sy*(0.0000)})
	--({\sx*(2.3700)},{\sy*(0.0000)})
	--({\sx*(2.3800)},{\sy*(-0.0000)})
	--({\sx*(2.3900)},{\sy*(0.0000)})
	--({\sx*(2.4000)},{\sy*(0.0000)})
	--({\sx*(2.4100)},{\sy*(-0.0000)})
	--({\sx*(2.4200)},{\sy*(0.0000)})
	--({\sx*(2.4300)},{\sy*(-0.0000)})
	--({\sx*(2.4400)},{\sy*(-0.0000)})
	--({\sx*(2.4500)},{\sy*(0.0000)})
	--({\sx*(2.4600)},{\sy*(0.0000)})
	--({\sx*(2.4700)},{\sy*(0.0000)})
	--({\sx*(2.4800)},{\sy*(0.0000)})
	--({\sx*(2.4900)},{\sy*(-0.0000)})
	--({\sx*(2.5000)},{\sy*(0.0000)})
	--({\sx*(2.5100)},{\sy*(-0.0000)})
	--({\sx*(2.5200)},{\sy*(-0.0000)})
	--({\sx*(2.5300)},{\sy*(0.0000)})
	--({\sx*(2.5400)},{\sy*(0.0000)})
	--({\sx*(2.5500)},{\sy*(-0.0000)})
	--({\sx*(2.5600)},{\sy*(0.0000)})
	--({\sx*(2.5700)},{\sy*(-0.0000)})
	--({\sx*(2.5800)},{\sy*(-0.0000)})
	--({\sx*(2.5900)},{\sy*(0.0000)})
	--({\sx*(2.6000)},{\sy*(0.0000)})
	--({\sx*(2.6100)},{\sy*(0.0000)})
	--({\sx*(2.6200)},{\sy*(0.0000)})
	--({\sx*(2.6300)},{\sy*(0.0000)})
	--({\sx*(2.6400)},{\sy*(0.0000)})
	--({\sx*(2.6500)},{\sy*(-0.0000)})
	--({\sx*(2.6600)},{\sy*(-0.0000)})
	--({\sx*(2.6700)},{\sy*(-0.0000)})
	--({\sx*(2.6800)},{\sy*(0.0000)})
	--({\sx*(2.6900)},{\sy*(-0.0000)})
	--({\sx*(2.7000)},{\sy*(0.0000)})
	--({\sx*(2.7100)},{\sy*(0.0000)})
	--({\sx*(2.7200)},{\sy*(0.0000)})
	--({\sx*(2.7300)},{\sy*(-0.0000)})
	--({\sx*(2.7400)},{\sy*(-0.0000)})
	--({\sx*(2.7500)},{\sy*(0.0000)})
	--({\sx*(2.7600)},{\sy*(-0.0000)})
	--({\sx*(2.7700)},{\sy*(-0.0000)})
	--({\sx*(2.7800)},{\sy*(-0.0000)})
	--({\sx*(2.7900)},{\sy*(-0.0000)})
	--({\sx*(2.8000)},{\sy*(0.0000)})
	--({\sx*(2.8100)},{\sy*(0.0000)})
	--({\sx*(2.8200)},{\sy*(0.0000)})
	--({\sx*(2.8300)},{\sy*(0.0000)})
	--({\sx*(2.8400)},{\sy*(0.0000)})
	--({\sx*(2.8500)},{\sy*(0.0000)})
	--({\sx*(2.8600)},{\sy*(0.0000)})
	--({\sx*(2.8700)},{\sy*(-0.0000)})
	--({\sx*(2.8800)},{\sy*(0.0000)})
	--({\sx*(2.8900)},{\sy*(0.0000)})
	--({\sx*(2.9000)},{\sy*(-0.0000)})
	--({\sx*(2.9100)},{\sy*(0.0000)})
	--({\sx*(2.9200)},{\sy*(-0.0000)})
	--({\sx*(2.9300)},{\sy*(0.0000)})
	--({\sx*(2.9400)},{\sy*(-0.0000)})
	--({\sx*(2.9500)},{\sy*(-0.0000)})
	--({\sx*(2.9600)},{\sy*(-0.0000)})
	--({\sx*(2.9700)},{\sy*(-0.0000)})
	--({\sx*(2.9800)},{\sy*(0.0000)})
	--({\sx*(2.9900)},{\sy*(-0.0000)})
	--({\sx*(3.0000)},{\sy*(-0.0000)})
	--({\sx*(3.0100)},{\sy*(0.0000)})
	--({\sx*(3.0200)},{\sy*(-0.0000)})
	--({\sx*(3.0300)},{\sy*(-0.0000)})
	--({\sx*(3.0400)},{\sy*(-0.0000)})
	--({\sx*(3.0500)},{\sy*(-0.0000)})
	--({\sx*(3.0600)},{\sy*(-0.0000)})
	--({\sx*(3.0700)},{\sy*(-0.0000)})
	--({\sx*(3.0800)},{\sy*(-0.0000)})
	--({\sx*(3.0900)},{\sy*(0.0000)})
	--({\sx*(3.1000)},{\sy*(-0.0000)})
	--({\sx*(3.1100)},{\sy*(-0.0000)})
	--({\sx*(3.1200)},{\sy*(0.0000)})
	--({\sx*(3.1300)},{\sy*(-0.0000)})
	--({\sx*(3.1400)},{\sy*(0.0000)})
	--({\sx*(3.1500)},{\sy*(0.0000)})
	--({\sx*(3.1600)},{\sy*(0.0000)})
	--({\sx*(3.1700)},{\sy*(0.0000)})
	--({\sx*(3.1800)},{\sy*(0.0000)})
	--({\sx*(3.1900)},{\sy*(0.0000)})
	--({\sx*(3.2000)},{\sy*(0.0000)})
	--({\sx*(3.2100)},{\sy*(0.0000)})
	--({\sx*(3.2200)},{\sy*(0.0000)})
	--({\sx*(3.2300)},{\sy*(-0.0000)})
	--({\sx*(3.2400)},{\sy*(-0.0000)})
	--({\sx*(3.2500)},{\sy*(-0.0000)})
	--({\sx*(3.2600)},{\sy*(-0.0000)})
	--({\sx*(3.2700)},{\sy*(-0.0000)})
	--({\sx*(3.2800)},{\sy*(0.0000)})
	--({\sx*(3.2900)},{\sy*(-0.0000)})
	--({\sx*(3.3000)},{\sy*(-0.0000)})
	--({\sx*(3.3100)},{\sy*(0.0000)})
	--({\sx*(3.3200)},{\sy*(0.0000)})
	--({\sx*(3.3300)},{\sy*(-0.0000)})
	--({\sx*(3.3400)},{\sy*(-0.0000)})
	--({\sx*(3.3500)},{\sy*(-0.0000)})
	--({\sx*(3.3600)},{\sy*(-0.0000)})
	--({\sx*(3.3700)},{\sy*(-0.0000)})
	--({\sx*(3.3800)},{\sy*(-0.0000)})
	--({\sx*(3.3900)},{\sy*(0.0000)})
	--({\sx*(3.4000)},{\sy*(0.0000)})
	--({\sx*(3.4100)},{\sy*(0.0000)})
	--({\sx*(3.4200)},{\sy*(-0.0000)})
	--({\sx*(3.4300)},{\sy*(0.0000)})
	--({\sx*(3.4400)},{\sy*(-0.0000)})
	--({\sx*(3.4500)},{\sy*(0.0000)})
	--({\sx*(3.4600)},{\sy*(-0.0000)})
	--({\sx*(3.4700)},{\sy*(0.0000)})
	--({\sx*(3.4800)},{\sy*(0.0000)})
	--({\sx*(3.4900)},{\sy*(0.0000)})
	--({\sx*(3.5000)},{\sy*(0.0000)})
	--({\sx*(3.5100)},{\sy*(0.0000)})
	--({\sx*(3.5200)},{\sy*(0.0000)})
	--({\sx*(3.5300)},{\sy*(-0.0000)})
	--({\sx*(3.5400)},{\sy*(-0.0000)})
	--({\sx*(3.5500)},{\sy*(-0.0000)})
	--({\sx*(3.5600)},{\sy*(0.0000)})
	--({\sx*(3.5700)},{\sy*(0.0000)})
	--({\sx*(3.5800)},{\sy*(-0.0000)})
	--({\sx*(3.5900)},{\sy*(0.0000)})
	--({\sx*(3.6000)},{\sy*(-0.0000)})
	--({\sx*(3.6100)},{\sy*(-0.0000)})
	--({\sx*(3.6200)},{\sy*(-0.0000)})
	--({\sx*(3.6300)},{\sy*(0.0000)})
	--({\sx*(3.6400)},{\sy*(-0.0000)})
	--({\sx*(3.6500)},{\sy*(-0.0000)})
	--({\sx*(3.6600)},{\sy*(0.0000)})
	--({\sx*(3.6700)},{\sy*(-0.0000)})
	--({\sx*(3.6800)},{\sy*(-0.0000)})
	--({\sx*(3.6900)},{\sy*(0.0000)})
	--({\sx*(3.7000)},{\sy*(0.0000)})
	--({\sx*(3.7100)},{\sy*(0.0000)})
	--({\sx*(3.7200)},{\sy*(-0.0000)})
	--({\sx*(3.7300)},{\sy*(0.0000)})
	--({\sx*(3.7400)},{\sy*(0.0000)})
	--({\sx*(3.7500)},{\sy*(-0.0000)})
	--({\sx*(3.7600)},{\sy*(-0.0000)})
	--({\sx*(3.7700)},{\sy*(0.0000)})
	--({\sx*(3.7800)},{\sy*(-0.0000)})
	--({\sx*(3.7900)},{\sy*(0.0000)})
	--({\sx*(3.8000)},{\sy*(-0.0000)})
	--({\sx*(3.8100)},{\sy*(-0.0000)})
	--({\sx*(3.8200)},{\sy*(0.0000)})
	--({\sx*(3.8300)},{\sy*(-0.0000)})
	--({\sx*(3.8400)},{\sy*(0.0000)})
	--({\sx*(3.8500)},{\sy*(-0.0000)})
	--({\sx*(3.8600)},{\sy*(-0.0000)})
	--({\sx*(3.8700)},{\sy*(-0.0000)})
	--({\sx*(3.8800)},{\sy*(-0.0000)})
	--({\sx*(3.8900)},{\sy*(0.0000)})
	--({\sx*(3.9000)},{\sy*(-0.0000)})
	--({\sx*(3.9100)},{\sy*(-0.0000)})
	--({\sx*(3.9200)},{\sy*(-0.0000)})
	--({\sx*(3.9300)},{\sy*(-0.0000)})
	--({\sx*(3.9400)},{\sy*(0.0000)})
	--({\sx*(3.9500)},{\sy*(-0.0000)})
	--({\sx*(3.9600)},{\sy*(-0.0000)})
	--({\sx*(3.9700)},{\sy*(-0.0000)})
	--({\sx*(3.9800)},{\sy*(0.0000)})
	--({\sx*(3.9900)},{\sy*(-0.0000)})
	--({\sx*(4.0000)},{\sy*(0.0000)})
	--({\sx*(4.0100)},{\sy*(0.0000)})
	--({\sx*(4.0200)},{\sy*(-0.0000)})
	--({\sx*(4.0300)},{\sy*(0.0000)})
	--({\sx*(4.0400)},{\sy*(0.0000)})
	--({\sx*(4.0500)},{\sy*(0.0000)})
	--({\sx*(4.0600)},{\sy*(0.0000)})
	--({\sx*(4.0700)},{\sy*(0.0000)})
	--({\sx*(4.0800)},{\sy*(-0.0000)})
	--({\sx*(4.0900)},{\sy*(0.0000)})
	--({\sx*(4.1000)},{\sy*(-0.0000)})
	--({\sx*(4.1100)},{\sy*(0.0000)})
	--({\sx*(4.1200)},{\sy*(-0.0000)})
	--({\sx*(4.1300)},{\sy*(-0.0000)})
	--({\sx*(4.1400)},{\sy*(-0.0000)})
	--({\sx*(4.1500)},{\sy*(0.0000)})
	--({\sx*(4.1600)},{\sy*(0.0000)})
	--({\sx*(4.1700)},{\sy*(-0.0000)})
	--({\sx*(4.1800)},{\sy*(-0.0000)})
	--({\sx*(4.1900)},{\sy*(-0.0000)})
	--({\sx*(4.2000)},{\sy*(-0.0000)})
	--({\sx*(4.2100)},{\sy*(-0.0000)})
	--({\sx*(4.2200)},{\sy*(-0.0000)})
	--({\sx*(4.2300)},{\sy*(-0.0000)})
	--({\sx*(4.2400)},{\sy*(-0.0000)})
	--({\sx*(4.2500)},{\sy*(-0.0000)})
	--({\sx*(4.2600)},{\sy*(-0.0000)})
	--({\sx*(4.2700)},{\sy*(0.0000)})
	--({\sx*(4.2800)},{\sy*(0.0000)})
	--({\sx*(4.2900)},{\sy*(0.0000)})
	--({\sx*(4.3000)},{\sy*(0.0000)})
	--({\sx*(4.3100)},{\sy*(0.0000)})
	--({\sx*(4.3200)},{\sy*(0.0000)})
	--({\sx*(4.3300)},{\sy*(-0.0000)})
	--({\sx*(4.3400)},{\sy*(0.0000)})
	--({\sx*(4.3500)},{\sy*(0.0000)})
	--({\sx*(4.3600)},{\sy*(0.0000)})
	--({\sx*(4.3700)},{\sy*(0.0000)})
	--({\sx*(4.3800)},{\sy*(-0.0000)})
	--({\sx*(4.3900)},{\sy*(0.0000)})
	--({\sx*(4.4000)},{\sy*(0.0000)})
	--({\sx*(4.4100)},{\sy*(-0.0000)})
	--({\sx*(4.4200)},{\sy*(-0.0000)})
	--({\sx*(4.4300)},{\sy*(-0.0000)})
	--({\sx*(4.4400)},{\sy*(0.0000)})
	--({\sx*(4.4500)},{\sy*(-0.0000)})
	--({\sx*(4.4600)},{\sy*(-0.0000)})
	--({\sx*(4.4700)},{\sy*(0.0000)})
	--({\sx*(4.4800)},{\sy*(-0.0000)})
	--({\sx*(4.4900)},{\sy*(-0.0000)})
	--({\sx*(4.5000)},{\sy*(-0.0000)})
	--({\sx*(4.5100)},{\sy*(-0.0000)})
	--({\sx*(4.5200)},{\sy*(-0.0000)})
	--({\sx*(4.5300)},{\sy*(-0.0000)})
	--({\sx*(4.5400)},{\sy*(-0.0000)})
	--({\sx*(4.5500)},{\sy*(-0.0000)})
	--({\sx*(4.5600)},{\sy*(0.0000)})
	--({\sx*(4.5700)},{\sy*(-0.0000)})
	--({\sx*(4.5800)},{\sy*(0.0000)})
	--({\sx*(4.5900)},{\sy*(0.0000)})
	--({\sx*(4.6000)},{\sy*(0.0000)})
	--({\sx*(4.6100)},{\sy*(0.0000)})
	--({\sx*(4.6200)},{\sy*(0.0000)})
	--({\sx*(4.6300)},{\sy*(0.0000)})
	--({\sx*(4.6400)},{\sy*(-0.0000)})
	--({\sx*(4.6500)},{\sy*(0.0000)})
	--({\sx*(4.6600)},{\sy*(0.0000)})
	--({\sx*(4.6700)},{\sy*(0.0000)})
	--({\sx*(4.6800)},{\sy*(0.0000)})
	--({\sx*(4.6900)},{\sy*(0.0000)})
	--({\sx*(4.7000)},{\sy*(-0.0000)})
	--({\sx*(4.7100)},{\sy*(0.0000)})
	--({\sx*(4.7200)},{\sy*(0.0000)})
	--({\sx*(4.7300)},{\sy*(0.0000)})
	--({\sx*(4.7400)},{\sy*(-0.0000)})
	--({\sx*(4.7500)},{\sy*(-0.0000)})
	--({\sx*(4.7600)},{\sy*(-0.0000)})
	--({\sx*(4.7700)},{\sy*(-0.0000)})
	--({\sx*(4.7800)},{\sy*(-0.0000)})
	--({\sx*(4.7900)},{\sy*(-0.0000)})
	--({\sx*(4.8000)},{\sy*(-0.0000)})
	--({\sx*(4.8100)},{\sy*(0.0000)})
	--({\sx*(4.8200)},{\sy*(-0.0000)})
	--({\sx*(4.8300)},{\sy*(-0.0000)})
	--({\sx*(4.8400)},{\sy*(-0.0000)})
	--({\sx*(4.8500)},{\sy*(-0.0000)})
	--({\sx*(4.8600)},{\sy*(0.0000)})
	--({\sx*(4.8700)},{\sy*(0.0000)})
	--({\sx*(4.8800)},{\sy*(0.0000)})
	--({\sx*(4.8900)},{\sy*(-0.0000)})
	--({\sx*(4.9000)},{\sy*(0.0000)})
	--({\sx*(4.9100)},{\sy*(0.0000)})
	--({\sx*(4.9200)},{\sy*(0.0000)})
	--({\sx*(4.9300)},{\sy*(0.0001)})
	--({\sx*(4.9400)},{\sy*(0.0001)})
	--({\sx*(4.9500)},{\sy*(0.0001)})
	--({\sx*(4.9600)},{\sy*(0.0002)})
	--({\sx*(4.9700)},{\sy*(0.0001)})
	--({\sx*(4.9800)},{\sy*(0.0002)})
	--({\sx*(4.9900)},{\sy*(0.0001)})
	--({\sx*(5.0000)},{\sy*(0.0000)});
}

\begin{tikzpicture}[>=latex,thick,scale=\skala]

\def\sx{1}
\def\sy{0.9}

\def\plot#1{
	\csname fehler#1\endcsname
	\csname xwerte#1\endcsname
	\draw ({-0.1/\skala},\sy)--({0.1/\skala},\sy);
	\node at ({-0.1/\skala},\sy) [left] {$\csname maxfehler#1\endcsname$};
	\draw ({-0.1/\skala},-\sy)--({0.1/\skala},-\sy);
	\node at ({-0.1/\skala},-\sy) [left] {$-\csname maxfehler#1\endcsname$};
	\draw[->] ({-0.1/\skala},0)--(5.2,0) coordinate[label={$x$}];
	\draw[->] (0,{-1.1*\sy})--(0,{1.2*\sy}) coordinate[label={right:$y$}];
	\node at ({-0.2/\skala},0) [left] {$n=\csname punkte#1\endcsname$};
	\foreach \x in {1,...,5}{
		\draw (\x,{-0.1/\skala})--(\x,{0.1/\skala});
		\node at (\x,{-0.1*\skala}) [below] {$\x$};
	}
}

\begin{scope}
\plot{a}
\end{scope}

\begin{scope}[yshift={-2.5cm*\sy}]
\plot{h}
\end{scope}

\begin{scope}[yshift={-5cm*\sy}]
\plot{o}
\end{scope}

\begin{scope}[yshift={-7.5cm*\sy}]
\plot{q}
\end{scope}

\end{tikzpicture}
\end{document}

