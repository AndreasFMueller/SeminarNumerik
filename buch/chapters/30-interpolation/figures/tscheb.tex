%
% norm.tex -- absoluter und relativer Fehler des Interpolationspolynoms
%
% (c) 2020 Prof Dr Andreas Müller, Hochschule Rapperswil
%
\documentclass[tikz]{standalone}
\usepackage{amsmath}
\usepackage{times}
\usepackage{txfonts}
\usepackage{pgfplots}
\usepackage{csvsimple}
\usetikzlibrary{arrows,intersections,math}
\begin{document}
\def\skala{2}
\def\xwertea{
\fill[color=red] (0.0000,0) circle[radius={0.07/\skala}];
\fill[color=white] (0.0000,0) circle[radius={0.05/\skala}];
\fill[color=red] (2.5000,0) circle[radius={0.07/\skala}];
\fill[color=white] (2.5000,0) circle[radius={0.05/\skala}];
\fill[color=red] (5.0000,0) circle[radius={0.07/\skala}];
\fill[color=white] (5.0000,0) circle[radius={0.05/\skala}];
}
\def\punktea{2}
\def\maxfehlera{5.917\cdot 10^{-2}}
\def\fehlera{
\draw[color=red,line width=1.4pt,line join=round] ({\sx*(0.000)},{\sy*(0.0000)})
	--({\sx*(0.0100)},{\sy*(-0.0377)})
	--({\sx*(0.0200)},{\sy*(-0.0746)})
	--({\sx*(0.0300)},{\sy*(-0.1108)})
	--({\sx*(0.0400)},{\sy*(-0.1462)})
	--({\sx*(0.0500)},{\sy*(-0.1808)})
	--({\sx*(0.0600)},{\sy*(-0.2146)})
	--({\sx*(0.0700)},{\sy*(-0.2477)})
	--({\sx*(0.0800)},{\sy*(-0.2800)})
	--({\sx*(0.0900)},{\sy*(-0.3115)})
	--({\sx*(0.1000)},{\sy*(-0.3423)})
	--({\sx*(0.1100)},{\sy*(-0.3723)})
	--({\sx*(0.1200)},{\sy*(-0.4016)})
	--({\sx*(0.1300)},{\sy*(-0.4300)})
	--({\sx*(0.1400)},{\sy*(-0.4578)})
	--({\sx*(0.1500)},{\sy*(-0.4848)})
	--({\sx*(0.1600)},{\sy*(-0.5110)})
	--({\sx*(0.1700)},{\sy*(-0.5365)})
	--({\sx*(0.1800)},{\sy*(-0.5612)})
	--({\sx*(0.1900)},{\sy*(-0.5852)})
	--({\sx*(0.2000)},{\sy*(-0.6085)})
	--({\sx*(0.2100)},{\sy*(-0.6310)})
	--({\sx*(0.2200)},{\sy*(-0.6528)})
	--({\sx*(0.2300)},{\sy*(-0.6739)})
	--({\sx*(0.2400)},{\sy*(-0.6943)})
	--({\sx*(0.2500)},{\sy*(-0.7139)})
	--({\sx*(0.2600)},{\sy*(-0.7329)})
	--({\sx*(0.2700)},{\sy*(-0.7511)})
	--({\sx*(0.2800)},{\sy*(-0.7686)})
	--({\sx*(0.2900)},{\sy*(-0.7854)})
	--({\sx*(0.3000)},{\sy*(-0.8016)})
	--({\sx*(0.3100)},{\sy*(-0.8170)})
	--({\sx*(0.3200)},{\sy*(-0.8318)})
	--({\sx*(0.3300)},{\sy*(-0.8459)})
	--({\sx*(0.3400)},{\sy*(-0.8593)})
	--({\sx*(0.3500)},{\sy*(-0.8721)})
	--({\sx*(0.3600)},{\sy*(-0.8842)})
	--({\sx*(0.3700)},{\sy*(-0.8957)})
	--({\sx*(0.3800)},{\sy*(-0.9065)})
	--({\sx*(0.3900)},{\sy*(-0.9167)})
	--({\sx*(0.4000)},{\sy*(-0.9263)})
	--({\sx*(0.4100)},{\sy*(-0.9352)})
	--({\sx*(0.4200)},{\sy*(-0.9435)})
	--({\sx*(0.4300)},{\sy*(-0.9513)})
	--({\sx*(0.4400)},{\sy*(-0.9584)})
	--({\sx*(0.4500)},{\sy*(-0.9649)})
	--({\sx*(0.4600)},{\sy*(-0.9709)})
	--({\sx*(0.4700)},{\sy*(-0.9762)})
	--({\sx*(0.4800)},{\sy*(-0.9810)})
	--({\sx*(0.4900)},{\sy*(-0.9853)})
	--({\sx*(0.5000)},{\sy*(-0.9890)})
	--({\sx*(0.5100)},{\sy*(-0.9921)})
	--({\sx*(0.5200)},{\sy*(-0.9947)})
	--({\sx*(0.5300)},{\sy*(-0.9968)})
	--({\sx*(0.5400)},{\sy*(-0.9984)})
	--({\sx*(0.5500)},{\sy*(-0.9994)})
	--({\sx*(0.5600)},{\sy*(-0.9999)})
	--({\sx*(0.5700)},{\sy*(-1.0000)})
	--({\sx*(0.5800)},{\sy*(-0.9996)})
	--({\sx*(0.5900)},{\sy*(-0.9987)})
	--({\sx*(0.6000)},{\sy*(-0.9973)})
	--({\sx*(0.6100)},{\sy*(-0.9954)})
	--({\sx*(0.6200)},{\sy*(-0.9932)})
	--({\sx*(0.6300)},{\sy*(-0.9904)})
	--({\sx*(0.6400)},{\sy*(-0.9873)})
	--({\sx*(0.6500)},{\sy*(-0.9837)})
	--({\sx*(0.6600)},{\sy*(-0.9797)})
	--({\sx*(0.6700)},{\sy*(-0.9753)})
	--({\sx*(0.6800)},{\sy*(-0.9705)})
	--({\sx*(0.6900)},{\sy*(-0.9653)})
	--({\sx*(0.7000)},{\sy*(-0.9597)})
	--({\sx*(0.7100)},{\sy*(-0.9538)})
	--({\sx*(0.7200)},{\sy*(-0.9475)})
	--({\sx*(0.7300)},{\sy*(-0.9409)})
	--({\sx*(0.7400)},{\sy*(-0.9339)})
	--({\sx*(0.7500)},{\sy*(-0.9266)})
	--({\sx*(0.7600)},{\sy*(-0.9190)})
	--({\sx*(0.7700)},{\sy*(-0.9111)})
	--({\sx*(0.7800)},{\sy*(-0.9028)})
	--({\sx*(0.7900)},{\sy*(-0.8943)})
	--({\sx*(0.8000)},{\sy*(-0.8855)})
	--({\sx*(0.8100)},{\sy*(-0.8764)})
	--({\sx*(0.8200)},{\sy*(-0.8670)})
	--({\sx*(0.8300)},{\sy*(-0.8574)})
	--({\sx*(0.8400)},{\sy*(-0.8475)})
	--({\sx*(0.8500)},{\sy*(-0.8375)})
	--({\sx*(0.8600)},{\sy*(-0.8271)})
	--({\sx*(0.8700)},{\sy*(-0.8166)})
	--({\sx*(0.8800)},{\sy*(-0.8058)})
	--({\sx*(0.8900)},{\sy*(-0.7949)})
	--({\sx*(0.9000)},{\sy*(-0.7837)})
	--({\sx*(0.9100)},{\sy*(-0.7724)})
	--({\sx*(0.9200)},{\sy*(-0.7609)})
	--({\sx*(0.9300)},{\sy*(-0.7492)})
	--({\sx*(0.9400)},{\sy*(-0.7374)})
	--({\sx*(0.9500)},{\sy*(-0.7254)})
	--({\sx*(0.9600)},{\sy*(-0.7133)})
	--({\sx*(0.9700)},{\sy*(-0.7010)})
	--({\sx*(0.9800)},{\sy*(-0.6886)})
	--({\sx*(0.9900)},{\sy*(-0.6761)})
	--({\sx*(1.0000)},{\sy*(-0.6635)})
	--({\sx*(1.0100)},{\sy*(-0.6508)})
	--({\sx*(1.0200)},{\sy*(-0.6380)})
	--({\sx*(1.0300)},{\sy*(-0.6252)})
	--({\sx*(1.0400)},{\sy*(-0.6122)})
	--({\sx*(1.0500)},{\sy*(-0.5992)})
	--({\sx*(1.0600)},{\sy*(-0.5861)})
	--({\sx*(1.0700)},{\sy*(-0.5730)})
	--({\sx*(1.0800)},{\sy*(-0.5599)})
	--({\sx*(1.0900)},{\sy*(-0.5467)})
	--({\sx*(1.1000)},{\sy*(-0.5334)})
	--({\sx*(1.1100)},{\sy*(-0.5202)})
	--({\sx*(1.1200)},{\sy*(-0.5069)})
	--({\sx*(1.1300)},{\sy*(-0.4937)})
	--({\sx*(1.1400)},{\sy*(-0.4804)})
	--({\sx*(1.1500)},{\sy*(-0.4671)})
	--({\sx*(1.1600)},{\sy*(-0.4539)})
	--({\sx*(1.1700)},{\sy*(-0.4406)})
	--({\sx*(1.1800)},{\sy*(-0.4274)})
	--({\sx*(1.1900)},{\sy*(-0.4143)})
	--({\sx*(1.2000)},{\sy*(-0.4011)})
	--({\sx*(1.2100)},{\sy*(-0.3881)})
	--({\sx*(1.2200)},{\sy*(-0.3750)})
	--({\sx*(1.2300)},{\sy*(-0.3621)})
	--({\sx*(1.2400)},{\sy*(-0.3492)})
	--({\sx*(1.2500)},{\sy*(-0.3363)})
	--({\sx*(1.2600)},{\sy*(-0.3236)})
	--({\sx*(1.2700)},{\sy*(-0.3109)})
	--({\sx*(1.2800)},{\sy*(-0.2983)})
	--({\sx*(1.2900)},{\sy*(-0.2858)})
	--({\sx*(1.3000)},{\sy*(-0.2734)})
	--({\sx*(1.3100)},{\sy*(-0.2611)})
	--({\sx*(1.3200)},{\sy*(-0.2489)})
	--({\sx*(1.3300)},{\sy*(-0.2368)})
	--({\sx*(1.3400)},{\sy*(-0.2248)})
	--({\sx*(1.3500)},{\sy*(-0.2129)})
	--({\sx*(1.3600)},{\sy*(-0.2012)})
	--({\sx*(1.3700)},{\sy*(-0.1896)})
	--({\sx*(1.3800)},{\sy*(-0.1781)})
	--({\sx*(1.3900)},{\sy*(-0.1668)})
	--({\sx*(1.4000)},{\sy*(-0.1556)})
	--({\sx*(1.4100)},{\sy*(-0.1446)})
	--({\sx*(1.4200)},{\sy*(-0.1337)})
	--({\sx*(1.4300)},{\sy*(-0.1229)})
	--({\sx*(1.4400)},{\sy*(-0.1123)})
	--({\sx*(1.4500)},{\sy*(-0.1019)})
	--({\sx*(1.4600)},{\sy*(-0.0916)})
	--({\sx*(1.4700)},{\sy*(-0.0815)})
	--({\sx*(1.4800)},{\sy*(-0.0716)})
	--({\sx*(1.4900)},{\sy*(-0.0618)})
	--({\sx*(1.5000)},{\sy*(-0.0522)})
	--({\sx*(1.5100)},{\sy*(-0.0428)})
	--({\sx*(1.5200)},{\sy*(-0.0336)})
	--({\sx*(1.5300)},{\sy*(-0.0245)})
	--({\sx*(1.5400)},{\sy*(-0.0156)})
	--({\sx*(1.5500)},{\sy*(-0.0069)})
	--({\sx*(1.5600)},{\sy*(0.0016)})
	--({\sx*(1.5700)},{\sy*(0.0099)})
	--({\sx*(1.5800)},{\sy*(0.0181)})
	--({\sx*(1.5900)},{\sy*(0.0260)})
	--({\sx*(1.6000)},{\sy*(0.0337)})
	--({\sx*(1.6100)},{\sy*(0.0413)})
	--({\sx*(1.6200)},{\sy*(0.0487)})
	--({\sx*(1.6300)},{\sy*(0.0558)})
	--({\sx*(1.6400)},{\sy*(0.0628)})
	--({\sx*(1.6500)},{\sy*(0.0696)})
	--({\sx*(1.6600)},{\sy*(0.0761)})
	--({\sx*(1.6700)},{\sy*(0.0825)})
	--({\sx*(1.6800)},{\sy*(0.0887)})
	--({\sx*(1.6900)},{\sy*(0.0946)})
	--({\sx*(1.7000)},{\sy*(0.1004)})
	--({\sx*(1.7100)},{\sy*(0.1060)})
	--({\sx*(1.7200)},{\sy*(0.1113)})
	--({\sx*(1.7300)},{\sy*(0.1165)})
	--({\sx*(1.7400)},{\sy*(0.1214)})
	--({\sx*(1.7500)},{\sy*(0.1262)})
	--({\sx*(1.7600)},{\sy*(0.1307)})
	--({\sx*(1.7700)},{\sy*(0.1351)})
	--({\sx*(1.7800)},{\sy*(0.1392)})
	--({\sx*(1.7900)},{\sy*(0.1432)})
	--({\sx*(1.8000)},{\sy*(0.1469)})
	--({\sx*(1.8100)},{\sy*(0.1505)})
	--({\sx*(1.8200)},{\sy*(0.1538)})
	--({\sx*(1.8300)},{\sy*(0.1570)})
	--({\sx*(1.8400)},{\sy*(0.1599)})
	--({\sx*(1.8500)},{\sy*(0.1627)})
	--({\sx*(1.8600)},{\sy*(0.1652)})
	--({\sx*(1.8700)},{\sy*(0.1676)})
	--({\sx*(1.8800)},{\sy*(0.1697)})
	--({\sx*(1.8900)},{\sy*(0.1717)})
	--({\sx*(1.9000)},{\sy*(0.1735)})
	--({\sx*(1.9100)},{\sy*(0.1751)})
	--({\sx*(1.9200)},{\sy*(0.1765)})
	--({\sx*(1.9300)},{\sy*(0.1777)})
	--({\sx*(1.9400)},{\sy*(0.1787)})
	--({\sx*(1.9500)},{\sy*(0.1795)})
	--({\sx*(1.9600)},{\sy*(0.1802)})
	--({\sx*(1.9700)},{\sy*(0.1807)})
	--({\sx*(1.9800)},{\sy*(0.1809)})
	--({\sx*(1.9900)},{\sy*(0.1811)})
	--({\sx*(2.0000)},{\sy*(0.1810)})
	--({\sx*(2.0100)},{\sy*(0.1807)})
	--({\sx*(2.0200)},{\sy*(0.1803)})
	--({\sx*(2.0300)},{\sy*(0.1797)})
	--({\sx*(2.0400)},{\sy*(0.1790)})
	--({\sx*(2.0500)},{\sy*(0.1781)})
	--({\sx*(2.0600)},{\sy*(0.1770)})
	--({\sx*(2.0700)},{\sy*(0.1757)})
	--({\sx*(2.0800)},{\sy*(0.1743)})
	--({\sx*(2.0900)},{\sy*(0.1727)})
	--({\sx*(2.1000)},{\sy*(0.1710)})
	--({\sx*(2.1100)},{\sy*(0.1691)})
	--({\sx*(2.1200)},{\sy*(0.1671)})
	--({\sx*(2.1300)},{\sy*(0.1649)})
	--({\sx*(2.1400)},{\sy*(0.1625)})
	--({\sx*(2.1500)},{\sy*(0.1601)})
	--({\sx*(2.1600)},{\sy*(0.1574)})
	--({\sx*(2.1700)},{\sy*(0.1547)})
	--({\sx*(2.1800)},{\sy*(0.1517)})
	--({\sx*(2.1900)},{\sy*(0.1487)})
	--({\sx*(2.2000)},{\sy*(0.1455)})
	--({\sx*(2.2100)},{\sy*(0.1422)})
	--({\sx*(2.2200)},{\sy*(0.1388)})
	--({\sx*(2.2300)},{\sy*(0.1352)})
	--({\sx*(2.2400)},{\sy*(0.1315)})
	--({\sx*(2.2500)},{\sy*(0.1277)})
	--({\sx*(2.2600)},{\sy*(0.1238)})
	--({\sx*(2.2700)},{\sy*(0.1197)})
	--({\sx*(2.2800)},{\sy*(0.1155)})
	--({\sx*(2.2900)},{\sy*(0.1113)})
	--({\sx*(2.3000)},{\sy*(0.1069)})
	--({\sx*(2.3100)},{\sy*(0.1024)})
	--({\sx*(2.3200)},{\sy*(0.0978)})
	--({\sx*(2.3300)},{\sy*(0.0931)})
	--({\sx*(2.3400)},{\sy*(0.0883)})
	--({\sx*(2.3500)},{\sy*(0.0834)})
	--({\sx*(2.3600)},{\sy*(0.0784)})
	--({\sx*(2.3700)},{\sy*(0.0733)})
	--({\sx*(2.3800)},{\sy*(0.0681)})
	--({\sx*(2.3900)},{\sy*(0.0629)})
	--({\sx*(2.4000)},{\sy*(0.0575)})
	--({\sx*(2.4100)},{\sy*(0.0521)})
	--({\sx*(2.4200)},{\sy*(0.0466)})
	--({\sx*(2.4300)},{\sy*(0.0410)})
	--({\sx*(2.4400)},{\sy*(0.0354)})
	--({\sx*(2.4500)},{\sy*(0.0296)})
	--({\sx*(2.4600)},{\sy*(0.0238)})
	--({\sx*(2.4700)},{\sy*(0.0180)})
	--({\sx*(2.4800)},{\sy*(0.0120)})
	--({\sx*(2.4900)},{\sy*(0.0060)})
	--({\sx*(2.5000)},{\sy*(0.0000)})
	--({\sx*(2.5100)},{\sy*(-0.0061)})
	--({\sx*(2.5200)},{\sy*(-0.0123)})
	--({\sx*(2.5300)},{\sy*(-0.0185)})
	--({\sx*(2.5400)},{\sy*(-0.0247)})
	--({\sx*(2.5500)},{\sy*(-0.0311)})
	--({\sx*(2.5600)},{\sy*(-0.0374)})
	--({\sx*(2.5700)},{\sy*(-0.0438)})
	--({\sx*(2.5800)},{\sy*(-0.0503)})
	--({\sx*(2.5900)},{\sy*(-0.0567)})
	--({\sx*(2.6000)},{\sy*(-0.0632)})
	--({\sx*(2.6100)},{\sy*(-0.0698)})
	--({\sx*(2.6200)},{\sy*(-0.0764)})
	--({\sx*(2.6300)},{\sy*(-0.0830)})
	--({\sx*(2.6400)},{\sy*(-0.0896)})
	--({\sx*(2.6500)},{\sy*(-0.0963)})
	--({\sx*(2.6600)},{\sy*(-0.1030)})
	--({\sx*(2.6700)},{\sy*(-0.1097)})
	--({\sx*(2.6800)},{\sy*(-0.1164)})
	--({\sx*(2.6900)},{\sy*(-0.1231)})
	--({\sx*(2.7000)},{\sy*(-0.1299)})
	--({\sx*(2.7100)},{\sy*(-0.1367)})
	--({\sx*(2.7200)},{\sy*(-0.1434)})
	--({\sx*(2.7300)},{\sy*(-0.1502)})
	--({\sx*(2.7400)},{\sy*(-0.1570)})
	--({\sx*(2.7500)},{\sy*(-0.1638)})
	--({\sx*(2.7600)},{\sy*(-0.1706)})
	--({\sx*(2.7700)},{\sy*(-0.1774)})
	--({\sx*(2.7800)},{\sy*(-0.1842)})
	--({\sx*(2.7900)},{\sy*(-0.1910)})
	--({\sx*(2.8000)},{\sy*(-0.1978)})
	--({\sx*(2.8100)},{\sy*(-0.2046)})
	--({\sx*(2.8200)},{\sy*(-0.2114)})
	--({\sx*(2.8300)},{\sy*(-0.2181)})
	--({\sx*(2.8400)},{\sy*(-0.2249)})
	--({\sx*(2.8500)},{\sy*(-0.2316)})
	--({\sx*(2.8600)},{\sy*(-0.2383)})
	--({\sx*(2.8700)},{\sy*(-0.2450)})
	--({\sx*(2.8800)},{\sy*(-0.2517)})
	--({\sx*(2.8900)},{\sy*(-0.2584)})
	--({\sx*(2.9000)},{\sy*(-0.2650)})
	--({\sx*(2.9100)},{\sy*(-0.2717)})
	--({\sx*(2.9200)},{\sy*(-0.2782)})
	--({\sx*(2.9300)},{\sy*(-0.2848)})
	--({\sx*(2.9400)},{\sy*(-0.2913)})
	--({\sx*(2.9500)},{\sy*(-0.2979)})
	--({\sx*(2.9600)},{\sy*(-0.3043)})
	--({\sx*(2.9700)},{\sy*(-0.3108)})
	--({\sx*(2.9800)},{\sy*(-0.3172)})
	--({\sx*(2.9900)},{\sy*(-0.3236)})
	--({\sx*(3.0000)},{\sy*(-0.3299)})
	--({\sx*(3.0100)},{\sy*(-0.3362)})
	--({\sx*(3.0200)},{\sy*(-0.3425)})
	--({\sx*(3.0300)},{\sy*(-0.3487)})
	--({\sx*(3.0400)},{\sy*(-0.3548)})
	--({\sx*(3.0500)},{\sy*(-0.3610)})
	--({\sx*(3.0600)},{\sy*(-0.3671)})
	--({\sx*(3.0700)},{\sy*(-0.3731)})
	--({\sx*(3.0800)},{\sy*(-0.3791)})
	--({\sx*(3.0900)},{\sy*(-0.3850)})
	--({\sx*(3.1000)},{\sy*(-0.3909)})
	--({\sx*(3.1100)},{\sy*(-0.3968)})
	--({\sx*(3.1200)},{\sy*(-0.4026)})
	--({\sx*(3.1300)},{\sy*(-0.4083)})
	--({\sx*(3.1400)},{\sy*(-0.4140)})
	--({\sx*(3.1500)},{\sy*(-0.4196)})
	--({\sx*(3.1600)},{\sy*(-0.4252)})
	--({\sx*(3.1700)},{\sy*(-0.4307)})
	--({\sx*(3.1800)},{\sy*(-0.4362)})
	--({\sx*(3.1900)},{\sy*(-0.4416)})
	--({\sx*(3.2000)},{\sy*(-0.4469)})
	--({\sx*(3.2100)},{\sy*(-0.4522)})
	--({\sx*(3.2200)},{\sy*(-0.4574)})
	--({\sx*(3.2300)},{\sy*(-0.4625)})
	--({\sx*(3.2400)},{\sy*(-0.4676)})
	--({\sx*(3.2500)},{\sy*(-0.4727)})
	--({\sx*(3.2600)},{\sy*(-0.4776)})
	--({\sx*(3.2700)},{\sy*(-0.4825)})
	--({\sx*(3.2800)},{\sy*(-0.4873)})
	--({\sx*(3.2900)},{\sy*(-0.4921)})
	--({\sx*(3.3000)},{\sy*(-0.4968)})
	--({\sx*(3.3100)},{\sy*(-0.5014)})
	--({\sx*(3.3200)},{\sy*(-0.5059)})
	--({\sx*(3.3300)},{\sy*(-0.5104)})
	--({\sx*(3.3400)},{\sy*(-0.5148)})
	--({\sx*(3.3500)},{\sy*(-0.5191)})
	--({\sx*(3.3600)},{\sy*(-0.5234)})
	--({\sx*(3.3700)},{\sy*(-0.5276)})
	--({\sx*(3.3800)},{\sy*(-0.5317)})
	--({\sx*(3.3900)},{\sy*(-0.5357)})
	--({\sx*(3.4000)},{\sy*(-0.5397)})
	--({\sx*(3.4100)},{\sy*(-0.5436)})
	--({\sx*(3.4200)},{\sy*(-0.5474)})
	--({\sx*(3.4300)},{\sy*(-0.5511)})
	--({\sx*(3.4400)},{\sy*(-0.5548)})
	--({\sx*(3.4500)},{\sy*(-0.5583)})
	--({\sx*(3.4600)},{\sy*(-0.5618)})
	--({\sx*(3.4700)},{\sy*(-0.5652)})
	--({\sx*(3.4800)},{\sy*(-0.5686)})
	--({\sx*(3.4900)},{\sy*(-0.5718)})
	--({\sx*(3.5000)},{\sy*(-0.5750)})
	--({\sx*(3.5100)},{\sy*(-0.5781)})
	--({\sx*(3.5200)},{\sy*(-0.5811)})
	--({\sx*(3.5300)},{\sy*(-0.5840)})
	--({\sx*(3.5400)},{\sy*(-0.5868)})
	--({\sx*(3.5500)},{\sy*(-0.5896)})
	--({\sx*(3.5600)},{\sy*(-0.5923)})
	--({\sx*(3.5700)},{\sy*(-0.5949)})
	--({\sx*(3.5800)},{\sy*(-0.5974)})
	--({\sx*(3.5900)},{\sy*(-0.5998)})
	--({\sx*(3.6000)},{\sy*(-0.6021)})
	--({\sx*(3.6100)},{\sy*(-0.6044)})
	--({\sx*(3.6200)},{\sy*(-0.6065)})
	--({\sx*(3.6300)},{\sy*(-0.6086)})
	--({\sx*(3.6400)},{\sy*(-0.6106)})
	--({\sx*(3.6500)},{\sy*(-0.6125)})
	--({\sx*(3.6600)},{\sy*(-0.6143)})
	--({\sx*(3.6700)},{\sy*(-0.6160)})
	--({\sx*(3.6800)},{\sy*(-0.6176)})
	--({\sx*(3.6900)},{\sy*(-0.6192)})
	--({\sx*(3.7000)},{\sy*(-0.6206)})
	--({\sx*(3.7100)},{\sy*(-0.6220)})
	--({\sx*(3.7200)},{\sy*(-0.6233)})
	--({\sx*(3.7300)},{\sy*(-0.6245)})
	--({\sx*(3.7400)},{\sy*(-0.6256)})
	--({\sx*(3.7500)},{\sy*(-0.6266)})
	--({\sx*(3.7600)},{\sy*(-0.6275)})
	--({\sx*(3.7700)},{\sy*(-0.6283)})
	--({\sx*(3.7800)},{\sy*(-0.6290)})
	--({\sx*(3.7900)},{\sy*(-0.6297)})
	--({\sx*(3.8000)},{\sy*(-0.6302)})
	--({\sx*(3.8100)},{\sy*(-0.6307)})
	--({\sx*(3.8200)},{\sy*(-0.6311)})
	--({\sx*(3.8300)},{\sy*(-0.6313)})
	--({\sx*(3.8400)},{\sy*(-0.6315)})
	--({\sx*(3.8500)},{\sy*(-0.6316)})
	--({\sx*(3.8600)},{\sy*(-0.6316)})
	--({\sx*(3.8700)},{\sy*(-0.6315)})
	--({\sx*(3.8800)},{\sy*(-0.6313)})
	--({\sx*(3.8900)},{\sy*(-0.6310)})
	--({\sx*(3.9000)},{\sy*(-0.6307)})
	--({\sx*(3.9100)},{\sy*(-0.6302)})
	--({\sx*(3.9200)},{\sy*(-0.6296)})
	--({\sx*(3.9300)},{\sy*(-0.6290)})
	--({\sx*(3.9400)},{\sy*(-0.6282)})
	--({\sx*(3.9500)},{\sy*(-0.6274)})
	--({\sx*(3.9600)},{\sy*(-0.6264)})
	--({\sx*(3.9700)},{\sy*(-0.6254)})
	--({\sx*(3.9800)},{\sy*(-0.6243)})
	--({\sx*(3.9900)},{\sy*(-0.6231)})
	--({\sx*(4.0000)},{\sy*(-0.6217)})
	--({\sx*(4.0100)},{\sy*(-0.6203)})
	--({\sx*(4.0200)},{\sy*(-0.6188)})
	--({\sx*(4.0300)},{\sy*(-0.6172)})
	--({\sx*(4.0400)},{\sy*(-0.6155)})
	--({\sx*(4.0500)},{\sy*(-0.6137)})
	--({\sx*(4.0600)},{\sy*(-0.6118)})
	--({\sx*(4.0700)},{\sy*(-0.6098)})
	--({\sx*(4.0800)},{\sy*(-0.6078)})
	--({\sx*(4.0900)},{\sy*(-0.6056)})
	--({\sx*(4.1000)},{\sy*(-0.6033)})
	--({\sx*(4.1100)},{\sy*(-0.6009)})
	--({\sx*(4.1200)},{\sy*(-0.5985)})
	--({\sx*(4.1300)},{\sy*(-0.5959)})
	--({\sx*(4.1400)},{\sy*(-0.5933)})
	--({\sx*(4.1500)},{\sy*(-0.5905)})
	--({\sx*(4.1600)},{\sy*(-0.5877)})
	--({\sx*(4.1700)},{\sy*(-0.5847)})
	--({\sx*(4.1800)},{\sy*(-0.5817)})
	--({\sx*(4.1900)},{\sy*(-0.5785)})
	--({\sx*(4.2000)},{\sy*(-0.5753)})
	--({\sx*(4.2100)},{\sy*(-0.5720)})
	--({\sx*(4.2200)},{\sy*(-0.5685)})
	--({\sx*(4.2300)},{\sy*(-0.5650)})
	--({\sx*(4.2400)},{\sy*(-0.5614)})
	--({\sx*(4.2500)},{\sy*(-0.5577)})
	--({\sx*(4.2600)},{\sy*(-0.5538)})
	--({\sx*(4.2700)},{\sy*(-0.5499)})
	--({\sx*(4.2800)},{\sy*(-0.5459)})
	--({\sx*(4.2900)},{\sy*(-0.5418)})
	--({\sx*(4.3000)},{\sy*(-0.5376)})
	--({\sx*(4.3100)},{\sy*(-0.5333)})
	--({\sx*(4.3200)},{\sy*(-0.5289)})
	--({\sx*(4.3300)},{\sy*(-0.5244)})
	--({\sx*(4.3400)},{\sy*(-0.5198)})
	--({\sx*(4.3500)},{\sy*(-0.5151)})
	--({\sx*(4.3600)},{\sy*(-0.5103)})
	--({\sx*(4.3700)},{\sy*(-0.5054)})
	--({\sx*(4.3800)},{\sy*(-0.5004)})
	--({\sx*(4.3900)},{\sy*(-0.4954)})
	--({\sx*(4.4000)},{\sy*(-0.4902)})
	--({\sx*(4.4100)},{\sy*(-0.4849)})
	--({\sx*(4.4200)},{\sy*(-0.4795)})
	--({\sx*(4.4300)},{\sy*(-0.4740)})
	--({\sx*(4.4400)},{\sy*(-0.4685)})
	--({\sx*(4.4500)},{\sy*(-0.4628)})
	--({\sx*(4.4600)},{\sy*(-0.4570)})
	--({\sx*(4.4700)},{\sy*(-0.4512)})
	--({\sx*(4.4800)},{\sy*(-0.4452)})
	--({\sx*(4.4900)},{\sy*(-0.4392)})
	--({\sx*(4.5000)},{\sy*(-0.4330)})
	--({\sx*(4.5100)},{\sy*(-0.4267)})
	--({\sx*(4.5200)},{\sy*(-0.4204)})
	--({\sx*(4.5300)},{\sy*(-0.4139)})
	--({\sx*(4.5400)},{\sy*(-0.4074)})
	--({\sx*(4.5500)},{\sy*(-0.4007)})
	--({\sx*(4.5600)},{\sy*(-0.3940)})
	--({\sx*(4.5700)},{\sy*(-0.3871)})
	--({\sx*(4.5800)},{\sy*(-0.3802)})
	--({\sx*(4.5900)},{\sy*(-0.3732)})
	--({\sx*(4.6000)},{\sy*(-0.3660)})
	--({\sx*(4.6100)},{\sy*(-0.3588)})
	--({\sx*(4.6200)},{\sy*(-0.3515)})
	--({\sx*(4.6300)},{\sy*(-0.3440)})
	--({\sx*(4.6400)},{\sy*(-0.3365)})
	--({\sx*(4.6500)},{\sy*(-0.3289)})
	--({\sx*(4.6600)},{\sy*(-0.3211)})
	--({\sx*(4.6700)},{\sy*(-0.3133)})
	--({\sx*(4.6800)},{\sy*(-0.3054)})
	--({\sx*(4.6900)},{\sy*(-0.2974)})
	--({\sx*(4.7000)},{\sy*(-0.2893)})
	--({\sx*(4.7100)},{\sy*(-0.2810)})
	--({\sx*(4.7200)},{\sy*(-0.2727)})
	--({\sx*(4.7300)},{\sy*(-0.2643)})
	--({\sx*(4.7400)},{\sy*(-0.2558)})
	--({\sx*(4.7500)},{\sy*(-0.2472)})
	--({\sx*(4.7600)},{\sy*(-0.2385)})
	--({\sx*(4.7700)},{\sy*(-0.2297)})
	--({\sx*(4.7800)},{\sy*(-0.2208)})
	--({\sx*(4.7900)},{\sy*(-0.2118)})
	--({\sx*(4.8000)},{\sy*(-0.2027)})
	--({\sx*(4.8100)},{\sy*(-0.1935)})
	--({\sx*(4.8200)},{\sy*(-0.1842)})
	--({\sx*(4.8300)},{\sy*(-0.1748)})
	--({\sx*(4.8400)},{\sy*(-0.1653)})
	--({\sx*(4.8500)},{\sy*(-0.1557)})
	--({\sx*(4.8600)},{\sy*(-0.1460)})
	--({\sx*(4.8700)},{\sy*(-0.1362)})
	--({\sx*(4.8800)},{\sy*(-0.1263)})
	--({\sx*(4.8900)},{\sy*(-0.1163)})
	--({\sx*(4.9000)},{\sy*(-0.1062)})
	--({\sx*(4.9100)},{\sy*(-0.0961)})
	--({\sx*(4.9200)},{\sy*(-0.0858)})
	--({\sx*(4.9300)},{\sy*(-0.0754)})
	--({\sx*(4.9400)},{\sy*(-0.0649)})
	--({\sx*(4.9500)},{\sy*(-0.0544)})
	--({\sx*(4.9600)},{\sy*(-0.0437)})
	--({\sx*(4.9700)},{\sy*(-0.0329)})
	--({\sx*(4.9800)},{\sy*(-0.0220)})
	--({\sx*(4.9900)},{\sy*(-0.0111)})
	--({\sx*(5.0000)},{\sy*(0.0000)});
}
\def\relfehlera{
\draw[color=blue,line width=1.4pt,line join=round] ({\sx*(0.000)},{\sy*(0.0000)})
	--({\sx*(0.0100)},{\sy*(-0.0056)})
	--({\sx*(0.0200)},{\sy*(-0.0112)})
	--({\sx*(0.0300)},{\sy*(-0.0167)})
	--({\sx*(0.0400)},{\sy*(-0.0222)})
	--({\sx*(0.0500)},{\sy*(-0.0276)})
	--({\sx*(0.0600)},{\sy*(-0.0329)})
	--({\sx*(0.0700)},{\sy*(-0.0382)})
	--({\sx*(0.0800)},{\sy*(-0.0435)})
	--({\sx*(0.0900)},{\sy*(-0.0486)})
	--({\sx*(0.1000)},{\sy*(-0.0538)})
	--({\sx*(0.1100)},{\sy*(-0.0588)})
	--({\sx*(0.1200)},{\sy*(-0.0638)})
	--({\sx*(0.1300)},{\sy*(-0.0687)})
	--({\sx*(0.1400)},{\sy*(-0.0736)})
	--({\sx*(0.1500)},{\sy*(-0.0784)})
	--({\sx*(0.1600)},{\sy*(-0.0831)})
	--({\sx*(0.1700)},{\sy*(-0.0878)})
	--({\sx*(0.1800)},{\sy*(-0.0924)})
	--({\sx*(0.1900)},{\sy*(-0.0969)})
	--({\sx*(0.2000)},{\sy*(-0.1014)})
	--({\sx*(0.2100)},{\sy*(-0.1058)})
	--({\sx*(0.2200)},{\sy*(-0.1101)})
	--({\sx*(0.2300)},{\sy*(-0.1144)})
	--({\sx*(0.2400)},{\sy*(-0.1185)})
	--({\sx*(0.2500)},{\sy*(-0.1226)})
	--({\sx*(0.2600)},{\sy*(-0.1267)})
	--({\sx*(0.2700)},{\sy*(-0.1306)})
	--({\sx*(0.2800)},{\sy*(-0.1345)})
	--({\sx*(0.2900)},{\sy*(-0.1383)})
	--({\sx*(0.3000)},{\sy*(-0.1420)})
	--({\sx*(0.3100)},{\sy*(-0.1457)})
	--({\sx*(0.3200)},{\sy*(-0.1492)})
	--({\sx*(0.3300)},{\sy*(-0.1527)})
	--({\sx*(0.3400)},{\sy*(-0.1561)})
	--({\sx*(0.3500)},{\sy*(-0.1594)})
	--({\sx*(0.3600)},{\sy*(-0.1627)})
	--({\sx*(0.3700)},{\sy*(-0.1659)})
	--({\sx*(0.3800)},{\sy*(-0.1689)})
	--({\sx*(0.3900)},{\sy*(-0.1719)})
	--({\sx*(0.4000)},{\sy*(-0.1748)})
	--({\sx*(0.4100)},{\sy*(-0.1777)})
	--({\sx*(0.4200)},{\sy*(-0.1804)})
	--({\sx*(0.4300)},{\sy*(-0.1831)})
	--({\sx*(0.4400)},{\sy*(-0.1857)})
	--({\sx*(0.4500)},{\sy*(-0.1882)})
	--({\sx*(0.4600)},{\sy*(-0.1906)})
	--({\sx*(0.4700)},{\sy*(-0.1929)})
	--({\sx*(0.4800)},{\sy*(-0.1951)})
	--({\sx*(0.4900)},{\sy*(-0.1973)})
	--({\sx*(0.5000)},{\sy*(-0.1993)})
	--({\sx*(0.5100)},{\sy*(-0.2013)})
	--({\sx*(0.5200)},{\sy*(-0.2032)})
	--({\sx*(0.5300)},{\sy*(-0.2050)})
	--({\sx*(0.5400)},{\sy*(-0.2067)})
	--({\sx*(0.5500)},{\sy*(-0.2084)})
	--({\sx*(0.5600)},{\sy*(-0.2099)})
	--({\sx*(0.5700)},{\sy*(-0.2114)})
	--({\sx*(0.5800)},{\sy*(-0.2127)})
	--({\sx*(0.5900)},{\sy*(-0.2140)})
	--({\sx*(0.6000)},{\sy*(-0.2152)})
	--({\sx*(0.6100)},{\sy*(-0.2163)})
	--({\sx*(0.6200)},{\sy*(-0.2173)})
	--({\sx*(0.6300)},{\sy*(-0.2182)})
	--({\sx*(0.6400)},{\sy*(-0.2191)})
	--({\sx*(0.6500)},{\sy*(-0.2198)})
	--({\sx*(0.6600)},{\sy*(-0.2205)})
	--({\sx*(0.6700)},{\sy*(-0.2211)})
	--({\sx*(0.6800)},{\sy*(-0.2216)})
	--({\sx*(0.6900)},{\sy*(-0.2220)})
	--({\sx*(0.7000)},{\sy*(-0.2223)})
	--({\sx*(0.7100)},{\sy*(-0.2225)})
	--({\sx*(0.7200)},{\sy*(-0.2227)})
	--({\sx*(0.7300)},{\sy*(-0.2227)})
	--({\sx*(0.7400)},{\sy*(-0.2227)})
	--({\sx*(0.7500)},{\sy*(-0.2226)})
	--({\sx*(0.7600)},{\sy*(-0.2224)})
	--({\sx*(0.7700)},{\sy*(-0.2221)})
	--({\sx*(0.7800)},{\sy*(-0.2218)})
	--({\sx*(0.7900)},{\sy*(-0.2213)})
	--({\sx*(0.8000)},{\sy*(-0.2208)})
	--({\sx*(0.8100)},{\sy*(-0.2202)})
	--({\sx*(0.8200)},{\sy*(-0.2195)})
	--({\sx*(0.8300)},{\sy*(-0.2187)})
	--({\sx*(0.8400)},{\sy*(-0.2179)})
	--({\sx*(0.8500)},{\sy*(-0.2169)})
	--({\sx*(0.8600)},{\sy*(-0.2159)})
	--({\sx*(0.8700)},{\sy*(-0.2148)})
	--({\sx*(0.8800)},{\sy*(-0.2136)})
	--({\sx*(0.8900)},{\sy*(-0.2124)})
	--({\sx*(0.9000)},{\sy*(-0.2111)})
	--({\sx*(0.9100)},{\sy*(-0.2097)})
	--({\sx*(0.9200)},{\sy*(-0.2082)})
	--({\sx*(0.9300)},{\sy*(-0.2066)})
	--({\sx*(0.9400)},{\sy*(-0.2050)})
	--({\sx*(0.9500)},{\sy*(-0.2033)})
	--({\sx*(0.9600)},{\sy*(-0.2015)})
	--({\sx*(0.9700)},{\sy*(-0.1997)})
	--({\sx*(0.9800)},{\sy*(-0.1977)})
	--({\sx*(0.9900)},{\sy*(-0.1957)})
	--({\sx*(1.0000)},{\sy*(-0.1937)})
	--({\sx*(1.0100)},{\sy*(-0.1915)})
	--({\sx*(1.0200)},{\sy*(-0.1894)})
	--({\sx*(1.0300)},{\sy*(-0.1871)})
	--({\sx*(1.0400)},{\sy*(-0.1848)})
	--({\sx*(1.0500)},{\sy*(-0.1824)})
	--({\sx*(1.0600)},{\sy*(-0.1799)})
	--({\sx*(1.0700)},{\sy*(-0.1774)})
	--({\sx*(1.0800)},{\sy*(-0.1748)})
	--({\sx*(1.0900)},{\sy*(-0.1721)})
	--({\sx*(1.1000)},{\sy*(-0.1694)})
	--({\sx*(1.1100)},{\sy*(-0.1667)})
	--({\sx*(1.1200)},{\sy*(-0.1638)})
	--({\sx*(1.1300)},{\sy*(-0.1610)})
	--({\sx*(1.1400)},{\sy*(-0.1580)})
	--({\sx*(1.1500)},{\sy*(-0.1550)})
	--({\sx*(1.1600)},{\sy*(-0.1520)})
	--({\sx*(1.1700)},{\sy*(-0.1489)})
	--({\sx*(1.1800)},{\sy*(-0.1457)})
	--({\sx*(1.1900)},{\sy*(-0.1425)})
	--({\sx*(1.2000)},{\sy*(-0.1393)})
	--({\sx*(1.2100)},{\sy*(-0.1359)})
	--({\sx*(1.2200)},{\sy*(-0.1326)})
	--({\sx*(1.2300)},{\sy*(-0.1292)})
	--({\sx*(1.2400)},{\sy*(-0.1258)})
	--({\sx*(1.2500)},{\sy*(-0.1223)})
	--({\sx*(1.2600)},{\sy*(-0.1187)})
	--({\sx*(1.2700)},{\sy*(-0.1152)})
	--({\sx*(1.2800)},{\sy*(-0.1116)})
	--({\sx*(1.2900)},{\sy*(-0.1079)})
	--({\sx*(1.3000)},{\sy*(-0.1042)})
	--({\sx*(1.3100)},{\sy*(-0.1005)})
	--({\sx*(1.3200)},{\sy*(-0.0967)})
	--({\sx*(1.3300)},{\sy*(-0.0929)})
	--({\sx*(1.3400)},{\sy*(-0.0891)})
	--({\sx*(1.3500)},{\sy*(-0.0853)})
	--({\sx*(1.3600)},{\sy*(-0.0814)})
	--({\sx*(1.3700)},{\sy*(-0.0774)})
	--({\sx*(1.3800)},{\sy*(-0.0735)})
	--({\sx*(1.3900)},{\sy*(-0.0695)})
	--({\sx*(1.4000)},{\sy*(-0.0655)})
	--({\sx*(1.4100)},{\sy*(-0.0615)})
	--({\sx*(1.4200)},{\sy*(-0.0575)})
	--({\sx*(1.4300)},{\sy*(-0.0534)})
	--({\sx*(1.4400)},{\sy*(-0.0493)})
	--({\sx*(1.4500)},{\sy*(-0.0452)})
	--({\sx*(1.4600)},{\sy*(-0.0411)})
	--({\sx*(1.4700)},{\sy*(-0.0369)})
	--({\sx*(1.4800)},{\sy*(-0.0328)})
	--({\sx*(1.4900)},{\sy*(-0.0286)})
	--({\sx*(1.5000)},{\sy*(-0.0244)})
	--({\sx*(1.5100)},{\sy*(-0.0203)})
	--({\sx*(1.5200)},{\sy*(-0.0161)})
	--({\sx*(1.5300)},{\sy*(-0.0119)})
	--({\sx*(1.5400)},{\sy*(-0.0076)})
	--({\sx*(1.5500)},{\sy*(-0.0034)})
	--({\sx*(1.5600)},{\sy*(0.0008)})
	--({\sx*(1.5700)},{\sy*(0.0050)})
	--({\sx*(1.5800)},{\sy*(0.0092)})
	--({\sx*(1.5900)},{\sy*(0.0135)})
	--({\sx*(1.6000)},{\sy*(0.0177)})
	--({\sx*(1.6100)},{\sy*(0.0219)})
	--({\sx*(1.6200)},{\sy*(0.0261)})
	--({\sx*(1.6300)},{\sy*(0.0303)})
	--({\sx*(1.6400)},{\sy*(0.0345)})
	--({\sx*(1.6500)},{\sy*(0.0387)})
	--({\sx*(1.6600)},{\sy*(0.0429)})
	--({\sx*(1.6700)},{\sy*(0.0470)})
	--({\sx*(1.6800)},{\sy*(0.0512)})
	--({\sx*(1.6900)},{\sy*(0.0553)})
	--({\sx*(1.7000)},{\sy*(0.0594)})
	--({\sx*(1.7100)},{\sy*(0.0635)})
	--({\sx*(1.7200)},{\sy*(0.0676)})
	--({\sx*(1.7300)},{\sy*(0.0716)})
	--({\sx*(1.7400)},{\sy*(0.0757)})
	--({\sx*(1.7500)},{\sy*(0.0797)})
	--({\sx*(1.7600)},{\sy*(0.0836)})
	--({\sx*(1.7700)},{\sy*(0.0876)})
	--({\sx*(1.7800)},{\sy*(0.0915)})
	--({\sx*(1.7900)},{\sy*(0.0954)})
	--({\sx*(1.8000)},{\sy*(0.0992)})
	--({\sx*(1.8100)},{\sy*(0.1030)})
	--({\sx*(1.8200)},{\sy*(0.1068)})
	--({\sx*(1.8300)},{\sy*(0.1105)})
	--({\sx*(1.8400)},{\sy*(0.1142)})
	--({\sx*(1.8500)},{\sy*(0.1178)})
	--({\sx*(1.8600)},{\sy*(0.1214)})
	--({\sx*(1.8700)},{\sy*(0.1250)})
	--({\sx*(1.8800)},{\sy*(0.1285)})
	--({\sx*(1.8900)},{\sy*(0.1319)})
	--({\sx*(1.9000)},{\sy*(0.1353)})
	--({\sx*(1.9100)},{\sy*(0.1386)})
	--({\sx*(1.9200)},{\sy*(0.1419)})
	--({\sx*(1.9300)},{\sy*(0.1451)})
	--({\sx*(1.9400)},{\sy*(0.1482)})
	--({\sx*(1.9500)},{\sy*(0.1513)})
	--({\sx*(1.9600)},{\sy*(0.1543)})
	--({\sx*(1.9700)},{\sy*(0.1572)})
	--({\sx*(1.9800)},{\sy*(0.1601)})
	--({\sx*(1.9900)},{\sy*(0.1628)})
	--({\sx*(2.0000)},{\sy*(0.1655)})
	--({\sx*(2.0100)},{\sy*(0.1681)})
	--({\sx*(2.0200)},{\sy*(0.1706)})
	--({\sx*(2.0300)},{\sy*(0.1730)})
	--({\sx*(2.0400)},{\sy*(0.1754)})
	--({\sx*(2.0500)},{\sy*(0.1776)})
	--({\sx*(2.0600)},{\sy*(0.1797)})
	--({\sx*(2.0700)},{\sy*(0.1817)})
	--({\sx*(2.0800)},{\sy*(0.1836)})
	--({\sx*(2.0900)},{\sy*(0.1854)})
	--({\sx*(2.1000)},{\sy*(0.1870)})
	--({\sx*(2.1100)},{\sy*(0.1885)})
	--({\sx*(2.1200)},{\sy*(0.1899)})
	--({\sx*(2.1300)},{\sy*(0.1912)})
	--({\sx*(2.1400)},{\sy*(0.1922)})
	--({\sx*(2.1500)},{\sy*(0.1932)})
	--({\sx*(2.1600)},{\sy*(0.1940)})
	--({\sx*(2.1700)},{\sy*(0.1946)})
	--({\sx*(2.1800)},{\sy*(0.1950)})
	--({\sx*(2.1900)},{\sy*(0.1953)})
	--({\sx*(2.2000)},{\sy*(0.1953)})
	--({\sx*(2.2100)},{\sy*(0.1952)})
	--({\sx*(2.2200)},{\sy*(0.1948)})
	--({\sx*(2.2300)},{\sy*(0.1942)})
	--({\sx*(2.2400)},{\sy*(0.1934)})
	--({\sx*(2.2500)},{\sy*(0.1923)})
	--({\sx*(2.2600)},{\sy*(0.1909)})
	--({\sx*(2.2700)},{\sy*(0.1893)})
	--({\sx*(2.2800)},{\sy*(0.1873)})
	--({\sx*(2.2900)},{\sy*(0.1851)})
	--({\sx*(2.3000)},{\sy*(0.1825)})
	--({\sx*(2.3100)},{\sy*(0.1795)})
	--({\sx*(2.3200)},{\sy*(0.1762)})
	--({\sx*(2.3300)},{\sy*(0.1725)})
	--({\sx*(2.3400)},{\sy*(0.1683)})
	--({\sx*(2.3500)},{\sy*(0.1636)})
	--({\sx*(2.3600)},{\sy*(0.1585)})
	--({\sx*(2.3700)},{\sy*(0.1528)})
	--({\sx*(2.3800)},{\sy*(0.1465)})
	--({\sx*(2.3900)},{\sy*(0.1395)})
	--({\sx*(2.4000)},{\sy*(0.1319)})
	--({\sx*(2.4100)},{\sy*(0.1236)})
	--({\sx*(2.4200)},{\sy*(0.1144)})
	--({\sx*(2.4300)},{\sy*(0.1043)})
	--({\sx*(2.4400)},{\sy*(0.0933)})
	--({\sx*(2.4500)},{\sy*(0.0812)})
	--({\sx*(2.4600)},{\sy*(0.0679)})
	--({\sx*(2.4700)},{\sy*(0.0533)})
	--({\sx*(2.4800)},{\sy*(0.0372)})
	--({\sx*(2.4900)},{\sy*(0.0195)})
	--({\sx*(2.5000)},{\sy*(0.0000)})
	--({\sx*(2.5100)},{\sy*(-0.0216)})
	--({\sx*(2.5200)},{\sy*(-0.0455)})
	--({\sx*(2.5300)},{\sy*(-0.0721)})
	--({\sx*(2.5400)},{\sy*(-0.1018)})
	--({\sx*(2.5500)},{\sy*(-0.1350)})
	--({\sx*(2.5600)},{\sy*(-0.1723)})
	--({\sx*(2.5700)},{\sy*(-0.2145)})
	--({\sx*(2.5800)},{\sy*(-0.2624)})
	--({\sx*(2.5900)},{\sy*(-0.3172)})
	--({\sx*(2.6000)},{\sy*(-0.3803)})
	--({\sx*(2.6100)},{\sy*(-0.4536)})
	--({\sx*(2.6200)},{\sy*(-0.5397)})
	--({\sx*(2.6300)},{\sy*(-0.6421)})
	--({\sx*(2.6400)},{\sy*(-0.7654)})
	--({\sx*(2.6500)},{\sy*(-0.9167)})
	--({\sx*(2.6600)},{\sy*(-1.1063)})
	--({\sx*(2.6700)},{\sy*(-1.3504)})
	--({\sx*(2.6800)},{\sy*(-1.6759)})
	--({\sx*(2.6900)},{\sy*(-2.1308)})
	--({\sx*(2.7000)},{\sy*(-2.8101)})
	--({\sx*(2.7100)},{\sy*(-3.9319)})
	--({\sx*(2.7200)},{\sy*(-6.1330)})
	--({\sx*(2.7300)},{\sy*(-12.3973)})
	--({\sx*(2.7400)},{\sy*(-166.0998)})
	--({\sx*(2.7500)},{\sy*(16.1736)})
	--({\sx*(2.7600)},{\sy*(8.0844)})
	--({\sx*(2.7700)},{\sy*(5.5475)})
	--({\sx*(2.7800)},{\sy*(4.3081)})
	--({\sx*(2.7900)},{\sy*(3.5742)})
	--({\sx*(2.8000)},{\sy*(3.0895)})
	--({\sx*(2.8100)},{\sy*(2.7458)})
	--({\sx*(2.8200)},{\sy*(2.4897)})
	--({\sx*(2.8300)},{\sy*(2.2917)})
	--({\sx*(2.8400)},{\sy*(2.1342)})
	--({\sx*(2.8500)},{\sy*(2.0060)})
	--({\sx*(2.8600)},{\sy*(1.8999)})
	--({\sx*(2.8700)},{\sy*(1.8105)})
	--({\sx*(2.8800)},{\sy*(1.7344)})
	--({\sx*(2.8900)},{\sy*(1.6688)})
	--({\sx*(2.9000)},{\sy*(1.6118)})
	--({\sx*(2.9100)},{\sy*(1.5618)})
	--({\sx*(2.9200)},{\sy*(1.5177)})
	--({\sx*(2.9300)},{\sy*(1.4785)})
	--({\sx*(2.9400)},{\sy*(1.4435)})
	--({\sx*(2.9500)},{\sy*(1.4120)})
	--({\sx*(2.9600)},{\sy*(1.3836)})
	--({\sx*(2.9700)},{\sy*(1.3579)})
	--({\sx*(2.9800)},{\sy*(1.3346)})
	--({\sx*(2.9900)},{\sy*(1.3132)})
	--({\sx*(3.0000)},{\sy*(1.2937)})
	--({\sx*(3.0100)},{\sy*(1.2758)})
	--({\sx*(3.0200)},{\sy*(1.2593)})
	--({\sx*(3.0300)},{\sy*(1.2442)})
	--({\sx*(3.0400)},{\sy*(1.2301)})
	--({\sx*(3.0500)},{\sy*(1.2171)})
	--({\sx*(3.0600)},{\sy*(1.2050)})
	--({\sx*(3.0700)},{\sy*(1.1938)})
	--({\sx*(3.0800)},{\sy*(1.1833)})
	--({\sx*(3.0900)},{\sy*(1.1736)})
	--({\sx*(3.1000)},{\sy*(1.1645)})
	--({\sx*(3.1100)},{\sy*(1.1559)})
	--({\sx*(3.1200)},{\sy*(1.1479)})
	--({\sx*(3.1300)},{\sy*(1.1405)})
	--({\sx*(3.1400)},{\sy*(1.1334)})
	--({\sx*(3.1500)},{\sy*(1.1268)})
	--({\sx*(3.1600)},{\sy*(1.1206)})
	--({\sx*(3.1700)},{\sy*(1.1147)})
	--({\sx*(3.1800)},{\sy*(1.1092)})
	--({\sx*(3.1900)},{\sy*(1.1040)})
	--({\sx*(3.2000)},{\sy*(1.0991)})
	--({\sx*(3.2100)},{\sy*(1.0944)})
	--({\sx*(3.2200)},{\sy*(1.0901)})
	--({\sx*(3.2300)},{\sy*(1.0859)})
	--({\sx*(3.2400)},{\sy*(1.0820)})
	--({\sx*(3.2500)},{\sy*(1.0782)})
	--({\sx*(3.2600)},{\sy*(1.0747)})
	--({\sx*(3.2700)},{\sy*(1.0713)})
	--({\sx*(3.2800)},{\sy*(1.0681)})
	--({\sx*(3.2900)},{\sy*(1.0651)})
	--({\sx*(3.3000)},{\sy*(1.0623)})
	--({\sx*(3.3100)},{\sy*(1.0595)})
	--({\sx*(3.3200)},{\sy*(1.0569)})
	--({\sx*(3.3300)},{\sy*(1.0544)})
	--({\sx*(3.3400)},{\sy*(1.0521)})
	--({\sx*(3.3500)},{\sy*(1.0499)})
	--({\sx*(3.3600)},{\sy*(1.0477)})
	--({\sx*(3.3700)},{\sy*(1.0457)})
	--({\sx*(3.3800)},{\sy*(1.0437)})
	--({\sx*(3.3900)},{\sy*(1.0419)})
	--({\sx*(3.4000)},{\sy*(1.0401)})
	--({\sx*(3.4100)},{\sy*(1.0385)})
	--({\sx*(3.4200)},{\sy*(1.0368)})
	--({\sx*(3.4300)},{\sy*(1.0353)})
	--({\sx*(3.4400)},{\sy*(1.0338)})
	--({\sx*(3.4500)},{\sy*(1.0324)})
	--({\sx*(3.4600)},{\sy*(1.0311)})
	--({\sx*(3.4700)},{\sy*(1.0298)})
	--({\sx*(3.4800)},{\sy*(1.0286)})
	--({\sx*(3.4900)},{\sy*(1.0274)})
	--({\sx*(3.5000)},{\sy*(1.0263)})
	--({\sx*(3.5100)},{\sy*(1.0253)})
	--({\sx*(3.5200)},{\sy*(1.0242)})
	--({\sx*(3.5300)},{\sy*(1.0233)})
	--({\sx*(3.5400)},{\sy*(1.0223)})
	--({\sx*(3.5500)},{\sy*(1.0214)})
	--({\sx*(3.5600)},{\sy*(1.0206)})
	--({\sx*(3.5700)},{\sy*(1.0197)})
	--({\sx*(3.5800)},{\sy*(1.0190)})
	--({\sx*(3.5900)},{\sy*(1.0182)})
	--({\sx*(3.6000)},{\sy*(1.0175)})
	--({\sx*(3.6100)},{\sy*(1.0168)})
	--({\sx*(3.6200)},{\sy*(1.0161)})
	--({\sx*(3.6300)},{\sy*(1.0155)})
	--({\sx*(3.6400)},{\sy*(1.0149)})
	--({\sx*(3.6500)},{\sy*(1.0143)})
	--({\sx*(3.6600)},{\sy*(1.0137)})
	--({\sx*(3.6700)},{\sy*(1.0132)})
	--({\sx*(3.6800)},{\sy*(1.0127)})
	--({\sx*(3.6900)},{\sy*(1.0122)})
	--({\sx*(3.7000)},{\sy*(1.0117)})
	--({\sx*(3.7100)},{\sy*(1.0112)})
	--({\sx*(3.7200)},{\sy*(1.0108)})
	--({\sx*(3.7300)},{\sy*(1.0104)})
	--({\sx*(3.7400)},{\sy*(1.0100)})
	--({\sx*(3.7500)},{\sy*(1.0096)})
	--({\sx*(3.7600)},{\sy*(1.0092)})
	--({\sx*(3.7700)},{\sy*(1.0089)})
	--({\sx*(3.7800)},{\sy*(1.0085)})
	--({\sx*(3.7900)},{\sy*(1.0082)})
	--({\sx*(3.8000)},{\sy*(1.0079)})
	--({\sx*(3.8100)},{\sy*(1.0076)})
	--({\sx*(3.8200)},{\sy*(1.0073)})
	--({\sx*(3.8300)},{\sy*(1.0070)})
	--({\sx*(3.8400)},{\sy*(1.0068)})
	--({\sx*(3.8500)},{\sy*(1.0065)})
	--({\sx*(3.8600)},{\sy*(1.0062)})
	--({\sx*(3.8700)},{\sy*(1.0060)})
	--({\sx*(3.8800)},{\sy*(1.0058)})
	--({\sx*(3.8900)},{\sy*(1.0056)})
	--({\sx*(3.9000)},{\sy*(1.0054)})
	--({\sx*(3.9100)},{\sy*(1.0051)})
	--({\sx*(3.9200)},{\sy*(1.0050)})
	--({\sx*(3.9300)},{\sy*(1.0048)})
	--({\sx*(3.9400)},{\sy*(1.0046)})
	--({\sx*(3.9500)},{\sy*(1.0044)})
	--({\sx*(3.9600)},{\sy*(1.0043)})
	--({\sx*(3.9700)},{\sy*(1.0041)})
	--({\sx*(3.9800)},{\sy*(1.0039)})
	--({\sx*(3.9900)},{\sy*(1.0038)})
	--({\sx*(4.0000)},{\sy*(1.0037)})
	--({\sx*(4.0100)},{\sy*(1.0035)})
	--({\sx*(4.0200)},{\sy*(1.0034)})
	--({\sx*(4.0300)},{\sy*(1.0033)})
	--({\sx*(4.0400)},{\sy*(1.0031)})
	--({\sx*(4.0500)},{\sy*(1.0030)})
	--({\sx*(4.0600)},{\sy*(1.0029)})
	--({\sx*(4.0700)},{\sy*(1.0028)})
	--({\sx*(4.0800)},{\sy*(1.0027)})
	--({\sx*(4.0900)},{\sy*(1.0026)})
	--({\sx*(4.1000)},{\sy*(1.0025)})
	--({\sx*(4.1100)},{\sy*(1.0024)})
	--({\sx*(4.1200)},{\sy*(1.0023)})
	--({\sx*(4.1300)},{\sy*(1.0022)})
	--({\sx*(4.1400)},{\sy*(1.0022)})
	--({\sx*(4.1500)},{\sy*(1.0021)})
	--({\sx*(4.1600)},{\sy*(1.0020)})
	--({\sx*(4.1700)},{\sy*(1.0019)})
	--({\sx*(4.1800)},{\sy*(1.0019)})
	--({\sx*(4.1900)},{\sy*(1.0018)})
	--({\sx*(4.2000)},{\sy*(1.0017)})
	--({\sx*(4.2100)},{\sy*(1.0017)})
	--({\sx*(4.2200)},{\sy*(1.0016)})
	--({\sx*(4.2300)},{\sy*(1.0016)})
	--({\sx*(4.2400)},{\sy*(1.0015)})
	--({\sx*(4.2500)},{\sy*(1.0014)})
	--({\sx*(4.2600)},{\sy*(1.0014)})
	--({\sx*(4.2700)},{\sy*(1.0013)})
	--({\sx*(4.2800)},{\sy*(1.0013)})
	--({\sx*(4.2900)},{\sy*(1.0013)})
	--({\sx*(4.3000)},{\sy*(1.0012)})
	--({\sx*(4.3100)},{\sy*(1.0012)})
	--({\sx*(4.3200)},{\sy*(1.0011)})
	--({\sx*(4.3300)},{\sy*(1.0011)})
	--({\sx*(4.3400)},{\sy*(1.0011)})
	--({\sx*(4.3500)},{\sy*(1.0010)})
	--({\sx*(4.3600)},{\sy*(1.0010)})
	--({\sx*(4.3700)},{\sy*(1.0010)})
	--({\sx*(4.3800)},{\sy*(1.0009)})
	--({\sx*(4.3900)},{\sy*(1.0009)})
	--({\sx*(4.4000)},{\sy*(1.0009)})
	--({\sx*(4.4100)},{\sy*(1.0008)})
	--({\sx*(4.4200)},{\sy*(1.0008)})
	--({\sx*(4.4300)},{\sy*(1.0008)})
	--({\sx*(4.4400)},{\sy*(1.0008)})
	--({\sx*(4.4500)},{\sy*(1.0007)})
	--({\sx*(4.4600)},{\sy*(1.0007)})
	--({\sx*(4.4700)},{\sy*(1.0007)})
	--({\sx*(4.4800)},{\sy*(1.0007)})
	--({\sx*(4.4900)},{\sy*(1.0006)})
	--({\sx*(4.5000)},{\sy*(1.0006)})
	--({\sx*(4.5100)},{\sy*(1.0006)})
	--({\sx*(4.5200)},{\sy*(1.0006)})
	--({\sx*(4.5300)},{\sy*(1.0006)})
	--({\sx*(4.5400)},{\sy*(1.0006)})
	--({\sx*(4.5500)},{\sy*(1.0005)})
	--({\sx*(4.5600)},{\sy*(1.0005)})
	--({\sx*(4.5700)},{\sy*(1.0005)})
	--({\sx*(4.5800)},{\sy*(1.0005)})
	--({\sx*(4.5900)},{\sy*(1.0005)})
	--({\sx*(4.6000)},{\sy*(1.0005)})
	--({\sx*(4.6100)},{\sy*(1.0005)})
	--({\sx*(4.6200)},{\sy*(1.0004)})
	--({\sx*(4.6300)},{\sy*(1.0004)})
	--({\sx*(4.6400)},{\sy*(1.0004)})
	--({\sx*(4.6500)},{\sy*(1.0004)})
	--({\sx*(4.6600)},{\sy*(1.0004)})
	--({\sx*(4.6700)},{\sy*(1.0004)})
	--({\sx*(4.6800)},{\sy*(1.0004)})
	--({\sx*(4.6900)},{\sy*(1.0004)})
	--({\sx*(4.7000)},{\sy*(1.0004)})
	--({\sx*(4.7100)},{\sy*(1.0004)})
	--({\sx*(4.7200)},{\sy*(1.0004)})
	--({\sx*(4.7300)},{\sy*(1.0004)})
	--({\sx*(4.7400)},{\sy*(1.0003)})
	--({\sx*(4.7500)},{\sy*(1.0003)})
	--({\sx*(4.7600)},{\sy*(1.0003)})
	--({\sx*(4.7700)},{\sy*(1.0003)})
	--({\sx*(4.7800)},{\sy*(1.0003)})
	--({\sx*(4.7900)},{\sy*(1.0003)})
	--({\sx*(4.8000)},{\sy*(1.0003)})
	--({\sx*(4.8100)},{\sy*(1.0003)})
	--({\sx*(4.8200)},{\sy*(1.0003)})
	--({\sx*(4.8300)},{\sy*(1.0003)})
	--({\sx*(4.8400)},{\sy*(1.0003)})
	--({\sx*(4.8500)},{\sy*(1.0003)})
	--({\sx*(4.8600)},{\sy*(1.0003)})
	--({\sx*(4.8700)},{\sy*(1.0004)})
	--({\sx*(4.8800)},{\sy*(1.0004)})
	--({\sx*(4.8900)},{\sy*(1.0004)})
	--({\sx*(4.9000)},{\sy*(1.0004)})
	--({\sx*(4.9100)},{\sy*(1.0004)})
	--({\sx*(4.9200)},{\sy*(1.0004)})
	--({\sx*(4.9300)},{\sy*(1.0005)})
	--({\sx*(4.9400)},{\sy*(1.0005)})
	--({\sx*(4.9500)},{\sy*(1.0006)})
	--({\sx*(4.9600)},{\sy*(1.0007)})
	--({\sx*(4.9700)},{\sy*(1.0009)})
	--({\sx*(4.9800)},{\sy*(1.0013)})
	--({\sx*(4.9900)},{\sy*(1.0024)})
	--({\sx*(5.0000)},{\sy*(0.0000)});
}
\def\xwerteb{
\fill[color=red] (0.0000,0) circle[radius={0.07/\skala}];
\fill[color=white] (0.0000,0) circle[radius={0.05/\skala}];
\fill[color=red] (0.7322,0) circle[radius={0.07/\skala}];
\fill[color=white] (0.7322,0) circle[radius={0.05/\skala}];
\fill[color=red] (2.5000,0) circle[radius={0.07/\skala}];
\fill[color=white] (2.5000,0) circle[radius={0.05/\skala}];
\fill[color=red] (4.2678,0) circle[radius={0.07/\skala}];
\fill[color=white] (4.2678,0) circle[radius={0.05/\skala}];
\fill[color=red] (5.0000,0) circle[radius={0.07/\skala}];
\fill[color=white] (5.0000,0) circle[radius={0.05/\skala}];
}
\def\punkteb{4}
\def\maxfehlerb{3.208\cdot 10^{-2}}
\def\fehlerb{
\draw[color=red,line width=1.4pt,line join=round] ({\sx*(0.000)},{\sy*(0.0000)})
	--({\sx*(0.0100)},{\sy*(-0.0179)})
	--({\sx*(0.0200)},{\sy*(-0.0354)})
	--({\sx*(0.0300)},{\sy*(-0.0524)})
	--({\sx*(0.0400)},{\sy*(-0.0690)})
	--({\sx*(0.0500)},{\sy*(-0.0850)})
	--({\sx*(0.0600)},{\sy*(-0.1006)})
	--({\sx*(0.0700)},{\sy*(-0.1157)})
	--({\sx*(0.0800)},{\sy*(-0.1303)})
	--({\sx*(0.0900)},{\sy*(-0.1443)})
	--({\sx*(0.1000)},{\sy*(-0.1579)})
	--({\sx*(0.1100)},{\sy*(-0.1709)})
	--({\sx*(0.1200)},{\sy*(-0.1835)})
	--({\sx*(0.1300)},{\sy*(-0.1955)})
	--({\sx*(0.1400)},{\sy*(-0.2069)})
	--({\sx*(0.1500)},{\sy*(-0.2179)})
	--({\sx*(0.1600)},{\sy*(-0.2283)})
	--({\sx*(0.1700)},{\sy*(-0.2381)})
	--({\sx*(0.1800)},{\sy*(-0.2475)})
	--({\sx*(0.1900)},{\sy*(-0.2562)})
	--({\sx*(0.2000)},{\sy*(-0.2645)})
	--({\sx*(0.2100)},{\sy*(-0.2722)})
	--({\sx*(0.2200)},{\sy*(-0.2793)})
	--({\sx*(0.2300)},{\sy*(-0.2859)})
	--({\sx*(0.2400)},{\sy*(-0.2920)})
	--({\sx*(0.2500)},{\sy*(-0.2975)})
	--({\sx*(0.2600)},{\sy*(-0.3025)})
	--({\sx*(0.2700)},{\sy*(-0.3069)})
	--({\sx*(0.2800)},{\sy*(-0.3108)})
	--({\sx*(0.2900)},{\sy*(-0.3141)})
	--({\sx*(0.3000)},{\sy*(-0.3169)})
	--({\sx*(0.3100)},{\sy*(-0.3192)})
	--({\sx*(0.3200)},{\sy*(-0.3209)})
	--({\sx*(0.3300)},{\sy*(-0.3221)})
	--({\sx*(0.3400)},{\sy*(-0.3228)})
	--({\sx*(0.3500)},{\sy*(-0.3229)})
	--({\sx*(0.3600)},{\sy*(-0.3225)})
	--({\sx*(0.3700)},{\sy*(-0.3216)})
	--({\sx*(0.3800)},{\sy*(-0.3203)})
	--({\sx*(0.3900)},{\sy*(-0.3183)})
	--({\sx*(0.4000)},{\sy*(-0.3159)})
	--({\sx*(0.4100)},{\sy*(-0.3130)})
	--({\sx*(0.4200)},{\sy*(-0.3097)})
	--({\sx*(0.4300)},{\sy*(-0.3058)})
	--({\sx*(0.4400)},{\sy*(-0.3014)})
	--({\sx*(0.4500)},{\sy*(-0.2966)})
	--({\sx*(0.4600)},{\sy*(-0.2913)})
	--({\sx*(0.4700)},{\sy*(-0.2856)})
	--({\sx*(0.4800)},{\sy*(-0.2794)})
	--({\sx*(0.4900)},{\sy*(-0.2728)})
	--({\sx*(0.5000)},{\sy*(-0.2657)})
	--({\sx*(0.5100)},{\sy*(-0.2582)})
	--({\sx*(0.5200)},{\sy*(-0.2503)})
	--({\sx*(0.5300)},{\sy*(-0.2420)})
	--({\sx*(0.5400)},{\sy*(-0.2332)})
	--({\sx*(0.5500)},{\sy*(-0.2241)})
	--({\sx*(0.5600)},{\sy*(-0.2146)})
	--({\sx*(0.5700)},{\sy*(-0.2047)})
	--({\sx*(0.5800)},{\sy*(-0.1945)})
	--({\sx*(0.5900)},{\sy*(-0.1838)})
	--({\sx*(0.6000)},{\sy*(-0.1729)})
	--({\sx*(0.6100)},{\sy*(-0.1616)})
	--({\sx*(0.6200)},{\sy*(-0.1500)})
	--({\sx*(0.6300)},{\sy*(-0.1380)})
	--({\sx*(0.6400)},{\sy*(-0.1258)})
	--({\sx*(0.6500)},{\sy*(-0.1132)})
	--({\sx*(0.6600)},{\sy*(-0.1004)})
	--({\sx*(0.6700)},{\sy*(-0.0873)})
	--({\sx*(0.6800)},{\sy*(-0.0739)})
	--({\sx*(0.6900)},{\sy*(-0.0602)})
	--({\sx*(0.7000)},{\sy*(-0.0463)})
	--({\sx*(0.7100)},{\sy*(-0.0322)})
	--({\sx*(0.7200)},{\sy*(-0.0179)})
	--({\sx*(0.7300)},{\sy*(-0.0033)})
	--({\sx*(0.7400)},{\sy*(0.0115)})
	--({\sx*(0.7500)},{\sy*(0.0265)})
	--({\sx*(0.7600)},{\sy*(0.0416)})
	--({\sx*(0.7700)},{\sy*(0.0569)})
	--({\sx*(0.7800)},{\sy*(0.0724)})
	--({\sx*(0.7900)},{\sy*(0.0881)})
	--({\sx*(0.8000)},{\sy*(0.1038)})
	--({\sx*(0.8100)},{\sy*(0.1197)})
	--({\sx*(0.8200)},{\sy*(0.1358)})
	--({\sx*(0.8300)},{\sy*(0.1519)})
	--({\sx*(0.8400)},{\sy*(0.1681)})
	--({\sx*(0.8500)},{\sy*(0.1844)})
	--({\sx*(0.8600)},{\sy*(0.2008)})
	--({\sx*(0.8700)},{\sy*(0.2172)})
	--({\sx*(0.8800)},{\sy*(0.2337)})
	--({\sx*(0.8900)},{\sy*(0.2503)})
	--({\sx*(0.9000)},{\sy*(0.2668)})
	--({\sx*(0.9100)},{\sy*(0.2834)})
	--({\sx*(0.9200)},{\sy*(0.3000)})
	--({\sx*(0.9300)},{\sy*(0.3166)})
	--({\sx*(0.9400)},{\sy*(0.3332)})
	--({\sx*(0.9500)},{\sy*(0.3498)})
	--({\sx*(0.9600)},{\sy*(0.3663)})
	--({\sx*(0.9700)},{\sy*(0.3828)})
	--({\sx*(0.9800)},{\sy*(0.3992)})
	--({\sx*(0.9900)},{\sy*(0.4156)})
	--({\sx*(1.0000)},{\sy*(0.4319)})
	--({\sx*(1.0100)},{\sy*(0.4482)})
	--({\sx*(1.0200)},{\sy*(0.4643)})
	--({\sx*(1.0300)},{\sy*(0.4803)})
	--({\sx*(1.0400)},{\sy*(0.4963)})
	--({\sx*(1.0500)},{\sy*(0.5121)})
	--({\sx*(1.0600)},{\sy*(0.5277)})
	--({\sx*(1.0700)},{\sy*(0.5433)})
	--({\sx*(1.0800)},{\sy*(0.5587)})
	--({\sx*(1.0900)},{\sy*(0.5739)})
	--({\sx*(1.1000)},{\sy*(0.5890)})
	--({\sx*(1.1100)},{\sy*(0.6039)})
	--({\sx*(1.1200)},{\sy*(0.6187)})
	--({\sx*(1.1300)},{\sy*(0.6332)})
	--({\sx*(1.1400)},{\sy*(0.6476)})
	--({\sx*(1.1500)},{\sy*(0.6617)})
	--({\sx*(1.1600)},{\sy*(0.6757)})
	--({\sx*(1.1700)},{\sy*(0.6894)})
	--({\sx*(1.1800)},{\sy*(0.7029)})
	--({\sx*(1.1900)},{\sy*(0.7162)})
	--({\sx*(1.2000)},{\sy*(0.7292)})
	--({\sx*(1.2100)},{\sy*(0.7420)})
	--({\sx*(1.2200)},{\sy*(0.7545)})
	--({\sx*(1.2300)},{\sy*(0.7668)})
	--({\sx*(1.2400)},{\sy*(0.7788)})
	--({\sx*(1.2500)},{\sy*(0.7905)})
	--({\sx*(1.2600)},{\sy*(0.8020)})
	--({\sx*(1.2700)},{\sy*(0.8132)})
	--({\sx*(1.2800)},{\sy*(0.8241)})
	--({\sx*(1.2900)},{\sy*(0.8347)})
	--({\sx*(1.3000)},{\sy*(0.8450)})
	--({\sx*(1.3100)},{\sy*(0.8550)})
	--({\sx*(1.3200)},{\sy*(0.8647)})
	--({\sx*(1.3300)},{\sy*(0.8741)})
	--({\sx*(1.3400)},{\sy*(0.8832)})
	--({\sx*(1.3500)},{\sy*(0.8919)})
	--({\sx*(1.3600)},{\sy*(0.9004)})
	--({\sx*(1.3700)},{\sy*(0.9085)})
	--({\sx*(1.3800)},{\sy*(0.9163)})
	--({\sx*(1.3900)},{\sy*(0.9237)})
	--({\sx*(1.4000)},{\sy*(0.9309)})
	--({\sx*(1.4100)},{\sy*(0.9376)})
	--({\sx*(1.4200)},{\sy*(0.9441)})
	--({\sx*(1.4300)},{\sy*(0.9502)})
	--({\sx*(1.4400)},{\sy*(0.9559)})
	--({\sx*(1.4500)},{\sy*(0.9613)})
	--({\sx*(1.4600)},{\sy*(0.9664)})
	--({\sx*(1.4700)},{\sy*(0.9711)})
	--({\sx*(1.4800)},{\sy*(0.9755)})
	--({\sx*(1.4900)},{\sy*(0.9795)})
	--({\sx*(1.5000)},{\sy*(0.9831)})
	--({\sx*(1.5100)},{\sy*(0.9864)})
	--({\sx*(1.5200)},{\sy*(0.9893)})
	--({\sx*(1.5300)},{\sy*(0.9919)})
	--({\sx*(1.5400)},{\sy*(0.9941)})
	--({\sx*(1.5500)},{\sy*(0.9960)})
	--({\sx*(1.5600)},{\sy*(0.9975)})
	--({\sx*(1.5700)},{\sy*(0.9987)})
	--({\sx*(1.5800)},{\sy*(0.9995)})
	--({\sx*(1.5900)},{\sy*(0.9999)})
	--({\sx*(1.6000)},{\sy*(1.0000)})
	--({\sx*(1.6100)},{\sy*(0.9997)})
	--({\sx*(1.6200)},{\sy*(0.9991)})
	--({\sx*(1.6300)},{\sy*(0.9981)})
	--({\sx*(1.6400)},{\sy*(0.9968)})
	--({\sx*(1.6500)},{\sy*(0.9952)})
	--({\sx*(1.6600)},{\sy*(0.9931)})
	--({\sx*(1.6700)},{\sy*(0.9908)})
	--({\sx*(1.6800)},{\sy*(0.9881)})
	--({\sx*(1.6900)},{\sy*(0.9850)})
	--({\sx*(1.7000)},{\sy*(0.9816)})
	--({\sx*(1.7100)},{\sy*(0.9779)})
	--({\sx*(1.7200)},{\sy*(0.9739)})
	--({\sx*(1.7300)},{\sy*(0.9695)})
	--({\sx*(1.7400)},{\sy*(0.9648)})
	--({\sx*(1.7500)},{\sy*(0.9597)})
	--({\sx*(1.7600)},{\sy*(0.9544)})
	--({\sx*(1.7700)},{\sy*(0.9487)})
	--({\sx*(1.7800)},{\sy*(0.9427)})
	--({\sx*(1.7900)},{\sy*(0.9364)})
	--({\sx*(1.8000)},{\sy*(0.9298)})
	--({\sx*(1.8100)},{\sy*(0.9229)})
	--({\sx*(1.8200)},{\sy*(0.9157)})
	--({\sx*(1.8300)},{\sy*(0.9082)})
	--({\sx*(1.8400)},{\sy*(0.9004)})
	--({\sx*(1.8500)},{\sy*(0.8923)})
	--({\sx*(1.8600)},{\sy*(0.8839)})
	--({\sx*(1.8700)},{\sy*(0.8753)})
	--({\sx*(1.8800)},{\sy*(0.8663)})
	--({\sx*(1.8900)},{\sy*(0.8572)})
	--({\sx*(1.9000)},{\sy*(0.8477)})
	--({\sx*(1.9100)},{\sy*(0.8380)})
	--({\sx*(1.9200)},{\sy*(0.8280)})
	--({\sx*(1.9300)},{\sy*(0.8178)})
	--({\sx*(1.9400)},{\sy*(0.8073)})
	--({\sx*(1.9500)},{\sy*(0.7966)})
	--({\sx*(1.9600)},{\sy*(0.7856)})
	--({\sx*(1.9700)},{\sy*(0.7744)})
	--({\sx*(1.9800)},{\sy*(0.7630)})
	--({\sx*(1.9900)},{\sy*(0.7514)})
	--({\sx*(2.0000)},{\sy*(0.7396)})
	--({\sx*(2.0100)},{\sy*(0.7275)})
	--({\sx*(2.0200)},{\sy*(0.7153)})
	--({\sx*(2.0300)},{\sy*(0.7028)})
	--({\sx*(2.0400)},{\sy*(0.6902)})
	--({\sx*(2.0500)},{\sy*(0.6773)})
	--({\sx*(2.0600)},{\sy*(0.6643)})
	--({\sx*(2.0700)},{\sy*(0.6511)})
	--({\sx*(2.0800)},{\sy*(0.6378)})
	--({\sx*(2.0900)},{\sy*(0.6243)})
	--({\sx*(2.1000)},{\sy*(0.6106)})
	--({\sx*(2.1100)},{\sy*(0.5967)})
	--({\sx*(2.1200)},{\sy*(0.5828)})
	--({\sx*(2.1300)},{\sy*(0.5686)})
	--({\sx*(2.1400)},{\sy*(0.5544)})
	--({\sx*(2.1500)},{\sy*(0.5400)})
	--({\sx*(2.1600)},{\sy*(0.5255)})
	--({\sx*(2.1700)},{\sy*(0.5109)})
	--({\sx*(2.1800)},{\sy*(0.4961)})
	--({\sx*(2.1900)},{\sy*(0.4813)})
	--({\sx*(2.2000)},{\sy*(0.4663)})
	--({\sx*(2.2100)},{\sy*(0.4513)})
	--({\sx*(2.2200)},{\sy*(0.4362)})
	--({\sx*(2.2300)},{\sy*(0.4209)})
	--({\sx*(2.2400)},{\sy*(0.4057)})
	--({\sx*(2.2500)},{\sy*(0.3903)})
	--({\sx*(2.2600)},{\sy*(0.3749)})
	--({\sx*(2.2700)},{\sy*(0.3594)})
	--({\sx*(2.2800)},{\sy*(0.3439)})
	--({\sx*(2.2900)},{\sy*(0.3283)})
	--({\sx*(2.3000)},{\sy*(0.3127)})
	--({\sx*(2.3100)},{\sy*(0.2970)})
	--({\sx*(2.3200)},{\sy*(0.2813)})
	--({\sx*(2.3300)},{\sy*(0.2656)})
	--({\sx*(2.3400)},{\sy*(0.2499)})
	--({\sx*(2.3500)},{\sy*(0.2341)})
	--({\sx*(2.3600)},{\sy*(0.2184)})
	--({\sx*(2.3700)},{\sy*(0.2026)})
	--({\sx*(2.3800)},{\sy*(0.1868)})
	--({\sx*(2.3900)},{\sy*(0.1711)})
	--({\sx*(2.4000)},{\sy*(0.1554)})
	--({\sx*(2.4100)},{\sy*(0.1397)})
	--({\sx*(2.4200)},{\sy*(0.1240)})
	--({\sx*(2.4300)},{\sy*(0.1083)})
	--({\sx*(2.4400)},{\sy*(0.0927)})
	--({\sx*(2.4500)},{\sy*(0.0771)})
	--({\sx*(2.4600)},{\sy*(0.0616)})
	--({\sx*(2.4700)},{\sy*(0.0461)})
	--({\sx*(2.4800)},{\sy*(0.0307)})
	--({\sx*(2.4900)},{\sy*(0.0153)})
	--({\sx*(2.5000)},{\sy*(0.0000)})
	--({\sx*(2.5100)},{\sy*(-0.0152)})
	--({\sx*(2.5200)},{\sy*(-0.0304)})
	--({\sx*(2.5300)},{\sy*(-0.0455)})
	--({\sx*(2.5400)},{\sy*(-0.0605)})
	--({\sx*(2.5500)},{\sy*(-0.0754)})
	--({\sx*(2.5600)},{\sy*(-0.0902)})
	--({\sx*(2.5700)},{\sy*(-0.1049)})
	--({\sx*(2.5800)},{\sy*(-0.1195)})
	--({\sx*(2.5900)},{\sy*(-0.1340)})
	--({\sx*(2.6000)},{\sy*(-0.1484)})
	--({\sx*(2.6100)},{\sy*(-0.1626)})
	--({\sx*(2.6200)},{\sy*(-0.1768)})
	--({\sx*(2.6300)},{\sy*(-0.1908)})
	--({\sx*(2.6400)},{\sy*(-0.2047)})
	--({\sx*(2.6500)},{\sy*(-0.2185)})
	--({\sx*(2.6600)},{\sy*(-0.2321)})
	--({\sx*(2.6700)},{\sy*(-0.2456)})
	--({\sx*(2.6800)},{\sy*(-0.2590)})
	--({\sx*(2.6900)},{\sy*(-0.2722)})
	--({\sx*(2.7000)},{\sy*(-0.2852)})
	--({\sx*(2.7100)},{\sy*(-0.2981)})
	--({\sx*(2.7200)},{\sy*(-0.3109)})
	--({\sx*(2.7300)},{\sy*(-0.3234)})
	--({\sx*(2.7400)},{\sy*(-0.3358)})
	--({\sx*(2.7500)},{\sy*(-0.3481)})
	--({\sx*(2.7600)},{\sy*(-0.3602)})
	--({\sx*(2.7700)},{\sy*(-0.3721)})
	--({\sx*(2.7800)},{\sy*(-0.3838)})
	--({\sx*(2.7900)},{\sy*(-0.3953)})
	--({\sx*(2.8000)},{\sy*(-0.4067)})
	--({\sx*(2.8100)},{\sy*(-0.4179)})
	--({\sx*(2.8200)},{\sy*(-0.4289)})
	--({\sx*(2.8300)},{\sy*(-0.4397)})
	--({\sx*(2.8400)},{\sy*(-0.4503)})
	--({\sx*(2.8500)},{\sy*(-0.4607)})
	--({\sx*(2.8600)},{\sy*(-0.4709)})
	--({\sx*(2.8700)},{\sy*(-0.4809)})
	--({\sx*(2.8800)},{\sy*(-0.4908)})
	--({\sx*(2.8900)},{\sy*(-0.5004)})
	--({\sx*(2.9000)},{\sy*(-0.5098)})
	--({\sx*(2.9100)},{\sy*(-0.5190)})
	--({\sx*(2.9200)},{\sy*(-0.5279)})
	--({\sx*(2.9300)},{\sy*(-0.5367)})
	--({\sx*(2.9400)},{\sy*(-0.5453)})
	--({\sx*(2.9500)},{\sy*(-0.5536)})
	--({\sx*(2.9600)},{\sy*(-0.5617)})
	--({\sx*(2.9700)},{\sy*(-0.5696)})
	--({\sx*(2.9800)},{\sy*(-0.5773)})
	--({\sx*(2.9900)},{\sy*(-0.5848)})
	--({\sx*(3.0000)},{\sy*(-0.5920)})
	--({\sx*(3.0100)},{\sy*(-0.5991)})
	--({\sx*(3.0200)},{\sy*(-0.6058)})
	--({\sx*(3.0300)},{\sy*(-0.6124)})
	--({\sx*(3.0400)},{\sy*(-0.6187)})
	--({\sx*(3.0500)},{\sy*(-0.6249)})
	--({\sx*(3.0600)},{\sy*(-0.6307)})
	--({\sx*(3.0700)},{\sy*(-0.6364)})
	--({\sx*(3.0800)},{\sy*(-0.6418)})
	--({\sx*(3.0900)},{\sy*(-0.6470)})
	--({\sx*(3.1000)},{\sy*(-0.6520)})
	--({\sx*(3.1100)},{\sy*(-0.6567)})
	--({\sx*(3.1200)},{\sy*(-0.6612)})
	--({\sx*(3.1300)},{\sy*(-0.6654)})
	--({\sx*(3.1400)},{\sy*(-0.6695)})
	--({\sx*(3.1500)},{\sy*(-0.6732)})
	--({\sx*(3.1600)},{\sy*(-0.6768)})
	--({\sx*(3.1700)},{\sy*(-0.6801)})
	--({\sx*(3.1800)},{\sy*(-0.6832)})
	--({\sx*(3.1900)},{\sy*(-0.6861)})
	--({\sx*(3.2000)},{\sy*(-0.6887)})
	--({\sx*(3.2100)},{\sy*(-0.6911)})
	--({\sx*(3.2200)},{\sy*(-0.6933)})
	--({\sx*(3.2300)},{\sy*(-0.6952)})
	--({\sx*(3.2400)},{\sy*(-0.6969)})
	--({\sx*(3.2500)},{\sy*(-0.6983)})
	--({\sx*(3.2600)},{\sy*(-0.6996)})
	--({\sx*(3.2700)},{\sy*(-0.7006)})
	--({\sx*(3.2800)},{\sy*(-0.7013)})
	--({\sx*(3.2900)},{\sy*(-0.7019)})
	--({\sx*(3.3000)},{\sy*(-0.7022)})
	--({\sx*(3.3100)},{\sy*(-0.7023)})
	--({\sx*(3.3200)},{\sy*(-0.7021)})
	--({\sx*(3.3300)},{\sy*(-0.7018)})
	--({\sx*(3.3400)},{\sy*(-0.7012)})
	--({\sx*(3.3500)},{\sy*(-0.7004)})
	--({\sx*(3.3600)},{\sy*(-0.6993)})
	--({\sx*(3.3700)},{\sy*(-0.6981)})
	--({\sx*(3.3800)},{\sy*(-0.6966)})
	--({\sx*(3.3900)},{\sy*(-0.6949)})
	--({\sx*(3.4000)},{\sy*(-0.6930)})
	--({\sx*(3.4100)},{\sy*(-0.6909)})
	--({\sx*(3.4200)},{\sy*(-0.6885)})
	--({\sx*(3.4300)},{\sy*(-0.6860)})
	--({\sx*(3.4400)},{\sy*(-0.6832)})
	--({\sx*(3.4500)},{\sy*(-0.6802)})
	--({\sx*(3.4600)},{\sy*(-0.6771)})
	--({\sx*(3.4700)},{\sy*(-0.6737)})
	--({\sx*(3.4800)},{\sy*(-0.6701)})
	--({\sx*(3.4900)},{\sy*(-0.6663)})
	--({\sx*(3.5000)},{\sy*(-0.6623)})
	--({\sx*(3.5100)},{\sy*(-0.6581)})
	--({\sx*(3.5200)},{\sy*(-0.6538)})
	--({\sx*(3.5300)},{\sy*(-0.6492)})
	--({\sx*(3.5400)},{\sy*(-0.6444)})
	--({\sx*(3.5500)},{\sy*(-0.6395)})
	--({\sx*(3.5600)},{\sy*(-0.6344)})
	--({\sx*(3.5700)},{\sy*(-0.6290)})
	--({\sx*(3.5800)},{\sy*(-0.6235)})
	--({\sx*(3.5900)},{\sy*(-0.6179)})
	--({\sx*(3.6000)},{\sy*(-0.6120)})
	--({\sx*(3.6100)},{\sy*(-0.6060)})
	--({\sx*(3.6200)},{\sy*(-0.5998)})
	--({\sx*(3.6300)},{\sy*(-0.5934)})
	--({\sx*(3.6400)},{\sy*(-0.5869)})
	--({\sx*(3.6500)},{\sy*(-0.5802)})
	--({\sx*(3.6600)},{\sy*(-0.5734)})
	--({\sx*(3.6700)},{\sy*(-0.5664)})
	--({\sx*(3.6800)},{\sy*(-0.5592)})
	--({\sx*(3.6900)},{\sy*(-0.5519)})
	--({\sx*(3.7000)},{\sy*(-0.5445)})
	--({\sx*(3.7100)},{\sy*(-0.5369)})
	--({\sx*(3.7200)},{\sy*(-0.5291)})
	--({\sx*(3.7300)},{\sy*(-0.5212)})
	--({\sx*(3.7400)},{\sy*(-0.5132)})
	--({\sx*(3.7500)},{\sy*(-0.5051)})
	--({\sx*(3.7600)},{\sy*(-0.4968)})
	--({\sx*(3.7700)},{\sy*(-0.4884)})
	--({\sx*(3.7800)},{\sy*(-0.4799)})
	--({\sx*(3.7900)},{\sy*(-0.4713)})
	--({\sx*(3.8000)},{\sy*(-0.4625)})
	--({\sx*(3.8100)},{\sy*(-0.4537)})
	--({\sx*(3.8200)},{\sy*(-0.4447)})
	--({\sx*(3.8300)},{\sy*(-0.4356)})
	--({\sx*(3.8400)},{\sy*(-0.4265)})
	--({\sx*(3.8500)},{\sy*(-0.4172)})
	--({\sx*(3.8600)},{\sy*(-0.4078)})
	--({\sx*(3.8700)},{\sy*(-0.3984)})
	--({\sx*(3.8800)},{\sy*(-0.3889)})
	--({\sx*(3.8900)},{\sy*(-0.3793)})
	--({\sx*(3.9000)},{\sy*(-0.3696)})
	--({\sx*(3.9100)},{\sy*(-0.3598)})
	--({\sx*(3.9200)},{\sy*(-0.3500)})
	--({\sx*(3.9300)},{\sy*(-0.3401)})
	--({\sx*(3.9400)},{\sy*(-0.3302)})
	--({\sx*(3.9500)},{\sy*(-0.3202)})
	--({\sx*(3.9600)},{\sy*(-0.3101)})
	--({\sx*(3.9700)},{\sy*(-0.3000)})
	--({\sx*(3.9800)},{\sy*(-0.2899)})
	--({\sx*(3.9900)},{\sy*(-0.2797)})
	--({\sx*(4.0000)},{\sy*(-0.2695)})
	--({\sx*(4.0100)},{\sy*(-0.2592)})
	--({\sx*(4.0200)},{\sy*(-0.2489)})
	--({\sx*(4.0300)},{\sy*(-0.2386)})
	--({\sx*(4.0400)},{\sy*(-0.2283)})
	--({\sx*(4.0500)},{\sy*(-0.2180)})
	--({\sx*(4.0600)},{\sy*(-0.2077)})
	--({\sx*(4.0700)},{\sy*(-0.1974)})
	--({\sx*(4.0800)},{\sy*(-0.1870)})
	--({\sx*(4.0900)},{\sy*(-0.1767)})
	--({\sx*(4.1000)},{\sy*(-0.1664)})
	--({\sx*(4.1100)},{\sy*(-0.1561)})
	--({\sx*(4.1200)},{\sy*(-0.1459)})
	--({\sx*(4.1300)},{\sy*(-0.1356)})
	--({\sx*(4.1400)},{\sy*(-0.1254)})
	--({\sx*(4.1500)},{\sy*(-0.1152)})
	--({\sx*(4.1600)},{\sy*(-0.1051)})
	--({\sx*(4.1700)},{\sy*(-0.0950)})
	--({\sx*(4.1800)},{\sy*(-0.0850)})
	--({\sx*(4.1900)},{\sy*(-0.0750)})
	--({\sx*(4.2000)},{\sy*(-0.0651)})
	--({\sx*(4.2100)},{\sy*(-0.0553)})
	--({\sx*(4.2200)},{\sy*(-0.0455)})
	--({\sx*(4.2300)},{\sy*(-0.0358)})
	--({\sx*(4.2400)},{\sy*(-0.0262)})
	--({\sx*(4.2500)},{\sy*(-0.0167)})
	--({\sx*(4.2600)},{\sy*(-0.0073)})
	--({\sx*(4.2700)},{\sy*(0.0021)})
	--({\sx*(4.2800)},{\sy*(0.0113)})
	--({\sx*(4.2900)},{\sy*(0.0204)})
	--({\sx*(4.3000)},{\sy*(0.0294)})
	--({\sx*(4.3100)},{\sy*(0.0383)})
	--({\sx*(4.3200)},{\sy*(0.0470)})
	--({\sx*(4.3300)},{\sy*(0.0556)})
	--({\sx*(4.3400)},{\sy*(0.0641)})
	--({\sx*(4.3500)},{\sy*(0.0724)})
	--({\sx*(4.3600)},{\sy*(0.0806)})
	--({\sx*(4.3700)},{\sy*(0.0886)})
	--({\sx*(4.3800)},{\sy*(0.0965)})
	--({\sx*(4.3900)},{\sy*(0.1042)})
	--({\sx*(4.4000)},{\sy*(0.1117)})
	--({\sx*(4.4100)},{\sy*(0.1190)})
	--({\sx*(4.4200)},{\sy*(0.1262)})
	--({\sx*(4.4300)},{\sy*(0.1332)})
	--({\sx*(4.4400)},{\sy*(0.1399)})
	--({\sx*(4.4500)},{\sy*(0.1465)})
	--({\sx*(4.4600)},{\sy*(0.1528)})
	--({\sx*(4.4700)},{\sy*(0.1590)})
	--({\sx*(4.4800)},{\sy*(0.1649)})
	--({\sx*(4.4900)},{\sy*(0.1706)})
	--({\sx*(4.5000)},{\sy*(0.1760)})
	--({\sx*(4.5100)},{\sy*(0.1812)})
	--({\sx*(4.5200)},{\sy*(0.1862)})
	--({\sx*(4.5300)},{\sy*(0.1909)})
	--({\sx*(4.5400)},{\sy*(0.1953)})
	--({\sx*(4.5500)},{\sy*(0.1995)})
	--({\sx*(4.5600)},{\sy*(0.2034)})
	--({\sx*(4.5700)},{\sy*(0.2070)})
	--({\sx*(4.5800)},{\sy*(0.2104)})
	--({\sx*(4.5900)},{\sy*(0.2134)})
	--({\sx*(4.6000)},{\sy*(0.2162)})
	--({\sx*(4.6100)},{\sy*(0.2186)})
	--({\sx*(4.6200)},{\sy*(0.2207)})
	--({\sx*(4.6300)},{\sy*(0.2225)})
	--({\sx*(4.6400)},{\sy*(0.2240)})
	--({\sx*(4.6500)},{\sy*(0.2252)})
	--({\sx*(4.6600)},{\sy*(0.2260)})
	--({\sx*(4.6700)},{\sy*(0.2264)})
	--({\sx*(4.6800)},{\sy*(0.2265)})
	--({\sx*(4.6900)},{\sy*(0.2263)})
	--({\sx*(4.7000)},{\sy*(0.2257)})
	--({\sx*(4.7100)},{\sy*(0.2247)})
	--({\sx*(4.7200)},{\sy*(0.2233)})
	--({\sx*(4.7300)},{\sy*(0.2215)})
	--({\sx*(4.7400)},{\sy*(0.2194)})
	--({\sx*(4.7500)},{\sy*(0.2168)})
	--({\sx*(4.7600)},{\sy*(0.2138)})
	--({\sx*(4.7700)},{\sy*(0.2104)})
	--({\sx*(4.7800)},{\sy*(0.2066)})
	--({\sx*(4.7900)},{\sy*(0.2023)})
	--({\sx*(4.8000)},{\sy*(0.1977)})
	--({\sx*(4.8100)},{\sy*(0.1925)})
	--({\sx*(4.8200)},{\sy*(0.1869)})
	--({\sx*(4.8300)},{\sy*(0.1809)})
	--({\sx*(4.8400)},{\sy*(0.1743)})
	--({\sx*(4.8500)},{\sy*(0.1674)})
	--({\sx*(4.8600)},{\sy*(0.1599)})
	--({\sx*(4.8700)},{\sy*(0.1519)})
	--({\sx*(4.8800)},{\sy*(0.1434)})
	--({\sx*(4.8900)},{\sy*(0.1344)})
	--({\sx*(4.9000)},{\sy*(0.1249)})
	--({\sx*(4.9100)},{\sy*(0.1149)})
	--({\sx*(4.9200)},{\sy*(0.1044)})
	--({\sx*(4.9300)},{\sy*(0.0933)})
	--({\sx*(4.9400)},{\sy*(0.0817)})
	--({\sx*(4.9500)},{\sy*(0.0695)})
	--({\sx*(4.9600)},{\sy*(0.0567)})
	--({\sx*(4.9700)},{\sy*(0.0434)})
	--({\sx*(4.9800)},{\sy*(0.0295)})
	--({\sx*(4.9900)},{\sy*(0.0151)})
	--({\sx*(5.0000)},{\sy*(0.0000)});
}
\def\relfehlerb{
\draw[color=blue,line width=1.4pt,line join=round] ({\sx*(0.000)},{\sy*(0.0000)})
	--({\sx*(0.0100)},{\sy*(-0.0014)})
	--({\sx*(0.0200)},{\sy*(-0.0029)})
	--({\sx*(0.0300)},{\sy*(-0.0042)})
	--({\sx*(0.0400)},{\sy*(-0.0056)})
	--({\sx*(0.0500)},{\sy*(-0.0069)})
	--({\sx*(0.0600)},{\sy*(-0.0082)})
	--({\sx*(0.0700)},{\sy*(-0.0094)})
	--({\sx*(0.0800)},{\sy*(-0.0106)})
	--({\sx*(0.0900)},{\sy*(-0.0118)})
	--({\sx*(0.1000)},{\sy*(-0.0129)})
	--({\sx*(0.1100)},{\sy*(-0.0140)})
	--({\sx*(0.1200)},{\sy*(-0.0151)})
	--({\sx*(0.1300)},{\sy*(-0.0161)})
	--({\sx*(0.1400)},{\sy*(-0.0171)})
	--({\sx*(0.1500)},{\sy*(-0.0180)})
	--({\sx*(0.1600)},{\sy*(-0.0189)})
	--({\sx*(0.1700)},{\sy*(-0.0198)})
	--({\sx*(0.1800)},{\sy*(-0.0206)})
	--({\sx*(0.1900)},{\sy*(-0.0214)})
	--({\sx*(0.2000)},{\sy*(-0.0222)})
	--({\sx*(0.2100)},{\sy*(-0.0229)})
	--({\sx*(0.2200)},{\sy*(-0.0236)})
	--({\sx*(0.2300)},{\sy*(-0.0242)})
	--({\sx*(0.2400)},{\sy*(-0.0248)})
	--({\sx*(0.2500)},{\sy*(-0.0253)})
	--({\sx*(0.2600)},{\sy*(-0.0258)})
	--({\sx*(0.2700)},{\sy*(-0.0263)})
	--({\sx*(0.2800)},{\sy*(-0.0267)})
	--({\sx*(0.2900)},{\sy*(-0.0271)})
	--({\sx*(0.3000)},{\sy*(-0.0274)})
	--({\sx*(0.3100)},{\sy*(-0.0277)})
	--({\sx*(0.3200)},{\sy*(-0.0279)})
	--({\sx*(0.3300)},{\sy*(-0.0281)})
	--({\sx*(0.3400)},{\sy*(-0.0283)})
	--({\sx*(0.3500)},{\sy*(-0.0284)})
	--({\sx*(0.3600)},{\sy*(-0.0285)})
	--({\sx*(0.3700)},{\sy*(-0.0285)})
	--({\sx*(0.3800)},{\sy*(-0.0285)})
	--({\sx*(0.3900)},{\sy*(-0.0284)})
	--({\sx*(0.4000)},{\sy*(-0.0283)})
	--({\sx*(0.4100)},{\sy*(-0.0281)})
	--({\sx*(0.4200)},{\sy*(-0.0280)})
	--({\sx*(0.4300)},{\sy*(-0.0277)})
	--({\sx*(0.4400)},{\sy*(-0.0274)})
	--({\sx*(0.4500)},{\sy*(-0.0271)})
	--({\sx*(0.4600)},{\sy*(-0.0267)})
	--({\sx*(0.4700)},{\sy*(-0.0263)})
	--({\sx*(0.4800)},{\sy*(-0.0259)})
	--({\sx*(0.4900)},{\sy*(-0.0254)})
	--({\sx*(0.5000)},{\sy*(-0.0248)})
	--({\sx*(0.5100)},{\sy*(-0.0242)})
	--({\sx*(0.5200)},{\sy*(-0.0236)})
	--({\sx*(0.5300)},{\sy*(-0.0229)})
	--({\sx*(0.5400)},{\sy*(-0.0222)})
	--({\sx*(0.5500)},{\sy*(-0.0214)})
	--({\sx*(0.5600)},{\sy*(-0.0206)})
	--({\sx*(0.5700)},{\sy*(-0.0197)})
	--({\sx*(0.5800)},{\sy*(-0.0188)})
	--({\sx*(0.5900)},{\sy*(-0.0179)})
	--({\sx*(0.6000)},{\sy*(-0.0169)})
	--({\sx*(0.6100)},{\sy*(-0.0159)})
	--({\sx*(0.6200)},{\sy*(-0.0148)})
	--({\sx*(0.6300)},{\sy*(-0.0137)})
	--({\sx*(0.6400)},{\sy*(-0.0126)})
	--({\sx*(0.6500)},{\sy*(-0.0114)})
	--({\sx*(0.6600)},{\sy*(-0.0101)})
	--({\sx*(0.6700)},{\sy*(-0.0089)})
	--({\sx*(0.6800)},{\sy*(-0.0075)})
	--({\sx*(0.6900)},{\sy*(-0.0062)})
	--({\sx*(0.7000)},{\sy*(-0.0048)})
	--({\sx*(0.7100)},{\sy*(-0.0033)})
	--({\sx*(0.7200)},{\sy*(-0.0019)})
	--({\sx*(0.7300)},{\sy*(-0.0003)})
	--({\sx*(0.7400)},{\sy*(0.0012)})
	--({\sx*(0.7500)},{\sy*(0.0028)})
	--({\sx*(0.7600)},{\sy*(0.0044)})
	--({\sx*(0.7700)},{\sy*(0.0061)})
	--({\sx*(0.7800)},{\sy*(0.0078)})
	--({\sx*(0.7900)},{\sy*(0.0096)})
	--({\sx*(0.8000)},{\sy*(0.0114)})
	--({\sx*(0.8100)},{\sy*(0.0132)})
	--({\sx*(0.8200)},{\sy*(0.0150)})
	--({\sx*(0.8300)},{\sy*(0.0169)})
	--({\sx*(0.8400)},{\sy*(0.0189)})
	--({\sx*(0.8500)},{\sy*(0.0208)})
	--({\sx*(0.8600)},{\sy*(0.0228)})
	--({\sx*(0.8700)},{\sy*(0.0249)})
	--({\sx*(0.8800)},{\sy*(0.0269)})
	--({\sx*(0.8900)},{\sy*(0.0290)})
	--({\sx*(0.9000)},{\sy*(0.0312)})
	--({\sx*(0.9100)},{\sy*(0.0333)})
	--({\sx*(0.9200)},{\sy*(0.0355)})
	--({\sx*(0.9300)},{\sy*(0.0377)})
	--({\sx*(0.9400)},{\sy*(0.0400)})
	--({\sx*(0.9500)},{\sy*(0.0423)})
	--({\sx*(0.9600)},{\sy*(0.0446)})
	--({\sx*(0.9700)},{\sy*(0.0470)})
	--({\sx*(0.9800)},{\sy*(0.0493)})
	--({\sx*(0.9900)},{\sy*(0.0517)})
	--({\sx*(1.0000)},{\sy*(0.0542)})
	--({\sx*(1.0100)},{\sy*(0.0566)})
	--({\sx*(1.0200)},{\sy*(0.0591)})
	--({\sx*(1.0300)},{\sy*(0.0616)})
	--({\sx*(1.0400)},{\sy*(0.0641)})
	--({\sx*(1.0500)},{\sy*(0.0667)})
	--({\sx*(1.0600)},{\sy*(0.0693)})
	--({\sx*(1.0700)},{\sy*(0.0719)})
	--({\sx*(1.0800)},{\sy*(0.0745)})
	--({\sx*(1.0900)},{\sy*(0.0771)})
	--({\sx*(1.1000)},{\sy*(0.0798)})
	--({\sx*(1.1100)},{\sy*(0.0825)})
	--({\sx*(1.1200)},{\sy*(0.0852)})
	--({\sx*(1.1300)},{\sy*(0.0879)})
	--({\sx*(1.1400)},{\sy*(0.0907)})
	--({\sx*(1.1500)},{\sy*(0.0934)})
	--({\sx*(1.1600)},{\sy*(0.0962)})
	--({\sx*(1.1700)},{\sy*(0.0990)})
	--({\sx*(1.1800)},{\sy*(0.1018)})
	--({\sx*(1.1900)},{\sy*(0.1047)})
	--({\sx*(1.2000)},{\sy*(0.1075)})
	--({\sx*(1.2100)},{\sy*(0.1104)})
	--({\sx*(1.2200)},{\sy*(0.1132)})
	--({\sx*(1.2300)},{\sy*(0.1161)})
	--({\sx*(1.2400)},{\sy*(0.1190)})
	--({\sx*(1.2500)},{\sy*(0.1219)})
	--({\sx*(1.2600)},{\sy*(0.1248)})
	--({\sx*(1.2700)},{\sy*(0.1277)})
	--({\sx*(1.2800)},{\sy*(0.1307)})
	--({\sx*(1.2900)},{\sy*(0.1336)})
	--({\sx*(1.3000)},{\sy*(0.1366)})
	--({\sx*(1.3100)},{\sy*(0.1395)})
	--({\sx*(1.3200)},{\sy*(0.1425)})
	--({\sx*(1.3300)},{\sy*(0.1454)})
	--({\sx*(1.3400)},{\sy*(0.1484)})
	--({\sx*(1.3500)},{\sy*(0.1514)})
	--({\sx*(1.3600)},{\sy*(0.1544)})
	--({\sx*(1.3700)},{\sy*(0.1573)})
	--({\sx*(1.3800)},{\sy*(0.1603)})
	--({\sx*(1.3900)},{\sy*(0.1633)})
	--({\sx*(1.4000)},{\sy*(0.1663)})
	--({\sx*(1.4100)},{\sy*(0.1692)})
	--({\sx*(1.4200)},{\sy*(0.1722)})
	--({\sx*(1.4300)},{\sy*(0.1752)})
	--({\sx*(1.4400)},{\sy*(0.1781)})
	--({\sx*(1.4500)},{\sy*(0.1811)})
	--({\sx*(1.4600)},{\sy*(0.1841)})
	--({\sx*(1.4700)},{\sy*(0.1870)})
	--({\sx*(1.4800)},{\sy*(0.1900)})
	--({\sx*(1.4900)},{\sy*(0.1929)})
	--({\sx*(1.5000)},{\sy*(0.1958)})
	--({\sx*(1.5100)},{\sy*(0.1987)})
	--({\sx*(1.5200)},{\sy*(0.2016)})
	--({\sx*(1.5300)},{\sy*(0.2045)})
	--({\sx*(1.5400)},{\sy*(0.2074)})
	--({\sx*(1.5500)},{\sy*(0.2102)})
	--({\sx*(1.5600)},{\sy*(0.2131)})
	--({\sx*(1.5700)},{\sy*(0.2159)})
	--({\sx*(1.5800)},{\sy*(0.2187)})
	--({\sx*(1.5900)},{\sy*(0.2215)})
	--({\sx*(1.6000)},{\sy*(0.2243)})
	--({\sx*(1.6100)},{\sy*(0.2271)})
	--({\sx*(1.6200)},{\sy*(0.2298)})
	--({\sx*(1.6300)},{\sy*(0.2325)})
	--({\sx*(1.6400)},{\sy*(0.2352)})
	--({\sx*(1.6500)},{\sy*(0.2379)})
	--({\sx*(1.6600)},{\sy*(0.2405)})
	--({\sx*(1.6700)},{\sy*(0.2432)})
	--({\sx*(1.6800)},{\sy*(0.2457)})
	--({\sx*(1.6900)},{\sy*(0.2483)})
	--({\sx*(1.7000)},{\sy*(0.2508)})
	--({\sx*(1.7100)},{\sy*(0.2533)})
	--({\sx*(1.7200)},{\sy*(0.2558)})
	--({\sx*(1.7300)},{\sy*(0.2582)})
	--({\sx*(1.7400)},{\sy*(0.2606)})
	--({\sx*(1.7500)},{\sy*(0.2630)})
	--({\sx*(1.7600)},{\sy*(0.2653)})
	--({\sx*(1.7700)},{\sy*(0.2676)})
	--({\sx*(1.7800)},{\sy*(0.2698)})
	--({\sx*(1.7900)},{\sy*(0.2720)})
	--({\sx*(1.8000)},{\sy*(0.2742)})
	--({\sx*(1.8100)},{\sy*(0.2763)})
	--({\sx*(1.8200)},{\sy*(0.2784)})
	--({\sx*(1.8300)},{\sy*(0.2804)})
	--({\sx*(1.8400)},{\sy*(0.2824)})
	--({\sx*(1.8500)},{\sy*(0.2843)})
	--({\sx*(1.8600)},{\sy*(0.2861)})
	--({\sx*(1.8700)},{\sy*(0.2879)})
	--({\sx*(1.8800)},{\sy*(0.2897)})
	--({\sx*(1.8900)},{\sy*(0.2914)})
	--({\sx*(1.9000)},{\sy*(0.2930)})
	--({\sx*(1.9100)},{\sy*(0.2945)})
	--({\sx*(1.9200)},{\sy*(0.2960)})
	--({\sx*(1.9300)},{\sy*(0.2975)})
	--({\sx*(1.9400)},{\sy*(0.2988)})
	--({\sx*(1.9500)},{\sy*(0.3001)})
	--({\sx*(1.9600)},{\sy*(0.3013)})
	--({\sx*(1.9700)},{\sy*(0.3024)})
	--({\sx*(1.9800)},{\sy*(0.3035)})
	--({\sx*(1.9900)},{\sy*(0.3044)})
	--({\sx*(2.0000)},{\sy*(0.3053)})
	--({\sx*(2.0100)},{\sy*(0.3060)})
	--({\sx*(2.0200)},{\sy*(0.3067)})
	--({\sx*(2.0300)},{\sy*(0.3073)})
	--({\sx*(2.0400)},{\sy*(0.3077)})
	--({\sx*(2.0500)},{\sy*(0.3081)})
	--({\sx*(2.0600)},{\sy*(0.3083)})
	--({\sx*(2.0700)},{\sy*(0.3085)})
	--({\sx*(2.0800)},{\sy*(0.3085)})
	--({\sx*(2.0900)},{\sy*(0.3084)})
	--({\sx*(2.1000)},{\sy*(0.3081)})
	--({\sx*(2.1100)},{\sy*(0.3077)})
	--({\sx*(2.1200)},{\sy*(0.3072)})
	--({\sx*(2.1300)},{\sy*(0.3065)})
	--({\sx*(2.1400)},{\sy*(0.3056)})
	--({\sx*(2.1500)},{\sy*(0.3046)})
	--({\sx*(2.1600)},{\sy*(0.3034)})
	--({\sx*(2.1700)},{\sy*(0.3020)})
	--({\sx*(2.1800)},{\sy*(0.3004)})
	--({\sx*(2.1900)},{\sy*(0.2986)})
	--({\sx*(2.2000)},{\sy*(0.2966)})
	--({\sx*(2.2100)},{\sy*(0.2944)})
	--({\sx*(2.2200)},{\sy*(0.2919)})
	--({\sx*(2.2300)},{\sy*(0.2892)})
	--({\sx*(2.2400)},{\sy*(0.2862)})
	--({\sx*(2.2500)},{\sy*(0.2829)})
	--({\sx*(2.2600)},{\sy*(0.2793)})
	--({\sx*(2.2700)},{\sy*(0.2754)})
	--({\sx*(2.2800)},{\sy*(0.2711)})
	--({\sx*(2.2900)},{\sy*(0.2665)})
	--({\sx*(2.3000)},{\sy*(0.2615)})
	--({\sx*(2.3100)},{\sy*(0.2560)})
	--({\sx*(2.3200)},{\sy*(0.2502)})
	--({\sx*(2.3300)},{\sy*(0.2438)})
	--({\sx*(2.3400)},{\sy*(0.2369)})
	--({\sx*(2.3500)},{\sy*(0.2295)})
	--({\sx*(2.3600)},{\sy*(0.2214)})
	--({\sx*(2.3700)},{\sy*(0.2127)})
	--({\sx*(2.3800)},{\sy*(0.2033)})
	--({\sx*(2.3900)},{\sy*(0.1931)})
	--({\sx*(2.4000)},{\sy*(0.1820)})
	--({\sx*(2.4100)},{\sy*(0.1701)})
	--({\sx*(2.4200)},{\sy*(0.1571)})
	--({\sx*(2.4300)},{\sy*(0.1430)})
	--({\sx*(2.4400)},{\sy*(0.1276)})
	--({\sx*(2.4500)},{\sy*(0.1109)})
	--({\sx*(2.4600)},{\sy*(0.0926)})
	--({\sx*(2.4700)},{\sy*(0.0726)})
	--({\sx*(2.4800)},{\sy*(0.0507)})
	--({\sx*(2.4900)},{\sy*(0.0266)})
	--({\sx*(2.5000)},{\sy*(0.0000)})
	--({\sx*(2.5100)},{\sy*(-0.0294)})
	--({\sx*(2.5200)},{\sy*(-0.0621)})
	--({\sx*(2.5300)},{\sy*(-0.0986)})
	--({\sx*(2.5400)},{\sy*(-0.1394)})
	--({\sx*(2.5500)},{\sy*(-0.1855)})
	--({\sx*(2.5600)},{\sy*(-0.2377)})
	--({\sx*(2.5700)},{\sy*(-0.2973)})
	--({\sx*(2.5800)},{\sy*(-0.3660)})
	--({\sx*(2.5900)},{\sy*(-0.4457)})
	--({\sx*(2.6000)},{\sy*(-0.5394)})
	--({\sx*(2.6100)},{\sy*(-0.6508)})
	--({\sx*(2.6200)},{\sy*(-0.7853)})
	--({\sx*(2.6300)},{\sy*(-0.9509)})
	--({\sx*(2.6400)},{\sy*(-1.1592)})
	--({\sx*(2.6500)},{\sy*(-1.4292)})
	--({\sx*(2.6600)},{\sy*(-1.7922)})
	--({\sx*(2.6700)},{\sy*(-2.3060)})
	--({\sx*(2.6800)},{\sy*(-3.0878)})
	--({\sx*(2.6900)},{\sy*(-4.4196)})
	--({\sx*(2.7000)},{\sy*(-7.1924)})
	--({\sx*(2.7100)},{\sy*(-16.4818)})
	--({\sx*(2.7200)},{\sy*(99.5632)})
	--({\sx*(2.7300)},{\sy*(13.4910)})
	--({\sx*(2.7400)},{\sy*(7.5525)})
	--({\sx*(2.7500)},{\sy*(5.3880)})
	--({\sx*(2.7600)},{\sy*(4.2681)})
	--({\sx*(2.7700)},{\sy*(3.5842)})
	--({\sx*(2.7800)},{\sy*(3.1234)})
	--({\sx*(2.7900)},{\sy*(2.7921)})
	--({\sx*(2.8000)},{\sy*(2.5427)})
	--({\sx*(2.8100)},{\sy*(2.3483)})
	--({\sx*(2.8200)},{\sy*(2.1926)})
	--({\sx*(2.8300)},{\sy*(2.0651)})
	--({\sx*(2.8400)},{\sy*(1.9590)})
	--({\sx*(2.8500)},{\sy*(1.8694)})
	--({\sx*(2.8600)},{\sy*(1.7926)})
	--({\sx*(2.8700)},{\sy*(1.7263)})
	--({\sx*(2.8800)},{\sy*(1.6684)})
	--({\sx*(2.8900)},{\sy*(1.6175)})
	--({\sx*(2.9000)},{\sy*(1.5724)})
	--({\sx*(2.9100)},{\sy*(1.5322)})
	--({\sx*(2.9200)},{\sy*(1.4962)})
	--({\sx*(2.9300)},{\sy*(1.4637)})
	--({\sx*(2.9400)},{\sy*(1.4343)})
	--({\sx*(2.9500)},{\sy*(1.4076)})
	--({\sx*(2.9600)},{\sy*(1.3833)})
	--({\sx*(2.9700)},{\sy*(1.3610)})
	--({\sx*(2.9800)},{\sy*(1.3406)})
	--({\sx*(2.9900)},{\sy*(1.3218)})
	--({\sx*(3.0000)},{\sy*(1.3044)})
	--({\sx*(3.0100)},{\sy*(1.2883)})
	--({\sx*(3.0200)},{\sy*(1.2734)})
	--({\sx*(3.0300)},{\sy*(1.2596)})
	--({\sx*(3.0400)},{\sy*(1.2467)})
	--({\sx*(3.0500)},{\sy*(1.2347)})
	--({\sx*(3.0600)},{\sy*(1.2234)})
	--({\sx*(3.0700)},{\sy*(1.2129)})
	--({\sx*(3.0800)},{\sy*(1.2031)})
	--({\sx*(3.0900)},{\sy*(1.1938)})
	--({\sx*(3.1000)},{\sy*(1.1851)})
	--({\sx*(3.1100)},{\sy*(1.1769)})
	--({\sx*(3.1200)},{\sy*(1.1692)})
	--({\sx*(3.1300)},{\sy*(1.1620)})
	--({\sx*(3.1400)},{\sy*(1.1551)})
	--({\sx*(3.1500)},{\sy*(1.1486)})
	--({\sx*(3.1600)},{\sy*(1.1425)})
	--({\sx*(3.1700)},{\sy*(1.1367)})
	--({\sx*(3.1800)},{\sy*(1.1312)})
	--({\sx*(3.1900)},{\sy*(1.1259)})
	--({\sx*(3.2000)},{\sy*(1.1210)})
	--({\sx*(3.2100)},{\sy*(1.1163)})
	--({\sx*(3.2200)},{\sy*(1.1118)})
	--({\sx*(3.2300)},{\sy*(1.1075)})
	--({\sx*(3.2400)},{\sy*(1.1035)})
	--({\sx*(3.2500)},{\sy*(1.0996)})
	--({\sx*(3.2600)},{\sy*(1.0959)})
	--({\sx*(3.2700)},{\sy*(1.0924)})
	--({\sx*(3.2800)},{\sy*(1.0891)})
	--({\sx*(3.2900)},{\sy*(1.0859)})
	--({\sx*(3.3000)},{\sy*(1.0828)})
	--({\sx*(3.3100)},{\sy*(1.0799)})
	--({\sx*(3.3200)},{\sy*(1.0771)})
	--({\sx*(3.3300)},{\sy*(1.0744)})
	--({\sx*(3.3400)},{\sy*(1.0719)})
	--({\sx*(3.3500)},{\sy*(1.0694)})
	--({\sx*(3.3600)},{\sy*(1.0671)})
	--({\sx*(3.3700)},{\sy*(1.0649)})
	--({\sx*(3.3800)},{\sy*(1.0627)})
	--({\sx*(3.3900)},{\sy*(1.0607)})
	--({\sx*(3.4000)},{\sy*(1.0587)})
	--({\sx*(3.4100)},{\sy*(1.0568)})
	--({\sx*(3.4200)},{\sy*(1.0550)})
	--({\sx*(3.4300)},{\sy*(1.0532)})
	--({\sx*(3.4400)},{\sy*(1.0516)})
	--({\sx*(3.4500)},{\sy*(1.0500)})
	--({\sx*(3.4600)},{\sy*(1.0484)})
	--({\sx*(3.4700)},{\sy*(1.0469)})
	--({\sx*(3.4800)},{\sy*(1.0455)})
	--({\sx*(3.4900)},{\sy*(1.0441)})
	--({\sx*(3.5000)},{\sy*(1.0428)})
	--({\sx*(3.5100)},{\sy*(1.0416)})
	--({\sx*(3.5200)},{\sy*(1.0404)})
	--({\sx*(3.5300)},{\sy*(1.0392)})
	--({\sx*(3.5400)},{\sy*(1.0381)})
	--({\sx*(3.5500)},{\sy*(1.0370)})
	--({\sx*(3.5600)},{\sy*(1.0359)})
	--({\sx*(3.5700)},{\sy*(1.0349)})
	--({\sx*(3.5800)},{\sy*(1.0340)})
	--({\sx*(3.5900)},{\sy*(1.0331)})
	--({\sx*(3.6000)},{\sy*(1.0322)})
	--({\sx*(3.6100)},{\sy*(1.0313)})
	--({\sx*(3.6200)},{\sy*(1.0305)})
	--({\sx*(3.6300)},{\sy*(1.0297)})
	--({\sx*(3.6400)},{\sy*(1.0289)})
	--({\sx*(3.6500)},{\sy*(1.0282)})
	--({\sx*(3.6600)},{\sy*(1.0275)})
	--({\sx*(3.6700)},{\sy*(1.0268)})
	--({\sx*(3.6800)},{\sy*(1.0262)})
	--({\sx*(3.6900)},{\sy*(1.0255)})
	--({\sx*(3.7000)},{\sy*(1.0249)})
	--({\sx*(3.7100)},{\sy*(1.0243)})
	--({\sx*(3.7200)},{\sy*(1.0238)})
	--({\sx*(3.7300)},{\sy*(1.0233)})
	--({\sx*(3.7400)},{\sy*(1.0227)})
	--({\sx*(3.7500)},{\sy*(1.0222)})
	--({\sx*(3.7600)},{\sy*(1.0218)})
	--({\sx*(3.7700)},{\sy*(1.0213)})
	--({\sx*(3.7800)},{\sy*(1.0209)})
	--({\sx*(3.7900)},{\sy*(1.0205)})
	--({\sx*(3.8000)},{\sy*(1.0201)})
	--({\sx*(3.8100)},{\sy*(1.0197)})
	--({\sx*(3.8200)},{\sy*(1.0193)})
	--({\sx*(3.8300)},{\sy*(1.0190)})
	--({\sx*(3.8400)},{\sy*(1.0187)})
	--({\sx*(3.8500)},{\sy*(1.0183)})
	--({\sx*(3.8600)},{\sy*(1.0181)})
	--({\sx*(3.8700)},{\sy*(1.0178)})
	--({\sx*(3.8800)},{\sy*(1.0175)})
	--({\sx*(3.8900)},{\sy*(1.0173)})
	--({\sx*(3.9000)},{\sy*(1.0170)})
	--({\sx*(3.9100)},{\sy*(1.0168)})
	--({\sx*(3.9200)},{\sy*(1.0166)})
	--({\sx*(3.9300)},{\sy*(1.0165)})
	--({\sx*(3.9400)},{\sy*(1.0163)})
	--({\sx*(3.9500)},{\sy*(1.0162)})
	--({\sx*(3.9600)},{\sy*(1.0160)})
	--({\sx*(3.9700)},{\sy*(1.0159)})
	--({\sx*(3.9800)},{\sy*(1.0158)})
	--({\sx*(3.9900)},{\sy*(1.0158)})
	--({\sx*(4.0000)},{\sy*(1.0157)})
	--({\sx*(4.0100)},{\sy*(1.0157)})
	--({\sx*(4.0200)},{\sy*(1.0157)})
	--({\sx*(4.0300)},{\sy*(1.0157)})
	--({\sx*(4.0400)},{\sy*(1.0158)})
	--({\sx*(4.0500)},{\sy*(1.0159)})
	--({\sx*(4.0600)},{\sy*(1.0160)})
	--({\sx*(4.0700)},{\sy*(1.0162)})
	--({\sx*(4.0800)},{\sy*(1.0164)})
	--({\sx*(4.0900)},{\sy*(1.0167)})
	--({\sx*(4.1000)},{\sy*(1.0170)})
	--({\sx*(4.1100)},{\sy*(1.0174)})
	--({\sx*(4.1200)},{\sy*(1.0179)})
	--({\sx*(4.1300)},{\sy*(1.0185)})
	--({\sx*(4.1400)},{\sy*(1.0192)})
	--({\sx*(4.1500)},{\sy*(1.0200)})
	--({\sx*(4.1600)},{\sy*(1.0211)})
	--({\sx*(4.1700)},{\sy*(1.0224)})
	--({\sx*(4.1800)},{\sy*(1.0241)})
	--({\sx*(4.1900)},{\sy*(1.0262)})
	--({\sx*(4.2000)},{\sy*(1.0290)})
	--({\sx*(4.2100)},{\sy*(1.0329)})
	--({\sx*(4.2200)},{\sy*(1.0386)})
	--({\sx*(4.2300)},{\sy*(1.0474)})
	--({\sx*(4.2400)},{\sy*(1.0630)})
	--({\sx*(4.2500)},{\sy*(1.0979)})
	--({\sx*(4.2600)},{\sy*(1.2447)})
	--({\sx*(4.2700)},{\sy*(0.6029)})
	--({\sx*(4.2800)},{\sy*(0.8961)})
	--({\sx*(4.2900)},{\sy*(0.9421)})
	--({\sx*(4.3000)},{\sy*(0.9607)})
	--({\sx*(4.3100)},{\sy*(0.9708)})
	--({\sx*(4.3200)},{\sy*(0.9771)})
	--({\sx*(4.3300)},{\sy*(0.9814)})
	--({\sx*(4.3400)},{\sy*(0.9845)})
	--({\sx*(4.3500)},{\sy*(0.9868)})
	--({\sx*(4.3600)},{\sy*(0.9886)})
	--({\sx*(4.3700)},{\sy*(0.9901)})
	--({\sx*(4.3800)},{\sy*(0.9913)})
	--({\sx*(4.3900)},{\sy*(0.9923)})
	--({\sx*(4.4000)},{\sy*(0.9931)})
	--({\sx*(4.4100)},{\sy*(0.9938)})
	--({\sx*(4.4200)},{\sy*(0.9944)})
	--({\sx*(4.4300)},{\sy*(0.9949)})
	--({\sx*(4.4400)},{\sy*(0.9954)})
	--({\sx*(4.4500)},{\sy*(0.9958)})
	--({\sx*(4.4600)},{\sy*(0.9961)})
	--({\sx*(4.4700)},{\sy*(0.9964)})
	--({\sx*(4.4800)},{\sy*(0.9967)})
	--({\sx*(4.4900)},{\sy*(0.9970)})
	--({\sx*(4.5000)},{\sy*(0.9972)})
	--({\sx*(4.5100)},{\sy*(0.9974)})
	--({\sx*(4.5200)},{\sy*(0.9976)})
	--({\sx*(4.5300)},{\sy*(0.9977)})
	--({\sx*(4.5400)},{\sy*(0.9979)})
	--({\sx*(4.5500)},{\sy*(0.9980)})
	--({\sx*(4.5600)},{\sy*(0.9981)})
	--({\sx*(4.5700)},{\sy*(0.9983)})
	--({\sx*(4.5800)},{\sy*(0.9984)})
	--({\sx*(4.5900)},{\sy*(0.9985)})
	--({\sx*(4.6000)},{\sy*(0.9985)})
	--({\sx*(4.6100)},{\sy*(0.9986)})
	--({\sx*(4.6200)},{\sy*(0.9987)})
	--({\sx*(4.6300)},{\sy*(0.9988)})
	--({\sx*(4.6400)},{\sy*(0.9988)})
	--({\sx*(4.6500)},{\sy*(0.9989)})
	--({\sx*(4.6600)},{\sy*(0.9989)})
	--({\sx*(4.6700)},{\sy*(0.9990)})
	--({\sx*(4.6800)},{\sy*(0.9990)})
	--({\sx*(4.6900)},{\sy*(0.9991)})
	--({\sx*(4.7000)},{\sy*(0.9991)})
	--({\sx*(4.7100)},{\sy*(0.9992)})
	--({\sx*(4.7200)},{\sy*(0.9992)})
	--({\sx*(4.7300)},{\sy*(0.9992)})
	--({\sx*(4.7400)},{\sy*(0.9993)})
	--({\sx*(4.7500)},{\sy*(0.9993)})
	--({\sx*(4.7600)},{\sy*(0.9993)})
	--({\sx*(4.7700)},{\sy*(0.9993)})
	--({\sx*(4.7800)},{\sy*(0.9993)})
	--({\sx*(4.7900)},{\sy*(0.9994)})
	--({\sx*(4.8000)},{\sy*(0.9994)})
	--({\sx*(4.8100)},{\sy*(0.9994)})
	--({\sx*(4.8200)},{\sy*(0.9994)})
	--({\sx*(4.8300)},{\sy*(0.9994)})
	--({\sx*(4.8400)},{\sy*(0.9994)})
	--({\sx*(4.8500)},{\sy*(0.9994)})
	--({\sx*(4.8600)},{\sy*(0.9994)})
	--({\sx*(4.8700)},{\sy*(0.9994)})
	--({\sx*(4.8800)},{\sy*(0.9994)})
	--({\sx*(4.8900)},{\sy*(0.9994)})
	--({\sx*(4.9000)},{\sy*(0.9994)})
	--({\sx*(4.9100)},{\sy*(0.9994)})
	--({\sx*(4.9200)},{\sy*(0.9993)})
	--({\sx*(4.9300)},{\sy*(0.9993)})
	--({\sx*(4.9400)},{\sy*(0.9992)})
	--({\sx*(4.9500)},{\sy*(0.9991)})
	--({\sx*(4.9600)},{\sy*(0.9990)})
	--({\sx*(4.9700)},{\sy*(0.9988)})
	--({\sx*(4.9800)},{\sy*(0.9983)})
	--({\sx*(4.9900)},{\sy*(0.9968)})
	--({\sx*(5.0000)},{\sy*(0.0000)});
}
\def\xwertec{
\fill[color=red] (0.0000,0) circle[radius={0.07/\skala}];
\fill[color=white] (0.0000,0) circle[radius={0.05/\skala}];
\fill[color=red] (0.3349,0) circle[radius={0.07/\skala}];
\fill[color=white] (0.3349,0) circle[radius={0.05/\skala}];
\fill[color=red] (1.2500,0) circle[radius={0.07/\skala}];
\fill[color=white] (1.2500,0) circle[radius={0.05/\skala}];
\fill[color=red] (2.5000,0) circle[radius={0.07/\skala}];
\fill[color=white] (2.5000,0) circle[radius={0.05/\skala}];
\fill[color=red] (3.7500,0) circle[radius={0.07/\skala}];
\fill[color=white] (3.7500,0) circle[radius={0.05/\skala}];
\fill[color=red] (4.6651,0) circle[radius={0.07/\skala}];
\fill[color=white] (4.6651,0) circle[radius={0.05/\skala}];
\fill[color=red] (5.0000,0) circle[radius={0.07/\skala}];
\fill[color=white] (5.0000,0) circle[radius={0.05/\skala}];
}
\def\punktec{6}
\def\maxfehlerc{4.664\cdot 10^{-3}}
\def\fehlerc{
\draw[color=red,line width=1.4pt,line join=round] ({\sx*(0.000)},{\sy*(0.0000)})
	--({\sx*(0.0100)},{\sy*(0.0660)})
	--({\sx*(0.0200)},{\sy*(0.1257)})
	--({\sx*(0.0300)},{\sy*(0.1792)})
	--({\sx*(0.0400)},{\sy*(0.2269)})
	--({\sx*(0.0500)},{\sy*(0.2690)})
	--({\sx*(0.0600)},{\sy*(0.3056)})
	--({\sx*(0.0700)},{\sy*(0.3371)})
	--({\sx*(0.0800)},{\sy*(0.3638)})
	--({\sx*(0.0900)},{\sy*(0.3857)})
	--({\sx*(0.1000)},{\sy*(0.4032)})
	--({\sx*(0.1100)},{\sy*(0.4164)})
	--({\sx*(0.1200)},{\sy*(0.4257)})
	--({\sx*(0.1300)},{\sy*(0.4311)})
	--({\sx*(0.1400)},{\sy*(0.4329)})
	--({\sx*(0.1500)},{\sy*(0.4313)})
	--({\sx*(0.1600)},{\sy*(0.4265)})
	--({\sx*(0.1700)},{\sy*(0.4186)})
	--({\sx*(0.1800)},{\sy*(0.4079)})
	--({\sx*(0.1900)},{\sy*(0.3946)})
	--({\sx*(0.2000)},{\sy*(0.3787)})
	--({\sx*(0.2100)},{\sy*(0.3605)})
	--({\sx*(0.2200)},{\sy*(0.3402)})
	--({\sx*(0.2300)},{\sy*(0.3179)})
	--({\sx*(0.2400)},{\sy*(0.2937)})
	--({\sx*(0.2500)},{\sy*(0.2679)})
	--({\sx*(0.2600)},{\sy*(0.2405)})
	--({\sx*(0.2700)},{\sy*(0.2117)})
	--({\sx*(0.2800)},{\sy*(0.1817)})
	--({\sx*(0.2900)},{\sy*(0.1505)})
	--({\sx*(0.3000)},{\sy*(0.1184)})
	--({\sx*(0.3100)},{\sy*(0.0854)})
	--({\sx*(0.3200)},{\sy*(0.0516)})
	--({\sx*(0.3300)},{\sy*(0.0172)})
	--({\sx*(0.3400)},{\sy*(-0.0177)})
	--({\sx*(0.3500)},{\sy*(-0.0531)})
	--({\sx*(0.3600)},{\sy*(-0.0887)})
	--({\sx*(0.3700)},{\sy*(-0.1246)})
	--({\sx*(0.3800)},{\sy*(-0.1605)})
	--({\sx*(0.3900)},{\sy*(-0.1965)})
	--({\sx*(0.4000)},{\sy*(-0.2324)})
	--({\sx*(0.4100)},{\sy*(-0.2681)})
	--({\sx*(0.4200)},{\sy*(-0.3037)})
	--({\sx*(0.4300)},{\sy*(-0.3388)})
	--({\sx*(0.4400)},{\sy*(-0.3736)})
	--({\sx*(0.4500)},{\sy*(-0.4080)})
	--({\sx*(0.4600)},{\sy*(-0.4418)})
	--({\sx*(0.4700)},{\sy*(-0.4750)})
	--({\sx*(0.4800)},{\sy*(-0.5075)})
	--({\sx*(0.4900)},{\sy*(-0.5393)})
	--({\sx*(0.5000)},{\sy*(-0.5704)})
	--({\sx*(0.5100)},{\sy*(-0.6007)})
	--({\sx*(0.5200)},{\sy*(-0.6301)})
	--({\sx*(0.5300)},{\sy*(-0.6585)})
	--({\sx*(0.5400)},{\sy*(-0.6861)})
	--({\sx*(0.5500)},{\sy*(-0.7126)})
	--({\sx*(0.5600)},{\sy*(-0.7381)})
	--({\sx*(0.5700)},{\sy*(-0.7626)})
	--({\sx*(0.5800)},{\sy*(-0.7859)})
	--({\sx*(0.5900)},{\sy*(-0.8081)})
	--({\sx*(0.6000)},{\sy*(-0.8292)})
	--({\sx*(0.6100)},{\sy*(-0.8492)})
	--({\sx*(0.6200)},{\sy*(-0.8679)})
	--({\sx*(0.6300)},{\sy*(-0.8855)})
	--({\sx*(0.6400)},{\sy*(-0.9018)})
	--({\sx*(0.6500)},{\sy*(-0.9169)})
	--({\sx*(0.6600)},{\sy*(-0.9307)})
	--({\sx*(0.6700)},{\sy*(-0.9433)})
	--({\sx*(0.6800)},{\sy*(-0.9547)})
	--({\sx*(0.6900)},{\sy*(-0.9647)})
	--({\sx*(0.7000)},{\sy*(-0.9735)})
	--({\sx*(0.7100)},{\sy*(-0.9811)})
	--({\sx*(0.7200)},{\sy*(-0.9874)})
	--({\sx*(0.7300)},{\sy*(-0.9924)})
	--({\sx*(0.7400)},{\sy*(-0.9962)})
	--({\sx*(0.7500)},{\sy*(-0.9987)})
	--({\sx*(0.7600)},{\sy*(-1.0000)})
	--({\sx*(0.7700)},{\sy*(-1.0000)})
	--({\sx*(0.7800)},{\sy*(-0.9988)})
	--({\sx*(0.7900)},{\sy*(-0.9964)})
	--({\sx*(0.8000)},{\sy*(-0.9929)})
	--({\sx*(0.8100)},{\sy*(-0.9881)})
	--({\sx*(0.8200)},{\sy*(-0.9822)})
	--({\sx*(0.8300)},{\sy*(-0.9752)})
	--({\sx*(0.8400)},{\sy*(-0.9670)})
	--({\sx*(0.8500)},{\sy*(-0.9577)})
	--({\sx*(0.8600)},{\sy*(-0.9474)})
	--({\sx*(0.8700)},{\sy*(-0.9360)})
	--({\sx*(0.8800)},{\sy*(-0.9235)})
	--({\sx*(0.8900)},{\sy*(-0.9101)})
	--({\sx*(0.9000)},{\sy*(-0.8956)})
	--({\sx*(0.9100)},{\sy*(-0.8802)})
	--({\sx*(0.9200)},{\sy*(-0.8639)})
	--({\sx*(0.9300)},{\sy*(-0.8466)})
	--({\sx*(0.9400)},{\sy*(-0.8285)})
	--({\sx*(0.9500)},{\sy*(-0.8096)})
	--({\sx*(0.9600)},{\sy*(-0.7898)})
	--({\sx*(0.9700)},{\sy*(-0.7692)})
	--({\sx*(0.9800)},{\sy*(-0.7479)})
	--({\sx*(0.9900)},{\sy*(-0.7258)})
	--({\sx*(1.0000)},{\sy*(-0.7030)})
	--({\sx*(1.0100)},{\sy*(-0.6796)})
	--({\sx*(1.0200)},{\sy*(-0.6555)})
	--({\sx*(1.0300)},{\sy*(-0.6309)})
	--({\sx*(1.0400)},{\sy*(-0.6056)})
	--({\sx*(1.0500)},{\sy*(-0.5798)})
	--({\sx*(1.0600)},{\sy*(-0.5535)})
	--({\sx*(1.0700)},{\sy*(-0.5268)})
	--({\sx*(1.0800)},{\sy*(-0.4996)})
	--({\sx*(1.0900)},{\sy*(-0.4720)})
	--({\sx*(1.1000)},{\sy*(-0.4440)})
	--({\sx*(1.1100)},{\sy*(-0.4156)})
	--({\sx*(1.1200)},{\sy*(-0.3869)})
	--({\sx*(1.1300)},{\sy*(-0.3580)})
	--({\sx*(1.1400)},{\sy*(-0.3288)})
	--({\sx*(1.1500)},{\sy*(-0.2994)})
	--({\sx*(1.1600)},{\sy*(-0.2698)})
	--({\sx*(1.1700)},{\sy*(-0.2400)})
	--({\sx*(1.1800)},{\sy*(-0.2102)})
	--({\sx*(1.1900)},{\sy*(-0.1802)})
	--({\sx*(1.2000)},{\sy*(-0.1501)})
	--({\sx*(1.2100)},{\sy*(-0.1201)})
	--({\sx*(1.2200)},{\sy*(-0.0900)})
	--({\sx*(1.2300)},{\sy*(-0.0599)})
	--({\sx*(1.2400)},{\sy*(-0.0299)})
	--({\sx*(1.2500)},{\sy*(0.0000)})
	--({\sx*(1.2600)},{\sy*(0.0298)})
	--({\sx*(1.2700)},{\sy*(0.0595)})
	--({\sx*(1.2800)},{\sy*(0.0890)})
	--({\sx*(1.2900)},{\sy*(0.1184)})
	--({\sx*(1.3000)},{\sy*(0.1475)})
	--({\sx*(1.3100)},{\sy*(0.1763)})
	--({\sx*(1.3200)},{\sy*(0.2050)})
	--({\sx*(1.3300)},{\sy*(0.2333)})
	--({\sx*(1.3400)},{\sy*(0.2613)})
	--({\sx*(1.3500)},{\sy*(0.2890)})
	--({\sx*(1.3600)},{\sy*(0.3163)})
	--({\sx*(1.3700)},{\sy*(0.3433)})
	--({\sx*(1.3800)},{\sy*(0.3698)})
	--({\sx*(1.3900)},{\sy*(0.3960)})
	--({\sx*(1.4000)},{\sy*(0.4217)})
	--({\sx*(1.4100)},{\sy*(0.4469)})
	--({\sx*(1.4200)},{\sy*(0.4717)})
	--({\sx*(1.4300)},{\sy*(0.4960)})
	--({\sx*(1.4400)},{\sy*(0.5197)})
	--({\sx*(1.4500)},{\sy*(0.5430)})
	--({\sx*(1.4600)},{\sy*(0.5657)})
	--({\sx*(1.4700)},{\sy*(0.5878)})
	--({\sx*(1.4800)},{\sy*(0.6094)})
	--({\sx*(1.4900)},{\sy*(0.6304)})
	--({\sx*(1.5000)},{\sy*(0.6508)})
	--({\sx*(1.5100)},{\sy*(0.6706)})
	--({\sx*(1.5200)},{\sy*(0.6898)})
	--({\sx*(1.5300)},{\sy*(0.7084)})
	--({\sx*(1.5400)},{\sy*(0.7263)})
	--({\sx*(1.5500)},{\sy*(0.7436)})
	--({\sx*(1.5600)},{\sy*(0.7602)})
	--({\sx*(1.5700)},{\sy*(0.7762)})
	--({\sx*(1.5800)},{\sy*(0.7915)})
	--({\sx*(1.5900)},{\sy*(0.8061)})
	--({\sx*(1.6000)},{\sy*(0.8201)})
	--({\sx*(1.6100)},{\sy*(0.8333)})
	--({\sx*(1.6200)},{\sy*(0.8459)})
	--({\sx*(1.6300)},{\sy*(0.8578)})
	--({\sx*(1.6400)},{\sy*(0.8689)})
	--({\sx*(1.6500)},{\sy*(0.8794)})
	--({\sx*(1.6600)},{\sy*(0.8892)})
	--({\sx*(1.6700)},{\sy*(0.8983)})
	--({\sx*(1.6800)},{\sy*(0.9067)})
	--({\sx*(1.6900)},{\sy*(0.9143)})
	--({\sx*(1.7000)},{\sy*(0.9213)})
	--({\sx*(1.7100)},{\sy*(0.9276)})
	--({\sx*(1.7200)},{\sy*(0.9332)})
	--({\sx*(1.7300)},{\sy*(0.9381)})
	--({\sx*(1.7400)},{\sy*(0.9423)})
	--({\sx*(1.7500)},{\sy*(0.9458)})
	--({\sx*(1.7600)},{\sy*(0.9486)})
	--({\sx*(1.7700)},{\sy*(0.9508)})
	--({\sx*(1.7800)},{\sy*(0.9522)})
	--({\sx*(1.7900)},{\sy*(0.9530)})
	--({\sx*(1.8000)},{\sy*(0.9532)})
	--({\sx*(1.8100)},{\sy*(0.9527)})
	--({\sx*(1.8200)},{\sy*(0.9515)})
	--({\sx*(1.8300)},{\sy*(0.9497)})
	--({\sx*(1.8400)},{\sy*(0.9473)})
	--({\sx*(1.8500)},{\sy*(0.9443)})
	--({\sx*(1.8600)},{\sy*(0.9406)})
	--({\sx*(1.8700)},{\sy*(0.9363)})
	--({\sx*(1.8800)},{\sy*(0.9315)})
	--({\sx*(1.8900)},{\sy*(0.9260)})
	--({\sx*(1.9000)},{\sy*(0.9200)})
	--({\sx*(1.9100)},{\sy*(0.9134)})
	--({\sx*(1.9200)},{\sy*(0.9063)})
	--({\sx*(1.9300)},{\sy*(0.8986)})
	--({\sx*(1.9400)},{\sy*(0.8903)})
	--({\sx*(1.9500)},{\sy*(0.8816)})
	--({\sx*(1.9600)},{\sy*(0.8724)})
	--({\sx*(1.9700)},{\sy*(0.8626)})
	--({\sx*(1.9800)},{\sy*(0.8524)})
	--({\sx*(1.9900)},{\sy*(0.8417)})
	--({\sx*(2.0000)},{\sy*(0.8306)})
	--({\sx*(2.0100)},{\sy*(0.8190)})
	--({\sx*(2.0200)},{\sy*(0.8069)})
	--({\sx*(2.0300)},{\sy*(0.7945)})
	--({\sx*(2.0400)},{\sy*(0.7817)})
	--({\sx*(2.0500)},{\sy*(0.7684)})
	--({\sx*(2.0600)},{\sy*(0.7548)})
	--({\sx*(2.0700)},{\sy*(0.7409)})
	--({\sx*(2.0800)},{\sy*(0.7266)})
	--({\sx*(2.0900)},{\sy*(0.7119)})
	--({\sx*(2.1000)},{\sy*(0.6970)})
	--({\sx*(2.1100)},{\sy*(0.6817)})
	--({\sx*(2.1200)},{\sy*(0.6662)})
	--({\sx*(2.1300)},{\sy*(0.6503)})
	--({\sx*(2.1400)},{\sy*(0.6342)})
	--({\sx*(2.1500)},{\sy*(0.6179)})
	--({\sx*(2.1600)},{\sy*(0.6013)})
	--({\sx*(2.1700)},{\sy*(0.5846)})
	--({\sx*(2.1800)},{\sy*(0.5676)})
	--({\sx*(2.1900)},{\sy*(0.5504)})
	--({\sx*(2.2000)},{\sy*(0.5331)})
	--({\sx*(2.2100)},{\sy*(0.5156)})
	--({\sx*(2.2200)},{\sy*(0.4980)})
	--({\sx*(2.2300)},{\sy*(0.4802)})
	--({\sx*(2.2400)},{\sy*(0.4623)})
	--({\sx*(2.2500)},{\sy*(0.4443)})
	--({\sx*(2.2600)},{\sy*(0.4262)})
	--({\sx*(2.2700)},{\sy*(0.4081)})
	--({\sx*(2.2800)},{\sy*(0.3899)})
	--({\sx*(2.2900)},{\sy*(0.3716)})
	--({\sx*(2.3000)},{\sy*(0.3533)})
	--({\sx*(2.3100)},{\sy*(0.3350)})
	--({\sx*(2.3200)},{\sy*(0.3167)})
	--({\sx*(2.3300)},{\sy*(0.2984)})
	--({\sx*(2.3400)},{\sy*(0.2801)})
	--({\sx*(2.3500)},{\sy*(0.2619)})
	--({\sx*(2.3600)},{\sy*(0.2437)})
	--({\sx*(2.3700)},{\sy*(0.2255)})
	--({\sx*(2.3800)},{\sy*(0.2075)})
	--({\sx*(2.3900)},{\sy*(0.1895)})
	--({\sx*(2.4000)},{\sy*(0.1716)})
	--({\sx*(2.4100)},{\sy*(0.1538)})
	--({\sx*(2.4200)},{\sy*(0.1361)})
	--({\sx*(2.4300)},{\sy*(0.1185)})
	--({\sx*(2.4400)},{\sy*(0.1011)})
	--({\sx*(2.4500)},{\sy*(0.0838)})
	--({\sx*(2.4600)},{\sy*(0.0667)})
	--({\sx*(2.4700)},{\sy*(0.0497)})
	--({\sx*(2.4800)},{\sy*(0.0330)})
	--({\sx*(2.4900)},{\sy*(0.0164)})
	--({\sx*(2.5000)},{\sy*(0.0000)})
	--({\sx*(2.5100)},{\sy*(-0.0162)})
	--({\sx*(2.5200)},{\sy*(-0.0321)})
	--({\sx*(2.5300)},{\sy*(-0.0479)})
	--({\sx*(2.5400)},{\sy*(-0.0634)})
	--({\sx*(2.5500)},{\sy*(-0.0787)})
	--({\sx*(2.5600)},{\sy*(-0.0937)})
	--({\sx*(2.5700)},{\sy*(-0.1085)})
	--({\sx*(2.5800)},{\sy*(-0.1230)})
	--({\sx*(2.5900)},{\sy*(-0.1372)})
	--({\sx*(2.6000)},{\sy*(-0.1512)})
	--({\sx*(2.6100)},{\sy*(-0.1649)})
	--({\sx*(2.6200)},{\sy*(-0.1783)})
	--({\sx*(2.6300)},{\sy*(-0.1914)})
	--({\sx*(2.6400)},{\sy*(-0.2042)})
	--({\sx*(2.6500)},{\sy*(-0.2167)})
	--({\sx*(2.6600)},{\sy*(-0.2289)})
	--({\sx*(2.6700)},{\sy*(-0.2408)})
	--({\sx*(2.6800)},{\sy*(-0.2524)})
	--({\sx*(2.6900)},{\sy*(-0.2636)})
	--({\sx*(2.7000)},{\sy*(-0.2746)})
	--({\sx*(2.7100)},{\sy*(-0.2852)})
	--({\sx*(2.7200)},{\sy*(-0.2954)})
	--({\sx*(2.7300)},{\sy*(-0.3054)})
	--({\sx*(2.7400)},{\sy*(-0.3150)})
	--({\sx*(2.7500)},{\sy*(-0.3242)})
	--({\sx*(2.7600)},{\sy*(-0.3331)})
	--({\sx*(2.7700)},{\sy*(-0.3417)})
	--({\sx*(2.7800)},{\sy*(-0.3499)})
	--({\sx*(2.7900)},{\sy*(-0.3578)})
	--({\sx*(2.8000)},{\sy*(-0.3653)})
	--({\sx*(2.8100)},{\sy*(-0.3725)})
	--({\sx*(2.8200)},{\sy*(-0.3794)})
	--({\sx*(2.8300)},{\sy*(-0.3858)})
	--({\sx*(2.8400)},{\sy*(-0.3920)})
	--({\sx*(2.8500)},{\sy*(-0.3978)})
	--({\sx*(2.8600)},{\sy*(-0.4032)})
	--({\sx*(2.8700)},{\sy*(-0.4083)})
	--({\sx*(2.8800)},{\sy*(-0.4131)})
	--({\sx*(2.8900)},{\sy*(-0.4175)})
	--({\sx*(2.9000)},{\sy*(-0.4215)})
	--({\sx*(2.9100)},{\sy*(-0.4253)})
	--({\sx*(2.9200)},{\sy*(-0.4286)})
	--({\sx*(2.9300)},{\sy*(-0.4317)})
	--({\sx*(2.9400)},{\sy*(-0.4344)})
	--({\sx*(2.9500)},{\sy*(-0.4368)})
	--({\sx*(2.9600)},{\sy*(-0.4388)})
	--({\sx*(2.9700)},{\sy*(-0.4405)})
	--({\sx*(2.9800)},{\sy*(-0.4419)})
	--({\sx*(2.9900)},{\sy*(-0.4430)})
	--({\sx*(3.0000)},{\sy*(-0.4437)})
	--({\sx*(3.0100)},{\sy*(-0.4441)})
	--({\sx*(3.0200)},{\sy*(-0.4443)})
	--({\sx*(3.0300)},{\sy*(-0.4441)})
	--({\sx*(3.0400)},{\sy*(-0.4436)})
	--({\sx*(3.0500)},{\sy*(-0.4428)})
	--({\sx*(3.0600)},{\sy*(-0.4417)})
	--({\sx*(3.0700)},{\sy*(-0.4404)})
	--({\sx*(3.0800)},{\sy*(-0.4387)})
	--({\sx*(3.0900)},{\sy*(-0.4368)})
	--({\sx*(3.1000)},{\sy*(-0.4346)})
	--({\sx*(3.1100)},{\sy*(-0.4321)})
	--({\sx*(3.1200)},{\sy*(-0.4293)})
	--({\sx*(3.1300)},{\sy*(-0.4263)})
	--({\sx*(3.1400)},{\sy*(-0.4231)})
	--({\sx*(3.1500)},{\sy*(-0.4196)})
	--({\sx*(3.1600)},{\sy*(-0.4158)})
	--({\sx*(3.1700)},{\sy*(-0.4119)})
	--({\sx*(3.1800)},{\sy*(-0.4077)})
	--({\sx*(3.1900)},{\sy*(-0.4032)})
	--({\sx*(3.2000)},{\sy*(-0.3986)})
	--({\sx*(3.2100)},{\sy*(-0.3937)})
	--({\sx*(3.2200)},{\sy*(-0.3887)})
	--({\sx*(3.2300)},{\sy*(-0.3834)})
	--({\sx*(3.2400)},{\sy*(-0.3779)})
	--({\sx*(3.2500)},{\sy*(-0.3723)})
	--({\sx*(3.2600)},{\sy*(-0.3665)})
	--({\sx*(3.2700)},{\sy*(-0.3605)})
	--({\sx*(3.2800)},{\sy*(-0.3543)})
	--({\sx*(3.2900)},{\sy*(-0.3480)})
	--({\sx*(3.3000)},{\sy*(-0.3415)})
	--({\sx*(3.3100)},{\sy*(-0.3349)})
	--({\sx*(3.3200)},{\sy*(-0.3281)})
	--({\sx*(3.3300)},{\sy*(-0.3213)})
	--({\sx*(3.3400)},{\sy*(-0.3142)})
	--({\sx*(3.3500)},{\sy*(-0.3071)})
	--({\sx*(3.3600)},{\sy*(-0.2999)})
	--({\sx*(3.3700)},{\sy*(-0.2925)})
	--({\sx*(3.3800)},{\sy*(-0.2851)})
	--({\sx*(3.3900)},{\sy*(-0.2775)})
	--({\sx*(3.4000)},{\sy*(-0.2699)})
	--({\sx*(3.4100)},{\sy*(-0.2622)})
	--({\sx*(3.4200)},{\sy*(-0.2544)})
	--({\sx*(3.4300)},{\sy*(-0.2466)})
	--({\sx*(3.4400)},{\sy*(-0.2387)})
	--({\sx*(3.4500)},{\sy*(-0.2307)})
	--({\sx*(3.4600)},{\sy*(-0.2228)})
	--({\sx*(3.4700)},{\sy*(-0.2147)})
	--({\sx*(3.4800)},{\sy*(-0.2067)})
	--({\sx*(3.4900)},{\sy*(-0.1986)})
	--({\sx*(3.5000)},{\sy*(-0.1905)})
	--({\sx*(3.5100)},{\sy*(-0.1824)})
	--({\sx*(3.5200)},{\sy*(-0.1743)})
	--({\sx*(3.5300)},{\sy*(-0.1662)})
	--({\sx*(3.5400)},{\sy*(-0.1581)})
	--({\sx*(3.5500)},{\sy*(-0.1500)})
	--({\sx*(3.5600)},{\sy*(-0.1419)})
	--({\sx*(3.5700)},{\sy*(-0.1339)})
	--({\sx*(3.5800)},{\sy*(-0.1258)})
	--({\sx*(3.5900)},{\sy*(-0.1179)})
	--({\sx*(3.6000)},{\sy*(-0.1099)})
	--({\sx*(3.6100)},{\sy*(-0.1021)})
	--({\sx*(3.6200)},{\sy*(-0.0942)})
	--({\sx*(3.6300)},{\sy*(-0.0865)})
	--({\sx*(3.6400)},{\sy*(-0.0788)})
	--({\sx*(3.6500)},{\sy*(-0.0712)})
	--({\sx*(3.6600)},{\sy*(-0.0636)})
	--({\sx*(3.6700)},{\sy*(-0.0562)})
	--({\sx*(3.6800)},{\sy*(-0.0488)})
	--({\sx*(3.6900)},{\sy*(-0.0415)})
	--({\sx*(3.7000)},{\sy*(-0.0343)})
	--({\sx*(3.7100)},{\sy*(-0.0272)})
	--({\sx*(3.7200)},{\sy*(-0.0203)})
	--({\sx*(3.7300)},{\sy*(-0.0134)})
	--({\sx*(3.7400)},{\sy*(-0.0066)})
	--({\sx*(3.7500)},{\sy*(0.0000)})
	--({\sx*(3.7600)},{\sy*(0.0065)})
	--({\sx*(3.7700)},{\sy*(0.0129)})
	--({\sx*(3.7800)},{\sy*(0.0191)})
	--({\sx*(3.7900)},{\sy*(0.0253)})
	--({\sx*(3.8000)},{\sy*(0.0312)})
	--({\sx*(3.8100)},{\sy*(0.0371)})
	--({\sx*(3.8200)},{\sy*(0.0428)})
	--({\sx*(3.8300)},{\sy*(0.0483)})
	--({\sx*(3.8400)},{\sy*(0.0537)})
	--({\sx*(3.8500)},{\sy*(0.0590)})
	--({\sx*(3.8600)},{\sy*(0.0641)})
	--({\sx*(3.8700)},{\sy*(0.0690)})
	--({\sx*(3.8800)},{\sy*(0.0737)})
	--({\sx*(3.8900)},{\sy*(0.0783)})
	--({\sx*(3.9000)},{\sy*(0.0828)})
	--({\sx*(3.9100)},{\sy*(0.0871)})
	--({\sx*(3.9200)},{\sy*(0.0912)})
	--({\sx*(3.9300)},{\sy*(0.0951)})
	--({\sx*(3.9400)},{\sy*(0.0988)})
	--({\sx*(3.9500)},{\sy*(0.1024)})
	--({\sx*(3.9600)},{\sy*(0.1058)})
	--({\sx*(3.9700)},{\sy*(0.1091)})
	--({\sx*(3.9800)},{\sy*(0.1121)})
	--({\sx*(3.9900)},{\sy*(0.1150)})
	--({\sx*(4.0000)},{\sy*(0.1177)})
	--({\sx*(4.0100)},{\sy*(0.1202)})
	--({\sx*(4.0200)},{\sy*(0.1226)})
	--({\sx*(4.0300)},{\sy*(0.1247)})
	--({\sx*(4.0400)},{\sy*(0.1267)})
	--({\sx*(4.0500)},{\sy*(0.1285)})
	--({\sx*(4.0600)},{\sy*(0.1301)})
	--({\sx*(4.0700)},{\sy*(0.1316)})
	--({\sx*(4.0800)},{\sy*(0.1329)})
	--({\sx*(4.0900)},{\sy*(0.1340)})
	--({\sx*(4.1000)},{\sy*(0.1349)})
	--({\sx*(4.1100)},{\sy*(0.1356)})
	--({\sx*(4.1200)},{\sy*(0.1362)})
	--({\sx*(4.1300)},{\sy*(0.1366)})
	--({\sx*(4.1400)},{\sy*(0.1368)})
	--({\sx*(4.1500)},{\sy*(0.1369)})
	--({\sx*(4.1600)},{\sy*(0.1368)})
	--({\sx*(4.1700)},{\sy*(0.1365)})
	--({\sx*(4.1800)},{\sy*(0.1361)})
	--({\sx*(4.1900)},{\sy*(0.1355)})
	--({\sx*(4.2000)},{\sy*(0.1348)})
	--({\sx*(4.2100)},{\sy*(0.1339)})
	--({\sx*(4.2200)},{\sy*(0.1328)})
	--({\sx*(4.2300)},{\sy*(0.1316)})
	--({\sx*(4.2400)},{\sy*(0.1303)})
	--({\sx*(4.2500)},{\sy*(0.1288)})
	--({\sx*(4.2600)},{\sy*(0.1271)})
	--({\sx*(4.2700)},{\sy*(0.1254)})
	--({\sx*(4.2800)},{\sy*(0.1235)})
	--({\sx*(4.2900)},{\sy*(0.1214)})
	--({\sx*(4.3000)},{\sy*(0.1193)})
	--({\sx*(4.3100)},{\sy*(0.1170)})
	--({\sx*(4.3200)},{\sy*(0.1146)})
	--({\sx*(4.3300)},{\sy*(0.1121)})
	--({\sx*(4.3400)},{\sy*(0.1095)})
	--({\sx*(4.3500)},{\sy*(0.1068)})
	--({\sx*(4.3600)},{\sy*(0.1040)})
	--({\sx*(4.3700)},{\sy*(0.1011)})
	--({\sx*(4.3800)},{\sy*(0.0981)})
	--({\sx*(4.3900)},{\sy*(0.0951)})
	--({\sx*(4.4000)},{\sy*(0.0919)})
	--({\sx*(4.4100)},{\sy*(0.0887)})
	--({\sx*(4.4200)},{\sy*(0.0854)})
	--({\sx*(4.4300)},{\sy*(0.0820)})
	--({\sx*(4.4400)},{\sy*(0.0786)})
	--({\sx*(4.4500)},{\sy*(0.0752)})
	--({\sx*(4.4600)},{\sy*(0.0716)})
	--({\sx*(4.4700)},{\sy*(0.0681)})
	--({\sx*(4.4800)},{\sy*(0.0645)})
	--({\sx*(4.4900)},{\sy*(0.0609)})
	--({\sx*(4.5000)},{\sy*(0.0573)})
	--({\sx*(4.5100)},{\sy*(0.0536)})
	--({\sx*(4.5200)},{\sy*(0.0500)})
	--({\sx*(4.5300)},{\sy*(0.0463)})
	--({\sx*(4.5400)},{\sy*(0.0426)})
	--({\sx*(4.5500)},{\sy*(0.0390)})
	--({\sx*(4.5600)},{\sy*(0.0354)})
	--({\sx*(4.5700)},{\sy*(0.0318)})
	--({\sx*(4.5800)},{\sy*(0.0282)})
	--({\sx*(4.5900)},{\sy*(0.0247)})
	--({\sx*(4.6000)},{\sy*(0.0212)})
	--({\sx*(4.6100)},{\sy*(0.0177)})
	--({\sx*(4.6200)},{\sy*(0.0143)})
	--({\sx*(4.6300)},{\sy*(0.0110)})
	--({\sx*(4.6400)},{\sy*(0.0078)})
	--({\sx*(4.6500)},{\sy*(0.0046)})
	--({\sx*(4.6600)},{\sy*(0.0015)})
	--({\sx*(4.6700)},{\sy*(-0.0015)})
	--({\sx*(4.6800)},{\sy*(-0.0043)})
	--({\sx*(4.6900)},{\sy*(-0.0071)})
	--({\sx*(4.7000)},{\sy*(-0.0098)})
	--({\sx*(4.7100)},{\sy*(-0.0123)})
	--({\sx*(4.7200)},{\sy*(-0.0147)})
	--({\sx*(4.7300)},{\sy*(-0.0170)})
	--({\sx*(4.7400)},{\sy*(-0.0191)})
	--({\sx*(4.7500)},{\sy*(-0.0211)})
	--({\sx*(4.7600)},{\sy*(-0.0229)})
	--({\sx*(4.7700)},{\sy*(-0.0246)})
	--({\sx*(4.7800)},{\sy*(-0.0261)})
	--({\sx*(4.7900)},{\sy*(-0.0274)})
	--({\sx*(4.8000)},{\sy*(-0.0285)})
	--({\sx*(4.8100)},{\sy*(-0.0294)})
	--({\sx*(4.8200)},{\sy*(-0.0301)})
	--({\sx*(4.8300)},{\sy*(-0.0306)})
	--({\sx*(4.8400)},{\sy*(-0.0308)})
	--({\sx*(4.8500)},{\sy*(-0.0309)})
	--({\sx*(4.8600)},{\sy*(-0.0307)})
	--({\sx*(4.8700)},{\sy*(-0.0303)})
	--({\sx*(4.8800)},{\sy*(-0.0296)})
	--({\sx*(4.8900)},{\sy*(-0.0287)})
	--({\sx*(4.9000)},{\sy*(-0.0275)})
	--({\sx*(4.9100)},{\sy*(-0.0261)})
	--({\sx*(4.9200)},{\sy*(-0.0244)})
	--({\sx*(4.9300)},{\sy*(-0.0224)})
	--({\sx*(4.9400)},{\sy*(-0.0201)})
	--({\sx*(4.9500)},{\sy*(-0.0175)})
	--({\sx*(4.9600)},{\sy*(-0.0146)})
	--({\sx*(4.9700)},{\sy*(-0.0115)})
	--({\sx*(4.9800)},{\sy*(-0.0080)})
	--({\sx*(4.9900)},{\sy*(-0.0041)})
	--({\sx*(5.0000)},{\sy*(0.0000)});
}
\def\relfehlerc{
\draw[color=blue,line width=1.4pt,line join=round] ({\sx*(0.000)},{\sy*(0.0000)})
	--({\sx*(0.0100)},{\sy*(0.0008)})
	--({\sx*(0.0200)},{\sy*(0.0015)})
	--({\sx*(0.0300)},{\sy*(0.0021)})
	--({\sx*(0.0400)},{\sy*(0.0026)})
	--({\sx*(0.0500)},{\sy*(0.0031)})
	--({\sx*(0.0600)},{\sy*(0.0036)})
	--({\sx*(0.0700)},{\sy*(0.0039)})
	--({\sx*(0.0800)},{\sy*(0.0042)})
	--({\sx*(0.0900)},{\sy*(0.0045)})
	--({\sx*(0.1000)},{\sy*(0.0047)})
	--({\sx*(0.1100)},{\sy*(0.0049)})
	--({\sx*(0.1200)},{\sy*(0.0050)})
	--({\sx*(0.1300)},{\sy*(0.0051)})
	--({\sx*(0.1400)},{\sy*(0.0051)})
	--({\sx*(0.1500)},{\sy*(0.0051)})
	--({\sx*(0.1600)},{\sy*(0.0050)})
	--({\sx*(0.1700)},{\sy*(0.0049)})
	--({\sx*(0.1800)},{\sy*(0.0048)})
	--({\sx*(0.1900)},{\sy*(0.0047)})
	--({\sx*(0.2000)},{\sy*(0.0045)})
	--({\sx*(0.2100)},{\sy*(0.0043)})
	--({\sx*(0.2200)},{\sy*(0.0041)})
	--({\sx*(0.2300)},{\sy*(0.0038)})
	--({\sx*(0.2400)},{\sy*(0.0035)})
	--({\sx*(0.2500)},{\sy*(0.0032)})
	--({\sx*(0.2600)},{\sy*(0.0029)})
	--({\sx*(0.2700)},{\sy*(0.0026)})
	--({\sx*(0.2800)},{\sy*(0.0022)})
	--({\sx*(0.2900)},{\sy*(0.0018)})
	--({\sx*(0.3000)},{\sy*(0.0014)})
	--({\sx*(0.3100)},{\sy*(0.0010)})
	--({\sx*(0.3200)},{\sy*(0.0006)})
	--({\sx*(0.3300)},{\sy*(0.0002)})
	--({\sx*(0.3400)},{\sy*(-0.0002)})
	--({\sx*(0.3500)},{\sy*(-0.0007)})
	--({\sx*(0.3600)},{\sy*(-0.0011)})
	--({\sx*(0.3700)},{\sy*(-0.0016)})
	--({\sx*(0.3800)},{\sy*(-0.0020)})
	--({\sx*(0.3900)},{\sy*(-0.0025)})
	--({\sx*(0.4000)},{\sy*(-0.0030)})
	--({\sx*(0.4100)},{\sy*(-0.0034)})
	--({\sx*(0.4200)},{\sy*(-0.0039)})
	--({\sx*(0.4300)},{\sy*(-0.0044)})
	--({\sx*(0.4400)},{\sy*(-0.0048)})
	--({\sx*(0.4500)},{\sy*(-0.0053)})
	--({\sx*(0.4600)},{\sy*(-0.0058)})
	--({\sx*(0.4700)},{\sy*(-0.0062)})
	--({\sx*(0.4800)},{\sy*(-0.0067)})
	--({\sx*(0.4900)},{\sy*(-0.0072)})
	--({\sx*(0.5000)},{\sy*(-0.0076)})
	--({\sx*(0.5100)},{\sy*(-0.0081)})
	--({\sx*(0.5200)},{\sy*(-0.0085)})
	--({\sx*(0.5300)},{\sy*(-0.0089)})
	--({\sx*(0.5400)},{\sy*(-0.0094)})
	--({\sx*(0.5500)},{\sy*(-0.0098)})
	--({\sx*(0.5600)},{\sy*(-0.0102)})
	--({\sx*(0.5700)},{\sy*(-0.0106)})
	--({\sx*(0.5800)},{\sy*(-0.0110)})
	--({\sx*(0.5900)},{\sy*(-0.0114)})
	--({\sx*(0.6000)},{\sy*(-0.0117)})
	--({\sx*(0.6100)},{\sy*(-0.0121)})
	--({\sx*(0.6200)},{\sy*(-0.0124)})
	--({\sx*(0.6300)},{\sy*(-0.0128)})
	--({\sx*(0.6400)},{\sy*(-0.0131)})
	--({\sx*(0.6500)},{\sy*(-0.0134)})
	--({\sx*(0.6600)},{\sy*(-0.0137)})
	--({\sx*(0.6700)},{\sy*(-0.0140)})
	--({\sx*(0.6800)},{\sy*(-0.0143)})
	--({\sx*(0.6900)},{\sy*(-0.0145)})
	--({\sx*(0.7000)},{\sy*(-0.0148)})
	--({\sx*(0.7100)},{\sy*(-0.0150)})
	--({\sx*(0.7200)},{\sy*(-0.0152)})
	--({\sx*(0.7300)},{\sy*(-0.0154)})
	--({\sx*(0.7400)},{\sy*(-0.0156)})
	--({\sx*(0.7500)},{\sy*(-0.0157)})
	--({\sx*(0.7600)},{\sy*(-0.0159)})
	--({\sx*(0.7700)},{\sy*(-0.0160)})
	--({\sx*(0.7800)},{\sy*(-0.0161)})
	--({\sx*(0.7900)},{\sy*(-0.0162)})
	--({\sx*(0.8000)},{\sy*(-0.0162)})
	--({\sx*(0.8100)},{\sy*(-0.0163)})
	--({\sx*(0.8200)},{\sy*(-0.0163)})
	--({\sx*(0.8300)},{\sy*(-0.0164)})
	--({\sx*(0.8400)},{\sy*(-0.0164)})
	--({\sx*(0.8500)},{\sy*(-0.0163)})
	--({\sx*(0.8600)},{\sy*(-0.0163)})
	--({\sx*(0.8700)},{\sy*(-0.0162)})
	--({\sx*(0.8800)},{\sy*(-0.0162)})
	--({\sx*(0.8900)},{\sy*(-0.0161)})
	--({\sx*(0.9000)},{\sy*(-0.0159)})
	--({\sx*(0.9100)},{\sy*(-0.0158)})
	--({\sx*(0.9200)},{\sy*(-0.0157)})
	--({\sx*(0.9300)},{\sy*(-0.0155)})
	--({\sx*(0.9400)},{\sy*(-0.0153)})
	--({\sx*(0.9500)},{\sy*(-0.0151)})
	--({\sx*(0.9600)},{\sy*(-0.0149)})
	--({\sx*(0.9700)},{\sy*(-0.0146)})
	--({\sx*(0.9800)},{\sy*(-0.0143)})
	--({\sx*(0.9900)},{\sy*(-0.0140)})
	--({\sx*(1.0000)},{\sy*(-0.0137)})
	--({\sx*(1.0100)},{\sy*(-0.0134)})
	--({\sx*(1.0200)},{\sy*(-0.0131)})
	--({\sx*(1.0300)},{\sy*(-0.0127)})
	--({\sx*(1.0400)},{\sy*(-0.0123)})
	--({\sx*(1.0500)},{\sy*(-0.0119)})
	--({\sx*(1.0600)},{\sy*(-0.0115)})
	--({\sx*(1.0700)},{\sy*(-0.0110)})
	--({\sx*(1.0800)},{\sy*(-0.0106)})
	--({\sx*(1.0900)},{\sy*(-0.0101)})
	--({\sx*(1.1000)},{\sy*(-0.0096)})
	--({\sx*(1.1100)},{\sy*(-0.0091)})
	--({\sx*(1.1200)},{\sy*(-0.0085)})
	--({\sx*(1.1300)},{\sy*(-0.0080)})
	--({\sx*(1.1400)},{\sy*(-0.0074)})
	--({\sx*(1.1500)},{\sy*(-0.0068)})
	--({\sx*(1.1600)},{\sy*(-0.0062)})
	--({\sx*(1.1700)},{\sy*(-0.0056)})
	--({\sx*(1.1800)},{\sy*(-0.0050)})
	--({\sx*(1.1900)},{\sy*(-0.0043)})
	--({\sx*(1.2000)},{\sy*(-0.0036)})
	--({\sx*(1.2100)},{\sy*(-0.0029)})
	--({\sx*(1.2200)},{\sy*(-0.0022)})
	--({\sx*(1.2300)},{\sy*(-0.0015)})
	--({\sx*(1.2400)},{\sy*(-0.0008)})
	--({\sx*(1.2500)},{\sy*(0.0000)})
	--({\sx*(1.2600)},{\sy*(0.0008)})
	--({\sx*(1.2700)},{\sy*(0.0016)})
	--({\sx*(1.2800)},{\sy*(0.0024)})
	--({\sx*(1.2900)},{\sy*(0.0032)})
	--({\sx*(1.3000)},{\sy*(0.0040)})
	--({\sx*(1.3100)},{\sy*(0.0048)})
	--({\sx*(1.3200)},{\sy*(0.0057)})
	--({\sx*(1.3300)},{\sy*(0.0066)})
	--({\sx*(1.3400)},{\sy*(0.0074)})
	--({\sx*(1.3500)},{\sy*(0.0083)})
	--({\sx*(1.3600)},{\sy*(0.0092)})
	--({\sx*(1.3700)},{\sy*(0.0102)})
	--({\sx*(1.3800)},{\sy*(0.0111)})
	--({\sx*(1.3900)},{\sy*(0.0120)})
	--({\sx*(1.4000)},{\sy*(0.0130)})
	--({\sx*(1.4100)},{\sy*(0.0139)})
	--({\sx*(1.4200)},{\sy*(0.0149)})
	--({\sx*(1.4300)},{\sy*(0.0159)})
	--({\sx*(1.4400)},{\sy*(0.0168)})
	--({\sx*(1.4500)},{\sy*(0.0178)})
	--({\sx*(1.4600)},{\sy*(0.0188)})
	--({\sx*(1.4700)},{\sy*(0.0198)})
	--({\sx*(1.4800)},{\sy*(0.0209)})
	--({\sx*(1.4900)},{\sy*(0.0219)})
	--({\sx*(1.5000)},{\sy*(0.0229)})
	--({\sx*(1.5100)},{\sy*(0.0239)})
	--({\sx*(1.5200)},{\sy*(0.0250)})
	--({\sx*(1.5300)},{\sy*(0.0260)})
	--({\sx*(1.5400)},{\sy*(0.0270)})
	--({\sx*(1.5500)},{\sy*(0.0281)})
	--({\sx*(1.5600)},{\sy*(0.0291)})
	--({\sx*(1.5700)},{\sy*(0.0302)})
	--({\sx*(1.5800)},{\sy*(0.0312)})
	--({\sx*(1.5900)},{\sy*(0.0323)})
	--({\sx*(1.6000)},{\sy*(0.0333)})
	--({\sx*(1.6100)},{\sy*(0.0344)})
	--({\sx*(1.6200)},{\sy*(0.0354)})
	--({\sx*(1.6300)},{\sy*(0.0365)})
	--({\sx*(1.6400)},{\sy*(0.0375)})
	--({\sx*(1.6500)},{\sy*(0.0386)})
	--({\sx*(1.6600)},{\sy*(0.0396)})
	--({\sx*(1.6700)},{\sy*(0.0406)})
	--({\sx*(1.6800)},{\sy*(0.0417)})
	--({\sx*(1.6900)},{\sy*(0.0427)})
	--({\sx*(1.7000)},{\sy*(0.0437)})
	--({\sx*(1.7100)},{\sy*(0.0447)})
	--({\sx*(1.7200)},{\sy*(0.0457)})
	--({\sx*(1.7300)},{\sy*(0.0467)})
	--({\sx*(1.7400)},{\sy*(0.0477)})
	--({\sx*(1.7500)},{\sy*(0.0486)})
	--({\sx*(1.7600)},{\sy*(0.0496)})
	--({\sx*(1.7700)},{\sy*(0.0505)})
	--({\sx*(1.7800)},{\sy*(0.0515)})
	--({\sx*(1.7900)},{\sy*(0.0524)})
	--({\sx*(1.8000)},{\sy*(0.0533)})
	--({\sx*(1.8100)},{\sy*(0.0542)})
	--({\sx*(1.8200)},{\sy*(0.0551)})
	--({\sx*(1.8300)},{\sy*(0.0559)})
	--({\sx*(1.8400)},{\sy*(0.0568)})
	--({\sx*(1.8500)},{\sy*(0.0576)})
	--({\sx*(1.8600)},{\sy*(0.0584)})
	--({\sx*(1.8700)},{\sy*(0.0592)})
	--({\sx*(1.8800)},{\sy*(0.0599)})
	--({\sx*(1.8900)},{\sy*(0.0607)})
	--({\sx*(1.9000)},{\sy*(0.0614)})
	--({\sx*(1.9100)},{\sy*(0.0621)})
	--({\sx*(1.9200)},{\sy*(0.0627)})
	--({\sx*(1.9300)},{\sy*(0.0634)})
	--({\sx*(1.9400)},{\sy*(0.0640)})
	--({\sx*(1.9500)},{\sy*(0.0645)})
	--({\sx*(1.9600)},{\sy*(0.0651)})
	--({\sx*(1.9700)},{\sy*(0.0656)})
	--({\sx*(1.9800)},{\sy*(0.0661)})
	--({\sx*(1.9900)},{\sy*(0.0665)})
	--({\sx*(2.0000)},{\sy*(0.0669)})
	--({\sx*(2.0100)},{\sy*(0.0673)})
	--({\sx*(2.0200)},{\sy*(0.0677)})
	--({\sx*(2.0300)},{\sy*(0.0680)})
	--({\sx*(2.0400)},{\sy*(0.0682)})
	--({\sx*(2.0500)},{\sy*(0.0684)})
	--({\sx*(2.0600)},{\sy*(0.0686)})
	--({\sx*(2.0700)},{\sy*(0.0687)})
	--({\sx*(2.0800)},{\sy*(0.0688)})
	--({\sx*(2.0900)},{\sy*(0.0688)})
	--({\sx*(2.1000)},{\sy*(0.0688)})
	--({\sx*(2.1100)},{\sy*(0.0687)})
	--({\sx*(2.1200)},{\sy*(0.0686)})
	--({\sx*(2.1300)},{\sy*(0.0684)})
	--({\sx*(2.1400)},{\sy*(0.0682)})
	--({\sx*(2.1500)},{\sy*(0.0679)})
	--({\sx*(2.1600)},{\sy*(0.0676)})
	--({\sx*(2.1700)},{\sy*(0.0671)})
	--({\sx*(2.1800)},{\sy*(0.0667)})
	--({\sx*(2.1900)},{\sy*(0.0661)})
	--({\sx*(2.2000)},{\sy*(0.0655)})
	--({\sx*(2.2100)},{\sy*(0.0648)})
	--({\sx*(2.2200)},{\sy*(0.0640)})
	--({\sx*(2.2300)},{\sy*(0.0632)})
	--({\sx*(2.2400)},{\sy*(0.0623)})
	--({\sx*(2.2500)},{\sy*(0.0613)})
	--({\sx*(2.2600)},{\sy*(0.0602)})
	--({\sx*(2.2700)},{\sy*(0.0590)})
	--({\sx*(2.2800)},{\sy*(0.0578)})
	--({\sx*(2.2900)},{\sy*(0.0564)})
	--({\sx*(2.3000)},{\sy*(0.0550)})
	--({\sx*(2.3100)},{\sy*(0.0534)})
	--({\sx*(2.3200)},{\sy*(0.0518)})
	--({\sx*(2.3300)},{\sy*(0.0500)})
	--({\sx*(2.3400)},{\sy*(0.0482)})
	--({\sx*(2.3500)},{\sy*(0.0462)})
	--({\sx*(2.3600)},{\sy*(0.0441)})
	--({\sx*(2.3700)},{\sy*(0.0419)})
	--({\sx*(2.3800)},{\sy*(0.0396)})
	--({\sx*(2.3900)},{\sy*(0.0371)})
	--({\sx*(2.4000)},{\sy*(0.0345)})
	--({\sx*(2.4100)},{\sy*(0.0318)})
	--({\sx*(2.4200)},{\sy*(0.0289)})
	--({\sx*(2.4300)},{\sy*(0.0258)})
	--({\sx*(2.4400)},{\sy*(0.0227)})
	--({\sx*(2.4500)},{\sy*(0.0193)})
	--({\sx*(2.4600)},{\sy*(0.0158)})
	--({\sx*(2.4700)},{\sy*(0.0121)})
	--({\sx*(2.4800)},{\sy*(0.0083)})
	--({\sx*(2.4900)},{\sy*(0.0042)})
	--({\sx*(2.5000)},{\sy*(0.0000)})
	--({\sx*(2.5100)},{\sy*(-0.0044)})
	--({\sx*(2.5200)},{\sy*(-0.0091)})
	--({\sx*(2.5300)},{\sy*(-0.0139)})
	--({\sx*(2.5400)},{\sy*(-0.0190)})
	--({\sx*(2.5500)},{\sy*(-0.0243)})
	--({\sx*(2.5600)},{\sy*(-0.0299)})
	--({\sx*(2.5700)},{\sy*(-0.0357)})
	--({\sx*(2.5800)},{\sy*(-0.0418)})
	--({\sx*(2.5900)},{\sy*(-0.0481)})
	--({\sx*(2.6000)},{\sy*(-0.0548)})
	--({\sx*(2.6100)},{\sy*(-0.0617)})
	--({\sx*(2.6200)},{\sy*(-0.0689)})
	--({\sx*(2.6300)},{\sy*(-0.0765)})
	--({\sx*(2.6400)},{\sy*(-0.0844)})
	--({\sx*(2.6500)},{\sy*(-0.0927)})
	--({\sx*(2.6600)},{\sy*(-0.1014)})
	--({\sx*(2.6700)},{\sy*(-0.1104)})
	--({\sx*(2.6800)},{\sy*(-0.1199)})
	--({\sx*(2.6900)},{\sy*(-0.1298)})
	--({\sx*(2.7000)},{\sy*(-0.1401)})
	--({\sx*(2.7100)},{\sy*(-0.1509)})
	--({\sx*(2.7200)},{\sy*(-0.1622)})
	--({\sx*(2.7300)},{\sy*(-0.1741)})
	--({\sx*(2.7400)},{\sy*(-0.1865)})
	--({\sx*(2.7500)},{\sy*(-0.1994)})
	--({\sx*(2.7600)},{\sy*(-0.2130)})
	--({\sx*(2.7700)},{\sy*(-0.2273)})
	--({\sx*(2.7800)},{\sy*(-0.2422)})
	--({\sx*(2.7900)},{\sy*(-0.2579)})
	--({\sx*(2.8000)},{\sy*(-0.2743)})
	--({\sx*(2.8100)},{\sy*(-0.2916)})
	--({\sx*(2.8200)},{\sy*(-0.3097)})
	--({\sx*(2.8300)},{\sy*(-0.3287)})
	--({\sx*(2.8400)},{\sy*(-0.3487)})
	--({\sx*(2.8500)},{\sy*(-0.3697)})
	--({\sx*(2.8600)},{\sy*(-0.3919)})
	--({\sx*(2.8700)},{\sy*(-0.4152)})
	--({\sx*(2.8800)},{\sy*(-0.4398)})
	--({\sx*(2.8900)},{\sy*(-0.4658)})
	--({\sx*(2.9000)},{\sy*(-0.4932)})
	--({\sx*(2.9100)},{\sy*(-0.5221)})
	--({\sx*(2.9200)},{\sy*(-0.5527)})
	--({\sx*(2.9300)},{\sy*(-0.5851)})
	--({\sx*(2.9400)},{\sy*(-0.6195)})
	--({\sx*(2.9500)},{\sy*(-0.6559)})
	--({\sx*(2.9600)},{\sy*(-0.6946)})
	--({\sx*(2.9700)},{\sy*(-0.7357)})
	--({\sx*(2.9800)},{\sy*(-0.7795)})
	--({\sx*(2.9900)},{\sy*(-0.8262)})
	--({\sx*(3.0000)},{\sy*(-0.8760)})
	--({\sx*(3.0100)},{\sy*(-0.9292)})
	--({\sx*(3.0200)},{\sy*(-0.9862)})
	--({\sx*(3.0300)},{\sy*(-1.0474)})
	--({\sx*(3.0400)},{\sy*(-1.1131)})
	--({\sx*(3.0500)},{\sy*(-1.1838)})
	--({\sx*(3.0600)},{\sy*(-1.2601)})
	--({\sx*(3.0700)},{\sy*(-1.3425)})
	--({\sx*(3.0800)},{\sy*(-1.4318)})
	--({\sx*(3.0900)},{\sy*(-1.5288)})
	--({\sx*(3.1000)},{\sy*(-1.6344)})
	--({\sx*(3.1100)},{\sy*(-1.7497)})
	--({\sx*(3.1200)},{\sy*(-1.8760)})
	--({\sx*(3.1300)},{\sy*(-2.0148)})
	--({\sx*(3.1400)},{\sy*(-2.1678)})
	--({\sx*(3.1500)},{\sy*(-2.3372)})
	--({\sx*(3.1600)},{\sy*(-2.5254)})
	--({\sx*(3.1700)},{\sy*(-2.7357)})
	--({\sx*(3.1800)},{\sy*(-2.9716)})
	--({\sx*(3.1900)},{\sy*(-3.2378)})
	--({\sx*(3.2000)},{\sy*(-3.5401)})
	--({\sx*(3.2100)},{\sy*(-3.8856)})
	--({\sx*(3.2200)},{\sy*(-4.2836)})
	--({\sx*(3.2300)},{\sy*(-4.7460)})
	--({\sx*(3.2400)},{\sy*(-5.2885)})
	--({\sx*(3.2500)},{\sy*(-5.9323)})
	--({\sx*(3.2600)},{\sy*(-6.7062)})
	--({\sx*(3.2700)},{\sy*(-7.6511)})
	--({\sx*(3.2800)},{\sy*(-8.8264)})
	--({\sx*(3.2900)},{\sy*(-10.3214)})
	--({\sx*(3.3000)},{\sy*(-12.2777)})
	--({\sx*(3.3100)},{\sy*(-14.9321)})
	--({\sx*(3.3200)},{\sy*(-18.7135)})
	--({\sx*(3.3300)},{\sy*(-24.4844)})
	--({\sx*(3.3400)},{\sy*(-34.2711)})
	--({\sx*(3.3500)},{\sy*(-54.2133)})
	--({\sx*(3.3600)},{\sy*(-115.6958)})
	--({\sx*(3.3700)},{\sy*(4498.3930)})
	--({\sx*(3.3800)},{\sy*(122.8238)})
	--({\sx*(3.3900)},{\sy*(66.2215)})
	--({\sx*(3.4000)},{\sy*(47.4374)})
	--({\sx*(3.4100)},{\sy*(38.3791)})
	--({\sx*(3.4200)},{\sy*(33.3374)})
	--({\sx*(3.4300)},{\sy*(30.4283)})
	--({\sx*(3.4400)},{\sy*(28.8903)})
	--({\sx*(3.4500)},{\sy*(28.4170)})
	--({\sx*(3.4600)},{\sy*(28.9614)})
	--({\sx*(3.4700)},{\sy*(30.7199)})
	--({\sx*(3.4800)},{\sy*(34.2826)})
	--({\sx*(3.4900)},{\sy*(41.1997)})
	--({\sx*(3.5000)},{\sy*(56.3941)})
	--({\sx*(3.5100)},{\sy*(106.4012)})
	--({\sx*(3.5200)},{\sy*(-1098.0118)})
	--({\sx*(3.5300)},{\sy*(-74.5634)})
	--({\sx*(3.5400)},{\sy*(-35.2563)})
	--({\sx*(3.5500)},{\sy*(-21.7003)})
	--({\sx*(3.5600)},{\sy*(-14.9363)})
	--({\sx*(3.5700)},{\sy*(-10.9361)})
	--({\sx*(3.5800)},{\sy*(-8.3240)})
	--({\sx*(3.5900)},{\sy*(-6.5033)})
	--({\sx*(3.6000)},{\sy*(-5.1740)})
	--({\sx*(3.6100)},{\sy*(-4.1693)})
	--({\sx*(3.6200)},{\sy*(-3.3891)})
	--({\sx*(3.6300)},{\sy*(-2.7699)})
	--({\sx*(3.6400)},{\sy*(-2.2697)})
	--({\sx*(3.6500)},{\sy*(-1.8595)})
	--({\sx*(3.6600)},{\sy*(-1.5188)})
	--({\sx*(3.6700)},{\sy*(-1.2328)})
	--({\sx*(3.6800)},{\sy*(-0.9904)})
	--({\sx*(3.6900)},{\sy*(-0.7832)})
	--({\sx*(3.7000)},{\sy*(-0.6047)})
	--({\sx*(3.7100)},{\sy*(-0.4499)})
	--({\sx*(3.7200)},{\sy*(-0.3150)})
	--({\sx*(3.7300)},{\sy*(-0.1966)})
	--({\sx*(3.7400)},{\sy*(-0.0923)})
	--({\sx*(3.7500)},{\sy*(0.0000)})
	--({\sx*(3.7600)},{\sy*(0.0821)})
	--({\sx*(3.7700)},{\sy*(0.1553)})
	--({\sx*(3.7800)},{\sy*(0.2209)})
	--({\sx*(3.7900)},{\sy*(0.2799)})
	--({\sx*(3.8000)},{\sy*(0.3330)})
	--({\sx*(3.8100)},{\sy*(0.3810)})
	--({\sx*(3.8200)},{\sy*(0.4245)})
	--({\sx*(3.8300)},{\sy*(0.4640)})
	--({\sx*(3.8400)},{\sy*(0.5000)})
	--({\sx*(3.8500)},{\sy*(0.5328)})
	--({\sx*(3.8600)},{\sy*(0.5629)})
	--({\sx*(3.8700)},{\sy*(0.5904)})
	--({\sx*(3.8800)},{\sy*(0.6157)})
	--({\sx*(3.8900)},{\sy*(0.6389)})
	--({\sx*(3.9000)},{\sy*(0.6603)})
	--({\sx*(3.9100)},{\sy*(0.6800)})
	--({\sx*(3.9200)},{\sy*(0.6983)})
	--({\sx*(3.9300)},{\sy*(0.7151)})
	--({\sx*(3.9400)},{\sy*(0.7308)})
	--({\sx*(3.9500)},{\sy*(0.7453)})
	--({\sx*(3.9600)},{\sy*(0.7588)})
	--({\sx*(3.9700)},{\sy*(0.7713)})
	--({\sx*(3.9800)},{\sy*(0.7830)})
	--({\sx*(3.9900)},{\sy*(0.7938)})
	--({\sx*(4.0000)},{\sy*(0.8040)})
	--({\sx*(4.0100)},{\sy*(0.8135)})
	--({\sx*(4.0200)},{\sy*(0.8223)})
	--({\sx*(4.0300)},{\sy*(0.8306)})
	--({\sx*(4.0400)},{\sy*(0.8383)})
	--({\sx*(4.0500)},{\sy*(0.8456)})
	--({\sx*(4.0600)},{\sy*(0.8524)})
	--({\sx*(4.0700)},{\sy*(0.8588)})
	--({\sx*(4.0800)},{\sy*(0.8648)})
	--({\sx*(4.0900)},{\sy*(0.8704)})
	--({\sx*(4.1000)},{\sy*(0.8757)})
	--({\sx*(4.1100)},{\sy*(0.8807)})
	--({\sx*(4.1200)},{\sy*(0.8854)})
	--({\sx*(4.1300)},{\sy*(0.8898)})
	--({\sx*(4.1400)},{\sy*(0.8940)})
	--({\sx*(4.1500)},{\sy*(0.8979)})
	--({\sx*(4.1600)},{\sy*(0.9015)})
	--({\sx*(4.1700)},{\sy*(0.9050)})
	--({\sx*(4.1800)},{\sy*(0.9083)})
	--({\sx*(4.1900)},{\sy*(0.9114)})
	--({\sx*(4.2000)},{\sy*(0.9143)})
	--({\sx*(4.2100)},{\sy*(0.9170)})
	--({\sx*(4.2200)},{\sy*(0.9196)})
	--({\sx*(4.2300)},{\sy*(0.9220)})
	--({\sx*(4.2400)},{\sy*(0.9243)})
	--({\sx*(4.2500)},{\sy*(0.9264)})
	--({\sx*(4.2600)},{\sy*(0.9284)})
	--({\sx*(4.2700)},{\sy*(0.9303)})
	--({\sx*(4.2800)},{\sy*(0.9320)})
	--({\sx*(4.2900)},{\sy*(0.9337)})
	--({\sx*(4.3000)},{\sy*(0.9352)})
	--({\sx*(4.3100)},{\sy*(0.9367)})
	--({\sx*(4.3200)},{\sy*(0.9380)})
	--({\sx*(4.3300)},{\sy*(0.9392)})
	--({\sx*(4.3400)},{\sy*(0.9403)})
	--({\sx*(4.3500)},{\sy*(0.9414)})
	--({\sx*(4.3600)},{\sy*(0.9423)})
	--({\sx*(4.3700)},{\sy*(0.9431)})
	--({\sx*(4.3800)},{\sy*(0.9438)})
	--({\sx*(4.3900)},{\sy*(0.9445)})
	--({\sx*(4.4000)},{\sy*(0.9450)})
	--({\sx*(4.4100)},{\sy*(0.9454)})
	--({\sx*(4.4200)},{\sy*(0.9458)})
	--({\sx*(4.4300)},{\sy*(0.9460)})
	--({\sx*(4.4400)},{\sy*(0.9461)})
	--({\sx*(4.4500)},{\sy*(0.9460)})
	--({\sx*(4.4600)},{\sy*(0.9459)})
	--({\sx*(4.4700)},{\sy*(0.9456)})
	--({\sx*(4.4800)},{\sy*(0.9451)})
	--({\sx*(4.4900)},{\sy*(0.9444)})
	--({\sx*(4.5000)},{\sy*(0.9435)})
	--({\sx*(4.5100)},{\sy*(0.9424)})
	--({\sx*(4.5200)},{\sy*(0.9410)})
	--({\sx*(4.5300)},{\sy*(0.9393)})
	--({\sx*(4.5400)},{\sy*(0.9371)})
	--({\sx*(4.5500)},{\sy*(0.9345)})
	--({\sx*(4.5600)},{\sy*(0.9312)})
	--({\sx*(4.5700)},{\sy*(0.9272)})
	--({\sx*(4.5800)},{\sy*(0.9220)})
	--({\sx*(4.5900)},{\sy*(0.9155)})
	--({\sx*(4.6000)},{\sy*(0.9068)})
	--({\sx*(4.6100)},{\sy*(0.8951)})
	--({\sx*(4.6200)},{\sy*(0.8785)})
	--({\sx*(4.6300)},{\sy*(0.8534)})
	--({\sx*(4.6400)},{\sy*(0.8113)})
	--({\sx*(4.6500)},{\sy*(0.7275)})
	--({\sx*(4.6600)},{\sy*(0.4807)})
	--({\sx*(4.6700)},{\sy*(-13.3531)})
	--({\sx*(4.6800)},{\sy*(1.5266)})
	--({\sx*(4.6900)},{\sy*(1.2513)})
	--({\sx*(4.7000)},{\sy*(1.1621)})
	--({\sx*(4.7100)},{\sy*(1.1182)})
	--({\sx*(4.7200)},{\sy*(1.0921)})
	--({\sx*(4.7300)},{\sy*(1.0749)})
	--({\sx*(4.7400)},{\sy*(1.0628)})
	--({\sx*(4.7500)},{\sy*(1.0538)})
	--({\sx*(4.7600)},{\sy*(1.0469)})
	--({\sx*(4.7700)},{\sy*(1.0415)})
	--({\sx*(4.7800)},{\sy*(1.0372)})
	--({\sx*(4.7900)},{\sy*(1.0337)})
	--({\sx*(4.8000)},{\sy*(1.0308)})
	--({\sx*(4.8100)},{\sy*(1.0283)})
	--({\sx*(4.8200)},{\sy*(1.0263)})
	--({\sx*(4.8300)},{\sy*(1.0246)})
	--({\sx*(4.8400)},{\sy*(1.0232)})
	--({\sx*(4.8500)},{\sy*(1.0221)})
	--({\sx*(4.8600)},{\sy*(1.0211)})
	--({\sx*(4.8700)},{\sy*(1.0204)})
	--({\sx*(4.8800)},{\sy*(1.0198)})
	--({\sx*(4.8900)},{\sy*(1.0195)})
	--({\sx*(4.9000)},{\sy*(1.0194)})
	--({\sx*(4.9100)},{\sy*(1.0194)})
	--({\sx*(4.9200)},{\sy*(1.0198)})
	--({\sx*(4.9300)},{\sy*(1.0206)})
	--({\sx*(4.9400)},{\sy*(1.0218)})
	--({\sx*(4.9500)},{\sy*(1.0239)})
	--({\sx*(4.9600)},{\sy*(1.0273)})
	--({\sx*(4.9700)},{\sy*(1.0334)})
	--({\sx*(4.9800)},{\sy*(1.0463)})
	--({\sx*(4.9900)},{\sy*(1.0880)})
	--({\sx*(5.0000)},{\sy*(0.0000)});
}
\def\xwerted{
\fill[color=red] (0.0000,0) circle[radius={0.07/\skala}];
\fill[color=white] (0.0000,0) circle[radius={0.05/\skala}];
\fill[color=red] (0.1903,0) circle[radius={0.07/\skala}];
\fill[color=white] (0.1903,0) circle[radius={0.05/\skala}];
\fill[color=red] (0.7322,0) circle[radius={0.07/\skala}];
\fill[color=white] (0.7322,0) circle[radius={0.05/\skala}];
\fill[color=red] (1.5433,0) circle[radius={0.07/\skala}];
\fill[color=white] (1.5433,0) circle[radius={0.05/\skala}];
\fill[color=red] (2.5000,0) circle[radius={0.07/\skala}];
\fill[color=white] (2.5000,0) circle[radius={0.05/\skala}];
\fill[color=red] (3.4567,0) circle[radius={0.07/\skala}];
\fill[color=white] (3.4567,0) circle[radius={0.05/\skala}];
\fill[color=red] (4.2678,0) circle[radius={0.07/\skala}];
\fill[color=white] (4.2678,0) circle[radius={0.05/\skala}];
\fill[color=red] (4.8097,0) circle[radius={0.07/\skala}];
\fill[color=white] (4.8097,0) circle[radius={0.05/\skala}];
\fill[color=red] (5.0000,0) circle[radius={0.07/\skala}];
\fill[color=white] (5.0000,0) circle[radius={0.05/\skala}];
}
\def\punkted{8}
\def\maxfehlerd{3.419\cdot 10^{-4}}
\def\fehlerd{
\draw[color=red,line width=1.4pt,line join=round] ({\sx*(0.000)},{\sy*(0.0000)})
	--({\sx*(0.0100)},{\sy*(-0.0833)})
	--({\sx*(0.0200)},{\sy*(-0.1512)})
	--({\sx*(0.0300)},{\sy*(-0.2053)})
	--({\sx*(0.0400)},{\sy*(-0.2466)})
	--({\sx*(0.0500)},{\sy*(-0.2764)})
	--({\sx*(0.0600)},{\sy*(-0.2959)})
	--({\sx*(0.0700)},{\sy*(-0.3059)})
	--({\sx*(0.0800)},{\sy*(-0.3076)})
	--({\sx*(0.0900)},{\sy*(-0.3018)})
	--({\sx*(0.1000)},{\sy*(-0.2895)})
	--({\sx*(0.1100)},{\sy*(-0.2714)})
	--({\sx*(0.1200)},{\sy*(-0.2483)})
	--({\sx*(0.1300)},{\sy*(-0.2210)})
	--({\sx*(0.1400)},{\sy*(-0.1900)})
	--({\sx*(0.1500)},{\sy*(-0.1560)})
	--({\sx*(0.1600)},{\sy*(-0.1197)})
	--({\sx*(0.1700)},{\sy*(-0.0814)})
	--({\sx*(0.1800)},{\sy*(-0.0418)})
	--({\sx*(0.1900)},{\sy*(-0.0012)})
	--({\sx*(0.2000)},{\sy*(0.0398)})
	--({\sx*(0.2100)},{\sy*(0.0811)})
	--({\sx*(0.2200)},{\sy*(0.1221)})
	--({\sx*(0.2300)},{\sy*(0.1625)})
	--({\sx*(0.2400)},{\sy*(0.2022)})
	--({\sx*(0.2500)},{\sy*(0.2407)})
	--({\sx*(0.2600)},{\sy*(0.2780)})
	--({\sx*(0.2700)},{\sy*(0.3138)})
	--({\sx*(0.2800)},{\sy*(0.3479)})
	--({\sx*(0.2900)},{\sy*(0.3801)})
	--({\sx*(0.3000)},{\sy*(0.4104)})
	--({\sx*(0.3100)},{\sy*(0.4386)})
	--({\sx*(0.3200)},{\sy*(0.4646)})
	--({\sx*(0.3300)},{\sy*(0.4884)})
	--({\sx*(0.3400)},{\sy*(0.5098)})
	--({\sx*(0.3500)},{\sy*(0.5290)})
	--({\sx*(0.3600)},{\sy*(0.5457)})
	--({\sx*(0.3700)},{\sy*(0.5601)})
	--({\sx*(0.3800)},{\sy*(0.5720)})
	--({\sx*(0.3900)},{\sy*(0.5816)})
	--({\sx*(0.4000)},{\sy*(0.5889)})
	--({\sx*(0.4100)},{\sy*(0.5938)})
	--({\sx*(0.4200)},{\sy*(0.5965)})
	--({\sx*(0.4300)},{\sy*(0.5969)})
	--({\sx*(0.4400)},{\sy*(0.5952)})
	--({\sx*(0.4500)},{\sy*(0.5914)})
	--({\sx*(0.4600)},{\sy*(0.5855)})
	--({\sx*(0.4700)},{\sy*(0.5778)})
	--({\sx*(0.4800)},{\sy*(0.5681)})
	--({\sx*(0.4900)},{\sy*(0.5567)})
	--({\sx*(0.5000)},{\sy*(0.5436)})
	--({\sx*(0.5100)},{\sy*(0.5290)})
	--({\sx*(0.5200)},{\sy*(0.5128)})
	--({\sx*(0.5300)},{\sy*(0.4953)})
	--({\sx*(0.5400)},{\sy*(0.4765)})
	--({\sx*(0.5500)},{\sy*(0.4564)})
	--({\sx*(0.5600)},{\sy*(0.4353)})
	--({\sx*(0.5700)},{\sy*(0.4132)})
	--({\sx*(0.5800)},{\sy*(0.3903)})
	--({\sx*(0.5900)},{\sy*(0.3665)})
	--({\sx*(0.6000)},{\sy*(0.3421)})
	--({\sx*(0.6100)},{\sy*(0.3171)})
	--({\sx*(0.6200)},{\sy*(0.2916)})
	--({\sx*(0.6300)},{\sy*(0.2658)})
	--({\sx*(0.6400)},{\sy*(0.2396)})
	--({\sx*(0.6500)},{\sy*(0.2133)})
	--({\sx*(0.6600)},{\sy*(0.1868)})
	--({\sx*(0.6700)},{\sy*(0.1603)})
	--({\sx*(0.6800)},{\sy*(0.1339)})
	--({\sx*(0.6900)},{\sy*(0.1076)})
	--({\sx*(0.7000)},{\sy*(0.0816)})
	--({\sx*(0.7100)},{\sy*(0.0558)})
	--({\sx*(0.7200)},{\sy*(0.0304)})
	--({\sx*(0.7300)},{\sy*(0.0055)})
	--({\sx*(0.7400)},{\sy*(-0.0189)})
	--({\sx*(0.7500)},{\sy*(-0.0428)})
	--({\sx*(0.7600)},{\sy*(-0.0661)})
	--({\sx*(0.7700)},{\sy*(-0.0887)})
	--({\sx*(0.7800)},{\sy*(-0.1105)})
	--({\sx*(0.7900)},{\sy*(-0.1317)})
	--({\sx*(0.8000)},{\sy*(-0.1519)})
	--({\sx*(0.8100)},{\sy*(-0.1714)})
	--({\sx*(0.8200)},{\sy*(-0.1900)})
	--({\sx*(0.8300)},{\sy*(-0.2076)})
	--({\sx*(0.8400)},{\sy*(-0.2243)})
	--({\sx*(0.8500)},{\sy*(-0.2401)})
	--({\sx*(0.8600)},{\sy*(-0.2548)})
	--({\sx*(0.8700)},{\sy*(-0.2686)})
	--({\sx*(0.8800)},{\sy*(-0.2814)})
	--({\sx*(0.8900)},{\sy*(-0.2931)})
	--({\sx*(0.9000)},{\sy*(-0.3038)})
	--({\sx*(0.9100)},{\sy*(-0.3135)})
	--({\sx*(0.9200)},{\sy*(-0.3221)})
	--({\sx*(0.9300)},{\sy*(-0.3298)})
	--({\sx*(0.9400)},{\sy*(-0.3364)})
	--({\sx*(0.9500)},{\sy*(-0.3420)})
	--({\sx*(0.9600)},{\sy*(-0.3466)})
	--({\sx*(0.9700)},{\sy*(-0.3502)})
	--({\sx*(0.9800)},{\sy*(-0.3528)})
	--({\sx*(0.9900)},{\sy*(-0.3545)})
	--({\sx*(1.0000)},{\sy*(-0.3553)})
	--({\sx*(1.0100)},{\sy*(-0.3552)})
	--({\sx*(1.0200)},{\sy*(-0.3542)})
	--({\sx*(1.0300)},{\sy*(-0.3524)})
	--({\sx*(1.0400)},{\sy*(-0.3497)})
	--({\sx*(1.0500)},{\sy*(-0.3463)})
	--({\sx*(1.0600)},{\sy*(-0.3421)})
	--({\sx*(1.0700)},{\sy*(-0.3372)})
	--({\sx*(1.0800)},{\sy*(-0.3316)})
	--({\sx*(1.0900)},{\sy*(-0.3254)})
	--({\sx*(1.1000)},{\sy*(-0.3186)})
	--({\sx*(1.1100)},{\sy*(-0.3112)})
	--({\sx*(1.1200)},{\sy*(-0.3033)})
	--({\sx*(1.1300)},{\sy*(-0.2949)})
	--({\sx*(1.1400)},{\sy*(-0.2860)})
	--({\sx*(1.1500)},{\sy*(-0.2768)})
	--({\sx*(1.1600)},{\sy*(-0.2672)})
	--({\sx*(1.1700)},{\sy*(-0.2573)})
	--({\sx*(1.1800)},{\sy*(-0.2471)})
	--({\sx*(1.1900)},{\sy*(-0.2366)})
	--({\sx*(1.2000)},{\sy*(-0.2260)})
	--({\sx*(1.2100)},{\sy*(-0.2152)})
	--({\sx*(1.2200)},{\sy*(-0.2043)})
	--({\sx*(1.2300)},{\sy*(-0.1934)})
	--({\sx*(1.2400)},{\sy*(-0.1824)})
	--({\sx*(1.2500)},{\sy*(-0.1714)})
	--({\sx*(1.2600)},{\sy*(-0.1605)})
	--({\sx*(1.2700)},{\sy*(-0.1497)})
	--({\sx*(1.2800)},{\sy*(-0.1389)})
	--({\sx*(1.2900)},{\sy*(-0.1284)})
	--({\sx*(1.3000)},{\sy*(-0.1181)})
	--({\sx*(1.3100)},{\sy*(-0.1080)})
	--({\sx*(1.3200)},{\sy*(-0.0981)})
	--({\sx*(1.3300)},{\sy*(-0.0886)})
	--({\sx*(1.3400)},{\sy*(-0.0794)})
	--({\sx*(1.3500)},{\sy*(-0.0705)})
	--({\sx*(1.3600)},{\sy*(-0.0621)})
	--({\sx*(1.3700)},{\sy*(-0.0540)})
	--({\sx*(1.3800)},{\sy*(-0.0464)})
	--({\sx*(1.3900)},{\sy*(-0.0393)})
	--({\sx*(1.4000)},{\sy*(-0.0327)})
	--({\sx*(1.4100)},{\sy*(-0.0265)})
	--({\sx*(1.4200)},{\sy*(-0.0209)})
	--({\sx*(1.4300)},{\sy*(-0.0158)})
	--({\sx*(1.4400)},{\sy*(-0.0113)})
	--({\sx*(1.4500)},{\sy*(-0.0074)})
	--({\sx*(1.4600)},{\sy*(-0.0041)})
	--({\sx*(1.4700)},{\sy*(-0.0013)})
	--({\sx*(1.4800)},{\sy*(0.0008)})
	--({\sx*(1.4900)},{\sy*(0.0024)})
	--({\sx*(1.5000)},{\sy*(0.0033)})
	--({\sx*(1.5100)},{\sy*(0.0036)})
	--({\sx*(1.5200)},{\sy*(0.0032)})
	--({\sx*(1.5300)},{\sy*(0.0023)})
	--({\sx*(1.5400)},{\sy*(0.0007)})
	--({\sx*(1.5500)},{\sy*(-0.0016)})
	--({\sx*(1.5600)},{\sy*(-0.0044)})
	--({\sx*(1.5700)},{\sy*(-0.0078)})
	--({\sx*(1.5800)},{\sy*(-0.0119)})
	--({\sx*(1.5900)},{\sy*(-0.0166)})
	--({\sx*(1.6000)},{\sy*(-0.0218)})
	--({\sx*(1.6100)},{\sy*(-0.0276)})
	--({\sx*(1.6200)},{\sy*(-0.0340)})
	--({\sx*(1.6300)},{\sy*(-0.0409)})
	--({\sx*(1.6400)},{\sy*(-0.0484)})
	--({\sx*(1.6500)},{\sy*(-0.0564)})
	--({\sx*(1.6600)},{\sy*(-0.0648)})
	--({\sx*(1.6700)},{\sy*(-0.0738)})
	--({\sx*(1.6800)},{\sy*(-0.0832)})
	--({\sx*(1.6900)},{\sy*(-0.0930)})
	--({\sx*(1.7000)},{\sy*(-0.1033)})
	--({\sx*(1.7100)},{\sy*(-0.1139)})
	--({\sx*(1.7200)},{\sy*(-0.1249)})
	--({\sx*(1.7300)},{\sy*(-0.1362)})
	--({\sx*(1.7400)},{\sy*(-0.1478)})
	--({\sx*(1.7500)},{\sy*(-0.1597)})
	--({\sx*(1.7600)},{\sy*(-0.1719)})
	--({\sx*(1.7700)},{\sy*(-0.1843)})
	--({\sx*(1.7800)},{\sy*(-0.1969)})
	--({\sx*(1.7900)},{\sy*(-0.2096)})
	--({\sx*(1.8000)},{\sy*(-0.2225)})
	--({\sx*(1.8100)},{\sy*(-0.2355)})
	--({\sx*(1.8200)},{\sy*(-0.2486)})
	--({\sx*(1.8300)},{\sy*(-0.2617)})
	--({\sx*(1.8400)},{\sy*(-0.2748)})
	--({\sx*(1.8500)},{\sy*(-0.2878)})
	--({\sx*(1.8600)},{\sy*(-0.3009)})
	--({\sx*(1.8700)},{\sy*(-0.3138)})
	--({\sx*(1.8800)},{\sy*(-0.3266)})
	--({\sx*(1.8900)},{\sy*(-0.3393)})
	--({\sx*(1.9000)},{\sy*(-0.3518)})
	--({\sx*(1.9100)},{\sy*(-0.3640)})
	--({\sx*(1.9200)},{\sy*(-0.3760)})
	--({\sx*(1.9300)},{\sy*(-0.3878)})
	--({\sx*(1.9400)},{\sy*(-0.3992)})
	--({\sx*(1.9500)},{\sy*(-0.4103)})
	--({\sx*(1.9600)},{\sy*(-0.4210)})
	--({\sx*(1.9700)},{\sy*(-0.4313)})
	--({\sx*(1.9800)},{\sy*(-0.4412)})
	--({\sx*(1.9900)},{\sy*(-0.4506)})
	--({\sx*(2.0000)},{\sy*(-0.4596)})
	--({\sx*(2.0100)},{\sy*(-0.4680)})
	--({\sx*(2.0200)},{\sy*(-0.4759)})
	--({\sx*(2.0300)},{\sy*(-0.4833)})
	--({\sx*(2.0400)},{\sy*(-0.4900)})
	--({\sx*(2.0500)},{\sy*(-0.4962)})
	--({\sx*(2.0600)},{\sy*(-0.5017)})
	--({\sx*(2.0700)},{\sy*(-0.5066)})
	--({\sx*(2.0800)},{\sy*(-0.5108)})
	--({\sx*(2.0900)},{\sy*(-0.5143)})
	--({\sx*(2.1000)},{\sy*(-0.5171)})
	--({\sx*(2.1100)},{\sy*(-0.5192)})
	--({\sx*(2.1200)},{\sy*(-0.5206)})
	--({\sx*(2.1300)},{\sy*(-0.5212)})
	--({\sx*(2.1400)},{\sy*(-0.5210)})
	--({\sx*(2.1500)},{\sy*(-0.5200)})
	--({\sx*(2.1600)},{\sy*(-0.5183)})
	--({\sx*(2.1700)},{\sy*(-0.5157)})
	--({\sx*(2.1800)},{\sy*(-0.5124)})
	--({\sx*(2.1900)},{\sy*(-0.5082)})
	--({\sx*(2.2000)},{\sy*(-0.5032)})
	--({\sx*(2.2100)},{\sy*(-0.4974)})
	--({\sx*(2.2200)},{\sy*(-0.4908)})
	--({\sx*(2.2300)},{\sy*(-0.4833)})
	--({\sx*(2.2400)},{\sy*(-0.4750)})
	--({\sx*(2.2500)},{\sy*(-0.4659)})
	--({\sx*(2.2600)},{\sy*(-0.4559)})
	--({\sx*(2.2700)},{\sy*(-0.4452)})
	--({\sx*(2.2800)},{\sy*(-0.4336)})
	--({\sx*(2.2900)},{\sy*(-0.4212)})
	--({\sx*(2.3000)},{\sy*(-0.4080)})
	--({\sx*(2.3100)},{\sy*(-0.3940)})
	--({\sx*(2.3200)},{\sy*(-0.3793)})
	--({\sx*(2.3300)},{\sy*(-0.3637)})
	--({\sx*(2.3400)},{\sy*(-0.3474)})
	--({\sx*(2.3500)},{\sy*(-0.3304)})
	--({\sx*(2.3600)},{\sy*(-0.3127)})
	--({\sx*(2.3700)},{\sy*(-0.2942)})
	--({\sx*(2.3800)},{\sy*(-0.2751)})
	--({\sx*(2.3900)},{\sy*(-0.2553)})
	--({\sx*(2.4000)},{\sy*(-0.2348)})
	--({\sx*(2.4100)},{\sy*(-0.2137)})
	--({\sx*(2.4200)},{\sy*(-0.1921)})
	--({\sx*(2.4300)},{\sy*(-0.1698)})
	--({\sx*(2.4400)},{\sy*(-0.1470)})
	--({\sx*(2.4500)},{\sy*(-0.1236)})
	--({\sx*(2.4600)},{\sy*(-0.0998)})
	--({\sx*(2.4700)},{\sy*(-0.0755)})
	--({\sx*(2.4800)},{\sy*(-0.0507)})
	--({\sx*(2.4900)},{\sy*(-0.0256)})
	--({\sx*(2.5000)},{\sy*(0.0000)})
	--({\sx*(2.5100)},{\sy*(0.0259)})
	--({\sx*(2.5200)},{\sy*(0.0521)})
	--({\sx*(2.5300)},{\sy*(0.0786)})
	--({\sx*(2.5400)},{\sy*(0.1054)})
	--({\sx*(2.5500)},{\sy*(0.1324)})
	--({\sx*(2.5600)},{\sy*(0.1595)})
	--({\sx*(2.5700)},{\sy*(0.1869)})
	--({\sx*(2.5800)},{\sy*(0.2143)})
	--({\sx*(2.5900)},{\sy*(0.2418)})
	--({\sx*(2.6000)},{\sy*(0.2693)})
	--({\sx*(2.6100)},{\sy*(0.2969)})
	--({\sx*(2.6200)},{\sy*(0.3244)})
	--({\sx*(2.6300)},{\sy*(0.3518)})
	--({\sx*(2.6400)},{\sy*(0.3792)})
	--({\sx*(2.6500)},{\sy*(0.4064)})
	--({\sx*(2.6600)},{\sy*(0.4334)})
	--({\sx*(2.6700)},{\sy*(0.4602)})
	--({\sx*(2.6800)},{\sy*(0.4867)})
	--({\sx*(2.6900)},{\sy*(0.5130)})
	--({\sx*(2.7000)},{\sy*(0.5389)})
	--({\sx*(2.7100)},{\sy*(0.5644)})
	--({\sx*(2.7200)},{\sy*(0.5896)})
	--({\sx*(2.7300)},{\sy*(0.6143)})
	--({\sx*(2.7400)},{\sy*(0.6385)})
	--({\sx*(2.7500)},{\sy*(0.6622)})
	--({\sx*(2.7600)},{\sy*(0.6854)})
	--({\sx*(2.7700)},{\sy*(0.7080)})
	--({\sx*(2.7800)},{\sy*(0.7299)})
	--({\sx*(2.7900)},{\sy*(0.7513)})
	--({\sx*(2.8000)},{\sy*(0.7719)})
	--({\sx*(2.8100)},{\sy*(0.7919)})
	--({\sx*(2.8200)},{\sy*(0.8110)})
	--({\sx*(2.8300)},{\sy*(0.8295)})
	--({\sx*(2.8400)},{\sy*(0.8471)})
	--({\sx*(2.8500)},{\sy*(0.8639)})
	--({\sx*(2.8600)},{\sy*(0.8798)})
	--({\sx*(2.8700)},{\sy*(0.8948)})
	--({\sx*(2.8800)},{\sy*(0.9089)})
	--({\sx*(2.8900)},{\sy*(0.9221)})
	--({\sx*(2.9000)},{\sy*(0.9343)})
	--({\sx*(2.9100)},{\sy*(0.9456)})
	--({\sx*(2.9200)},{\sy*(0.9558)})
	--({\sx*(2.9300)},{\sy*(0.9651)})
	--({\sx*(2.9400)},{\sy*(0.9732)})
	--({\sx*(2.9500)},{\sy*(0.9803)})
	--({\sx*(2.9600)},{\sy*(0.9864)})
	--({\sx*(2.9700)},{\sy*(0.9913)})
	--({\sx*(2.9800)},{\sy*(0.9952)})
	--({\sx*(2.9900)},{\sy*(0.9979)})
	--({\sx*(3.0000)},{\sy*(0.9995)})
	--({\sx*(3.0100)},{\sy*(1.0000)})
	--({\sx*(3.0200)},{\sy*(0.9993)})
	--({\sx*(3.0300)},{\sy*(0.9975)})
	--({\sx*(3.0400)},{\sy*(0.9945)})
	--({\sx*(3.0500)},{\sy*(0.9904)})
	--({\sx*(3.0600)},{\sy*(0.9851)})
	--({\sx*(3.0700)},{\sy*(0.9787)})
	--({\sx*(3.0800)},{\sy*(0.9710)})
	--({\sx*(3.0900)},{\sy*(0.9623)})
	--({\sx*(3.1000)},{\sy*(0.9524)})
	--({\sx*(3.1100)},{\sy*(0.9413)})
	--({\sx*(3.1200)},{\sy*(0.9291)})
	--({\sx*(3.1300)},{\sy*(0.9158)})
	--({\sx*(3.1400)},{\sy*(0.9013)})
	--({\sx*(3.1500)},{\sy*(0.8858)})
	--({\sx*(3.1600)},{\sy*(0.8691)})
	--({\sx*(3.1700)},{\sy*(0.8514)})
	--({\sx*(3.1800)},{\sy*(0.8326)})
	--({\sx*(3.1900)},{\sy*(0.8128)})
	--({\sx*(3.2000)},{\sy*(0.7920)})
	--({\sx*(3.2100)},{\sy*(0.7701)})
	--({\sx*(3.2200)},{\sy*(0.7473)})
	--({\sx*(3.2300)},{\sy*(0.7235)})
	--({\sx*(3.2400)},{\sy*(0.6988)})
	--({\sx*(3.2500)},{\sy*(0.6732)})
	--({\sx*(3.2600)},{\sy*(0.6467)})
	--({\sx*(3.2700)},{\sy*(0.6194)})
	--({\sx*(3.2800)},{\sy*(0.5912)})
	--({\sx*(3.2900)},{\sy*(0.5623)})
	--({\sx*(3.3000)},{\sy*(0.5327)})
	--({\sx*(3.3100)},{\sy*(0.5023)})
	--({\sx*(3.3200)},{\sy*(0.4712)})
	--({\sx*(3.3300)},{\sy*(0.4395)})
	--({\sx*(3.3400)},{\sy*(0.4072)})
	--({\sx*(3.3500)},{\sy*(0.3744)})
	--({\sx*(3.3600)},{\sy*(0.3410)})
	--({\sx*(3.3700)},{\sy*(0.3071)})
	--({\sx*(3.3800)},{\sy*(0.2729)})
	--({\sx*(3.3900)},{\sy*(0.2382)})
	--({\sx*(3.4000)},{\sy*(0.2031)})
	--({\sx*(3.4100)},{\sy*(0.1678)})
	--({\sx*(3.4200)},{\sy*(0.1322)})
	--({\sx*(3.4300)},{\sy*(0.0964)})
	--({\sx*(3.4400)},{\sy*(0.0604)})
	--({\sx*(3.4500)},{\sy*(0.0243)})
	--({\sx*(3.4600)},{\sy*(-0.0119)})
	--({\sx*(3.4700)},{\sy*(-0.0481)})
	--({\sx*(3.4800)},{\sy*(-0.0843)})
	--({\sx*(3.4900)},{\sy*(-0.1204)})
	--({\sx*(3.5000)},{\sy*(-0.1564)})
	--({\sx*(3.5100)},{\sy*(-0.1922)})
	--({\sx*(3.5200)},{\sy*(-0.2278)})
	--({\sx*(3.5300)},{\sy*(-0.2631)})
	--({\sx*(3.5400)},{\sy*(-0.2981)})
	--({\sx*(3.5500)},{\sy*(-0.3327)})
	--({\sx*(3.5600)},{\sy*(-0.3670)})
	--({\sx*(3.5700)},{\sy*(-0.4007)})
	--({\sx*(3.5800)},{\sy*(-0.4339)})
	--({\sx*(3.5900)},{\sy*(-0.4666)})
	--({\sx*(3.6000)},{\sy*(-0.4986)})
	--({\sx*(3.6100)},{\sy*(-0.5300)})
	--({\sx*(3.6200)},{\sy*(-0.5607)})
	--({\sx*(3.6300)},{\sy*(-0.5906)})
	--({\sx*(3.6400)},{\sy*(-0.6198)})
	--({\sx*(3.6500)},{\sy*(-0.6481)})
	--({\sx*(3.6600)},{\sy*(-0.6755)})
	--({\sx*(3.6700)},{\sy*(-0.7019)})
	--({\sx*(3.6800)},{\sy*(-0.7274)})
	--({\sx*(3.6900)},{\sy*(-0.7519)})
	--({\sx*(3.7000)},{\sy*(-0.7753)})
	--({\sx*(3.7100)},{\sy*(-0.7977)})
	--({\sx*(3.7200)},{\sy*(-0.8189)})
	--({\sx*(3.7300)},{\sy*(-0.8389)})
	--({\sx*(3.7400)},{\sy*(-0.8578)})
	--({\sx*(3.7500)},{\sy*(-0.8754)})
	--({\sx*(3.7600)},{\sy*(-0.8917)})
	--({\sx*(3.7700)},{\sy*(-0.9068)})
	--({\sx*(3.7800)},{\sy*(-0.9205)})
	--({\sx*(3.7900)},{\sy*(-0.9329)})
	--({\sx*(3.8000)},{\sy*(-0.9439)})
	--({\sx*(3.8100)},{\sy*(-0.9536)})
	--({\sx*(3.8200)},{\sy*(-0.9618)})
	--({\sx*(3.8300)},{\sy*(-0.9686)})
	--({\sx*(3.8400)},{\sy*(-0.9739)})
	--({\sx*(3.8500)},{\sy*(-0.9777)})
	--({\sx*(3.8600)},{\sy*(-0.9801)})
	--({\sx*(3.8700)},{\sy*(-0.9810)})
	--({\sx*(3.8800)},{\sy*(-0.9804)})
	--({\sx*(3.8900)},{\sy*(-0.9783)})
	--({\sx*(3.9000)},{\sy*(-0.9747)})
	--({\sx*(3.9100)},{\sy*(-0.9697)})
	--({\sx*(3.9200)},{\sy*(-0.9631)})
	--({\sx*(3.9300)},{\sy*(-0.9550)})
	--({\sx*(3.9400)},{\sy*(-0.9454)})
	--({\sx*(3.9500)},{\sy*(-0.9343)})
	--({\sx*(3.9600)},{\sy*(-0.9218)})
	--({\sx*(3.9700)},{\sy*(-0.9078)})
	--({\sx*(3.9800)},{\sy*(-0.8924)})
	--({\sx*(3.9900)},{\sy*(-0.8755)})
	--({\sx*(4.0000)},{\sy*(-0.8573)})
	--({\sx*(4.0100)},{\sy*(-0.8377)})
	--({\sx*(4.0200)},{\sy*(-0.8167)})
	--({\sx*(4.0300)},{\sy*(-0.7945)})
	--({\sx*(4.0400)},{\sy*(-0.7709)})
	--({\sx*(4.0500)},{\sy*(-0.7461)})
	--({\sx*(4.0600)},{\sy*(-0.7201)})
	--({\sx*(4.0700)},{\sy*(-0.6930)})
	--({\sx*(4.0800)},{\sy*(-0.6647)})
	--({\sx*(4.0900)},{\sy*(-0.6354)})
	--({\sx*(4.1000)},{\sy*(-0.6050)})
	--({\sx*(4.1100)},{\sy*(-0.5737)})
	--({\sx*(4.1200)},{\sy*(-0.5415)})
	--({\sx*(4.1300)},{\sy*(-0.5084)})
	--({\sx*(4.1400)},{\sy*(-0.4744)})
	--({\sx*(4.1500)},{\sy*(-0.4398)})
	--({\sx*(4.1600)},{\sy*(-0.4045)})
	--({\sx*(4.1700)},{\sy*(-0.3685)})
	--({\sx*(4.1800)},{\sy*(-0.3320)})
	--({\sx*(4.1900)},{\sy*(-0.2951)})
	--({\sx*(4.2000)},{\sy*(-0.2577)})
	--({\sx*(4.2100)},{\sy*(-0.2201)})
	--({\sx*(4.2200)},{\sy*(-0.1821)})
	--({\sx*(4.2300)},{\sy*(-0.1440)})
	--({\sx*(4.2400)},{\sy*(-0.1058)})
	--({\sx*(4.2500)},{\sy*(-0.0676)})
	--({\sx*(4.2600)},{\sy*(-0.0295)})
	--({\sx*(4.2700)},{\sy*(0.0085)})
	--({\sx*(4.2800)},{\sy*(0.0462)})
	--({\sx*(4.2900)},{\sy*(0.0836)})
	--({\sx*(4.3000)},{\sy*(0.1206)})
	--({\sx*(4.3100)},{\sy*(0.1571)})
	--({\sx*(4.3200)},{\sy*(0.1930)})
	--({\sx*(4.3300)},{\sy*(0.2283)})
	--({\sx*(4.3400)},{\sy*(0.2627)})
	--({\sx*(4.3500)},{\sy*(0.2964)})
	--({\sx*(4.3600)},{\sy*(0.3291)})
	--({\sx*(4.3700)},{\sy*(0.3607)})
	--({\sx*(4.3800)},{\sy*(0.3913)})
	--({\sx*(4.3900)},{\sy*(0.4206)})
	--({\sx*(4.4000)},{\sy*(0.4487)})
	--({\sx*(4.4100)},{\sy*(0.4754)})
	--({\sx*(4.4200)},{\sy*(0.5007)})
	--({\sx*(4.4300)},{\sy*(0.5244)})
	--({\sx*(4.4400)},{\sy*(0.5465)})
	--({\sx*(4.4500)},{\sy*(0.5670)})
	--({\sx*(4.4600)},{\sy*(0.5857)})
	--({\sx*(4.4700)},{\sy*(0.6026)})
	--({\sx*(4.4800)},{\sy*(0.6176)})
	--({\sx*(4.4900)},{\sy*(0.6306)})
	--({\sx*(4.5000)},{\sy*(0.6416)})
	--({\sx*(4.5100)},{\sy*(0.6506)})
	--({\sx*(4.5200)},{\sy*(0.6575)})
	--({\sx*(4.5300)},{\sy*(0.6622)})
	--({\sx*(4.5400)},{\sy*(0.6648)})
	--({\sx*(4.5500)},{\sy*(0.6651)})
	--({\sx*(4.5600)},{\sy*(0.6632)})
	--({\sx*(4.5700)},{\sy*(0.6590)})
	--({\sx*(4.5800)},{\sy*(0.6525)})
	--({\sx*(4.5900)},{\sy*(0.6438)})
	--({\sx*(4.6000)},{\sy*(0.6328)})
	--({\sx*(4.6100)},{\sy*(0.6195)})
	--({\sx*(4.6200)},{\sy*(0.6040)})
	--({\sx*(4.6300)},{\sy*(0.5863)})
	--({\sx*(4.6400)},{\sy*(0.5665)})
	--({\sx*(4.6500)},{\sy*(0.5445)})
	--({\sx*(4.6600)},{\sy*(0.5204)})
	--({\sx*(4.6700)},{\sy*(0.4944)})
	--({\sx*(4.6800)},{\sy*(0.4666)})
	--({\sx*(4.6900)},{\sy*(0.4369)})
	--({\sx*(4.7000)},{\sy*(0.4056)})
	--({\sx*(4.7100)},{\sy*(0.3727)})
	--({\sx*(4.7200)},{\sy*(0.3384)})
	--({\sx*(4.7300)},{\sy*(0.3029)})
	--({\sx*(4.7400)},{\sy*(0.2663)})
	--({\sx*(4.7500)},{\sy*(0.2289)})
	--({\sx*(4.7600)},{\sy*(0.1908)})
	--({\sx*(4.7700)},{\sy*(0.1522)})
	--({\sx*(4.7800)},{\sy*(0.1135)})
	--({\sx*(4.7900)},{\sy*(0.0748)})
	--({\sx*(4.8000)},{\sy*(0.0365)})
	--({\sx*(4.8100)},{\sy*(-0.0011)})
	--({\sx*(4.8200)},{\sy*(-0.0378)})
	--({\sx*(4.8300)},{\sy*(-0.0731)})
	--({\sx*(4.8400)},{\sy*(-0.1067)})
	--({\sx*(4.8500)},{\sy*(-0.1381)})
	--({\sx*(4.8600)},{\sy*(-0.1671)})
	--({\sx*(4.8700)},{\sy*(-0.1930)})
	--({\sx*(4.8800)},{\sy*(-0.2155)})
	--({\sx*(4.8900)},{\sy*(-0.2340)})
	--({\sx*(4.9000)},{\sy*(-0.2480)})
	--({\sx*(4.9100)},{\sy*(-0.2569)})
	--({\sx*(4.9200)},{\sy*(-0.2601)})
	--({\sx*(4.9300)},{\sy*(-0.2571)})
	--({\sx*(4.9400)},{\sy*(-0.2471)})
	--({\sx*(4.9500)},{\sy*(-0.2295)})
	--({\sx*(4.9600)},{\sy*(-0.2035)})
	--({\sx*(4.9700)},{\sy*(-0.1684)})
	--({\sx*(4.9800)},{\sy*(-0.1233)})
	--({\sx*(4.9900)},{\sy*(-0.0675)})
	--({\sx*(5.0000)},{\sy*(0.0000)});
}
\def\relfehlerd{
\draw[color=blue,line width=1.4pt,line join=round] ({\sx*(0.000)},{\sy*(0.0000)})
	--({\sx*(0.0100)},{\sy*(-0.0001)})
	--({\sx*(0.0200)},{\sy*(-0.0001)})
	--({\sx*(0.0300)},{\sy*(-0.0002)})
	--({\sx*(0.0400)},{\sy*(-0.0002)})
	--({\sx*(0.0500)},{\sy*(-0.0002)})
	--({\sx*(0.0600)},{\sy*(-0.0003)})
	--({\sx*(0.0700)},{\sy*(-0.0003)})
	--({\sx*(0.0800)},{\sy*(-0.0003)})
	--({\sx*(0.0900)},{\sy*(-0.0003)})
	--({\sx*(0.1000)},{\sy*(-0.0002)})
	--({\sx*(0.1100)},{\sy*(-0.0002)})
	--({\sx*(0.1200)},{\sy*(-0.0002)})
	--({\sx*(0.1300)},{\sy*(-0.0002)})
	--({\sx*(0.1400)},{\sy*(-0.0002)})
	--({\sx*(0.1500)},{\sy*(-0.0001)})
	--({\sx*(0.1600)},{\sy*(-0.0001)})
	--({\sx*(0.1700)},{\sy*(-0.0001)})
	--({\sx*(0.1800)},{\sy*(-0.0000)})
	--({\sx*(0.1900)},{\sy*(-0.0000)})
	--({\sx*(0.2000)},{\sy*(0.0000)})
	--({\sx*(0.2100)},{\sy*(0.0001)})
	--({\sx*(0.2200)},{\sy*(0.0001)})
	--({\sx*(0.2300)},{\sy*(0.0001)})
	--({\sx*(0.2400)},{\sy*(0.0002)})
	--({\sx*(0.2500)},{\sy*(0.0002)})
	--({\sx*(0.2600)},{\sy*(0.0002)})
	--({\sx*(0.2700)},{\sy*(0.0003)})
	--({\sx*(0.2800)},{\sy*(0.0003)})
	--({\sx*(0.2900)},{\sy*(0.0003)})
	--({\sx*(0.3000)},{\sy*(0.0004)})
	--({\sx*(0.3100)},{\sy*(0.0004)})
	--({\sx*(0.3200)},{\sy*(0.0004)})
	--({\sx*(0.3300)},{\sy*(0.0004)})
	--({\sx*(0.3400)},{\sy*(0.0005)})
	--({\sx*(0.3500)},{\sy*(0.0005)})
	--({\sx*(0.3600)},{\sy*(0.0005)})
	--({\sx*(0.3700)},{\sy*(0.0005)})
	--({\sx*(0.3800)},{\sy*(0.0005)})
	--({\sx*(0.3900)},{\sy*(0.0005)})
	--({\sx*(0.4000)},{\sy*(0.0005)})
	--({\sx*(0.4100)},{\sy*(0.0006)})
	--({\sx*(0.4200)},{\sy*(0.0006)})
	--({\sx*(0.4300)},{\sy*(0.0006)})
	--({\sx*(0.4400)},{\sy*(0.0006)})
	--({\sx*(0.4500)},{\sy*(0.0006)})
	--({\sx*(0.4600)},{\sy*(0.0006)})
	--({\sx*(0.4700)},{\sy*(0.0006)})
	--({\sx*(0.4800)},{\sy*(0.0005)})
	--({\sx*(0.4900)},{\sy*(0.0005)})
	--({\sx*(0.5000)},{\sy*(0.0005)})
	--({\sx*(0.5100)},{\sy*(0.0005)})
	--({\sx*(0.5200)},{\sy*(0.0005)})
	--({\sx*(0.5300)},{\sy*(0.0005)})
	--({\sx*(0.5400)},{\sy*(0.0005)})
	--({\sx*(0.5500)},{\sy*(0.0005)})
	--({\sx*(0.5600)},{\sy*(0.0004)})
	--({\sx*(0.5700)},{\sy*(0.0004)})
	--({\sx*(0.5800)},{\sy*(0.0004)})
	--({\sx*(0.5900)},{\sy*(0.0004)})
	--({\sx*(0.6000)},{\sy*(0.0004)})
	--({\sx*(0.6100)},{\sy*(0.0003)})
	--({\sx*(0.6200)},{\sy*(0.0003)})
	--({\sx*(0.6300)},{\sy*(0.0003)})
	--({\sx*(0.6400)},{\sy*(0.0003)})
	--({\sx*(0.6500)},{\sy*(0.0002)})
	--({\sx*(0.6600)},{\sy*(0.0002)})
	--({\sx*(0.6700)},{\sy*(0.0002)})
	--({\sx*(0.6800)},{\sy*(0.0001)})
	--({\sx*(0.6900)},{\sy*(0.0001)})
	--({\sx*(0.7000)},{\sy*(0.0001)})
	--({\sx*(0.7100)},{\sy*(0.0001)})
	--({\sx*(0.7200)},{\sy*(0.0000)})
	--({\sx*(0.7300)},{\sy*(0.0000)})
	--({\sx*(0.7400)},{\sy*(-0.0000)})
	--({\sx*(0.7500)},{\sy*(-0.0000)})
	--({\sx*(0.7600)},{\sy*(-0.0001)})
	--({\sx*(0.7700)},{\sy*(-0.0001)})
	--({\sx*(0.7800)},{\sy*(-0.0001)})
	--({\sx*(0.7900)},{\sy*(-0.0002)})
	--({\sx*(0.8000)},{\sy*(-0.0002)})
	--({\sx*(0.8100)},{\sy*(-0.0002)})
	--({\sx*(0.8200)},{\sy*(-0.0002)})
	--({\sx*(0.8300)},{\sy*(-0.0003)})
	--({\sx*(0.8400)},{\sy*(-0.0003)})
	--({\sx*(0.8500)},{\sy*(-0.0003)})
	--({\sx*(0.8600)},{\sy*(-0.0003)})
	--({\sx*(0.8700)},{\sy*(-0.0003)})
	--({\sx*(0.8800)},{\sy*(-0.0004)})
	--({\sx*(0.8900)},{\sy*(-0.0004)})
	--({\sx*(0.9000)},{\sy*(-0.0004)})
	--({\sx*(0.9100)},{\sy*(-0.0004)})
	--({\sx*(0.9200)},{\sy*(-0.0004)})
	--({\sx*(0.9300)},{\sy*(-0.0004)})
	--({\sx*(0.9400)},{\sy*(-0.0004)})
	--({\sx*(0.9500)},{\sy*(-0.0005)})
	--({\sx*(0.9600)},{\sy*(-0.0005)})
	--({\sx*(0.9700)},{\sy*(-0.0005)})
	--({\sx*(0.9800)},{\sy*(-0.0005)})
	--({\sx*(0.9900)},{\sy*(-0.0005)})
	--({\sx*(1.0000)},{\sy*(-0.0005)})
	--({\sx*(1.0100)},{\sy*(-0.0005)})
	--({\sx*(1.0200)},{\sy*(-0.0005)})
	--({\sx*(1.0300)},{\sy*(-0.0005)})
	--({\sx*(1.0400)},{\sy*(-0.0005)})
	--({\sx*(1.0500)},{\sy*(-0.0005)})
	--({\sx*(1.0600)},{\sy*(-0.0005)})
	--({\sx*(1.0700)},{\sy*(-0.0005)})
	--({\sx*(1.0800)},{\sy*(-0.0005)})
	--({\sx*(1.0900)},{\sy*(-0.0005)})
	--({\sx*(1.1000)},{\sy*(-0.0005)})
	--({\sx*(1.1100)},{\sy*(-0.0005)})
	--({\sx*(1.1200)},{\sy*(-0.0005)})
	--({\sx*(1.1300)},{\sy*(-0.0005)})
	--({\sx*(1.1400)},{\sy*(-0.0005)})
	--({\sx*(1.1500)},{\sy*(-0.0005)})
	--({\sx*(1.1600)},{\sy*(-0.0004)})
	--({\sx*(1.1700)},{\sy*(-0.0004)})
	--({\sx*(1.1800)},{\sy*(-0.0004)})
	--({\sx*(1.1900)},{\sy*(-0.0004)})
	--({\sx*(1.2000)},{\sy*(-0.0004)})
	--({\sx*(1.2100)},{\sy*(-0.0004)})
	--({\sx*(1.2200)},{\sy*(-0.0004)})
	--({\sx*(1.2300)},{\sy*(-0.0004)})
	--({\sx*(1.2400)},{\sy*(-0.0003)})
	--({\sx*(1.2500)},{\sy*(-0.0003)})
	--({\sx*(1.2600)},{\sy*(-0.0003)})
	--({\sx*(1.2700)},{\sy*(-0.0003)})
	--({\sx*(1.2800)},{\sy*(-0.0003)})
	--({\sx*(1.2900)},{\sy*(-0.0003)})
	--({\sx*(1.3000)},{\sy*(-0.0002)})
	--({\sx*(1.3100)},{\sy*(-0.0002)})
	--({\sx*(1.3200)},{\sy*(-0.0002)})
	--({\sx*(1.3300)},{\sy*(-0.0002)})
	--({\sx*(1.3400)},{\sy*(-0.0002)})
	--({\sx*(1.3500)},{\sy*(-0.0002)})
	--({\sx*(1.3600)},{\sy*(-0.0001)})
	--({\sx*(1.3700)},{\sy*(-0.0001)})
	--({\sx*(1.3800)},{\sy*(-0.0001)})
	--({\sx*(1.3900)},{\sy*(-0.0001)})
	--({\sx*(1.4000)},{\sy*(-0.0001)})
	--({\sx*(1.4100)},{\sy*(-0.0001)})
	--({\sx*(1.4200)},{\sy*(-0.0000)})
	--({\sx*(1.4300)},{\sy*(-0.0000)})
	--({\sx*(1.4400)},{\sy*(-0.0000)})
	--({\sx*(1.4500)},{\sy*(-0.0000)})
	--({\sx*(1.4600)},{\sy*(-0.0000)})
	--({\sx*(1.4700)},{\sy*(-0.0000)})
	--({\sx*(1.4800)},{\sy*(0.0000)})
	--({\sx*(1.4900)},{\sy*(0.0000)})
	--({\sx*(1.5000)},{\sy*(0.0000)})
	--({\sx*(1.5100)},{\sy*(0.0000)})
	--({\sx*(1.5200)},{\sy*(0.0000)})
	--({\sx*(1.5300)},{\sy*(0.0000)})
	--({\sx*(1.5400)},{\sy*(0.0000)})
	--({\sx*(1.5500)},{\sy*(-0.0000)})
	--({\sx*(1.5600)},{\sy*(-0.0000)})
	--({\sx*(1.5700)},{\sy*(-0.0000)})
	--({\sx*(1.5800)},{\sy*(-0.0000)})
	--({\sx*(1.5900)},{\sy*(-0.0001)})
	--({\sx*(1.6000)},{\sy*(-0.0001)})
	--({\sx*(1.6100)},{\sy*(-0.0001)})
	--({\sx*(1.6200)},{\sy*(-0.0001)})
	--({\sx*(1.6300)},{\sy*(-0.0001)})
	--({\sx*(1.6400)},{\sy*(-0.0002)})
	--({\sx*(1.6500)},{\sy*(-0.0002)})
	--({\sx*(1.6600)},{\sy*(-0.0002)})
	--({\sx*(1.6700)},{\sy*(-0.0003)})
	--({\sx*(1.6800)},{\sy*(-0.0003)})
	--({\sx*(1.6900)},{\sy*(-0.0003)})
	--({\sx*(1.7000)},{\sy*(-0.0004)})
	--({\sx*(1.7100)},{\sy*(-0.0004)})
	--({\sx*(1.7200)},{\sy*(-0.0005)})
	--({\sx*(1.7300)},{\sy*(-0.0005)})
	--({\sx*(1.7400)},{\sy*(-0.0006)})
	--({\sx*(1.7500)},{\sy*(-0.0006)})
	--({\sx*(1.7600)},{\sy*(-0.0007)})
	--({\sx*(1.7700)},{\sy*(-0.0008)})
	--({\sx*(1.7800)},{\sy*(-0.0008)})
	--({\sx*(1.7900)},{\sy*(-0.0009)})
	--({\sx*(1.8000)},{\sy*(-0.0010)})
	--({\sx*(1.8100)},{\sy*(-0.0010)})
	--({\sx*(1.8200)},{\sy*(-0.0011)})
	--({\sx*(1.8300)},{\sy*(-0.0012)})
	--({\sx*(1.8400)},{\sy*(-0.0013)})
	--({\sx*(1.8500)},{\sy*(-0.0014)})
	--({\sx*(1.8600)},{\sy*(-0.0015)})
	--({\sx*(1.8700)},{\sy*(-0.0015)})
	--({\sx*(1.8800)},{\sy*(-0.0016)})
	--({\sx*(1.8900)},{\sy*(-0.0017)})
	--({\sx*(1.9000)},{\sy*(-0.0018)})
	--({\sx*(1.9100)},{\sy*(-0.0019)})
	--({\sx*(1.9200)},{\sy*(-0.0020)})
	--({\sx*(1.9300)},{\sy*(-0.0021)})
	--({\sx*(1.9400)},{\sy*(-0.0023)})
	--({\sx*(1.9500)},{\sy*(-0.0024)})
	--({\sx*(1.9600)},{\sy*(-0.0025)})
	--({\sx*(1.9700)},{\sy*(-0.0026)})
	--({\sx*(1.9800)},{\sy*(-0.0027)})
	--({\sx*(1.9900)},{\sy*(-0.0028)})
	--({\sx*(2.0000)},{\sy*(-0.0029)})
	--({\sx*(2.0100)},{\sy*(-0.0030)})
	--({\sx*(2.0200)},{\sy*(-0.0031)})
	--({\sx*(2.0300)},{\sy*(-0.0033)})
	--({\sx*(2.0400)},{\sy*(-0.0034)})
	--({\sx*(2.0500)},{\sy*(-0.0035)})
	--({\sx*(2.0600)},{\sy*(-0.0036)})
	--({\sx*(2.0700)},{\sy*(-0.0037)})
	--({\sx*(2.0800)},{\sy*(-0.0038)})
	--({\sx*(2.0900)},{\sy*(-0.0039)})
	--({\sx*(2.1000)},{\sy*(-0.0040)})
	--({\sx*(2.1100)},{\sy*(-0.0041)})
	--({\sx*(2.1200)},{\sy*(-0.0042)})
	--({\sx*(2.1300)},{\sy*(-0.0043)})
	--({\sx*(2.1400)},{\sy*(-0.0044)})
	--({\sx*(2.1500)},{\sy*(-0.0045)})
	--({\sx*(2.1600)},{\sy*(-0.0046)})
	--({\sx*(2.1700)},{\sy*(-0.0047)})
	--({\sx*(2.1800)},{\sy*(-0.0047)})
	--({\sx*(2.1900)},{\sy*(-0.0048)})
	--({\sx*(2.2000)},{\sy*(-0.0049)})
	--({\sx*(2.2100)},{\sy*(-0.0049)})
	--({\sx*(2.2200)},{\sy*(-0.0050)})
	--({\sx*(2.2300)},{\sy*(-0.0050)})
	--({\sx*(2.2400)},{\sy*(-0.0050)})
	--({\sx*(2.2500)},{\sy*(-0.0050)})
	--({\sx*(2.2600)},{\sy*(-0.0050)})
	--({\sx*(2.2700)},{\sy*(-0.0050)})
	--({\sx*(2.2800)},{\sy*(-0.0050)})
	--({\sx*(2.2900)},{\sy*(-0.0050)})
	--({\sx*(2.3000)},{\sy*(-0.0049)})
	--({\sx*(2.3100)},{\sy*(-0.0049)})
	--({\sx*(2.3200)},{\sy*(-0.0048)})
	--({\sx*(2.3300)},{\sy*(-0.0047)})
	--({\sx*(2.3400)},{\sy*(-0.0046)})
	--({\sx*(2.3500)},{\sy*(-0.0045)})
	--({\sx*(2.3600)},{\sy*(-0.0044)})
	--({\sx*(2.3700)},{\sy*(-0.0042)})
	--({\sx*(2.3800)},{\sy*(-0.0040)})
	--({\sx*(2.3900)},{\sy*(-0.0038)})
	--({\sx*(2.4000)},{\sy*(-0.0036)})
	--({\sx*(2.4100)},{\sy*(-0.0034)})
	--({\sx*(2.4200)},{\sy*(-0.0031)})
	--({\sx*(2.4300)},{\sy*(-0.0028)})
	--({\sx*(2.4400)},{\sy*(-0.0025)})
	--({\sx*(2.4500)},{\sy*(-0.0021)})
	--({\sx*(2.4600)},{\sy*(-0.0018)})
	--({\sx*(2.4700)},{\sy*(-0.0014)})
	--({\sx*(2.4800)},{\sy*(-0.0009)})
	--({\sx*(2.4900)},{\sy*(-0.0005)})
	--({\sx*(2.5000)},{\sy*(0.0000)})
	--({\sx*(2.5100)},{\sy*(0.0005)})
	--({\sx*(2.5200)},{\sy*(0.0011)})
	--({\sx*(2.5300)},{\sy*(0.0017)})
	--({\sx*(2.5400)},{\sy*(0.0023)})
	--({\sx*(2.5500)},{\sy*(0.0029)})
	--({\sx*(2.5600)},{\sy*(0.0036)})
	--({\sx*(2.5700)},{\sy*(0.0043)})
	--({\sx*(2.5800)},{\sy*(0.0051)})
	--({\sx*(2.5900)},{\sy*(0.0059)})
	--({\sx*(2.6000)},{\sy*(0.0067)})
	--({\sx*(2.6100)},{\sy*(0.0076)})
	--({\sx*(2.6200)},{\sy*(0.0085)})
	--({\sx*(2.6300)},{\sy*(0.0095)})
	--({\sx*(2.6400)},{\sy*(0.0105)})
	--({\sx*(2.6500)},{\sy*(0.0115)})
	--({\sx*(2.6600)},{\sy*(0.0126)})
	--({\sx*(2.6700)},{\sy*(0.0137)})
	--({\sx*(2.6800)},{\sy*(0.0149)})
	--({\sx*(2.6900)},{\sy*(0.0161)})
	--({\sx*(2.7000)},{\sy*(0.0174)})
	--({\sx*(2.7100)},{\sy*(0.0187)})
	--({\sx*(2.7200)},{\sy*(0.0200)})
	--({\sx*(2.7300)},{\sy*(0.0214)})
	--({\sx*(2.7400)},{\sy*(0.0228)})
	--({\sx*(2.7500)},{\sy*(0.0243)})
	--({\sx*(2.7600)},{\sy*(0.0258)})
	--({\sx*(2.7700)},{\sy*(0.0274)})
	--({\sx*(2.7800)},{\sy*(0.0290)})
	--({\sx*(2.7900)},{\sy*(0.0306)})
	--({\sx*(2.8000)},{\sy*(0.0323)})
	--({\sx*(2.8100)},{\sy*(0.0340)})
	--({\sx*(2.8200)},{\sy*(0.0357)})
	--({\sx*(2.8300)},{\sy*(0.0375)})
	--({\sx*(2.8400)},{\sy*(0.0393)})
	--({\sx*(2.8500)},{\sy*(0.0412)})
	--({\sx*(2.8600)},{\sy*(0.0431)})
	--({\sx*(2.8700)},{\sy*(0.0450)})
	--({\sx*(2.8800)},{\sy*(0.0470)})
	--({\sx*(2.8900)},{\sy*(0.0489)})
	--({\sx*(2.9000)},{\sy*(0.0509)})
	--({\sx*(2.9100)},{\sy*(0.0529)})
	--({\sx*(2.9200)},{\sy*(0.0550)})
	--({\sx*(2.9300)},{\sy*(0.0570)})
	--({\sx*(2.9400)},{\sy*(0.0591)})
	--({\sx*(2.9500)},{\sy*(0.0612)})
	--({\sx*(2.9600)},{\sy*(0.0633)})
	--({\sx*(2.9700)},{\sy*(0.0654)})
	--({\sx*(2.9800)},{\sy*(0.0674)})
	--({\sx*(2.9900)},{\sy*(0.0695)})
	--({\sx*(3.0000)},{\sy*(0.0716)})
	--({\sx*(3.0100)},{\sy*(0.0736)})
	--({\sx*(3.0200)},{\sy*(0.0757)})
	--({\sx*(3.0300)},{\sy*(0.0777)})
	--({\sx*(3.0400)},{\sy*(0.0797)})
	--({\sx*(3.0500)},{\sy*(0.0816)})
	--({\sx*(3.0600)},{\sy*(0.0835)})
	--({\sx*(3.0700)},{\sy*(0.0854)})
	--({\sx*(3.0800)},{\sy*(0.0872)})
	--({\sx*(3.0900)},{\sy*(0.0889)})
	--({\sx*(3.1000)},{\sy*(0.0906)})
	--({\sx*(3.1100)},{\sy*(0.0922)})
	--({\sx*(3.1200)},{\sy*(0.0938)})
	--({\sx*(3.1300)},{\sy*(0.0952)})
	--({\sx*(3.1400)},{\sy*(0.0965)})
	--({\sx*(3.1500)},{\sy*(0.0978)})
	--({\sx*(3.1600)},{\sy*(0.0989)})
	--({\sx*(3.1700)},{\sy*(0.0999)})
	--({\sx*(3.1800)},{\sy*(0.1007)})
	--({\sx*(3.1900)},{\sy*(0.1014)})
	--({\sx*(3.2000)},{\sy*(0.1020)})
	--({\sx*(3.2100)},{\sy*(0.1024)})
	--({\sx*(3.2200)},{\sy*(0.1025)})
	--({\sx*(3.2300)},{\sy*(0.1025)})
	--({\sx*(3.2400)},{\sy*(0.1023)})
	--({\sx*(3.2500)},{\sy*(0.1019)})
	--({\sx*(3.2600)},{\sy*(0.1012)})
	--({\sx*(3.2700)},{\sy*(0.1002)})
	--({\sx*(3.2800)},{\sy*(0.0990)})
	--({\sx*(3.2900)},{\sy*(0.0975)})
	--({\sx*(3.3000)},{\sy*(0.0956)})
	--({\sx*(3.3100)},{\sy*(0.0934)})
	--({\sx*(3.3200)},{\sy*(0.0908)})
	--({\sx*(3.3300)},{\sy*(0.0879)})
	--({\sx*(3.3400)},{\sy*(0.0845)})
	--({\sx*(3.3500)},{\sy*(0.0807)})
	--({\sx*(3.3600)},{\sy*(0.0763)})
	--({\sx*(3.3700)},{\sy*(0.0715)})
	--({\sx*(3.3800)},{\sy*(0.0661)})
	--({\sx*(3.3900)},{\sy*(0.0600)})
	--({\sx*(3.4000)},{\sy*(0.0533)})
	--({\sx*(3.4100)},{\sy*(0.0459)})
	--({\sx*(3.4200)},{\sy*(0.0378)})
	--({\sx*(3.4300)},{\sy*(0.0288)})
	--({\sx*(3.4400)},{\sy*(0.0188)})
	--({\sx*(3.4500)},{\sy*(0.0079)})
	--({\sx*(3.4600)},{\sy*(-0.0041)})
	--({\sx*(3.4700)},{\sy*(-0.0173)})
	--({\sx*(3.4800)},{\sy*(-0.0318)})
	--({\sx*(3.4900)},{\sy*(-0.0477)})
	--({\sx*(3.5000)},{\sy*(-0.0653)})
	--({\sx*(3.5100)},{\sy*(-0.0846)})
	--({\sx*(3.5200)},{\sy*(-0.1059)})
	--({\sx*(3.5300)},{\sy*(-0.1293)})
	--({\sx*(3.5400)},{\sy*(-0.1553)})
	--({\sx*(3.5500)},{\sy*(-0.1841)})
	--({\sx*(3.5600)},{\sy*(-0.2160)})
	--({\sx*(3.5700)},{\sy*(-0.2516)})
	--({\sx*(3.5800)},{\sy*(-0.2914)})
	--({\sx*(3.5900)},{\sy*(-0.3360)})
	--({\sx*(3.6000)},{\sy*(-0.3862)})
	--({\sx*(3.6100)},{\sy*(-0.4430)})
	--({\sx*(3.6200)},{\sy*(-0.5077)})
	--({\sx*(3.6300)},{\sy*(-0.5818)})
	--({\sx*(3.6400)},{\sy*(-0.6673)})
	--({\sx*(3.6500)},{\sy*(-0.7668)})
	--({\sx*(3.6600)},{\sy*(-0.8840)})
	--({\sx*(3.6700)},{\sy*(-1.0234)})
	--({\sx*(3.6800)},{\sy*(-1.1920)})
	--({\sx*(3.6900)},{\sy*(-1.3992)})
	--({\sx*(3.7000)},{\sy*(-1.6595)})
	--({\sx*(3.7100)},{\sy*(-1.9958)})
	--({\sx*(3.7200)},{\sy*(-2.4459)})
	--({\sx*(3.7300)},{\sy*(-3.0779)})
	--({\sx*(3.7400)},{\sy*(-4.0277)})
	--({\sx*(3.7500)},{\sy*(-5.6116)})
	--({\sx*(3.7600)},{\sy*(-8.7739)})
	--({\sx*(3.7700)},{\sy*(-18.1824)})
	--({\sx*(3.7800)},{\sy*(-1326.7317)})
	--({\sx*(3.7900)},{\sy*(20.3193)})
	--({\sx*(3.8000)},{\sy*(10.4925)})
	--({\sx*(3.8100)},{\sy*(7.2530)})
	--({\sx*(3.8200)},{\sy*(5.6425)})
	--({\sx*(3.8300)},{\sy*(4.6807)})
	--({\sx*(3.8400)},{\sy*(4.0425)})
	--({\sx*(3.8500)},{\sy*(3.5891)})
	--({\sx*(3.8600)},{\sy*(3.2511)})
	--({\sx*(3.8700)},{\sy*(2.9899)})
	--({\sx*(3.8800)},{\sy*(2.7827)})
	--({\sx*(3.8900)},{\sy*(2.6147)})
	--({\sx*(3.9000)},{\sy*(2.4761)})
	--({\sx*(3.9100)},{\sy*(2.3603)})
	--({\sx*(3.9200)},{\sy*(2.2624)})
	--({\sx*(3.9300)},{\sy*(2.1790)})
	--({\sx*(3.9400)},{\sy*(2.1073)})
	--({\sx*(3.9500)},{\sy*(2.0455)})
	--({\sx*(3.9600)},{\sy*(1.9920)})
	--({\sx*(3.9700)},{\sy*(1.9455)})
	--({\sx*(3.9800)},{\sy*(1.9053)})
	--({\sx*(3.9900)},{\sy*(1.8704)})
	--({\sx*(4.0000)},{\sy*(1.8404)})
	--({\sx*(4.0100)},{\sy*(1.8148)})
	--({\sx*(4.0200)},{\sy*(1.7933)})
	--({\sx*(4.0300)},{\sy*(1.7757)})
	--({\sx*(4.0400)},{\sy*(1.7617)})
	--({\sx*(4.0500)},{\sy*(1.7514)})
	--({\sx*(4.0600)},{\sy*(1.7447)})
	--({\sx*(4.0700)},{\sy*(1.7418)})
	--({\sx*(4.0800)},{\sy*(1.7430)})
	--({\sx*(4.0900)},{\sy*(1.7486)})
	--({\sx*(4.1000)},{\sy*(1.7591)})
	--({\sx*(4.1100)},{\sy*(1.7756)})
	--({\sx*(4.1200)},{\sy*(1.7991)})
	--({\sx*(4.1300)},{\sy*(1.8314)})
	--({\sx*(4.1400)},{\sy*(1.8752)})
	--({\sx*(4.1500)},{\sy*(1.9344)})
	--({\sx*(4.1600)},{\sy*(2.0156)})
	--({\sx*(4.1700)},{\sy*(2.1296)})
	--({\sx*(4.1800)},{\sy*(2.2970)})
	--({\sx*(4.1900)},{\sy*(2.5596)})
	--({\sx*(4.2000)},{\sy*(3.0207)})
	--({\sx*(4.2100)},{\sy*(4.0196)})
	--({\sx*(4.2200)},{\sy*(7.7033)})
	--({\sx*(4.2300)},{\sy*(-18.2378)})
	--({\sx*(4.2400)},{\sy*(-2.6602)})
	--({\sx*(4.2500)},{\sy*(-0.9403)})
	--({\sx*(4.2600)},{\sy*(-0.2831)})
	--({\sx*(4.2700)},{\sy*(0.0619)})
	--({\sx*(4.2800)},{\sy*(0.2733)})
	--({\sx*(4.2900)},{\sy*(0.4154)})
	--({\sx*(4.3000)},{\sy*(0.5169)})
	--({\sx*(4.3100)},{\sy*(0.5927)})
	--({\sx*(4.3200)},{\sy*(0.6512)})
	--({\sx*(4.3300)},{\sy*(0.6974)})
	--({\sx*(4.3400)},{\sy*(0.7348)})
	--({\sx*(4.3500)},{\sy*(0.7655)})
	--({\sx*(4.3600)},{\sy*(0.7910)})
	--({\sx*(4.3700)},{\sy*(0.8126)})
	--({\sx*(4.3800)},{\sy*(0.8309)})
	--({\sx*(4.3900)},{\sy*(0.8466)})
	--({\sx*(4.4000)},{\sy*(0.8601)})
	--({\sx*(4.4100)},{\sy*(0.8720)})
	--({\sx*(4.4200)},{\sy*(0.8823)})
	--({\sx*(4.4300)},{\sy*(0.8914)})
	--({\sx*(4.4400)},{\sy*(0.8994)})
	--({\sx*(4.4500)},{\sy*(0.9065)})
	--({\sx*(4.4600)},{\sy*(0.9128)})
	--({\sx*(4.4700)},{\sy*(0.9185)})
	--({\sx*(4.4800)},{\sy*(0.9235)})
	--({\sx*(4.4900)},{\sy*(0.9280)})
	--({\sx*(4.5000)},{\sy*(0.9321)})
	--({\sx*(4.5100)},{\sy*(0.9357)})
	--({\sx*(4.5200)},{\sy*(0.9390)})
	--({\sx*(4.5300)},{\sy*(0.9419)})
	--({\sx*(4.5400)},{\sy*(0.9446)})
	--({\sx*(4.5500)},{\sy*(0.9469)})
	--({\sx*(4.5600)},{\sy*(0.9490)})
	--({\sx*(4.5700)},{\sy*(0.9509)})
	--({\sx*(4.5800)},{\sy*(0.9525)})
	--({\sx*(4.5900)},{\sy*(0.9540)})
	--({\sx*(4.6000)},{\sy*(0.9552)})
	--({\sx*(4.6100)},{\sy*(0.9563)})
	--({\sx*(4.6200)},{\sy*(0.9571)})
	--({\sx*(4.6300)},{\sy*(0.9578)})
	--({\sx*(4.6400)},{\sy*(0.9583)})
	--({\sx*(4.6500)},{\sy*(0.9586)})
	--({\sx*(4.6600)},{\sy*(0.9586)})
	--({\sx*(4.6700)},{\sy*(0.9584)})
	--({\sx*(4.6800)},{\sy*(0.9580)})
	--({\sx*(4.6900)},{\sy*(0.9572)})
	--({\sx*(4.7000)},{\sy*(0.9561)})
	--({\sx*(4.7100)},{\sy*(0.9545)})
	--({\sx*(4.7200)},{\sy*(0.9523)})
	--({\sx*(4.7300)},{\sy*(0.9493)})
	--({\sx*(4.7400)},{\sy*(0.9452)})
	--({\sx*(4.7500)},{\sy*(0.9396)})
	--({\sx*(4.7600)},{\sy*(0.9315)})
	--({\sx*(4.7700)},{\sy*(0.9192)})
	--({\sx*(4.7800)},{\sy*(0.8990)})
	--({\sx*(4.7900)},{\sy*(0.8603)})
	--({\sx*(4.8000)},{\sy*(0.7592)})
	--({\sx*(4.8100)},{\sy*(-0.1130)})
	--({\sx*(4.8200)},{\sy*(1.3863)})
	--({\sx*(4.8300)},{\sy*(1.1591)})
	--({\sx*(4.8400)},{\sy*(1.0984)})
	--({\sx*(4.8500)},{\sy*(1.0706)})
	--({\sx*(4.8600)},{\sy*(1.0547)})
	--({\sx*(4.8700)},{\sy*(1.0447)})
	--({\sx*(4.8800)},{\sy*(1.0379)})
	--({\sx*(4.8900)},{\sy*(1.0331)})
	--({\sx*(4.9000)},{\sy*(1.0296)})
	--({\sx*(4.9100)},{\sy*(1.0272)})
	--({\sx*(4.9200)},{\sy*(1.0255)})
	--({\sx*(4.9300)},{\sy*(1.0245)})
	--({\sx*(4.9400)},{\sy*(1.0243)})
	--({\sx*(4.9500)},{\sy*(1.0249)})
	--({\sx*(4.9600)},{\sy*(1.0268)})
	--({\sx*(4.9700)},{\sy*(1.0309)})
	--({\sx*(4.9800)},{\sy*(1.0405)})
	--({\sx*(4.9900)},{\sy*(1.0727)})
	--({\sx*(5.0000)},{\sy*(0.0000)});
}
\def\xwertee{
\fill[color=red] (0.0000,0) circle[radius={0.07/\skala}];
\fill[color=white] (0.0000,0) circle[radius={0.05/\skala}];
\fill[color=red] (0.1224,0) circle[radius={0.07/\skala}];
\fill[color=white] (0.1224,0) circle[radius={0.05/\skala}];
\fill[color=red] (0.4775,0) circle[radius={0.07/\skala}];
\fill[color=white] (0.4775,0) circle[radius={0.05/\skala}];
\fill[color=red] (1.0305,0) circle[radius={0.07/\skala}];
\fill[color=white] (1.0305,0) circle[radius={0.05/\skala}];
\fill[color=red] (1.7275,0) circle[radius={0.07/\skala}];
\fill[color=white] (1.7275,0) circle[radius={0.05/\skala}];
\fill[color=red] (2.5000,0) circle[radius={0.07/\skala}];
\fill[color=white] (2.5000,0) circle[radius={0.05/\skala}];
\fill[color=red] (3.2725,0) circle[radius={0.07/\skala}];
\fill[color=white] (3.2725,0) circle[radius={0.05/\skala}];
\fill[color=red] (3.9695,0) circle[radius={0.07/\skala}];
\fill[color=white] (3.9695,0) circle[radius={0.05/\skala}];
\fill[color=red] (4.5225,0) circle[radius={0.07/\skala}];
\fill[color=white] (4.5225,0) circle[radius={0.05/\skala}];
\fill[color=red] (4.8776,0) circle[radius={0.07/\skala}];
\fill[color=white] (4.8776,0) circle[radius={0.05/\skala}];
\fill[color=red] (5.0000,0) circle[radius={0.07/\skala}];
\fill[color=white] (5.0000,0) circle[radius={0.05/\skala}];
}
\def\punktee{10}
\def\maxfehlere{1.003\cdot 10^{-4}}
\def\fehlere{
\draw[color=red,line width=1.4pt,line join=round] ({\sx*(0.000)},{\sy*(0.0000)})
	--({\sx*(0.0100)},{\sy*(-0.0313)})
	--({\sx*(0.0200)},{\sy*(-0.0546)})
	--({\sx*(0.0300)},{\sy*(-0.0706)})
	--({\sx*(0.0400)},{\sy*(-0.0801)})
	--({\sx*(0.0500)},{\sy*(-0.0840)})
	--({\sx*(0.0600)},{\sy*(-0.0828)})
	--({\sx*(0.0700)},{\sy*(-0.0773)})
	--({\sx*(0.0800)},{\sy*(-0.0680)})
	--({\sx*(0.0900)},{\sy*(-0.0556)})
	--({\sx*(0.1000)},{\sy*(-0.0405)})
	--({\sx*(0.1100)},{\sy*(-0.0234)})
	--({\sx*(0.1200)},{\sy*(-0.0046)})
	--({\sx*(0.1300)},{\sy*(0.0153)})
	--({\sx*(0.1400)},{\sy*(0.0361)})
	--({\sx*(0.1500)},{\sy*(0.0573)})
	--({\sx*(0.1600)},{\sy*(0.0785)})
	--({\sx*(0.1700)},{\sy*(0.0996)})
	--({\sx*(0.1800)},{\sy*(0.1201)})
	--({\sx*(0.1900)},{\sy*(0.1399)})
	--({\sx*(0.2000)},{\sy*(0.1586)})
	--({\sx*(0.2100)},{\sy*(0.1762)})
	--({\sx*(0.2200)},{\sy*(0.1924)})
	--({\sx*(0.2300)},{\sy*(0.2071)})
	--({\sx*(0.2400)},{\sy*(0.2201)})
	--({\sx*(0.2500)},{\sy*(0.2314)})
	--({\sx*(0.2600)},{\sy*(0.2409)})
	--({\sx*(0.2700)},{\sy*(0.2484)})
	--({\sx*(0.2800)},{\sy*(0.2540)})
	--({\sx*(0.2900)},{\sy*(0.2575)})
	--({\sx*(0.3000)},{\sy*(0.2591)})
	--({\sx*(0.3100)},{\sy*(0.2586)})
	--({\sx*(0.3200)},{\sy*(0.2561)})
	--({\sx*(0.3300)},{\sy*(0.2517)})
	--({\sx*(0.3400)},{\sy*(0.2453)})
	--({\sx*(0.3500)},{\sy*(0.2370)})
	--({\sx*(0.3600)},{\sy*(0.2268)})
	--({\sx*(0.3700)},{\sy*(0.2149)})
	--({\sx*(0.3800)},{\sy*(0.2013)})
	--({\sx*(0.3900)},{\sy*(0.1861)})
	--({\sx*(0.4000)},{\sy*(0.1694)})
	--({\sx*(0.4100)},{\sy*(0.1512)})
	--({\sx*(0.4200)},{\sy*(0.1318)})
	--({\sx*(0.4300)},{\sy*(0.1111)})
	--({\sx*(0.4400)},{\sy*(0.0893)})
	--({\sx*(0.4500)},{\sy*(0.0666)})
	--({\sx*(0.4600)},{\sy*(0.0429)})
	--({\sx*(0.4700)},{\sy*(0.0186)})
	--({\sx*(0.4800)},{\sy*(-0.0064)})
	--({\sx*(0.4900)},{\sy*(-0.0319)})
	--({\sx*(0.5000)},{\sy*(-0.0578)})
	--({\sx*(0.5100)},{\sy*(-0.0839)})
	--({\sx*(0.5200)},{\sy*(-0.1101)})
	--({\sx*(0.5300)},{\sy*(-0.1364)})
	--({\sx*(0.5400)},{\sy*(-0.1626)})
	--({\sx*(0.5500)},{\sy*(-0.1885)})
	--({\sx*(0.5600)},{\sy*(-0.2142)})
	--({\sx*(0.5700)},{\sy*(-0.2393)})
	--({\sx*(0.5800)},{\sy*(-0.2640)})
	--({\sx*(0.5900)},{\sy*(-0.2879)})
	--({\sx*(0.6000)},{\sy*(-0.3112)})
	--({\sx*(0.6100)},{\sy*(-0.3335)})
	--({\sx*(0.6200)},{\sy*(-0.3550)})
	--({\sx*(0.6300)},{\sy*(-0.3754)})
	--({\sx*(0.6400)},{\sy*(-0.3947)})
	--({\sx*(0.6500)},{\sy*(-0.4128)})
	--({\sx*(0.6600)},{\sy*(-0.4297)})
	--({\sx*(0.6700)},{\sy*(-0.4452)})
	--({\sx*(0.6800)},{\sy*(-0.4593)})
	--({\sx*(0.6900)},{\sy*(-0.4720)})
	--({\sx*(0.7000)},{\sy*(-0.4832)})
	--({\sx*(0.7100)},{\sy*(-0.4929)})
	--({\sx*(0.7200)},{\sy*(-0.5010)})
	--({\sx*(0.7300)},{\sy*(-0.5075)})
	--({\sx*(0.7400)},{\sy*(-0.5124)})
	--({\sx*(0.7500)},{\sy*(-0.5156)})
	--({\sx*(0.7600)},{\sy*(-0.5171)})
	--({\sx*(0.7700)},{\sy*(-0.5169)})
	--({\sx*(0.7800)},{\sy*(-0.5150)})
	--({\sx*(0.7900)},{\sy*(-0.5115)})
	--({\sx*(0.8000)},{\sy*(-0.5063)})
	--({\sx*(0.8100)},{\sy*(-0.4994)})
	--({\sx*(0.8200)},{\sy*(-0.4908)})
	--({\sx*(0.8300)},{\sy*(-0.4806)})
	--({\sx*(0.8400)},{\sy*(-0.4688)})
	--({\sx*(0.8500)},{\sy*(-0.4554)})
	--({\sx*(0.8600)},{\sy*(-0.4404)})
	--({\sx*(0.8700)},{\sy*(-0.4240)})
	--({\sx*(0.8800)},{\sy*(-0.4060)})
	--({\sx*(0.8900)},{\sy*(-0.3867)})
	--({\sx*(0.9000)},{\sy*(-0.3660)})
	--({\sx*(0.9100)},{\sy*(-0.3440)})
	--({\sx*(0.9200)},{\sy*(-0.3207)})
	--({\sx*(0.9300)},{\sy*(-0.2963)})
	--({\sx*(0.9400)},{\sy*(-0.2707)})
	--({\sx*(0.9500)},{\sy*(-0.2441)})
	--({\sx*(0.9600)},{\sy*(-0.2164)})
	--({\sx*(0.9700)},{\sy*(-0.1879)})
	--({\sx*(0.9800)},{\sy*(-0.1585)})
	--({\sx*(0.9900)},{\sy*(-0.1284)})
	--({\sx*(1.0000)},{\sy*(-0.0975)})
	--({\sx*(1.0100)},{\sy*(-0.0661)})
	--({\sx*(1.0200)},{\sy*(-0.0341)})
	--({\sx*(1.0300)},{\sy*(-0.0018)})
	--({\sx*(1.0400)},{\sy*(0.0310)})
	--({\sx*(1.0500)},{\sy*(0.0640)})
	--({\sx*(1.0600)},{\sy*(0.0973)})
	--({\sx*(1.0700)},{\sy*(0.1306)})
	--({\sx*(1.0800)},{\sy*(0.1639)})
	--({\sx*(1.0900)},{\sy*(0.1972)})
	--({\sx*(1.1000)},{\sy*(0.2304)})
	--({\sx*(1.1100)},{\sy*(0.2633)})
	--({\sx*(1.1200)},{\sy*(0.2959)})
	--({\sx*(1.1300)},{\sy*(0.3281)})
	--({\sx*(1.1400)},{\sy*(0.3599)})
	--({\sx*(1.1500)},{\sy*(0.3911)})
	--({\sx*(1.1600)},{\sy*(0.4216)})
	--({\sx*(1.1700)},{\sy*(0.4515)})
	--({\sx*(1.1800)},{\sy*(0.4806)})
	--({\sx*(1.1900)},{\sy*(0.5088)})
	--({\sx*(1.2000)},{\sy*(0.5362)})
	--({\sx*(1.2100)},{\sy*(0.5625)})
	--({\sx*(1.2200)},{\sy*(0.5878)})
	--({\sx*(1.2300)},{\sy*(0.6120)})
	--({\sx*(1.2400)},{\sy*(0.6350)})
	--({\sx*(1.2500)},{\sy*(0.6568)})
	--({\sx*(1.2600)},{\sy*(0.6773)})
	--({\sx*(1.2700)},{\sy*(0.6965)})
	--({\sx*(1.2800)},{\sy*(0.7143)})
	--({\sx*(1.2900)},{\sy*(0.7307)})
	--({\sx*(1.3000)},{\sy*(0.7457)})
	--({\sx*(1.3100)},{\sy*(0.7591)})
	--({\sx*(1.3200)},{\sy*(0.7710)})
	--({\sx*(1.3300)},{\sy*(0.7814)})
	--({\sx*(1.3400)},{\sy*(0.7902)})
	--({\sx*(1.3500)},{\sy*(0.7973)})
	--({\sx*(1.3600)},{\sy*(0.8029)})
	--({\sx*(1.3700)},{\sy*(0.8067)})
	--({\sx*(1.3800)},{\sy*(0.8090)})
	--({\sx*(1.3900)},{\sy*(0.8095)})
	--({\sx*(1.4000)},{\sy*(0.8084)})
	--({\sx*(1.4100)},{\sy*(0.8056)})
	--({\sx*(1.4200)},{\sy*(0.8012)})
	--({\sx*(1.4300)},{\sy*(0.7950)})
	--({\sx*(1.4400)},{\sy*(0.7872)})
	--({\sx*(1.4500)},{\sy*(0.7778)})
	--({\sx*(1.4600)},{\sy*(0.7668)})
	--({\sx*(1.4700)},{\sy*(0.7541)})
	--({\sx*(1.4800)},{\sy*(0.7398)})
	--({\sx*(1.4900)},{\sy*(0.7240)})
	--({\sx*(1.5000)},{\sy*(0.7066)})
	--({\sx*(1.5100)},{\sy*(0.6878)})
	--({\sx*(1.5200)},{\sy*(0.6674)})
	--({\sx*(1.5300)},{\sy*(0.6457)})
	--({\sx*(1.5400)},{\sy*(0.6225)})
	--({\sx*(1.5500)},{\sy*(0.5980)})
	--({\sx*(1.5600)},{\sy*(0.5722)})
	--({\sx*(1.5700)},{\sy*(0.5452)})
	--({\sx*(1.5800)},{\sy*(0.5169)})
	--({\sx*(1.5900)},{\sy*(0.4876)})
	--({\sx*(1.6000)},{\sy*(0.4571)})
	--({\sx*(1.6100)},{\sy*(0.4256)})
	--({\sx*(1.6200)},{\sy*(0.3931)})
	--({\sx*(1.6300)},{\sy*(0.3597)})
	--({\sx*(1.6400)},{\sy*(0.3254)})
	--({\sx*(1.6500)},{\sy*(0.2904)})
	--({\sx*(1.6600)},{\sy*(0.2546)})
	--({\sx*(1.6700)},{\sy*(0.2183)})
	--({\sx*(1.6800)},{\sy*(0.1813)})
	--({\sx*(1.6900)},{\sy*(0.1438)})
	--({\sx*(1.7000)},{\sy*(0.1058)})
	--({\sx*(1.7100)},{\sy*(0.0675)})
	--({\sx*(1.7200)},{\sy*(0.0289)})
	--({\sx*(1.7300)},{\sy*(-0.0099)})
	--({\sx*(1.7400)},{\sy*(-0.0489)})
	--({\sx*(1.7500)},{\sy*(-0.0879)})
	--({\sx*(1.7600)},{\sy*(-0.1270)})
	--({\sx*(1.7700)},{\sy*(-0.1660)})
	--({\sx*(1.7800)},{\sy*(-0.2049)})
	--({\sx*(1.7900)},{\sy*(-0.2435)})
	--({\sx*(1.8000)},{\sy*(-0.2819)})
	--({\sx*(1.8100)},{\sy*(-0.3199)})
	--({\sx*(1.8200)},{\sy*(-0.3575)})
	--({\sx*(1.8300)},{\sy*(-0.3946)})
	--({\sx*(1.8400)},{\sy*(-0.4311)})
	--({\sx*(1.8500)},{\sy*(-0.4670)})
	--({\sx*(1.8600)},{\sy*(-0.5022)})
	--({\sx*(1.8700)},{\sy*(-0.5367)})
	--({\sx*(1.8800)},{\sy*(-0.5703)})
	--({\sx*(1.8900)},{\sy*(-0.6030)})
	--({\sx*(1.9000)},{\sy*(-0.6348)})
	--({\sx*(1.9100)},{\sy*(-0.6656)})
	--({\sx*(1.9200)},{\sy*(-0.6953)})
	--({\sx*(1.9300)},{\sy*(-0.7239)})
	--({\sx*(1.9400)},{\sy*(-0.7513)})
	--({\sx*(1.9500)},{\sy*(-0.7775)})
	--({\sx*(1.9600)},{\sy*(-0.8025)})
	--({\sx*(1.9700)},{\sy*(-0.8261)})
	--({\sx*(1.9800)},{\sy*(-0.8484)})
	--({\sx*(1.9900)},{\sy*(-0.8693)})
	--({\sx*(2.0000)},{\sy*(-0.8887)})
	--({\sx*(2.0100)},{\sy*(-0.9067)})
	--({\sx*(2.0200)},{\sy*(-0.9232)})
	--({\sx*(2.0300)},{\sy*(-0.9381)})
	--({\sx*(2.0400)},{\sy*(-0.9515)})
	--({\sx*(2.0500)},{\sy*(-0.9633)})
	--({\sx*(2.0600)},{\sy*(-0.9735)})
	--({\sx*(2.0700)},{\sy*(-0.9820)})
	--({\sx*(2.0800)},{\sy*(-0.9889)})
	--({\sx*(2.0900)},{\sy*(-0.9942)})
	--({\sx*(2.1000)},{\sy*(-0.9978)})
	--({\sx*(2.1100)},{\sy*(-0.9997)})
	--({\sx*(2.1200)},{\sy*(-1.0000)})
	--({\sx*(2.1300)},{\sy*(-0.9986)})
	--({\sx*(2.1400)},{\sy*(-0.9955)})
	--({\sx*(2.1500)},{\sy*(-0.9907)})
	--({\sx*(2.1600)},{\sy*(-0.9843)})
	--({\sx*(2.1700)},{\sy*(-0.9762)})
	--({\sx*(2.1800)},{\sy*(-0.9665)})
	--({\sx*(2.1900)},{\sy*(-0.9551)})
	--({\sx*(2.2000)},{\sy*(-0.9422)})
	--({\sx*(2.2100)},{\sy*(-0.9276)})
	--({\sx*(2.2200)},{\sy*(-0.9115)})
	--({\sx*(2.2300)},{\sy*(-0.8939)})
	--({\sx*(2.2400)},{\sy*(-0.8748)})
	--({\sx*(2.2500)},{\sy*(-0.8542)})
	--({\sx*(2.2600)},{\sy*(-0.8321)})
	--({\sx*(2.2700)},{\sy*(-0.8087)})
	--({\sx*(2.2800)},{\sy*(-0.7839)})
	--({\sx*(2.2900)},{\sy*(-0.7578)})
	--({\sx*(2.3000)},{\sy*(-0.7304)})
	--({\sx*(2.3100)},{\sy*(-0.7018)})
	--({\sx*(2.3200)},{\sy*(-0.6721)})
	--({\sx*(2.3300)},{\sy*(-0.6412)})
	--({\sx*(2.3400)},{\sy*(-0.6092)})
	--({\sx*(2.3500)},{\sy*(-0.5762)})
	--({\sx*(2.3600)},{\sy*(-0.5422)})
	--({\sx*(2.3700)},{\sy*(-0.5073)})
	--({\sx*(2.3800)},{\sy*(-0.4716)})
	--({\sx*(2.3900)},{\sy*(-0.4351)})
	--({\sx*(2.4000)},{\sy*(-0.3979)})
	--({\sx*(2.4100)},{\sy*(-0.3600)})
	--({\sx*(2.4200)},{\sy*(-0.3215)})
	--({\sx*(2.4300)},{\sy*(-0.2825)})
	--({\sx*(2.4400)},{\sy*(-0.2430)})
	--({\sx*(2.4500)},{\sy*(-0.2031)})
	--({\sx*(2.4600)},{\sy*(-0.1629)})
	--({\sx*(2.4700)},{\sy*(-0.1224)})
	--({\sx*(2.4800)},{\sy*(-0.0817)})
	--({\sx*(2.4900)},{\sy*(-0.0409)})
	--({\sx*(2.5000)},{\sy*(0.0000)})
	--({\sx*(2.5100)},{\sy*(0.0409)})
	--({\sx*(2.5200)},{\sy*(0.0817)})
	--({\sx*(2.5300)},{\sy*(0.1223)})
	--({\sx*(2.5400)},{\sy*(0.1628)})
	--({\sx*(2.5500)},{\sy*(0.2029)})
	--({\sx*(2.5600)},{\sy*(0.2427)})
	--({\sx*(2.5700)},{\sy*(0.2821)})
	--({\sx*(2.5800)},{\sy*(0.3210)})
	--({\sx*(2.5900)},{\sy*(0.3594)})
	--({\sx*(2.6000)},{\sy*(0.3972)})
	--({\sx*(2.6100)},{\sy*(0.4342)})
	--({\sx*(2.6200)},{\sy*(0.4706)})
	--({\sx*(2.6300)},{\sy*(0.5061)})
	--({\sx*(2.6400)},{\sy*(0.5408)})
	--({\sx*(2.6500)},{\sy*(0.5746)})
	--({\sx*(2.6600)},{\sy*(0.6074)})
	--({\sx*(2.6700)},{\sy*(0.6392)})
	--({\sx*(2.6800)},{\sy*(0.6699)})
	--({\sx*(2.6900)},{\sy*(0.6995)})
	--({\sx*(2.7000)},{\sy*(0.7279)})
	--({\sx*(2.7100)},{\sy*(0.7550)})
	--({\sx*(2.7200)},{\sy*(0.7809)})
	--({\sx*(2.7300)},{\sy*(0.8055)})
	--({\sx*(2.7400)},{\sy*(0.8288)})
	--({\sx*(2.7500)},{\sy*(0.8506)})
	--({\sx*(2.7600)},{\sy*(0.8710)})
	--({\sx*(2.7700)},{\sy*(0.8900)})
	--({\sx*(2.7800)},{\sy*(0.9074)})
	--({\sx*(2.7900)},{\sy*(0.9233)})
	--({\sx*(2.8000)},{\sy*(0.9377)})
	--({\sx*(2.8100)},{\sy*(0.9505)})
	--({\sx*(2.8200)},{\sy*(0.9617)})
	--({\sx*(2.8300)},{\sy*(0.9713)})
	--({\sx*(2.8400)},{\sy*(0.9793)})
	--({\sx*(2.8500)},{\sy*(0.9857)})
	--({\sx*(2.8600)},{\sy*(0.9904)})
	--({\sx*(2.8700)},{\sy*(0.9934)})
	--({\sx*(2.8800)},{\sy*(0.9948)})
	--({\sx*(2.8900)},{\sy*(0.9945)})
	--({\sx*(2.9000)},{\sy*(0.9925)})
	--({\sx*(2.9100)},{\sy*(0.9890)})
	--({\sx*(2.9200)},{\sy*(0.9837)})
	--({\sx*(2.9300)},{\sy*(0.9768)})
	--({\sx*(2.9400)},{\sy*(0.9683)})
	--({\sx*(2.9500)},{\sy*(0.9582)})
	--({\sx*(2.9600)},{\sy*(0.9465)})
	--({\sx*(2.9700)},{\sy*(0.9332)})
	--({\sx*(2.9800)},{\sy*(0.9184)})
	--({\sx*(2.9900)},{\sy*(0.9021)})
	--({\sx*(3.0000)},{\sy*(0.8843)})
	--({\sx*(3.0100)},{\sy*(0.8650)})
	--({\sx*(3.0200)},{\sy*(0.8443)})
	--({\sx*(3.0300)},{\sy*(0.8222)})
	--({\sx*(3.0400)},{\sy*(0.7988)})
	--({\sx*(3.0500)},{\sy*(0.7741)})
	--({\sx*(3.0600)},{\sy*(0.7481)})
	--({\sx*(3.0700)},{\sy*(0.7209)})
	--({\sx*(3.0800)},{\sy*(0.6925)})
	--({\sx*(3.0900)},{\sy*(0.6630)})
	--({\sx*(3.1000)},{\sy*(0.6325)})
	--({\sx*(3.1100)},{\sy*(0.6010)})
	--({\sx*(3.1200)},{\sy*(0.5685)})
	--({\sx*(3.1300)},{\sy*(0.5351)})
	--({\sx*(3.1400)},{\sy*(0.5009)})
	--({\sx*(3.1500)},{\sy*(0.4659)})
	--({\sx*(3.1600)},{\sy*(0.4302)})
	--({\sx*(3.1700)},{\sy*(0.3938)})
	--({\sx*(3.1800)},{\sy*(0.3569)})
	--({\sx*(3.1900)},{\sy*(0.3195)})
	--({\sx*(3.2000)},{\sy*(0.2816)})
	--({\sx*(3.2100)},{\sy*(0.2434)})
	--({\sx*(3.2200)},{\sy*(0.2048)})
	--({\sx*(3.2300)},{\sy*(0.1660)})
	--({\sx*(3.2400)},{\sy*(0.1271)})
	--({\sx*(3.2500)},{\sy*(0.0880)})
	--({\sx*(3.2600)},{\sy*(0.0489)})
	--({\sx*(3.2700)},{\sy*(0.0099)})
	--({\sx*(3.2800)},{\sy*(-0.0290)})
	--({\sx*(3.2900)},{\sy*(-0.0677)})
	--({\sx*(3.3000)},{\sy*(-0.1062)})
	--({\sx*(3.3100)},{\sy*(-0.1443)})
	--({\sx*(3.3200)},{\sy*(-0.1821)})
	--({\sx*(3.3300)},{\sy*(-0.2193)})
	--({\sx*(3.3400)},{\sy*(-0.2560)})
	--({\sx*(3.3500)},{\sy*(-0.2922)})
	--({\sx*(3.3600)},{\sy*(-0.3276)})
	--({\sx*(3.3700)},{\sy*(-0.3623)})
	--({\sx*(3.3800)},{\sy*(-0.3962)})
	--({\sx*(3.3900)},{\sy*(-0.4292)})
	--({\sx*(3.4000)},{\sy*(-0.4613)})
	--({\sx*(3.4100)},{\sy*(-0.4924)})
	--({\sx*(3.4200)},{\sy*(-0.5225)})
	--({\sx*(3.4300)},{\sy*(-0.5514)})
	--({\sx*(3.4400)},{\sy*(-0.5792)})
	--({\sx*(3.4500)},{\sy*(-0.6058)})
	--({\sx*(3.4600)},{\sy*(-0.6311)})
	--({\sx*(3.4700)},{\sy*(-0.6551)})
	--({\sx*(3.4800)},{\sy*(-0.6777)})
	--({\sx*(3.4900)},{\sy*(-0.6990)})
	--({\sx*(3.5000)},{\sy*(-0.7188)})
	--({\sx*(3.5100)},{\sy*(-0.7371)})
	--({\sx*(3.5200)},{\sy*(-0.7539)})
	--({\sx*(3.5300)},{\sy*(-0.7692)})
	--({\sx*(3.5400)},{\sy*(-0.7829)})
	--({\sx*(3.5500)},{\sy*(-0.7949)})
	--({\sx*(3.5600)},{\sy*(-0.8054)})
	--({\sx*(3.5700)},{\sy*(-0.8142)})
	--({\sx*(3.5800)},{\sy*(-0.8213)})
	--({\sx*(3.5900)},{\sy*(-0.8268)})
	--({\sx*(3.6000)},{\sy*(-0.8306)})
	--({\sx*(3.6100)},{\sy*(-0.8327)})
	--({\sx*(3.6200)},{\sy*(-0.8330)})
	--({\sx*(3.6300)},{\sy*(-0.8317)})
	--({\sx*(3.6400)},{\sy*(-0.8287)})
	--({\sx*(3.6500)},{\sy*(-0.8240)})
	--({\sx*(3.6600)},{\sy*(-0.8176)})
	--({\sx*(3.6700)},{\sy*(-0.8096)})
	--({\sx*(3.6800)},{\sy*(-0.7999)})
	--({\sx*(3.6900)},{\sy*(-0.7886)})
	--({\sx*(3.7000)},{\sy*(-0.7756)})
	--({\sx*(3.7100)},{\sy*(-0.7611)})
	--({\sx*(3.7200)},{\sy*(-0.7451)})
	--({\sx*(3.7300)},{\sy*(-0.7275)})
	--({\sx*(3.7400)},{\sy*(-0.7085)})
	--({\sx*(3.7500)},{\sy*(-0.6881)})
	--({\sx*(3.7600)},{\sy*(-0.6662)})
	--({\sx*(3.7700)},{\sy*(-0.6431)})
	--({\sx*(3.7800)},{\sy*(-0.6186)})
	--({\sx*(3.7900)},{\sy*(-0.5929)})
	--({\sx*(3.8000)},{\sy*(-0.5661)})
	--({\sx*(3.8100)},{\sy*(-0.5381)})
	--({\sx*(3.8200)},{\sy*(-0.5091)})
	--({\sx*(3.8300)},{\sy*(-0.4791)})
	--({\sx*(3.8400)},{\sy*(-0.4482)})
	--({\sx*(3.8500)},{\sy*(-0.4164)})
	--({\sx*(3.8600)},{\sy*(-0.3839)})
	--({\sx*(3.8700)},{\sy*(-0.3506)})
	--({\sx*(3.8800)},{\sy*(-0.3168)})
	--({\sx*(3.8900)},{\sy*(-0.2824)})
	--({\sx*(3.9000)},{\sy*(-0.2476)})
	--({\sx*(3.9100)},{\sy*(-0.2124)})
	--({\sx*(3.9200)},{\sy*(-0.1769)})
	--({\sx*(3.9300)},{\sy*(-0.1412)})
	--({\sx*(3.9400)},{\sy*(-0.1054)})
	--({\sx*(3.9500)},{\sy*(-0.0695)})
	--({\sx*(3.9600)},{\sy*(-0.0337)})
	--({\sx*(3.9700)},{\sy*(0.0019)})
	--({\sx*(3.9800)},{\sy*(0.0373)})
	--({\sx*(3.9900)},{\sy*(0.0724)})
	--({\sx*(4.0000)},{\sy*(0.1070)})
	--({\sx*(4.0100)},{\sy*(0.1411)})
	--({\sx*(4.0200)},{\sy*(0.1747)})
	--({\sx*(4.0300)},{\sy*(0.2075)})
	--({\sx*(4.0400)},{\sy*(0.2396)})
	--({\sx*(4.0500)},{\sy*(0.2708)})
	--({\sx*(4.0600)},{\sy*(0.3011)})
	--({\sx*(4.0700)},{\sy*(0.3304)})
	--({\sx*(4.0800)},{\sy*(0.3585)})
	--({\sx*(4.0900)},{\sy*(0.3855)})
	--({\sx*(4.1000)},{\sy*(0.4112)})
	--({\sx*(4.1100)},{\sy*(0.4356)})
	--({\sx*(4.1200)},{\sy*(0.4586)})
	--({\sx*(4.1300)},{\sy*(0.4801)})
	--({\sx*(4.1400)},{\sy*(0.5000)})
	--({\sx*(4.1500)},{\sy*(0.5184)})
	--({\sx*(4.1600)},{\sy*(0.5351)})
	--({\sx*(4.1700)},{\sy*(0.5502)})
	--({\sx*(4.1800)},{\sy*(0.5634)})
	--({\sx*(4.1900)},{\sy*(0.5749)})
	--({\sx*(4.2000)},{\sy*(0.5846)})
	--({\sx*(4.2100)},{\sy*(0.5923)})
	--({\sx*(4.2200)},{\sy*(0.5982)})
	--({\sx*(4.2300)},{\sy*(0.6022)})
	--({\sx*(4.2400)},{\sy*(0.6042)})
	--({\sx*(4.2500)},{\sy*(0.6043)})
	--({\sx*(4.2600)},{\sy*(0.6024)})
	--({\sx*(4.2700)},{\sy*(0.5986)})
	--({\sx*(4.2800)},{\sy*(0.5928)})
	--({\sx*(4.2900)},{\sy*(0.5851)})
	--({\sx*(4.3000)},{\sy*(0.5755)})
	--({\sx*(4.3100)},{\sy*(0.5641)})
	--({\sx*(4.3200)},{\sy*(0.5507)})
	--({\sx*(4.3300)},{\sy*(0.5356)})
	--({\sx*(4.3400)},{\sy*(0.5187)})
	--({\sx*(4.3500)},{\sy*(0.5001)})
	--({\sx*(4.3600)},{\sy*(0.4798)})
	--({\sx*(4.3700)},{\sy*(0.4580)})
	--({\sx*(4.3800)},{\sy*(0.4346)})
	--({\sx*(4.3900)},{\sy*(0.4099)})
	--({\sx*(4.4000)},{\sy*(0.3838)})
	--({\sx*(4.4100)},{\sy*(0.3565)})
	--({\sx*(4.4200)},{\sy*(0.3280)})
	--({\sx*(4.4300)},{\sy*(0.2985)})
	--({\sx*(4.4400)},{\sy*(0.2682)})
	--({\sx*(4.4500)},{\sy*(0.2370)})
	--({\sx*(4.4600)},{\sy*(0.2052)})
	--({\sx*(4.4700)},{\sy*(0.1728)})
	--({\sx*(4.4800)},{\sy*(0.1401)})
	--({\sx*(4.4900)},{\sy*(0.1071)})
	--({\sx*(4.5000)},{\sy*(0.0741)})
	--({\sx*(4.5100)},{\sy*(0.0411)})
	--({\sx*(4.5200)},{\sy*(0.0083)})
	--({\sx*(4.5300)},{\sy*(-0.0241)})
	--({\sx*(4.5400)},{\sy*(-0.0560)})
	--({\sx*(4.5500)},{\sy*(-0.0872)})
	--({\sx*(4.5600)},{\sy*(-0.1175)})
	--({\sx*(4.5700)},{\sy*(-0.1468)})
	--({\sx*(4.5800)},{\sy*(-0.1750)})
	--({\sx*(4.5900)},{\sy*(-0.2017)})
	--({\sx*(4.6000)},{\sy*(-0.2270)})
	--({\sx*(4.6100)},{\sy*(-0.2505)})
	--({\sx*(4.6200)},{\sy*(-0.2723)})
	--({\sx*(4.6300)},{\sy*(-0.2921)})
	--({\sx*(4.6400)},{\sy*(-0.3097)})
	--({\sx*(4.6500)},{\sy*(-0.3252)})
	--({\sx*(4.6600)},{\sy*(-0.3382)})
	--({\sx*(4.6700)},{\sy*(-0.3487)})
	--({\sx*(4.6800)},{\sy*(-0.3567)})
	--({\sx*(4.6900)},{\sy*(-0.3620)})
	--({\sx*(4.7000)},{\sy*(-0.3645)})
	--({\sx*(4.7100)},{\sy*(-0.3642)})
	--({\sx*(4.7200)},{\sy*(-0.3610)})
	--({\sx*(4.7300)},{\sy*(-0.3550)})
	--({\sx*(4.7400)},{\sy*(-0.3460)})
	--({\sx*(4.7500)},{\sy*(-0.3343)})
	--({\sx*(4.7600)},{\sy*(-0.3197)})
	--({\sx*(4.7700)},{\sy*(-0.3024)})
	--({\sx*(4.7800)},{\sy*(-0.2825)})
	--({\sx*(4.7900)},{\sy*(-0.2602)})
	--({\sx*(4.8000)},{\sy*(-0.2355)})
	--({\sx*(4.8100)},{\sy*(-0.2089)})
	--({\sx*(4.8200)},{\sy*(-0.1804)})
	--({\sx*(4.8300)},{\sy*(-0.1504)})
	--({\sx*(4.8400)},{\sy*(-0.1194)})
	--({\sx*(4.8500)},{\sy*(-0.0876)})
	--({\sx*(4.8600)},{\sy*(-0.0555)})
	--({\sx*(4.8700)},{\sy*(-0.0238)})
	--({\sx*(4.8800)},{\sy*(0.0072)})
	--({\sx*(4.8900)},{\sy*(0.0366)})
	--({\sx*(4.9000)},{\sy*(0.0639)})
	--({\sx*(4.9100)},{\sy*(0.0881)})
	--({\sx*(4.9200)},{\sy*(0.1085)})
	--({\sx*(4.9300)},{\sy*(0.1241)})
	--({\sx*(4.9400)},{\sy*(0.1339)})
	--({\sx*(4.9500)},{\sy*(0.1367)})
	--({\sx*(4.9600)},{\sy*(0.1313)})
	--({\sx*(4.9700)},{\sy*(0.1164)})
	--({\sx*(4.9800)},{\sy*(0.0906)})
	--({\sx*(4.9900)},{\sy*(0.0524)})
	--({\sx*(5.0000)},{\sy*(0.0000)});
}
\def\relfehlere{
\draw[color=blue,line width=1.4pt,line join=round] ({\sx*(0.000)},{\sy*(0.0000)})
	--({\sx*(0.0100)},{\sy*(-0.0000)})
	--({\sx*(0.0200)},{\sy*(-0.0000)})
	--({\sx*(0.0300)},{\sy*(-0.0000)})
	--({\sx*(0.0400)},{\sy*(-0.0000)})
	--({\sx*(0.0500)},{\sy*(-0.0000)})
	--({\sx*(0.0600)},{\sy*(-0.0000)})
	--({\sx*(0.0700)},{\sy*(-0.0000)})
	--({\sx*(0.0800)},{\sy*(-0.0000)})
	--({\sx*(0.0900)},{\sy*(-0.0000)})
	--({\sx*(0.1000)},{\sy*(-0.0000)})
	--({\sx*(0.1100)},{\sy*(-0.0000)})
	--({\sx*(0.1200)},{\sy*(-0.0000)})
	--({\sx*(0.1300)},{\sy*(0.0000)})
	--({\sx*(0.1400)},{\sy*(0.0000)})
	--({\sx*(0.1500)},{\sy*(0.0000)})
	--({\sx*(0.1600)},{\sy*(0.0000)})
	--({\sx*(0.1700)},{\sy*(0.0000)})
	--({\sx*(0.1800)},{\sy*(0.0000)})
	--({\sx*(0.1900)},{\sy*(0.0000)})
	--({\sx*(0.2000)},{\sy*(0.0000)})
	--({\sx*(0.2100)},{\sy*(0.0000)})
	--({\sx*(0.2200)},{\sy*(0.0000)})
	--({\sx*(0.2300)},{\sy*(0.0001)})
	--({\sx*(0.2400)},{\sy*(0.0001)})
	--({\sx*(0.2500)},{\sy*(0.0001)})
	--({\sx*(0.2600)},{\sy*(0.0001)})
	--({\sx*(0.2700)},{\sy*(0.0001)})
	--({\sx*(0.2800)},{\sy*(0.0001)})
	--({\sx*(0.2900)},{\sy*(0.0001)})
	--({\sx*(0.3000)},{\sy*(0.0001)})
	--({\sx*(0.3100)},{\sy*(0.0001)})
	--({\sx*(0.3200)},{\sy*(0.0001)})
	--({\sx*(0.3300)},{\sy*(0.0001)})
	--({\sx*(0.3400)},{\sy*(0.0001)})
	--({\sx*(0.3500)},{\sy*(0.0001)})
	--({\sx*(0.3600)},{\sy*(0.0001)})
	--({\sx*(0.3700)},{\sy*(0.0001)})
	--({\sx*(0.3800)},{\sy*(0.0001)})
	--({\sx*(0.3900)},{\sy*(0.0001)})
	--({\sx*(0.4000)},{\sy*(0.0000)})
	--({\sx*(0.4100)},{\sy*(0.0000)})
	--({\sx*(0.4200)},{\sy*(0.0000)})
	--({\sx*(0.4300)},{\sy*(0.0000)})
	--({\sx*(0.4400)},{\sy*(0.0000)})
	--({\sx*(0.4500)},{\sy*(0.0000)})
	--({\sx*(0.4600)},{\sy*(0.0000)})
	--({\sx*(0.4700)},{\sy*(0.0000)})
	--({\sx*(0.4800)},{\sy*(-0.0000)})
	--({\sx*(0.4900)},{\sy*(-0.0000)})
	--({\sx*(0.5000)},{\sy*(-0.0000)})
	--({\sx*(0.5100)},{\sy*(-0.0000)})
	--({\sx*(0.5200)},{\sy*(-0.0000)})
	--({\sx*(0.5300)},{\sy*(-0.0000)})
	--({\sx*(0.5400)},{\sy*(-0.0000)})
	--({\sx*(0.5500)},{\sy*(-0.0001)})
	--({\sx*(0.5600)},{\sy*(-0.0001)})
	--({\sx*(0.5700)},{\sy*(-0.0001)})
	--({\sx*(0.5800)},{\sy*(-0.0001)})
	--({\sx*(0.5900)},{\sy*(-0.0001)})
	--({\sx*(0.6000)},{\sy*(-0.0001)})
	--({\sx*(0.6100)},{\sy*(-0.0001)})
	--({\sx*(0.6200)},{\sy*(-0.0001)})
	--({\sx*(0.6300)},{\sy*(-0.0001)})
	--({\sx*(0.6400)},{\sy*(-0.0001)})
	--({\sx*(0.6500)},{\sy*(-0.0001)})
	--({\sx*(0.6600)},{\sy*(-0.0001)})
	--({\sx*(0.6700)},{\sy*(-0.0001)})
	--({\sx*(0.6800)},{\sy*(-0.0001)})
	--({\sx*(0.6900)},{\sy*(-0.0002)})
	--({\sx*(0.7000)},{\sy*(-0.0002)})
	--({\sx*(0.7100)},{\sy*(-0.0002)})
	--({\sx*(0.7200)},{\sy*(-0.0002)})
	--({\sx*(0.7300)},{\sy*(-0.0002)})
	--({\sx*(0.7400)},{\sy*(-0.0002)})
	--({\sx*(0.7500)},{\sy*(-0.0002)})
	--({\sx*(0.7600)},{\sy*(-0.0002)})
	--({\sx*(0.7700)},{\sy*(-0.0002)})
	--({\sx*(0.7800)},{\sy*(-0.0002)})
	--({\sx*(0.7900)},{\sy*(-0.0002)})
	--({\sx*(0.8000)},{\sy*(-0.0002)})
	--({\sx*(0.8100)},{\sy*(-0.0002)})
	--({\sx*(0.8200)},{\sy*(-0.0002)})
	--({\sx*(0.8300)},{\sy*(-0.0002)})
	--({\sx*(0.8400)},{\sy*(-0.0002)})
	--({\sx*(0.8500)},{\sy*(-0.0002)})
	--({\sx*(0.8600)},{\sy*(-0.0002)})
	--({\sx*(0.8700)},{\sy*(-0.0002)})
	--({\sx*(0.8800)},{\sy*(-0.0002)})
	--({\sx*(0.8900)},{\sy*(-0.0001)})
	--({\sx*(0.9000)},{\sy*(-0.0001)})
	--({\sx*(0.9100)},{\sy*(-0.0001)})
	--({\sx*(0.9200)},{\sy*(-0.0001)})
	--({\sx*(0.9300)},{\sy*(-0.0001)})
	--({\sx*(0.9400)},{\sy*(-0.0001)})
	--({\sx*(0.9500)},{\sy*(-0.0001)})
	--({\sx*(0.9600)},{\sy*(-0.0001)})
	--({\sx*(0.9700)},{\sy*(-0.0001)})
	--({\sx*(0.9800)},{\sy*(-0.0001)})
	--({\sx*(0.9900)},{\sy*(-0.0001)})
	--({\sx*(1.0000)},{\sy*(-0.0000)})
	--({\sx*(1.0100)},{\sy*(-0.0000)})
	--({\sx*(1.0200)},{\sy*(-0.0000)})
	--({\sx*(1.0300)},{\sy*(-0.0000)})
	--({\sx*(1.0400)},{\sy*(0.0000)})
	--({\sx*(1.0500)},{\sy*(0.0000)})
	--({\sx*(1.0600)},{\sy*(0.0000)})
	--({\sx*(1.0700)},{\sy*(0.0001)})
	--({\sx*(1.0800)},{\sy*(0.0001)})
	--({\sx*(1.0900)},{\sy*(0.0001)})
	--({\sx*(1.1000)},{\sy*(0.0001)})
	--({\sx*(1.1100)},{\sy*(0.0001)})
	--({\sx*(1.1200)},{\sy*(0.0001)})
	--({\sx*(1.1300)},{\sy*(0.0002)})
	--({\sx*(1.1400)},{\sy*(0.0002)})
	--({\sx*(1.1500)},{\sy*(0.0002)})
	--({\sx*(1.1600)},{\sy*(0.0002)})
	--({\sx*(1.1700)},{\sy*(0.0002)})
	--({\sx*(1.1800)},{\sy*(0.0002)})
	--({\sx*(1.1900)},{\sy*(0.0003)})
	--({\sx*(1.2000)},{\sy*(0.0003)})
	--({\sx*(1.2100)},{\sy*(0.0003)})
	--({\sx*(1.2200)},{\sy*(0.0003)})
	--({\sx*(1.2300)},{\sy*(0.0003)})
	--({\sx*(1.2400)},{\sy*(0.0003)})
	--({\sx*(1.2500)},{\sy*(0.0004)})
	--({\sx*(1.2600)},{\sy*(0.0004)})
	--({\sx*(1.2700)},{\sy*(0.0004)})
	--({\sx*(1.2800)},{\sy*(0.0004)})
	--({\sx*(1.2900)},{\sy*(0.0004)})
	--({\sx*(1.3000)},{\sy*(0.0004)})
	--({\sx*(1.3100)},{\sy*(0.0004)})
	--({\sx*(1.3200)},{\sy*(0.0005)})
	--({\sx*(1.3300)},{\sy*(0.0005)})
	--({\sx*(1.3400)},{\sy*(0.0005)})
	--({\sx*(1.3500)},{\sy*(0.0005)})
	--({\sx*(1.3600)},{\sy*(0.0005)})
	--({\sx*(1.3700)},{\sy*(0.0005)})
	--({\sx*(1.3800)},{\sy*(0.0005)})
	--({\sx*(1.3900)},{\sy*(0.0005)})
	--({\sx*(1.4000)},{\sy*(0.0005)})
	--({\sx*(1.4100)},{\sy*(0.0005)})
	--({\sx*(1.4200)},{\sy*(0.0006)})
	--({\sx*(1.4300)},{\sy*(0.0006)})
	--({\sx*(1.4400)},{\sy*(0.0006)})
	--({\sx*(1.4500)},{\sy*(0.0006)})
	--({\sx*(1.4600)},{\sy*(0.0006)})
	--({\sx*(1.4700)},{\sy*(0.0006)})
	--({\sx*(1.4800)},{\sy*(0.0006)})
	--({\sx*(1.4900)},{\sy*(0.0006)})
	--({\sx*(1.5000)},{\sy*(0.0005)})
	--({\sx*(1.5100)},{\sy*(0.0005)})
	--({\sx*(1.5200)},{\sy*(0.0005)})
	--({\sx*(1.5300)},{\sy*(0.0005)})
	--({\sx*(1.5400)},{\sy*(0.0005)})
	--({\sx*(1.5500)},{\sy*(0.0005)})
	--({\sx*(1.5600)},{\sy*(0.0005)})
	--({\sx*(1.5700)},{\sy*(0.0005)})
	--({\sx*(1.5800)},{\sy*(0.0005)})
	--({\sx*(1.5900)},{\sy*(0.0004)})
	--({\sx*(1.6000)},{\sy*(0.0004)})
	--({\sx*(1.6100)},{\sy*(0.0004)})
	--({\sx*(1.6200)},{\sy*(0.0004)})
	--({\sx*(1.6300)},{\sy*(0.0003)})
	--({\sx*(1.6400)},{\sy*(0.0003)})
	--({\sx*(1.6500)},{\sy*(0.0003)})
	--({\sx*(1.6600)},{\sy*(0.0003)})
	--({\sx*(1.6700)},{\sy*(0.0002)})
	--({\sx*(1.6800)},{\sy*(0.0002)})
	--({\sx*(1.6900)},{\sy*(0.0002)})
	--({\sx*(1.7000)},{\sy*(0.0001)})
	--({\sx*(1.7100)},{\sy*(0.0001)})
	--({\sx*(1.7200)},{\sy*(0.0000)})
	--({\sx*(1.7300)},{\sy*(-0.0000)})
	--({\sx*(1.7400)},{\sy*(-0.0001)})
	--({\sx*(1.7500)},{\sy*(-0.0001)})
	--({\sx*(1.7600)},{\sy*(-0.0002)})
	--({\sx*(1.7700)},{\sy*(-0.0002)})
	--({\sx*(1.7800)},{\sy*(-0.0003)})
	--({\sx*(1.7900)},{\sy*(-0.0003)})
	--({\sx*(1.8000)},{\sy*(-0.0004)})
	--({\sx*(1.8100)},{\sy*(-0.0004)})
	--({\sx*(1.8200)},{\sy*(-0.0005)})
	--({\sx*(1.8300)},{\sy*(-0.0005)})
	--({\sx*(1.8400)},{\sy*(-0.0006)})
	--({\sx*(1.8500)},{\sy*(-0.0007)})
	--({\sx*(1.8600)},{\sy*(-0.0007)})
	--({\sx*(1.8700)},{\sy*(-0.0008)})
	--({\sx*(1.8800)},{\sy*(-0.0008)})
	--({\sx*(1.8900)},{\sy*(-0.0009)})
	--({\sx*(1.9000)},{\sy*(-0.0010)})
	--({\sx*(1.9100)},{\sy*(-0.0010)})
	--({\sx*(1.9200)},{\sy*(-0.0011)})
	--({\sx*(1.9300)},{\sy*(-0.0012)})
	--({\sx*(1.9400)},{\sy*(-0.0012)})
	--({\sx*(1.9500)},{\sy*(-0.0013)})
	--({\sx*(1.9600)},{\sy*(-0.0014)})
	--({\sx*(1.9700)},{\sy*(-0.0014)})
	--({\sx*(1.9800)},{\sy*(-0.0015)})
	--({\sx*(1.9900)},{\sy*(-0.0016)})
	--({\sx*(2.0000)},{\sy*(-0.0017)})
	--({\sx*(2.0100)},{\sy*(-0.0017)})
	--({\sx*(2.0200)},{\sy*(-0.0018)})
	--({\sx*(2.0300)},{\sy*(-0.0019)})
	--({\sx*(2.0400)},{\sy*(-0.0019)})
	--({\sx*(2.0500)},{\sy*(-0.0020)})
	--({\sx*(2.0600)},{\sy*(-0.0020)})
	--({\sx*(2.0700)},{\sy*(-0.0021)})
	--({\sx*(2.0800)},{\sy*(-0.0022)})
	--({\sx*(2.0900)},{\sy*(-0.0022)})
	--({\sx*(2.1000)},{\sy*(-0.0023)})
	--({\sx*(2.1100)},{\sy*(-0.0023)})
	--({\sx*(2.1200)},{\sy*(-0.0024)})
	--({\sx*(2.1300)},{\sy*(-0.0024)})
	--({\sx*(2.1400)},{\sy*(-0.0025)})
	--({\sx*(2.1500)},{\sy*(-0.0025)})
	--({\sx*(2.1600)},{\sy*(-0.0026)})
	--({\sx*(2.1700)},{\sy*(-0.0026)})
	--({\sx*(2.1800)},{\sy*(-0.0026)})
	--({\sx*(2.1900)},{\sy*(-0.0026)})
	--({\sx*(2.2000)},{\sy*(-0.0027)})
	--({\sx*(2.2100)},{\sy*(-0.0027)})
	--({\sx*(2.2200)},{\sy*(-0.0027)})
	--({\sx*(2.2300)},{\sy*(-0.0027)})
	--({\sx*(2.2400)},{\sy*(-0.0027)})
	--({\sx*(2.2500)},{\sy*(-0.0027)})
	--({\sx*(2.2600)},{\sy*(-0.0027)})
	--({\sx*(2.2700)},{\sy*(-0.0027)})
	--({\sx*(2.2800)},{\sy*(-0.0027)})
	--({\sx*(2.2900)},{\sy*(-0.0026)})
	--({\sx*(2.3000)},{\sy*(-0.0026)})
	--({\sx*(2.3100)},{\sy*(-0.0025)})
	--({\sx*(2.3200)},{\sy*(-0.0025)})
	--({\sx*(2.3300)},{\sy*(-0.0024)})
	--({\sx*(2.3400)},{\sy*(-0.0024)})
	--({\sx*(2.3500)},{\sy*(-0.0023)})
	--({\sx*(2.3600)},{\sy*(-0.0022)})
	--({\sx*(2.3700)},{\sy*(-0.0021)})
	--({\sx*(2.3800)},{\sy*(-0.0020)})
	--({\sx*(2.3900)},{\sy*(-0.0019)})
	--({\sx*(2.4000)},{\sy*(-0.0018)})
	--({\sx*(2.4100)},{\sy*(-0.0017)})
	--({\sx*(2.4200)},{\sy*(-0.0015)})
	--({\sx*(2.4300)},{\sy*(-0.0014)})
	--({\sx*(2.4400)},{\sy*(-0.0012)})
	--({\sx*(2.4500)},{\sy*(-0.0010)})
	--({\sx*(2.4600)},{\sy*(-0.0008)})
	--({\sx*(2.4700)},{\sy*(-0.0007)})
	--({\sx*(2.4800)},{\sy*(-0.0004)})
	--({\sx*(2.4900)},{\sy*(-0.0002)})
	--({\sx*(2.5000)},{\sy*(0.0000)})
	--({\sx*(2.5100)},{\sy*(0.0002)})
	--({\sx*(2.5200)},{\sy*(0.0005)})
	--({\sx*(2.5300)},{\sy*(0.0008)})
	--({\sx*(2.5400)},{\sy*(0.0010)})
	--({\sx*(2.5500)},{\sy*(0.0013)})
	--({\sx*(2.5600)},{\sy*(0.0016)})
	--({\sx*(2.5700)},{\sy*(0.0019)})
	--({\sx*(2.5800)},{\sy*(0.0022)})
	--({\sx*(2.5900)},{\sy*(0.0026)})
	--({\sx*(2.6000)},{\sy*(0.0029)})
	--({\sx*(2.6100)},{\sy*(0.0033)})
	--({\sx*(2.6200)},{\sy*(0.0036)})
	--({\sx*(2.6300)},{\sy*(0.0040)})
	--({\sx*(2.6400)},{\sy*(0.0044)})
	--({\sx*(2.6500)},{\sy*(0.0048)})
	--({\sx*(2.6600)},{\sy*(0.0052)})
	--({\sx*(2.6700)},{\sy*(0.0056)})
	--({\sx*(2.6800)},{\sy*(0.0061)})
	--({\sx*(2.6900)},{\sy*(0.0065)})
	--({\sx*(2.7000)},{\sy*(0.0070)})
	--({\sx*(2.7100)},{\sy*(0.0074)})
	--({\sx*(2.7200)},{\sy*(0.0079)})
	--({\sx*(2.7300)},{\sy*(0.0083)})
	--({\sx*(2.7400)},{\sy*(0.0088)})
	--({\sx*(2.7500)},{\sy*(0.0093)})
	--({\sx*(2.7600)},{\sy*(0.0098)})
	--({\sx*(2.7700)},{\sy*(0.0103)})
	--({\sx*(2.7800)},{\sy*(0.0108)})
	--({\sx*(2.7900)},{\sy*(0.0112)})
	--({\sx*(2.8000)},{\sy*(0.0117)})
	--({\sx*(2.8100)},{\sy*(0.0122)})
	--({\sx*(2.8200)},{\sy*(0.0127)})
	--({\sx*(2.8300)},{\sy*(0.0132)})
	--({\sx*(2.8400)},{\sy*(0.0137)})
	--({\sx*(2.8500)},{\sy*(0.0142)})
	--({\sx*(2.8600)},{\sy*(0.0147)})
	--({\sx*(2.8700)},{\sy*(0.0151)})
	--({\sx*(2.8800)},{\sy*(0.0156)})
	--({\sx*(2.8900)},{\sy*(0.0160)})
	--({\sx*(2.9000)},{\sy*(0.0164)})
	--({\sx*(2.9100)},{\sy*(0.0169)})
	--({\sx*(2.9200)},{\sy*(0.0173)})
	--({\sx*(2.9300)},{\sy*(0.0176)})
	--({\sx*(2.9400)},{\sy*(0.0180)})
	--({\sx*(2.9500)},{\sy*(0.0183)})
	--({\sx*(2.9600)},{\sy*(0.0187)})
	--({\sx*(2.9700)},{\sy*(0.0189)})
	--({\sx*(2.9800)},{\sy*(0.0192)})
	--({\sx*(2.9900)},{\sy*(0.0194)})
	--({\sx*(3.0000)},{\sy*(0.0196)})
	--({\sx*(3.0100)},{\sy*(0.0198)})
	--({\sx*(3.0200)},{\sy*(0.0199)})
	--({\sx*(3.0300)},{\sy*(0.0200)})
	--({\sx*(3.0400)},{\sy*(0.0200)})
	--({\sx*(3.0500)},{\sy*(0.0200)})
	--({\sx*(3.0600)},{\sy*(0.0199)})
	--({\sx*(3.0700)},{\sy*(0.0198)})
	--({\sx*(3.0800)},{\sy*(0.0196)})
	--({\sx*(3.0900)},{\sy*(0.0194)})
	--({\sx*(3.1000)},{\sy*(0.0190)})
	--({\sx*(3.1100)},{\sy*(0.0187)})
	--({\sx*(3.1200)},{\sy*(0.0182)})
	--({\sx*(3.1300)},{\sy*(0.0177)})
	--({\sx*(3.1400)},{\sy*(0.0171)})
	--({\sx*(3.1500)},{\sy*(0.0164)})
	--({\sx*(3.1600)},{\sy*(0.0157)})
	--({\sx*(3.1700)},{\sy*(0.0148)})
	--({\sx*(3.1800)},{\sy*(0.0139)})
	--({\sx*(3.1900)},{\sy*(0.0128)})
	--({\sx*(3.2000)},{\sy*(0.0117)})
	--({\sx*(3.2100)},{\sy*(0.0105)})
	--({\sx*(3.2200)},{\sy*(0.0091)})
	--({\sx*(3.2300)},{\sy*(0.0076)})
	--({\sx*(3.2400)},{\sy*(0.0060)})
	--({\sx*(3.2500)},{\sy*(0.0043)})
	--({\sx*(3.2600)},{\sy*(0.0025)})
	--({\sx*(3.2700)},{\sy*(0.0005)})
	--({\sx*(3.2800)},{\sy*(-0.0016)})
	--({\sx*(3.2900)},{\sy*(-0.0038)})
	--({\sx*(3.3000)},{\sy*(-0.0062)})
	--({\sx*(3.3100)},{\sy*(-0.0088)})
	--({\sx*(3.3200)},{\sy*(-0.0115)})
	--({\sx*(3.3300)},{\sy*(-0.0143)})
	--({\sx*(3.3400)},{\sy*(-0.0173)})
	--({\sx*(3.3500)},{\sy*(-0.0205)})
	--({\sx*(3.3600)},{\sy*(-0.0238)})
	--({\sx*(3.3700)},{\sy*(-0.0274)})
	--({\sx*(3.3800)},{\sy*(-0.0311)})
	--({\sx*(3.3900)},{\sy*(-0.0349)})
	--({\sx*(3.4000)},{\sy*(-0.0390)})
	--({\sx*(3.4100)},{\sy*(-0.0433)})
	--({\sx*(3.4200)},{\sy*(-0.0477)})
	--({\sx*(3.4300)},{\sy*(-0.0523)})
	--({\sx*(3.4400)},{\sy*(-0.0571)})
	--({\sx*(3.4500)},{\sy*(-0.0621)})
	--({\sx*(3.4600)},{\sy*(-0.0673)})
	--({\sx*(3.4700)},{\sy*(-0.0727)})
	--({\sx*(3.4800)},{\sy*(-0.0783)})
	--({\sx*(3.4900)},{\sy*(-0.0841)})
	--({\sx*(3.5000)},{\sy*(-0.0900)})
	--({\sx*(3.5100)},{\sy*(-0.0962)})
	--({\sx*(3.5200)},{\sy*(-0.1025)})
	--({\sx*(3.5300)},{\sy*(-0.1089)})
	--({\sx*(3.5400)},{\sy*(-0.1155)})
	--({\sx*(3.5500)},{\sy*(-0.1223)})
	--({\sx*(3.5600)},{\sy*(-0.1291)})
	--({\sx*(3.5700)},{\sy*(-0.1361)})
	--({\sx*(3.5800)},{\sy*(-0.1432)})
	--({\sx*(3.5900)},{\sy*(-0.1504)})
	--({\sx*(3.6000)},{\sy*(-0.1576)})
	--({\sx*(3.6100)},{\sy*(-0.1648)})
	--({\sx*(3.6200)},{\sy*(-0.1720)})
	--({\sx*(3.6300)},{\sy*(-0.1791)})
	--({\sx*(3.6400)},{\sy*(-0.1862)})
	--({\sx*(3.6500)},{\sy*(-0.1931)})
	--({\sx*(3.6600)},{\sy*(-0.1999)})
	--({\sx*(3.6700)},{\sy*(-0.2064)})
	--({\sx*(3.6800)},{\sy*(-0.2127)})
	--({\sx*(3.6900)},{\sy*(-0.2186)})
	--({\sx*(3.7000)},{\sy*(-0.2241)})
	--({\sx*(3.7100)},{\sy*(-0.2292)})
	--({\sx*(3.7200)},{\sy*(-0.2337)})
	--({\sx*(3.7300)},{\sy*(-0.2376)})
	--({\sx*(3.7400)},{\sy*(-0.2408)})
	--({\sx*(3.7500)},{\sy*(-0.2433)})
	--({\sx*(3.7600)},{\sy*(-0.2449)})
	--({\sx*(3.7700)},{\sy*(-0.2456)})
	--({\sx*(3.7800)},{\sy*(-0.2453)})
	--({\sx*(3.7900)},{\sy*(-0.2439)})
	--({\sx*(3.8000)},{\sy*(-0.2414)})
	--({\sx*(3.8100)},{\sy*(-0.2376)})
	--({\sx*(3.8200)},{\sy*(-0.2326)})
	--({\sx*(3.8300)},{\sy*(-0.2262)})
	--({\sx*(3.8400)},{\sy*(-0.2185)})
	--({\sx*(3.8500)},{\sy*(-0.2094)})
	--({\sx*(3.8600)},{\sy*(-0.1989)})
	--({\sx*(3.8700)},{\sy*(-0.1870)})
	--({\sx*(3.8800)},{\sy*(-0.1736)})
	--({\sx*(3.8900)},{\sy*(-0.1589)})
	--({\sx*(3.9000)},{\sy*(-0.1428)})
	--({\sx*(3.9100)},{\sy*(-0.1254)})
	--({\sx*(3.9200)},{\sy*(-0.1069)})
	--({\sx*(3.9300)},{\sy*(-0.0871)})
	--({\sx*(3.9400)},{\sy*(-0.0663)})
	--({\sx*(3.9500)},{\sy*(-0.0446)})
	--({\sx*(3.9600)},{\sy*(-0.0220)})
	--({\sx*(3.9700)},{\sy*(0.0013)})
	--({\sx*(3.9800)},{\sy*(0.0252)})
	--({\sx*(3.9900)},{\sy*(0.0495)})
	--({\sx*(4.0000)},{\sy*(0.0742)})
	--({\sx*(4.0100)},{\sy*(0.0992)})
	--({\sx*(4.0200)},{\sy*(0.1242)})
	--({\sx*(4.0300)},{\sy*(0.1492)})
	--({\sx*(4.0400)},{\sy*(0.1741)})
	--({\sx*(4.0500)},{\sy*(0.1988)})
	--({\sx*(4.0600)},{\sy*(0.2232)})
	--({\sx*(4.0700)},{\sy*(0.2472)})
	--({\sx*(4.0800)},{\sy*(0.2707)})
	--({\sx*(4.0900)},{\sy*(0.2936)})
	--({\sx*(4.1000)},{\sy*(0.3160)})
	--({\sx*(4.1100)},{\sy*(0.3377)})
	--({\sx*(4.1200)},{\sy*(0.3587)})
	--({\sx*(4.1300)},{\sy*(0.3790)})
	--({\sx*(4.1400)},{\sy*(0.3985)})
	--({\sx*(4.1500)},{\sy*(0.4172)})
	--({\sx*(4.1600)},{\sy*(0.4351)})
	--({\sx*(4.1700)},{\sy*(0.4522)})
	--({\sx*(4.1800)},{\sy*(0.4685)})
	--({\sx*(4.1900)},{\sy*(0.4840)})
	--({\sx*(4.2000)},{\sy*(0.4986)})
	--({\sx*(4.2100)},{\sy*(0.5124)})
	--({\sx*(4.2200)},{\sy*(0.5254)})
	--({\sx*(4.2300)},{\sy*(0.5376)})
	--({\sx*(4.2400)},{\sy*(0.5489)})
	--({\sx*(4.2500)},{\sy*(0.5594)})
	--({\sx*(4.2600)},{\sy*(0.5691)})
	--({\sx*(4.2700)},{\sy*(0.5780)})
	--({\sx*(4.2800)},{\sy*(0.5861)})
	--({\sx*(4.2900)},{\sy*(0.5933)})
	--({\sx*(4.3000)},{\sy*(0.5996)})
	--({\sx*(4.3100)},{\sy*(0.6051)})
	--({\sx*(4.3200)},{\sy*(0.6097)})
	--({\sx*(4.3300)},{\sy*(0.6133)})
	--({\sx*(4.3400)},{\sy*(0.6160)})
	--({\sx*(4.3500)},{\sy*(0.6177)})
	--({\sx*(4.3600)},{\sy*(0.6182)})
	--({\sx*(4.3700)},{\sy*(0.6175)})
	--({\sx*(4.3800)},{\sy*(0.6155)})
	--({\sx*(4.3900)},{\sy*(0.6119)})
	--({\sx*(4.4000)},{\sy*(0.6068)})
	--({\sx*(4.4100)},{\sy*(0.5996)})
	--({\sx*(4.4200)},{\sy*(0.5902)})
	--({\sx*(4.4300)},{\sy*(0.5781)})
	--({\sx*(4.4400)},{\sy*(0.5627)})
	--({\sx*(4.4500)},{\sy*(0.5431)})
	--({\sx*(4.4600)},{\sy*(0.5183)})
	--({\sx*(4.4700)},{\sy*(0.4866)})
	--({\sx*(4.4800)},{\sy*(0.4455)})
	--({\sx*(4.4900)},{\sy*(0.3912)})
	--({\sx*(4.5000)},{\sy*(0.3173)})
	--({\sx*(4.5100)},{\sy*(0.2123)})
	--({\sx*(4.5200)},{\sy*(0.0538)})
	--({\sx*(4.5300)},{\sy*(-0.2097)})
	--({\sx*(4.5400)},{\sy*(-0.7274)})
	--({\sx*(4.5500)},{\sy*(-2.1849)})
	--({\sx*(4.5600)},{\sy*(-29.9468)})
	--({\sx*(4.5700)},{\sy*(4.7678)})
	--({\sx*(4.5800)},{\sy*(2.7293)})
	--({\sx*(4.5900)},{\sy*(2.1049)})
	--({\sx*(4.6000)},{\sy*(1.8037)})
	--({\sx*(4.6100)},{\sy*(1.6273)})
	--({\sx*(4.6200)},{\sy*(1.5121)})
	--({\sx*(4.6300)},{\sy*(1.4316)})
	--({\sx*(4.6400)},{\sy*(1.3725)})
	--({\sx*(4.6500)},{\sy*(1.3277)})
	--({\sx*(4.6600)},{\sy*(1.2928)})
	--({\sx*(4.6700)},{\sy*(1.2652)})
	--({\sx*(4.6800)},{\sy*(1.2432)})
	--({\sx*(4.6900)},{\sy*(1.2254)})
	--({\sx*(4.7000)},{\sy*(1.2111)})
	--({\sx*(4.7100)},{\sy*(1.1996)})
	--({\sx*(4.7200)},{\sy*(1.1907)})
	--({\sx*(4.7300)},{\sy*(1.1839)})
	--({\sx*(4.7400)},{\sy*(1.1792)})
	--({\sx*(4.7500)},{\sy*(1.1765)})
	--({\sx*(4.7600)},{\sy*(1.1759)})
	--({\sx*(4.7700)},{\sy*(1.1776)})
	--({\sx*(4.7800)},{\sy*(1.1819)})
	--({\sx*(4.7900)},{\sy*(1.1895)})
	--({\sx*(4.8000)},{\sy*(1.2015)})
	--({\sx*(4.8100)},{\sy*(1.2199)})
	--({\sx*(4.8200)},{\sy*(1.2483)})
	--({\sx*(4.8300)},{\sy*(1.2941)})
	--({\sx*(4.8400)},{\sy*(1.3753)})
	--({\sx*(4.8500)},{\sy*(1.5489)})
	--({\sx*(4.8600)},{\sy*(2.1385)})
	--({\sx*(4.8700)},{\sy*(-5.3854)})
	--({\sx*(4.8800)},{\sy*(0.2114)})
	--({\sx*(4.8900)},{\sy*(0.5892)})
	--({\sx*(4.9000)},{\sy*(0.7243)})
	--({\sx*(4.9100)},{\sy*(0.7919)})
	--({\sx*(4.9200)},{\sy*(0.8312)})
	--({\sx*(4.9300)},{\sy*(0.8554)})
	--({\sx*(4.9400)},{\sy*(0.8702)})
	--({\sx*(4.9500)},{\sy*(0.8779)})
	--({\sx*(4.9600)},{\sy*(0.8789)})
	--({\sx*(4.9700)},{\sy*(0.8712)})
	--({\sx*(4.9800)},{\sy*(0.8469)})
	--({\sx*(4.9900)},{\sy*(0.7707)})
	--({\sx*(5.0000)},{\sy*(0.0000)});
}
\def\xwertef{
\fill[color=red] (0.0000,0) circle[radius={0.07/\skala}];
\fill[color=white] (0.0000,0) circle[radius={0.05/\skala}];
\fill[color=red] (0.0852,0) circle[radius={0.07/\skala}];
\fill[color=white] (0.0852,0) circle[radius={0.05/\skala}];
\fill[color=red] (0.3349,0) circle[radius={0.07/\skala}];
\fill[color=white] (0.3349,0) circle[radius={0.05/\skala}];
\fill[color=red] (0.7322,0) circle[radius={0.07/\skala}];
\fill[color=white] (0.7322,0) circle[radius={0.05/\skala}];
\fill[color=red] (1.2500,0) circle[radius={0.07/\skala}];
\fill[color=white] (1.2500,0) circle[radius={0.05/\skala}];
\fill[color=red] (1.8530,0) circle[radius={0.07/\skala}];
\fill[color=white] (1.8530,0) circle[radius={0.05/\skala}];
\fill[color=red] (2.5000,0) circle[radius={0.07/\skala}];
\fill[color=white] (2.5000,0) circle[radius={0.05/\skala}];
\fill[color=red] (3.1470,0) circle[radius={0.07/\skala}];
\fill[color=white] (3.1470,0) circle[radius={0.05/\skala}];
\fill[color=red] (3.7500,0) circle[radius={0.07/\skala}];
\fill[color=white] (3.7500,0) circle[radius={0.05/\skala}];
\fill[color=red] (4.2678,0) circle[radius={0.07/\skala}];
\fill[color=white] (4.2678,0) circle[radius={0.05/\skala}];
\fill[color=red] (4.6651,0) circle[radius={0.07/\skala}];
\fill[color=white] (4.6651,0) circle[radius={0.05/\skala}];
\fill[color=red] (4.9148,0) circle[radius={0.07/\skala}];
\fill[color=white] (4.9148,0) circle[radius={0.05/\skala}];
\fill[color=red] (5.0000,0) circle[radius={0.07/\skala}];
\fill[color=white] (5.0000,0) circle[radius={0.05/\skala}];
}
\def\punktef{12}
\def\maxfehlerf{1.258\cdot 10^{-5}}
\def\fehlerf{
\draw[color=red,line width=1.4pt,line join=round] ({\sx*(0.000)},{\sy*(0.0000)})
	--({\sx*(0.0100)},{\sy*(0.0779)})
	--({\sx*(0.0200)},{\sy*(0.1253)})
	--({\sx*(0.0300)},{\sy*(0.1474)})
	--({\sx*(0.0400)},{\sy*(0.1488)})
	--({\sx*(0.0500)},{\sy*(0.1337)})
	--({\sx*(0.0600)},{\sy*(0.1059)})
	--({\sx*(0.0700)},{\sy*(0.0686)})
	--({\sx*(0.0800)},{\sy*(0.0246)})
	--({\sx*(0.0900)},{\sy*(-0.0235)})
	--({\sx*(0.1000)},{\sy*(-0.0736)})
	--({\sx*(0.1100)},{\sy*(-0.1238)})
	--({\sx*(0.1200)},{\sy*(-0.1726)})
	--({\sx*(0.1300)},{\sy*(-0.2186)})
	--({\sx*(0.1400)},{\sy*(-0.2608)})
	--({\sx*(0.1500)},{\sy*(-0.2984)})
	--({\sx*(0.1600)},{\sy*(-0.3306)})
	--({\sx*(0.1700)},{\sy*(-0.3570)})
	--({\sx*(0.1800)},{\sy*(-0.3772)})
	--({\sx*(0.1900)},{\sy*(-0.3911)})
	--({\sx*(0.2000)},{\sy*(-0.3985)})
	--({\sx*(0.2100)},{\sy*(-0.3996)})
	--({\sx*(0.2200)},{\sy*(-0.3945)})
	--({\sx*(0.2300)},{\sy*(-0.3833)})
	--({\sx*(0.2400)},{\sy*(-0.3664)})
	--({\sx*(0.2500)},{\sy*(-0.3440)})
	--({\sx*(0.2600)},{\sy*(-0.3167)})
	--({\sx*(0.2700)},{\sy*(-0.2848)})
	--({\sx*(0.2800)},{\sy*(-0.2488)})
	--({\sx*(0.2900)},{\sy*(-0.2092)})
	--({\sx*(0.3000)},{\sy*(-0.1665)})
	--({\sx*(0.3100)},{\sy*(-0.1211)})
	--({\sx*(0.3200)},{\sy*(-0.0737)})
	--({\sx*(0.3300)},{\sy*(-0.0246)})
	--({\sx*(0.3400)},{\sy*(0.0255)})
	--({\sx*(0.3500)},{\sy*(0.0761)})
	--({\sx*(0.3600)},{\sy*(0.1269)})
	--({\sx*(0.3700)},{\sy*(0.1773)})
	--({\sx*(0.3800)},{\sy*(0.2269)})
	--({\sx*(0.3900)},{\sy*(0.2753)})
	--({\sx*(0.4000)},{\sy*(0.3220)})
	--({\sx*(0.4100)},{\sy*(0.3668)})
	--({\sx*(0.4200)},{\sy*(0.4092)})
	--({\sx*(0.4300)},{\sy*(0.4489)})
	--({\sx*(0.4400)},{\sy*(0.4857)})
	--({\sx*(0.4500)},{\sy*(0.5194)})
	--({\sx*(0.4600)},{\sy*(0.5496)})
	--({\sx*(0.4700)},{\sy*(0.5762)})
	--({\sx*(0.4800)},{\sy*(0.5991)})
	--({\sx*(0.4900)},{\sy*(0.6181)})
	--({\sx*(0.5000)},{\sy*(0.6332)})
	--({\sx*(0.5100)},{\sy*(0.6441)})
	--({\sx*(0.5200)},{\sy*(0.6510)})
	--({\sx*(0.5300)},{\sy*(0.6538)})
	--({\sx*(0.5400)},{\sy*(0.6525)})
	--({\sx*(0.5500)},{\sy*(0.6471)})
	--({\sx*(0.5600)},{\sy*(0.6377)})
	--({\sx*(0.5700)},{\sy*(0.6243)})
	--({\sx*(0.5800)},{\sy*(0.6072)})
	--({\sx*(0.5900)},{\sy*(0.5864)})
	--({\sx*(0.6000)},{\sy*(0.5620)})
	--({\sx*(0.6100)},{\sy*(0.5342)})
	--({\sx*(0.6200)},{\sy*(0.5032)})
	--({\sx*(0.6300)},{\sy*(0.4692)})
	--({\sx*(0.6400)},{\sy*(0.4324)})
	--({\sx*(0.6500)},{\sy*(0.3930)})
	--({\sx*(0.6600)},{\sy*(0.3511)})
	--({\sx*(0.6700)},{\sy*(0.3071)})
	--({\sx*(0.6800)},{\sy*(0.2612)})
	--({\sx*(0.6900)},{\sy*(0.2137)})
	--({\sx*(0.7000)},{\sy*(0.1647)})
	--({\sx*(0.7100)},{\sy*(0.1145)})
	--({\sx*(0.7200)},{\sy*(0.0634)})
	--({\sx*(0.7300)},{\sy*(0.0116)})
	--({\sx*(0.7400)},{\sy*(-0.0406)})
	--({\sx*(0.7500)},{\sy*(-0.0929)})
	--({\sx*(0.7600)},{\sy*(-0.1452)})
	--({\sx*(0.7700)},{\sy*(-0.1971)})
	--({\sx*(0.7800)},{\sy*(-0.2485)})
	--({\sx*(0.7900)},{\sy*(-0.2992)})
	--({\sx*(0.8000)},{\sy*(-0.3488)})
	--({\sx*(0.8100)},{\sy*(-0.3971)})
	--({\sx*(0.8200)},{\sy*(-0.4441)})
	--({\sx*(0.8300)},{\sy*(-0.4894)})
	--({\sx*(0.8400)},{\sy*(-0.5328)})
	--({\sx*(0.8500)},{\sy*(-0.5743)})
	--({\sx*(0.8600)},{\sy*(-0.6136)})
	--({\sx*(0.8700)},{\sy*(-0.6506)})
	--({\sx*(0.8800)},{\sy*(-0.6850)})
	--({\sx*(0.8900)},{\sy*(-0.7169)})
	--({\sx*(0.9000)},{\sy*(-0.7461)})
	--({\sx*(0.9100)},{\sy*(-0.7724)})
	--({\sx*(0.9200)},{\sy*(-0.7959)})
	--({\sx*(0.9300)},{\sy*(-0.8163)})
	--({\sx*(0.9400)},{\sy*(-0.8336)})
	--({\sx*(0.9500)},{\sy*(-0.8478)})
	--({\sx*(0.9600)},{\sy*(-0.8588)})
	--({\sx*(0.9700)},{\sy*(-0.8666)})
	--({\sx*(0.9800)},{\sy*(-0.8711)})
	--({\sx*(0.9900)},{\sy*(-0.8725)})
	--({\sx*(1.0000)},{\sy*(-0.8706)})
	--({\sx*(1.0100)},{\sy*(-0.8655)})
	--({\sx*(1.0200)},{\sy*(-0.8572)})
	--({\sx*(1.0300)},{\sy*(-0.8457)})
	--({\sx*(1.0400)},{\sy*(-0.8312)})
	--({\sx*(1.0500)},{\sy*(-0.8137)})
	--({\sx*(1.0600)},{\sy*(-0.7932)})
	--({\sx*(1.0700)},{\sy*(-0.7699)})
	--({\sx*(1.0800)},{\sy*(-0.7438)})
	--({\sx*(1.0900)},{\sy*(-0.7150)})
	--({\sx*(1.1000)},{\sy*(-0.6836)})
	--({\sx*(1.1100)},{\sy*(-0.6499)})
	--({\sx*(1.1200)},{\sy*(-0.6137)})
	--({\sx*(1.1300)},{\sy*(-0.5755)})
	--({\sx*(1.1400)},{\sy*(-0.5351)})
	--({\sx*(1.1500)},{\sy*(-0.4929)})
	--({\sx*(1.1600)},{\sy*(-0.4488)})
	--({\sx*(1.1700)},{\sy*(-0.4032)})
	--({\sx*(1.1800)},{\sy*(-0.3562)})
	--({\sx*(1.1900)},{\sy*(-0.3078)})
	--({\sx*(1.2000)},{\sy*(-0.2584)})
	--({\sx*(1.2100)},{\sy*(-0.2079)})
	--({\sx*(1.2200)},{\sy*(-0.1567)})
	--({\sx*(1.2300)},{\sy*(-0.1049)})
	--({\sx*(1.2400)},{\sy*(-0.0526)})
	--({\sx*(1.2500)},{\sy*(0.0000)})
	--({\sx*(1.2600)},{\sy*(0.0527)})
	--({\sx*(1.2700)},{\sy*(0.1053)})
	--({\sx*(1.2800)},{\sy*(0.1578)})
	--({\sx*(1.2900)},{\sy*(0.2098)})
	--({\sx*(1.3000)},{\sy*(0.2613)})
	--({\sx*(1.3100)},{\sy*(0.3121)})
	--({\sx*(1.3200)},{\sy*(0.3621)})
	--({\sx*(1.3300)},{\sy*(0.4110)})
	--({\sx*(1.3400)},{\sy*(0.4588)})
	--({\sx*(1.3500)},{\sy*(0.5053)})
	--({\sx*(1.3600)},{\sy*(0.5504)})
	--({\sx*(1.3700)},{\sy*(0.5939)})
	--({\sx*(1.3800)},{\sy*(0.6358)})
	--({\sx*(1.3900)},{\sy*(0.6758)})
	--({\sx*(1.4000)},{\sy*(0.7139)})
	--({\sx*(1.4100)},{\sy*(0.7500)})
	--({\sx*(1.4200)},{\sy*(0.7839)})
	--({\sx*(1.4300)},{\sy*(0.8157)})
	--({\sx*(1.4400)},{\sy*(0.8451)})
	--({\sx*(1.4500)},{\sy*(0.8722)})
	--({\sx*(1.4600)},{\sy*(0.8968)})
	--({\sx*(1.4700)},{\sy*(0.9189)})
	--({\sx*(1.4800)},{\sy*(0.9384)})
	--({\sx*(1.4900)},{\sy*(0.9553)})
	--({\sx*(1.5000)},{\sy*(0.9695)})
	--({\sx*(1.5100)},{\sy*(0.9811)})
	--({\sx*(1.5200)},{\sy*(0.9899)})
	--({\sx*(1.5300)},{\sy*(0.9960)})
	--({\sx*(1.5400)},{\sy*(0.9994)})
	--({\sx*(1.5500)},{\sy*(1.0000)})
	--({\sx*(1.5600)},{\sy*(0.9979)})
	--({\sx*(1.5700)},{\sy*(0.9931)})
	--({\sx*(1.5800)},{\sy*(0.9856)})
	--({\sx*(1.5900)},{\sy*(0.9754)})
	--({\sx*(1.6000)},{\sy*(0.9626)})
	--({\sx*(1.6100)},{\sy*(0.9472)})
	--({\sx*(1.6200)},{\sy*(0.9292)})
	--({\sx*(1.6300)},{\sy*(0.9088)})
	--({\sx*(1.6400)},{\sy*(0.8860)})
	--({\sx*(1.6500)},{\sy*(0.8608)})
	--({\sx*(1.6600)},{\sy*(0.8334)})
	--({\sx*(1.6700)},{\sy*(0.8037)})
	--({\sx*(1.6800)},{\sy*(0.7720)})
	--({\sx*(1.6900)},{\sy*(0.7383)})
	--({\sx*(1.7000)},{\sy*(0.7026)})
	--({\sx*(1.7100)},{\sy*(0.6651)})
	--({\sx*(1.7200)},{\sy*(0.6260)})
	--({\sx*(1.7300)},{\sy*(0.5852)})
	--({\sx*(1.7400)},{\sy*(0.5429)})
	--({\sx*(1.7500)},{\sy*(0.4993)})
	--({\sx*(1.7600)},{\sy*(0.4544)})
	--({\sx*(1.7700)},{\sy*(0.4084)})
	--({\sx*(1.7800)},{\sy*(0.3614)})
	--({\sx*(1.7900)},{\sy*(0.3135)})
	--({\sx*(1.8000)},{\sy*(0.2649)})
	--({\sx*(1.8100)},{\sy*(0.2156)})
	--({\sx*(1.8200)},{\sy*(0.1659)})
	--({\sx*(1.8300)},{\sy*(0.1157)})
	--({\sx*(1.8400)},{\sy*(0.0654)})
	--({\sx*(1.8500)},{\sy*(0.0149)})
	--({\sx*(1.8600)},{\sy*(-0.0355)})
	--({\sx*(1.8700)},{\sy*(-0.0858)})
	--({\sx*(1.8800)},{\sy*(-0.1359)})
	--({\sx*(1.8900)},{\sy*(-0.1855)})
	--({\sx*(1.9000)},{\sy*(-0.2346)})
	--({\sx*(1.9100)},{\sy*(-0.2830)})
	--({\sx*(1.9200)},{\sy*(-0.3307)})
	--({\sx*(1.9300)},{\sy*(-0.3774)})
	--({\sx*(1.9400)},{\sy*(-0.4232)})
	--({\sx*(1.9500)},{\sy*(-0.4679)})
	--({\sx*(1.9600)},{\sy*(-0.5113)})
	--({\sx*(1.9700)},{\sy*(-0.5534)})
	--({\sx*(1.9800)},{\sy*(-0.5941)})
	--({\sx*(1.9900)},{\sy*(-0.6333)})
	--({\sx*(2.0000)},{\sy*(-0.6708)})
	--({\sx*(2.0100)},{\sy*(-0.7067)})
	--({\sx*(2.0200)},{\sy*(-0.7407)})
	--({\sx*(2.0300)},{\sy*(-0.7730)})
	--({\sx*(2.0400)},{\sy*(-0.8032)})
	--({\sx*(2.0500)},{\sy*(-0.8315)})
	--({\sx*(2.0600)},{\sy*(-0.8577)})
	--({\sx*(2.0700)},{\sy*(-0.8818)})
	--({\sx*(2.0800)},{\sy*(-0.9037)})
	--({\sx*(2.0900)},{\sy*(-0.9234)})
	--({\sx*(2.1000)},{\sy*(-0.9409)})
	--({\sx*(2.1100)},{\sy*(-0.9560)})
	--({\sx*(2.1200)},{\sy*(-0.9689)})
	--({\sx*(2.1300)},{\sy*(-0.9794)})
	--({\sx*(2.1400)},{\sy*(-0.9875)})
	--({\sx*(2.1500)},{\sy*(-0.9932)})
	--({\sx*(2.1600)},{\sy*(-0.9966)})
	--({\sx*(2.1700)},{\sy*(-0.9976)})
	--({\sx*(2.1800)},{\sy*(-0.9963)})
	--({\sx*(2.1900)},{\sy*(-0.9926)})
	--({\sx*(2.2000)},{\sy*(-0.9865)})
	--({\sx*(2.2100)},{\sy*(-0.9781)})
	--({\sx*(2.2200)},{\sy*(-0.9675)})
	--({\sx*(2.2300)},{\sy*(-0.9545)})
	--({\sx*(2.2400)},{\sy*(-0.9394)})
	--({\sx*(2.2500)},{\sy*(-0.9221)})
	--({\sx*(2.2600)},{\sy*(-0.9027)})
	--({\sx*(2.2700)},{\sy*(-0.8812)})
	--({\sx*(2.2800)},{\sy*(-0.8577)})
	--({\sx*(2.2900)},{\sy*(-0.8323)})
	--({\sx*(2.3000)},{\sy*(-0.8050)})
	--({\sx*(2.3100)},{\sy*(-0.7759)})
	--({\sx*(2.3200)},{\sy*(-0.7451)})
	--({\sx*(2.3300)},{\sy*(-0.7126)})
	--({\sx*(2.3400)},{\sy*(-0.6785)})
	--({\sx*(2.3500)},{\sy*(-0.6430)})
	--({\sx*(2.3600)},{\sy*(-0.6061)})
	--({\sx*(2.3700)},{\sy*(-0.5679)})
	--({\sx*(2.3800)},{\sy*(-0.5285)})
	--({\sx*(2.3900)},{\sy*(-0.4881)})
	--({\sx*(2.4000)},{\sy*(-0.4466)})
	--({\sx*(2.4100)},{\sy*(-0.4042)})
	--({\sx*(2.4200)},{\sy*(-0.3610)})
	--({\sx*(2.4300)},{\sy*(-0.3172)})
	--({\sx*(2.4400)},{\sy*(-0.2727)})
	--({\sx*(2.4500)},{\sy*(-0.2278)})
	--({\sx*(2.4600)},{\sy*(-0.1826)})
	--({\sx*(2.4700)},{\sy*(-0.1370)})
	--({\sx*(2.4800)},{\sy*(-0.0914)})
	--({\sx*(2.4900)},{\sy*(-0.0456)})
	--({\sx*(2.5000)},{\sy*(0.0000)})
	--({\sx*(2.5100)},{\sy*(0.0455)})
	--({\sx*(2.5200)},{\sy*(0.0906)})
	--({\sx*(2.5300)},{\sy*(0.1354)})
	--({\sx*(2.5400)},{\sy*(0.1797)})
	--({\sx*(2.5500)},{\sy*(0.2234)})
	--({\sx*(2.5600)},{\sy*(0.2664)})
	--({\sx*(2.5700)},{\sy*(0.3086)})
	--({\sx*(2.5800)},{\sy*(0.3499)})
	--({\sx*(2.5900)},{\sy*(0.3902)})
	--({\sx*(2.6000)},{\sy*(0.4295)})
	--({\sx*(2.6100)},{\sy*(0.4675)})
	--({\sx*(2.6200)},{\sy*(0.5044)})
	--({\sx*(2.6300)},{\sy*(0.5398)})
	--({\sx*(2.6400)},{\sy*(0.5739)})
	--({\sx*(2.6500)},{\sy*(0.6065)})
	--({\sx*(2.6600)},{\sy*(0.6375)})
	--({\sx*(2.6700)},{\sy*(0.6668)})
	--({\sx*(2.6800)},{\sy*(0.6945)})
	--({\sx*(2.6900)},{\sy*(0.7205)})
	--({\sx*(2.7000)},{\sy*(0.7446)})
	--({\sx*(2.7100)},{\sy*(0.7669)})
	--({\sx*(2.7200)},{\sy*(0.7872)})
	--({\sx*(2.7300)},{\sy*(0.8057)})
	--({\sx*(2.7400)},{\sy*(0.8221)})
	--({\sx*(2.7500)},{\sy*(0.8366)})
	--({\sx*(2.7600)},{\sy*(0.8490)})
	--({\sx*(2.7700)},{\sy*(0.8593)})
	--({\sx*(2.7800)},{\sy*(0.8676)})
	--({\sx*(2.7900)},{\sy*(0.8738)})
	--({\sx*(2.8000)},{\sy*(0.8779)})
	--({\sx*(2.8100)},{\sy*(0.8799)})
	--({\sx*(2.8200)},{\sy*(0.8798)})
	--({\sx*(2.8300)},{\sy*(0.8776)})
	--({\sx*(2.8400)},{\sy*(0.8733)})
	--({\sx*(2.8500)},{\sy*(0.8671)})
	--({\sx*(2.8600)},{\sy*(0.8587)})
	--({\sx*(2.8700)},{\sy*(0.8484)})
	--({\sx*(2.8800)},{\sy*(0.8361)})
	--({\sx*(2.8900)},{\sy*(0.8219)})
	--({\sx*(2.9000)},{\sy*(0.8058)})
	--({\sx*(2.9100)},{\sy*(0.7879)})
	--({\sx*(2.9200)},{\sy*(0.7681)})
	--({\sx*(2.9300)},{\sy*(0.7467)})
	--({\sx*(2.9400)},{\sy*(0.7235)})
	--({\sx*(2.9500)},{\sy*(0.6988)})
	--({\sx*(2.9600)},{\sy*(0.6725)})
	--({\sx*(2.9700)},{\sy*(0.6447)})
	--({\sx*(2.9800)},{\sy*(0.6155)})
	--({\sx*(2.9900)},{\sy*(0.5850)})
	--({\sx*(3.0000)},{\sy*(0.5532)})
	--({\sx*(3.0100)},{\sy*(0.5203)})
	--({\sx*(3.0200)},{\sy*(0.4863)})
	--({\sx*(3.0300)},{\sy*(0.4513)})
	--({\sx*(3.0400)},{\sy*(0.4154)})
	--({\sx*(3.0500)},{\sy*(0.3787)})
	--({\sx*(3.0600)},{\sy*(0.3413)})
	--({\sx*(3.0700)},{\sy*(0.3033)})
	--({\sx*(3.0800)},{\sy*(0.2647)})
	--({\sx*(3.0900)},{\sy*(0.2257)})
	--({\sx*(3.1000)},{\sy*(0.1864)})
	--({\sx*(3.1100)},{\sy*(0.1468)})
	--({\sx*(3.1200)},{\sy*(0.1072)})
	--({\sx*(3.1300)},{\sy*(0.0675)})
	--({\sx*(3.1400)},{\sy*(0.0278)})
	--({\sx*(3.1500)},{\sy*(-0.0116)})
	--({\sx*(3.1600)},{\sy*(-0.0508)})
	--({\sx*(3.1700)},{\sy*(-0.0896)})
	--({\sx*(3.1800)},{\sy*(-0.1280)})
	--({\sx*(3.1900)},{\sy*(-0.1658)})
	--({\sx*(3.2000)},{\sy*(-0.2030)})
	--({\sx*(3.2100)},{\sy*(-0.2394)})
	--({\sx*(3.2200)},{\sy*(-0.2749)})
	--({\sx*(3.2300)},{\sy*(-0.3096)})
	--({\sx*(3.2400)},{\sy*(-0.3432)})
	--({\sx*(3.2500)},{\sy*(-0.3758)})
	--({\sx*(3.2600)},{\sy*(-0.4072)})
	--({\sx*(3.2700)},{\sy*(-0.4373)})
	--({\sx*(3.2800)},{\sy*(-0.4661)})
	--({\sx*(3.2900)},{\sy*(-0.4936)})
	--({\sx*(3.3000)},{\sy*(-0.5195)})
	--({\sx*(3.3100)},{\sy*(-0.5440)})
	--({\sx*(3.3200)},{\sy*(-0.5669)})
	--({\sx*(3.3300)},{\sy*(-0.5881)})
	--({\sx*(3.3400)},{\sy*(-0.6077)})
	--({\sx*(3.3500)},{\sy*(-0.6255)})
	--({\sx*(3.3600)},{\sy*(-0.6416)})
	--({\sx*(3.3700)},{\sy*(-0.6559)})
	--({\sx*(3.3800)},{\sy*(-0.6683)})
	--({\sx*(3.3900)},{\sy*(-0.6789)})
	--({\sx*(3.4000)},{\sy*(-0.6875)})
	--({\sx*(3.4100)},{\sy*(-0.6943)})
	--({\sx*(3.4200)},{\sy*(-0.6992)})
	--({\sx*(3.4300)},{\sy*(-0.7022)})
	--({\sx*(3.4400)},{\sy*(-0.7032)})
	--({\sx*(3.4500)},{\sy*(-0.7023)})
	--({\sx*(3.4600)},{\sy*(-0.6995)})
	--({\sx*(3.4700)},{\sy*(-0.6949)})
	--({\sx*(3.4800)},{\sy*(-0.6883)})
	--({\sx*(3.4900)},{\sy*(-0.6799)})
	--({\sx*(3.5000)},{\sy*(-0.6697)})
	--({\sx*(3.5100)},{\sy*(-0.6577)})
	--({\sx*(3.5200)},{\sy*(-0.6439)})
	--({\sx*(3.5300)},{\sy*(-0.6285)})
	--({\sx*(3.5400)},{\sy*(-0.6114)})
	--({\sx*(3.5500)},{\sy*(-0.5927)})
	--({\sx*(3.5600)},{\sy*(-0.5725)})
	--({\sx*(3.5700)},{\sy*(-0.5507)})
	--({\sx*(3.5800)},{\sy*(-0.5276)})
	--({\sx*(3.5900)},{\sy*(-0.5032)})
	--({\sx*(3.6000)},{\sy*(-0.4774)})
	--({\sx*(3.6100)},{\sy*(-0.4505)})
	--({\sx*(3.6200)},{\sy*(-0.4225)})
	--({\sx*(3.6300)},{\sy*(-0.3935)})
	--({\sx*(3.6400)},{\sy*(-0.3635)})
	--({\sx*(3.6500)},{\sy*(-0.3327)})
	--({\sx*(3.6600)},{\sy*(-0.3011)})
	--({\sx*(3.6700)},{\sy*(-0.2689)})
	--({\sx*(3.6800)},{\sy*(-0.2362)})
	--({\sx*(3.6900)},{\sy*(-0.2030)})
	--({\sx*(3.7000)},{\sy*(-0.1694)})
	--({\sx*(3.7100)},{\sy*(-0.1356)})
	--({\sx*(3.7200)},{\sy*(-0.1017)})
	--({\sx*(3.7300)},{\sy*(-0.0677)})
	--({\sx*(3.7400)},{\sy*(-0.0338)})
	--({\sx*(3.7500)},{\sy*(0.0000)})
	--({\sx*(3.7600)},{\sy*(0.0335)})
	--({\sx*(3.7700)},{\sy*(0.0666)})
	--({\sx*(3.7800)},{\sy*(0.0992)})
	--({\sx*(3.7900)},{\sy*(0.1312)})
	--({\sx*(3.8000)},{\sy*(0.1626)})
	--({\sx*(3.8100)},{\sy*(0.1931)})
	--({\sx*(3.8200)},{\sy*(0.2228)})
	--({\sx*(3.8300)},{\sy*(0.2515)})
	--({\sx*(3.8400)},{\sy*(0.2792)})
	--({\sx*(3.8500)},{\sy*(0.3057)})
	--({\sx*(3.8600)},{\sy*(0.3310)})
	--({\sx*(3.8700)},{\sy*(0.3549)})
	--({\sx*(3.8800)},{\sy*(0.3775)})
	--({\sx*(3.8900)},{\sy*(0.3986)})
	--({\sx*(3.9000)},{\sy*(0.4181)})
	--({\sx*(3.9100)},{\sy*(0.4361)})
	--({\sx*(3.9200)},{\sy*(0.4524)})
	--({\sx*(3.9300)},{\sy*(0.4670)})
	--({\sx*(3.9400)},{\sy*(0.4799)})
	--({\sx*(3.9500)},{\sy*(0.4910)})
	--({\sx*(3.9600)},{\sy*(0.5002)})
	--({\sx*(3.9700)},{\sy*(0.5076)})
	--({\sx*(3.9800)},{\sy*(0.5131)})
	--({\sx*(3.9900)},{\sy*(0.5166)})
	--({\sx*(4.0000)},{\sy*(0.5183)})
	--({\sx*(4.0100)},{\sy*(0.5181)})
	--({\sx*(4.0200)},{\sy*(0.5160)})
	--({\sx*(4.0300)},{\sy*(0.5120)})
	--({\sx*(4.0400)},{\sy*(0.5060)})
	--({\sx*(4.0500)},{\sy*(0.4983)})
	--({\sx*(4.0600)},{\sy*(0.4887)})
	--({\sx*(4.0700)},{\sy*(0.4774)})
	--({\sx*(4.0800)},{\sy*(0.4643)})
	--({\sx*(4.0900)},{\sy*(0.4495)})
	--({\sx*(4.1000)},{\sy*(0.4331)})
	--({\sx*(4.1100)},{\sy*(0.4152)})
	--({\sx*(4.1200)},{\sy*(0.3957)})
	--({\sx*(4.1300)},{\sy*(0.3749)})
	--({\sx*(4.1400)},{\sy*(0.3527)})
	--({\sx*(4.1500)},{\sy*(0.3294)})
	--({\sx*(4.1600)},{\sy*(0.3048)})
	--({\sx*(4.1700)},{\sy*(0.2793)})
	--({\sx*(4.1800)},{\sy*(0.2529)})
	--({\sx*(4.1900)},{\sy*(0.2256)})
	--({\sx*(4.2000)},{\sy*(0.1977)})
	--({\sx*(4.2100)},{\sy*(0.1692)})
	--({\sx*(4.2200)},{\sy*(0.1402)})
	--({\sx*(4.2300)},{\sy*(0.1110)})
	--({\sx*(4.2400)},{\sy*(0.0816)})
	--({\sx*(4.2500)},{\sy*(0.0521)})
	--({\sx*(4.2600)},{\sy*(0.0227)})
	--({\sx*(4.2700)},{\sy*(-0.0065)})
	--({\sx*(4.2800)},{\sy*(-0.0353)})
	--({\sx*(4.2900)},{\sy*(-0.0636)})
	--({\sx*(4.3000)},{\sy*(-0.0913)})
	--({\sx*(4.3100)},{\sy*(-0.1182)})
	--({\sx*(4.3200)},{\sy*(-0.1442)})
	--({\sx*(4.3300)},{\sy*(-0.1692)})
	--({\sx*(4.3400)},{\sy*(-0.1930)})
	--({\sx*(4.3500)},{\sy*(-0.2156)})
	--({\sx*(4.3600)},{\sy*(-0.2367)})
	--({\sx*(4.3700)},{\sy*(-0.2564)})
	--({\sx*(4.3800)},{\sy*(-0.2744)})
	--({\sx*(4.3900)},{\sy*(-0.2907)})
	--({\sx*(4.4000)},{\sy*(-0.3052)})
	--({\sx*(4.4100)},{\sy*(-0.3178)})
	--({\sx*(4.4200)},{\sy*(-0.3285)})
	--({\sx*(4.4300)},{\sy*(-0.3371)})
	--({\sx*(4.4400)},{\sy*(-0.3436)})
	--({\sx*(4.4500)},{\sy*(-0.3481)})
	--({\sx*(4.4600)},{\sy*(-0.3503)})
	--({\sx*(4.4700)},{\sy*(-0.3504)})
	--({\sx*(4.4800)},{\sy*(-0.3482)})
	--({\sx*(4.4900)},{\sy*(-0.3439)})
	--({\sx*(4.5000)},{\sy*(-0.3375)})
	--({\sx*(4.5100)},{\sy*(-0.3289)})
	--({\sx*(4.5200)},{\sy*(-0.3182)})
	--({\sx*(4.5300)},{\sy*(-0.3055)})
	--({\sx*(4.5400)},{\sy*(-0.2908)})
	--({\sx*(4.5500)},{\sy*(-0.2744)})
	--({\sx*(4.5600)},{\sy*(-0.2562)})
	--({\sx*(4.5700)},{\sy*(-0.2363)})
	--({\sx*(4.5800)},{\sy*(-0.2151)})
	--({\sx*(4.5900)},{\sy*(-0.1924)})
	--({\sx*(4.6000)},{\sy*(-0.1687)})
	--({\sx*(4.6100)},{\sy*(-0.1440)})
	--({\sx*(4.6200)},{\sy*(-0.1185)})
	--({\sx*(4.6300)},{\sy*(-0.0925)})
	--({\sx*(4.6400)},{\sy*(-0.0661)})
	--({\sx*(4.6500)},{\sy*(-0.0396)})
	--({\sx*(4.6600)},{\sy*(-0.0132)})
	--({\sx*(4.6700)},{\sy*(0.0128)})
	--({\sx*(4.6800)},{\sy*(0.0381)})
	--({\sx*(4.6900)},{\sy*(0.0626)})
	--({\sx*(4.7000)},{\sy*(0.0859)})
	--({\sx*(4.7100)},{\sy*(0.1078)})
	--({\sx*(4.7200)},{\sy*(0.1280)})
	--({\sx*(4.7300)},{\sy*(0.1463)})
	--({\sx*(4.7400)},{\sy*(0.1624)})
	--({\sx*(4.7500)},{\sy*(0.1762)})
	--({\sx*(4.7600)},{\sy*(0.1874)})
	--({\sx*(4.7700)},{\sy*(0.1958)})
	--({\sx*(4.7800)},{\sy*(0.2012)})
	--({\sx*(4.7900)},{\sy*(0.2036)})
	--({\sx*(4.8000)},{\sy*(0.2028)})
	--({\sx*(4.8100)},{\sy*(0.1987)})
	--({\sx*(4.8200)},{\sy*(0.1914)})
	--({\sx*(4.8300)},{\sy*(0.1809)})
	--({\sx*(4.8400)},{\sy*(0.1673)})
	--({\sx*(4.8500)},{\sy*(0.1509)})
	--({\sx*(4.8600)},{\sy*(0.1317)})
	--({\sx*(4.8700)},{\sy*(0.1103)})
	--({\sx*(4.8800)},{\sy*(0.0870)})
	--({\sx*(4.8900)},{\sy*(0.0623)})
	--({\sx*(4.9000)},{\sy*(0.0370)})
	--({\sx*(4.9100)},{\sy*(0.0118)})
	--({\sx*(4.9200)},{\sy*(-0.0123)})
	--({\sx*(4.9300)},{\sy*(-0.0343)})
	--({\sx*(4.9400)},{\sy*(-0.0530)})
	--({\sx*(4.9500)},{\sy*(-0.0668)})
	--({\sx*(4.9600)},{\sy*(-0.0743)})
	--({\sx*(4.9700)},{\sy*(-0.0735)})
	--({\sx*(4.9800)},{\sy*(-0.0624)})
	--({\sx*(4.9900)},{\sy*(-0.0388)})
	--({\sx*(5.0000)},{\sy*(0.0000)});
}
\def\relfehlerf{
\draw[color=blue,line width=1.4pt,line join=round] ({\sx*(0.000)},{\sy*(0.0000)})
	--({\sx*(0.0100)},{\sy*(0.0000)})
	--({\sx*(0.0200)},{\sy*(0.0000)})
	--({\sx*(0.0300)},{\sy*(0.0000)})
	--({\sx*(0.0400)},{\sy*(0.0000)})
	--({\sx*(0.0500)},{\sy*(0.0000)})
	--({\sx*(0.0600)},{\sy*(0.0000)})
	--({\sx*(0.0700)},{\sy*(0.0000)})
	--({\sx*(0.0800)},{\sy*(0.0000)})
	--({\sx*(0.0900)},{\sy*(-0.0000)})
	--({\sx*(0.1000)},{\sy*(-0.0000)})
	--({\sx*(0.1100)},{\sy*(-0.0000)})
	--({\sx*(0.1200)},{\sy*(-0.0000)})
	--({\sx*(0.1300)},{\sy*(-0.0000)})
	--({\sx*(0.1400)},{\sy*(-0.0000)})
	--({\sx*(0.1500)},{\sy*(-0.0000)})
	--({\sx*(0.1600)},{\sy*(-0.0000)})
	--({\sx*(0.1700)},{\sy*(-0.0000)})
	--({\sx*(0.1800)},{\sy*(-0.0000)})
	--({\sx*(0.1900)},{\sy*(-0.0000)})
	--({\sx*(0.2000)},{\sy*(-0.0000)})
	--({\sx*(0.2100)},{\sy*(-0.0000)})
	--({\sx*(0.2200)},{\sy*(-0.0000)})
	--({\sx*(0.2300)},{\sy*(-0.0000)})
	--({\sx*(0.2400)},{\sy*(-0.0000)})
	--({\sx*(0.2500)},{\sy*(-0.0000)})
	--({\sx*(0.2600)},{\sy*(-0.0000)})
	--({\sx*(0.2700)},{\sy*(-0.0000)})
	--({\sx*(0.2800)},{\sy*(-0.0000)})
	--({\sx*(0.2900)},{\sy*(-0.0000)})
	--({\sx*(0.3000)},{\sy*(-0.0000)})
	--({\sx*(0.3100)},{\sy*(-0.0000)})
	--({\sx*(0.3200)},{\sy*(-0.0000)})
	--({\sx*(0.3300)},{\sy*(-0.0000)})
	--({\sx*(0.3400)},{\sy*(0.0000)})
	--({\sx*(0.3500)},{\sy*(0.0000)})
	--({\sx*(0.3600)},{\sy*(0.0000)})
	--({\sx*(0.3700)},{\sy*(0.0000)})
	--({\sx*(0.3800)},{\sy*(0.0000)})
	--({\sx*(0.3900)},{\sy*(0.0000)})
	--({\sx*(0.4000)},{\sy*(0.0000)})
	--({\sx*(0.4100)},{\sy*(0.0000)})
	--({\sx*(0.4200)},{\sy*(0.0000)})
	--({\sx*(0.4300)},{\sy*(0.0000)})
	--({\sx*(0.4400)},{\sy*(0.0000)})
	--({\sx*(0.4500)},{\sy*(0.0000)})
	--({\sx*(0.4600)},{\sy*(0.0000)})
	--({\sx*(0.4700)},{\sy*(0.0000)})
	--({\sx*(0.4800)},{\sy*(0.0000)})
	--({\sx*(0.4900)},{\sy*(0.0000)})
	--({\sx*(0.5000)},{\sy*(0.0000)})
	--({\sx*(0.5100)},{\sy*(0.0000)})
	--({\sx*(0.5200)},{\sy*(0.0000)})
	--({\sx*(0.5300)},{\sy*(0.0000)})
	--({\sx*(0.5400)},{\sy*(0.0000)})
	--({\sx*(0.5500)},{\sy*(0.0000)})
	--({\sx*(0.5600)},{\sy*(0.0000)})
	--({\sx*(0.5700)},{\sy*(0.0000)})
	--({\sx*(0.5800)},{\sy*(0.0000)})
	--({\sx*(0.5900)},{\sy*(0.0000)})
	--({\sx*(0.6000)},{\sy*(0.0000)})
	--({\sx*(0.6100)},{\sy*(0.0000)})
	--({\sx*(0.6200)},{\sy*(0.0000)})
	--({\sx*(0.6300)},{\sy*(0.0000)})
	--({\sx*(0.6400)},{\sy*(0.0000)})
	--({\sx*(0.6500)},{\sy*(0.0000)})
	--({\sx*(0.6600)},{\sy*(0.0000)})
	--({\sx*(0.6700)},{\sy*(0.0000)})
	--({\sx*(0.6800)},{\sy*(0.0000)})
	--({\sx*(0.6900)},{\sy*(0.0000)})
	--({\sx*(0.7000)},{\sy*(0.0000)})
	--({\sx*(0.7100)},{\sy*(0.0000)})
	--({\sx*(0.7200)},{\sy*(0.0000)})
	--({\sx*(0.7300)},{\sy*(0.0000)})
	--({\sx*(0.7400)},{\sy*(-0.0000)})
	--({\sx*(0.7500)},{\sy*(-0.0000)})
	--({\sx*(0.7600)},{\sy*(-0.0000)})
	--({\sx*(0.7700)},{\sy*(-0.0000)})
	--({\sx*(0.7800)},{\sy*(-0.0000)})
	--({\sx*(0.7900)},{\sy*(-0.0000)})
	--({\sx*(0.8000)},{\sy*(-0.0000)})
	--({\sx*(0.8100)},{\sy*(-0.0000)})
	--({\sx*(0.8200)},{\sy*(-0.0000)})
	--({\sx*(0.8300)},{\sy*(-0.0000)})
	--({\sx*(0.8400)},{\sy*(-0.0000)})
	--({\sx*(0.8500)},{\sy*(-0.0000)})
	--({\sx*(0.8600)},{\sy*(-0.0000)})
	--({\sx*(0.8700)},{\sy*(-0.0000)})
	--({\sx*(0.8800)},{\sy*(-0.0000)})
	--({\sx*(0.8900)},{\sy*(-0.0000)})
	--({\sx*(0.9000)},{\sy*(-0.0000)})
	--({\sx*(0.9100)},{\sy*(-0.0000)})
	--({\sx*(0.9200)},{\sy*(-0.0000)})
	--({\sx*(0.9300)},{\sy*(-0.0000)})
	--({\sx*(0.9400)},{\sy*(-0.0000)})
	--({\sx*(0.9500)},{\sy*(-0.0000)})
	--({\sx*(0.9600)},{\sy*(-0.0000)})
	--({\sx*(0.9700)},{\sy*(-0.0000)})
	--({\sx*(0.9800)},{\sy*(-0.0000)})
	--({\sx*(0.9900)},{\sy*(-0.0000)})
	--({\sx*(1.0000)},{\sy*(-0.0000)})
	--({\sx*(1.0100)},{\sy*(-0.0000)})
	--({\sx*(1.0200)},{\sy*(-0.0000)})
	--({\sx*(1.0300)},{\sy*(-0.0000)})
	--({\sx*(1.0400)},{\sy*(-0.0000)})
	--({\sx*(1.0500)},{\sy*(-0.0000)})
	--({\sx*(1.0600)},{\sy*(-0.0000)})
	--({\sx*(1.0700)},{\sy*(-0.0000)})
	--({\sx*(1.0800)},{\sy*(-0.0000)})
	--({\sx*(1.0900)},{\sy*(-0.0000)})
	--({\sx*(1.1000)},{\sy*(-0.0000)})
	--({\sx*(1.1100)},{\sy*(-0.0000)})
	--({\sx*(1.1200)},{\sy*(-0.0000)})
	--({\sx*(1.1300)},{\sy*(-0.0000)})
	--({\sx*(1.1400)},{\sy*(-0.0000)})
	--({\sx*(1.1500)},{\sy*(-0.0000)})
	--({\sx*(1.1600)},{\sy*(-0.0000)})
	--({\sx*(1.1700)},{\sy*(-0.0000)})
	--({\sx*(1.1800)},{\sy*(-0.0000)})
	--({\sx*(1.1900)},{\sy*(-0.0000)})
	--({\sx*(1.2000)},{\sy*(-0.0000)})
	--({\sx*(1.2100)},{\sy*(-0.0000)})
	--({\sx*(1.2200)},{\sy*(-0.0000)})
	--({\sx*(1.2300)},{\sy*(-0.0000)})
	--({\sx*(1.2400)},{\sy*(-0.0000)})
	--({\sx*(1.2500)},{\sy*(0.0000)})
	--({\sx*(1.2600)},{\sy*(0.0000)})
	--({\sx*(1.2700)},{\sy*(0.0000)})
	--({\sx*(1.2800)},{\sy*(0.0000)})
	--({\sx*(1.2900)},{\sy*(0.0000)})
	--({\sx*(1.3000)},{\sy*(0.0000)})
	--({\sx*(1.3100)},{\sy*(0.0000)})
	--({\sx*(1.3200)},{\sy*(0.0000)})
	--({\sx*(1.3300)},{\sy*(0.0000)})
	--({\sx*(1.3400)},{\sy*(0.0000)})
	--({\sx*(1.3500)},{\sy*(0.0000)})
	--({\sx*(1.3600)},{\sy*(0.0000)})
	--({\sx*(1.3700)},{\sy*(0.0000)})
	--({\sx*(1.3800)},{\sy*(0.0001)})
	--({\sx*(1.3900)},{\sy*(0.0001)})
	--({\sx*(1.4000)},{\sy*(0.0001)})
	--({\sx*(1.4100)},{\sy*(0.0001)})
	--({\sx*(1.4200)},{\sy*(0.0001)})
	--({\sx*(1.4300)},{\sy*(0.0001)})
	--({\sx*(1.4400)},{\sy*(0.0001)})
	--({\sx*(1.4500)},{\sy*(0.0001)})
	--({\sx*(1.4600)},{\sy*(0.0001)})
	--({\sx*(1.4700)},{\sy*(0.0001)})
	--({\sx*(1.4800)},{\sy*(0.0001)})
	--({\sx*(1.4900)},{\sy*(0.0001)})
	--({\sx*(1.5000)},{\sy*(0.0001)})
	--({\sx*(1.5100)},{\sy*(0.0001)})
	--({\sx*(1.5200)},{\sy*(0.0001)})
	--({\sx*(1.5300)},{\sy*(0.0001)})
	--({\sx*(1.5400)},{\sy*(0.0001)})
	--({\sx*(1.5500)},{\sy*(0.0001)})
	--({\sx*(1.5600)},{\sy*(0.0001)})
	--({\sx*(1.5700)},{\sy*(0.0001)})
	--({\sx*(1.5800)},{\sy*(0.0001)})
	--({\sx*(1.5900)},{\sy*(0.0001)})
	--({\sx*(1.6000)},{\sy*(0.0001)})
	--({\sx*(1.6100)},{\sy*(0.0001)})
	--({\sx*(1.6200)},{\sy*(0.0001)})
	--({\sx*(1.6300)},{\sy*(0.0001)})
	--({\sx*(1.6400)},{\sy*(0.0001)})
	--({\sx*(1.6500)},{\sy*(0.0001)})
	--({\sx*(1.6600)},{\sy*(0.0001)})
	--({\sx*(1.6700)},{\sy*(0.0001)})
	--({\sx*(1.6800)},{\sy*(0.0001)})
	--({\sx*(1.6900)},{\sy*(0.0001)})
	--({\sx*(1.7000)},{\sy*(0.0001)})
	--({\sx*(1.7100)},{\sy*(0.0001)})
	--({\sx*(1.7200)},{\sy*(0.0001)})
	--({\sx*(1.7300)},{\sy*(0.0001)})
	--({\sx*(1.7400)},{\sy*(0.0001)})
	--({\sx*(1.7500)},{\sy*(0.0001)})
	--({\sx*(1.7600)},{\sy*(0.0001)})
	--({\sx*(1.7700)},{\sy*(0.0001)})
	--({\sx*(1.7800)},{\sy*(0.0001)})
	--({\sx*(1.7900)},{\sy*(0.0000)})
	--({\sx*(1.8000)},{\sy*(0.0000)})
	--({\sx*(1.8100)},{\sy*(0.0000)})
	--({\sx*(1.8200)},{\sy*(0.0000)})
	--({\sx*(1.8300)},{\sy*(0.0000)})
	--({\sx*(1.8400)},{\sy*(0.0000)})
	--({\sx*(1.8500)},{\sy*(0.0000)})
	--({\sx*(1.8600)},{\sy*(-0.0000)})
	--({\sx*(1.8700)},{\sy*(-0.0000)})
	--({\sx*(1.8800)},{\sy*(-0.0000)})
	--({\sx*(1.8900)},{\sy*(-0.0000)})
	--({\sx*(1.9000)},{\sy*(-0.0000)})
	--({\sx*(1.9100)},{\sy*(-0.0001)})
	--({\sx*(1.9200)},{\sy*(-0.0001)})
	--({\sx*(1.9300)},{\sy*(-0.0001)})
	--({\sx*(1.9400)},{\sy*(-0.0001)})
	--({\sx*(1.9500)},{\sy*(-0.0001)})
	--({\sx*(1.9600)},{\sy*(-0.0001)})
	--({\sx*(1.9700)},{\sy*(-0.0001)})
	--({\sx*(1.9800)},{\sy*(-0.0001)})
	--({\sx*(1.9900)},{\sy*(-0.0001)})
	--({\sx*(2.0000)},{\sy*(-0.0002)})
	--({\sx*(2.0100)},{\sy*(-0.0002)})
	--({\sx*(2.0200)},{\sy*(-0.0002)})
	--({\sx*(2.0300)},{\sy*(-0.0002)})
	--({\sx*(2.0400)},{\sy*(-0.0002)})
	--({\sx*(2.0500)},{\sy*(-0.0002)})
	--({\sx*(2.0600)},{\sy*(-0.0002)})
	--({\sx*(2.0700)},{\sy*(-0.0002)})
	--({\sx*(2.0800)},{\sy*(-0.0002)})
	--({\sx*(2.0900)},{\sy*(-0.0003)})
	--({\sx*(2.1000)},{\sy*(-0.0003)})
	--({\sx*(2.1100)},{\sy*(-0.0003)})
	--({\sx*(2.1200)},{\sy*(-0.0003)})
	--({\sx*(2.1300)},{\sy*(-0.0003)})
	--({\sx*(2.1400)},{\sy*(-0.0003)})
	--({\sx*(2.1500)},{\sy*(-0.0003)})
	--({\sx*(2.1600)},{\sy*(-0.0003)})
	--({\sx*(2.1700)},{\sy*(-0.0003)})
	--({\sx*(2.1800)},{\sy*(-0.0003)})
	--({\sx*(2.1900)},{\sy*(-0.0003)})
	--({\sx*(2.2000)},{\sy*(-0.0004)})
	--({\sx*(2.2100)},{\sy*(-0.0004)})
	--({\sx*(2.2200)},{\sy*(-0.0004)})
	--({\sx*(2.2300)},{\sy*(-0.0004)})
	--({\sx*(2.2400)},{\sy*(-0.0004)})
	--({\sx*(2.2500)},{\sy*(-0.0004)})
	--({\sx*(2.2600)},{\sy*(-0.0004)})
	--({\sx*(2.2700)},{\sy*(-0.0004)})
	--({\sx*(2.2800)},{\sy*(-0.0004)})
	--({\sx*(2.2900)},{\sy*(-0.0004)})
	--({\sx*(2.3000)},{\sy*(-0.0004)})
	--({\sx*(2.3100)},{\sy*(-0.0004)})
	--({\sx*(2.3200)},{\sy*(-0.0003)})
	--({\sx*(2.3300)},{\sy*(-0.0003)})
	--({\sx*(2.3400)},{\sy*(-0.0003)})
	--({\sx*(2.3500)},{\sy*(-0.0003)})
	--({\sx*(2.3600)},{\sy*(-0.0003)})
	--({\sx*(2.3700)},{\sy*(-0.0003)})
	--({\sx*(2.3800)},{\sy*(-0.0003)})
	--({\sx*(2.3900)},{\sy*(-0.0003)})
	--({\sx*(2.4000)},{\sy*(-0.0003)})
	--({\sx*(2.4100)},{\sy*(-0.0002)})
	--({\sx*(2.4200)},{\sy*(-0.0002)})
	--({\sx*(2.4300)},{\sy*(-0.0002)})
	--({\sx*(2.4400)},{\sy*(-0.0002)})
	--({\sx*(2.4500)},{\sy*(-0.0001)})
	--({\sx*(2.4600)},{\sy*(-0.0001)})
	--({\sx*(2.4700)},{\sy*(-0.0001)})
	--({\sx*(2.4800)},{\sy*(-0.0001)})
	--({\sx*(2.4900)},{\sy*(-0.0000)})
	--({\sx*(2.5000)},{\sy*(0.0000)})
	--({\sx*(2.5100)},{\sy*(0.0000)})
	--({\sx*(2.5200)},{\sy*(0.0001)})
	--({\sx*(2.5300)},{\sy*(0.0001)})
	--({\sx*(2.5400)},{\sy*(0.0001)})
	--({\sx*(2.5500)},{\sy*(0.0002)})
	--({\sx*(2.5600)},{\sy*(0.0002)})
	--({\sx*(2.5700)},{\sy*(0.0003)})
	--({\sx*(2.5800)},{\sy*(0.0003)})
	--({\sx*(2.5900)},{\sy*(0.0004)})
	--({\sx*(2.6000)},{\sy*(0.0004)})
	--({\sx*(2.6100)},{\sy*(0.0004)})
	--({\sx*(2.6200)},{\sy*(0.0005)})
	--({\sx*(2.6300)},{\sy*(0.0005)})
	--({\sx*(2.6400)},{\sy*(0.0006)})
	--({\sx*(2.6500)},{\sy*(0.0006)})
	--({\sx*(2.6600)},{\sy*(0.0007)})
	--({\sx*(2.6700)},{\sy*(0.0007)})
	--({\sx*(2.6800)},{\sy*(0.0008)})
	--({\sx*(2.6900)},{\sy*(0.0008)})
	--({\sx*(2.7000)},{\sy*(0.0009)})
	--({\sx*(2.7100)},{\sy*(0.0010)})
	--({\sx*(2.7200)},{\sy*(0.0010)})
	--({\sx*(2.7300)},{\sy*(0.0011)})
	--({\sx*(2.7400)},{\sy*(0.0011)})
	--({\sx*(2.7500)},{\sy*(0.0012)})
	--({\sx*(2.7600)},{\sy*(0.0012)})
	--({\sx*(2.7700)},{\sy*(0.0013)})
	--({\sx*(2.7800)},{\sy*(0.0013)})
	--({\sx*(2.7900)},{\sy*(0.0013)})
	--({\sx*(2.8000)},{\sy*(0.0014)})
	--({\sx*(2.8100)},{\sy*(0.0014)})
	--({\sx*(2.8200)},{\sy*(0.0015)})
	--({\sx*(2.8300)},{\sy*(0.0015)})
	--({\sx*(2.8400)},{\sy*(0.0016)})
	--({\sx*(2.8500)},{\sy*(0.0016)})
	--({\sx*(2.8600)},{\sy*(0.0016)})
	--({\sx*(2.8700)},{\sy*(0.0016)})
	--({\sx*(2.8800)},{\sy*(0.0017)})
	--({\sx*(2.8900)},{\sy*(0.0017)})
	--({\sx*(2.9000)},{\sy*(0.0017)})
	--({\sx*(2.9100)},{\sy*(0.0017)})
	--({\sx*(2.9200)},{\sy*(0.0017)})
	--({\sx*(2.9300)},{\sy*(0.0017)})
	--({\sx*(2.9400)},{\sy*(0.0017)})
	--({\sx*(2.9500)},{\sy*(0.0017)})
	--({\sx*(2.9600)},{\sy*(0.0017)})
	--({\sx*(2.9700)},{\sy*(0.0017)})
	--({\sx*(2.9800)},{\sy*(0.0016)})
	--({\sx*(2.9900)},{\sy*(0.0016)})
	--({\sx*(3.0000)},{\sy*(0.0016)})
	--({\sx*(3.0100)},{\sy*(0.0015)})
	--({\sx*(3.0200)},{\sy*(0.0015)})
	--({\sx*(3.0300)},{\sy*(0.0014)})
	--({\sx*(3.0400)},{\sy*(0.0013)})
	--({\sx*(3.0500)},{\sy*(0.0012)})
	--({\sx*(3.0600)},{\sy*(0.0012)})
	--({\sx*(3.0700)},{\sy*(0.0011)})
	--({\sx*(3.0800)},{\sy*(0.0010)})
	--({\sx*(3.0900)},{\sy*(0.0008)})
	--({\sx*(3.1000)},{\sy*(0.0007)})
	--({\sx*(3.1100)},{\sy*(0.0006)})
	--({\sx*(3.1200)},{\sy*(0.0004)})
	--({\sx*(3.1300)},{\sy*(0.0003)})
	--({\sx*(3.1400)},{\sy*(0.0001)})
	--({\sx*(3.1500)},{\sy*(-0.0001)})
	--({\sx*(3.1600)},{\sy*(-0.0002)})
	--({\sx*(3.1700)},{\sy*(-0.0004)})
	--({\sx*(3.1800)},{\sy*(-0.0006)})
	--({\sx*(3.1900)},{\sy*(-0.0008)})
	--({\sx*(3.2000)},{\sy*(-0.0011)})
	--({\sx*(3.2100)},{\sy*(-0.0013)})
	--({\sx*(3.2200)},{\sy*(-0.0015)})
	--({\sx*(3.2300)},{\sy*(-0.0018)})
	--({\sx*(3.2400)},{\sy*(-0.0021)})
	--({\sx*(3.2500)},{\sy*(-0.0023)})
	--({\sx*(3.2600)},{\sy*(-0.0026)})
	--({\sx*(3.2700)},{\sy*(-0.0029)})
	--({\sx*(3.2800)},{\sy*(-0.0032)})
	--({\sx*(3.2900)},{\sy*(-0.0035)})
	--({\sx*(3.3000)},{\sy*(-0.0038)})
	--({\sx*(3.3100)},{\sy*(-0.0041)})
	--({\sx*(3.3200)},{\sy*(-0.0044)})
	--({\sx*(3.3300)},{\sy*(-0.0048)})
	--({\sx*(3.3400)},{\sy*(-0.0051)})
	--({\sx*(3.3500)},{\sy*(-0.0054)})
	--({\sx*(3.3600)},{\sy*(-0.0058)})
	--({\sx*(3.3700)},{\sy*(-0.0061)})
	--({\sx*(3.3800)},{\sy*(-0.0064)})
	--({\sx*(3.3900)},{\sy*(-0.0067)})
	--({\sx*(3.4000)},{\sy*(-0.0071)})
	--({\sx*(3.4100)},{\sy*(-0.0074)})
	--({\sx*(3.4200)},{\sy*(-0.0077)})
	--({\sx*(3.4300)},{\sy*(-0.0080)})
	--({\sx*(3.4400)},{\sy*(-0.0083)})
	--({\sx*(3.4500)},{\sy*(-0.0086)})
	--({\sx*(3.4600)},{\sy*(-0.0089)})
	--({\sx*(3.4700)},{\sy*(-0.0091)})
	--({\sx*(3.4800)},{\sy*(-0.0093)})
	--({\sx*(3.4900)},{\sy*(-0.0096)})
	--({\sx*(3.5000)},{\sy*(-0.0098)})
	--({\sx*(3.5100)},{\sy*(-0.0099)})
	--({\sx*(3.5200)},{\sy*(-0.0101)})
	--({\sx*(3.5300)},{\sy*(-0.0102)})
	--({\sx*(3.5400)},{\sy*(-0.0103)})
	--({\sx*(3.5500)},{\sy*(-0.0103)})
	--({\sx*(3.5600)},{\sy*(-0.0103)})
	--({\sx*(3.5700)},{\sy*(-0.0103)})
	--({\sx*(3.5800)},{\sy*(-0.0102)})
	--({\sx*(3.5900)},{\sy*(-0.0101)})
	--({\sx*(3.6000)},{\sy*(-0.0099)})
	--({\sx*(3.6100)},{\sy*(-0.0097)})
	--({\sx*(3.6200)},{\sy*(-0.0094)})
	--({\sx*(3.6300)},{\sy*(-0.0091)})
	--({\sx*(3.6400)},{\sy*(-0.0087)})
	--({\sx*(3.6500)},{\sy*(-0.0083)})
	--({\sx*(3.6600)},{\sy*(-0.0078)})
	--({\sx*(3.6700)},{\sy*(-0.0072)})
	--({\sx*(3.6800)},{\sy*(-0.0065)})
	--({\sx*(3.6900)},{\sy*(-0.0058)})
	--({\sx*(3.7000)},{\sy*(-0.0050)})
	--({\sx*(3.7100)},{\sy*(-0.0042)})
	--({\sx*(3.7200)},{\sy*(-0.0033)})
	--({\sx*(3.7300)},{\sy*(-0.0022)})
	--({\sx*(3.7400)},{\sy*(-0.0012)})
	--({\sx*(3.7500)},{\sy*(0.0000)})
	--({\sx*(3.7600)},{\sy*(0.0012)})
	--({\sx*(3.7700)},{\sy*(0.0026)})
	--({\sx*(3.7800)},{\sy*(0.0039)})
	--({\sx*(3.7900)},{\sy*(0.0054)})
	--({\sx*(3.8000)},{\sy*(0.0070)})
	--({\sx*(3.8100)},{\sy*(0.0086)})
	--({\sx*(3.8200)},{\sy*(0.0103)})
	--({\sx*(3.8300)},{\sy*(0.0120)})
	--({\sx*(3.8400)},{\sy*(0.0138)})
	--({\sx*(3.8500)},{\sy*(0.0157)})
	--({\sx*(3.8600)},{\sy*(0.0176)})
	--({\sx*(3.8700)},{\sy*(0.0196)})
	--({\sx*(3.8800)},{\sy*(0.0216)})
	--({\sx*(3.8900)},{\sy*(0.0237)})
	--({\sx*(3.9000)},{\sy*(0.0258)})
	--({\sx*(3.9100)},{\sy*(0.0279)})
	--({\sx*(3.9200)},{\sy*(0.0301)})
	--({\sx*(3.9300)},{\sy*(0.0322)})
	--({\sx*(3.9400)},{\sy*(0.0343)})
	--({\sx*(3.9500)},{\sy*(0.0365)})
	--({\sx*(3.9600)},{\sy*(0.0386)})
	--({\sx*(3.9700)},{\sy*(0.0406)})
	--({\sx*(3.9800)},{\sy*(0.0426)})
	--({\sx*(3.9900)},{\sy*(0.0446)})
	--({\sx*(4.0000)},{\sy*(0.0465)})
	--({\sx*(4.0100)},{\sy*(0.0483)})
	--({\sx*(4.0200)},{\sy*(0.0499)})
	--({\sx*(4.0300)},{\sy*(0.0515)})
	--({\sx*(4.0400)},{\sy*(0.0529)})
	--({\sx*(4.0500)},{\sy*(0.0542)})
	--({\sx*(4.0600)},{\sy*(0.0553)})
	--({\sx*(4.0700)},{\sy*(0.0562)})
	--({\sx*(4.0800)},{\sy*(0.0569)})
	--({\sx*(4.0900)},{\sy*(0.0573)})
	--({\sx*(4.1000)},{\sy*(0.0575)})
	--({\sx*(4.1100)},{\sy*(0.0575)})
	--({\sx*(4.1200)},{\sy*(0.0571)})
	--({\sx*(4.1300)},{\sy*(0.0564)})
	--({\sx*(4.1400)},{\sy*(0.0554)})
	--({\sx*(4.1500)},{\sy*(0.0540)})
	--({\sx*(4.1600)},{\sy*(0.0522)})
	--({\sx*(4.1700)},{\sy*(0.0500)})
	--({\sx*(4.1800)},{\sy*(0.0473)})
	--({\sx*(4.1900)},{\sy*(0.0442)})
	--({\sx*(4.2000)},{\sy*(0.0405)})
	--({\sx*(4.2100)},{\sy*(0.0363)})
	--({\sx*(4.2200)},{\sy*(0.0315)})
	--({\sx*(4.2300)},{\sy*(0.0262)})
	--({\sx*(4.2400)},{\sy*(0.0202)})
	--({\sx*(4.2500)},{\sy*(0.0135)})
	--({\sx*(4.2600)},{\sy*(0.0062)})
	--({\sx*(4.2700)},{\sy*(-0.0019)})
	--({\sx*(4.2800)},{\sy*(-0.0107)})
	--({\sx*(4.2900)},{\sy*(-0.0203)})
	--({\sx*(4.3000)},{\sy*(-0.0307)})
	--({\sx*(4.3100)},{\sy*(-0.0420)})
	--({\sx*(4.3200)},{\sy*(-0.0541)})
	--({\sx*(4.3300)},{\sy*(-0.0671)})
	--({\sx*(4.3400)},{\sy*(-0.0810)})
	--({\sx*(4.3500)},{\sy*(-0.0958)})
	--({\sx*(4.3600)},{\sy*(-0.1114)})
	--({\sx*(4.3700)},{\sy*(-0.1279)})
	--({\sx*(4.3800)},{\sy*(-0.1452)})
	--({\sx*(4.3900)},{\sy*(-0.1633)})
	--({\sx*(4.4000)},{\sy*(-0.1820)})
	--({\sx*(4.4100)},{\sy*(-0.2013)})
	--({\sx*(4.4200)},{\sy*(-0.2210)})
	--({\sx*(4.4300)},{\sy*(-0.2410)})
	--({\sx*(4.4400)},{\sy*(-0.2609)})
	--({\sx*(4.4500)},{\sy*(-0.2806)})
	--({\sx*(4.4600)},{\sy*(-0.2996)})
	--({\sx*(4.4700)},{\sy*(-0.3177)})
	--({\sx*(4.4800)},{\sy*(-0.3344)})
	--({\sx*(4.4900)},{\sy*(-0.3493)})
	--({\sx*(4.5000)},{\sy*(-0.3618)})
	--({\sx*(4.5100)},{\sy*(-0.3714)})
	--({\sx*(4.5200)},{\sy*(-0.3777)})
	--({\sx*(4.5300)},{\sy*(-0.3801)})
	--({\sx*(4.5400)},{\sy*(-0.3781)})
	--({\sx*(4.5500)},{\sy*(-0.3715)})
	--({\sx*(4.5600)},{\sy*(-0.3599)})
	--({\sx*(4.5700)},{\sy*(-0.3434)})
	--({\sx*(4.5800)},{\sy*(-0.3218)})
	--({\sx*(4.5900)},{\sy*(-0.2955)})
	--({\sx*(4.6000)},{\sy*(-0.2648)})
	--({\sx*(4.6100)},{\sy*(-0.2301)})
	--({\sx*(4.6200)},{\sy*(-0.1922)})
	--({\sx*(4.6300)},{\sy*(-0.1518)})
	--({\sx*(4.6400)},{\sy*(-0.1094)})
	--({\sx*(4.6500)},{\sy*(-0.0660)})
	--({\sx*(4.6600)},{\sy*(-0.0221)})
	--({\sx*(4.6700)},{\sy*(0.0214)})
	--({\sx*(4.6800)},{\sy*(0.0641)})
	--({\sx*(4.6900)},{\sy*(0.1055)})
	--({\sx*(4.7000)},{\sy*(0.1450)})
	--({\sx*(4.7100)},{\sy*(0.1824)})
	--({\sx*(4.7200)},{\sy*(0.2174)})
	--({\sx*(4.7300)},{\sy*(0.2498)})
	--({\sx*(4.7400)},{\sy*(0.2793)})
	--({\sx*(4.7500)},{\sy*(0.3060)})
	--({\sx*(4.7600)},{\sy*(0.3296)})
	--({\sx*(4.7700)},{\sy*(0.3501)})
	--({\sx*(4.7800)},{\sy*(0.3674)})
	--({\sx*(4.7900)},{\sy*(0.3813)})
	--({\sx*(4.8000)},{\sy*(0.3918)})
	--({\sx*(4.8100)},{\sy*(0.3984)})
	--({\sx*(4.8200)},{\sy*(0.4010)})
	--({\sx*(4.8300)},{\sy*(0.3990)})
	--({\sx*(4.8400)},{\sy*(0.3920)})
	--({\sx*(4.8500)},{\sy*(0.3789)})
	--({\sx*(4.8600)},{\sy*(0.3586)})
	--({\sx*(4.8700)},{\sy*(0.3295)})
	--({\sx*(4.8800)},{\sy*(0.2892)})
	--({\sx*(4.8900)},{\sy*(0.2344)})
	--({\sx*(4.9000)},{\sy*(0.1603)})
	--({\sx*(4.9100)},{\sy*(0.0602)})
	--({\sx*(4.9200)},{\sy*(-0.0754)})
	--({\sx*(4.9300)},{\sy*(-0.2584)})
	--({\sx*(4.9400)},{\sy*(-0.4990)})
	--({\sx*(4.9500)},{\sy*(-0.7896)})
	--({\sx*(4.9600)},{\sy*(-1.0632)})
	--({\sx*(4.9700)},{\sy*(-1.1544)})
	--({\sx*(4.9800)},{\sy*(-0.9170)})
	--({\sx*(4.9900)},{\sy*(-0.4543)})
	--({\sx*(5.0000)},{\sy*(0.0000)});
}
\def\xwerteg{
\fill[color=red] (0.0000,0) circle[radius={0.07/\skala}];
\fill[color=white] (0.0000,0) circle[radius={0.05/\skala}];
\fill[color=red] (0.0627,0) circle[radius={0.07/\skala}];
\fill[color=white] (0.0627,0) circle[radius={0.05/\skala}];
\fill[color=red] (0.2476,0) circle[radius={0.07/\skala}];
\fill[color=white] (0.2476,0) circle[radius={0.05/\skala}];
\fill[color=red] (0.5454,0) circle[radius={0.07/\skala}];
\fill[color=white] (0.5454,0) circle[radius={0.05/\skala}];
\fill[color=red] (0.9413,0) circle[radius={0.07/\skala}];
\fill[color=white] (0.9413,0) circle[radius={0.05/\skala}];
\fill[color=red] (1.4153,0) circle[radius={0.07/\skala}];
\fill[color=white] (1.4153,0) circle[radius={0.05/\skala}];
\fill[color=red] (1.9437,0) circle[radius={0.07/\skala}];
\fill[color=white] (1.9437,0) circle[radius={0.05/\skala}];
\fill[color=red] (2.5000,0) circle[radius={0.07/\skala}];
\fill[color=white] (2.5000,0) circle[radius={0.05/\skala}];
\fill[color=red] (3.0563,0) circle[radius={0.07/\skala}];
\fill[color=white] (3.0563,0) circle[radius={0.05/\skala}];
\fill[color=red] (3.5847,0) circle[radius={0.07/\skala}];
\fill[color=white] (3.5847,0) circle[radius={0.05/\skala}];
\fill[color=red] (4.0587,0) circle[radius={0.07/\skala}];
\fill[color=white] (4.0587,0) circle[radius={0.05/\skala}];
\fill[color=red] (4.4546,0) circle[radius={0.07/\skala}];
\fill[color=white] (4.4546,0) circle[radius={0.05/\skala}];
\fill[color=red] (4.7524,0) circle[radius={0.07/\skala}];
\fill[color=white] (4.7524,0) circle[radius={0.05/\skala}];
\fill[color=red] (4.9373,0) circle[radius={0.07/\skala}];
\fill[color=white] (4.9373,0) circle[radius={0.05/\skala}];
\fill[color=red] (5.0000,0) circle[radius={0.07/\skala}];
\fill[color=white] (5.0000,0) circle[radius={0.05/\skala}];
}
\def\punkteg{14}
\def\maxfehlerg{8.862\cdot 10^{-7}}
\def\fehlerg{
\draw[color=red,line width=1.4pt,line join=round] ({\sx*(0.000)},{\sy*(0.0000)})
	--({\sx*(0.0100)},{\sy*(-0.1171)})
	--({\sx*(0.0200)},{\sy*(-0.1706)})
	--({\sx*(0.0300)},{\sy*(-0.1757)})
	--({\sx*(0.0400)},{\sy*(-0.1454)})
	--({\sx*(0.0500)},{\sy*(-0.0906)})
	--({\sx*(0.0600)},{\sy*(-0.0204)})
	--({\sx*(0.0700)},{\sy*(0.0577)})
	--({\sx*(0.0800)},{\sy*(0.1375)})
	--({\sx*(0.0900)},{\sy*(0.2144)})
	--({\sx*(0.1000)},{\sy*(0.2844)})
	--({\sx*(0.1100)},{\sy*(0.3449)})
	--({\sx*(0.1200)},{\sy*(0.3938)})
	--({\sx*(0.1300)},{\sy*(0.4299)})
	--({\sx*(0.1400)},{\sy*(0.4525)})
	--({\sx*(0.1500)},{\sy*(0.4615)})
	--({\sx*(0.1600)},{\sy*(0.4572)})
	--({\sx*(0.1700)},{\sy*(0.4401)})
	--({\sx*(0.1800)},{\sy*(0.4111)})
	--({\sx*(0.1900)},{\sy*(0.3713)})
	--({\sx*(0.2000)},{\sy*(0.3220)})
	--({\sx*(0.2100)},{\sy*(0.2645)})
	--({\sx*(0.2200)},{\sy*(0.2003)})
	--({\sx*(0.2300)},{\sy*(0.1307)})
	--({\sx*(0.2400)},{\sy*(0.0573)})
	--({\sx*(0.2500)},{\sy*(-0.0185)})
	--({\sx*(0.2600)},{\sy*(-0.0953)})
	--({\sx*(0.2700)},{\sy*(-0.1719)})
	--({\sx*(0.2800)},{\sy*(-0.2468)})
	--({\sx*(0.2900)},{\sy*(-0.3191)})
	--({\sx*(0.3000)},{\sy*(-0.3877)})
	--({\sx*(0.3100)},{\sy*(-0.4516)})
	--({\sx*(0.3200)},{\sy*(-0.5100)})
	--({\sx*(0.3300)},{\sy*(-0.5622)})
	--({\sx*(0.3400)},{\sy*(-0.6076)})
	--({\sx*(0.3500)},{\sy*(-0.6457)})
	--({\sx*(0.3600)},{\sy*(-0.6761)})
	--({\sx*(0.3700)},{\sy*(-0.6987)})
	--({\sx*(0.3800)},{\sy*(-0.7132)})
	--({\sx*(0.3900)},{\sy*(-0.7196)})
	--({\sx*(0.4000)},{\sy*(-0.7179)})
	--({\sx*(0.4100)},{\sy*(-0.7084)})
	--({\sx*(0.4200)},{\sy*(-0.6911)})
	--({\sx*(0.4300)},{\sy*(-0.6664)})
	--({\sx*(0.4400)},{\sy*(-0.6346)})
	--({\sx*(0.4500)},{\sy*(-0.5962)})
	--({\sx*(0.4600)},{\sy*(-0.5516)})
	--({\sx*(0.4700)},{\sy*(-0.5013)})
	--({\sx*(0.4800)},{\sy*(-0.4459)})
	--({\sx*(0.4900)},{\sy*(-0.3860)})
	--({\sx*(0.5000)},{\sy*(-0.3221)})
	--({\sx*(0.5100)},{\sy*(-0.2550)})
	--({\sx*(0.5200)},{\sy*(-0.1851)})
	--({\sx*(0.5300)},{\sy*(-0.1133)})
	--({\sx*(0.5400)},{\sy*(-0.0401)})
	--({\sx*(0.5500)},{\sy*(0.0339)})
	--({\sx*(0.5600)},{\sy*(0.1080)})
	--({\sx*(0.5700)},{\sy*(0.1817)})
	--({\sx*(0.5800)},{\sy*(0.2542)})
	--({\sx*(0.5900)},{\sy*(0.3252)})
	--({\sx*(0.6000)},{\sy*(0.3939)})
	--({\sx*(0.6100)},{\sy*(0.4600)})
	--({\sx*(0.6200)},{\sy*(0.5230)})
	--({\sx*(0.6300)},{\sy*(0.5823)})
	--({\sx*(0.6400)},{\sy*(0.6377)})
	--({\sx*(0.6500)},{\sy*(0.6887)})
	--({\sx*(0.6600)},{\sy*(0.7350)})
	--({\sx*(0.6700)},{\sy*(0.7763)})
	--({\sx*(0.6800)},{\sy*(0.8125)})
	--({\sx*(0.6900)},{\sy*(0.8432)})
	--({\sx*(0.7000)},{\sy*(0.8684)})
	--({\sx*(0.7100)},{\sy*(0.8878)})
	--({\sx*(0.7200)},{\sy*(0.9016)})
	--({\sx*(0.7300)},{\sy*(0.9095)})
	--({\sx*(0.7400)},{\sy*(0.9116)})
	--({\sx*(0.7500)},{\sy*(0.9079)})
	--({\sx*(0.7600)},{\sy*(0.8986)})
	--({\sx*(0.7700)},{\sy*(0.8837)})
	--({\sx*(0.7800)},{\sy*(0.8633)})
	--({\sx*(0.7900)},{\sy*(0.8377)})
	--({\sx*(0.8000)},{\sy*(0.8070)})
	--({\sx*(0.8100)},{\sy*(0.7715)})
	--({\sx*(0.8200)},{\sy*(0.7314)})
	--({\sx*(0.8300)},{\sy*(0.6870)})
	--({\sx*(0.8400)},{\sy*(0.6386)})
	--({\sx*(0.8500)},{\sy*(0.5864)})
	--({\sx*(0.8600)},{\sy*(0.5309)})
	--({\sx*(0.8700)},{\sy*(0.4724)})
	--({\sx*(0.8800)},{\sy*(0.4113)})
	--({\sx*(0.8900)},{\sy*(0.3478)})
	--({\sx*(0.9000)},{\sy*(0.2824)})
	--({\sx*(0.9100)},{\sy*(0.2154)})
	--({\sx*(0.9200)},{\sy*(0.1472)})
	--({\sx*(0.9300)},{\sy*(0.0783)})
	--({\sx*(0.9400)},{\sy*(0.0089)})
	--({\sx*(0.9500)},{\sy*(-0.0606)})
	--({\sx*(0.9600)},{\sy*(-0.1297)})
	--({\sx*(0.9700)},{\sy*(-0.1982)})
	--({\sx*(0.9800)},{\sy*(-0.2657)})
	--({\sx*(0.9900)},{\sy*(-0.3318)})
	--({\sx*(1.0000)},{\sy*(-0.3963)})
	--({\sx*(1.0100)},{\sy*(-0.4587)})
	--({\sx*(1.0200)},{\sy*(-0.5189)})
	--({\sx*(1.0300)},{\sy*(-0.5765)})
	--({\sx*(1.0400)},{\sy*(-0.6312)})
	--({\sx*(1.0500)},{\sy*(-0.6829)})
	--({\sx*(1.0600)},{\sy*(-0.7312)})
	--({\sx*(1.0700)},{\sy*(-0.7761)})
	--({\sx*(1.0800)},{\sy*(-0.8172)})
	--({\sx*(1.0900)},{\sy*(-0.8545)})
	--({\sx*(1.1000)},{\sy*(-0.8877)})
	--({\sx*(1.1100)},{\sy*(-0.9168)})
	--({\sx*(1.1200)},{\sy*(-0.9417)})
	--({\sx*(1.1300)},{\sy*(-0.9622)})
	--({\sx*(1.1400)},{\sy*(-0.9783)})
	--({\sx*(1.1500)},{\sy*(-0.9900)})
	--({\sx*(1.1600)},{\sy*(-0.9972)})
	--({\sx*(1.1700)},{\sy*(-1.0000)})
	--({\sx*(1.1800)},{\sy*(-0.9983)})
	--({\sx*(1.1900)},{\sy*(-0.9923)})
	--({\sx*(1.2000)},{\sy*(-0.9819)})
	--({\sx*(1.2100)},{\sy*(-0.9673)})
	--({\sx*(1.2200)},{\sy*(-0.9485)})
	--({\sx*(1.2300)},{\sy*(-0.9257)})
	--({\sx*(1.2400)},{\sy*(-0.8990)})
	--({\sx*(1.2500)},{\sy*(-0.8685)})
	--({\sx*(1.2600)},{\sy*(-0.8345)})
	--({\sx*(1.2700)},{\sy*(-0.7970)})
	--({\sx*(1.2800)},{\sy*(-0.7562)})
	--({\sx*(1.2900)},{\sy*(-0.7124)})
	--({\sx*(1.3000)},{\sy*(-0.6658)})
	--({\sx*(1.3100)},{\sy*(-0.6166)})
	--({\sx*(1.3200)},{\sy*(-0.5650)})
	--({\sx*(1.3300)},{\sy*(-0.5112)})
	--({\sx*(1.3400)},{\sy*(-0.4555)})
	--({\sx*(1.3500)},{\sy*(-0.3981)})
	--({\sx*(1.3600)},{\sy*(-0.3393)})
	--({\sx*(1.3700)},{\sy*(-0.2794)})
	--({\sx*(1.3800)},{\sy*(-0.2185)})
	--({\sx*(1.3900)},{\sy*(-0.1569)})
	--({\sx*(1.4000)},{\sy*(-0.0950)})
	--({\sx*(1.4100)},{\sy*(-0.0328)})
	--({\sx*(1.4200)},{\sy*(0.0292)})
	--({\sx*(1.4300)},{\sy*(0.0909)})
	--({\sx*(1.4400)},{\sy*(0.1520)})
	--({\sx*(1.4500)},{\sy*(0.2123)})
	--({\sx*(1.4600)},{\sy*(0.2716)})
	--({\sx*(1.4700)},{\sy*(0.3296)})
	--({\sx*(1.4800)},{\sy*(0.3861)})
	--({\sx*(1.4900)},{\sy*(0.4410)})
	--({\sx*(1.5000)},{\sy*(0.4939)})
	--({\sx*(1.5100)},{\sy*(0.5448)})
	--({\sx*(1.5200)},{\sy*(0.5934)})
	--({\sx*(1.5300)},{\sy*(0.6395)})
	--({\sx*(1.5400)},{\sy*(0.6831)})
	--({\sx*(1.5500)},{\sy*(0.7240)})
	--({\sx*(1.5600)},{\sy*(0.7620)})
	--({\sx*(1.5700)},{\sy*(0.7970)})
	--({\sx*(1.5800)},{\sy*(0.8290)})
	--({\sx*(1.5900)},{\sy*(0.8577)})
	--({\sx*(1.6000)},{\sy*(0.8832)})
	--({\sx*(1.6100)},{\sy*(0.9053)})
	--({\sx*(1.6200)},{\sy*(0.9241)})
	--({\sx*(1.6300)},{\sy*(0.9394)})
	--({\sx*(1.6400)},{\sy*(0.9513)})
	--({\sx*(1.6500)},{\sy*(0.9597)})
	--({\sx*(1.6600)},{\sy*(0.9646)})
	--({\sx*(1.6700)},{\sy*(0.9660)})
	--({\sx*(1.6800)},{\sy*(0.9640)})
	--({\sx*(1.6900)},{\sy*(0.9586)})
	--({\sx*(1.7000)},{\sy*(0.9498)})
	--({\sx*(1.7100)},{\sy*(0.9377)})
	--({\sx*(1.7200)},{\sy*(0.9223)})
	--({\sx*(1.7300)},{\sy*(0.9038)})
	--({\sx*(1.7400)},{\sy*(0.8823)})
	--({\sx*(1.7500)},{\sy*(0.8578)})
	--({\sx*(1.7600)},{\sy*(0.8304)})
	--({\sx*(1.7700)},{\sy*(0.8003)})
	--({\sx*(1.7800)},{\sy*(0.7676)})
	--({\sx*(1.7900)},{\sy*(0.7325)})
	--({\sx*(1.8000)},{\sy*(0.6950)})
	--({\sx*(1.8100)},{\sy*(0.6553)})
	--({\sx*(1.8200)},{\sy*(0.6137)})
	--({\sx*(1.8300)},{\sy*(0.5701)})
	--({\sx*(1.8400)},{\sy*(0.5250)})
	--({\sx*(1.8500)},{\sy*(0.4782)})
	--({\sx*(1.8600)},{\sy*(0.4302)})
	--({\sx*(1.8700)},{\sy*(0.3810)})
	--({\sx*(1.8800)},{\sy*(0.3308)})
	--({\sx*(1.8900)},{\sy*(0.2798)})
	--({\sx*(1.9000)},{\sy*(0.2282)})
	--({\sx*(1.9100)},{\sy*(0.1762)})
	--({\sx*(1.9200)},{\sy*(0.1239)})
	--({\sx*(1.9300)},{\sy*(0.0715)})
	--({\sx*(1.9400)},{\sy*(0.0193)})
	--({\sx*(1.9500)},{\sy*(-0.0327)})
	--({\sx*(1.9600)},{\sy*(-0.0842)})
	--({\sx*(1.9700)},{\sy*(-0.1351)})
	--({\sx*(1.9800)},{\sy*(-0.1852)})
	--({\sx*(1.9900)},{\sy*(-0.2343)})
	--({\sx*(2.0000)},{\sy*(-0.2823)})
	--({\sx*(2.0100)},{\sy*(-0.3291)})
	--({\sx*(2.0200)},{\sy*(-0.3744)})
	--({\sx*(2.0300)},{\sy*(-0.4182)})
	--({\sx*(2.0400)},{\sy*(-0.4602)})
	--({\sx*(2.0500)},{\sy*(-0.5005)})
	--({\sx*(2.0600)},{\sy*(-0.5388)})
	--({\sx*(2.0700)},{\sy*(-0.5751)})
	--({\sx*(2.0800)},{\sy*(-0.6093)})
	--({\sx*(2.0900)},{\sy*(-0.6412)})
	--({\sx*(2.1000)},{\sy*(-0.6707)})
	--({\sx*(2.1100)},{\sy*(-0.6979)})
	--({\sx*(2.1200)},{\sy*(-0.7226)})
	--({\sx*(2.1300)},{\sy*(-0.7447)})
	--({\sx*(2.1400)},{\sy*(-0.7643)})
	--({\sx*(2.1500)},{\sy*(-0.7812)})
	--({\sx*(2.1600)},{\sy*(-0.7955)})
	--({\sx*(2.1700)},{\sy*(-0.8071)})
	--({\sx*(2.1800)},{\sy*(-0.8159)})
	--({\sx*(2.1900)},{\sy*(-0.8221)})
	--({\sx*(2.2000)},{\sy*(-0.8256)})
	--({\sx*(2.2100)},{\sy*(-0.8264)})
	--({\sx*(2.2200)},{\sy*(-0.8246)})
	--({\sx*(2.2300)},{\sy*(-0.8201)})
	--({\sx*(2.2400)},{\sy*(-0.8130)})
	--({\sx*(2.2500)},{\sy*(-0.8033)})
	--({\sx*(2.2600)},{\sy*(-0.7912)})
	--({\sx*(2.2700)},{\sy*(-0.7766)})
	--({\sx*(2.2800)},{\sy*(-0.7597)})
	--({\sx*(2.2900)},{\sy*(-0.7405)})
	--({\sx*(2.3000)},{\sy*(-0.7190)})
	--({\sx*(2.3100)},{\sy*(-0.6955)})
	--({\sx*(2.3200)},{\sy*(-0.6699)})
	--({\sx*(2.3300)},{\sy*(-0.6425)})
	--({\sx*(2.3400)},{\sy*(-0.6132)})
	--({\sx*(2.3500)},{\sy*(-0.5822)})
	--({\sx*(2.3600)},{\sy*(-0.5497)})
	--({\sx*(2.3700)},{\sy*(-0.5157)})
	--({\sx*(2.3800)},{\sy*(-0.4803)})
	--({\sx*(2.3900)},{\sy*(-0.4438)})
	--({\sx*(2.4000)},{\sy*(-0.4061)})
	--({\sx*(2.4100)},{\sy*(-0.3675)})
	--({\sx*(2.4200)},{\sy*(-0.3281)})
	--({\sx*(2.4300)},{\sy*(-0.2881)})
	--({\sx*(2.4400)},{\sy*(-0.2475)})
	--({\sx*(2.4500)},{\sy*(-0.2064)})
	--({\sx*(2.4600)},{\sy*(-0.1651)})
	--({\sx*(2.4700)},{\sy*(-0.1237)})
	--({\sx*(2.4800)},{\sy*(-0.0823)})
	--({\sx*(2.4900)},{\sy*(-0.0410)})
	--({\sx*(2.5000)},{\sy*(0.0000)})
	--({\sx*(2.5100)},{\sy*(0.0406)})
	--({\sx*(2.5200)},{\sy*(0.0807)})
	--({\sx*(2.5300)},{\sy*(0.1202)})
	--({\sx*(2.5400)},{\sy*(0.1588)})
	--({\sx*(2.5500)},{\sy*(0.1966)})
	--({\sx*(2.5600)},{\sy*(0.2334)})
	--({\sx*(2.5700)},{\sy*(0.2691)})
	--({\sx*(2.5800)},{\sy*(0.3035)})
	--({\sx*(2.5900)},{\sy*(0.3367)})
	--({\sx*(2.6000)},{\sy*(0.3684)})
	--({\sx*(2.6100)},{\sy*(0.3986)})
	--({\sx*(2.6200)},{\sy*(0.4273)})
	--({\sx*(2.6300)},{\sy*(0.4543)})
	--({\sx*(2.6400)},{\sy*(0.4796)})
	--({\sx*(2.6500)},{\sy*(0.5030)})
	--({\sx*(2.6600)},{\sy*(0.5247)})
	--({\sx*(2.6700)},{\sy*(0.5444)})
	--({\sx*(2.6800)},{\sy*(0.5621)})
	--({\sx*(2.6900)},{\sy*(0.5779)})
	--({\sx*(2.7000)},{\sy*(0.5917)})
	--({\sx*(2.7100)},{\sy*(0.6034)})
	--({\sx*(2.7200)},{\sy*(0.6130)})
	--({\sx*(2.7300)},{\sy*(0.6206)})
	--({\sx*(2.7400)},{\sy*(0.6261)})
	--({\sx*(2.7500)},{\sy*(0.6296)})
	--({\sx*(2.7600)},{\sy*(0.6310)})
	--({\sx*(2.7700)},{\sy*(0.6303)})
	--({\sx*(2.7800)},{\sy*(0.6276)})
	--({\sx*(2.7900)},{\sy*(0.6229)})
	--({\sx*(2.8000)},{\sy*(0.6163)})
	--({\sx*(2.8100)},{\sy*(0.6077)})
	--({\sx*(2.8200)},{\sy*(0.5973)})
	--({\sx*(2.8300)},{\sy*(0.5850)})
	--({\sx*(2.8400)},{\sy*(0.5710)})
	--({\sx*(2.8500)},{\sy*(0.5553)})
	--({\sx*(2.8600)},{\sy*(0.5380)})
	--({\sx*(2.8700)},{\sy*(0.5192)})
	--({\sx*(2.8800)},{\sy*(0.4988)})
	--({\sx*(2.8900)},{\sy*(0.4771)})
	--({\sx*(2.9000)},{\sy*(0.4541)})
	--({\sx*(2.9100)},{\sy*(0.4299)})
	--({\sx*(2.9200)},{\sy*(0.4045)})
	--({\sx*(2.9300)},{\sy*(0.3781)})
	--({\sx*(2.9400)},{\sy*(0.3508)})
	--({\sx*(2.9500)},{\sy*(0.3227)})
	--({\sx*(2.9600)},{\sy*(0.2939)})
	--({\sx*(2.9700)},{\sy*(0.2644)})
	--({\sx*(2.9800)},{\sy*(0.2344)})
	--({\sx*(2.9900)},{\sy*(0.2040)})
	--({\sx*(3.0000)},{\sy*(0.1734)})
	--({\sx*(3.0100)},{\sy*(0.1425)})
	--({\sx*(3.0200)},{\sy*(0.1115)})
	--({\sx*(3.0300)},{\sy*(0.0806)})
	--({\sx*(3.0400)},{\sy*(0.0497)})
	--({\sx*(3.0500)},{\sy*(0.0191)})
	--({\sx*(3.0600)},{\sy*(-0.0112)})
	--({\sx*(3.0700)},{\sy*(-0.0410)})
	--({\sx*(3.0800)},{\sy*(-0.0703)})
	--({\sx*(3.0900)},{\sy*(-0.0991)})
	--({\sx*(3.1000)},{\sy*(-0.1271)})
	--({\sx*(3.1100)},{\sy*(-0.1543)})
	--({\sx*(3.1200)},{\sy*(-0.1806)})
	--({\sx*(3.1300)},{\sy*(-0.2060)})
	--({\sx*(3.1400)},{\sy*(-0.2304)})
	--({\sx*(3.1500)},{\sy*(-0.2536)})
	--({\sx*(3.1600)},{\sy*(-0.2757)})
	--({\sx*(3.1700)},{\sy*(-0.2965)})
	--({\sx*(3.1800)},{\sy*(-0.3160)})
	--({\sx*(3.1900)},{\sy*(-0.3341)})
	--({\sx*(3.2000)},{\sy*(-0.3509)})
	--({\sx*(3.2100)},{\sy*(-0.3662)})
	--({\sx*(3.2200)},{\sy*(-0.3801)})
	--({\sx*(3.2300)},{\sy*(-0.3924)})
	--({\sx*(3.2400)},{\sy*(-0.4032)})
	--({\sx*(3.2500)},{\sy*(-0.4124)})
	--({\sx*(3.2600)},{\sy*(-0.4200)})
	--({\sx*(3.2700)},{\sy*(-0.4261)})
	--({\sx*(3.2800)},{\sy*(-0.4306)})
	--({\sx*(3.2900)},{\sy*(-0.4334)})
	--({\sx*(3.3000)},{\sy*(-0.4347)})
	--({\sx*(3.3100)},{\sy*(-0.4345)})
	--({\sx*(3.3200)},{\sy*(-0.4327)})
	--({\sx*(3.3300)},{\sy*(-0.4293)})
	--({\sx*(3.3400)},{\sy*(-0.4245)})
	--({\sx*(3.3500)},{\sy*(-0.4182)})
	--({\sx*(3.3600)},{\sy*(-0.4105)})
	--({\sx*(3.3700)},{\sy*(-0.4014)})
	--({\sx*(3.3800)},{\sy*(-0.3910)})
	--({\sx*(3.3900)},{\sy*(-0.3793)})
	--({\sx*(3.4000)},{\sy*(-0.3664)})
	--({\sx*(3.4100)},{\sy*(-0.3523)})
	--({\sx*(3.4200)},{\sy*(-0.3372)})
	--({\sx*(3.4300)},{\sy*(-0.3210)})
	--({\sx*(3.4400)},{\sy*(-0.3039)})
	--({\sx*(3.4500)},{\sy*(-0.2859)})
	--({\sx*(3.4600)},{\sy*(-0.2671)})
	--({\sx*(3.4700)},{\sy*(-0.2476)})
	--({\sx*(3.4800)},{\sy*(-0.2275)})
	--({\sx*(3.4900)},{\sy*(-0.2068)})
	--({\sx*(3.5000)},{\sy*(-0.1856)})
	--({\sx*(3.5100)},{\sy*(-0.1641)})
	--({\sx*(3.5200)},{\sy*(-0.1423)})
	--({\sx*(3.5300)},{\sy*(-0.1202)})
	--({\sx*(3.5400)},{\sy*(-0.0981)})
	--({\sx*(3.5500)},{\sy*(-0.0759)})
	--({\sx*(3.5600)},{\sy*(-0.0538)})
	--({\sx*(3.5700)},{\sy*(-0.0319)})
	--({\sx*(3.5800)},{\sy*(-0.0101)})
	--({\sx*(3.5900)},{\sy*(0.0113)})
	--({\sx*(3.6000)},{\sy*(0.0323)})
	--({\sx*(3.6100)},{\sy*(0.0529)})
	--({\sx*(3.6200)},{\sy*(0.0729)})
	--({\sx*(3.6300)},{\sy*(0.0923)})
	--({\sx*(3.6400)},{\sy*(0.1110)})
	--({\sx*(3.6500)},{\sy*(0.1289)})
	--({\sx*(3.6600)},{\sy*(0.1460)})
	--({\sx*(3.6700)},{\sy*(0.1622)})
	--({\sx*(3.6800)},{\sy*(0.1775)})
	--({\sx*(3.6900)},{\sy*(0.1918)})
	--({\sx*(3.7000)},{\sy*(0.2050)})
	--({\sx*(3.7100)},{\sy*(0.2172)})
	--({\sx*(3.7200)},{\sy*(0.2283)})
	--({\sx*(3.7300)},{\sy*(0.2382)})
	--({\sx*(3.7400)},{\sy*(0.2469)})
	--({\sx*(3.7500)},{\sy*(0.2544)})
	--({\sx*(3.7600)},{\sy*(0.2607)})
	--({\sx*(3.7700)},{\sy*(0.2657)})
	--({\sx*(3.7800)},{\sy*(0.2695)})
	--({\sx*(3.7900)},{\sy*(0.2721)})
	--({\sx*(3.8000)},{\sy*(0.2734)})
	--({\sx*(3.8100)},{\sy*(0.2736)})
	--({\sx*(3.8200)},{\sy*(0.2724)})
	--({\sx*(3.8300)},{\sy*(0.2701)})
	--({\sx*(3.8400)},{\sy*(0.2667)})
	--({\sx*(3.8500)},{\sy*(0.2621)})
	--({\sx*(3.8600)},{\sy*(0.2564)})
	--({\sx*(3.8700)},{\sy*(0.2496)})
	--({\sx*(3.8800)},{\sy*(0.2418)})
	--({\sx*(3.8900)},{\sy*(0.2330)})
	--({\sx*(3.9000)},{\sy*(0.2233)})
	--({\sx*(3.9100)},{\sy*(0.2128)})
	--({\sx*(3.9200)},{\sy*(0.2014)})
	--({\sx*(3.9300)},{\sy*(0.1893)})
	--({\sx*(3.9400)},{\sy*(0.1766)})
	--({\sx*(3.9500)},{\sy*(0.1632)})
	--({\sx*(3.9600)},{\sy*(0.1493)})
	--({\sx*(3.9700)},{\sy*(0.1350)})
	--({\sx*(3.9800)},{\sy*(0.1202)})
	--({\sx*(3.9900)},{\sy*(0.1052)})
	--({\sx*(4.0000)},{\sy*(0.0899)})
	--({\sx*(4.0100)},{\sy*(0.0745)})
	--({\sx*(4.0200)},{\sy*(0.0591)})
	--({\sx*(4.0300)},{\sy*(0.0436)})
	--({\sx*(4.0400)},{\sy*(0.0283)})
	--({\sx*(4.0500)},{\sy*(0.0131)})
	--({\sx*(4.0600)},{\sy*(-0.0019)})
	--({\sx*(4.0700)},{\sy*(-0.0165)})
	--({\sx*(4.0800)},{\sy*(-0.0307)})
	--({\sx*(4.0900)},{\sy*(-0.0445)})
	--({\sx*(4.1000)},{\sy*(-0.0577)})
	--({\sx*(4.1100)},{\sy*(-0.0704)})
	--({\sx*(4.1200)},{\sy*(-0.0824)})
	--({\sx*(4.1300)},{\sy*(-0.0936)})
	--({\sx*(4.1400)},{\sy*(-0.1041)})
	--({\sx*(4.1500)},{\sy*(-0.1138)})
	--({\sx*(4.1600)},{\sy*(-0.1226)})
	--({\sx*(4.1700)},{\sy*(-0.1305)})
	--({\sx*(4.1800)},{\sy*(-0.1374)})
	--({\sx*(4.1900)},{\sy*(-0.1434)})
	--({\sx*(4.2000)},{\sy*(-0.1484)})
	--({\sx*(4.2100)},{\sy*(-0.1524)})
	--({\sx*(4.2200)},{\sy*(-0.1554)})
	--({\sx*(4.2300)},{\sy*(-0.1574)})
	--({\sx*(4.2400)},{\sy*(-0.1583)})
	--({\sx*(4.2500)},{\sy*(-0.1582)})
	--({\sx*(4.2600)},{\sy*(-0.1571)})
	--({\sx*(4.2700)},{\sy*(-0.1551)})
	--({\sx*(4.2800)},{\sy*(-0.1521)})
	--({\sx*(4.2900)},{\sy*(-0.1481)})
	--({\sx*(4.3000)},{\sy*(-0.1433)})
	--({\sx*(4.3100)},{\sy*(-0.1376)})
	--({\sx*(4.3200)},{\sy*(-0.1312)})
	--({\sx*(4.3300)},{\sy*(-0.1239)})
	--({\sx*(4.3400)},{\sy*(-0.1161)})
	--({\sx*(4.3500)},{\sy*(-0.1075)})
	--({\sx*(4.3600)},{\sy*(-0.0985)})
	--({\sx*(4.3700)},{\sy*(-0.0889)})
	--({\sx*(4.3800)},{\sy*(-0.0790)})
	--({\sx*(4.3900)},{\sy*(-0.0687)})
	--({\sx*(4.4000)},{\sy*(-0.0582)})
	--({\sx*(4.4100)},{\sy*(-0.0475)})
	--({\sx*(4.4200)},{\sy*(-0.0367)})
	--({\sx*(4.4300)},{\sy*(-0.0259)})
	--({\sx*(4.4400)},{\sy*(-0.0152)})
	--({\sx*(4.4500)},{\sy*(-0.0047)})
	--({\sx*(4.4600)},{\sy*(0.0055)})
	--({\sx*(4.4700)},{\sy*(0.0154)})
	--({\sx*(4.4800)},{\sy*(0.0249)})
	--({\sx*(4.4900)},{\sy*(0.0340)})
	--({\sx*(4.5000)},{\sy*(0.0424)})
	--({\sx*(4.5100)},{\sy*(0.0502)})
	--({\sx*(4.5200)},{\sy*(0.0573)})
	--({\sx*(4.5300)},{\sy*(0.0637)})
	--({\sx*(4.5400)},{\sy*(0.0693)})
	--({\sx*(4.5500)},{\sy*(0.0740)})
	--({\sx*(4.5600)},{\sy*(0.0778)})
	--({\sx*(4.5700)},{\sy*(0.0808)})
	--({\sx*(4.5800)},{\sy*(0.0828)})
	--({\sx*(4.5900)},{\sy*(0.0838)})
	--({\sx*(4.6000)},{\sy*(0.0839)})
	--({\sx*(4.6100)},{\sy*(0.0831)})
	--({\sx*(4.6200)},{\sy*(0.0814)})
	--({\sx*(4.6300)},{\sy*(0.0787)})
	--({\sx*(4.6400)},{\sy*(0.0753)})
	--({\sx*(4.6500)},{\sy*(0.0710)})
	--({\sx*(4.6600)},{\sy*(0.0660)})
	--({\sx*(4.6700)},{\sy*(0.0603)})
	--({\sx*(4.6800)},{\sy*(0.0540)})
	--({\sx*(4.6900)},{\sy*(0.0472)})
	--({\sx*(4.7000)},{\sy*(0.0400)})
	--({\sx*(4.7100)},{\sy*(0.0325)})
	--({\sx*(4.7200)},{\sy*(0.0248)})
	--({\sx*(4.7300)},{\sy*(0.0171)})
	--({\sx*(4.7400)},{\sy*(0.0094)})
	--({\sx*(4.7500)},{\sy*(0.0018)})
	--({\sx*(4.7600)},{\sy*(-0.0055)})
	--({\sx*(4.7700)},{\sy*(-0.0123)})
	--({\sx*(4.7800)},{\sy*(-0.0186)})
	--({\sx*(4.7900)},{\sy*(-0.0243)})
	--({\sx*(4.8000)},{\sy*(-0.0292)})
	--({\sx*(4.8100)},{\sy*(-0.0332)})
	--({\sx*(4.8200)},{\sy*(-0.0362)})
	--({\sx*(4.8300)},{\sy*(-0.0383)})
	--({\sx*(4.8400)},{\sy*(-0.0392)})
	--({\sx*(4.8500)},{\sy*(-0.0391)})
	--({\sx*(4.8600)},{\sy*(-0.0378)})
	--({\sx*(4.8700)},{\sy*(-0.0354)})
	--({\sx*(4.8800)},{\sy*(-0.0319)})
	--({\sx*(4.8900)},{\sy*(-0.0276)})
	--({\sx*(4.9000)},{\sy*(-0.0224)})
	--({\sx*(4.9100)},{\sy*(-0.0167)})
	--({\sx*(4.9200)},{\sy*(-0.0105)})
	--({\sx*(4.9300)},{\sy*(-0.0044)})
	--({\sx*(4.9400)},{\sy*(0.0015)})
	--({\sx*(4.9500)},{\sy*(0.0066)})
	--({\sx*(4.9600)},{\sy*(0.0105)})
	--({\sx*(4.9700)},{\sy*(0.0125)})
	--({\sx*(4.9800)},{\sy*(0.0119)})
	--({\sx*(4.9900)},{\sy*(0.0081)})
	--({\sx*(5.0000)},{\sy*(0.0000)});
}
\def\relfehlerg{
\draw[color=blue,line width=1.4pt,line join=round] ({\sx*(0.000)},{\sy*(0.0000)})
	--({\sx*(0.0100)},{\sy*(-0.0000)})
	--({\sx*(0.0200)},{\sy*(-0.0000)})
	--({\sx*(0.0300)},{\sy*(-0.0000)})
	--({\sx*(0.0400)},{\sy*(-0.0000)})
	--({\sx*(0.0500)},{\sy*(-0.0000)})
	--({\sx*(0.0600)},{\sy*(-0.0000)})
	--({\sx*(0.0700)},{\sy*(0.0000)})
	--({\sx*(0.0800)},{\sy*(0.0000)})
	--({\sx*(0.0900)},{\sy*(0.0000)})
	--({\sx*(0.1000)},{\sy*(0.0000)})
	--({\sx*(0.1100)},{\sy*(0.0000)})
	--({\sx*(0.1200)},{\sy*(0.0000)})
	--({\sx*(0.1300)},{\sy*(0.0000)})
	--({\sx*(0.1400)},{\sy*(0.0000)})
	--({\sx*(0.1500)},{\sy*(0.0000)})
	--({\sx*(0.1600)},{\sy*(0.0000)})
	--({\sx*(0.1700)},{\sy*(0.0000)})
	--({\sx*(0.1800)},{\sy*(0.0000)})
	--({\sx*(0.1900)},{\sy*(0.0000)})
	--({\sx*(0.2000)},{\sy*(0.0000)})
	--({\sx*(0.2100)},{\sy*(0.0000)})
	--({\sx*(0.2200)},{\sy*(0.0000)})
	--({\sx*(0.2300)},{\sy*(0.0000)})
	--({\sx*(0.2400)},{\sy*(0.0000)})
	--({\sx*(0.2500)},{\sy*(-0.0000)})
	--({\sx*(0.2600)},{\sy*(-0.0000)})
	--({\sx*(0.2700)},{\sy*(-0.0000)})
	--({\sx*(0.2800)},{\sy*(-0.0000)})
	--({\sx*(0.2900)},{\sy*(-0.0000)})
	--({\sx*(0.3000)},{\sy*(-0.0000)})
	--({\sx*(0.3100)},{\sy*(-0.0000)})
	--({\sx*(0.3200)},{\sy*(-0.0000)})
	--({\sx*(0.3300)},{\sy*(-0.0000)})
	--({\sx*(0.3400)},{\sy*(-0.0000)})
	--({\sx*(0.3500)},{\sy*(-0.0000)})
	--({\sx*(0.3600)},{\sy*(-0.0000)})
	--({\sx*(0.3700)},{\sy*(-0.0000)})
	--({\sx*(0.3800)},{\sy*(-0.0000)})
	--({\sx*(0.3900)},{\sy*(-0.0000)})
	--({\sx*(0.4000)},{\sy*(-0.0000)})
	--({\sx*(0.4100)},{\sy*(-0.0000)})
	--({\sx*(0.4200)},{\sy*(-0.0000)})
	--({\sx*(0.4300)},{\sy*(-0.0000)})
	--({\sx*(0.4400)},{\sy*(-0.0000)})
	--({\sx*(0.4500)},{\sy*(-0.0000)})
	--({\sx*(0.4600)},{\sy*(-0.0000)})
	--({\sx*(0.4700)},{\sy*(-0.0000)})
	--({\sx*(0.4800)},{\sy*(-0.0000)})
	--({\sx*(0.4900)},{\sy*(-0.0000)})
	--({\sx*(0.5000)},{\sy*(-0.0000)})
	--({\sx*(0.5100)},{\sy*(-0.0000)})
	--({\sx*(0.5200)},{\sy*(-0.0000)})
	--({\sx*(0.5300)},{\sy*(-0.0000)})
	--({\sx*(0.5400)},{\sy*(-0.0000)})
	--({\sx*(0.5500)},{\sy*(0.0000)})
	--({\sx*(0.5600)},{\sy*(0.0000)})
	--({\sx*(0.5700)},{\sy*(0.0000)})
	--({\sx*(0.5800)},{\sy*(0.0000)})
	--({\sx*(0.5900)},{\sy*(0.0000)})
	--({\sx*(0.6000)},{\sy*(0.0000)})
	--({\sx*(0.6100)},{\sy*(0.0000)})
	--({\sx*(0.6200)},{\sy*(0.0000)})
	--({\sx*(0.6300)},{\sy*(0.0000)})
	--({\sx*(0.6400)},{\sy*(0.0000)})
	--({\sx*(0.6500)},{\sy*(0.0000)})
	--({\sx*(0.6600)},{\sy*(0.0000)})
	--({\sx*(0.6700)},{\sy*(0.0000)})
	--({\sx*(0.6800)},{\sy*(0.0000)})
	--({\sx*(0.6900)},{\sy*(0.0000)})
	--({\sx*(0.7000)},{\sy*(0.0000)})
	--({\sx*(0.7100)},{\sy*(0.0000)})
	--({\sx*(0.7200)},{\sy*(0.0000)})
	--({\sx*(0.7300)},{\sy*(0.0000)})
	--({\sx*(0.7400)},{\sy*(0.0000)})
	--({\sx*(0.7500)},{\sy*(0.0000)})
	--({\sx*(0.7600)},{\sy*(0.0000)})
	--({\sx*(0.7700)},{\sy*(0.0000)})
	--({\sx*(0.7800)},{\sy*(0.0000)})
	--({\sx*(0.7900)},{\sy*(0.0000)})
	--({\sx*(0.8000)},{\sy*(0.0000)})
	--({\sx*(0.8100)},{\sy*(0.0000)})
	--({\sx*(0.8200)},{\sy*(0.0000)})
	--({\sx*(0.8300)},{\sy*(0.0000)})
	--({\sx*(0.8400)},{\sy*(0.0000)})
	--({\sx*(0.8500)},{\sy*(0.0000)})
	--({\sx*(0.8600)},{\sy*(0.0000)})
	--({\sx*(0.8700)},{\sy*(0.0000)})
	--({\sx*(0.8800)},{\sy*(0.0000)})
	--({\sx*(0.8900)},{\sy*(0.0000)})
	--({\sx*(0.9000)},{\sy*(0.0000)})
	--({\sx*(0.9100)},{\sy*(0.0000)})
	--({\sx*(0.9200)},{\sy*(0.0000)})
	--({\sx*(0.9300)},{\sy*(0.0000)})
	--({\sx*(0.9400)},{\sy*(0.0000)})
	--({\sx*(0.9500)},{\sy*(-0.0000)})
	--({\sx*(0.9600)},{\sy*(-0.0000)})
	--({\sx*(0.9700)},{\sy*(-0.0000)})
	--({\sx*(0.9800)},{\sy*(-0.0000)})
	--({\sx*(0.9900)},{\sy*(-0.0000)})
	--({\sx*(1.0000)},{\sy*(-0.0000)})
	--({\sx*(1.0100)},{\sy*(-0.0000)})
	--({\sx*(1.0200)},{\sy*(-0.0000)})
	--({\sx*(1.0300)},{\sy*(-0.0000)})
	--({\sx*(1.0400)},{\sy*(-0.0000)})
	--({\sx*(1.0500)},{\sy*(-0.0000)})
	--({\sx*(1.0600)},{\sy*(-0.0000)})
	--({\sx*(1.0700)},{\sy*(-0.0000)})
	--({\sx*(1.0800)},{\sy*(-0.0000)})
	--({\sx*(1.0900)},{\sy*(-0.0000)})
	--({\sx*(1.1000)},{\sy*(-0.0000)})
	--({\sx*(1.1100)},{\sy*(-0.0000)})
	--({\sx*(1.1200)},{\sy*(-0.0000)})
	--({\sx*(1.1300)},{\sy*(-0.0000)})
	--({\sx*(1.1400)},{\sy*(-0.0000)})
	--({\sx*(1.1500)},{\sy*(-0.0000)})
	--({\sx*(1.1600)},{\sy*(-0.0000)})
	--({\sx*(1.1700)},{\sy*(-0.0000)})
	--({\sx*(1.1800)},{\sy*(-0.0000)})
	--({\sx*(1.1900)},{\sy*(-0.0000)})
	--({\sx*(1.2000)},{\sy*(-0.0000)})
	--({\sx*(1.2100)},{\sy*(-0.0000)})
	--({\sx*(1.2200)},{\sy*(-0.0000)})
	--({\sx*(1.2300)},{\sy*(-0.0000)})
	--({\sx*(1.2400)},{\sy*(-0.0000)})
	--({\sx*(1.2500)},{\sy*(-0.0000)})
	--({\sx*(1.2600)},{\sy*(-0.0000)})
	--({\sx*(1.2700)},{\sy*(-0.0000)})
	--({\sx*(1.2800)},{\sy*(-0.0000)})
	--({\sx*(1.2900)},{\sy*(-0.0000)})
	--({\sx*(1.3000)},{\sy*(-0.0000)})
	--({\sx*(1.3100)},{\sy*(-0.0000)})
	--({\sx*(1.3200)},{\sy*(-0.0000)})
	--({\sx*(1.3300)},{\sy*(-0.0000)})
	--({\sx*(1.3400)},{\sy*(-0.0000)})
	--({\sx*(1.3500)},{\sy*(-0.0000)})
	--({\sx*(1.3600)},{\sy*(-0.0000)})
	--({\sx*(1.3700)},{\sy*(-0.0000)})
	--({\sx*(1.3800)},{\sy*(-0.0000)})
	--({\sx*(1.3900)},{\sy*(-0.0000)})
	--({\sx*(1.4000)},{\sy*(-0.0000)})
	--({\sx*(1.4100)},{\sy*(-0.0000)})
	--({\sx*(1.4200)},{\sy*(0.0000)})
	--({\sx*(1.4300)},{\sy*(0.0000)})
	--({\sx*(1.4400)},{\sy*(0.0000)})
	--({\sx*(1.4500)},{\sy*(0.0000)})
	--({\sx*(1.4600)},{\sy*(0.0000)})
	--({\sx*(1.4700)},{\sy*(0.0000)})
	--({\sx*(1.4800)},{\sy*(0.0000)})
	--({\sx*(1.4900)},{\sy*(0.0000)})
	--({\sx*(1.5000)},{\sy*(0.0000)})
	--({\sx*(1.5100)},{\sy*(0.0000)})
	--({\sx*(1.5200)},{\sy*(0.0000)})
	--({\sx*(1.5300)},{\sy*(0.0000)})
	--({\sx*(1.5400)},{\sy*(0.0000)})
	--({\sx*(1.5500)},{\sy*(0.0000)})
	--({\sx*(1.5600)},{\sy*(0.0000)})
	--({\sx*(1.5700)},{\sy*(0.0000)})
	--({\sx*(1.5800)},{\sy*(0.0000)})
	--({\sx*(1.5900)},{\sy*(0.0000)})
	--({\sx*(1.6000)},{\sy*(0.0000)})
	--({\sx*(1.6100)},{\sy*(0.0000)})
	--({\sx*(1.6200)},{\sy*(0.0000)})
	--({\sx*(1.6300)},{\sy*(0.0000)})
	--({\sx*(1.6400)},{\sy*(0.0000)})
	--({\sx*(1.6500)},{\sy*(0.0000)})
	--({\sx*(1.6600)},{\sy*(0.0000)})
	--({\sx*(1.6700)},{\sy*(0.0000)})
	--({\sx*(1.6800)},{\sy*(0.0000)})
	--({\sx*(1.6900)},{\sy*(0.0000)})
	--({\sx*(1.7000)},{\sy*(0.0000)})
	--({\sx*(1.7100)},{\sy*(0.0000)})
	--({\sx*(1.7200)},{\sy*(0.0000)})
	--({\sx*(1.7300)},{\sy*(0.0000)})
	--({\sx*(1.7400)},{\sy*(0.0000)})
	--({\sx*(1.7500)},{\sy*(0.0000)})
	--({\sx*(1.7600)},{\sy*(0.0000)})
	--({\sx*(1.7700)},{\sy*(0.0000)})
	--({\sx*(1.7800)},{\sy*(0.0000)})
	--({\sx*(1.7900)},{\sy*(0.0000)})
	--({\sx*(1.8000)},{\sy*(0.0000)})
	--({\sx*(1.8100)},{\sy*(0.0000)})
	--({\sx*(1.8200)},{\sy*(0.0000)})
	--({\sx*(1.8300)},{\sy*(0.0000)})
	--({\sx*(1.8400)},{\sy*(0.0000)})
	--({\sx*(1.8500)},{\sy*(0.0000)})
	--({\sx*(1.8600)},{\sy*(0.0000)})
	--({\sx*(1.8700)},{\sy*(0.0000)})
	--({\sx*(1.8800)},{\sy*(0.0000)})
	--({\sx*(1.8900)},{\sy*(0.0000)})
	--({\sx*(1.9000)},{\sy*(0.0000)})
	--({\sx*(1.9100)},{\sy*(0.0000)})
	--({\sx*(1.9200)},{\sy*(0.0000)})
	--({\sx*(1.9300)},{\sy*(0.0000)})
	--({\sx*(1.9400)},{\sy*(0.0000)})
	--({\sx*(1.9500)},{\sy*(-0.0000)})
	--({\sx*(1.9600)},{\sy*(-0.0000)})
	--({\sx*(1.9700)},{\sy*(-0.0000)})
	--({\sx*(1.9800)},{\sy*(-0.0000)})
	--({\sx*(1.9900)},{\sy*(-0.0000)})
	--({\sx*(2.0000)},{\sy*(-0.0000)})
	--({\sx*(2.0100)},{\sy*(-0.0000)})
	--({\sx*(2.0200)},{\sy*(-0.0000)})
	--({\sx*(2.0300)},{\sy*(-0.0000)})
	--({\sx*(2.0400)},{\sy*(-0.0000)})
	--({\sx*(2.0500)},{\sy*(-0.0000)})
	--({\sx*(2.0600)},{\sy*(-0.0000)})
	--({\sx*(2.0700)},{\sy*(-0.0000)})
	--({\sx*(2.0800)},{\sy*(-0.0000)})
	--({\sx*(2.0900)},{\sy*(-0.0000)})
	--({\sx*(2.1000)},{\sy*(-0.0000)})
	--({\sx*(2.1100)},{\sy*(-0.0000)})
	--({\sx*(2.1200)},{\sy*(-0.0000)})
	--({\sx*(2.1300)},{\sy*(-0.0000)})
	--({\sx*(2.1400)},{\sy*(-0.0000)})
	--({\sx*(2.1500)},{\sy*(-0.0000)})
	--({\sx*(2.1600)},{\sy*(-0.0000)})
	--({\sx*(2.1700)},{\sy*(-0.0000)})
	--({\sx*(2.1800)},{\sy*(-0.0000)})
	--({\sx*(2.1900)},{\sy*(-0.0000)})
	--({\sx*(2.2000)},{\sy*(-0.0000)})
	--({\sx*(2.2100)},{\sy*(-0.0000)})
	--({\sx*(2.2200)},{\sy*(-0.0000)})
	--({\sx*(2.2300)},{\sy*(-0.0000)})
	--({\sx*(2.2400)},{\sy*(-0.0000)})
	--({\sx*(2.2500)},{\sy*(-0.0000)})
	--({\sx*(2.2600)},{\sy*(-0.0000)})
	--({\sx*(2.2700)},{\sy*(-0.0000)})
	--({\sx*(2.2800)},{\sy*(-0.0000)})
	--({\sx*(2.2900)},{\sy*(-0.0000)})
	--({\sx*(2.3000)},{\sy*(-0.0000)})
	--({\sx*(2.3100)},{\sy*(-0.0000)})
	--({\sx*(2.3200)},{\sy*(-0.0000)})
	--({\sx*(2.3300)},{\sy*(-0.0000)})
	--({\sx*(2.3400)},{\sy*(-0.0000)})
	--({\sx*(2.3500)},{\sy*(-0.0000)})
	--({\sx*(2.3600)},{\sy*(-0.0000)})
	--({\sx*(2.3700)},{\sy*(-0.0000)})
	--({\sx*(2.3800)},{\sy*(-0.0000)})
	--({\sx*(2.3900)},{\sy*(-0.0000)})
	--({\sx*(2.4000)},{\sy*(-0.0000)})
	--({\sx*(2.4100)},{\sy*(-0.0000)})
	--({\sx*(2.4200)},{\sy*(-0.0000)})
	--({\sx*(2.4300)},{\sy*(-0.0000)})
	--({\sx*(2.4400)},{\sy*(-0.0000)})
	--({\sx*(2.4500)},{\sy*(-0.0000)})
	--({\sx*(2.4600)},{\sy*(-0.0000)})
	--({\sx*(2.4700)},{\sy*(-0.0000)})
	--({\sx*(2.4800)},{\sy*(-0.0000)})
	--({\sx*(2.4900)},{\sy*(-0.0000)})
	--({\sx*(2.5000)},{\sy*(0.0000)})
	--({\sx*(2.5100)},{\sy*(0.0000)})
	--({\sx*(2.5200)},{\sy*(0.0000)})
	--({\sx*(2.5300)},{\sy*(0.0000)})
	--({\sx*(2.5400)},{\sy*(0.0000)})
	--({\sx*(2.5500)},{\sy*(0.0000)})
	--({\sx*(2.5600)},{\sy*(0.0000)})
	--({\sx*(2.5700)},{\sy*(0.0000)})
	--({\sx*(2.5800)},{\sy*(0.0000)})
	--({\sx*(2.5900)},{\sy*(0.0000)})
	--({\sx*(2.6000)},{\sy*(0.0000)})
	--({\sx*(2.6100)},{\sy*(0.0000)})
	--({\sx*(2.6200)},{\sy*(0.0000)})
	--({\sx*(2.6300)},{\sy*(0.0000)})
	--({\sx*(2.6400)},{\sy*(0.0000)})
	--({\sx*(2.6500)},{\sy*(0.0000)})
	--({\sx*(2.6600)},{\sy*(0.0000)})
	--({\sx*(2.6700)},{\sy*(0.0000)})
	--({\sx*(2.6800)},{\sy*(0.0000)})
	--({\sx*(2.6900)},{\sy*(0.0000)})
	--({\sx*(2.7000)},{\sy*(0.0001)})
	--({\sx*(2.7100)},{\sy*(0.0001)})
	--({\sx*(2.7200)},{\sy*(0.0001)})
	--({\sx*(2.7300)},{\sy*(0.0001)})
	--({\sx*(2.7400)},{\sy*(0.0001)})
	--({\sx*(2.7500)},{\sy*(0.0001)})
	--({\sx*(2.7600)},{\sy*(0.0001)})
	--({\sx*(2.7700)},{\sy*(0.0001)})
	--({\sx*(2.7800)},{\sy*(0.0001)})
	--({\sx*(2.7900)},{\sy*(0.0001)})
	--({\sx*(2.8000)},{\sy*(0.0001)})
	--({\sx*(2.8100)},{\sy*(0.0001)})
	--({\sx*(2.8200)},{\sy*(0.0001)})
	--({\sx*(2.8300)},{\sy*(0.0001)})
	--({\sx*(2.8400)},{\sy*(0.0001)})
	--({\sx*(2.8500)},{\sy*(0.0001)})
	--({\sx*(2.8600)},{\sy*(0.0001)})
	--({\sx*(2.8700)},{\sy*(0.0001)})
	--({\sx*(2.8800)},{\sy*(0.0001)})
	--({\sx*(2.8900)},{\sy*(0.0001)})
	--({\sx*(2.9000)},{\sy*(0.0001)})
	--({\sx*(2.9100)},{\sy*(0.0001)})
	--({\sx*(2.9200)},{\sy*(0.0001)})
	--({\sx*(2.9300)},{\sy*(0.0001)})
	--({\sx*(2.9400)},{\sy*(0.0001)})
	--({\sx*(2.9500)},{\sy*(0.0001)})
	--({\sx*(2.9600)},{\sy*(0.0001)})
	--({\sx*(2.9700)},{\sy*(0.0000)})
	--({\sx*(2.9800)},{\sy*(0.0000)})
	--({\sx*(2.9900)},{\sy*(0.0000)})
	--({\sx*(3.0000)},{\sy*(0.0000)})
	--({\sx*(3.0100)},{\sy*(0.0000)})
	--({\sx*(3.0200)},{\sy*(0.0000)})
	--({\sx*(3.0300)},{\sy*(0.0000)})
	--({\sx*(3.0400)},{\sy*(0.0000)})
	--({\sx*(3.0500)},{\sy*(0.0000)})
	--({\sx*(3.0600)},{\sy*(-0.0000)})
	--({\sx*(3.0700)},{\sy*(-0.0000)})
	--({\sx*(3.0800)},{\sy*(-0.0000)})
	--({\sx*(3.0900)},{\sy*(-0.0000)})
	--({\sx*(3.1000)},{\sy*(-0.0000)})
	--({\sx*(3.1100)},{\sy*(-0.0000)})
	--({\sx*(3.1200)},{\sy*(-0.0001)})
	--({\sx*(3.1300)},{\sy*(-0.0001)})
	--({\sx*(3.1400)},{\sy*(-0.0001)})
	--({\sx*(3.1500)},{\sy*(-0.0001)})
	--({\sx*(3.1600)},{\sy*(-0.0001)})
	--({\sx*(3.1700)},{\sy*(-0.0001)})
	--({\sx*(3.1800)},{\sy*(-0.0001)})
	--({\sx*(3.1900)},{\sy*(-0.0001)})
	--({\sx*(3.2000)},{\sy*(-0.0001)})
	--({\sx*(3.2100)},{\sy*(-0.0001)})
	--({\sx*(3.2200)},{\sy*(-0.0002)})
	--({\sx*(3.2300)},{\sy*(-0.0002)})
	--({\sx*(3.2400)},{\sy*(-0.0002)})
	--({\sx*(3.2500)},{\sy*(-0.0002)})
	--({\sx*(3.2600)},{\sy*(-0.0002)})
	--({\sx*(3.2700)},{\sy*(-0.0002)})
	--({\sx*(3.2800)},{\sy*(-0.0002)})
	--({\sx*(3.2900)},{\sy*(-0.0002)})
	--({\sx*(3.3000)},{\sy*(-0.0002)})
	--({\sx*(3.3100)},{\sy*(-0.0002)})
	--({\sx*(3.3200)},{\sy*(-0.0002)})
	--({\sx*(3.3300)},{\sy*(-0.0002)})
	--({\sx*(3.3400)},{\sy*(-0.0002)})
	--({\sx*(3.3500)},{\sy*(-0.0003)})
	--({\sx*(3.3600)},{\sy*(-0.0003)})
	--({\sx*(3.3700)},{\sy*(-0.0003)})
	--({\sx*(3.3800)},{\sy*(-0.0003)})
	--({\sx*(3.3900)},{\sy*(-0.0003)})
	--({\sx*(3.4000)},{\sy*(-0.0003)})
	--({\sx*(3.4100)},{\sy*(-0.0003)})
	--({\sx*(3.4200)},{\sy*(-0.0003)})
	--({\sx*(3.4300)},{\sy*(-0.0003)})
	--({\sx*(3.4400)},{\sy*(-0.0003)})
	--({\sx*(3.4500)},{\sy*(-0.0002)})
	--({\sx*(3.4600)},{\sy*(-0.0002)})
	--({\sx*(3.4700)},{\sy*(-0.0002)})
	--({\sx*(3.4800)},{\sy*(-0.0002)})
	--({\sx*(3.4900)},{\sy*(-0.0002)})
	--({\sx*(3.5000)},{\sy*(-0.0002)})
	--({\sx*(3.5100)},{\sy*(-0.0002)})
	--({\sx*(3.5200)},{\sy*(-0.0002)})
	--({\sx*(3.5300)},{\sy*(-0.0001)})
	--({\sx*(3.5400)},{\sy*(-0.0001)})
	--({\sx*(3.5500)},{\sy*(-0.0001)})
	--({\sx*(3.5600)},{\sy*(-0.0001)})
	--({\sx*(3.5700)},{\sy*(-0.0000)})
	--({\sx*(3.5800)},{\sy*(-0.0000)})
	--({\sx*(3.5900)},{\sy*(0.0000)})
	--({\sx*(3.6000)},{\sy*(0.0000)})
	--({\sx*(3.6100)},{\sy*(0.0001)})
	--({\sx*(3.6200)},{\sy*(0.0001)})
	--({\sx*(3.6300)},{\sy*(0.0001)})
	--({\sx*(3.6400)},{\sy*(0.0002)})
	--({\sx*(3.6500)},{\sy*(0.0002)})
	--({\sx*(3.6600)},{\sy*(0.0003)})
	--({\sx*(3.6700)},{\sy*(0.0003)})
	--({\sx*(3.6800)},{\sy*(0.0003)})
	--({\sx*(3.6900)},{\sy*(0.0004)})
	--({\sx*(3.7000)},{\sy*(0.0004)})
	--({\sx*(3.7100)},{\sy*(0.0005)})
	--({\sx*(3.7200)},{\sy*(0.0005)})
	--({\sx*(3.7300)},{\sy*(0.0006)})
	--({\sx*(3.7400)},{\sy*(0.0006)})
	--({\sx*(3.7500)},{\sy*(0.0006)})
	--({\sx*(3.7600)},{\sy*(0.0007)})
	--({\sx*(3.7700)},{\sy*(0.0007)})
	--({\sx*(3.7800)},{\sy*(0.0008)})
	--({\sx*(3.7900)},{\sy*(0.0008)})
	--({\sx*(3.8000)},{\sy*(0.0008)})
	--({\sx*(3.8100)},{\sy*(0.0009)})
	--({\sx*(3.8200)},{\sy*(0.0009)})
	--({\sx*(3.8300)},{\sy*(0.0009)})
	--({\sx*(3.8400)},{\sy*(0.0009)})
	--({\sx*(3.8500)},{\sy*(0.0010)})
	--({\sx*(3.8600)},{\sy*(0.0010)})
	--({\sx*(3.8700)},{\sy*(0.0010)})
	--({\sx*(3.8800)},{\sy*(0.0010)})
	--({\sx*(3.8900)},{\sy*(0.0010)})
	--({\sx*(3.9000)},{\sy*(0.0010)})
	--({\sx*(3.9100)},{\sy*(0.0010)})
	--({\sx*(3.9200)},{\sy*(0.0010)})
	--({\sx*(3.9300)},{\sy*(0.0009)})
	--({\sx*(3.9400)},{\sy*(0.0009)})
	--({\sx*(3.9500)},{\sy*(0.0009)})
	--({\sx*(3.9600)},{\sy*(0.0008)})
	--({\sx*(3.9700)},{\sy*(0.0008)})
	--({\sx*(3.9800)},{\sy*(0.0007)})
	--({\sx*(3.9900)},{\sy*(0.0007)})
	--({\sx*(4.0000)},{\sy*(0.0006)})
	--({\sx*(4.0100)},{\sy*(0.0005)})
	--({\sx*(4.0200)},{\sy*(0.0004)})
	--({\sx*(4.0300)},{\sy*(0.0003)})
	--({\sx*(4.0400)},{\sy*(0.0002)})
	--({\sx*(4.0500)},{\sy*(0.0001)})
	--({\sx*(4.0600)},{\sy*(-0.0000)})
	--({\sx*(4.0700)},{\sy*(-0.0001)})
	--({\sx*(4.0800)},{\sy*(-0.0003)})
	--({\sx*(4.0900)},{\sy*(-0.0004)})
	--({\sx*(4.1000)},{\sy*(-0.0006)})
	--({\sx*(4.1100)},{\sy*(-0.0007)})
	--({\sx*(4.1200)},{\sy*(-0.0009)})
	--({\sx*(4.1300)},{\sy*(-0.0011)})
	--({\sx*(4.1400)},{\sy*(-0.0012)})
	--({\sx*(4.1500)},{\sy*(-0.0014)})
	--({\sx*(4.1600)},{\sy*(-0.0016)})
	--({\sx*(4.1700)},{\sy*(-0.0017)})
	--({\sx*(4.1800)},{\sy*(-0.0019)})
	--({\sx*(4.1900)},{\sy*(-0.0021)})
	--({\sx*(4.2000)},{\sy*(-0.0022)})
	--({\sx*(4.2100)},{\sy*(-0.0024)})
	--({\sx*(4.2200)},{\sy*(-0.0025)})
	--({\sx*(4.2300)},{\sy*(-0.0027)})
	--({\sx*(4.2400)},{\sy*(-0.0028)})
	--({\sx*(4.2500)},{\sy*(-0.0029)})
	--({\sx*(4.2600)},{\sy*(-0.0031)})
	--({\sx*(4.2700)},{\sy*(-0.0031)})
	--({\sx*(4.2800)},{\sy*(-0.0032)})
	--({\sx*(4.2900)},{\sy*(-0.0033)})
	--({\sx*(4.3000)},{\sy*(-0.0033)})
	--({\sx*(4.3100)},{\sy*(-0.0033)})
	--({\sx*(4.3200)},{\sy*(-0.0033)})
	--({\sx*(4.3300)},{\sy*(-0.0033)})
	--({\sx*(4.3400)},{\sy*(-0.0032)})
	--({\sx*(4.3500)},{\sy*(-0.0031)})
	--({\sx*(4.3600)},{\sy*(-0.0029)})
	--({\sx*(4.3700)},{\sy*(-0.0028)})
	--({\sx*(4.3800)},{\sy*(-0.0026)})
	--({\sx*(4.3900)},{\sy*(-0.0023)})
	--({\sx*(4.4000)},{\sy*(-0.0021)})
	--({\sx*(4.4100)},{\sy*(-0.0018)})
	--({\sx*(4.4200)},{\sy*(-0.0014)})
	--({\sx*(4.4300)},{\sy*(-0.0011)})
	--({\sx*(4.4400)},{\sy*(-0.0006)})
	--({\sx*(4.4500)},{\sy*(-0.0002)})
	--({\sx*(4.4600)},{\sy*(0.0003)})
	--({\sx*(4.4700)},{\sy*(0.0007)})
	--({\sx*(4.4800)},{\sy*(0.0013)})
	--({\sx*(4.4900)},{\sy*(0.0018)})
	--({\sx*(4.5000)},{\sy*(0.0023)})
	--({\sx*(4.5100)},{\sy*(0.0029)})
	--({\sx*(4.5200)},{\sy*(0.0035)})
	--({\sx*(4.5300)},{\sy*(0.0040)})
	--({\sx*(4.5400)},{\sy*(0.0046)})
	--({\sx*(4.5500)},{\sy*(0.0051)})
	--({\sx*(4.5600)},{\sy*(0.0056)})
	--({\sx*(4.5700)},{\sy*(0.0061)})
	--({\sx*(4.5800)},{\sy*(0.0066)})
	--({\sx*(4.5900)},{\sy*(0.0069)})
	--({\sx*(4.6000)},{\sy*(0.0073)})
	--({\sx*(4.6100)},{\sy*(0.0075)})
	--({\sx*(4.6200)},{\sy*(0.0077)})
	--({\sx*(4.6300)},{\sy*(0.0078)})
	--({\sx*(4.6400)},{\sy*(0.0078)})
	--({\sx*(4.6500)},{\sy*(0.0078)})
	--({\sx*(4.6600)},{\sy*(0.0076)})
	--({\sx*(4.6700)},{\sy*(0.0072)})
	--({\sx*(4.6800)},{\sy*(0.0068)})
	--({\sx*(4.6900)},{\sy*(0.0062)})
	--({\sx*(4.7000)},{\sy*(0.0055)})
	--({\sx*(4.7100)},{\sy*(0.0047)})
	--({\sx*(4.7200)},{\sy*(0.0038)})
	--({\sx*(4.7300)},{\sy*(0.0027)})
	--({\sx*(4.7400)},{\sy*(0.0016)})
	--({\sx*(4.7500)},{\sy*(0.0003)})
	--({\sx*(4.7600)},{\sy*(-0.0010)})
	--({\sx*(4.7700)},{\sy*(-0.0024)})
	--({\sx*(4.7800)},{\sy*(-0.0038)})
	--({\sx*(4.7900)},{\sy*(-0.0052)})
	--({\sx*(4.8000)},{\sy*(-0.0066)})
	--({\sx*(4.8100)},{\sy*(-0.0079)})
	--({\sx*(4.8200)},{\sy*(-0.0090)})
	--({\sx*(4.8300)},{\sy*(-0.0100)})
	--({\sx*(4.8400)},{\sy*(-0.0108)})
	--({\sx*(4.8500)},{\sy*(-0.0112)})
	--({\sx*(4.8600)},{\sy*(-0.0114)})
	--({\sx*(4.8700)},{\sy*(-0.0112)})
	--({\sx*(4.8800)},{\sy*(-0.0106)})
	--({\sx*(4.8900)},{\sy*(-0.0096)})
	--({\sx*(4.9000)},{\sy*(-0.0082)})
	--({\sx*(4.9100)},{\sy*(-0.0064)})
	--({\sx*(4.9200)},{\sy*(-0.0042)})
	--({\sx*(4.9300)},{\sy*(-0.0018)})
	--({\sx*(4.9400)},{\sy*(0.0007)})
	--({\sx*(4.9500)},{\sy*(0.0031)})
	--({\sx*(4.9600)},{\sy*(0.0051)})
	--({\sx*(4.9700)},{\sy*(0.0064)})
	--({\sx*(4.9800)},{\sy*(0.0064)})
	--({\sx*(4.9900)},{\sy*(0.0046)})
	--({\sx*(5.0000)},{\sy*(0.0000)});
}
\def\xwerteh{
\fill[color=red] (0.0000,0) circle[radius={0.07/\skala}];
\fill[color=white] (0.0000,0) circle[radius={0.05/\skala}];
\fill[color=red] (0.0480,0) circle[radius={0.07/\skala}];
\fill[color=white] (0.0480,0) circle[radius={0.05/\skala}];
\fill[color=red] (0.1903,0) circle[radius={0.07/\skala}];
\fill[color=white] (0.1903,0) circle[radius={0.05/\skala}];
\fill[color=red] (0.4213,0) circle[radius={0.07/\skala}];
\fill[color=white] (0.4213,0) circle[radius={0.05/\skala}];
\fill[color=red] (0.7322,0) circle[radius={0.07/\skala}];
\fill[color=white] (0.7322,0) circle[radius={0.05/\skala}];
\fill[color=red] (1.1111,0) circle[radius={0.07/\skala}];
\fill[color=white] (1.1111,0) circle[radius={0.05/\skala}];
\fill[color=red] (1.5433,0) circle[radius={0.07/\skala}];
\fill[color=white] (1.5433,0) circle[radius={0.05/\skala}];
\fill[color=red] (2.0123,0) circle[radius={0.07/\skala}];
\fill[color=white] (2.0123,0) circle[radius={0.05/\skala}];
\fill[color=red] (2.5000,0) circle[radius={0.07/\skala}];
\fill[color=white] (2.5000,0) circle[radius={0.05/\skala}];
\fill[color=red] (2.9877,0) circle[radius={0.07/\skala}];
\fill[color=white] (2.9877,0) circle[radius={0.05/\skala}];
\fill[color=red] (3.4567,0) circle[radius={0.07/\skala}];
\fill[color=white] (3.4567,0) circle[radius={0.05/\skala}];
\fill[color=red] (3.8889,0) circle[radius={0.07/\skala}];
\fill[color=white] (3.8889,0) circle[radius={0.05/\skala}];
\fill[color=red] (4.2678,0) circle[radius={0.07/\skala}];
\fill[color=white] (4.2678,0) circle[radius={0.05/\skala}];
\fill[color=red] (4.5787,0) circle[radius={0.07/\skala}];
\fill[color=white] (4.5787,0) circle[radius={0.05/\skala}];
\fill[color=red] (4.8097,0) circle[radius={0.07/\skala}];
\fill[color=white] (4.8097,0) circle[radius={0.05/\skala}];
\fill[color=red] (4.9520,0) circle[radius={0.07/\skala}];
\fill[color=white] (4.9520,0) circle[radius={0.05/\skala}];
\fill[color=red] (5.0000,0) circle[radius={0.07/\skala}];
\fill[color=white] (5.0000,0) circle[radius={0.05/\skala}];
}
\def\punkteh{16}
\def\maxfehlerh{2.884\cdot 10^{-8}}
\def\fehlerh{
\draw[color=red,line width=1.4pt,line join=round] ({\sx*(0.000)},{\sy*(0.0000)})
	--({\sx*(0.0100)},{\sy*(0.1723)})
	--({\sx*(0.0200)},{\sy*(0.2200)})
	--({\sx*(0.0300)},{\sy*(0.1830)})
	--({\sx*(0.0400)},{\sy*(0.0933)})
	--({\sx*(0.0500)},{\sy*(-0.0243)})
	--({\sx*(0.0600)},{\sy*(-0.1507)})
	--({\sx*(0.0700)},{\sy*(-0.2717)})
	--({\sx*(0.0800)},{\sy*(-0.3773)})
	--({\sx*(0.0900)},{\sy*(-0.4609)})
	--({\sx*(0.1000)},{\sy*(-0.5184)})
	--({\sx*(0.1100)},{\sy*(-0.5482)})
	--({\sx*(0.1200)},{\sy*(-0.5505)})
	--({\sx*(0.1300)},{\sy*(-0.5266)})
	--({\sx*(0.1400)},{\sy*(-0.4788)})
	--({\sx*(0.1500)},{\sy*(-0.4104)})
	--({\sx*(0.1600)},{\sy*(-0.3248)})
	--({\sx*(0.1700)},{\sy*(-0.2259)})
	--({\sx*(0.1800)},{\sy*(-0.1175)})
	--({\sx*(0.1900)},{\sy*(-0.0035)})
	--({\sx*(0.2000)},{\sy*(0.1126)})
	--({\sx*(0.2100)},{\sy*(0.2273)})
	--({\sx*(0.2200)},{\sy*(0.3375)})
	--({\sx*(0.2300)},{\sy*(0.4406)})
	--({\sx*(0.2400)},{\sy*(0.5342)})
	--({\sx*(0.2500)},{\sy*(0.6162)})
	--({\sx*(0.2600)},{\sy*(0.6853)})
	--({\sx*(0.2700)},{\sy*(0.7403)})
	--({\sx*(0.2800)},{\sy*(0.7803)})
	--({\sx*(0.2900)},{\sy*(0.8049)})
	--({\sx*(0.3000)},{\sy*(0.8142)})
	--({\sx*(0.3100)},{\sy*(0.8082)})
	--({\sx*(0.3200)},{\sy*(0.7875)})
	--({\sx*(0.3300)},{\sy*(0.7529)})
	--({\sx*(0.3400)},{\sy*(0.7053)})
	--({\sx*(0.3500)},{\sy*(0.6458)})
	--({\sx*(0.3600)},{\sy*(0.5757)})
	--({\sx*(0.3700)},{\sy*(0.4964)})
	--({\sx*(0.3800)},{\sy*(0.4093)})
	--({\sx*(0.3900)},{\sy*(0.3159)})
	--({\sx*(0.4000)},{\sy*(0.2178)})
	--({\sx*(0.4100)},{\sy*(0.1166)})
	--({\sx*(0.4200)},{\sy*(0.0137)})
	--({\sx*(0.4300)},{\sy*(-0.0894)})
	--({\sx*(0.4400)},{\sy*(-0.1912)})
	--({\sx*(0.4500)},{\sy*(-0.2905)})
	--({\sx*(0.4600)},{\sy*(-0.3860)})
	--({\sx*(0.4700)},{\sy*(-0.4764)})
	--({\sx*(0.4800)},{\sy*(-0.5609)})
	--({\sx*(0.4900)},{\sy*(-0.6385)})
	--({\sx*(0.5000)},{\sy*(-0.7083)})
	--({\sx*(0.5100)},{\sy*(-0.7697)})
	--({\sx*(0.5200)},{\sy*(-0.8221)})
	--({\sx*(0.5300)},{\sy*(-0.8652)})
	--({\sx*(0.5400)},{\sy*(-0.8986)})
	--({\sx*(0.5500)},{\sy*(-0.9222)})
	--({\sx*(0.5600)},{\sy*(-0.9358)})
	--({\sx*(0.5700)},{\sy*(-0.9396)})
	--({\sx*(0.5800)},{\sy*(-0.9337)})
	--({\sx*(0.5900)},{\sy*(-0.9183)})
	--({\sx*(0.6000)},{\sy*(-0.8938)})
	--({\sx*(0.6100)},{\sy*(-0.8607)})
	--({\sx*(0.6200)},{\sy*(-0.8194)})
	--({\sx*(0.6300)},{\sy*(-0.7705)})
	--({\sx*(0.6400)},{\sy*(-0.7146)})
	--({\sx*(0.6500)},{\sy*(-0.6524)})
	--({\sx*(0.6600)},{\sy*(-0.5847)})
	--({\sx*(0.6700)},{\sy*(-0.5121)})
	--({\sx*(0.6800)},{\sy*(-0.4355)})
	--({\sx*(0.6900)},{\sy*(-0.3556)})
	--({\sx*(0.7000)},{\sy*(-0.2732)})
	--({\sx*(0.7100)},{\sy*(-0.1891)})
	--({\sx*(0.7200)},{\sy*(-0.1041)})
	--({\sx*(0.7300)},{\sy*(-0.0190)})
	--({\sx*(0.7400)},{\sy*(0.0656)})
	--({\sx*(0.7500)},{\sy*(0.1488)})
	--({\sx*(0.7600)},{\sy*(0.2301)})
	--({\sx*(0.7700)},{\sy*(0.3087)})
	--({\sx*(0.7800)},{\sy*(0.3841)})
	--({\sx*(0.7900)},{\sy*(0.4558)})
	--({\sx*(0.8000)},{\sy*(0.5231)})
	--({\sx*(0.8100)},{\sy*(0.5856)})
	--({\sx*(0.8200)},{\sy*(0.6430)})
	--({\sx*(0.8300)},{\sy*(0.6949)})
	--({\sx*(0.8400)},{\sy*(0.7410)})
	--({\sx*(0.8500)},{\sy*(0.7810)})
	--({\sx*(0.8600)},{\sy*(0.8148)})
	--({\sx*(0.8700)},{\sy*(0.8423)})
	--({\sx*(0.8800)},{\sy*(0.8632)})
	--({\sx*(0.8900)},{\sy*(0.8777)})
	--({\sx*(0.9000)},{\sy*(0.8858)})
	--({\sx*(0.9100)},{\sy*(0.8875)})
	--({\sx*(0.9200)},{\sy*(0.8829)})
	--({\sx*(0.9300)},{\sy*(0.8723)})
	--({\sx*(0.9400)},{\sy*(0.8558)})
	--({\sx*(0.9500)},{\sy*(0.8336)})
	--({\sx*(0.9600)},{\sy*(0.8061)})
	--({\sx*(0.9700)},{\sy*(0.7735)})
	--({\sx*(0.9800)},{\sy*(0.7362)})
	--({\sx*(0.9900)},{\sy*(0.6946)})
	--({\sx*(1.0000)},{\sy*(0.6490)})
	--({\sx*(1.0100)},{\sy*(0.5999)})
	--({\sx*(1.0200)},{\sy*(0.5475)})
	--({\sx*(1.0300)},{\sy*(0.4925)})
	--({\sx*(1.0400)},{\sy*(0.4352)})
	--({\sx*(1.0500)},{\sy*(0.3760)})
	--({\sx*(1.0600)},{\sy*(0.3155)})
	--({\sx*(1.0700)},{\sy*(0.2539)})
	--({\sx*(1.0800)},{\sy*(0.1919)})
	--({\sx*(1.0900)},{\sy*(0.1297)})
	--({\sx*(1.1000)},{\sy*(0.0677)})
	--({\sx*(1.1100)},{\sy*(0.0065)})
	--({\sx*(1.1200)},{\sy*(-0.0536)})
	--({\sx*(1.1300)},{\sy*(-0.1123)})
	--({\sx*(1.1400)},{\sy*(-0.1693)})
	--({\sx*(1.1500)},{\sy*(-0.2241)})
	--({\sx*(1.1600)},{\sy*(-0.2765)})
	--({\sx*(1.1700)},{\sy*(-0.3263)})
	--({\sx*(1.1800)},{\sy*(-0.3731)})
	--({\sx*(1.1900)},{\sy*(-0.4168)})
	--({\sx*(1.2000)},{\sy*(-0.4571)})
	--({\sx*(1.2100)},{\sy*(-0.4939)})
	--({\sx*(1.2200)},{\sy*(-0.5270)})
	--({\sx*(1.2300)},{\sy*(-0.5564)})
	--({\sx*(1.2400)},{\sy*(-0.5819)})
	--({\sx*(1.2500)},{\sy*(-0.6035)})
	--({\sx*(1.2600)},{\sy*(-0.6212)})
	--({\sx*(1.2700)},{\sy*(-0.6349)})
	--({\sx*(1.2800)},{\sy*(-0.6447)})
	--({\sx*(1.2900)},{\sy*(-0.6506)})
	--({\sx*(1.3000)},{\sy*(-0.6527)})
	--({\sx*(1.3100)},{\sy*(-0.6510)})
	--({\sx*(1.3200)},{\sy*(-0.6458)})
	--({\sx*(1.3300)},{\sy*(-0.6371)})
	--({\sx*(1.3400)},{\sy*(-0.6250)})
	--({\sx*(1.3500)},{\sy*(-0.6098)})
	--({\sx*(1.3600)},{\sy*(-0.5916)})
	--({\sx*(1.3700)},{\sy*(-0.5707)})
	--({\sx*(1.3800)},{\sy*(-0.5471)})
	--({\sx*(1.3900)},{\sy*(-0.5212)})
	--({\sx*(1.4000)},{\sy*(-0.4932)})
	--({\sx*(1.4100)},{\sy*(-0.4633)})
	--({\sx*(1.4200)},{\sy*(-0.4318)})
	--({\sx*(1.4300)},{\sy*(-0.3988)})
	--({\sx*(1.4400)},{\sy*(-0.3647)})
	--({\sx*(1.4500)},{\sy*(-0.3296)})
	--({\sx*(1.4600)},{\sy*(-0.2938)})
	--({\sx*(1.4700)},{\sy*(-0.2577)})
	--({\sx*(1.4800)},{\sy*(-0.2212)})
	--({\sx*(1.4900)},{\sy*(-0.1849)})
	--({\sx*(1.5000)},{\sy*(-0.1487)})
	--({\sx*(1.5100)},{\sy*(-0.1130)})
	--({\sx*(1.5200)},{\sy*(-0.0780)})
	--({\sx*(1.5300)},{\sy*(-0.0438)})
	--({\sx*(1.5400)},{\sy*(-0.0107)})
	--({\sx*(1.5500)},{\sy*(0.0213)})
	--({\sx*(1.5600)},{\sy*(0.0519)})
	--({\sx*(1.5700)},{\sy*(0.0810)})
	--({\sx*(1.5800)},{\sy*(0.1085)})
	--({\sx*(1.5900)},{\sy*(0.1342)})
	--({\sx*(1.6000)},{\sy*(0.1582)})
	--({\sx*(1.6100)},{\sy*(0.1802)})
	--({\sx*(1.6200)},{\sy*(0.2002)})
	--({\sx*(1.6300)},{\sy*(0.2181)})
	--({\sx*(1.6400)},{\sy*(0.2340)})
	--({\sx*(1.6500)},{\sy*(0.2479)})
	--({\sx*(1.6600)},{\sy*(0.2596)})
	--({\sx*(1.6700)},{\sy*(0.2692)})
	--({\sx*(1.6800)},{\sy*(0.2767)})
	--({\sx*(1.6900)},{\sy*(0.2822)})
	--({\sx*(1.7000)},{\sy*(0.2858)})
	--({\sx*(1.7100)},{\sy*(0.2874)})
	--({\sx*(1.7200)},{\sy*(0.2871)})
	--({\sx*(1.7300)},{\sy*(0.2851)})
	--({\sx*(1.7400)},{\sy*(0.2814)})
	--({\sx*(1.7500)},{\sy*(0.2762)})
	--({\sx*(1.7600)},{\sy*(0.2694)})
	--({\sx*(1.7700)},{\sy*(0.2614)})
	--({\sx*(1.7800)},{\sy*(0.2521)})
	--({\sx*(1.7900)},{\sy*(0.2417)})
	--({\sx*(1.8000)},{\sy*(0.2303)})
	--({\sx*(1.8100)},{\sy*(0.2182)})
	--({\sx*(1.8200)},{\sy*(0.2053)})
	--({\sx*(1.8300)},{\sy*(0.1919)})
	--({\sx*(1.8400)},{\sy*(0.1781)})
	--({\sx*(1.8500)},{\sy*(0.1641)})
	--({\sx*(1.8600)},{\sy*(0.1499)})
	--({\sx*(1.8700)},{\sy*(0.1357)})
	--({\sx*(1.8800)},{\sy*(0.1217)})
	--({\sx*(1.8900)},{\sy*(0.1079)})
	--({\sx*(1.9000)},{\sy*(0.0946)})
	--({\sx*(1.9100)},{\sy*(0.0817)})
	--({\sx*(1.9200)},{\sy*(0.0694)})
	--({\sx*(1.9300)},{\sy*(0.0579)})
	--({\sx*(1.9400)},{\sy*(0.0471)})
	--({\sx*(1.9500)},{\sy*(0.0373)})
	--({\sx*(1.9600)},{\sy*(0.0284)})
	--({\sx*(1.9700)},{\sy*(0.0206)})
	--({\sx*(1.9800)},{\sy*(0.0138)})
	--({\sx*(1.9900)},{\sy*(0.0082)})
	--({\sx*(2.0000)},{\sy*(0.0038)})
	--({\sx*(2.0100)},{\sy*(0.0006)})
	--({\sx*(2.0200)},{\sy*(-0.0015)})
	--({\sx*(2.0300)},{\sy*(-0.0023)})
	--({\sx*(2.0400)},{\sy*(-0.0018)})
	--({\sx*(2.0500)},{\sy*(-0.0002)})
	--({\sx*(2.0600)},{\sy*(0.0026)})
	--({\sx*(2.0700)},{\sy*(0.0065)})
	--({\sx*(2.0800)},{\sy*(0.0115)})
	--({\sx*(2.0900)},{\sy*(0.0175)})
	--({\sx*(2.1000)},{\sy*(0.0245)})
	--({\sx*(2.1100)},{\sy*(0.0324)})
	--({\sx*(2.1200)},{\sy*(0.0411)})
	--({\sx*(2.1300)},{\sy*(0.0506)})
	--({\sx*(2.1400)},{\sy*(0.0607)})
	--({\sx*(2.1500)},{\sy*(0.0713)})
	--({\sx*(2.1600)},{\sy*(0.0823)})
	--({\sx*(2.1700)},{\sy*(0.0937)})
	--({\sx*(2.1800)},{\sy*(0.1052)})
	--({\sx*(2.1900)},{\sy*(0.1169)})
	--({\sx*(2.2000)},{\sy*(0.1285)})
	--({\sx*(2.2100)},{\sy*(0.1400)})
	--({\sx*(2.2200)},{\sy*(0.1512)})
	--({\sx*(2.2300)},{\sy*(0.1621)})
	--({\sx*(2.2400)},{\sy*(0.1724)})
	--({\sx*(2.2500)},{\sy*(0.1821)})
	--({\sx*(2.2600)},{\sy*(0.1910)})
	--({\sx*(2.2700)},{\sy*(0.1991)})
	--({\sx*(2.2800)},{\sy*(0.2063)})
	--({\sx*(2.2900)},{\sy*(0.2124)})
	--({\sx*(2.3000)},{\sy*(0.2173)})
	--({\sx*(2.3100)},{\sy*(0.2210)})
	--({\sx*(2.3200)},{\sy*(0.2234)})
	--({\sx*(2.3300)},{\sy*(0.2243)})
	--({\sx*(2.3400)},{\sy*(0.2238)})
	--({\sx*(2.3500)},{\sy*(0.2218)})
	--({\sx*(2.3600)},{\sy*(0.2182)})
	--({\sx*(2.3700)},{\sy*(0.2130)})
	--({\sx*(2.3800)},{\sy*(0.2061)})
	--({\sx*(2.3900)},{\sy*(0.1976)})
	--({\sx*(2.4000)},{\sy*(0.1875)})
	--({\sx*(2.4100)},{\sy*(0.1756)})
	--({\sx*(2.4200)},{\sy*(0.1621)})
	--({\sx*(2.4300)},{\sy*(0.1470)})
	--({\sx*(2.4400)},{\sy*(0.1303)})
	--({\sx*(2.4500)},{\sy*(0.1121)})
	--({\sx*(2.4600)},{\sy*(0.0923)})
	--({\sx*(2.4700)},{\sy*(0.0712)})
	--({\sx*(2.4800)},{\sy*(0.0487)})
	--({\sx*(2.4900)},{\sy*(0.0249)})
	--({\sx*(2.5000)},{\sy*(0.0000)})
	--({\sx*(2.5100)},{\sy*(-0.0260)})
	--({\sx*(2.5200)},{\sy*(-0.0529)})
	--({\sx*(2.5300)},{\sy*(-0.0806)})
	--({\sx*(2.5400)},{\sy*(-0.1090)})
	--({\sx*(2.5500)},{\sy*(-0.1379)})
	--({\sx*(2.5600)},{\sy*(-0.1672)})
	--({\sx*(2.5700)},{\sy*(-0.1968)})
	--({\sx*(2.5800)},{\sy*(-0.2265)})
	--({\sx*(2.5900)},{\sy*(-0.2561)})
	--({\sx*(2.6000)},{\sy*(-0.2854)})
	--({\sx*(2.6100)},{\sy*(-0.3144)})
	--({\sx*(2.6200)},{\sy*(-0.3429)})
	--({\sx*(2.6300)},{\sy*(-0.3706)})
	--({\sx*(2.6400)},{\sy*(-0.3974)})
	--({\sx*(2.6500)},{\sy*(-0.4232)})
	--({\sx*(2.6600)},{\sy*(-0.4478)})
	--({\sx*(2.6700)},{\sy*(-0.4709)})
	--({\sx*(2.6800)},{\sy*(-0.4926)})
	--({\sx*(2.6900)},{\sy*(-0.5126)})
	--({\sx*(2.7000)},{\sy*(-0.5307)})
	--({\sx*(2.7100)},{\sy*(-0.5470)})
	--({\sx*(2.7200)},{\sy*(-0.5611)})
	--({\sx*(2.7300)},{\sy*(-0.5730)})
	--({\sx*(2.7400)},{\sy*(-0.5825)})
	--({\sx*(2.7500)},{\sy*(-0.5897)})
	--({\sx*(2.7600)},{\sy*(-0.5943)})
	--({\sx*(2.7700)},{\sy*(-0.5964)})
	--({\sx*(2.7800)},{\sy*(-0.5958)})
	--({\sx*(2.7900)},{\sy*(-0.5924)})
	--({\sx*(2.8000)},{\sy*(-0.5863)})
	--({\sx*(2.8100)},{\sy*(-0.5774)})
	--({\sx*(2.8200)},{\sy*(-0.5656)})
	--({\sx*(2.8300)},{\sy*(-0.5511)})
	--({\sx*(2.8400)},{\sy*(-0.5337)})
	--({\sx*(2.8500)},{\sy*(-0.5136)})
	--({\sx*(2.8600)},{\sy*(-0.4908)})
	--({\sx*(2.8700)},{\sy*(-0.4652)})
	--({\sx*(2.8800)},{\sy*(-0.4371)})
	--({\sx*(2.8900)},{\sy*(-0.4064)})
	--({\sx*(2.9000)},{\sy*(-0.3733)})
	--({\sx*(2.9100)},{\sy*(-0.3379)})
	--({\sx*(2.9200)},{\sy*(-0.3002)})
	--({\sx*(2.9300)},{\sy*(-0.2606)})
	--({\sx*(2.9400)},{\sy*(-0.2190)})
	--({\sx*(2.9500)},{\sy*(-0.1757)})
	--({\sx*(2.9600)},{\sy*(-0.1309)})
	--({\sx*(2.9700)},{\sy*(-0.0847)})
	--({\sx*(2.9800)},{\sy*(-0.0373)})
	--({\sx*(2.9900)},{\sy*(0.0111)})
	--({\sx*(3.0000)},{\sy*(0.0602)})
	--({\sx*(3.0100)},{\sy*(0.1098)})
	--({\sx*(3.0200)},{\sy*(0.1598)})
	--({\sx*(3.0300)},{\sy*(0.2098)})
	--({\sx*(3.0400)},{\sy*(0.2597)})
	--({\sx*(3.0500)},{\sy*(0.3091)})
	--({\sx*(3.0600)},{\sy*(0.3579)})
	--({\sx*(3.0700)},{\sy*(0.4058)})
	--({\sx*(3.0800)},{\sy*(0.4526)})
	--({\sx*(3.0900)},{\sy*(0.4980)})
	--({\sx*(3.1000)},{\sy*(0.5418)})
	--({\sx*(3.1100)},{\sy*(0.5837)})
	--({\sx*(3.1200)},{\sy*(0.6236)})
	--({\sx*(3.1300)},{\sy*(0.6612)})
	--({\sx*(3.1400)},{\sy*(0.6962)})
	--({\sx*(3.1500)},{\sy*(0.7286)})
	--({\sx*(3.1600)},{\sy*(0.7580)})
	--({\sx*(3.1700)},{\sy*(0.7844)})
	--({\sx*(3.1800)},{\sy*(0.8074)})
	--({\sx*(3.1900)},{\sy*(0.8271)})
	--({\sx*(3.2000)},{\sy*(0.8432)})
	--({\sx*(3.2100)},{\sy*(0.8556)})
	--({\sx*(3.2200)},{\sy*(0.8642)})
	--({\sx*(3.2300)},{\sy*(0.8689)})
	--({\sx*(3.2400)},{\sy*(0.8696)})
	--({\sx*(3.2500)},{\sy*(0.8663)})
	--({\sx*(3.2600)},{\sy*(0.8590)})
	--({\sx*(3.2700)},{\sy*(0.8475)})
	--({\sx*(3.2800)},{\sy*(0.8320)})
	--({\sx*(3.2900)},{\sy*(0.8124)})
	--({\sx*(3.3000)},{\sy*(0.7888)})
	--({\sx*(3.3100)},{\sy*(0.7612)})
	--({\sx*(3.3200)},{\sy*(0.7298)})
	--({\sx*(3.3300)},{\sy*(0.6946)})
	--({\sx*(3.3400)},{\sy*(0.6558)})
	--({\sx*(3.3500)},{\sy*(0.6135)})
	--({\sx*(3.3600)},{\sy*(0.5679)})
	--({\sx*(3.3700)},{\sy*(0.5191)})
	--({\sx*(3.3800)},{\sy*(0.4674)})
	--({\sx*(3.3900)},{\sy*(0.4130)})
	--({\sx*(3.4000)},{\sy*(0.3561)})
	--({\sx*(3.4100)},{\sy*(0.2971)})
	--({\sx*(3.4200)},{\sy*(0.2360)})
	--({\sx*(3.4300)},{\sy*(0.1733)})
	--({\sx*(3.4400)},{\sy*(0.1092)})
	--({\sx*(3.4500)},{\sy*(0.0441)})
	--({\sx*(3.4600)},{\sy*(-0.0217)})
	--({\sx*(3.4700)},{\sy*(-0.0880)})
	--({\sx*(3.4800)},{\sy*(-0.1544)})
	--({\sx*(3.4900)},{\sy*(-0.2205)})
	--({\sx*(3.5000)},{\sy*(-0.2860)})
	--({\sx*(3.5100)},{\sy*(-0.3505)})
	--({\sx*(3.5200)},{\sy*(-0.4138)})
	--({\sx*(3.5300)},{\sy*(-0.4754)})
	--({\sx*(3.5400)},{\sy*(-0.5351)})
	--({\sx*(3.5500)},{\sy*(-0.5925)})
	--({\sx*(3.5600)},{\sy*(-0.6472)})
	--({\sx*(3.5700)},{\sy*(-0.6990)})
	--({\sx*(3.5800)},{\sy*(-0.7476)})
	--({\sx*(3.5900)},{\sy*(-0.7927)})
	--({\sx*(3.6000)},{\sy*(-0.8339)})
	--({\sx*(3.6100)},{\sy*(-0.8711)})
	--({\sx*(3.6200)},{\sy*(-0.9039)})
	--({\sx*(3.6300)},{\sy*(-0.9323)})
	--({\sx*(3.6400)},{\sy*(-0.9559)})
	--({\sx*(3.6500)},{\sy*(-0.9746)})
	--({\sx*(3.6600)},{\sy*(-0.9883)})
	--({\sx*(3.6700)},{\sy*(-0.9968)})
	--({\sx*(3.6800)},{\sy*(-1.0000)})
	--({\sx*(3.6900)},{\sy*(-0.9979)})
	--({\sx*(3.7000)},{\sy*(-0.9904)})
	--({\sx*(3.7100)},{\sy*(-0.9775)})
	--({\sx*(3.7200)},{\sy*(-0.9592)})
	--({\sx*(3.7300)},{\sy*(-0.9356)})
	--({\sx*(3.7400)},{\sy*(-0.9067)})
	--({\sx*(3.7500)},{\sy*(-0.8727)})
	--({\sx*(3.7600)},{\sy*(-0.8337)})
	--({\sx*(3.7700)},{\sy*(-0.7899)})
	--({\sx*(3.7800)},{\sy*(-0.7415)})
	--({\sx*(3.7900)},{\sy*(-0.6888)})
	--({\sx*(3.8000)},{\sy*(-0.6319)})
	--({\sx*(3.8100)},{\sy*(-0.5712)})
	--({\sx*(3.8200)},{\sy*(-0.5070)})
	--({\sx*(3.8300)},{\sy*(-0.4397)})
	--({\sx*(3.8400)},{\sy*(-0.3695)})
	--({\sx*(3.8500)},{\sy*(-0.2970)})
	--({\sx*(3.8600)},{\sy*(-0.2226)})
	--({\sx*(3.8700)},{\sy*(-0.1465)})
	--({\sx*(3.8800)},{\sy*(-0.0694)})
	--({\sx*(3.8900)},{\sy*(0.0084)})
	--({\sx*(3.9000)},{\sy*(0.0863)})
	--({\sx*(3.9100)},{\sy*(0.1639)})
	--({\sx*(3.9200)},{\sy*(0.2407)})
	--({\sx*(3.9300)},{\sy*(0.3163)})
	--({\sx*(3.9400)},{\sy*(0.3900)})
	--({\sx*(3.9500)},{\sy*(0.4615)})
	--({\sx*(3.9600)},{\sy*(0.5303)})
	--({\sx*(3.9700)},{\sy*(0.5959)})
	--({\sx*(3.9800)},{\sy*(0.6578)})
	--({\sx*(3.9900)},{\sy*(0.7156)})
	--({\sx*(4.0000)},{\sy*(0.7689)})
	--({\sx*(4.0100)},{\sy*(0.8173)})
	--({\sx*(4.0200)},{\sy*(0.8605)})
	--({\sx*(4.0300)},{\sy*(0.8980)})
	--({\sx*(4.0400)},{\sy*(0.9297)})
	--({\sx*(4.0500)},{\sy*(0.9551)})
	--({\sx*(4.0600)},{\sy*(0.9742)})
	--({\sx*(4.0700)},{\sy*(0.9866)})
	--({\sx*(4.0800)},{\sy*(0.9923)})
	--({\sx*(4.0900)},{\sy*(0.9912)})
	--({\sx*(4.1000)},{\sy*(0.9832)})
	--({\sx*(4.1100)},{\sy*(0.9682)})
	--({\sx*(4.1200)},{\sy*(0.9464)})
	--({\sx*(4.1300)},{\sy*(0.9179)})
	--({\sx*(4.1400)},{\sy*(0.8827)})
	--({\sx*(4.1500)},{\sy*(0.8411)})
	--({\sx*(4.1600)},{\sy*(0.7933)})
	--({\sx*(4.1700)},{\sy*(0.7396)})
	--({\sx*(4.1800)},{\sy*(0.6805)})
	--({\sx*(4.1900)},{\sy*(0.6162)})
	--({\sx*(4.2000)},{\sy*(0.5473)})
	--({\sx*(4.2100)},{\sy*(0.4742)})
	--({\sx*(4.2200)},{\sy*(0.3974)})
	--({\sx*(4.2300)},{\sy*(0.3177)})
	--({\sx*(4.2400)},{\sy*(0.2355)})
	--({\sx*(4.2500)},{\sy*(0.1515)})
	--({\sx*(4.2600)},{\sy*(0.0664)})
	--({\sx*(4.2700)},{\sy*(-0.0191)})
	--({\sx*(4.2800)},{\sy*(-0.1043)})
	--({\sx*(4.2900)},{\sy*(-0.1884)})
	--({\sx*(4.3000)},{\sy*(-0.2708)})
	--({\sx*(4.3100)},{\sy*(-0.3507)})
	--({\sx*(4.3200)},{\sy*(-0.4273)})
	--({\sx*(4.3300)},{\sy*(-0.5000)})
	--({\sx*(4.3400)},{\sy*(-0.5680)})
	--({\sx*(4.3500)},{\sy*(-0.6307)})
	--({\sx*(4.3600)},{\sy*(-0.6874)})
	--({\sx*(4.3700)},{\sy*(-0.7375)})
	--({\sx*(4.3800)},{\sy*(-0.7806)})
	--({\sx*(4.3900)},{\sy*(-0.8160)})
	--({\sx*(4.4000)},{\sy*(-0.8434)})
	--({\sx*(4.4100)},{\sy*(-0.8624)})
	--({\sx*(4.4200)},{\sy*(-0.8728)})
	--({\sx*(4.4300)},{\sy*(-0.8743)})
	--({\sx*(4.4400)},{\sy*(-0.8668)})
	--({\sx*(4.4500)},{\sy*(-0.8502)})
	--({\sx*(4.4600)},{\sy*(-0.8248)})
	--({\sx*(4.4700)},{\sy*(-0.7906)})
	--({\sx*(4.4800)},{\sy*(-0.7479)})
	--({\sx*(4.4900)},{\sy*(-0.6971)})
	--({\sx*(4.5000)},{\sy*(-0.6387)})
	--({\sx*(4.5100)},{\sy*(-0.5732)})
	--({\sx*(4.5200)},{\sy*(-0.5014)})
	--({\sx*(4.5300)},{\sy*(-0.4241)})
	--({\sx*(4.5400)},{\sy*(-0.3421)})
	--({\sx*(4.5500)},{\sy*(-0.2564)})
	--({\sx*(4.5600)},{\sy*(-0.1681)})
	--({\sx*(4.5700)},{\sy*(-0.0782)})
	--({\sx*(4.5800)},{\sy*(0.0119)})
	--({\sx*(4.5900)},{\sy*(0.1012)})
	--({\sx*(4.6000)},{\sy*(0.1884)})
	--({\sx*(4.6100)},{\sy*(0.2721)})
	--({\sx*(4.6200)},{\sy*(0.3511)})
	--({\sx*(4.6300)},{\sy*(0.4241)})
	--({\sx*(4.6400)},{\sy*(0.4900)})
	--({\sx*(4.6500)},{\sy*(0.5475)})
	--({\sx*(4.6600)},{\sy*(0.5957)})
	--({\sx*(4.6700)},{\sy*(0.6335)})
	--({\sx*(4.6800)},{\sy*(0.6601)})
	--({\sx*(4.6900)},{\sy*(0.6749)})
	--({\sx*(4.7000)},{\sy*(0.6773)})
	--({\sx*(4.7100)},{\sy*(0.6671)})
	--({\sx*(4.7200)},{\sy*(0.6443)})
	--({\sx*(4.7300)},{\sy*(0.6091)})
	--({\sx*(4.7400)},{\sy*(0.5618)})
	--({\sx*(4.7500)},{\sy*(0.5034)})
	--({\sx*(4.7600)},{\sy*(0.4348)})
	--({\sx*(4.7700)},{\sy*(0.3574)})
	--({\sx*(4.7800)},{\sy*(0.2728)})
	--({\sx*(4.7900)},{\sy*(0.1831)})
	--({\sx*(4.8000)},{\sy*(0.0904)})
	--({\sx*(4.8100)},{\sy*(-0.0028)})
	--({\sx*(4.8200)},{\sy*(-0.0937)})
	--({\sx*(4.8300)},{\sy*(-0.1795)})
	--({\sx*(4.8400)},{\sy*(-0.2572)})
	--({\sx*(4.8500)},{\sy*(-0.3239)})
	--({\sx*(4.8600)},{\sy*(-0.3766)})
	--({\sx*(4.8700)},{\sy*(-0.4128)})
	--({\sx*(4.8800)},{\sy*(-0.4302)})
	--({\sx*(4.8900)},{\sy*(-0.4271)})
	--({\sx*(4.9000)},{\sy*(-0.4025)})
	--({\sx*(4.9100)},{\sy*(-0.3567)})
	--({\sx*(4.9200)},{\sy*(-0.2911)})
	--({\sx*(4.9300)},{\sy*(-0.2090)})
	--({\sx*(4.9400)},{\sy*(-0.1156)})
	--({\sx*(4.9500)},{\sy*(-0.0186)})
	--({\sx*(4.9600)},{\sy*(0.0711)})
	--({\sx*(4.9700)},{\sy*(0.1391)})
	--({\sx*(4.9800)},{\sy*(0.1666)})
	--({\sx*(4.9900)},{\sy*(0.1301)})
	--({\sx*(5.0000)},{\sy*(0.0000)});
}
\def\relfehlerh{
\draw[color=blue,line width=1.4pt,line join=round] ({\sx*(0.000)},{\sy*(0.0000)})
	--({\sx*(0.0100)},{\sy*(0.0000)})
	--({\sx*(0.0200)},{\sy*(0.0000)})
	--({\sx*(0.0300)},{\sy*(0.0000)})
	--({\sx*(0.0400)},{\sy*(0.0000)})
	--({\sx*(0.0500)},{\sy*(-0.0000)})
	--({\sx*(0.0600)},{\sy*(-0.0000)})
	--({\sx*(0.0700)},{\sy*(-0.0000)})
	--({\sx*(0.0800)},{\sy*(-0.0000)})
	--({\sx*(0.0900)},{\sy*(-0.0000)})
	--({\sx*(0.1000)},{\sy*(-0.0000)})
	--({\sx*(0.1100)},{\sy*(-0.0000)})
	--({\sx*(0.1200)},{\sy*(-0.0000)})
	--({\sx*(0.1300)},{\sy*(-0.0000)})
	--({\sx*(0.1400)},{\sy*(-0.0000)})
	--({\sx*(0.1500)},{\sy*(-0.0000)})
	--({\sx*(0.1600)},{\sy*(-0.0000)})
	--({\sx*(0.1700)},{\sy*(-0.0000)})
	--({\sx*(0.1800)},{\sy*(-0.0000)})
	--({\sx*(0.1900)},{\sy*(-0.0000)})
	--({\sx*(0.2000)},{\sy*(0.0000)})
	--({\sx*(0.2100)},{\sy*(0.0000)})
	--({\sx*(0.2200)},{\sy*(0.0000)})
	--({\sx*(0.2300)},{\sy*(0.0000)})
	--({\sx*(0.2400)},{\sy*(0.0000)})
	--({\sx*(0.2500)},{\sy*(0.0000)})
	--({\sx*(0.2600)},{\sy*(0.0000)})
	--({\sx*(0.2700)},{\sy*(0.0000)})
	--({\sx*(0.2800)},{\sy*(0.0000)})
	--({\sx*(0.2900)},{\sy*(0.0000)})
	--({\sx*(0.3000)},{\sy*(0.0000)})
	--({\sx*(0.3100)},{\sy*(0.0000)})
	--({\sx*(0.3200)},{\sy*(0.0000)})
	--({\sx*(0.3300)},{\sy*(0.0000)})
	--({\sx*(0.3400)},{\sy*(0.0000)})
	--({\sx*(0.3500)},{\sy*(0.0000)})
	--({\sx*(0.3600)},{\sy*(0.0000)})
	--({\sx*(0.3700)},{\sy*(0.0000)})
	--({\sx*(0.3800)},{\sy*(0.0000)})
	--({\sx*(0.3900)},{\sy*(0.0000)})
	--({\sx*(0.4000)},{\sy*(0.0000)})
	--({\sx*(0.4100)},{\sy*(0.0000)})
	--({\sx*(0.4200)},{\sy*(0.0000)})
	--({\sx*(0.4300)},{\sy*(-0.0000)})
	--({\sx*(0.4400)},{\sy*(-0.0000)})
	--({\sx*(0.4500)},{\sy*(-0.0000)})
	--({\sx*(0.4600)},{\sy*(-0.0000)})
	--({\sx*(0.4700)},{\sy*(-0.0000)})
	--({\sx*(0.4800)},{\sy*(-0.0000)})
	--({\sx*(0.4900)},{\sy*(-0.0000)})
	--({\sx*(0.5000)},{\sy*(-0.0000)})
	--({\sx*(0.5100)},{\sy*(-0.0000)})
	--({\sx*(0.5200)},{\sy*(-0.0000)})
	--({\sx*(0.5300)},{\sy*(-0.0000)})
	--({\sx*(0.5400)},{\sy*(-0.0000)})
	--({\sx*(0.5500)},{\sy*(-0.0000)})
	--({\sx*(0.5600)},{\sy*(-0.0000)})
	--({\sx*(0.5700)},{\sy*(-0.0000)})
	--({\sx*(0.5800)},{\sy*(-0.0000)})
	--({\sx*(0.5900)},{\sy*(-0.0000)})
	--({\sx*(0.6000)},{\sy*(-0.0000)})
	--({\sx*(0.6100)},{\sy*(-0.0000)})
	--({\sx*(0.6200)},{\sy*(-0.0000)})
	--({\sx*(0.6300)},{\sy*(-0.0000)})
	--({\sx*(0.6400)},{\sy*(-0.0000)})
	--({\sx*(0.6500)},{\sy*(-0.0000)})
	--({\sx*(0.6600)},{\sy*(-0.0000)})
	--({\sx*(0.6700)},{\sy*(-0.0000)})
	--({\sx*(0.6800)},{\sy*(-0.0000)})
	--({\sx*(0.6900)},{\sy*(-0.0000)})
	--({\sx*(0.7000)},{\sy*(-0.0000)})
	--({\sx*(0.7100)},{\sy*(-0.0000)})
	--({\sx*(0.7200)},{\sy*(-0.0000)})
	--({\sx*(0.7300)},{\sy*(-0.0000)})
	--({\sx*(0.7400)},{\sy*(0.0000)})
	--({\sx*(0.7500)},{\sy*(0.0000)})
	--({\sx*(0.7600)},{\sy*(0.0000)})
	--({\sx*(0.7700)},{\sy*(0.0000)})
	--({\sx*(0.7800)},{\sy*(0.0000)})
	--({\sx*(0.7900)},{\sy*(0.0000)})
	--({\sx*(0.8000)},{\sy*(0.0000)})
	--({\sx*(0.8100)},{\sy*(0.0000)})
	--({\sx*(0.8200)},{\sy*(0.0000)})
	--({\sx*(0.8300)},{\sy*(0.0000)})
	--({\sx*(0.8400)},{\sy*(0.0000)})
	--({\sx*(0.8500)},{\sy*(0.0000)})
	--({\sx*(0.8600)},{\sy*(0.0000)})
	--({\sx*(0.8700)},{\sy*(0.0000)})
	--({\sx*(0.8800)},{\sy*(0.0000)})
	--({\sx*(0.8900)},{\sy*(0.0000)})
	--({\sx*(0.9000)},{\sy*(0.0000)})
	--({\sx*(0.9100)},{\sy*(0.0000)})
	--({\sx*(0.9200)},{\sy*(0.0000)})
	--({\sx*(0.9300)},{\sy*(0.0000)})
	--({\sx*(0.9400)},{\sy*(0.0000)})
	--({\sx*(0.9500)},{\sy*(0.0000)})
	--({\sx*(0.9600)},{\sy*(0.0000)})
	--({\sx*(0.9700)},{\sy*(0.0000)})
	--({\sx*(0.9800)},{\sy*(0.0000)})
	--({\sx*(0.9900)},{\sy*(0.0000)})
	--({\sx*(1.0000)},{\sy*(0.0000)})
	--({\sx*(1.0100)},{\sy*(0.0000)})
	--({\sx*(1.0200)},{\sy*(0.0000)})
	--({\sx*(1.0300)},{\sy*(0.0000)})
	--({\sx*(1.0400)},{\sy*(0.0000)})
	--({\sx*(1.0500)},{\sy*(0.0000)})
	--({\sx*(1.0600)},{\sy*(0.0000)})
	--({\sx*(1.0700)},{\sy*(0.0000)})
	--({\sx*(1.0800)},{\sy*(0.0000)})
	--({\sx*(1.0900)},{\sy*(0.0000)})
	--({\sx*(1.1000)},{\sy*(0.0000)})
	--({\sx*(1.1100)},{\sy*(0.0000)})
	--({\sx*(1.1200)},{\sy*(-0.0000)})
	--({\sx*(1.1300)},{\sy*(-0.0000)})
	--({\sx*(1.1400)},{\sy*(-0.0000)})
	--({\sx*(1.1500)},{\sy*(-0.0000)})
	--({\sx*(1.1600)},{\sy*(-0.0000)})
	--({\sx*(1.1700)},{\sy*(-0.0000)})
	--({\sx*(1.1800)},{\sy*(-0.0000)})
	--({\sx*(1.1900)},{\sy*(-0.0000)})
	--({\sx*(1.2000)},{\sy*(-0.0000)})
	--({\sx*(1.2100)},{\sy*(-0.0000)})
	--({\sx*(1.2200)},{\sy*(-0.0000)})
	--({\sx*(1.2300)},{\sy*(-0.0000)})
	--({\sx*(1.2400)},{\sy*(-0.0000)})
	--({\sx*(1.2500)},{\sy*(-0.0000)})
	--({\sx*(1.2600)},{\sy*(-0.0000)})
	--({\sx*(1.2700)},{\sy*(-0.0000)})
	--({\sx*(1.2800)},{\sy*(-0.0000)})
	--({\sx*(1.2900)},{\sy*(-0.0000)})
	--({\sx*(1.3000)},{\sy*(-0.0000)})
	--({\sx*(1.3100)},{\sy*(-0.0000)})
	--({\sx*(1.3200)},{\sy*(-0.0000)})
	--({\sx*(1.3300)},{\sy*(-0.0000)})
	--({\sx*(1.3400)},{\sy*(-0.0000)})
	--({\sx*(1.3500)},{\sy*(-0.0000)})
	--({\sx*(1.3600)},{\sy*(-0.0000)})
	--({\sx*(1.3700)},{\sy*(-0.0000)})
	--({\sx*(1.3800)},{\sy*(-0.0000)})
	--({\sx*(1.3900)},{\sy*(-0.0000)})
	--({\sx*(1.4000)},{\sy*(-0.0000)})
	--({\sx*(1.4100)},{\sy*(-0.0000)})
	--({\sx*(1.4200)},{\sy*(-0.0000)})
	--({\sx*(1.4300)},{\sy*(-0.0000)})
	--({\sx*(1.4400)},{\sy*(-0.0000)})
	--({\sx*(1.4500)},{\sy*(-0.0000)})
	--({\sx*(1.4600)},{\sy*(-0.0000)})
	--({\sx*(1.4700)},{\sy*(-0.0000)})
	--({\sx*(1.4800)},{\sy*(-0.0000)})
	--({\sx*(1.4900)},{\sy*(-0.0000)})
	--({\sx*(1.5000)},{\sy*(-0.0000)})
	--({\sx*(1.5100)},{\sy*(-0.0000)})
	--({\sx*(1.5200)},{\sy*(-0.0000)})
	--({\sx*(1.5300)},{\sy*(-0.0000)})
	--({\sx*(1.5400)},{\sy*(-0.0000)})
	--({\sx*(1.5500)},{\sy*(0.0000)})
	--({\sx*(1.5600)},{\sy*(0.0000)})
	--({\sx*(1.5700)},{\sy*(0.0000)})
	--({\sx*(1.5800)},{\sy*(0.0000)})
	--({\sx*(1.5900)},{\sy*(0.0000)})
	--({\sx*(1.6000)},{\sy*(0.0000)})
	--({\sx*(1.6100)},{\sy*(0.0000)})
	--({\sx*(1.6200)},{\sy*(0.0000)})
	--({\sx*(1.6300)},{\sy*(0.0000)})
	--({\sx*(1.6400)},{\sy*(0.0000)})
	--({\sx*(1.6500)},{\sy*(0.0000)})
	--({\sx*(1.6600)},{\sy*(0.0000)})
	--({\sx*(1.6700)},{\sy*(0.0000)})
	--({\sx*(1.6800)},{\sy*(0.0000)})
	--({\sx*(1.6900)},{\sy*(0.0000)})
	--({\sx*(1.7000)},{\sy*(0.0000)})
	--({\sx*(1.7100)},{\sy*(0.0000)})
	--({\sx*(1.7200)},{\sy*(0.0000)})
	--({\sx*(1.7300)},{\sy*(0.0000)})
	--({\sx*(1.7400)},{\sy*(0.0000)})
	--({\sx*(1.7500)},{\sy*(0.0000)})
	--({\sx*(1.7600)},{\sy*(0.0000)})
	--({\sx*(1.7700)},{\sy*(0.0000)})
	--({\sx*(1.7800)},{\sy*(0.0000)})
	--({\sx*(1.7900)},{\sy*(0.0000)})
	--({\sx*(1.8000)},{\sy*(0.0000)})
	--({\sx*(1.8100)},{\sy*(0.0000)})
	--({\sx*(1.8200)},{\sy*(0.0000)})
	--({\sx*(1.8300)},{\sy*(0.0000)})
	--({\sx*(1.8400)},{\sy*(0.0000)})
	--({\sx*(1.8500)},{\sy*(0.0000)})
	--({\sx*(1.8600)},{\sy*(0.0000)})
	--({\sx*(1.8700)},{\sy*(0.0000)})
	--({\sx*(1.8800)},{\sy*(0.0000)})
	--({\sx*(1.8900)},{\sy*(0.0000)})
	--({\sx*(1.9000)},{\sy*(0.0000)})
	--({\sx*(1.9100)},{\sy*(0.0000)})
	--({\sx*(1.9200)},{\sy*(0.0000)})
	--({\sx*(1.9300)},{\sy*(0.0000)})
	--({\sx*(1.9400)},{\sy*(0.0000)})
	--({\sx*(1.9500)},{\sy*(0.0000)})
	--({\sx*(1.9600)},{\sy*(0.0000)})
	--({\sx*(1.9700)},{\sy*(0.0000)})
	--({\sx*(1.9800)},{\sy*(0.0000)})
	--({\sx*(1.9900)},{\sy*(0.0000)})
	--({\sx*(2.0000)},{\sy*(0.0000)})
	--({\sx*(2.0100)},{\sy*(0.0000)})
	--({\sx*(2.0200)},{\sy*(-0.0000)})
	--({\sx*(2.0300)},{\sy*(-0.0000)})
	--({\sx*(2.0400)},{\sy*(-0.0000)})
	--({\sx*(2.0500)},{\sy*(-0.0000)})
	--({\sx*(2.0600)},{\sy*(0.0000)})
	--({\sx*(2.0700)},{\sy*(0.0000)})
	--({\sx*(2.0800)},{\sy*(0.0000)})
	--({\sx*(2.0900)},{\sy*(0.0000)})
	--({\sx*(2.1000)},{\sy*(0.0000)})
	--({\sx*(2.1100)},{\sy*(0.0000)})
	--({\sx*(2.1200)},{\sy*(0.0000)})
	--({\sx*(2.1300)},{\sy*(0.0000)})
	--({\sx*(2.1400)},{\sy*(0.0000)})
	--({\sx*(2.1500)},{\sy*(0.0000)})
	--({\sx*(2.1600)},{\sy*(0.0000)})
	--({\sx*(2.1700)},{\sy*(0.0000)})
	--({\sx*(2.1800)},{\sy*(0.0000)})
	--({\sx*(2.1900)},{\sy*(0.0000)})
	--({\sx*(2.2000)},{\sy*(0.0000)})
	--({\sx*(2.2100)},{\sy*(0.0000)})
	--({\sx*(2.2200)},{\sy*(0.0000)})
	--({\sx*(2.2300)},{\sy*(0.0000)})
	--({\sx*(2.2400)},{\sy*(0.0000)})
	--({\sx*(2.2500)},{\sy*(0.0000)})
	--({\sx*(2.2600)},{\sy*(0.0000)})
	--({\sx*(2.2700)},{\sy*(0.0000)})
	--({\sx*(2.2800)},{\sy*(0.0000)})
	--({\sx*(2.2900)},{\sy*(0.0000)})
	--({\sx*(2.3000)},{\sy*(0.0000)})
	--({\sx*(2.3100)},{\sy*(0.0000)})
	--({\sx*(2.3200)},{\sy*(0.0000)})
	--({\sx*(2.3300)},{\sy*(0.0000)})
	--({\sx*(2.3400)},{\sy*(0.0000)})
	--({\sx*(2.3500)},{\sy*(0.0000)})
	--({\sx*(2.3600)},{\sy*(0.0000)})
	--({\sx*(2.3700)},{\sy*(0.0000)})
	--({\sx*(2.3800)},{\sy*(0.0000)})
	--({\sx*(2.3900)},{\sy*(0.0000)})
	--({\sx*(2.4000)},{\sy*(0.0000)})
	--({\sx*(2.4100)},{\sy*(0.0000)})
	--({\sx*(2.4200)},{\sy*(0.0000)})
	--({\sx*(2.4300)},{\sy*(0.0000)})
	--({\sx*(2.4400)},{\sy*(0.0000)})
	--({\sx*(2.4500)},{\sy*(0.0000)})
	--({\sx*(2.4600)},{\sy*(0.0000)})
	--({\sx*(2.4700)},{\sy*(0.0000)})
	--({\sx*(2.4800)},{\sy*(0.0000)})
	--({\sx*(2.4900)},{\sy*(0.0000)})
	--({\sx*(2.5000)},{\sy*(0.0000)})
	--({\sx*(2.5100)},{\sy*(-0.0000)})
	--({\sx*(2.5200)},{\sy*(-0.0000)})
	--({\sx*(2.5300)},{\sy*(-0.0000)})
	--({\sx*(2.5400)},{\sy*(-0.0000)})
	--({\sx*(2.5500)},{\sy*(-0.0000)})
	--({\sx*(2.5600)},{\sy*(-0.0000)})
	--({\sx*(2.5700)},{\sy*(-0.0000)})
	--({\sx*(2.5800)},{\sy*(-0.0000)})
	--({\sx*(2.5900)},{\sy*(-0.0000)})
	--({\sx*(2.6000)},{\sy*(-0.0000)})
	--({\sx*(2.6100)},{\sy*(-0.0000)})
	--({\sx*(2.6200)},{\sy*(-0.0000)})
	--({\sx*(2.6300)},{\sy*(-0.0000)})
	--({\sx*(2.6400)},{\sy*(-0.0000)})
	--({\sx*(2.6500)},{\sy*(-0.0000)})
	--({\sx*(2.6600)},{\sy*(-0.0000)})
	--({\sx*(2.6700)},{\sy*(-0.0000)})
	--({\sx*(2.6800)},{\sy*(-0.0000)})
	--({\sx*(2.6900)},{\sy*(-0.0000)})
	--({\sx*(2.7000)},{\sy*(-0.0000)})
	--({\sx*(2.7100)},{\sy*(-0.0000)})
	--({\sx*(2.7200)},{\sy*(-0.0000)})
	--({\sx*(2.7300)},{\sy*(-0.0000)})
	--({\sx*(2.7400)},{\sy*(-0.0000)})
	--({\sx*(2.7500)},{\sy*(-0.0000)})
	--({\sx*(2.7600)},{\sy*(-0.0000)})
	--({\sx*(2.7700)},{\sy*(-0.0000)})
	--({\sx*(2.7800)},{\sy*(-0.0000)})
	--({\sx*(2.7900)},{\sy*(-0.0000)})
	--({\sx*(2.8000)},{\sy*(-0.0000)})
	--({\sx*(2.8100)},{\sy*(-0.0000)})
	--({\sx*(2.8200)},{\sy*(-0.0000)})
	--({\sx*(2.8300)},{\sy*(-0.0000)})
	--({\sx*(2.8400)},{\sy*(-0.0000)})
	--({\sx*(2.8500)},{\sy*(-0.0000)})
	--({\sx*(2.8600)},{\sy*(-0.0000)})
	--({\sx*(2.8700)},{\sy*(-0.0000)})
	--({\sx*(2.8800)},{\sy*(-0.0000)})
	--({\sx*(2.8900)},{\sy*(-0.0000)})
	--({\sx*(2.9000)},{\sy*(-0.0000)})
	--({\sx*(2.9100)},{\sy*(-0.0000)})
	--({\sx*(2.9200)},{\sy*(-0.0000)})
	--({\sx*(2.9300)},{\sy*(-0.0000)})
	--({\sx*(2.9400)},{\sy*(-0.0000)})
	--({\sx*(2.9500)},{\sy*(-0.0000)})
	--({\sx*(2.9600)},{\sy*(-0.0000)})
	--({\sx*(2.9700)},{\sy*(-0.0000)})
	--({\sx*(2.9800)},{\sy*(-0.0000)})
	--({\sx*(2.9900)},{\sy*(0.0000)})
	--({\sx*(3.0000)},{\sy*(0.0000)})
	--({\sx*(3.0100)},{\sy*(0.0000)})
	--({\sx*(3.0200)},{\sy*(0.0000)})
	--({\sx*(3.0300)},{\sy*(0.0000)})
	--({\sx*(3.0400)},{\sy*(0.0000)})
	--({\sx*(3.0500)},{\sy*(0.0000)})
	--({\sx*(3.0600)},{\sy*(0.0000)})
	--({\sx*(3.0700)},{\sy*(0.0000)})
	--({\sx*(3.0800)},{\sy*(0.0000)})
	--({\sx*(3.0900)},{\sy*(0.0000)})
	--({\sx*(3.1000)},{\sy*(0.0000)})
	--({\sx*(3.1100)},{\sy*(0.0000)})
	--({\sx*(3.1200)},{\sy*(0.0000)})
	--({\sx*(3.1300)},{\sy*(0.0000)})
	--({\sx*(3.1400)},{\sy*(0.0000)})
	--({\sx*(3.1500)},{\sy*(0.0000)})
	--({\sx*(3.1600)},{\sy*(0.0000)})
	--({\sx*(3.1700)},{\sy*(0.0000)})
	--({\sx*(3.1800)},{\sy*(0.0000)})
	--({\sx*(3.1900)},{\sy*(0.0000)})
	--({\sx*(3.2000)},{\sy*(0.0000)})
	--({\sx*(3.2100)},{\sy*(0.0000)})
	--({\sx*(3.2200)},{\sy*(0.0000)})
	--({\sx*(3.2300)},{\sy*(0.0000)})
	--({\sx*(3.2400)},{\sy*(0.0000)})
	--({\sx*(3.2500)},{\sy*(0.0000)})
	--({\sx*(3.2600)},{\sy*(0.0000)})
	--({\sx*(3.2700)},{\sy*(0.0000)})
	--({\sx*(3.2800)},{\sy*(0.0000)})
	--({\sx*(3.2900)},{\sy*(0.0000)})
	--({\sx*(3.3000)},{\sy*(0.0000)})
	--({\sx*(3.3100)},{\sy*(0.0000)})
	--({\sx*(3.3200)},{\sy*(0.0000)})
	--({\sx*(3.3300)},{\sy*(0.0000)})
	--({\sx*(3.3400)},{\sy*(0.0000)})
	--({\sx*(3.3500)},{\sy*(0.0000)})
	--({\sx*(3.3600)},{\sy*(0.0000)})
	--({\sx*(3.3700)},{\sy*(0.0000)})
	--({\sx*(3.3800)},{\sy*(0.0000)})
	--({\sx*(3.3900)},{\sy*(0.0000)})
	--({\sx*(3.4000)},{\sy*(0.0000)})
	--({\sx*(3.4100)},{\sy*(0.0000)})
	--({\sx*(3.4200)},{\sy*(0.0000)})
	--({\sx*(3.4300)},{\sy*(0.0000)})
	--({\sx*(3.4400)},{\sy*(0.0000)})
	--({\sx*(3.4500)},{\sy*(0.0000)})
	--({\sx*(3.4600)},{\sy*(-0.0000)})
	--({\sx*(3.4700)},{\sy*(-0.0000)})
	--({\sx*(3.4800)},{\sy*(-0.0000)})
	--({\sx*(3.4900)},{\sy*(-0.0000)})
	--({\sx*(3.5000)},{\sy*(-0.0000)})
	--({\sx*(3.5100)},{\sy*(-0.0000)})
	--({\sx*(3.5200)},{\sy*(-0.0000)})
	--({\sx*(3.5300)},{\sy*(-0.0000)})
	--({\sx*(3.5400)},{\sy*(-0.0000)})
	--({\sx*(3.5500)},{\sy*(-0.0000)})
	--({\sx*(3.5600)},{\sy*(-0.0000)})
	--({\sx*(3.5700)},{\sy*(-0.0000)})
	--({\sx*(3.5800)},{\sy*(-0.0000)})
	--({\sx*(3.5900)},{\sy*(-0.0000)})
	--({\sx*(3.6000)},{\sy*(-0.0000)})
	--({\sx*(3.6100)},{\sy*(-0.0000)})
	--({\sx*(3.6200)},{\sy*(-0.0000)})
	--({\sx*(3.6300)},{\sy*(-0.0000)})
	--({\sx*(3.6400)},{\sy*(-0.0001)})
	--({\sx*(3.6500)},{\sy*(-0.0001)})
	--({\sx*(3.6600)},{\sy*(-0.0001)})
	--({\sx*(3.6700)},{\sy*(-0.0001)})
	--({\sx*(3.6800)},{\sy*(-0.0001)})
	--({\sx*(3.6900)},{\sy*(-0.0001)})
	--({\sx*(3.7000)},{\sy*(-0.0001)})
	--({\sx*(3.7100)},{\sy*(-0.0001)})
	--({\sx*(3.7200)},{\sy*(-0.0001)})
	--({\sx*(3.7300)},{\sy*(-0.0001)})
	--({\sx*(3.7400)},{\sy*(-0.0001)})
	--({\sx*(3.7500)},{\sy*(-0.0001)})
	--({\sx*(3.7600)},{\sy*(-0.0001)})
	--({\sx*(3.7700)},{\sy*(-0.0001)})
	--({\sx*(3.7800)},{\sy*(-0.0001)})
	--({\sx*(3.7900)},{\sy*(-0.0001)})
	--({\sx*(3.8000)},{\sy*(-0.0001)})
	--({\sx*(3.8100)},{\sy*(-0.0001)})
	--({\sx*(3.8200)},{\sy*(-0.0001)})
	--({\sx*(3.8300)},{\sy*(-0.0000)})
	--({\sx*(3.8400)},{\sy*(-0.0000)})
	--({\sx*(3.8500)},{\sy*(-0.0000)})
	--({\sx*(3.8600)},{\sy*(-0.0000)})
	--({\sx*(3.8700)},{\sy*(-0.0000)})
	--({\sx*(3.8800)},{\sy*(-0.0000)})
	--({\sx*(3.8900)},{\sy*(0.0000)})
	--({\sx*(3.9000)},{\sy*(0.0000)})
	--({\sx*(3.9100)},{\sy*(0.0000)})
	--({\sx*(3.9200)},{\sy*(0.0000)})
	--({\sx*(3.9300)},{\sy*(0.0001)})
	--({\sx*(3.9400)},{\sy*(0.0001)})
	--({\sx*(3.9500)},{\sy*(0.0001)})
	--({\sx*(3.9600)},{\sy*(0.0001)})
	--({\sx*(3.9700)},{\sy*(0.0001)})
	--({\sx*(3.9800)},{\sy*(0.0001)})
	--({\sx*(3.9900)},{\sy*(0.0001)})
	--({\sx*(4.0000)},{\sy*(0.0002)})
	--({\sx*(4.0100)},{\sy*(0.0002)})
	--({\sx*(4.0200)},{\sy*(0.0002)})
	--({\sx*(4.0300)},{\sy*(0.0002)})
	--({\sx*(4.0400)},{\sy*(0.0002)})
	--({\sx*(4.0500)},{\sy*(0.0003)})
	--({\sx*(4.0600)},{\sy*(0.0003)})
	--({\sx*(4.0700)},{\sy*(0.0003)})
	--({\sx*(4.0800)},{\sy*(0.0003)})
	--({\sx*(4.0900)},{\sy*(0.0003)})
	--({\sx*(4.1000)},{\sy*(0.0003)})
	--({\sx*(4.1100)},{\sy*(0.0003)})
	--({\sx*(4.1200)},{\sy*(0.0003)})
	--({\sx*(4.1300)},{\sy*(0.0003)})
	--({\sx*(4.1400)},{\sy*(0.0003)})
	--({\sx*(4.1500)},{\sy*(0.0003)})
	--({\sx*(4.1600)},{\sy*(0.0003)})
	--({\sx*(4.1700)},{\sy*(0.0003)})
	--({\sx*(4.1800)},{\sy*(0.0003)})
	--({\sx*(4.1900)},{\sy*(0.0003)})
	--({\sx*(4.2000)},{\sy*(0.0003)})
	--({\sx*(4.2100)},{\sy*(0.0002)})
	--({\sx*(4.2200)},{\sy*(0.0002)})
	--({\sx*(4.2300)},{\sy*(0.0002)})
	--({\sx*(4.2400)},{\sy*(0.0001)})
	--({\sx*(4.2500)},{\sy*(0.0001)})
	--({\sx*(4.2600)},{\sy*(0.0000)})
	--({\sx*(4.2700)},{\sy*(-0.0000)})
	--({\sx*(4.2800)},{\sy*(-0.0001)})
	--({\sx*(4.2900)},{\sy*(-0.0001)})
	--({\sx*(4.3000)},{\sy*(-0.0002)})
	--({\sx*(4.3100)},{\sy*(-0.0003)})
	--({\sx*(4.3200)},{\sy*(-0.0003)})
	--({\sx*(4.3300)},{\sy*(-0.0004)})
	--({\sx*(4.3400)},{\sy*(-0.0005)})
	--({\sx*(4.3500)},{\sy*(-0.0006)})
	--({\sx*(4.3600)},{\sy*(-0.0007)})
	--({\sx*(4.3700)},{\sy*(-0.0007)})
	--({\sx*(4.3800)},{\sy*(-0.0008)})
	--({\sx*(4.3900)},{\sy*(-0.0009)})
	--({\sx*(4.4000)},{\sy*(-0.0010)})
	--({\sx*(4.4100)},{\sy*(-0.0010)})
	--({\sx*(4.4200)},{\sy*(-0.0011)})
	--({\sx*(4.4300)},{\sy*(-0.0012)})
	--({\sx*(4.4400)},{\sy*(-0.0012)})
	--({\sx*(4.4500)},{\sy*(-0.0012)})
	--({\sx*(4.4600)},{\sy*(-0.0012)})
	--({\sx*(4.4700)},{\sy*(-0.0012)})
	--({\sx*(4.4800)},{\sy*(-0.0012)})
	--({\sx*(4.4900)},{\sy*(-0.0012)})
	--({\sx*(4.5000)},{\sy*(-0.0012)})
	--({\sx*(4.5100)},{\sy*(-0.0011)})
	--({\sx*(4.5200)},{\sy*(-0.0010)})
	--({\sx*(4.5300)},{\sy*(-0.0009)})
	--({\sx*(4.5400)},{\sy*(-0.0007)})
	--({\sx*(4.5500)},{\sy*(-0.0006)})
	--({\sx*(4.5600)},{\sy*(-0.0004)})
	--({\sx*(4.5700)},{\sy*(-0.0002)})
	--({\sx*(4.5800)},{\sy*(0.0000)})
	--({\sx*(4.5900)},{\sy*(0.0003)})
	--({\sx*(4.6000)},{\sy*(0.0005)})
	--({\sx*(4.6100)},{\sy*(0.0008)})
	--({\sx*(4.6200)},{\sy*(0.0011)})
	--({\sx*(4.6300)},{\sy*(0.0014)})
	--({\sx*(4.6400)},{\sy*(0.0017)})
	--({\sx*(4.6500)},{\sy*(0.0020)})
	--({\sx*(4.6600)},{\sy*(0.0022)})
	--({\sx*(4.6700)},{\sy*(0.0025)})
	--({\sx*(4.6800)},{\sy*(0.0027)})
	--({\sx*(4.6900)},{\sy*(0.0029)})
	--({\sx*(4.7000)},{\sy*(0.0031)})
	--({\sx*(4.7100)},{\sy*(0.0032)})
	--({\sx*(4.7200)},{\sy*(0.0032)})
	--({\sx*(4.7300)},{\sy*(0.0032)})
	--({\sx*(4.7400)},{\sy*(0.0031)})
	--({\sx*(4.7500)},{\sy*(0.0029)})
	--({\sx*(4.7600)},{\sy*(0.0026)})
	--({\sx*(4.7700)},{\sy*(0.0022)})
	--({\sx*(4.7800)},{\sy*(0.0018)})
	--({\sx*(4.7900)},{\sy*(0.0013)})
	--({\sx*(4.8000)},{\sy*(0.0007)})
	--({\sx*(4.8100)},{\sy*(-0.0000)})
	--({\sx*(4.8200)},{\sy*(-0.0008)})
	--({\sx*(4.8300)},{\sy*(-0.0015)})
	--({\sx*(4.8400)},{\sy*(-0.0023)})
	--({\sx*(4.8500)},{\sy*(-0.0030)})
	--({\sx*(4.8600)},{\sy*(-0.0037)})
	--({\sx*(4.8700)},{\sy*(-0.0042)})
	--({\sx*(4.8800)},{\sy*(-0.0046)})
	--({\sx*(4.8900)},{\sy*(-0.0048)})
	--({\sx*(4.9000)},{\sy*(-0.0048)})
	--({\sx*(4.9100)},{\sy*(-0.0045)})
	--({\sx*(4.9200)},{\sy*(-0.0038)})
	--({\sx*(4.9300)},{\sy*(-0.0029)})
	--({\sx*(4.9400)},{\sy*(-0.0017)})
	--({\sx*(4.9500)},{\sy*(-0.0003)})
	--({\sx*(4.9600)},{\sy*(0.0011)})
	--({\sx*(4.9700)},{\sy*(0.0023)})
	--({\sx*(4.9800)},{\sy*(0.0029)})
	--({\sx*(4.9900)},{\sy*(0.0024)})
	--({\sx*(5.0000)},{\sy*(0.0000)});
}
\def\xwertei{
\fill[color=red] (0.0000,0) circle[radius={0.07/\skala}];
\fill[color=white] (0.0000,0) circle[radius={0.05/\skala}];
\fill[color=red] (0.0380,0) circle[radius={0.07/\skala}];
\fill[color=white] (0.0380,0) circle[radius={0.05/\skala}];
\fill[color=red] (0.1508,0) circle[radius={0.07/\skala}];
\fill[color=white] (0.1508,0) circle[radius={0.05/\skala}];
\fill[color=red] (0.3349,0) circle[radius={0.07/\skala}];
\fill[color=white] (0.3349,0) circle[radius={0.05/\skala}];
\fill[color=red] (0.5849,0) circle[radius={0.07/\skala}];
\fill[color=white] (0.5849,0) circle[radius={0.05/\skala}];
\fill[color=red] (0.8930,0) circle[radius={0.07/\skala}];
\fill[color=white] (0.8930,0) circle[radius={0.05/\skala}];
\fill[color=red] (1.2500,0) circle[radius={0.07/\skala}];
\fill[color=white] (1.2500,0) circle[radius={0.05/\skala}];
\fill[color=red] (1.6449,0) circle[radius={0.07/\skala}];
\fill[color=white] (1.6449,0) circle[radius={0.05/\skala}];
\fill[color=red] (2.0659,0) circle[radius={0.07/\skala}];
\fill[color=white] (2.0659,0) circle[radius={0.05/\skala}];
\fill[color=red] (2.5000,0) circle[radius={0.07/\skala}];
\fill[color=white] (2.5000,0) circle[radius={0.05/\skala}];
\fill[color=red] (2.9341,0) circle[radius={0.07/\skala}];
\fill[color=white] (2.9341,0) circle[radius={0.05/\skala}];
\fill[color=red] (3.3551,0) circle[radius={0.07/\skala}];
\fill[color=white] (3.3551,0) circle[radius={0.05/\skala}];
\fill[color=red] (3.7500,0) circle[radius={0.07/\skala}];
\fill[color=white] (3.7500,0) circle[radius={0.05/\skala}];
\fill[color=red] (4.1070,0) circle[radius={0.07/\skala}];
\fill[color=white] (4.1070,0) circle[radius={0.05/\skala}];
\fill[color=red] (4.4151,0) circle[radius={0.07/\skala}];
\fill[color=white] (4.4151,0) circle[radius={0.05/\skala}];
\fill[color=red] (4.6651,0) circle[radius={0.07/\skala}];
\fill[color=white] (4.6651,0) circle[radius={0.05/\skala}];
\fill[color=red] (4.8492,0) circle[radius={0.07/\skala}];
\fill[color=white] (4.8492,0) circle[radius={0.05/\skala}];
\fill[color=red] (4.9620,0) circle[radius={0.07/\skala}];
\fill[color=white] (4.9620,0) circle[radius={0.05/\skala}];
\fill[color=red] (5.0000,0) circle[radius={0.07/\skala}];
\fill[color=white] (5.0000,0) circle[radius={0.05/\skala}];
}
\def\punktei{18}
\def\maxfehleri{5.942\cdot 10^{-9}}
\def\fehleri{
\draw[color=red,line width=1.4pt,line join=round] ({\sx*(0.000)},{\sy*(0.0000)})
	--({\sx*(0.0100)},{\sy*(0.0177)})
	--({\sx*(0.0200)},{\sy*(0.0194)})
	--({\sx*(0.0300)},{\sy*(0.0109)})
	--({\sx*(0.0400)},{\sy*(-0.0031)})
	--({\sx*(0.0500)},{\sy*(-0.0189)})
	--({\sx*(0.0600)},{\sy*(-0.0341)})
	--({\sx*(0.0700)},{\sy*(-0.0467)})
	--({\sx*(0.0800)},{\sy*(-0.0554)})
	--({\sx*(0.0900)},{\sy*(-0.0598)})
	--({\sx*(0.1000)},{\sy*(-0.0596)})
	--({\sx*(0.1100)},{\sy*(-0.0548)})
	--({\sx*(0.1200)},{\sy*(-0.0460)})
	--({\sx*(0.1300)},{\sy*(-0.0336)})
	--({\sx*(0.1400)},{\sy*(-0.0185)})
	--({\sx*(0.1500)},{\sy*(-0.0014)})
	--({\sx*(0.1600)},{\sy*(0.0169)})
	--({\sx*(0.1700)},{\sy*(0.0355)})
	--({\sx*(0.1800)},{\sy*(0.0537)})
	--({\sx*(0.1900)},{\sy*(0.0708)})
	--({\sx*(0.2000)},{\sy*(0.0860)})
	--({\sx*(0.2100)},{\sy*(0.0989)})
	--({\sx*(0.2200)},{\sy*(0.1091)})
	--({\sx*(0.2300)},{\sy*(0.1161)})
	--({\sx*(0.2400)},{\sy*(0.1197)})
	--({\sx*(0.2500)},{\sy*(0.1199)})
	--({\sx*(0.2600)},{\sy*(0.1166)})
	--({\sx*(0.2700)},{\sy*(0.1099)})
	--({\sx*(0.2800)},{\sy*(0.0999)})
	--({\sx*(0.2900)},{\sy*(0.0869)})
	--({\sx*(0.3000)},{\sy*(0.0711)})
	--({\sx*(0.3100)},{\sy*(0.0529)})
	--({\sx*(0.3200)},{\sy*(0.0328)})
	--({\sx*(0.3300)},{\sy*(0.0111)})
	--({\sx*(0.3400)},{\sy*(-0.0116)})
	--({\sx*(0.3500)},{\sy*(-0.0350)})
	--({\sx*(0.3600)},{\sy*(-0.0585)})
	--({\sx*(0.3700)},{\sy*(-0.0816)})
	--({\sx*(0.3800)},{\sy*(-0.1039)})
	--({\sx*(0.3900)},{\sy*(-0.1250)})
	--({\sx*(0.4000)},{\sy*(-0.1444)})
	--({\sx*(0.4100)},{\sy*(-0.1618)})
	--({\sx*(0.4200)},{\sy*(-0.1768)})
	--({\sx*(0.4300)},{\sy*(-0.1892)})
	--({\sx*(0.4400)},{\sy*(-0.1987)})
	--({\sx*(0.4500)},{\sy*(-0.2051)})
	--({\sx*(0.4600)},{\sy*(-0.2083)})
	--({\sx*(0.4700)},{\sy*(-0.2083)})
	--({\sx*(0.4800)},{\sy*(-0.2049)})
	--({\sx*(0.4900)},{\sy*(-0.1982)})
	--({\sx*(0.5000)},{\sy*(-0.1882)})
	--({\sx*(0.5100)},{\sy*(-0.1752)})
	--({\sx*(0.5200)},{\sy*(-0.1591)})
	--({\sx*(0.5300)},{\sy*(-0.1403)})
	--({\sx*(0.5400)},{\sy*(-0.1190)})
	--({\sx*(0.5500)},{\sy*(-0.0954)})
	--({\sx*(0.5600)},{\sy*(-0.0699)})
	--({\sx*(0.5700)},{\sy*(-0.0427)})
	--({\sx*(0.5800)},{\sy*(-0.0143)})
	--({\sx*(0.5900)},{\sy*(0.0151)})
	--({\sx*(0.6000)},{\sy*(0.0450)})
	--({\sx*(0.6100)},{\sy*(0.0751)})
	--({\sx*(0.6200)},{\sy*(0.1051)})
	--({\sx*(0.6300)},{\sy*(0.1344)})
	--({\sx*(0.6400)},{\sy*(0.1628)})
	--({\sx*(0.6500)},{\sy*(0.1899)})
	--({\sx*(0.6600)},{\sy*(0.2155)})
	--({\sx*(0.6700)},{\sy*(0.2390)})
	--({\sx*(0.6800)},{\sy*(0.2604)})
	--({\sx*(0.6900)},{\sy*(0.2792)})
	--({\sx*(0.7000)},{\sy*(0.2954)})
	--({\sx*(0.7100)},{\sy*(0.3086)})
	--({\sx*(0.7200)},{\sy*(0.3187)})
	--({\sx*(0.7300)},{\sy*(0.3255)})
	--({\sx*(0.7400)},{\sy*(0.3290)})
	--({\sx*(0.7500)},{\sy*(0.3290)})
	--({\sx*(0.7600)},{\sy*(0.3256)})
	--({\sx*(0.7700)},{\sy*(0.3187)})
	--({\sx*(0.7800)},{\sy*(0.3084)})
	--({\sx*(0.7900)},{\sy*(0.2947)})
	--({\sx*(0.8000)},{\sy*(0.2778)})
	--({\sx*(0.8100)},{\sy*(0.2578)})
	--({\sx*(0.8200)},{\sy*(0.2349)})
	--({\sx*(0.8300)},{\sy*(0.2092)})
	--({\sx*(0.8400)},{\sy*(0.1809)})
	--({\sx*(0.8500)},{\sy*(0.1504)})
	--({\sx*(0.8600)},{\sy*(0.1179)})
	--({\sx*(0.8700)},{\sy*(0.0837)})
	--({\sx*(0.8800)},{\sy*(0.0481)})
	--({\sx*(0.8900)},{\sy*(0.0113)})
	--({\sx*(0.9000)},{\sy*(-0.0262)})
	--({\sx*(0.9100)},{\sy*(-0.0642)})
	--({\sx*(0.9200)},{\sy*(-0.1023)})
	--({\sx*(0.9300)},{\sy*(-0.1401)})
	--({\sx*(0.9400)},{\sy*(-0.1774)})
	--({\sx*(0.9500)},{\sy*(-0.2137)})
	--({\sx*(0.9600)},{\sy*(-0.2489)})
	--({\sx*(0.9700)},{\sy*(-0.2825)})
	--({\sx*(0.9800)},{\sy*(-0.3143)})
	--({\sx*(0.9900)},{\sy*(-0.3439)})
	--({\sx*(1.0000)},{\sy*(-0.3711)})
	--({\sx*(1.0100)},{\sy*(-0.3957)})
	--({\sx*(1.0200)},{\sy*(-0.4175)})
	--({\sx*(1.0300)},{\sy*(-0.4361)})
	--({\sx*(1.0400)},{\sy*(-0.4515)})
	--({\sx*(1.0500)},{\sy*(-0.4635)})
	--({\sx*(1.0600)},{\sy*(-0.4720)})
	--({\sx*(1.0700)},{\sy*(-0.4768)})
	--({\sx*(1.0800)},{\sy*(-0.4780)})
	--({\sx*(1.0900)},{\sy*(-0.4754)})
	--({\sx*(1.1000)},{\sy*(-0.4690)})
	--({\sx*(1.1100)},{\sy*(-0.4589)})
	--({\sx*(1.1200)},{\sy*(-0.4452)})
	--({\sx*(1.1300)},{\sy*(-0.4278)})
	--({\sx*(1.1400)},{\sy*(-0.4070)})
	--({\sx*(1.1500)},{\sy*(-0.3827)})
	--({\sx*(1.1600)},{\sy*(-0.3553)})
	--({\sx*(1.1700)},{\sy*(-0.3248)})
	--({\sx*(1.1800)},{\sy*(-0.2915)})
	--({\sx*(1.1900)},{\sy*(-0.2555)})
	--({\sx*(1.2000)},{\sy*(-0.2172)})
	--({\sx*(1.2100)},{\sy*(-0.1768)})
	--({\sx*(1.2200)},{\sy*(-0.1346)})
	--({\sx*(1.2300)},{\sy*(-0.0909)})
	--({\sx*(1.2400)},{\sy*(-0.0459)})
	--({\sx*(1.2500)},{\sy*(0.0000)})
	--({\sx*(1.2600)},{\sy*(0.0465)})
	--({\sx*(1.2700)},{\sy*(0.0933)})
	--({\sx*(1.2800)},{\sy*(0.1400)})
	--({\sx*(1.2900)},{\sy*(0.1863)})
	--({\sx*(1.3000)},{\sy*(0.2319)})
	--({\sx*(1.3100)},{\sy*(0.2765)})
	--({\sx*(1.3200)},{\sy*(0.3198)})
	--({\sx*(1.3300)},{\sy*(0.3614)})
	--({\sx*(1.3400)},{\sy*(0.4010)})
	--({\sx*(1.3500)},{\sy*(0.4385)})
	--({\sx*(1.3600)},{\sy*(0.4734)})
	--({\sx*(1.3700)},{\sy*(0.5056)})
	--({\sx*(1.3800)},{\sy*(0.5347)})
	--({\sx*(1.3900)},{\sy*(0.5607)})
	--({\sx*(1.4000)},{\sy*(0.5833)})
	--({\sx*(1.4100)},{\sy*(0.6023)})
	--({\sx*(1.4200)},{\sy*(0.6177)})
	--({\sx*(1.4300)},{\sy*(0.6291)})
	--({\sx*(1.4400)},{\sy*(0.6366)})
	--({\sx*(1.4500)},{\sy*(0.6401)})
	--({\sx*(1.4600)},{\sy*(0.6395)})
	--({\sx*(1.4700)},{\sy*(0.6349)})
	--({\sx*(1.4800)},{\sy*(0.6261)})
	--({\sx*(1.4900)},{\sy*(0.6133)})
	--({\sx*(1.5000)},{\sy*(0.5965)})
	--({\sx*(1.5100)},{\sy*(0.5757)})
	--({\sx*(1.5200)},{\sy*(0.5512)})
	--({\sx*(1.5300)},{\sy*(0.5229)})
	--({\sx*(1.5400)},{\sy*(0.4911)})
	--({\sx*(1.5500)},{\sy*(0.4560)})
	--({\sx*(1.5600)},{\sy*(0.4177)})
	--({\sx*(1.5700)},{\sy*(0.3765)})
	--({\sx*(1.5800)},{\sy*(0.3325)})
	--({\sx*(1.5900)},{\sy*(0.2861)})
	--({\sx*(1.6000)},{\sy*(0.2375)})
	--({\sx*(1.6100)},{\sy*(0.1870)})
	--({\sx*(1.6200)},{\sy*(0.1349)})
	--({\sx*(1.6300)},{\sy*(0.0815)})
	--({\sx*(1.6400)},{\sy*(0.0272)})
	--({\sx*(1.6500)},{\sy*(-0.0279)})
	--({\sx*(1.6600)},{\sy*(-0.0832)})
	--({\sx*(1.6700)},{\sy*(-0.1386)})
	--({\sx*(1.6800)},{\sy*(-0.1936)})
	--({\sx*(1.6900)},{\sy*(-0.2480)})
	--({\sx*(1.7000)},{\sy*(-0.3014)})
	--({\sx*(1.7100)},{\sy*(-0.3534)})
	--({\sx*(1.7200)},{\sy*(-0.4039)})
	--({\sx*(1.7300)},{\sy*(-0.4524)})
	--({\sx*(1.7400)},{\sy*(-0.4987)})
	--({\sx*(1.7500)},{\sy*(-0.5425)})
	--({\sx*(1.7600)},{\sy*(-0.5835)})
	--({\sx*(1.7700)},{\sy*(-0.6215)})
	--({\sx*(1.7800)},{\sy*(-0.6562)})
	--({\sx*(1.7900)},{\sy*(-0.6875)})
	--({\sx*(1.8000)},{\sy*(-0.7150)})
	--({\sx*(1.8100)},{\sy*(-0.7387)})
	--({\sx*(1.8200)},{\sy*(-0.7584)})
	--({\sx*(1.8300)},{\sy*(-0.7739)})
	--({\sx*(1.8400)},{\sy*(-0.7851)})
	--({\sx*(1.8500)},{\sy*(-0.7920)})
	--({\sx*(1.8600)},{\sy*(-0.7945)})
	--({\sx*(1.8700)},{\sy*(-0.7925)})
	--({\sx*(1.8800)},{\sy*(-0.7860)})
	--({\sx*(1.8900)},{\sy*(-0.7751)})
	--({\sx*(1.9000)},{\sy*(-0.7599)})
	--({\sx*(1.9100)},{\sy*(-0.7402)})
	--({\sx*(1.9200)},{\sy*(-0.7163)})
	--({\sx*(1.9300)},{\sy*(-0.6883)})
	--({\sx*(1.9400)},{\sy*(-0.6562)})
	--({\sx*(1.9500)},{\sy*(-0.6203)})
	--({\sx*(1.9600)},{\sy*(-0.5808)})
	--({\sx*(1.9700)},{\sy*(-0.5377)})
	--({\sx*(1.9800)},{\sy*(-0.4915)})
	--({\sx*(1.9900)},{\sy*(-0.4422)})
	--({\sx*(2.0000)},{\sy*(-0.3902)})
	--({\sx*(2.0100)},{\sy*(-0.3357)})
	--({\sx*(2.0200)},{\sy*(-0.2791)})
	--({\sx*(2.0300)},{\sy*(-0.2206)})
	--({\sx*(2.0400)},{\sy*(-0.1605)})
	--({\sx*(2.0500)},{\sy*(-0.0991)})
	--({\sx*(2.0600)},{\sy*(-0.0369)})
	--({\sx*(2.0700)},{\sy*(0.0259)})
	--({\sx*(2.0800)},{\sy*(0.0890)})
	--({\sx*(2.0900)},{\sy*(0.1519)})
	--({\sx*(2.1000)},{\sy*(0.2144)})
	--({\sx*(2.1100)},{\sy*(0.2761)})
	--({\sx*(2.1200)},{\sy*(0.3367)})
	--({\sx*(2.1300)},{\sy*(0.3958)})
	--({\sx*(2.1400)},{\sy*(0.4531)})
	--({\sx*(2.1500)},{\sy*(0.5083)})
	--({\sx*(2.1600)},{\sy*(0.5611)})
	--({\sx*(2.1700)},{\sy*(0.6111)})
	--({\sx*(2.1800)},{\sy*(0.6582)})
	--({\sx*(2.1900)},{\sy*(0.7020)})
	--({\sx*(2.2000)},{\sy*(0.7423)})
	--({\sx*(2.2100)},{\sy*(0.7789)})
	--({\sx*(2.2200)},{\sy*(0.8115)})
	--({\sx*(2.2300)},{\sy*(0.8400)})
	--({\sx*(2.2400)},{\sy*(0.8642)})
	--({\sx*(2.2500)},{\sy*(0.8839)})
	--({\sx*(2.2600)},{\sy*(0.8990)})
	--({\sx*(2.2700)},{\sy*(0.9095)})
	--({\sx*(2.2800)},{\sy*(0.9152)})
	--({\sx*(2.2900)},{\sy*(0.9160)})
	--({\sx*(2.3000)},{\sy*(0.9121)})
	--({\sx*(2.3100)},{\sy*(0.9034)})
	--({\sx*(2.3200)},{\sy*(0.8898)})
	--({\sx*(2.3300)},{\sy*(0.8715)})
	--({\sx*(2.3400)},{\sy*(0.8486)})
	--({\sx*(2.3500)},{\sy*(0.8210)})
	--({\sx*(2.3600)},{\sy*(0.7891)})
	--({\sx*(2.3700)},{\sy*(0.7529)})
	--({\sx*(2.3800)},{\sy*(0.7126)})
	--({\sx*(2.3900)},{\sy*(0.6683)})
	--({\sx*(2.4000)},{\sy*(0.6204)})
	--({\sx*(2.4100)},{\sy*(0.5691)})
	--({\sx*(2.4200)},{\sy*(0.5145)})
	--({\sx*(2.4300)},{\sy*(0.4571)})
	--({\sx*(2.4400)},{\sy*(0.3971)})
	--({\sx*(2.4500)},{\sy*(0.3347)})
	--({\sx*(2.4600)},{\sy*(0.2704)})
	--({\sx*(2.4700)},{\sy*(0.2044)})
	--({\sx*(2.4800)},{\sy*(0.1371)})
	--({\sx*(2.4900)},{\sy*(0.0688)})
	--({\sx*(2.5000)},{\sy*(0.0000)})
	--({\sx*(2.5100)},{\sy*(-0.0691)})
	--({\sx*(2.5200)},{\sy*(-0.1380)})
	--({\sx*(2.5300)},{\sy*(-0.2065)})
	--({\sx*(2.5400)},{\sy*(-0.2741)})
	--({\sx*(2.5500)},{\sy*(-0.3405)})
	--({\sx*(2.5600)},{\sy*(-0.4054)})
	--({\sx*(2.5700)},{\sy*(-0.4683)})
	--({\sx*(2.5800)},{\sy*(-0.5289)})
	--({\sx*(2.5900)},{\sy*(-0.5870)})
	--({\sx*(2.6000)},{\sy*(-0.6422)})
	--({\sx*(2.6100)},{\sy*(-0.6942)})
	--({\sx*(2.6200)},{\sy*(-0.7427)})
	--({\sx*(2.6300)},{\sy*(-0.7874)})
	--({\sx*(2.6400)},{\sy*(-0.8281)})
	--({\sx*(2.6500)},{\sy*(-0.8647)})
	--({\sx*(2.6600)},{\sy*(-0.8967)})
	--({\sx*(2.6700)},{\sy*(-0.9242)})
	--({\sx*(2.6800)},{\sy*(-0.9469)})
	--({\sx*(2.6900)},{\sy*(-0.9646)})
	--({\sx*(2.7000)},{\sy*(-0.9774)})
	--({\sx*(2.7100)},{\sy*(-0.9850)})
	--({\sx*(2.7200)},{\sy*(-0.9875)})
	--({\sx*(2.7300)},{\sy*(-0.9848)})
	--({\sx*(2.7400)},{\sy*(-0.9769)})
	--({\sx*(2.7500)},{\sy*(-0.9638)})
	--({\sx*(2.7600)},{\sy*(-0.9456)})
	--({\sx*(2.7700)},{\sy*(-0.9224)})
	--({\sx*(2.7800)},{\sy*(-0.8943)})
	--({\sx*(2.7900)},{\sy*(-0.8614)})
	--({\sx*(2.8000)},{\sy*(-0.8239)})
	--({\sx*(2.8100)},{\sy*(-0.7819)})
	--({\sx*(2.8200)},{\sy*(-0.7357)})
	--({\sx*(2.8300)},{\sy*(-0.6855)})
	--({\sx*(2.8400)},{\sy*(-0.6316)})
	--({\sx*(2.8500)},{\sy*(-0.5742)})
	--({\sx*(2.8600)},{\sy*(-0.5137)})
	--({\sx*(2.8700)},{\sy*(-0.4503)})
	--({\sx*(2.8800)},{\sy*(-0.3844)})
	--({\sx*(2.8900)},{\sy*(-0.3164)})
	--({\sx*(2.9000)},{\sy*(-0.2466)})
	--({\sx*(2.9100)},{\sy*(-0.1754)})
	--({\sx*(2.9200)},{\sy*(-0.1031)})
	--({\sx*(2.9300)},{\sy*(-0.0302)})
	--({\sx*(2.9400)},{\sy*(0.0430)})
	--({\sx*(2.9500)},{\sy*(0.1161)})
	--({\sx*(2.9600)},{\sy*(0.1886)})
	--({\sx*(2.9700)},{\sy*(0.2602)})
	--({\sx*(2.9800)},{\sy*(0.3305)})
	--({\sx*(2.9900)},{\sy*(0.3990)})
	--({\sx*(3.0000)},{\sy*(0.4655)})
	--({\sx*(3.0100)},{\sy*(0.5295)})
	--({\sx*(3.0200)},{\sy*(0.5907)})
	--({\sx*(3.0300)},{\sy*(0.6487)})
	--({\sx*(3.0400)},{\sy*(0.7032)})
	--({\sx*(3.0500)},{\sy*(0.7539)})
	--({\sx*(3.0600)},{\sy*(0.8006)})
	--({\sx*(3.0700)},{\sy*(0.8428)})
	--({\sx*(3.0800)},{\sy*(0.8805)})
	--({\sx*(3.0900)},{\sy*(0.9133)})
	--({\sx*(3.1000)},{\sy*(0.9411)})
	--({\sx*(3.1100)},{\sy*(0.9638)})
	--({\sx*(3.1200)},{\sy*(0.9811)})
	--({\sx*(3.1300)},{\sy*(0.9929)})
	--({\sx*(3.1400)},{\sy*(0.9992)})
	--({\sx*(3.1500)},{\sy*(1.0000)})
	--({\sx*(3.1600)},{\sy*(0.9952)})
	--({\sx*(3.1700)},{\sy*(0.9848)})
	--({\sx*(3.1800)},{\sy*(0.9688)})
	--({\sx*(3.1900)},{\sy*(0.9474)})
	--({\sx*(3.2000)},{\sy*(0.9207)})
	--({\sx*(3.2100)},{\sy*(0.8887)})
	--({\sx*(3.2200)},{\sy*(0.8517)})
	--({\sx*(3.2300)},{\sy*(0.8099)})
	--({\sx*(3.2400)},{\sy*(0.7635)})
	--({\sx*(3.2500)},{\sy*(0.7127)})
	--({\sx*(3.2600)},{\sy*(0.6578)})
	--({\sx*(3.2700)},{\sy*(0.5991)})
	--({\sx*(3.2800)},{\sy*(0.5371)})
	--({\sx*(3.2900)},{\sy*(0.4719)})
	--({\sx*(3.3000)},{\sy*(0.4040)})
	--({\sx*(3.3100)},{\sy*(0.3338)})
	--({\sx*(3.3200)},{\sy*(0.2617)})
	--({\sx*(3.3300)},{\sy*(0.1881)})
	--({\sx*(3.3400)},{\sy*(0.1134)})
	--({\sx*(3.3500)},{\sy*(0.0381)})
	--({\sx*(3.3600)},{\sy*(-0.0374)})
	--({\sx*(3.3700)},{\sy*(-0.1126)})
	--({\sx*(3.3800)},{\sy*(-0.1871)})
	--({\sx*(3.3900)},{\sy*(-0.2604)})
	--({\sx*(3.4000)},{\sy*(-0.3321)})
	--({\sx*(3.4100)},{\sy*(-0.4018)})
	--({\sx*(3.4200)},{\sy*(-0.4690)})
	--({\sx*(3.4300)},{\sy*(-0.5333)})
	--({\sx*(3.4400)},{\sy*(-0.5943)})
	--({\sx*(3.4500)},{\sy*(-0.6517)})
	--({\sx*(3.4600)},{\sy*(-0.7050)})
	--({\sx*(3.4700)},{\sy*(-0.7540)})
	--({\sx*(3.4800)},{\sy*(-0.7983)})
	--({\sx*(3.4900)},{\sy*(-0.8376)})
	--({\sx*(3.5000)},{\sy*(-0.8717)})
	--({\sx*(3.5100)},{\sy*(-0.9004)})
	--({\sx*(3.5200)},{\sy*(-0.9234)})
	--({\sx*(3.5300)},{\sy*(-0.9406)})
	--({\sx*(3.5400)},{\sy*(-0.9519)})
	--({\sx*(3.5500)},{\sy*(-0.9572)})
	--({\sx*(3.5600)},{\sy*(-0.9564)})
	--({\sx*(3.5700)},{\sy*(-0.9495)})
	--({\sx*(3.5800)},{\sy*(-0.9366)})
	--({\sx*(3.5900)},{\sy*(-0.9178)})
	--({\sx*(3.6000)},{\sy*(-0.8930)})
	--({\sx*(3.6100)},{\sy*(-0.8626)})
	--({\sx*(3.6200)},{\sy*(-0.8266)})
	--({\sx*(3.6300)},{\sy*(-0.7852)})
	--({\sx*(3.6400)},{\sy*(-0.7388)})
	--({\sx*(3.6500)},{\sy*(-0.6877)})
	--({\sx*(3.6600)},{\sy*(-0.6321)})
	--({\sx*(3.6700)},{\sy*(-0.5724)})
	--({\sx*(3.6800)},{\sy*(-0.5091)})
	--({\sx*(3.6900)},{\sy*(-0.4424)})
	--({\sx*(3.7000)},{\sy*(-0.3729)})
	--({\sx*(3.7100)},{\sy*(-0.3011)})
	--({\sx*(3.7200)},{\sy*(-0.2274)})
	--({\sx*(3.7300)},{\sy*(-0.1523)})
	--({\sx*(3.7400)},{\sy*(-0.0763)})
	--({\sx*(3.7500)},{\sy*(0.0000)})
	--({\sx*(3.7600)},{\sy*(0.0761)})
	--({\sx*(3.7700)},{\sy*(0.1515)})
	--({\sx*(3.7800)},{\sy*(0.2256)})
	--({\sx*(3.7900)},{\sy*(0.2980)})
	--({\sx*(3.8000)},{\sy*(0.3680)})
	--({\sx*(3.8100)},{\sy*(0.4352)})
	--({\sx*(3.8200)},{\sy*(0.4991)})
	--({\sx*(3.8300)},{\sy*(0.5592)})
	--({\sx*(3.8400)},{\sy*(0.6150)})
	--({\sx*(3.8500)},{\sy*(0.6662)})
	--({\sx*(3.8600)},{\sy*(0.7124)})
	--({\sx*(3.8700)},{\sy*(0.7531)})
	--({\sx*(3.8800)},{\sy*(0.7881)})
	--({\sx*(3.8900)},{\sy*(0.8171)})
	--({\sx*(3.9000)},{\sy*(0.8398)})
	--({\sx*(3.9100)},{\sy*(0.8561)})
	--({\sx*(3.9200)},{\sy*(0.8657)})
	--({\sx*(3.9300)},{\sy*(0.8687)})
	--({\sx*(3.9400)},{\sy*(0.8650)})
	--({\sx*(3.9500)},{\sy*(0.8545)})
	--({\sx*(3.9600)},{\sy*(0.8374)})
	--({\sx*(3.9700)},{\sy*(0.8137)})
	--({\sx*(3.9800)},{\sy*(0.7836)})
	--({\sx*(3.9900)},{\sy*(0.7473)})
	--({\sx*(4.0000)},{\sy*(0.7052)})
	--({\sx*(4.0100)},{\sy*(0.6574)})
	--({\sx*(4.0200)},{\sy*(0.6045)})
	--({\sx*(4.0300)},{\sy*(0.5468)})
	--({\sx*(4.0400)},{\sy*(0.4848)})
	--({\sx*(4.0500)},{\sy*(0.4190)})
	--({\sx*(4.0600)},{\sy*(0.3499)})
	--({\sx*(4.0700)},{\sy*(0.2782)})
	--({\sx*(4.0800)},{\sy*(0.2044)})
	--({\sx*(4.0900)},{\sy*(0.1291)})
	--({\sx*(4.1000)},{\sy*(0.0531)})
	--({\sx*(4.1100)},{\sy*(-0.0230)})
	--({\sx*(4.1200)},{\sy*(-0.0986)})
	--({\sx*(4.1300)},{\sy*(-0.1729)})
	--({\sx*(4.1400)},{\sy*(-0.2453)})
	--({\sx*(4.1500)},{\sy*(-0.3151)})
	--({\sx*(4.1600)},{\sy*(-0.3816)})
	--({\sx*(4.1700)},{\sy*(-0.4441)})
	--({\sx*(4.1800)},{\sy*(-0.5022)})
	--({\sx*(4.1900)},{\sy*(-0.5552)})
	--({\sx*(4.2000)},{\sy*(-0.6026)})
	--({\sx*(4.2100)},{\sy*(-0.6438)})
	--({\sx*(4.2200)},{\sy*(-0.6785)})
	--({\sx*(4.2300)},{\sy*(-0.7063)})
	--({\sx*(4.2400)},{\sy*(-0.7269)})
	--({\sx*(4.2500)},{\sy*(-0.7400)})
	--({\sx*(4.2600)},{\sy*(-0.7454)})
	--({\sx*(4.2700)},{\sy*(-0.7432)})
	--({\sx*(4.2800)},{\sy*(-0.7331)})
	--({\sx*(4.2900)},{\sy*(-0.7154)})
	--({\sx*(4.3000)},{\sy*(-0.6901)})
	--({\sx*(4.3100)},{\sy*(-0.6575)})
	--({\sx*(4.3200)},{\sy*(-0.6179)})
	--({\sx*(4.3300)},{\sy*(-0.5717)})
	--({\sx*(4.3400)},{\sy*(-0.5195)})
	--({\sx*(4.3500)},{\sy*(-0.4617)})
	--({\sx*(4.3600)},{\sy*(-0.3990)})
	--({\sx*(4.3700)},{\sy*(-0.3320)})
	--({\sx*(4.3800)},{\sy*(-0.2617)})
	--({\sx*(4.3900)},{\sy*(-0.1888)})
	--({\sx*(4.4000)},{\sy*(-0.1141)})
	--({\sx*(4.4100)},{\sy*(-0.0386)})
	--({\sx*(4.4200)},{\sy*(0.0368)})
	--({\sx*(4.4300)},{\sy*(0.1110)})
	--({\sx*(4.4400)},{\sy*(0.1832)})
	--({\sx*(4.4500)},{\sy*(0.2524)})
	--({\sx*(4.4600)},{\sy*(0.3176)})
	--({\sx*(4.4700)},{\sy*(0.3779)})
	--({\sx*(4.4800)},{\sy*(0.4324)})
	--({\sx*(4.4900)},{\sy*(0.4804)})
	--({\sx*(4.5000)},{\sy*(0.5210)})
	--({\sx*(4.5100)},{\sy*(0.5537)})
	--({\sx*(4.5200)},{\sy*(0.5779)})
	--({\sx*(4.5300)},{\sy*(0.5931)})
	--({\sx*(4.5400)},{\sy*(0.5991)})
	--({\sx*(4.5500)},{\sy*(0.5956)})
	--({\sx*(4.5600)},{\sy*(0.5826)})
	--({\sx*(4.5700)},{\sy*(0.5603)})
	--({\sx*(4.5800)},{\sy*(0.5289)})
	--({\sx*(4.5900)},{\sy*(0.4888)})
	--({\sx*(4.6000)},{\sy*(0.4408)})
	--({\sx*(4.6100)},{\sy*(0.3855)})
	--({\sx*(4.6200)},{\sy*(0.3239)})
	--({\sx*(4.6300)},{\sy*(0.2570)})
	--({\sx*(4.6400)},{\sy*(0.1861)})
	--({\sx*(4.6500)},{\sy*(0.1126)})
	--({\sx*(4.6600)},{\sy*(0.0378)})
	--({\sx*(4.6700)},{\sy*(-0.0366)})
	--({\sx*(4.6800)},{\sy*(-0.1092)})
	--({\sx*(4.6900)},{\sy*(-0.1782)})
	--({\sx*(4.7000)},{\sy*(-0.2421)})
	--({\sx*(4.7100)},{\sy*(-0.2992)})
	--({\sx*(4.7200)},{\sy*(-0.3482)})
	--({\sx*(4.7300)},{\sy*(-0.3876)})
	--({\sx*(4.7400)},{\sy*(-0.4163)})
	--({\sx*(4.7500)},{\sy*(-0.4334)})
	--({\sx*(4.7600)},{\sy*(-0.4380)})
	--({\sx*(4.7700)},{\sy*(-0.4299)})
	--({\sx*(4.7800)},{\sy*(-0.4091)})
	--({\sx*(4.7900)},{\sy*(-0.3758)})
	--({\sx*(4.8000)},{\sy*(-0.3310)})
	--({\sx*(4.8100)},{\sy*(-0.2760)})
	--({\sx*(4.8200)},{\sy*(-0.2123)})
	--({\sx*(4.8300)},{\sy*(-0.1423)})
	--({\sx*(4.8400)},{\sy*(-0.0687)})
	--({\sx*(4.8500)},{\sy*(0.0057)})
	--({\sx*(4.8600)},{\sy*(0.0772)})
	--({\sx*(4.8700)},{\sy*(0.1424)})
	--({\sx*(4.8800)},{\sy*(0.1976)})
	--({\sx*(4.8900)},{\sy*(0.2391)})
	--({\sx*(4.9000)},{\sy*(0.2637)})
	--({\sx*(4.9100)},{\sy*(0.2689)})
	--({\sx*(4.9200)},{\sy*(0.2531)})
	--({\sx*(4.9300)},{\sy*(0.2163)})
	--({\sx*(4.9400)},{\sy*(0.1606)})
	--({\sx*(4.9500)},{\sy*(0.0907)})
	--({\sx*(4.9600)},{\sy*(0.0150)})
	--({\sx*(4.9700)},{\sy*(-0.0539)})
	--({\sx*(4.9800)},{\sy*(-0.0974)})
	--({\sx*(4.9900)},{\sy*(-0.0906)})
	--({\sx*(5.0000)},{\sy*(0.0000)});
}
\def\relfehleri{
\draw[color=blue,line width=1.4pt,line join=round] ({\sx*(0.000)},{\sy*(0.0000)})
	--({\sx*(0.0100)},{\sy*(0.0000)})
	--({\sx*(0.0200)},{\sy*(0.0000)})
	--({\sx*(0.0300)},{\sy*(0.0000)})
	--({\sx*(0.0400)},{\sy*(-0.0000)})
	--({\sx*(0.0500)},{\sy*(-0.0000)})
	--({\sx*(0.0600)},{\sy*(-0.0000)})
	--({\sx*(0.0700)},{\sy*(-0.0000)})
	--({\sx*(0.0800)},{\sy*(-0.0000)})
	--({\sx*(0.0900)},{\sy*(-0.0000)})
	--({\sx*(0.1000)},{\sy*(-0.0000)})
	--({\sx*(0.1100)},{\sy*(-0.0000)})
	--({\sx*(0.1200)},{\sy*(-0.0000)})
	--({\sx*(0.1300)},{\sy*(-0.0000)})
	--({\sx*(0.1400)},{\sy*(-0.0000)})
	--({\sx*(0.1500)},{\sy*(-0.0000)})
	--({\sx*(0.1600)},{\sy*(0.0000)})
	--({\sx*(0.1700)},{\sy*(0.0000)})
	--({\sx*(0.1800)},{\sy*(0.0000)})
	--({\sx*(0.1900)},{\sy*(0.0000)})
	--({\sx*(0.2000)},{\sy*(0.0000)})
	--({\sx*(0.2100)},{\sy*(0.0000)})
	--({\sx*(0.2200)},{\sy*(0.0000)})
	--({\sx*(0.2300)},{\sy*(0.0000)})
	--({\sx*(0.2400)},{\sy*(0.0000)})
	--({\sx*(0.2500)},{\sy*(0.0000)})
	--({\sx*(0.2600)},{\sy*(0.0000)})
	--({\sx*(0.2700)},{\sy*(0.0000)})
	--({\sx*(0.2800)},{\sy*(0.0000)})
	--({\sx*(0.2900)},{\sy*(0.0000)})
	--({\sx*(0.3000)},{\sy*(0.0000)})
	--({\sx*(0.3100)},{\sy*(0.0000)})
	--({\sx*(0.3200)},{\sy*(0.0000)})
	--({\sx*(0.3300)},{\sy*(0.0000)})
	--({\sx*(0.3400)},{\sy*(-0.0000)})
	--({\sx*(0.3500)},{\sy*(-0.0000)})
	--({\sx*(0.3600)},{\sy*(-0.0000)})
	--({\sx*(0.3700)},{\sy*(-0.0000)})
	--({\sx*(0.3800)},{\sy*(-0.0000)})
	--({\sx*(0.3900)},{\sy*(-0.0000)})
	--({\sx*(0.4000)},{\sy*(-0.0000)})
	--({\sx*(0.4100)},{\sy*(-0.0000)})
	--({\sx*(0.4200)},{\sy*(-0.0000)})
	--({\sx*(0.4300)},{\sy*(-0.0000)})
	--({\sx*(0.4400)},{\sy*(-0.0000)})
	--({\sx*(0.4500)},{\sy*(-0.0000)})
	--({\sx*(0.4600)},{\sy*(-0.0000)})
	--({\sx*(0.4700)},{\sy*(-0.0000)})
	--({\sx*(0.4800)},{\sy*(-0.0000)})
	--({\sx*(0.4900)},{\sy*(-0.0000)})
	--({\sx*(0.5000)},{\sy*(-0.0000)})
	--({\sx*(0.5100)},{\sy*(-0.0000)})
	--({\sx*(0.5200)},{\sy*(-0.0000)})
	--({\sx*(0.5300)},{\sy*(-0.0000)})
	--({\sx*(0.5400)},{\sy*(-0.0000)})
	--({\sx*(0.5500)},{\sy*(-0.0000)})
	--({\sx*(0.5600)},{\sy*(-0.0000)})
	--({\sx*(0.5700)},{\sy*(-0.0000)})
	--({\sx*(0.5800)},{\sy*(-0.0000)})
	--({\sx*(0.5900)},{\sy*(0.0000)})
	--({\sx*(0.6000)},{\sy*(0.0000)})
	--({\sx*(0.6100)},{\sy*(0.0000)})
	--({\sx*(0.6200)},{\sy*(0.0000)})
	--({\sx*(0.6300)},{\sy*(0.0000)})
	--({\sx*(0.6400)},{\sy*(0.0000)})
	--({\sx*(0.6500)},{\sy*(0.0000)})
	--({\sx*(0.6600)},{\sy*(0.0000)})
	--({\sx*(0.6700)},{\sy*(0.0000)})
	--({\sx*(0.6800)},{\sy*(0.0000)})
	--({\sx*(0.6900)},{\sy*(0.0000)})
	--({\sx*(0.7000)},{\sy*(0.0000)})
	--({\sx*(0.7100)},{\sy*(0.0000)})
	--({\sx*(0.7200)},{\sy*(0.0000)})
	--({\sx*(0.7300)},{\sy*(0.0000)})
	--({\sx*(0.7400)},{\sy*(0.0000)})
	--({\sx*(0.7500)},{\sy*(0.0000)})
	--({\sx*(0.7600)},{\sy*(0.0000)})
	--({\sx*(0.7700)},{\sy*(0.0000)})
	--({\sx*(0.7800)},{\sy*(0.0000)})
	--({\sx*(0.7900)},{\sy*(0.0000)})
	--({\sx*(0.8000)},{\sy*(0.0000)})
	--({\sx*(0.8100)},{\sy*(0.0000)})
	--({\sx*(0.8200)},{\sy*(0.0000)})
	--({\sx*(0.8300)},{\sy*(0.0000)})
	--({\sx*(0.8400)},{\sy*(0.0000)})
	--({\sx*(0.8500)},{\sy*(0.0000)})
	--({\sx*(0.8600)},{\sy*(0.0000)})
	--({\sx*(0.8700)},{\sy*(0.0000)})
	--({\sx*(0.8800)},{\sy*(0.0000)})
	--({\sx*(0.8900)},{\sy*(0.0000)})
	--({\sx*(0.9000)},{\sy*(-0.0000)})
	--({\sx*(0.9100)},{\sy*(-0.0000)})
	--({\sx*(0.9200)},{\sy*(-0.0000)})
	--({\sx*(0.9300)},{\sy*(-0.0000)})
	--({\sx*(0.9400)},{\sy*(-0.0000)})
	--({\sx*(0.9500)},{\sy*(-0.0000)})
	--({\sx*(0.9600)},{\sy*(-0.0000)})
	--({\sx*(0.9700)},{\sy*(-0.0000)})
	--({\sx*(0.9800)},{\sy*(-0.0000)})
	--({\sx*(0.9900)},{\sy*(-0.0000)})
	--({\sx*(1.0000)},{\sy*(-0.0000)})
	--({\sx*(1.0100)},{\sy*(-0.0000)})
	--({\sx*(1.0200)},{\sy*(-0.0000)})
	--({\sx*(1.0300)},{\sy*(-0.0000)})
	--({\sx*(1.0400)},{\sy*(-0.0000)})
	--({\sx*(1.0500)},{\sy*(-0.0000)})
	--({\sx*(1.0600)},{\sy*(-0.0000)})
	--({\sx*(1.0700)},{\sy*(-0.0000)})
	--({\sx*(1.0800)},{\sy*(-0.0000)})
	--({\sx*(1.0900)},{\sy*(-0.0000)})
	--({\sx*(1.1000)},{\sy*(-0.0000)})
	--({\sx*(1.1100)},{\sy*(-0.0000)})
	--({\sx*(1.1200)},{\sy*(-0.0000)})
	--({\sx*(1.1300)},{\sy*(-0.0000)})
	--({\sx*(1.1400)},{\sy*(-0.0000)})
	--({\sx*(1.1500)},{\sy*(-0.0000)})
	--({\sx*(1.1600)},{\sy*(-0.0000)})
	--({\sx*(1.1700)},{\sy*(-0.0000)})
	--({\sx*(1.1800)},{\sy*(-0.0000)})
	--({\sx*(1.1900)},{\sy*(-0.0000)})
	--({\sx*(1.2000)},{\sy*(-0.0000)})
	--({\sx*(1.2100)},{\sy*(-0.0000)})
	--({\sx*(1.2200)},{\sy*(-0.0000)})
	--({\sx*(1.2300)},{\sy*(-0.0000)})
	--({\sx*(1.2400)},{\sy*(-0.0000)})
	--({\sx*(1.2500)},{\sy*(0.0000)})
	--({\sx*(1.2600)},{\sy*(0.0000)})
	--({\sx*(1.2700)},{\sy*(0.0000)})
	--({\sx*(1.2800)},{\sy*(0.0000)})
	--({\sx*(1.2900)},{\sy*(0.0000)})
	--({\sx*(1.3000)},{\sy*(0.0000)})
	--({\sx*(1.3100)},{\sy*(0.0000)})
	--({\sx*(1.3200)},{\sy*(0.0000)})
	--({\sx*(1.3300)},{\sy*(0.0000)})
	--({\sx*(1.3400)},{\sy*(0.0000)})
	--({\sx*(1.3500)},{\sy*(0.0000)})
	--({\sx*(1.3600)},{\sy*(0.0000)})
	--({\sx*(1.3700)},{\sy*(0.0000)})
	--({\sx*(1.3800)},{\sy*(0.0000)})
	--({\sx*(1.3900)},{\sy*(0.0000)})
	--({\sx*(1.4000)},{\sy*(0.0000)})
	--({\sx*(1.4100)},{\sy*(0.0000)})
	--({\sx*(1.4200)},{\sy*(0.0000)})
	--({\sx*(1.4300)},{\sy*(0.0000)})
	--({\sx*(1.4400)},{\sy*(0.0000)})
	--({\sx*(1.4500)},{\sy*(0.0000)})
	--({\sx*(1.4600)},{\sy*(0.0000)})
	--({\sx*(1.4700)},{\sy*(0.0000)})
	--({\sx*(1.4800)},{\sy*(0.0000)})
	--({\sx*(1.4900)},{\sy*(0.0000)})
	--({\sx*(1.5000)},{\sy*(0.0000)})
	--({\sx*(1.5100)},{\sy*(0.0000)})
	--({\sx*(1.5200)},{\sy*(0.0000)})
	--({\sx*(1.5300)},{\sy*(0.0000)})
	--({\sx*(1.5400)},{\sy*(0.0000)})
	--({\sx*(1.5500)},{\sy*(0.0000)})
	--({\sx*(1.5600)},{\sy*(0.0000)})
	--({\sx*(1.5700)},{\sy*(0.0000)})
	--({\sx*(1.5800)},{\sy*(0.0000)})
	--({\sx*(1.5900)},{\sy*(0.0000)})
	--({\sx*(1.6000)},{\sy*(0.0000)})
	--({\sx*(1.6100)},{\sy*(0.0000)})
	--({\sx*(1.6200)},{\sy*(0.0000)})
	--({\sx*(1.6300)},{\sy*(0.0000)})
	--({\sx*(1.6400)},{\sy*(0.0000)})
	--({\sx*(1.6500)},{\sy*(-0.0000)})
	--({\sx*(1.6600)},{\sy*(-0.0000)})
	--({\sx*(1.6700)},{\sy*(-0.0000)})
	--({\sx*(1.6800)},{\sy*(-0.0000)})
	--({\sx*(1.6900)},{\sy*(-0.0000)})
	--({\sx*(1.7000)},{\sy*(-0.0000)})
	--({\sx*(1.7100)},{\sy*(-0.0000)})
	--({\sx*(1.7200)},{\sy*(-0.0000)})
	--({\sx*(1.7300)},{\sy*(-0.0000)})
	--({\sx*(1.7400)},{\sy*(-0.0000)})
	--({\sx*(1.7500)},{\sy*(-0.0000)})
	--({\sx*(1.7600)},{\sy*(-0.0000)})
	--({\sx*(1.7700)},{\sy*(-0.0000)})
	--({\sx*(1.7800)},{\sy*(-0.0000)})
	--({\sx*(1.7900)},{\sy*(-0.0000)})
	--({\sx*(1.8000)},{\sy*(-0.0000)})
	--({\sx*(1.8100)},{\sy*(-0.0000)})
	--({\sx*(1.8200)},{\sy*(-0.0000)})
	--({\sx*(1.8300)},{\sy*(-0.0000)})
	--({\sx*(1.8400)},{\sy*(-0.0000)})
	--({\sx*(1.8500)},{\sy*(-0.0000)})
	--({\sx*(1.8600)},{\sy*(-0.0000)})
	--({\sx*(1.8700)},{\sy*(-0.0000)})
	--({\sx*(1.8800)},{\sy*(-0.0000)})
	--({\sx*(1.8900)},{\sy*(-0.0000)})
	--({\sx*(1.9000)},{\sy*(-0.0000)})
	--({\sx*(1.9100)},{\sy*(-0.0000)})
	--({\sx*(1.9200)},{\sy*(-0.0000)})
	--({\sx*(1.9300)},{\sy*(-0.0000)})
	--({\sx*(1.9400)},{\sy*(-0.0000)})
	--({\sx*(1.9500)},{\sy*(-0.0000)})
	--({\sx*(1.9600)},{\sy*(-0.0000)})
	--({\sx*(1.9700)},{\sy*(-0.0000)})
	--({\sx*(1.9800)},{\sy*(-0.0000)})
	--({\sx*(1.9900)},{\sy*(-0.0000)})
	--({\sx*(2.0000)},{\sy*(-0.0000)})
	--({\sx*(2.0100)},{\sy*(-0.0000)})
	--({\sx*(2.0200)},{\sy*(-0.0000)})
	--({\sx*(2.0300)},{\sy*(-0.0000)})
	--({\sx*(2.0400)},{\sy*(-0.0000)})
	--({\sx*(2.0500)},{\sy*(-0.0000)})
	--({\sx*(2.0600)},{\sy*(-0.0000)})
	--({\sx*(2.0700)},{\sy*(0.0000)})
	--({\sx*(2.0800)},{\sy*(0.0000)})
	--({\sx*(2.0900)},{\sy*(0.0000)})
	--({\sx*(2.1000)},{\sy*(0.0000)})
	--({\sx*(2.1100)},{\sy*(0.0000)})
	--({\sx*(2.1200)},{\sy*(0.0000)})
	--({\sx*(2.1300)},{\sy*(0.0000)})
	--({\sx*(2.1400)},{\sy*(0.0000)})
	--({\sx*(2.1500)},{\sy*(0.0000)})
	--({\sx*(2.1600)},{\sy*(0.0000)})
	--({\sx*(2.1700)},{\sy*(0.0000)})
	--({\sx*(2.1800)},{\sy*(0.0000)})
	--({\sx*(2.1900)},{\sy*(0.0000)})
	--({\sx*(2.2000)},{\sy*(0.0000)})
	--({\sx*(2.2100)},{\sy*(0.0000)})
	--({\sx*(2.2200)},{\sy*(0.0000)})
	--({\sx*(2.2300)},{\sy*(0.0000)})
	--({\sx*(2.2400)},{\sy*(0.0000)})
	--({\sx*(2.2500)},{\sy*(0.0000)})
	--({\sx*(2.2600)},{\sy*(0.0000)})
	--({\sx*(2.2700)},{\sy*(0.0000)})
	--({\sx*(2.2800)},{\sy*(0.0000)})
	--({\sx*(2.2900)},{\sy*(0.0000)})
	--({\sx*(2.3000)},{\sy*(0.0000)})
	--({\sx*(2.3100)},{\sy*(0.0000)})
	--({\sx*(2.3200)},{\sy*(0.0000)})
	--({\sx*(2.3300)},{\sy*(0.0000)})
	--({\sx*(2.3400)},{\sy*(0.0000)})
	--({\sx*(2.3500)},{\sy*(0.0000)})
	--({\sx*(2.3600)},{\sy*(0.0000)})
	--({\sx*(2.3700)},{\sy*(0.0000)})
	--({\sx*(2.3800)},{\sy*(0.0000)})
	--({\sx*(2.3900)},{\sy*(0.0000)})
	--({\sx*(2.4000)},{\sy*(0.0000)})
	--({\sx*(2.4100)},{\sy*(0.0000)})
	--({\sx*(2.4200)},{\sy*(0.0000)})
	--({\sx*(2.4300)},{\sy*(0.0000)})
	--({\sx*(2.4400)},{\sy*(0.0000)})
	--({\sx*(2.4500)},{\sy*(0.0000)})
	--({\sx*(2.4600)},{\sy*(0.0000)})
	--({\sx*(2.4700)},{\sy*(0.0000)})
	--({\sx*(2.4800)},{\sy*(0.0000)})
	--({\sx*(2.4900)},{\sy*(0.0000)})
	--({\sx*(2.5000)},{\sy*(0.0000)})
	--({\sx*(2.5100)},{\sy*(-0.0000)})
	--({\sx*(2.5200)},{\sy*(-0.0000)})
	--({\sx*(2.5300)},{\sy*(-0.0000)})
	--({\sx*(2.5400)},{\sy*(-0.0000)})
	--({\sx*(2.5500)},{\sy*(-0.0000)})
	--({\sx*(2.5600)},{\sy*(-0.0000)})
	--({\sx*(2.5700)},{\sy*(-0.0000)})
	--({\sx*(2.5800)},{\sy*(-0.0000)})
	--({\sx*(2.5900)},{\sy*(-0.0000)})
	--({\sx*(2.6000)},{\sy*(-0.0000)})
	--({\sx*(2.6100)},{\sy*(-0.0000)})
	--({\sx*(2.6200)},{\sy*(-0.0000)})
	--({\sx*(2.6300)},{\sy*(-0.0000)})
	--({\sx*(2.6400)},{\sy*(-0.0000)})
	--({\sx*(2.6500)},{\sy*(-0.0000)})
	--({\sx*(2.6600)},{\sy*(-0.0000)})
	--({\sx*(2.6700)},{\sy*(-0.0000)})
	--({\sx*(2.6800)},{\sy*(-0.0000)})
	--({\sx*(2.6900)},{\sy*(-0.0000)})
	--({\sx*(2.7000)},{\sy*(-0.0000)})
	--({\sx*(2.7100)},{\sy*(-0.0000)})
	--({\sx*(2.7200)},{\sy*(-0.0000)})
	--({\sx*(2.7300)},{\sy*(-0.0000)})
	--({\sx*(2.7400)},{\sy*(-0.0000)})
	--({\sx*(2.7500)},{\sy*(-0.0000)})
	--({\sx*(2.7600)},{\sy*(-0.0000)})
	--({\sx*(2.7700)},{\sy*(-0.0000)})
	--({\sx*(2.7800)},{\sy*(-0.0000)})
	--({\sx*(2.7900)},{\sy*(-0.0000)})
	--({\sx*(2.8000)},{\sy*(-0.0000)})
	--({\sx*(2.8100)},{\sy*(-0.0000)})
	--({\sx*(2.8200)},{\sy*(-0.0000)})
	--({\sx*(2.8300)},{\sy*(-0.0000)})
	--({\sx*(2.8400)},{\sy*(-0.0000)})
	--({\sx*(2.8500)},{\sy*(-0.0000)})
	--({\sx*(2.8600)},{\sy*(-0.0000)})
	--({\sx*(2.8700)},{\sy*(-0.0000)})
	--({\sx*(2.8800)},{\sy*(-0.0000)})
	--({\sx*(2.8900)},{\sy*(-0.0000)})
	--({\sx*(2.9000)},{\sy*(-0.0000)})
	--({\sx*(2.9100)},{\sy*(-0.0000)})
	--({\sx*(2.9200)},{\sy*(-0.0000)})
	--({\sx*(2.9300)},{\sy*(-0.0000)})
	--({\sx*(2.9400)},{\sy*(0.0000)})
	--({\sx*(2.9500)},{\sy*(0.0000)})
	--({\sx*(2.9600)},{\sy*(0.0000)})
	--({\sx*(2.9700)},{\sy*(0.0000)})
	--({\sx*(2.9800)},{\sy*(0.0000)})
	--({\sx*(2.9900)},{\sy*(0.0000)})
	--({\sx*(3.0000)},{\sy*(0.0000)})
	--({\sx*(3.0100)},{\sy*(0.0000)})
	--({\sx*(3.0200)},{\sy*(0.0000)})
	--({\sx*(3.0300)},{\sy*(0.0000)})
	--({\sx*(3.0400)},{\sy*(0.0000)})
	--({\sx*(3.0500)},{\sy*(0.0000)})
	--({\sx*(3.0600)},{\sy*(0.0000)})
	--({\sx*(3.0700)},{\sy*(0.0000)})
	--({\sx*(3.0800)},{\sy*(0.0000)})
	--({\sx*(3.0900)},{\sy*(0.0000)})
	--({\sx*(3.1000)},{\sy*(0.0000)})
	--({\sx*(3.1100)},{\sy*(0.0000)})
	--({\sx*(3.1200)},{\sy*(0.0000)})
	--({\sx*(3.1300)},{\sy*(0.0000)})
	--({\sx*(3.1400)},{\sy*(0.0000)})
	--({\sx*(3.1500)},{\sy*(0.0000)})
	--({\sx*(3.1600)},{\sy*(0.0000)})
	--({\sx*(3.1700)},{\sy*(0.0000)})
	--({\sx*(3.1800)},{\sy*(0.0000)})
	--({\sx*(3.1900)},{\sy*(0.0000)})
	--({\sx*(3.2000)},{\sy*(0.0000)})
	--({\sx*(3.2100)},{\sy*(0.0000)})
	--({\sx*(3.2200)},{\sy*(0.0000)})
	--({\sx*(3.2300)},{\sy*(0.0000)})
	--({\sx*(3.2400)},{\sy*(0.0000)})
	--({\sx*(3.2500)},{\sy*(0.0000)})
	--({\sx*(3.2600)},{\sy*(0.0000)})
	--({\sx*(3.2700)},{\sy*(0.0000)})
	--({\sx*(3.2800)},{\sy*(0.0000)})
	--({\sx*(3.2900)},{\sy*(0.0000)})
	--({\sx*(3.3000)},{\sy*(0.0000)})
	--({\sx*(3.3100)},{\sy*(0.0000)})
	--({\sx*(3.3200)},{\sy*(0.0000)})
	--({\sx*(3.3300)},{\sy*(0.0000)})
	--({\sx*(3.3400)},{\sy*(0.0000)})
	--({\sx*(3.3500)},{\sy*(0.0000)})
	--({\sx*(3.3600)},{\sy*(-0.0000)})
	--({\sx*(3.3700)},{\sy*(-0.0000)})
	--({\sx*(3.3800)},{\sy*(-0.0000)})
	--({\sx*(3.3900)},{\sy*(-0.0000)})
	--({\sx*(3.4000)},{\sy*(-0.0000)})
	--({\sx*(3.4100)},{\sy*(-0.0000)})
	--({\sx*(3.4200)},{\sy*(-0.0000)})
	--({\sx*(3.4300)},{\sy*(-0.0000)})
	--({\sx*(3.4400)},{\sy*(-0.0000)})
	--({\sx*(3.4500)},{\sy*(-0.0000)})
	--({\sx*(3.4600)},{\sy*(-0.0000)})
	--({\sx*(3.4700)},{\sy*(-0.0000)})
	--({\sx*(3.4800)},{\sy*(-0.0000)})
	--({\sx*(3.4900)},{\sy*(-0.0000)})
	--({\sx*(3.5000)},{\sy*(-0.0000)})
	--({\sx*(3.5100)},{\sy*(-0.0000)})
	--({\sx*(3.5200)},{\sy*(-0.0000)})
	--({\sx*(3.5300)},{\sy*(-0.0000)})
	--({\sx*(3.5400)},{\sy*(-0.0000)})
	--({\sx*(3.5500)},{\sy*(-0.0000)})
	--({\sx*(3.5600)},{\sy*(-0.0000)})
	--({\sx*(3.5700)},{\sy*(-0.0000)})
	--({\sx*(3.5800)},{\sy*(-0.0000)})
	--({\sx*(3.5900)},{\sy*(-0.0000)})
	--({\sx*(3.6000)},{\sy*(-0.0000)})
	--({\sx*(3.6100)},{\sy*(-0.0000)})
	--({\sx*(3.6200)},{\sy*(-0.0000)})
	--({\sx*(3.6300)},{\sy*(-0.0000)})
	--({\sx*(3.6400)},{\sy*(-0.0000)})
	--({\sx*(3.6500)},{\sy*(-0.0000)})
	--({\sx*(3.6600)},{\sy*(-0.0000)})
	--({\sx*(3.6700)},{\sy*(-0.0000)})
	--({\sx*(3.6800)},{\sy*(-0.0000)})
	--({\sx*(3.6900)},{\sy*(-0.0000)})
	--({\sx*(3.7000)},{\sy*(-0.0000)})
	--({\sx*(3.7100)},{\sy*(-0.0000)})
	--({\sx*(3.7200)},{\sy*(-0.0000)})
	--({\sx*(3.7300)},{\sy*(-0.0000)})
	--({\sx*(3.7400)},{\sy*(-0.0000)})
	--({\sx*(3.7500)},{\sy*(0.0000)})
	--({\sx*(3.7600)},{\sy*(0.0000)})
	--({\sx*(3.7700)},{\sy*(0.0000)})
	--({\sx*(3.7800)},{\sy*(0.0000)})
	--({\sx*(3.7900)},{\sy*(0.0000)})
	--({\sx*(3.8000)},{\sy*(0.0000)})
	--({\sx*(3.8100)},{\sy*(0.0000)})
	--({\sx*(3.8200)},{\sy*(0.0000)})
	--({\sx*(3.8300)},{\sy*(0.0000)})
	--({\sx*(3.8400)},{\sy*(0.0000)})
	--({\sx*(3.8500)},{\sy*(0.0000)})
	--({\sx*(3.8600)},{\sy*(0.0000)})
	--({\sx*(3.8700)},{\sy*(0.0000)})
	--({\sx*(3.8800)},{\sy*(0.0000)})
	--({\sx*(3.8900)},{\sy*(0.0000)})
	--({\sx*(3.9000)},{\sy*(0.0000)})
	--({\sx*(3.9100)},{\sy*(0.0000)})
	--({\sx*(3.9200)},{\sy*(0.0000)})
	--({\sx*(3.9300)},{\sy*(0.0000)})
	--({\sx*(3.9400)},{\sy*(0.0000)})
	--({\sx*(3.9500)},{\sy*(0.0000)})
	--({\sx*(3.9600)},{\sy*(0.0000)})
	--({\sx*(3.9700)},{\sy*(0.0000)})
	--({\sx*(3.9800)},{\sy*(0.0000)})
	--({\sx*(3.9900)},{\sy*(0.0000)})
	--({\sx*(4.0000)},{\sy*(0.0000)})
	--({\sx*(4.0100)},{\sy*(0.0000)})
	--({\sx*(4.0200)},{\sy*(0.0000)})
	--({\sx*(4.0300)},{\sy*(0.0000)})
	--({\sx*(4.0400)},{\sy*(0.0000)})
	--({\sx*(4.0500)},{\sy*(0.0000)})
	--({\sx*(4.0600)},{\sy*(0.0000)})
	--({\sx*(4.0700)},{\sy*(0.0000)})
	--({\sx*(4.0800)},{\sy*(0.0000)})
	--({\sx*(4.0900)},{\sy*(0.0000)})
	--({\sx*(4.1000)},{\sy*(0.0000)})
	--({\sx*(4.1100)},{\sy*(-0.0000)})
	--({\sx*(4.1200)},{\sy*(-0.0000)})
	--({\sx*(4.1300)},{\sy*(-0.0000)})
	--({\sx*(4.1400)},{\sy*(-0.0000)})
	--({\sx*(4.1500)},{\sy*(-0.0000)})
	--({\sx*(4.1600)},{\sy*(-0.0000)})
	--({\sx*(4.1700)},{\sy*(-0.0000)})
	--({\sx*(4.1800)},{\sy*(-0.0000)})
	--({\sx*(4.1900)},{\sy*(-0.0001)})
	--({\sx*(4.2000)},{\sy*(-0.0001)})
	--({\sx*(4.2100)},{\sy*(-0.0001)})
	--({\sx*(4.2200)},{\sy*(-0.0001)})
	--({\sx*(4.2300)},{\sy*(-0.0001)})
	--({\sx*(4.2400)},{\sy*(-0.0001)})
	--({\sx*(4.2500)},{\sy*(-0.0001)})
	--({\sx*(4.2600)},{\sy*(-0.0001)})
	--({\sx*(4.2700)},{\sy*(-0.0001)})
	--({\sx*(4.2800)},{\sy*(-0.0001)})
	--({\sx*(4.2900)},{\sy*(-0.0001)})
	--({\sx*(4.3000)},{\sy*(-0.0001)})
	--({\sx*(4.3100)},{\sy*(-0.0001)})
	--({\sx*(4.3200)},{\sy*(-0.0001)})
	--({\sx*(4.3300)},{\sy*(-0.0001)})
	--({\sx*(4.3400)},{\sy*(-0.0001)})
	--({\sx*(4.3500)},{\sy*(-0.0001)})
	--({\sx*(4.3600)},{\sy*(-0.0001)})
	--({\sx*(4.3700)},{\sy*(-0.0001)})
	--({\sx*(4.3800)},{\sy*(-0.0001)})
	--({\sx*(4.3900)},{\sy*(-0.0000)})
	--({\sx*(4.4000)},{\sy*(-0.0000)})
	--({\sx*(4.4100)},{\sy*(-0.0000)})
	--({\sx*(4.4200)},{\sy*(0.0000)})
	--({\sx*(4.4300)},{\sy*(0.0000)})
	--({\sx*(4.4400)},{\sy*(0.0001)})
	--({\sx*(4.4500)},{\sy*(0.0001)})
	--({\sx*(4.4600)},{\sy*(0.0001)})
	--({\sx*(4.4700)},{\sy*(0.0001)})
	--({\sx*(4.4800)},{\sy*(0.0001)})
	--({\sx*(4.4900)},{\sy*(0.0002)})
	--({\sx*(4.5000)},{\sy*(0.0002)})
	--({\sx*(4.5100)},{\sy*(0.0002)})
	--({\sx*(4.5200)},{\sy*(0.0002)})
	--({\sx*(4.5300)},{\sy*(0.0003)})
	--({\sx*(4.5400)},{\sy*(0.0003)})
	--({\sx*(4.5500)},{\sy*(0.0003)})
	--({\sx*(4.5600)},{\sy*(0.0003)})
	--({\sx*(4.5700)},{\sy*(0.0003)})
	--({\sx*(4.5800)},{\sy*(0.0003)})
	--({\sx*(4.5900)},{\sy*(0.0003)})
	--({\sx*(4.6000)},{\sy*(0.0003)})
	--({\sx*(4.6100)},{\sy*(0.0002)})
	--({\sx*(4.6200)},{\sy*(0.0002)})
	--({\sx*(4.6300)},{\sy*(0.0002)})
	--({\sx*(4.6400)},{\sy*(0.0001)})
	--({\sx*(4.6500)},{\sy*(0.0001)})
	--({\sx*(4.6600)},{\sy*(0.0000)})
	--({\sx*(4.6700)},{\sy*(-0.0000)})
	--({\sx*(4.6800)},{\sy*(-0.0001)})
	--({\sx*(4.6900)},{\sy*(-0.0002)})
	--({\sx*(4.7000)},{\sy*(-0.0002)})
	--({\sx*(4.7100)},{\sy*(-0.0003)})
	--({\sx*(4.7200)},{\sy*(-0.0004)})
	--({\sx*(4.7300)},{\sy*(-0.0004)})
	--({\sx*(4.7400)},{\sy*(-0.0005)})
	--({\sx*(4.7500)},{\sy*(-0.0005)})
	--({\sx*(4.7600)},{\sy*(-0.0005)})
	--({\sx*(4.7700)},{\sy*(-0.0006)})
	--({\sx*(4.7800)},{\sy*(-0.0006)})
	--({\sx*(4.7900)},{\sy*(-0.0005)})
	--({\sx*(4.8000)},{\sy*(-0.0005)})
	--({\sx*(4.8100)},{\sy*(-0.0004)})
	--({\sx*(4.8200)},{\sy*(-0.0004)})
	--({\sx*(4.8300)},{\sy*(-0.0002)})
	--({\sx*(4.8400)},{\sy*(-0.0001)})
	--({\sx*(4.8500)},{\sy*(0.0000)})
	--({\sx*(4.8600)},{\sy*(0.0002)})
	--({\sx*(4.8700)},{\sy*(0.0003)})
	--({\sx*(4.8800)},{\sy*(0.0004)})
	--({\sx*(4.8900)},{\sy*(0.0006)})
	--({\sx*(4.9000)},{\sy*(0.0006)})
	--({\sx*(4.9100)},{\sy*(0.0007)})
	--({\sx*(4.9200)},{\sy*(0.0007)})
	--({\sx*(4.9300)},{\sy*(0.0006)})
	--({\sx*(4.9400)},{\sy*(0.0005)})
	--({\sx*(4.9500)},{\sy*(0.0003)})
	--({\sx*(4.9600)},{\sy*(0.0000)})
	--({\sx*(4.9700)},{\sy*(-0.0002)})
	--({\sx*(4.9800)},{\sy*(-0.0004)})
	--({\sx*(4.9900)},{\sy*(-0.0003)})
	--({\sx*(5.0000)},{\sy*(0.0000)});
}
\def\xwertej{
\fill[color=red] (0.0000,0) circle[radius={0.07/\skala}];
\fill[color=white] (0.0000,0) circle[radius={0.05/\skala}];
\fill[color=red] (0.0308,0) circle[radius={0.07/\skala}];
\fill[color=white] (0.0308,0) circle[radius={0.05/\skala}];
\fill[color=red] (0.1224,0) circle[radius={0.07/\skala}];
\fill[color=white] (0.1224,0) circle[radius={0.05/\skala}];
\fill[color=red] (0.2725,0) circle[radius={0.07/\skala}];
\fill[color=white] (0.2725,0) circle[radius={0.05/\skala}];
\fill[color=red] (0.4775,0) circle[radius={0.07/\skala}];
\fill[color=white] (0.4775,0) circle[radius={0.05/\skala}];
\fill[color=red] (0.7322,0) circle[radius={0.07/\skala}];
\fill[color=white] (0.7322,0) circle[radius={0.05/\skala}];
\fill[color=red] (1.0305,0) circle[radius={0.07/\skala}];
\fill[color=white] (1.0305,0) circle[radius={0.05/\skala}];
\fill[color=red] (1.3650,0) circle[radius={0.07/\skala}];
\fill[color=white] (1.3650,0) circle[radius={0.05/\skala}];
\fill[color=red] (1.7275,0) circle[radius={0.07/\skala}];
\fill[color=white] (1.7275,0) circle[radius={0.05/\skala}];
\fill[color=red] (2.1089,0) circle[radius={0.07/\skala}];
\fill[color=white] (2.1089,0) circle[radius={0.05/\skala}];
\fill[color=red] (2.5000,0) circle[radius={0.07/\skala}];
\fill[color=white] (2.5000,0) circle[radius={0.05/\skala}];
\fill[color=red] (2.8911,0) circle[radius={0.07/\skala}];
\fill[color=white] (2.8911,0) circle[radius={0.05/\skala}];
\fill[color=red] (3.2725,0) circle[radius={0.07/\skala}];
\fill[color=white] (3.2725,0) circle[radius={0.05/\skala}];
\fill[color=red] (3.6350,0) circle[radius={0.07/\skala}];
\fill[color=white] (3.6350,0) circle[radius={0.05/\skala}];
\fill[color=red] (3.9695,0) circle[radius={0.07/\skala}];
\fill[color=white] (3.9695,0) circle[radius={0.05/\skala}];
\fill[color=red] (4.2678,0) circle[radius={0.07/\skala}];
\fill[color=white] (4.2678,0) circle[radius={0.05/\skala}];
\fill[color=red] (4.5225,0) circle[radius={0.07/\skala}];
\fill[color=white] (4.5225,0) circle[radius={0.05/\skala}];
\fill[color=red] (4.7275,0) circle[radius={0.07/\skala}];
\fill[color=white] (4.7275,0) circle[radius={0.05/\skala}];
\fill[color=red] (4.8776,0) circle[radius={0.07/\skala}];
\fill[color=white] (4.8776,0) circle[radius={0.05/\skala}];
\fill[color=red] (4.9692,0) circle[radius={0.07/\skala}];
\fill[color=white] (4.9692,0) circle[radius={0.05/\skala}];
\fill[color=red] (5.0000,0) circle[radius={0.07/\skala}];
\fill[color=white] (5.0000,0) circle[radius={0.05/\skala}];
}
\def\punktej{20}
\def\maxfehlerj{5.983\cdot 10^{-10}}
\def\fehlerj{
\draw[color=red,line width=1.4pt,line join=round] ({\sx*(0.000)},{\sy*(0.0000)})
	--({\sx*(0.0100)},{\sy*(-0.0633)})
	--({\sx*(0.0200)},{\sy*(-0.0526)})
	--({\sx*(0.0300)},{\sy*(-0.0045)})
	--({\sx*(0.0400)},{\sy*(0.0556)})
	--({\sx*(0.0500)},{\sy*(0.1111)})
	--({\sx*(0.0600)},{\sy*(0.1520)})
	--({\sx*(0.0700)},{\sy*(0.1733)})
	--({\sx*(0.0800)},{\sy*(0.1737)})
	--({\sx*(0.0900)},{\sy*(0.1546)})
	--({\sx*(0.1000)},{\sy*(0.1189)})
	--({\sx*(0.1100)},{\sy*(0.0706)})
	--({\sx*(0.1200)},{\sy*(0.0140)})
	--({\sx*(0.1300)},{\sy*(-0.0462)})
	--({\sx*(0.1400)},{\sy*(-0.1061)})
	--({\sx*(0.1500)},{\sy*(-0.1618)})
	--({\sx*(0.1600)},{\sy*(-0.2103)})
	--({\sx*(0.1700)},{\sy*(-0.2491)})
	--({\sx*(0.1800)},{\sy*(-0.2765)})
	--({\sx*(0.1900)},{\sy*(-0.2915)})
	--({\sx*(0.2000)},{\sy*(-0.2935)})
	--({\sx*(0.2100)},{\sy*(-0.2829)})
	--({\sx*(0.2200)},{\sy*(-0.2602)})
	--({\sx*(0.2300)},{\sy*(-0.2266)})
	--({\sx*(0.2400)},{\sy*(-0.1836)})
	--({\sx*(0.2500)},{\sy*(-0.1327)})
	--({\sx*(0.2600)},{\sy*(-0.0760)})
	--({\sx*(0.2700)},{\sy*(-0.0154)})
	--({\sx*(0.2800)},{\sy*(0.0471)})
	--({\sx*(0.2900)},{\sy*(0.1095)})
	--({\sx*(0.3000)},{\sy*(0.1700)})
	--({\sx*(0.3100)},{\sy*(0.2267)})
	--({\sx*(0.3200)},{\sy*(0.2782)})
	--({\sx*(0.3300)},{\sy*(0.3231)})
	--({\sx*(0.3400)},{\sy*(0.3603)})
	--({\sx*(0.3500)},{\sy*(0.3888)})
	--({\sx*(0.3600)},{\sy*(0.4080)})
	--({\sx*(0.3700)},{\sy*(0.4175)})
	--({\sx*(0.3800)},{\sy*(0.4172)})
	--({\sx*(0.3900)},{\sy*(0.4072)})
	--({\sx*(0.4000)},{\sy*(0.3877)})
	--({\sx*(0.4100)},{\sy*(0.3593)})
	--({\sx*(0.4200)},{\sy*(0.3227)})
	--({\sx*(0.4300)},{\sy*(0.2786)})
	--({\sx*(0.4400)},{\sy*(0.2280)})
	--({\sx*(0.4500)},{\sy*(0.1720)})
	--({\sx*(0.4600)},{\sy*(0.1118)})
	--({\sx*(0.4700)},{\sy*(0.0485)})
	--({\sx*(0.4800)},{\sy*(-0.0167)})
	--({\sx*(0.4900)},{\sy*(-0.0824)})
	--({\sx*(0.5000)},{\sy*(-0.1475)})
	--({\sx*(0.5100)},{\sy*(-0.2108)})
	--({\sx*(0.5200)},{\sy*(-0.2712)})
	--({\sx*(0.5300)},{\sy*(-0.3277)})
	--({\sx*(0.5400)},{\sy*(-0.3792)})
	--({\sx*(0.5500)},{\sy*(-0.4250)})
	--({\sx*(0.5600)},{\sy*(-0.4643)})
	--({\sx*(0.5700)},{\sy*(-0.4965)})
	--({\sx*(0.5800)},{\sy*(-0.5211)})
	--({\sx*(0.5900)},{\sy*(-0.5377)})
	--({\sx*(0.6000)},{\sy*(-0.5460)})
	--({\sx*(0.6100)},{\sy*(-0.5461)})
	--({\sx*(0.6200)},{\sy*(-0.5379)})
	--({\sx*(0.6300)},{\sy*(-0.5216)})
	--({\sx*(0.6400)},{\sy*(-0.4974)})
	--({\sx*(0.6500)},{\sy*(-0.4656)})
	--({\sx*(0.6600)},{\sy*(-0.4269)})
	--({\sx*(0.6700)},{\sy*(-0.3817)})
	--({\sx*(0.6800)},{\sy*(-0.3307)})
	--({\sx*(0.6900)},{\sy*(-0.2746)})
	--({\sx*(0.7000)},{\sy*(-0.2141)})
	--({\sx*(0.7100)},{\sy*(-0.1502)})
	--({\sx*(0.7200)},{\sy*(-0.0837)})
	--({\sx*(0.7300)},{\sy*(-0.0154)})
	--({\sx*(0.7400)},{\sy*(0.0538)})
	--({\sx*(0.7500)},{\sy*(0.1229)})
	--({\sx*(0.7600)},{\sy*(0.1911)})
	--({\sx*(0.7700)},{\sy*(0.2575)})
	--({\sx*(0.7800)},{\sy*(0.3213)})
	--({\sx*(0.7900)},{\sy*(0.3818)})
	--({\sx*(0.8000)},{\sy*(0.4381)})
	--({\sx*(0.8100)},{\sy*(0.4897)})
	--({\sx*(0.8200)},{\sy*(0.5359)})
	--({\sx*(0.8300)},{\sy*(0.5761)})
	--({\sx*(0.8400)},{\sy*(0.6099)})
	--({\sx*(0.8500)},{\sy*(0.6370)})
	--({\sx*(0.8600)},{\sy*(0.6570)})
	--({\sx*(0.8700)},{\sy*(0.6696)})
	--({\sx*(0.8800)},{\sy*(0.6749)})
	--({\sx*(0.8900)},{\sy*(0.6726)})
	--({\sx*(0.9000)},{\sy*(0.6629)})
	--({\sx*(0.9100)},{\sy*(0.6459)})
	--({\sx*(0.9200)},{\sy*(0.6217)})
	--({\sx*(0.9300)},{\sy*(0.5906)})
	--({\sx*(0.9400)},{\sy*(0.5529)})
	--({\sx*(0.9500)},{\sy*(0.5092)})
	--({\sx*(0.9600)},{\sy*(0.4597)})
	--({\sx*(0.9700)},{\sy*(0.4052)})
	--({\sx*(0.9800)},{\sy*(0.3460)})
	--({\sx*(0.9900)},{\sy*(0.2828)})
	--({\sx*(1.0000)},{\sy*(0.2164)})
	--({\sx*(1.0100)},{\sy*(0.1473)})
	--({\sx*(1.0200)},{\sy*(0.0762)})
	--({\sx*(1.0300)},{\sy*(0.0039)})
	--({\sx*(1.0400)},{\sy*(-0.0689)})
	--({\sx*(1.0500)},{\sy*(-0.1416)})
	--({\sx*(1.0600)},{\sy*(-0.2133)})
	--({\sx*(1.0700)},{\sy*(-0.2834)})
	--({\sx*(1.0800)},{\sy*(-0.3513)})
	--({\sx*(1.0900)},{\sy*(-0.4163)})
	--({\sx*(1.1000)},{\sy*(-0.4777)})
	--({\sx*(1.1100)},{\sy*(-0.5350)})
	--({\sx*(1.1200)},{\sy*(-0.5876)})
	--({\sx*(1.1300)},{\sy*(-0.6351)})
	--({\sx*(1.1400)},{\sy*(-0.6771)})
	--({\sx*(1.1500)},{\sy*(-0.7131)})
	--({\sx*(1.1600)},{\sy*(-0.7427)})
	--({\sx*(1.1700)},{\sy*(-0.7659)})
	--({\sx*(1.1800)},{\sy*(-0.7822)})
	--({\sx*(1.1900)},{\sy*(-0.7917)})
	--({\sx*(1.2000)},{\sy*(-0.7942)})
	--({\sx*(1.2100)},{\sy*(-0.7896)})
	--({\sx*(1.2200)},{\sy*(-0.7781)})
	--({\sx*(1.2300)},{\sy*(-0.7597)})
	--({\sx*(1.2400)},{\sy*(-0.7346)})
	--({\sx*(1.2500)},{\sy*(-0.7031)})
	--({\sx*(1.2600)},{\sy*(-0.6653)})
	--({\sx*(1.2700)},{\sy*(-0.6216)})
	--({\sx*(1.2800)},{\sy*(-0.5724)})
	--({\sx*(1.2900)},{\sy*(-0.5181)})
	--({\sx*(1.3000)},{\sy*(-0.4592)})
	--({\sx*(1.3100)},{\sy*(-0.3962)})
	--({\sx*(1.3200)},{\sy*(-0.3295)})
	--({\sx*(1.3300)},{\sy*(-0.2598)})
	--({\sx*(1.3400)},{\sy*(-0.1876)})
	--({\sx*(1.3500)},{\sy*(-0.1135)})
	--({\sx*(1.3600)},{\sy*(-0.0381)})
	--({\sx*(1.3700)},{\sy*(0.0379)})
	--({\sx*(1.3800)},{\sy*(0.1139)})
	--({\sx*(1.3900)},{\sy*(0.1894)})
	--({\sx*(1.4000)},{\sy*(0.2637)})
	--({\sx*(1.4100)},{\sy*(0.3361)})
	--({\sx*(1.4200)},{\sy*(0.4063)})
	--({\sx*(1.4300)},{\sy*(0.4735)})
	--({\sx*(1.4400)},{\sy*(0.5372)})
	--({\sx*(1.4500)},{\sy*(0.5971)})
	--({\sx*(1.4600)},{\sy*(0.6524)})
	--({\sx*(1.4700)},{\sy*(0.7029)})
	--({\sx*(1.4800)},{\sy*(0.7482)})
	--({\sx*(1.4900)},{\sy*(0.7879)})
	--({\sx*(1.5000)},{\sy*(0.8216)})
	--({\sx*(1.5100)},{\sy*(0.8492)})
	--({\sx*(1.5200)},{\sy*(0.8704)})
	--({\sx*(1.5300)},{\sy*(0.8850)})
	--({\sx*(1.5400)},{\sy*(0.8930)})
	--({\sx*(1.5500)},{\sy*(0.8942)})
	--({\sx*(1.5600)},{\sy*(0.8888)})
	--({\sx*(1.5700)},{\sy*(0.8766)})
	--({\sx*(1.5800)},{\sy*(0.8579)})
	--({\sx*(1.5900)},{\sy*(0.8327)})
	--({\sx*(1.6000)},{\sy*(0.8012)})
	--({\sx*(1.6100)},{\sy*(0.7637)})
	--({\sx*(1.6200)},{\sy*(0.7205)})
	--({\sx*(1.6300)},{\sy*(0.6718)})
	--({\sx*(1.6400)},{\sy*(0.6180)})
	--({\sx*(1.6500)},{\sy*(0.5596)})
	--({\sx*(1.6600)},{\sy*(0.4969)})
	--({\sx*(1.6700)},{\sy*(0.4304)})
	--({\sx*(1.6800)},{\sy*(0.3605)})
	--({\sx*(1.6900)},{\sy*(0.2879)})
	--({\sx*(1.7000)},{\sy*(0.2130)})
	--({\sx*(1.7100)},{\sy*(0.1363)})
	--({\sx*(1.7200)},{\sy*(0.0585)})
	--({\sx*(1.7300)},{\sy*(-0.0200)})
	--({\sx*(1.7400)},{\sy*(-0.0985)})
	--({\sx*(1.7500)},{\sy*(-0.1766)})
	--({\sx*(1.7600)},{\sy*(-0.2536)})
	--({\sx*(1.7700)},{\sy*(-0.3291)})
	--({\sx*(1.7800)},{\sy*(-0.4024)})
	--({\sx*(1.7900)},{\sy*(-0.4731)})
	--({\sx*(1.8000)},{\sy*(-0.5406)})
	--({\sx*(1.8100)},{\sy*(-0.6046)})
	--({\sx*(1.8200)},{\sy*(-0.6644)})
	--({\sx*(1.8300)},{\sy*(-0.7198)})
	--({\sx*(1.8400)},{\sy*(-0.7704)})
	--({\sx*(1.8500)},{\sy*(-0.8157)})
	--({\sx*(1.8600)},{\sy*(-0.8555)})
	--({\sx*(1.8700)},{\sy*(-0.8895)})
	--({\sx*(1.8800)},{\sy*(-0.9175)})
	--({\sx*(1.8900)},{\sy*(-0.9392)})
	--({\sx*(1.9000)},{\sy*(-0.9546)})
	--({\sx*(1.9100)},{\sy*(-0.9634)})
	--({\sx*(1.9200)},{\sy*(-0.9658)})
	--({\sx*(1.9300)},{\sy*(-0.9615)})
	--({\sx*(1.9400)},{\sy*(-0.9508)})
	--({\sx*(1.9500)},{\sy*(-0.9335)})
	--({\sx*(1.9600)},{\sy*(-0.9100)})
	--({\sx*(1.9700)},{\sy*(-0.8802)})
	--({\sx*(1.9800)},{\sy*(-0.8445)})
	--({\sx*(1.9900)},{\sy*(-0.8031)})
	--({\sx*(2.0000)},{\sy*(-0.7561)})
	--({\sx*(2.0100)},{\sy*(-0.7041)})
	--({\sx*(2.0200)},{\sy*(-0.6473)})
	--({\sx*(2.0300)},{\sy*(-0.5860)})
	--({\sx*(2.0400)},{\sy*(-0.5208)})
	--({\sx*(2.0500)},{\sy*(-0.4519)})
	--({\sx*(2.0600)},{\sy*(-0.3800)})
	--({\sx*(2.0700)},{\sy*(-0.3055)})
	--({\sx*(2.0800)},{\sy*(-0.2289)})
	--({\sx*(2.0900)},{\sy*(-0.1506)})
	--({\sx*(2.1000)},{\sy*(-0.0712)})
	--({\sx*(2.1100)},{\sy*(0.0087)})
	--({\sx*(2.1200)},{\sy*(0.0887)})
	--({\sx*(2.1300)},{\sy*(0.1681)})
	--({\sx*(2.1400)},{\sy*(0.2466)})
	--({\sx*(2.1500)},{\sy*(0.3235)})
	--({\sx*(2.1600)},{\sy*(0.3984)})
	--({\sx*(2.1700)},{\sy*(0.4707)})
	--({\sx*(2.1800)},{\sy*(0.5401)})
	--({\sx*(2.1900)},{\sy*(0.6059)})
	--({\sx*(2.2000)},{\sy*(0.6679)})
	--({\sx*(2.2100)},{\sy*(0.7256)})
	--({\sx*(2.2200)},{\sy*(0.7786)})
	--({\sx*(2.2300)},{\sy*(0.8266)})
	--({\sx*(2.2400)},{\sy*(0.8692)})
	--({\sx*(2.2500)},{\sy*(0.9062)})
	--({\sx*(2.2600)},{\sy*(0.9374)})
	--({\sx*(2.2700)},{\sy*(0.9625)})
	--({\sx*(2.2800)},{\sy*(0.9814)})
	--({\sx*(2.2900)},{\sy*(0.9939)})
	--({\sx*(2.3000)},{\sy*(1.0000)})
	--({\sx*(2.3100)},{\sy*(0.9996)})
	--({\sx*(2.3200)},{\sy*(0.9928)})
	--({\sx*(2.3300)},{\sy*(0.9796)})
	--({\sx*(2.3400)},{\sy*(0.9601)})
	--({\sx*(2.3500)},{\sy*(0.9343)})
	--({\sx*(2.3600)},{\sy*(0.9025)})
	--({\sx*(2.3700)},{\sy*(0.8649)})
	--({\sx*(2.3800)},{\sy*(0.8218)})
	--({\sx*(2.3900)},{\sy*(0.7733)})
	--({\sx*(2.4000)},{\sy*(0.7198)})
	--({\sx*(2.4100)},{\sy*(0.6617)})
	--({\sx*(2.4200)},{\sy*(0.5993)})
	--({\sx*(2.4300)},{\sy*(0.5331)})
	--({\sx*(2.4400)},{\sy*(0.4635)})
	--({\sx*(2.4500)},{\sy*(0.3908)})
	--({\sx*(2.4600)},{\sy*(0.3157)})
	--({\sx*(2.4700)},{\sy*(0.2386)})
	--({\sx*(2.4800)},{\sy*(0.1599)})
	--({\sx*(2.4900)},{\sy*(0.0802)})
	--({\sx*(2.5000)},{\sy*(0.0000)})
	--({\sx*(2.5100)},{\sy*(-0.0802)})
	--({\sx*(2.5200)},{\sy*(-0.1598)})
	--({\sx*(2.5300)},{\sy*(-0.2384)})
	--({\sx*(2.5400)},{\sy*(-0.3154)})
	--({\sx*(2.5500)},{\sy*(-0.3904)})
	--({\sx*(2.5600)},{\sy*(-0.4629)})
	--({\sx*(2.5700)},{\sy*(-0.5323)})
	--({\sx*(2.5800)},{\sy*(-0.5983)})
	--({\sx*(2.5900)},{\sy*(-0.6604)})
	--({\sx*(2.6000)},{\sy*(-0.7182)})
	--({\sx*(2.6100)},{\sy*(-0.7714)})
	--({\sx*(2.6200)},{\sy*(-0.8196)})
	--({\sx*(2.6300)},{\sy*(-0.8625)})
	--({\sx*(2.6400)},{\sy*(-0.8998)})
	--({\sx*(2.6500)},{\sy*(-0.9313)})
	--({\sx*(2.6600)},{\sy*(-0.9567)})
	--({\sx*(2.6700)},{\sy*(-0.9760)})
	--({\sx*(2.6800)},{\sy*(-0.9890)})
	--({\sx*(2.6900)},{\sy*(-0.9955)})
	--({\sx*(2.7000)},{\sy*(-0.9957)})
	--({\sx*(2.7100)},{\sy*(-0.9894)})
	--({\sx*(2.7200)},{\sy*(-0.9767)})
	--({\sx*(2.7300)},{\sy*(-0.9577)})
	--({\sx*(2.7400)},{\sy*(-0.9326)})
	--({\sx*(2.7500)},{\sy*(-0.9014)})
	--({\sx*(2.7600)},{\sy*(-0.8644)})
	--({\sx*(2.7700)},{\sy*(-0.8218)})
	--({\sx*(2.7800)},{\sy*(-0.7740)})
	--({\sx*(2.7900)},{\sy*(-0.7212)})
	--({\sx*(2.8000)},{\sy*(-0.6637)})
	--({\sx*(2.8100)},{\sy*(-0.6020)})
	--({\sx*(2.8200)},{\sy*(-0.5365)})
	--({\sx*(2.8300)},{\sy*(-0.4675)})
	--({\sx*(2.8400)},{\sy*(-0.3956)})
	--({\sx*(2.8500)},{\sy*(-0.3212)})
	--({\sx*(2.8600)},{\sy*(-0.2448)})
	--({\sx*(2.8700)},{\sy*(-0.1669)})
	--({\sx*(2.8800)},{\sy*(-0.0880)})
	--({\sx*(2.8900)},{\sy*(-0.0086)})
	--({\sx*(2.9000)},{\sy*(0.0707)})
	--({\sx*(2.9100)},{\sy*(0.1494)})
	--({\sx*(2.9200)},{\sy*(0.2269)})
	--({\sx*(2.9300)},{\sy*(0.3029)})
	--({\sx*(2.9400)},{\sy*(0.3767)})
	--({\sx*(2.9500)},{\sy*(0.4479)})
	--({\sx*(2.9600)},{\sy*(0.5161)})
	--({\sx*(2.9700)},{\sy*(0.5806)})
	--({\sx*(2.9800)},{\sy*(0.6412)})
	--({\sx*(2.9900)},{\sy*(0.6974)})
	--({\sx*(3.0000)},{\sy*(0.7489)})
	--({\sx*(3.0100)},{\sy*(0.7952)})
	--({\sx*(3.0200)},{\sy*(0.8362)})
	--({\sx*(3.0300)},{\sy*(0.8714)})
	--({\sx*(3.0400)},{\sy*(0.9008)})
	--({\sx*(3.0500)},{\sy*(0.9240)})
	--({\sx*(3.0600)},{\sy*(0.9409)})
	--({\sx*(3.0700)},{\sy*(0.9515)})
	--({\sx*(3.0800)},{\sy*(0.9555)})
	--({\sx*(3.0900)},{\sy*(0.9531)})
	--({\sx*(3.1000)},{\sy*(0.9443)})
	--({\sx*(3.1100)},{\sy*(0.9290)})
	--({\sx*(3.1200)},{\sy*(0.9074)})
	--({\sx*(3.1300)},{\sy*(0.8797)})
	--({\sx*(3.1400)},{\sy*(0.8460)})
	--({\sx*(3.1500)},{\sy*(0.8066)})
	--({\sx*(3.1600)},{\sy*(0.7617)})
	--({\sx*(3.1700)},{\sy*(0.7116)})
	--({\sx*(3.1800)},{\sy*(0.6568)})
	--({\sx*(3.1900)},{\sy*(0.5976)})
	--({\sx*(3.2000)},{\sy*(0.5344)})
	--({\sx*(3.2100)},{\sy*(0.4676)})
	--({\sx*(3.2200)},{\sy*(0.3977)})
	--({\sx*(3.2300)},{\sy*(0.3252)})
	--({\sx*(3.2400)},{\sy*(0.2507)})
	--({\sx*(3.2500)},{\sy*(0.1745)})
	--({\sx*(3.2600)},{\sy*(0.0974)})
	--({\sx*(3.2700)},{\sy*(0.0197)})
	--({\sx*(3.2800)},{\sy*(-0.0578)})
	--({\sx*(3.2900)},{\sy*(-0.1347)})
	--({\sx*(3.3000)},{\sy*(-0.2105)})
	--({\sx*(3.3100)},{\sy*(-0.2845)})
	--({\sx*(3.3200)},{\sy*(-0.3562)})
	--({\sx*(3.3300)},{\sy*(-0.4252)})
	--({\sx*(3.3400)},{\sy*(-0.4910)})
	--({\sx*(3.3500)},{\sy*(-0.5529)})
	--({\sx*(3.3600)},{\sy*(-0.6107)})
	--({\sx*(3.3700)},{\sy*(-0.6638)})
	--({\sx*(3.3800)},{\sy*(-0.7120)})
	--({\sx*(3.3900)},{\sy*(-0.7548)})
	--({\sx*(3.4000)},{\sy*(-0.7918)})
	--({\sx*(3.4100)},{\sy*(-0.8230)})
	--({\sx*(3.4200)},{\sy*(-0.8479)})
	--({\sx*(3.4300)},{\sy*(-0.8665)})
	--({\sx*(3.4400)},{\sy*(-0.8786)})
	--({\sx*(3.4500)},{\sy*(-0.8840)})
	--({\sx*(3.4600)},{\sy*(-0.8829)})
	--({\sx*(3.4700)},{\sy*(-0.8750)})
	--({\sx*(3.4800)},{\sy*(-0.8607)})
	--({\sx*(3.4900)},{\sy*(-0.8398)})
	--({\sx*(3.5000)},{\sy*(-0.8126)})
	--({\sx*(3.5100)},{\sy*(-0.7793)})
	--({\sx*(3.5200)},{\sy*(-0.7402)})
	--({\sx*(3.5300)},{\sy*(-0.6955)})
	--({\sx*(3.5400)},{\sy*(-0.6456)})
	--({\sx*(3.5500)},{\sy*(-0.5909)})
	--({\sx*(3.5600)},{\sy*(-0.5318)})
	--({\sx*(3.5700)},{\sy*(-0.4687)})
	--({\sx*(3.5800)},{\sy*(-0.4023)})
	--({\sx*(3.5900)},{\sy*(-0.3329)})
	--({\sx*(3.6000)},{\sy*(-0.2611)})
	--({\sx*(3.6100)},{\sy*(-0.1876)})
	--({\sx*(3.6200)},{\sy*(-0.1129)})
	--({\sx*(3.6300)},{\sy*(-0.0375)})
	--({\sx*(3.6400)},{\sy*(0.0378)})
	--({\sx*(3.6500)},{\sy*(0.1125)})
	--({\sx*(3.6600)},{\sy*(0.1860)})
	--({\sx*(3.6700)},{\sy*(0.2576)})
	--({\sx*(3.6800)},{\sy*(0.3269)})
	--({\sx*(3.6900)},{\sy*(0.3931)})
	--({\sx*(3.7000)},{\sy*(0.4558)})
	--({\sx*(3.7100)},{\sy*(0.5144)})
	--({\sx*(3.7200)},{\sy*(0.5684)})
	--({\sx*(3.7300)},{\sy*(0.6174)})
	--({\sx*(3.7400)},{\sy*(0.6610)})
	--({\sx*(3.7500)},{\sy*(0.6987)})
	--({\sx*(3.7600)},{\sy*(0.7303)})
	--({\sx*(3.7700)},{\sy*(0.7555)})
	--({\sx*(3.7800)},{\sy*(0.7740)})
	--({\sx*(3.7900)},{\sy*(0.7857)})
	--({\sx*(3.8000)},{\sy*(0.7905)})
	--({\sx*(3.8100)},{\sy*(0.7883)})
	--({\sx*(3.8200)},{\sy*(0.7791)})
	--({\sx*(3.8300)},{\sy*(0.7631)})
	--({\sx*(3.8400)},{\sy*(0.7403)})
	--({\sx*(3.8500)},{\sy*(0.7109)})
	--({\sx*(3.8600)},{\sy*(0.6753)})
	--({\sx*(3.8700)},{\sy*(0.6337)})
	--({\sx*(3.8800)},{\sy*(0.5865)})
	--({\sx*(3.8900)},{\sy*(0.5342)})
	--({\sx*(3.9000)},{\sy*(0.4772)})
	--({\sx*(3.9100)},{\sy*(0.4160)})
	--({\sx*(3.9200)},{\sy*(0.3512)})
	--({\sx*(3.9300)},{\sy*(0.2835)})
	--({\sx*(3.9400)},{\sy*(0.2134)})
	--({\sx*(3.9500)},{\sy*(0.1417)})
	--({\sx*(3.9600)},{\sy*(0.0690)})
	--({\sx*(3.9700)},{\sy*(-0.0039)})
	--({\sx*(3.9800)},{\sy*(-0.0764)})
	--({\sx*(3.9900)},{\sy*(-0.1477)})
	--({\sx*(4.0000)},{\sy*(-0.2171)})
	--({\sx*(4.0100)},{\sy*(-0.2839)})
	--({\sx*(4.0200)},{\sy*(-0.3475)})
	--({\sx*(4.0300)},{\sy*(-0.4071)})
	--({\sx*(4.0400)},{\sy*(-0.4622)})
	--({\sx*(4.0500)},{\sy*(-0.5122)})
	--({\sx*(4.0600)},{\sy*(-0.5565)})
	--({\sx*(4.0700)},{\sy*(-0.5947)})
	--({\sx*(4.0800)},{\sy*(-0.6264)})
	--({\sx*(4.0900)},{\sy*(-0.6511)})
	--({\sx*(4.1000)},{\sy*(-0.6687)})
	--({\sx*(4.1100)},{\sy*(-0.6789)})
	--({\sx*(4.1200)},{\sy*(-0.6816)})
	--({\sx*(4.1300)},{\sy*(-0.6767)})
	--({\sx*(4.1400)},{\sy*(-0.6643)})
	--({\sx*(4.1500)},{\sy*(-0.6445)})
	--({\sx*(4.1600)},{\sy*(-0.6175)})
	--({\sx*(4.1700)},{\sy*(-0.5836)})
	--({\sx*(4.1800)},{\sy*(-0.5432)})
	--({\sx*(4.1900)},{\sy*(-0.4967)})
	--({\sx*(4.2000)},{\sy*(-0.4447)})
	--({\sx*(4.2100)},{\sy*(-0.3878)})
	--({\sx*(4.2200)},{\sy*(-0.3266)})
	--({\sx*(4.2300)},{\sy*(-0.2619)})
	--({\sx*(4.2400)},{\sy*(-0.1945)})
	--({\sx*(4.2500)},{\sy*(-0.1251)})
	--({\sx*(4.2600)},{\sy*(-0.0548)})
	--({\sx*(4.2700)},{\sy*(0.0157)})
	--({\sx*(4.2800)},{\sy*(0.0854)})
	--({\sx*(4.2900)},{\sy*(0.1535)})
	--({\sx*(4.3000)},{\sy*(0.2189)})
	--({\sx*(4.3100)},{\sy*(0.2809)})
	--({\sx*(4.3200)},{\sy*(0.3386)})
	--({\sx*(4.3300)},{\sy*(0.3911)})
	--({\sx*(4.3400)},{\sy*(0.4378)})
	--({\sx*(4.3500)},{\sy*(0.4779)})
	--({\sx*(4.3600)},{\sy*(0.5109)})
	--({\sx*(4.3700)},{\sy*(0.5362)})
	--({\sx*(4.3800)},{\sy*(0.5535)})
	--({\sx*(4.3900)},{\sy*(0.5624)})
	--({\sx*(4.4000)},{\sy*(0.5628)})
	--({\sx*(4.4100)},{\sy*(0.5547)})
	--({\sx*(4.4200)},{\sy*(0.5380)})
	--({\sx*(4.4300)},{\sy*(0.5131)})
	--({\sx*(4.4400)},{\sy*(0.4803)})
	--({\sx*(4.4500)},{\sy*(0.4401)})
	--({\sx*(4.4600)},{\sy*(0.3930)})
	--({\sx*(4.4700)},{\sy*(0.3399)})
	--({\sx*(4.4800)},{\sy*(0.2816)})
	--({\sx*(4.4900)},{\sy*(0.2191)})
	--({\sx*(4.5000)},{\sy*(0.1534)})
	--({\sx*(4.5100)},{\sy*(0.0858)})
	--({\sx*(4.5200)},{\sy*(0.0174)})
	--({\sx*(4.5300)},{\sy*(-0.0506)})
	--({\sx*(4.5400)},{\sy*(-0.1167)})
	--({\sx*(4.5500)},{\sy*(-0.1798)})
	--({\sx*(4.5600)},{\sy*(-0.2386)})
	--({\sx*(4.5700)},{\sy*(-0.2918)})
	--({\sx*(4.5800)},{\sy*(-0.3384)})
	--({\sx*(4.5900)},{\sy*(-0.3772)})
	--({\sx*(4.6000)},{\sy*(-0.4075)})
	--({\sx*(4.6100)},{\sy*(-0.4284)})
	--({\sx*(4.6200)},{\sy*(-0.4394)})
	--({\sx*(4.6300)},{\sy*(-0.4402)})
	--({\sx*(4.6400)},{\sy*(-0.4306)})
	--({\sx*(4.6500)},{\sy*(-0.4108)})
	--({\sx*(4.6600)},{\sy*(-0.3811)})
	--({\sx*(4.6700)},{\sy*(-0.3422)})
	--({\sx*(4.6800)},{\sy*(-0.2950)})
	--({\sx*(4.6900)},{\sy*(-0.2407)})
	--({\sx*(4.7000)},{\sy*(-0.1807)})
	--({\sx*(4.7100)},{\sy*(-0.1165)})
	--({\sx*(4.7200)},{\sy*(-0.0502)})
	--({\sx*(4.7300)},{\sy*(0.0165)})
	--({\sx*(4.7400)},{\sy*(0.0812)})
	--({\sx*(4.7500)},{\sy*(0.1419)})
	--({\sx*(4.7600)},{\sy*(0.1965)})
	--({\sx*(4.7700)},{\sy*(0.2429)})
	--({\sx*(4.7800)},{\sy*(0.2793)})
	--({\sx*(4.7900)},{\sy*(0.3040)})
	--({\sx*(4.8000)},{\sy*(0.3159)})
	--({\sx*(4.8100)},{\sy*(0.3141)})
	--({\sx*(4.8200)},{\sy*(0.2983)})
	--({\sx*(4.8300)},{\sy*(0.2691)})
	--({\sx*(4.8400)},{\sy*(0.2275)})
	--({\sx*(4.8500)},{\sy*(0.1753)})
	--({\sx*(4.8600)},{\sy*(0.1151)})
	--({\sx*(4.8700)},{\sy*(0.0502)})
	--({\sx*(4.8800)},{\sy*(-0.0153)})
	--({\sx*(4.8900)},{\sy*(-0.0768)})
	--({\sx*(4.9000)},{\sy*(-0.1296)})
	--({\sx*(4.9100)},{\sy*(-0.1688)})
	--({\sx*(4.9200)},{\sy*(-0.1900)})
	--({\sx*(4.9300)},{\sy*(-0.1898)})
	--({\sx*(4.9400)},{\sy*(-0.1667)})
	--({\sx*(4.9500)},{\sy*(-0.1221)})
	--({\sx*(4.9600)},{\sy*(-0.0612)})
	--({\sx*(4.9700)},{\sy*(0.0050)})
	--({\sx*(4.9800)},{\sy*(0.0580)})
	--({\sx*(4.9900)},{\sy*(0.0699)})
	--({\sx*(5.0000)},{\sy*(0.0000)});
}
\def\relfehlerj{
\draw[color=blue,line width=1.4pt,line join=round] ({\sx*(0.000)},{\sy*(0.0000)})
	--({\sx*(0.0100)},{\sy*(-0.0000)})
	--({\sx*(0.0200)},{\sy*(-0.0000)})
	--({\sx*(0.0300)},{\sy*(-0.0000)})
	--({\sx*(0.0400)},{\sy*(0.0000)})
	--({\sx*(0.0500)},{\sy*(0.0000)})
	--({\sx*(0.0600)},{\sy*(0.0000)})
	--({\sx*(0.0700)},{\sy*(0.0000)})
	--({\sx*(0.0800)},{\sy*(0.0000)})
	--({\sx*(0.0900)},{\sy*(0.0000)})
	--({\sx*(0.1000)},{\sy*(0.0000)})
	--({\sx*(0.1100)},{\sy*(0.0000)})
	--({\sx*(0.1200)},{\sy*(0.0000)})
	--({\sx*(0.1300)},{\sy*(-0.0000)})
	--({\sx*(0.1400)},{\sy*(-0.0000)})
	--({\sx*(0.1500)},{\sy*(-0.0000)})
	--({\sx*(0.1600)},{\sy*(-0.0000)})
	--({\sx*(0.1700)},{\sy*(-0.0000)})
	--({\sx*(0.1800)},{\sy*(-0.0000)})
	--({\sx*(0.1900)},{\sy*(-0.0000)})
	--({\sx*(0.2000)},{\sy*(-0.0000)})
	--({\sx*(0.2100)},{\sy*(-0.0000)})
	--({\sx*(0.2200)},{\sy*(-0.0000)})
	--({\sx*(0.2300)},{\sy*(-0.0000)})
	--({\sx*(0.2400)},{\sy*(-0.0000)})
	--({\sx*(0.2500)},{\sy*(-0.0000)})
	--({\sx*(0.2600)},{\sy*(-0.0000)})
	--({\sx*(0.2700)},{\sy*(-0.0000)})
	--({\sx*(0.2800)},{\sy*(0.0000)})
	--({\sx*(0.2900)},{\sy*(0.0000)})
	--({\sx*(0.3000)},{\sy*(0.0000)})
	--({\sx*(0.3100)},{\sy*(0.0000)})
	--({\sx*(0.3200)},{\sy*(0.0000)})
	--({\sx*(0.3300)},{\sy*(0.0000)})
	--({\sx*(0.3400)},{\sy*(0.0000)})
	--({\sx*(0.3500)},{\sy*(0.0000)})
	--({\sx*(0.3600)},{\sy*(0.0000)})
	--({\sx*(0.3700)},{\sy*(0.0000)})
	--({\sx*(0.3800)},{\sy*(0.0000)})
	--({\sx*(0.3900)},{\sy*(0.0000)})
	--({\sx*(0.4000)},{\sy*(0.0000)})
	--({\sx*(0.4100)},{\sy*(0.0000)})
	--({\sx*(0.4200)},{\sy*(0.0000)})
	--({\sx*(0.4300)},{\sy*(0.0000)})
	--({\sx*(0.4400)},{\sy*(0.0000)})
	--({\sx*(0.4500)},{\sy*(0.0000)})
	--({\sx*(0.4600)},{\sy*(0.0000)})
	--({\sx*(0.4700)},{\sy*(0.0000)})
	--({\sx*(0.4800)},{\sy*(-0.0000)})
	--({\sx*(0.4900)},{\sy*(-0.0000)})
	--({\sx*(0.5000)},{\sy*(-0.0000)})
	--({\sx*(0.5100)},{\sy*(-0.0000)})
	--({\sx*(0.5200)},{\sy*(-0.0000)})
	--({\sx*(0.5300)},{\sy*(-0.0000)})
	--({\sx*(0.5400)},{\sy*(-0.0000)})
	--({\sx*(0.5500)},{\sy*(-0.0000)})
	--({\sx*(0.5600)},{\sy*(-0.0000)})
	--({\sx*(0.5700)},{\sy*(-0.0000)})
	--({\sx*(0.5800)},{\sy*(-0.0000)})
	--({\sx*(0.5900)},{\sy*(-0.0000)})
	--({\sx*(0.6000)},{\sy*(-0.0000)})
	--({\sx*(0.6100)},{\sy*(-0.0000)})
	--({\sx*(0.6200)},{\sy*(-0.0000)})
	--({\sx*(0.6300)},{\sy*(-0.0000)})
	--({\sx*(0.6400)},{\sy*(-0.0000)})
	--({\sx*(0.6500)},{\sy*(-0.0000)})
	--({\sx*(0.6600)},{\sy*(-0.0000)})
	--({\sx*(0.6700)},{\sy*(-0.0000)})
	--({\sx*(0.6800)},{\sy*(-0.0000)})
	--({\sx*(0.6900)},{\sy*(-0.0000)})
	--({\sx*(0.7000)},{\sy*(-0.0000)})
	--({\sx*(0.7100)},{\sy*(-0.0000)})
	--({\sx*(0.7200)},{\sy*(-0.0000)})
	--({\sx*(0.7300)},{\sy*(-0.0000)})
	--({\sx*(0.7400)},{\sy*(0.0000)})
	--({\sx*(0.7500)},{\sy*(0.0000)})
	--({\sx*(0.7600)},{\sy*(0.0000)})
	--({\sx*(0.7700)},{\sy*(0.0000)})
	--({\sx*(0.7800)},{\sy*(0.0000)})
	--({\sx*(0.7900)},{\sy*(0.0000)})
	--({\sx*(0.8000)},{\sy*(0.0000)})
	--({\sx*(0.8100)},{\sy*(0.0000)})
	--({\sx*(0.8200)},{\sy*(0.0000)})
	--({\sx*(0.8300)},{\sy*(0.0000)})
	--({\sx*(0.8400)},{\sy*(0.0000)})
	--({\sx*(0.8500)},{\sy*(0.0000)})
	--({\sx*(0.8600)},{\sy*(0.0000)})
	--({\sx*(0.8700)},{\sy*(0.0000)})
	--({\sx*(0.8800)},{\sy*(0.0000)})
	--({\sx*(0.8900)},{\sy*(0.0000)})
	--({\sx*(0.9000)},{\sy*(0.0000)})
	--({\sx*(0.9100)},{\sy*(0.0000)})
	--({\sx*(0.9200)},{\sy*(0.0000)})
	--({\sx*(0.9300)},{\sy*(0.0000)})
	--({\sx*(0.9400)},{\sy*(0.0000)})
	--({\sx*(0.9500)},{\sy*(0.0000)})
	--({\sx*(0.9600)},{\sy*(0.0000)})
	--({\sx*(0.9700)},{\sy*(0.0000)})
	--({\sx*(0.9800)},{\sy*(0.0000)})
	--({\sx*(0.9900)},{\sy*(0.0000)})
	--({\sx*(1.0000)},{\sy*(0.0000)})
	--({\sx*(1.0100)},{\sy*(0.0000)})
	--({\sx*(1.0200)},{\sy*(0.0000)})
	--({\sx*(1.0300)},{\sy*(0.0000)})
	--({\sx*(1.0400)},{\sy*(-0.0000)})
	--({\sx*(1.0500)},{\sy*(-0.0000)})
	--({\sx*(1.0600)},{\sy*(-0.0000)})
	--({\sx*(1.0700)},{\sy*(-0.0000)})
	--({\sx*(1.0800)},{\sy*(-0.0000)})
	--({\sx*(1.0900)},{\sy*(-0.0000)})
	--({\sx*(1.1000)},{\sy*(-0.0000)})
	--({\sx*(1.1100)},{\sy*(-0.0000)})
	--({\sx*(1.1200)},{\sy*(-0.0000)})
	--({\sx*(1.1300)},{\sy*(-0.0000)})
	--({\sx*(1.1400)},{\sy*(-0.0000)})
	--({\sx*(1.1500)},{\sy*(-0.0000)})
	--({\sx*(1.1600)},{\sy*(-0.0000)})
	--({\sx*(1.1700)},{\sy*(-0.0000)})
	--({\sx*(1.1800)},{\sy*(-0.0000)})
	--({\sx*(1.1900)},{\sy*(-0.0000)})
	--({\sx*(1.2000)},{\sy*(-0.0000)})
	--({\sx*(1.2100)},{\sy*(-0.0000)})
	--({\sx*(1.2200)},{\sy*(-0.0000)})
	--({\sx*(1.2300)},{\sy*(-0.0000)})
	--({\sx*(1.2400)},{\sy*(-0.0000)})
	--({\sx*(1.2500)},{\sy*(-0.0000)})
	--({\sx*(1.2600)},{\sy*(-0.0000)})
	--({\sx*(1.2700)},{\sy*(-0.0000)})
	--({\sx*(1.2800)},{\sy*(-0.0000)})
	--({\sx*(1.2900)},{\sy*(-0.0000)})
	--({\sx*(1.3000)},{\sy*(-0.0000)})
	--({\sx*(1.3100)},{\sy*(-0.0000)})
	--({\sx*(1.3200)},{\sy*(-0.0000)})
	--({\sx*(1.3300)},{\sy*(-0.0000)})
	--({\sx*(1.3400)},{\sy*(-0.0000)})
	--({\sx*(1.3500)},{\sy*(-0.0000)})
	--({\sx*(1.3600)},{\sy*(-0.0000)})
	--({\sx*(1.3700)},{\sy*(0.0000)})
	--({\sx*(1.3800)},{\sy*(0.0000)})
	--({\sx*(1.3900)},{\sy*(0.0000)})
	--({\sx*(1.4000)},{\sy*(0.0000)})
	--({\sx*(1.4100)},{\sy*(0.0000)})
	--({\sx*(1.4200)},{\sy*(0.0000)})
	--({\sx*(1.4300)},{\sy*(0.0000)})
	--({\sx*(1.4400)},{\sy*(0.0000)})
	--({\sx*(1.4500)},{\sy*(0.0000)})
	--({\sx*(1.4600)},{\sy*(0.0000)})
	--({\sx*(1.4700)},{\sy*(0.0000)})
	--({\sx*(1.4800)},{\sy*(0.0000)})
	--({\sx*(1.4900)},{\sy*(0.0000)})
	--({\sx*(1.5000)},{\sy*(0.0000)})
	--({\sx*(1.5100)},{\sy*(0.0000)})
	--({\sx*(1.5200)},{\sy*(0.0000)})
	--({\sx*(1.5300)},{\sy*(0.0000)})
	--({\sx*(1.5400)},{\sy*(0.0000)})
	--({\sx*(1.5500)},{\sy*(0.0000)})
	--({\sx*(1.5600)},{\sy*(0.0000)})
	--({\sx*(1.5700)},{\sy*(0.0000)})
	--({\sx*(1.5800)},{\sy*(0.0000)})
	--({\sx*(1.5900)},{\sy*(0.0000)})
	--({\sx*(1.6000)},{\sy*(0.0000)})
	--({\sx*(1.6100)},{\sy*(0.0000)})
	--({\sx*(1.6200)},{\sy*(0.0000)})
	--({\sx*(1.6300)},{\sy*(0.0000)})
	--({\sx*(1.6400)},{\sy*(0.0000)})
	--({\sx*(1.6500)},{\sy*(0.0000)})
	--({\sx*(1.6600)},{\sy*(0.0000)})
	--({\sx*(1.6700)},{\sy*(0.0000)})
	--({\sx*(1.6800)},{\sy*(0.0000)})
	--({\sx*(1.6900)},{\sy*(0.0000)})
	--({\sx*(1.7000)},{\sy*(0.0000)})
	--({\sx*(1.7100)},{\sy*(0.0000)})
	--({\sx*(1.7200)},{\sy*(0.0000)})
	--({\sx*(1.7300)},{\sy*(-0.0000)})
	--({\sx*(1.7400)},{\sy*(-0.0000)})
	--({\sx*(1.7500)},{\sy*(-0.0000)})
	--({\sx*(1.7600)},{\sy*(-0.0000)})
	--({\sx*(1.7700)},{\sy*(-0.0000)})
	--({\sx*(1.7800)},{\sy*(-0.0000)})
	--({\sx*(1.7900)},{\sy*(-0.0000)})
	--({\sx*(1.8000)},{\sy*(-0.0000)})
	--({\sx*(1.8100)},{\sy*(-0.0000)})
	--({\sx*(1.8200)},{\sy*(-0.0000)})
	--({\sx*(1.8300)},{\sy*(-0.0000)})
	--({\sx*(1.8400)},{\sy*(-0.0000)})
	--({\sx*(1.8500)},{\sy*(-0.0000)})
	--({\sx*(1.8600)},{\sy*(-0.0000)})
	--({\sx*(1.8700)},{\sy*(-0.0000)})
	--({\sx*(1.8800)},{\sy*(-0.0000)})
	--({\sx*(1.8900)},{\sy*(-0.0000)})
	--({\sx*(1.9000)},{\sy*(-0.0000)})
	--({\sx*(1.9100)},{\sy*(-0.0000)})
	--({\sx*(1.9200)},{\sy*(-0.0000)})
	--({\sx*(1.9300)},{\sy*(-0.0000)})
	--({\sx*(1.9400)},{\sy*(-0.0000)})
	--({\sx*(1.9500)},{\sy*(-0.0000)})
	--({\sx*(1.9600)},{\sy*(-0.0000)})
	--({\sx*(1.9700)},{\sy*(-0.0000)})
	--({\sx*(1.9800)},{\sy*(-0.0000)})
	--({\sx*(1.9900)},{\sy*(-0.0000)})
	--({\sx*(2.0000)},{\sy*(-0.0000)})
	--({\sx*(2.0100)},{\sy*(-0.0000)})
	--({\sx*(2.0200)},{\sy*(-0.0000)})
	--({\sx*(2.0300)},{\sy*(-0.0000)})
	--({\sx*(2.0400)},{\sy*(-0.0000)})
	--({\sx*(2.0500)},{\sy*(-0.0000)})
	--({\sx*(2.0600)},{\sy*(-0.0000)})
	--({\sx*(2.0700)},{\sy*(-0.0000)})
	--({\sx*(2.0800)},{\sy*(-0.0000)})
	--({\sx*(2.0900)},{\sy*(-0.0000)})
	--({\sx*(2.1000)},{\sy*(-0.0000)})
	--({\sx*(2.1100)},{\sy*(0.0000)})
	--({\sx*(2.1200)},{\sy*(0.0000)})
	--({\sx*(2.1300)},{\sy*(0.0000)})
	--({\sx*(2.1400)},{\sy*(0.0000)})
	--({\sx*(2.1500)},{\sy*(0.0000)})
	--({\sx*(2.1600)},{\sy*(0.0000)})
	--({\sx*(2.1700)},{\sy*(0.0000)})
	--({\sx*(2.1800)},{\sy*(0.0000)})
	--({\sx*(2.1900)},{\sy*(0.0000)})
	--({\sx*(2.2000)},{\sy*(0.0000)})
	--({\sx*(2.2100)},{\sy*(0.0000)})
	--({\sx*(2.2200)},{\sy*(0.0000)})
	--({\sx*(2.2300)},{\sy*(0.0000)})
	--({\sx*(2.2400)},{\sy*(0.0000)})
	--({\sx*(2.2500)},{\sy*(0.0000)})
	--({\sx*(2.2600)},{\sy*(0.0000)})
	--({\sx*(2.2700)},{\sy*(0.0000)})
	--({\sx*(2.2800)},{\sy*(0.0000)})
	--({\sx*(2.2900)},{\sy*(0.0000)})
	--({\sx*(2.3000)},{\sy*(0.0000)})
	--({\sx*(2.3100)},{\sy*(0.0000)})
	--({\sx*(2.3200)},{\sy*(0.0000)})
	--({\sx*(2.3300)},{\sy*(0.0000)})
	--({\sx*(2.3400)},{\sy*(0.0000)})
	--({\sx*(2.3500)},{\sy*(0.0000)})
	--({\sx*(2.3600)},{\sy*(0.0000)})
	--({\sx*(2.3700)},{\sy*(0.0000)})
	--({\sx*(2.3800)},{\sy*(0.0000)})
	--({\sx*(2.3900)},{\sy*(0.0000)})
	--({\sx*(2.4000)},{\sy*(0.0000)})
	--({\sx*(2.4100)},{\sy*(0.0000)})
	--({\sx*(2.4200)},{\sy*(0.0000)})
	--({\sx*(2.4300)},{\sy*(0.0000)})
	--({\sx*(2.4400)},{\sy*(0.0000)})
	--({\sx*(2.4500)},{\sy*(0.0000)})
	--({\sx*(2.4600)},{\sy*(0.0000)})
	--({\sx*(2.4700)},{\sy*(0.0000)})
	--({\sx*(2.4800)},{\sy*(0.0000)})
	--({\sx*(2.4900)},{\sy*(0.0000)})
	--({\sx*(2.5000)},{\sy*(0.0000)})
	--({\sx*(2.5100)},{\sy*(-0.0000)})
	--({\sx*(2.5200)},{\sy*(-0.0000)})
	--({\sx*(2.5300)},{\sy*(-0.0000)})
	--({\sx*(2.5400)},{\sy*(-0.0000)})
	--({\sx*(2.5500)},{\sy*(-0.0000)})
	--({\sx*(2.5600)},{\sy*(-0.0000)})
	--({\sx*(2.5700)},{\sy*(-0.0000)})
	--({\sx*(2.5800)},{\sy*(-0.0000)})
	--({\sx*(2.5900)},{\sy*(-0.0000)})
	--({\sx*(2.6000)},{\sy*(-0.0000)})
	--({\sx*(2.6100)},{\sy*(-0.0000)})
	--({\sx*(2.6200)},{\sy*(-0.0000)})
	--({\sx*(2.6300)},{\sy*(-0.0000)})
	--({\sx*(2.6400)},{\sy*(-0.0000)})
	--({\sx*(2.6500)},{\sy*(-0.0000)})
	--({\sx*(2.6600)},{\sy*(-0.0000)})
	--({\sx*(2.6700)},{\sy*(-0.0000)})
	--({\sx*(2.6800)},{\sy*(-0.0000)})
	--({\sx*(2.6900)},{\sy*(-0.0000)})
	--({\sx*(2.7000)},{\sy*(-0.0000)})
	--({\sx*(2.7100)},{\sy*(-0.0000)})
	--({\sx*(2.7200)},{\sy*(-0.0000)})
	--({\sx*(2.7300)},{\sy*(-0.0000)})
	--({\sx*(2.7400)},{\sy*(-0.0000)})
	--({\sx*(2.7500)},{\sy*(-0.0000)})
	--({\sx*(2.7600)},{\sy*(-0.0000)})
	--({\sx*(2.7700)},{\sy*(-0.0000)})
	--({\sx*(2.7800)},{\sy*(-0.0000)})
	--({\sx*(2.7900)},{\sy*(-0.0000)})
	--({\sx*(2.8000)},{\sy*(-0.0000)})
	--({\sx*(2.8100)},{\sy*(-0.0000)})
	--({\sx*(2.8200)},{\sy*(-0.0000)})
	--({\sx*(2.8300)},{\sy*(-0.0000)})
	--({\sx*(2.8400)},{\sy*(-0.0000)})
	--({\sx*(2.8500)},{\sy*(-0.0000)})
	--({\sx*(2.8600)},{\sy*(-0.0000)})
	--({\sx*(2.8700)},{\sy*(-0.0000)})
	--({\sx*(2.8800)},{\sy*(-0.0000)})
	--({\sx*(2.8900)},{\sy*(-0.0000)})
	--({\sx*(2.9000)},{\sy*(0.0000)})
	--({\sx*(2.9100)},{\sy*(0.0000)})
	--({\sx*(2.9200)},{\sy*(0.0000)})
	--({\sx*(2.9300)},{\sy*(0.0000)})
	--({\sx*(2.9400)},{\sy*(0.0000)})
	--({\sx*(2.9500)},{\sy*(0.0000)})
	--({\sx*(2.9600)},{\sy*(0.0000)})
	--({\sx*(2.9700)},{\sy*(0.0000)})
	--({\sx*(2.9800)},{\sy*(0.0000)})
	--({\sx*(2.9900)},{\sy*(0.0000)})
	--({\sx*(3.0000)},{\sy*(0.0000)})
	--({\sx*(3.0100)},{\sy*(0.0000)})
	--({\sx*(3.0200)},{\sy*(0.0000)})
	--({\sx*(3.0300)},{\sy*(0.0000)})
	--({\sx*(3.0400)},{\sy*(0.0000)})
	--({\sx*(3.0500)},{\sy*(0.0000)})
	--({\sx*(3.0600)},{\sy*(0.0000)})
	--({\sx*(3.0700)},{\sy*(0.0000)})
	--({\sx*(3.0800)},{\sy*(0.0000)})
	--({\sx*(3.0900)},{\sy*(0.0000)})
	--({\sx*(3.1000)},{\sy*(0.0000)})
	--({\sx*(3.1100)},{\sy*(0.0000)})
	--({\sx*(3.1200)},{\sy*(0.0000)})
	--({\sx*(3.1300)},{\sy*(0.0000)})
	--({\sx*(3.1400)},{\sy*(0.0000)})
	--({\sx*(3.1500)},{\sy*(0.0000)})
	--({\sx*(3.1600)},{\sy*(0.0000)})
	--({\sx*(3.1700)},{\sy*(0.0000)})
	--({\sx*(3.1800)},{\sy*(0.0000)})
	--({\sx*(3.1900)},{\sy*(0.0000)})
	--({\sx*(3.2000)},{\sy*(0.0000)})
	--({\sx*(3.2100)},{\sy*(0.0000)})
	--({\sx*(3.2200)},{\sy*(0.0000)})
	--({\sx*(3.2300)},{\sy*(0.0000)})
	--({\sx*(3.2400)},{\sy*(0.0000)})
	--({\sx*(3.2500)},{\sy*(0.0000)})
	--({\sx*(3.2600)},{\sy*(0.0000)})
	--({\sx*(3.2700)},{\sy*(0.0000)})
	--({\sx*(3.2800)},{\sy*(-0.0000)})
	--({\sx*(3.2900)},{\sy*(-0.0000)})
	--({\sx*(3.3000)},{\sy*(-0.0000)})
	--({\sx*(3.3100)},{\sy*(-0.0000)})
	--({\sx*(3.3200)},{\sy*(-0.0000)})
	--({\sx*(3.3300)},{\sy*(-0.0000)})
	--({\sx*(3.3400)},{\sy*(-0.0000)})
	--({\sx*(3.3500)},{\sy*(-0.0000)})
	--({\sx*(3.3600)},{\sy*(-0.0000)})
	--({\sx*(3.3700)},{\sy*(-0.0000)})
	--({\sx*(3.3800)},{\sy*(-0.0000)})
	--({\sx*(3.3900)},{\sy*(-0.0000)})
	--({\sx*(3.4000)},{\sy*(-0.0000)})
	--({\sx*(3.4100)},{\sy*(-0.0000)})
	--({\sx*(3.4200)},{\sy*(-0.0000)})
	--({\sx*(3.4300)},{\sy*(-0.0000)})
	--({\sx*(3.4400)},{\sy*(-0.0000)})
	--({\sx*(3.4500)},{\sy*(-0.0000)})
	--({\sx*(3.4600)},{\sy*(-0.0000)})
	--({\sx*(3.4700)},{\sy*(-0.0000)})
	--({\sx*(3.4800)},{\sy*(-0.0000)})
	--({\sx*(3.4900)},{\sy*(-0.0000)})
	--({\sx*(3.5000)},{\sy*(-0.0000)})
	--({\sx*(3.5100)},{\sy*(-0.0000)})
	--({\sx*(3.5200)},{\sy*(-0.0000)})
	--({\sx*(3.5300)},{\sy*(-0.0000)})
	--({\sx*(3.5400)},{\sy*(-0.0000)})
	--({\sx*(3.5500)},{\sy*(-0.0000)})
	--({\sx*(3.5600)},{\sy*(-0.0000)})
	--({\sx*(3.5700)},{\sy*(-0.0000)})
	--({\sx*(3.5800)},{\sy*(-0.0000)})
	--({\sx*(3.5900)},{\sy*(-0.0000)})
	--({\sx*(3.6000)},{\sy*(-0.0000)})
	--({\sx*(3.6100)},{\sy*(-0.0000)})
	--({\sx*(3.6200)},{\sy*(-0.0000)})
	--({\sx*(3.6300)},{\sy*(-0.0000)})
	--({\sx*(3.6400)},{\sy*(0.0000)})
	--({\sx*(3.6500)},{\sy*(0.0000)})
	--({\sx*(3.6600)},{\sy*(0.0000)})
	--({\sx*(3.6700)},{\sy*(0.0000)})
	--({\sx*(3.6800)},{\sy*(0.0000)})
	--({\sx*(3.6900)},{\sy*(0.0000)})
	--({\sx*(3.7000)},{\sy*(0.0000)})
	--({\sx*(3.7100)},{\sy*(0.0000)})
	--({\sx*(3.7200)},{\sy*(0.0000)})
	--({\sx*(3.7300)},{\sy*(0.0000)})
	--({\sx*(3.7400)},{\sy*(0.0000)})
	--({\sx*(3.7500)},{\sy*(0.0000)})
	--({\sx*(3.7600)},{\sy*(0.0000)})
	--({\sx*(3.7700)},{\sy*(0.0000)})
	--({\sx*(3.7800)},{\sy*(0.0000)})
	--({\sx*(3.7900)},{\sy*(0.0000)})
	--({\sx*(3.8000)},{\sy*(0.0000)})
	--({\sx*(3.8100)},{\sy*(0.0000)})
	--({\sx*(3.8200)},{\sy*(0.0000)})
	--({\sx*(3.8300)},{\sy*(0.0000)})
	--({\sx*(3.8400)},{\sy*(0.0000)})
	--({\sx*(3.8500)},{\sy*(0.0000)})
	--({\sx*(3.8600)},{\sy*(0.0000)})
	--({\sx*(3.8700)},{\sy*(0.0000)})
	--({\sx*(3.8800)},{\sy*(0.0000)})
	--({\sx*(3.8900)},{\sy*(0.0000)})
	--({\sx*(3.9000)},{\sy*(0.0000)})
	--({\sx*(3.9100)},{\sy*(0.0000)})
	--({\sx*(3.9200)},{\sy*(0.0000)})
	--({\sx*(3.9300)},{\sy*(0.0000)})
	--({\sx*(3.9400)},{\sy*(0.0000)})
	--({\sx*(3.9500)},{\sy*(0.0000)})
	--({\sx*(3.9600)},{\sy*(0.0000)})
	--({\sx*(3.9700)},{\sy*(-0.0000)})
	--({\sx*(3.9800)},{\sy*(-0.0000)})
	--({\sx*(3.9900)},{\sy*(-0.0000)})
	--({\sx*(4.0000)},{\sy*(-0.0000)})
	--({\sx*(4.0100)},{\sy*(-0.0000)})
	--({\sx*(4.0200)},{\sy*(-0.0000)})
	--({\sx*(4.0300)},{\sy*(-0.0000)})
	--({\sx*(4.0400)},{\sy*(-0.0000)})
	--({\sx*(4.0500)},{\sy*(-0.0000)})
	--({\sx*(4.0600)},{\sy*(-0.0000)})
	--({\sx*(4.0700)},{\sy*(-0.0000)})
	--({\sx*(4.0800)},{\sy*(-0.0000)})
	--({\sx*(4.0900)},{\sy*(-0.0000)})
	--({\sx*(4.1000)},{\sy*(-0.0000)})
	--({\sx*(4.1100)},{\sy*(-0.0000)})
	--({\sx*(4.1200)},{\sy*(-0.0000)})
	--({\sx*(4.1300)},{\sy*(-0.0000)})
	--({\sx*(4.1400)},{\sy*(-0.0000)})
	--({\sx*(4.1500)},{\sy*(-0.0000)})
	--({\sx*(4.1600)},{\sy*(-0.0000)})
	--({\sx*(4.1700)},{\sy*(-0.0000)})
	--({\sx*(4.1800)},{\sy*(-0.0000)})
	--({\sx*(4.1900)},{\sy*(-0.0000)})
	--({\sx*(4.2000)},{\sy*(-0.0000)})
	--({\sx*(4.2100)},{\sy*(-0.0000)})
	--({\sx*(4.2200)},{\sy*(-0.0000)})
	--({\sx*(4.2300)},{\sy*(-0.0000)})
	--({\sx*(4.2400)},{\sy*(-0.0000)})
	--({\sx*(4.2500)},{\sy*(-0.0000)})
	--({\sx*(4.2600)},{\sy*(-0.0000)})
	--({\sx*(4.2700)},{\sy*(0.0000)})
	--({\sx*(4.2800)},{\sy*(0.0000)})
	--({\sx*(4.2900)},{\sy*(0.0000)})
	--({\sx*(4.3000)},{\sy*(0.0000)})
	--({\sx*(4.3100)},{\sy*(0.0000)})
	--({\sx*(4.3200)},{\sy*(0.0000)})
	--({\sx*(4.3300)},{\sy*(0.0000)})
	--({\sx*(4.3400)},{\sy*(0.0000)})
	--({\sx*(4.3500)},{\sy*(0.0000)})
	--({\sx*(4.3600)},{\sy*(0.0000)})
	--({\sx*(4.3700)},{\sy*(0.0000)})
	--({\sx*(4.3800)},{\sy*(0.0000)})
	--({\sx*(4.3900)},{\sy*(0.0000)})
	--({\sx*(4.4000)},{\sy*(0.0000)})
	--({\sx*(4.4100)},{\sy*(0.0000)})
	--({\sx*(4.4200)},{\sy*(0.0000)})
	--({\sx*(4.4300)},{\sy*(0.0000)})
	--({\sx*(4.4400)},{\sy*(0.0000)})
	--({\sx*(4.4500)},{\sy*(0.0000)})
	--({\sx*(4.4600)},{\sy*(0.0000)})
	--({\sx*(4.4700)},{\sy*(0.0000)})
	--({\sx*(4.4800)},{\sy*(0.0000)})
	--({\sx*(4.4900)},{\sy*(0.0000)})
	--({\sx*(4.5000)},{\sy*(0.0000)})
	--({\sx*(4.5100)},{\sy*(0.0000)})
	--({\sx*(4.5200)},{\sy*(0.0000)})
	--({\sx*(4.5300)},{\sy*(-0.0000)})
	--({\sx*(4.5400)},{\sy*(-0.0000)})
	--({\sx*(4.5500)},{\sy*(-0.0000)})
	--({\sx*(4.5600)},{\sy*(-0.0000)})
	--({\sx*(4.5700)},{\sy*(-0.0000)})
	--({\sx*(4.5800)},{\sy*(-0.0000)})
	--({\sx*(4.5900)},{\sy*(-0.0000)})
	--({\sx*(4.6000)},{\sy*(-0.0000)})
	--({\sx*(4.6100)},{\sy*(-0.0000)})
	--({\sx*(4.6200)},{\sy*(-0.0000)})
	--({\sx*(4.6300)},{\sy*(-0.0000)})
	--({\sx*(4.6400)},{\sy*(-0.0000)})
	--({\sx*(4.6500)},{\sy*(-0.0000)})
	--({\sx*(4.6600)},{\sy*(-0.0000)})
	--({\sx*(4.6700)},{\sy*(-0.0000)})
	--({\sx*(4.6800)},{\sy*(-0.0000)})
	--({\sx*(4.6900)},{\sy*(-0.0000)})
	--({\sx*(4.7000)},{\sy*(-0.0000)})
	--({\sx*(4.7100)},{\sy*(-0.0000)})
	--({\sx*(4.7200)},{\sy*(-0.0000)})
	--({\sx*(4.7300)},{\sy*(0.0000)})
	--({\sx*(4.7400)},{\sy*(0.0000)})
	--({\sx*(4.7500)},{\sy*(0.0000)})
	--({\sx*(4.7600)},{\sy*(0.0000)})
	--({\sx*(4.7700)},{\sy*(0.0000)})
	--({\sx*(4.7800)},{\sy*(0.0000)})
	--({\sx*(4.7900)},{\sy*(0.0000)})
	--({\sx*(4.8000)},{\sy*(0.0000)})
	--({\sx*(4.8100)},{\sy*(0.0000)})
	--({\sx*(4.8200)},{\sy*(0.0000)})
	--({\sx*(4.8300)},{\sy*(0.0000)})
	--({\sx*(4.8400)},{\sy*(0.0000)})
	--({\sx*(4.8500)},{\sy*(0.0000)})
	--({\sx*(4.8600)},{\sy*(0.0000)})
	--({\sx*(4.8700)},{\sy*(0.0000)})
	--({\sx*(4.8800)},{\sy*(-0.0000)})
	--({\sx*(4.8900)},{\sy*(-0.0000)})
	--({\sx*(4.9000)},{\sy*(-0.0000)})
	--({\sx*(4.9100)},{\sy*(-0.0000)})
	--({\sx*(4.9200)},{\sy*(-0.0001)})
	--({\sx*(4.9300)},{\sy*(-0.0001)})
	--({\sx*(4.9400)},{\sy*(-0.0000)})
	--({\sx*(4.9500)},{\sy*(-0.0000)})
	--({\sx*(4.9600)},{\sy*(-0.0000)})
	--({\sx*(4.9700)},{\sy*(0.0000)})
	--({\sx*(4.9800)},{\sy*(0.0000)})
	--({\sx*(4.9900)},{\sy*(0.0000)})
	--({\sx*(5.0000)},{\sy*(0.0000)});
}
\def\xwertek{
\fill[color=red] (0.0000,0) circle[radius={0.07/\skala}];
\fill[color=white] (0.0000,0) circle[radius={0.05/\skala}];
\fill[color=red] (0.0254,0) circle[radius={0.07/\skala}];
\fill[color=white] (0.0254,0) circle[radius={0.05/\skala}];
\fill[color=red] (0.1013,0) circle[radius={0.07/\skala}];
\fill[color=white] (0.1013,0) circle[radius={0.05/\skala}];
\fill[color=red] (0.2259,0) circle[radius={0.07/\skala}];
\fill[color=white] (0.2259,0) circle[radius={0.05/\skala}];
\fill[color=red] (0.3969,0) circle[radius={0.07/\skala}];
\fill[color=white] (0.3969,0) circle[radius={0.05/\skala}];
\fill[color=red] (0.6106,0) circle[radius={0.07/\skala}];
\fill[color=white] (0.6106,0) circle[radius={0.05/\skala}];
\fill[color=red] (0.8628,0) circle[radius={0.07/\skala}];
\fill[color=white] (0.8628,0) circle[radius={0.05/\skala}];
\fill[color=red] (1.1484,0) circle[radius={0.07/\skala}];
\fill[color=white] (1.1484,0) circle[radius={0.05/\skala}];
\fill[color=red] (1.4615,0) circle[radius={0.07/\skala}];
\fill[color=white] (1.4615,0) circle[radius={0.05/\skala}];
\fill[color=red] (1.7957,0) circle[radius={0.07/\skala}];
\fill[color=white] (1.7957,0) circle[radius={0.05/\skala}];
\fill[color=red] (2.1442,0) circle[radius={0.07/\skala}];
\fill[color=white] (2.1442,0) circle[radius={0.05/\skala}];
\fill[color=red] (2.5000,0) circle[radius={0.07/\skala}];
\fill[color=white] (2.5000,0) circle[radius={0.05/\skala}];
\fill[color=red] (2.8558,0) circle[radius={0.07/\skala}];
\fill[color=white] (2.8558,0) circle[radius={0.05/\skala}];
\fill[color=red] (3.2043,0) circle[radius={0.07/\skala}];
\fill[color=white] (3.2043,0) circle[radius={0.05/\skala}];
\fill[color=red] (3.5385,0) circle[radius={0.07/\skala}];
\fill[color=white] (3.5385,0) circle[radius={0.05/\skala}];
\fill[color=red] (3.8516,0) circle[radius={0.07/\skala}];
\fill[color=white] (3.8516,0) circle[radius={0.05/\skala}];
\fill[color=red] (4.1372,0) circle[radius={0.07/\skala}];
\fill[color=white] (4.1372,0) circle[radius={0.05/\skala}];
\fill[color=red] (4.3894,0) circle[radius={0.07/\skala}];
\fill[color=white] (4.3894,0) circle[radius={0.05/\skala}];
\fill[color=red] (4.6031,0) circle[radius={0.07/\skala}];
\fill[color=white] (4.6031,0) circle[radius={0.05/\skala}];
\fill[color=red] (4.7741,0) circle[radius={0.07/\skala}];
\fill[color=white] (4.7741,0) circle[radius={0.05/\skala}];
\fill[color=red] (4.8987,0) circle[radius={0.07/\skala}];
\fill[color=white] (4.8987,0) circle[radius={0.05/\skala}];
\fill[color=red] (4.9746,0) circle[radius={0.07/\skala}];
\fill[color=white] (4.9746,0) circle[radius={0.05/\skala}];
\fill[color=red] (5.0000,0) circle[radius={0.07/\skala}];
\fill[color=white] (5.0000,0) circle[radius={0.05/\skala}];
}
\def\punktek{22}
\def\maxfehlerk{4.069\cdot 10^{-11}}
\def\fehlerk{
\draw[color=red,line width=1.4pt,line join=round] ({\sx*(0.000)},{\sy*(0.0000)})
	--({\sx*(0.0100)},{\sy*(0.0828)})
	--({\sx*(0.0200)},{\sy*(0.0444)})
	--({\sx*(0.0300)},{\sy*(-0.0415)})
	--({\sx*(0.0400)},{\sy*(-0.1288)})
	--({\sx*(0.0500)},{\sy*(-0.1916)})
	--({\sx*(0.0600)},{\sy*(-0.2185)})
	--({\sx*(0.0700)},{\sy*(-0.2078)})
	--({\sx*(0.0800)},{\sy*(-0.1641)})
	--({\sx*(0.0900)},{\sy*(-0.0954)})
	--({\sx*(0.1000)},{\sy*(-0.0113)})
	--({\sx*(0.1100)},{\sy*(0.0786)})
	--({\sx*(0.1200)},{\sy*(0.1654)})
	--({\sx*(0.1300)},{\sy*(0.2415)})
	--({\sx*(0.1400)},{\sy*(0.3013)})
	--({\sx*(0.1500)},{\sy*(0.3409)})
	--({\sx*(0.1600)},{\sy*(0.3581)})
	--({\sx*(0.1700)},{\sy*(0.3526)})
	--({\sx*(0.1800)},{\sy*(0.3252)})
	--({\sx*(0.1900)},{\sy*(0.2782)})
	--({\sx*(0.2000)},{\sy*(0.2146)})
	--({\sx*(0.2100)},{\sy*(0.1381)})
	--({\sx*(0.2200)},{\sy*(0.0529)})
	--({\sx*(0.2300)},{\sy*(-0.0369)})
	--({\sx*(0.2400)},{\sy*(-0.1271)})
	--({\sx*(0.2500)},{\sy*(-0.2137)})
	--({\sx*(0.2600)},{\sy*(-0.2932)})
	--({\sx*(0.2700)},{\sy*(-0.3624)})
	--({\sx*(0.2800)},{\sy*(-0.4190)})
	--({\sx*(0.2900)},{\sy*(-0.4608)})
	--({\sx*(0.3000)},{\sy*(-0.4867)})
	--({\sx*(0.3100)},{\sy*(-0.4960)})
	--({\sx*(0.3200)},{\sy*(-0.4885)})
	--({\sx*(0.3300)},{\sy*(-0.4648)})
	--({\sx*(0.3400)},{\sy*(-0.4258)})
	--({\sx*(0.3500)},{\sy*(-0.3730)})
	--({\sx*(0.3600)},{\sy*(-0.3080)})
	--({\sx*(0.3700)},{\sy*(-0.2331)})
	--({\sx*(0.3800)},{\sy*(-0.1503)})
	--({\sx*(0.3900)},{\sy*(-0.0623)})
	--({\sx*(0.4000)},{\sy*(0.0287)})
	--({\sx*(0.4100)},{\sy*(0.1200)})
	--({\sx*(0.4200)},{\sy*(0.2093)})
	--({\sx*(0.4300)},{\sy*(0.2944)})
	--({\sx*(0.4400)},{\sy*(0.3731)})
	--({\sx*(0.4500)},{\sy*(0.4437)})
	--({\sx*(0.4600)},{\sy*(0.5043)})
	--({\sx*(0.4700)},{\sy*(0.5538)})
	--({\sx*(0.4800)},{\sy*(0.5910)})
	--({\sx*(0.4900)},{\sy*(0.6153)})
	--({\sx*(0.5000)},{\sy*(0.6262)})
	--({\sx*(0.5100)},{\sy*(0.6235)})
	--({\sx*(0.5200)},{\sy*(0.6074)})
	--({\sx*(0.5300)},{\sy*(0.5784)})
	--({\sx*(0.5400)},{\sy*(0.5372)})
	--({\sx*(0.5500)},{\sy*(0.4846)})
	--({\sx*(0.5600)},{\sy*(0.4219)})
	--({\sx*(0.5700)},{\sy*(0.3503)})
	--({\sx*(0.5800)},{\sy*(0.2713)})
	--({\sx*(0.5900)},{\sy*(0.1865)})
	--({\sx*(0.6000)},{\sy*(0.0974)})
	--({\sx*(0.6100)},{\sy*(0.0058)})
	--({\sx*(0.6200)},{\sy*(-0.0867)})
	--({\sx*(0.6300)},{\sy*(-0.1783)})
	--({\sx*(0.6400)},{\sy*(-0.2675)})
	--({\sx*(0.6500)},{\sy*(-0.3526)})
	--({\sx*(0.6600)},{\sy*(-0.4323)})
	--({\sx*(0.6700)},{\sy*(-0.5052)})
	--({\sx*(0.6800)},{\sy*(-0.5701)})
	--({\sx*(0.6900)},{\sy*(-0.6259)})
	--({\sx*(0.7000)},{\sy*(-0.6719)})
	--({\sx*(0.7100)},{\sy*(-0.7073)})
	--({\sx*(0.7200)},{\sy*(-0.7315)})
	--({\sx*(0.7300)},{\sy*(-0.7444)})
	--({\sx*(0.7400)},{\sy*(-0.7456)})
	--({\sx*(0.7500)},{\sy*(-0.7354)})
	--({\sx*(0.7600)},{\sy*(-0.7138)})
	--({\sx*(0.7700)},{\sy*(-0.6812)})
	--({\sx*(0.7800)},{\sy*(-0.6383)})
	--({\sx*(0.7900)},{\sy*(-0.5856)})
	--({\sx*(0.8000)},{\sy*(-0.5240)})
	--({\sx*(0.8100)},{\sy*(-0.4545)})
	--({\sx*(0.8200)},{\sy*(-0.3781)})
	--({\sx*(0.8300)},{\sy*(-0.2959)})
	--({\sx*(0.8400)},{\sy*(-0.2090)})
	--({\sx*(0.8500)},{\sy*(-0.1188)})
	--({\sx*(0.8600)},{\sy*(-0.0265)})
	--({\sx*(0.8700)},{\sy*(0.0666)})
	--({\sx*(0.8800)},{\sy*(0.1594)})
	--({\sx*(0.8900)},{\sy*(0.2504)})
	--({\sx*(0.9000)},{\sy*(0.3385)})
	--({\sx*(0.9100)},{\sy*(0.4225)})
	--({\sx*(0.9200)},{\sy*(0.5015)})
	--({\sx*(0.9300)},{\sy*(0.5742)})
	--({\sx*(0.9400)},{\sy*(0.6399)})
	--({\sx*(0.9500)},{\sy*(0.6978)})
	--({\sx*(0.9600)},{\sy*(0.7470)})
	--({\sx*(0.9700)},{\sy*(0.7871)})
	--({\sx*(0.9800)},{\sy*(0.8175)})
	--({\sx*(0.9900)},{\sy*(0.8379)})
	--({\sx*(1.0000)},{\sy*(0.8482)})
	--({\sx*(1.0100)},{\sy*(0.8481)})
	--({\sx*(1.0200)},{\sy*(0.8378)})
	--({\sx*(1.0300)},{\sy*(0.8174)})
	--({\sx*(1.0400)},{\sy*(0.7872)})
	--({\sx*(1.0500)},{\sy*(0.7476)})
	--({\sx*(1.0600)},{\sy*(0.6991)})
	--({\sx*(1.0700)},{\sy*(0.6423)})
	--({\sx*(1.0800)},{\sy*(0.5778)})
	--({\sx*(1.0900)},{\sy*(0.5065)})
	--({\sx*(1.1000)},{\sy*(0.4291)})
	--({\sx*(1.1100)},{\sy*(0.3466)})
	--({\sx*(1.1200)},{\sy*(0.2600)})
	--({\sx*(1.1300)},{\sy*(0.1702)})
	--({\sx*(1.1400)},{\sy*(0.0782)})
	--({\sx*(1.1500)},{\sy*(-0.0150)})
	--({\sx*(1.1600)},{\sy*(-0.1082)})
	--({\sx*(1.1700)},{\sy*(-0.2005)})
	--({\sx*(1.1800)},{\sy*(-0.2909)})
	--({\sx*(1.1900)},{\sy*(-0.3785)})
	--({\sx*(1.2000)},{\sy*(-0.4622)})
	--({\sx*(1.2100)},{\sy*(-0.5412)})
	--({\sx*(1.2200)},{\sy*(-0.6146)})
	--({\sx*(1.2300)},{\sy*(-0.6818)})
	--({\sx*(1.2400)},{\sy*(-0.7420)})
	--({\sx*(1.2500)},{\sy*(-0.7946)})
	--({\sx*(1.2600)},{\sy*(-0.8390)})
	--({\sx*(1.2700)},{\sy*(-0.8750)})
	--({\sx*(1.2800)},{\sy*(-0.9020)})
	--({\sx*(1.2900)},{\sy*(-0.9199)})
	--({\sx*(1.3000)},{\sy*(-0.9284)})
	--({\sx*(1.3100)},{\sy*(-0.9276)})
	--({\sx*(1.3200)},{\sy*(-0.9175)})
	--({\sx*(1.3300)},{\sy*(-0.8982)})
	--({\sx*(1.3400)},{\sy*(-0.8698)})
	--({\sx*(1.3500)},{\sy*(-0.8329)})
	--({\sx*(1.3600)},{\sy*(-0.7876)})
	--({\sx*(1.3700)},{\sy*(-0.7345)})
	--({\sx*(1.3800)},{\sy*(-0.6742)})
	--({\sx*(1.3900)},{\sy*(-0.6071)})
	--({\sx*(1.4000)},{\sy*(-0.5341)})
	--({\sx*(1.4100)},{\sy*(-0.4559)})
	--({\sx*(1.4200)},{\sy*(-0.3731)})
	--({\sx*(1.4300)},{\sy*(-0.2866)})
	--({\sx*(1.4400)},{\sy*(-0.1973)})
	--({\sx*(1.4500)},{\sy*(-0.1060)})
	--({\sx*(1.4600)},{\sy*(-0.0136)})
	--({\sx*(1.4700)},{\sy*(0.0791)})
	--({\sx*(1.4800)},{\sy*(0.1712)})
	--({\sx*(1.4900)},{\sy*(0.2618)})
	--({\sx*(1.5000)},{\sy*(0.3500)})
	--({\sx*(1.5100)},{\sy*(0.4352)})
	--({\sx*(1.5200)},{\sy*(0.5164)})
	--({\sx*(1.5300)},{\sy*(0.5929)})
	--({\sx*(1.5400)},{\sy*(0.6641)})
	--({\sx*(1.5500)},{\sy*(0.7293)})
	--({\sx*(1.5600)},{\sy*(0.7880)})
	--({\sx*(1.5700)},{\sy*(0.8396)})
	--({\sx*(1.5800)},{\sy*(0.8836)})
	--({\sx*(1.5900)},{\sy*(0.9197)})
	--({\sx*(1.6000)},{\sy*(0.9477)})
	--({\sx*(1.6100)},{\sy*(0.9672)})
	--({\sx*(1.6200)},{\sy*(0.9781)})
	--({\sx*(1.6300)},{\sy*(0.9803)})
	--({\sx*(1.6400)},{\sy*(0.9739)})
	--({\sx*(1.6500)},{\sy*(0.9589)})
	--({\sx*(1.6600)},{\sy*(0.9354)})
	--({\sx*(1.6700)},{\sy*(0.9038)})
	--({\sx*(1.6800)},{\sy*(0.8642)})
	--({\sx*(1.6900)},{\sy*(0.8171)})
	--({\sx*(1.7000)},{\sy*(0.7630)})
	--({\sx*(1.7100)},{\sy*(0.7021)})
	--({\sx*(1.7200)},{\sy*(0.6353)})
	--({\sx*(1.7300)},{\sy*(0.5629)})
	--({\sx*(1.7400)},{\sy*(0.4857)})
	--({\sx*(1.7500)},{\sy*(0.4043)})
	--({\sx*(1.7600)},{\sy*(0.3195)})
	--({\sx*(1.7700)},{\sy*(0.2319)})
	--({\sx*(1.7800)},{\sy*(0.1424)})
	--({\sx*(1.7900)},{\sy*(0.0517)})
	--({\sx*(1.8000)},{\sy*(-0.0395)})
	--({\sx*(1.8100)},{\sy*(-0.1303)})
	--({\sx*(1.8200)},{\sy*(-0.2200)})
	--({\sx*(1.8300)},{\sy*(-0.3079)})
	--({\sx*(1.8400)},{\sy*(-0.3932)})
	--({\sx*(1.8500)},{\sy*(-0.4751)})
	--({\sx*(1.8600)},{\sy*(-0.5531)})
	--({\sx*(1.8700)},{\sy*(-0.6265)})
	--({\sx*(1.8800)},{\sy*(-0.6946)})
	--({\sx*(1.8900)},{\sy*(-0.7570)})
	--({\sx*(1.9000)},{\sy*(-0.8131)})
	--({\sx*(1.9100)},{\sy*(-0.8625)})
	--({\sx*(1.9200)},{\sy*(-0.9047)})
	--({\sx*(1.9300)},{\sy*(-0.9396)})
	--({\sx*(1.9400)},{\sy*(-0.9667)})
	--({\sx*(1.9500)},{\sy*(-0.9859)})
	--({\sx*(1.9600)},{\sy*(-0.9970)})
	--({\sx*(1.9700)},{\sy*(-1.0000)})
	--({\sx*(1.9800)},{\sy*(-0.9949)})
	--({\sx*(1.9900)},{\sy*(-0.9817)})
	--({\sx*(2.0000)},{\sy*(-0.9606)})
	--({\sx*(2.0100)},{\sy*(-0.9317)})
	--({\sx*(2.0200)},{\sy*(-0.8953)})
	--({\sx*(2.0300)},{\sy*(-0.8518)})
	--({\sx*(2.0400)},{\sy*(-0.8014)})
	--({\sx*(2.0500)},{\sy*(-0.7447)})
	--({\sx*(2.0600)},{\sy*(-0.6820)})
	--({\sx*(2.0700)},{\sy*(-0.6139)})
	--({\sx*(2.0800)},{\sy*(-0.5410)})
	--({\sx*(2.0900)},{\sy*(-0.4639)})
	--({\sx*(2.1000)},{\sy*(-0.3831)})
	--({\sx*(2.1100)},{\sy*(-0.2994)})
	--({\sx*(2.1200)},{\sy*(-0.2134)})
	--({\sx*(2.1300)},{\sy*(-0.1258)})
	--({\sx*(2.1400)},{\sy*(-0.0374)})
	--({\sx*(2.1500)},{\sy*(0.0513)})
	--({\sx*(2.1600)},{\sy*(0.1394)})
	--({\sx*(2.1700)},{\sy*(0.2263)})
	--({\sx*(2.1800)},{\sy*(0.3113)})
	--({\sx*(2.1900)},{\sy*(0.3938)})
	--({\sx*(2.2000)},{\sy*(0.4730)})
	--({\sx*(2.2100)},{\sy*(0.5483)})
	--({\sx*(2.2200)},{\sy*(0.6193)})
	--({\sx*(2.2300)},{\sy*(0.6852)})
	--({\sx*(2.2400)},{\sy*(0.7457)})
	--({\sx*(2.2500)},{\sy*(0.8003)})
	--({\sx*(2.2600)},{\sy*(0.8484)})
	--({\sx*(2.2700)},{\sy*(0.8899)})
	--({\sx*(2.2800)},{\sy*(0.9243)})
	--({\sx*(2.2900)},{\sy*(0.9514)})
	--({\sx*(2.3000)},{\sy*(0.9711)})
	--({\sx*(2.3100)},{\sy*(0.9831)})
	--({\sx*(2.3200)},{\sy*(0.9875)})
	--({\sx*(2.3300)},{\sy*(0.9841)})
	--({\sx*(2.3400)},{\sy*(0.9731)})
	--({\sx*(2.3500)},{\sy*(0.9545)})
	--({\sx*(2.3600)},{\sy*(0.9285)})
	--({\sx*(2.3700)},{\sy*(0.8954)})
	--({\sx*(2.3800)},{\sy*(0.8553)})
	--({\sx*(2.3900)},{\sy*(0.8087)})
	--({\sx*(2.4000)},{\sy*(0.7559)})
	--({\sx*(2.4100)},{\sy*(0.6974)})
	--({\sx*(2.4200)},{\sy*(0.6335)})
	--({\sx*(2.4300)},{\sy*(0.5649)})
	--({\sx*(2.4400)},{\sy*(0.4920)})
	--({\sx*(2.4500)},{\sy*(0.4155)})
	--({\sx*(2.4600)},{\sy*(0.3360)})
	--({\sx*(2.4700)},{\sy*(0.2540)})
	--({\sx*(2.4800)},{\sy*(0.1702)})
	--({\sx*(2.4900)},{\sy*(0.0853)})
	--({\sx*(2.5000)},{\sy*(-0.0000)})
	--({\sx*(2.5100)},{\sy*(-0.0851)})
	--({\sx*(2.5200)},{\sy*(-0.1694)})
	--({\sx*(2.5300)},{\sy*(-0.2521)})
	--({\sx*(2.5400)},{\sy*(-0.3327)})
	--({\sx*(2.5500)},{\sy*(-0.4105)})
	--({\sx*(2.5600)},{\sy*(-0.4849)})
	--({\sx*(2.5700)},{\sy*(-0.5554)})
	--({\sx*(2.5800)},{\sy*(-0.6213)})
	--({\sx*(2.5900)},{\sy*(-0.6823)})
	--({\sx*(2.6000)},{\sy*(-0.7378)})
	--({\sx*(2.6100)},{\sy*(-0.7874)})
	--({\sx*(2.6200)},{\sy*(-0.8308)})
	--({\sx*(2.6300)},{\sy*(-0.8675)})
	--({\sx*(2.6400)},{\sy*(-0.8975)})
	--({\sx*(2.6500)},{\sy*(-0.9203)})
	--({\sx*(2.6600)},{\sy*(-0.9360)})
	--({\sx*(2.6700)},{\sy*(-0.9443)})
	--({\sx*(2.6800)},{\sy*(-0.9452)})
	--({\sx*(2.6900)},{\sy*(-0.9388)})
	--({\sx*(2.7000)},{\sy*(-0.9251)})
	--({\sx*(2.7100)},{\sy*(-0.9042)})
	--({\sx*(2.7200)},{\sy*(-0.8762)})
	--({\sx*(2.7300)},{\sy*(-0.8416)})
	--({\sx*(2.7400)},{\sy*(-0.8004)})
	--({\sx*(2.7500)},{\sy*(-0.7532)})
	--({\sx*(2.7600)},{\sy*(-0.7002)})
	--({\sx*(2.7700)},{\sy*(-0.6418)})
	--({\sx*(2.7800)},{\sy*(-0.5786)})
	--({\sx*(2.7900)},{\sy*(-0.5111)})
	--({\sx*(2.8000)},{\sy*(-0.4398)})
	--({\sx*(2.8100)},{\sy*(-0.3653)})
	--({\sx*(2.8200)},{\sy*(-0.2881)})
	--({\sx*(2.8300)},{\sy*(-0.2089)})
	--({\sx*(2.8400)},{\sy*(-0.1284)})
	--({\sx*(2.8500)},{\sy*(-0.0471)})
	--({\sx*(2.8600)},{\sy*(0.0343)})
	--({\sx*(2.8700)},{\sy*(0.1151)})
	--({\sx*(2.8800)},{\sy*(0.1947)})
	--({\sx*(2.8900)},{\sy*(0.2725)})
	--({\sx*(2.9000)},{\sy*(0.3478)})
	--({\sx*(2.9100)},{\sy*(0.4201)})
	--({\sx*(2.9200)},{\sy*(0.4888)})
	--({\sx*(2.9300)},{\sy*(0.5534)})
	--({\sx*(2.9400)},{\sy*(0.6133)})
	--({\sx*(2.9500)},{\sy*(0.6680)})
	--({\sx*(2.9600)},{\sy*(0.7172)})
	--({\sx*(2.9700)},{\sy*(0.7605)})
	--({\sx*(2.9800)},{\sy*(0.7975)})
	--({\sx*(2.9900)},{\sy*(0.8279)})
	--({\sx*(3.0000)},{\sy*(0.8515)})
	--({\sx*(3.0100)},{\sy*(0.8682)})
	--({\sx*(3.0200)},{\sy*(0.8778)})
	--({\sx*(3.0300)},{\sy*(0.8802)})
	--({\sx*(3.0400)},{\sy*(0.8755)})
	--({\sx*(3.0500)},{\sy*(0.8637)})
	--({\sx*(3.0600)},{\sy*(0.8449)})
	--({\sx*(3.0700)},{\sy*(0.8192)})
	--({\sx*(3.0800)},{\sy*(0.7870)})
	--({\sx*(3.0900)},{\sy*(0.7485)})
	--({\sx*(3.1000)},{\sy*(0.7040)})
	--({\sx*(3.1100)},{\sy*(0.6539)})
	--({\sx*(3.1200)},{\sy*(0.5986)})
	--({\sx*(3.1300)},{\sy*(0.5386)})
	--({\sx*(3.1400)},{\sy*(0.4744)})
	--({\sx*(3.1500)},{\sy*(0.4066)})
	--({\sx*(3.1600)},{\sy*(0.3357)})
	--({\sx*(3.1700)},{\sy*(0.2623)})
	--({\sx*(3.1800)},{\sy*(0.1870)})
	--({\sx*(3.1900)},{\sy*(0.1105)})
	--({\sx*(3.2000)},{\sy*(0.0334)})
	--({\sx*(3.2100)},{\sy*(-0.0436)})
	--({\sx*(3.2200)},{\sy*(-0.1199)})
	--({\sx*(3.2300)},{\sy*(-0.1948)})
	--({\sx*(3.2400)},{\sy*(-0.2678)})
	--({\sx*(3.2500)},{\sy*(-0.3381)})
	--({\sx*(3.2600)},{\sy*(-0.4052)})
	--({\sx*(3.2700)},{\sy*(-0.4686)})
	--({\sx*(3.2800)},{\sy*(-0.5276)})
	--({\sx*(3.2900)},{\sy*(-0.5818)})
	--({\sx*(3.3000)},{\sy*(-0.6308)})
	--({\sx*(3.3100)},{\sy*(-0.6740)})
	--({\sx*(3.3200)},{\sy*(-0.7113)})
	--({\sx*(3.3300)},{\sy*(-0.7421)})
	--({\sx*(3.3400)},{\sy*(-0.7664)})
	--({\sx*(3.3500)},{\sy*(-0.7838)})
	--({\sx*(3.3600)},{\sy*(-0.7943)})
	--({\sx*(3.3700)},{\sy*(-0.7977)})
	--({\sx*(3.3800)},{\sy*(-0.7941)})
	--({\sx*(3.3900)},{\sy*(-0.7835)})
	--({\sx*(3.4000)},{\sy*(-0.7660)})
	--({\sx*(3.4100)},{\sy*(-0.7418)})
	--({\sx*(3.4200)},{\sy*(-0.7111)})
	--({\sx*(3.4300)},{\sy*(-0.6741)})
	--({\sx*(3.4400)},{\sy*(-0.6313)})
	--({\sx*(3.4500)},{\sy*(-0.5830)})
	--({\sx*(3.4600)},{\sy*(-0.5297)})
	--({\sx*(3.4700)},{\sy*(-0.4719)})
	--({\sx*(3.4800)},{\sy*(-0.4101)})
	--({\sx*(3.4900)},{\sy*(-0.3448)})
	--({\sx*(3.5000)},{\sy*(-0.2768)})
	--({\sx*(3.5100)},{\sy*(-0.2065)})
	--({\sx*(3.5200)},{\sy*(-0.1348)})
	--({\sx*(3.5300)},{\sy*(-0.0622)})
	--({\sx*(3.5400)},{\sy*(0.0106)})
	--({\sx*(3.5500)},{\sy*(0.0829)})
	--({\sx*(3.5600)},{\sy*(0.1540)})
	--({\sx*(3.5700)},{\sy*(0.2232)})
	--({\sx*(3.5800)},{\sy*(0.2899)})
	--({\sx*(3.5900)},{\sy*(0.3535)})
	--({\sx*(3.6000)},{\sy*(0.4133)})
	--({\sx*(3.6100)},{\sy*(0.4687)})
	--({\sx*(3.6200)},{\sy*(0.5194)})
	--({\sx*(3.6300)},{\sy*(0.5647)})
	--({\sx*(3.6400)},{\sy*(0.6042)})
	--({\sx*(3.6500)},{\sy*(0.6376)})
	--({\sx*(3.6600)},{\sy*(0.6645)})
	--({\sx*(3.6700)},{\sy*(0.6847)})
	--({\sx*(3.6800)},{\sy*(0.6979)})
	--({\sx*(3.6900)},{\sy*(0.7042)})
	--({\sx*(3.7000)},{\sy*(0.7033)})
	--({\sx*(3.7100)},{\sy*(0.6954)})
	--({\sx*(3.7200)},{\sy*(0.6804)})
	--({\sx*(3.7300)},{\sy*(0.6587)})
	--({\sx*(3.7400)},{\sy*(0.6303)})
	--({\sx*(3.7500)},{\sy*(0.5957)})
	--({\sx*(3.7600)},{\sy*(0.5551)})
	--({\sx*(3.7700)},{\sy*(0.5090)})
	--({\sx*(3.7800)},{\sy*(0.4580)})
	--({\sx*(3.7900)},{\sy*(0.4024)})
	--({\sx*(3.8000)},{\sy*(0.3430)})
	--({\sx*(3.8100)},{\sy*(0.2803)})
	--({\sx*(3.8200)},{\sy*(0.2150)})
	--({\sx*(3.8300)},{\sy*(0.1479)})
	--({\sx*(3.8400)},{\sy*(0.0796)})
	--({\sx*(3.8500)},{\sy*(0.0110)})
	--({\sx*(3.8600)},{\sy*(-0.0573)})
	--({\sx*(3.8700)},{\sy*(-0.1245)})
	--({\sx*(3.8800)},{\sy*(-0.1899)})
	--({\sx*(3.8900)},{\sy*(-0.2526)})
	--({\sx*(3.9000)},{\sy*(-0.3121)})
	--({\sx*(3.9100)},{\sy*(-0.3676)})
	--({\sx*(3.9200)},{\sy*(-0.4186)})
	--({\sx*(3.9300)},{\sy*(-0.4644)})
	--({\sx*(3.9400)},{\sy*(-0.5045)})
	--({\sx*(3.9500)},{\sy*(-0.5385)})
	--({\sx*(3.9600)},{\sy*(-0.5659)})
	--({\sx*(3.9700)},{\sy*(-0.5865)})
	--({\sx*(3.9800)},{\sy*(-0.5999)})
	--({\sx*(3.9900)},{\sy*(-0.6061)})
	--({\sx*(4.0000)},{\sy*(-0.6050)})
	--({\sx*(4.0100)},{\sy*(-0.5966)})
	--({\sx*(4.0200)},{\sy*(-0.5809)})
	--({\sx*(4.0300)},{\sy*(-0.5582)})
	--({\sx*(4.0400)},{\sy*(-0.5288)})
	--({\sx*(4.0500)},{\sy*(-0.4930)})
	--({\sx*(4.0600)},{\sy*(-0.4513)})
	--({\sx*(4.0700)},{\sy*(-0.4042)})
	--({\sx*(4.0800)},{\sy*(-0.3523)})
	--({\sx*(4.0900)},{\sy*(-0.2963)})
	--({\sx*(4.1000)},{\sy*(-0.2369)})
	--({\sx*(4.1100)},{\sy*(-0.1749)})
	--({\sx*(4.1200)},{\sy*(-0.1111)})
	--({\sx*(4.1300)},{\sy*(-0.0464)})
	--({\sx*(4.1400)},{\sy*(0.0184)})
	--({\sx*(4.1500)},{\sy*(0.0824)})
	--({\sx*(4.1600)},{\sy*(0.1447)})
	--({\sx*(4.1700)},{\sy*(0.2045)})
	--({\sx*(4.1800)},{\sy*(0.2608)})
	--({\sx*(4.1900)},{\sy*(0.3130)})
	--({\sx*(4.2000)},{\sy*(0.3602)})
	--({\sx*(4.2100)},{\sy*(0.4018)})
	--({\sx*(4.2200)},{\sy*(0.4372)})
	--({\sx*(4.2300)},{\sy*(0.4658)})
	--({\sx*(4.2400)},{\sy*(0.4872)})
	--({\sx*(4.2500)},{\sy*(0.5010)})
	--({\sx*(4.2600)},{\sy*(0.5072)})
	--({\sx*(4.2700)},{\sy*(0.5054)})
	--({\sx*(4.2800)},{\sy*(0.4958)})
	--({\sx*(4.2900)},{\sy*(0.4786)})
	--({\sx*(4.3000)},{\sy*(0.4538)})
	--({\sx*(4.3100)},{\sy*(0.4221)})
	--({\sx*(4.3200)},{\sy*(0.3837)})
	--({\sx*(4.3300)},{\sy*(0.3395)})
	--({\sx*(4.3400)},{\sy*(0.2900)})
	--({\sx*(4.3500)},{\sy*(0.2361)})
	--({\sx*(4.3600)},{\sy*(0.1788)})
	--({\sx*(4.3700)},{\sy*(0.1190)})
	--({\sx*(4.3800)},{\sy*(0.0578)})
	--({\sx*(4.3900)},{\sy*(-0.0038)})
	--({\sx*(4.4000)},{\sy*(-0.0647)})
	--({\sx*(4.4100)},{\sy*(-0.1236)})
	--({\sx*(4.4200)},{\sy*(-0.1796)})
	--({\sx*(4.4300)},{\sy*(-0.2316)})
	--({\sx*(4.4400)},{\sy*(-0.2784)})
	--({\sx*(4.4500)},{\sy*(-0.3193)})
	--({\sx*(4.4600)},{\sy*(-0.3533)})
	--({\sx*(4.4700)},{\sy*(-0.3799)})
	--({\sx*(4.4800)},{\sy*(-0.3983)})
	--({\sx*(4.4900)},{\sy*(-0.4081)})
	--({\sx*(4.5000)},{\sy*(-0.4093)})
	--({\sx*(4.5100)},{\sy*(-0.4015)})
	--({\sx*(4.5200)},{\sy*(-0.3851)})
	--({\sx*(4.5300)},{\sy*(-0.3603)})
	--({\sx*(4.5400)},{\sy*(-0.3276)})
	--({\sx*(4.5500)},{\sy*(-0.2877)})
	--({\sx*(4.5600)},{\sy*(-0.2416)})
	--({\sx*(4.5700)},{\sy*(-0.1903)})
	--({\sx*(4.5800)},{\sy*(-0.1351)})
	--({\sx*(4.5900)},{\sy*(-0.0773)})
	--({\sx*(4.6000)},{\sy*(-0.0184)})
	--({\sx*(4.6100)},{\sy*(0.0400)})
	--({\sx*(4.6200)},{\sy*(0.0965)})
	--({\sx*(4.6300)},{\sy*(0.1493)})
	--({\sx*(4.6400)},{\sy*(0.1971)})
	--({\sx*(4.6500)},{\sy*(0.2383)})
	--({\sx*(4.6600)},{\sy*(0.2716)})
	--({\sx*(4.6700)},{\sy*(0.2960)})
	--({\sx*(4.6800)},{\sy*(0.3107)})
	--({\sx*(4.6900)},{\sy*(0.3150)})
	--({\sx*(4.7000)},{\sy*(0.3087)})
	--({\sx*(4.7100)},{\sy*(0.2918)})
	--({\sx*(4.7200)},{\sy*(0.2650)})
	--({\sx*(4.7300)},{\sy*(0.2289)})
	--({\sx*(4.7400)},{\sy*(0.1849)})
	--({\sx*(4.7500)},{\sy*(0.1346)})
	--({\sx*(4.7600)},{\sy*(0.0799)})
	--({\sx*(4.7700)},{\sy*(0.0232)})
	--({\sx*(4.7800)},{\sy*(-0.0332)})
	--({\sx*(4.7900)},{\sy*(-0.0865)})
	--({\sx*(4.8000)},{\sy*(-0.1342)})
	--({\sx*(4.8100)},{\sy*(-0.1738)})
	--({\sx*(4.8200)},{\sy*(-0.2029)})
	--({\sx*(4.8300)},{\sy*(-0.2196)})
	--({\sx*(4.8400)},{\sy*(-0.2228)})
	--({\sx*(4.8500)},{\sy*(-0.2118)})
	--({\sx*(4.8600)},{\sy*(-0.1870)})
	--({\sx*(4.8700)},{\sy*(-0.1497)})
	--({\sx*(4.8800)},{\sy*(-0.1023)})
	--({\sx*(4.8900)},{\sy*(-0.0486)})
	--({\sx*(4.9000)},{\sy*(0.0070)})
	--({\sx*(4.9100)},{\sy*(0.0588)})
	--({\sx*(4.9200)},{\sy*(0.1010)})
	--({\sx*(4.9300)},{\sy*(0.1278)})
	--({\sx*(4.9400)},{\sy*(0.1342)})
	--({\sx*(4.9500)},{\sy*(0.1175)})
	--({\sx*(4.9600)},{\sy*(0.0789)})
	--({\sx*(4.9700)},{\sy*(0.0254)})
	--({\sx*(4.9800)},{\sy*(-0.0271)})
	--({\sx*(4.9900)},{\sy*(-0.0505)})
	--({\sx*(5.0000)},{\sy*(0.0000)});
}
\def\relfehlerk{
\draw[color=blue,line width=1.4pt,line join=round] ({\sx*(0.000)},{\sy*(0.0000)})
	--({\sx*(0.0100)},{\sy*(0.0000)})
	--({\sx*(0.0200)},{\sy*(0.0000)})
	--({\sx*(0.0300)},{\sy*(-0.0000)})
	--({\sx*(0.0400)},{\sy*(-0.0000)})
	--({\sx*(0.0500)},{\sy*(-0.0000)})
	--({\sx*(0.0600)},{\sy*(-0.0000)})
	--({\sx*(0.0700)},{\sy*(-0.0000)})
	--({\sx*(0.0800)},{\sy*(-0.0000)})
	--({\sx*(0.0900)},{\sy*(-0.0000)})
	--({\sx*(0.1000)},{\sy*(-0.0000)})
	--({\sx*(0.1100)},{\sy*(0.0000)})
	--({\sx*(0.1200)},{\sy*(0.0000)})
	--({\sx*(0.1300)},{\sy*(0.0000)})
	--({\sx*(0.1400)},{\sy*(0.0000)})
	--({\sx*(0.1500)},{\sy*(0.0000)})
	--({\sx*(0.1600)},{\sy*(0.0000)})
	--({\sx*(0.1700)},{\sy*(0.0000)})
	--({\sx*(0.1800)},{\sy*(0.0000)})
	--({\sx*(0.1900)},{\sy*(0.0000)})
	--({\sx*(0.2000)},{\sy*(0.0000)})
	--({\sx*(0.2100)},{\sy*(0.0000)})
	--({\sx*(0.2200)},{\sy*(0.0000)})
	--({\sx*(0.2300)},{\sy*(-0.0000)})
	--({\sx*(0.2400)},{\sy*(-0.0000)})
	--({\sx*(0.2500)},{\sy*(-0.0000)})
	--({\sx*(0.2600)},{\sy*(-0.0000)})
	--({\sx*(0.2700)},{\sy*(-0.0000)})
	--({\sx*(0.2800)},{\sy*(-0.0000)})
	--({\sx*(0.2900)},{\sy*(-0.0000)})
	--({\sx*(0.3000)},{\sy*(-0.0000)})
	--({\sx*(0.3100)},{\sy*(-0.0000)})
	--({\sx*(0.3200)},{\sy*(-0.0000)})
	--({\sx*(0.3300)},{\sy*(-0.0000)})
	--({\sx*(0.3400)},{\sy*(-0.0000)})
	--({\sx*(0.3500)},{\sy*(-0.0000)})
	--({\sx*(0.3600)},{\sy*(-0.0000)})
	--({\sx*(0.3700)},{\sy*(-0.0000)})
	--({\sx*(0.3800)},{\sy*(-0.0000)})
	--({\sx*(0.3900)},{\sy*(-0.0000)})
	--({\sx*(0.4000)},{\sy*(0.0000)})
	--({\sx*(0.4100)},{\sy*(0.0000)})
	--({\sx*(0.4200)},{\sy*(0.0000)})
	--({\sx*(0.4300)},{\sy*(0.0000)})
	--({\sx*(0.4400)},{\sy*(0.0000)})
	--({\sx*(0.4500)},{\sy*(0.0000)})
	--({\sx*(0.4600)},{\sy*(0.0000)})
	--({\sx*(0.4700)},{\sy*(0.0000)})
	--({\sx*(0.4800)},{\sy*(0.0000)})
	--({\sx*(0.4900)},{\sy*(0.0000)})
	--({\sx*(0.5000)},{\sy*(0.0000)})
	--({\sx*(0.5100)},{\sy*(0.0000)})
	--({\sx*(0.5200)},{\sy*(0.0000)})
	--({\sx*(0.5300)},{\sy*(0.0000)})
	--({\sx*(0.5400)},{\sy*(0.0000)})
	--({\sx*(0.5500)},{\sy*(0.0000)})
	--({\sx*(0.5600)},{\sy*(0.0000)})
	--({\sx*(0.5700)},{\sy*(0.0000)})
	--({\sx*(0.5800)},{\sy*(0.0000)})
	--({\sx*(0.5900)},{\sy*(0.0000)})
	--({\sx*(0.6000)},{\sy*(0.0000)})
	--({\sx*(0.6100)},{\sy*(0.0000)})
	--({\sx*(0.6200)},{\sy*(-0.0000)})
	--({\sx*(0.6300)},{\sy*(-0.0000)})
	--({\sx*(0.6400)},{\sy*(-0.0000)})
	--({\sx*(0.6500)},{\sy*(-0.0000)})
	--({\sx*(0.6600)},{\sy*(-0.0000)})
	--({\sx*(0.6700)},{\sy*(-0.0000)})
	--({\sx*(0.6800)},{\sy*(-0.0000)})
	--({\sx*(0.6900)},{\sy*(-0.0000)})
	--({\sx*(0.7000)},{\sy*(-0.0000)})
	--({\sx*(0.7100)},{\sy*(-0.0000)})
	--({\sx*(0.7200)},{\sy*(-0.0000)})
	--({\sx*(0.7300)},{\sy*(-0.0000)})
	--({\sx*(0.7400)},{\sy*(-0.0000)})
	--({\sx*(0.7500)},{\sy*(-0.0000)})
	--({\sx*(0.7600)},{\sy*(-0.0000)})
	--({\sx*(0.7700)},{\sy*(-0.0000)})
	--({\sx*(0.7800)},{\sy*(-0.0000)})
	--({\sx*(0.7900)},{\sy*(-0.0000)})
	--({\sx*(0.8000)},{\sy*(-0.0000)})
	--({\sx*(0.8100)},{\sy*(-0.0000)})
	--({\sx*(0.8200)},{\sy*(-0.0000)})
	--({\sx*(0.8300)},{\sy*(-0.0000)})
	--({\sx*(0.8400)},{\sy*(-0.0000)})
	--({\sx*(0.8500)},{\sy*(-0.0000)})
	--({\sx*(0.8600)},{\sy*(-0.0000)})
	--({\sx*(0.8700)},{\sy*(0.0000)})
	--({\sx*(0.8800)},{\sy*(0.0000)})
	--({\sx*(0.8900)},{\sy*(0.0000)})
	--({\sx*(0.9000)},{\sy*(0.0000)})
	--({\sx*(0.9100)},{\sy*(0.0000)})
	--({\sx*(0.9200)},{\sy*(0.0000)})
	--({\sx*(0.9300)},{\sy*(0.0000)})
	--({\sx*(0.9400)},{\sy*(0.0000)})
	--({\sx*(0.9500)},{\sy*(0.0000)})
	--({\sx*(0.9600)},{\sy*(0.0000)})
	--({\sx*(0.9700)},{\sy*(0.0000)})
	--({\sx*(0.9800)},{\sy*(0.0000)})
	--({\sx*(0.9900)},{\sy*(0.0000)})
	--({\sx*(1.0000)},{\sy*(0.0000)})
	--({\sx*(1.0100)},{\sy*(0.0000)})
	--({\sx*(1.0200)},{\sy*(0.0000)})
	--({\sx*(1.0300)},{\sy*(0.0000)})
	--({\sx*(1.0400)},{\sy*(0.0000)})
	--({\sx*(1.0500)},{\sy*(0.0000)})
	--({\sx*(1.0600)},{\sy*(0.0000)})
	--({\sx*(1.0700)},{\sy*(0.0000)})
	--({\sx*(1.0800)},{\sy*(0.0000)})
	--({\sx*(1.0900)},{\sy*(0.0000)})
	--({\sx*(1.1000)},{\sy*(0.0000)})
	--({\sx*(1.1100)},{\sy*(0.0000)})
	--({\sx*(1.1200)},{\sy*(0.0000)})
	--({\sx*(1.1300)},{\sy*(0.0000)})
	--({\sx*(1.1400)},{\sy*(0.0000)})
	--({\sx*(1.1500)},{\sy*(-0.0000)})
	--({\sx*(1.1600)},{\sy*(-0.0000)})
	--({\sx*(1.1700)},{\sy*(-0.0000)})
	--({\sx*(1.1800)},{\sy*(-0.0000)})
	--({\sx*(1.1900)},{\sy*(-0.0000)})
	--({\sx*(1.2000)},{\sy*(-0.0000)})
	--({\sx*(1.2100)},{\sy*(-0.0000)})
	--({\sx*(1.2200)},{\sy*(-0.0000)})
	--({\sx*(1.2300)},{\sy*(-0.0000)})
	--({\sx*(1.2400)},{\sy*(-0.0000)})
	--({\sx*(1.2500)},{\sy*(-0.0000)})
	--({\sx*(1.2600)},{\sy*(-0.0000)})
	--({\sx*(1.2700)},{\sy*(-0.0000)})
	--({\sx*(1.2800)},{\sy*(-0.0000)})
	--({\sx*(1.2900)},{\sy*(-0.0000)})
	--({\sx*(1.3000)},{\sy*(-0.0000)})
	--({\sx*(1.3100)},{\sy*(-0.0000)})
	--({\sx*(1.3200)},{\sy*(-0.0000)})
	--({\sx*(1.3300)},{\sy*(-0.0000)})
	--({\sx*(1.3400)},{\sy*(-0.0000)})
	--({\sx*(1.3500)},{\sy*(-0.0000)})
	--({\sx*(1.3600)},{\sy*(-0.0000)})
	--({\sx*(1.3700)},{\sy*(-0.0000)})
	--({\sx*(1.3800)},{\sy*(-0.0000)})
	--({\sx*(1.3900)},{\sy*(-0.0000)})
	--({\sx*(1.4000)},{\sy*(-0.0000)})
	--({\sx*(1.4100)},{\sy*(-0.0000)})
	--({\sx*(1.4200)},{\sy*(-0.0000)})
	--({\sx*(1.4300)},{\sy*(-0.0000)})
	--({\sx*(1.4400)},{\sy*(-0.0000)})
	--({\sx*(1.4500)},{\sy*(-0.0000)})
	--({\sx*(1.4600)},{\sy*(-0.0000)})
	--({\sx*(1.4700)},{\sy*(0.0000)})
	--({\sx*(1.4800)},{\sy*(0.0000)})
	--({\sx*(1.4900)},{\sy*(0.0000)})
	--({\sx*(1.5000)},{\sy*(0.0000)})
	--({\sx*(1.5100)},{\sy*(0.0000)})
	--({\sx*(1.5200)},{\sy*(0.0000)})
	--({\sx*(1.5300)},{\sy*(0.0000)})
	--({\sx*(1.5400)},{\sy*(0.0000)})
	--({\sx*(1.5500)},{\sy*(0.0000)})
	--({\sx*(1.5600)},{\sy*(0.0000)})
	--({\sx*(1.5700)},{\sy*(0.0000)})
	--({\sx*(1.5800)},{\sy*(0.0000)})
	--({\sx*(1.5900)},{\sy*(0.0000)})
	--({\sx*(1.6000)},{\sy*(0.0000)})
	--({\sx*(1.6100)},{\sy*(0.0000)})
	--({\sx*(1.6200)},{\sy*(0.0000)})
	--({\sx*(1.6300)},{\sy*(0.0000)})
	--({\sx*(1.6400)},{\sy*(0.0000)})
	--({\sx*(1.6500)},{\sy*(0.0000)})
	--({\sx*(1.6600)},{\sy*(0.0000)})
	--({\sx*(1.6700)},{\sy*(0.0000)})
	--({\sx*(1.6800)},{\sy*(0.0000)})
	--({\sx*(1.6900)},{\sy*(0.0000)})
	--({\sx*(1.7000)},{\sy*(0.0000)})
	--({\sx*(1.7100)},{\sy*(0.0000)})
	--({\sx*(1.7200)},{\sy*(0.0000)})
	--({\sx*(1.7300)},{\sy*(0.0000)})
	--({\sx*(1.7400)},{\sy*(0.0000)})
	--({\sx*(1.7500)},{\sy*(0.0000)})
	--({\sx*(1.7600)},{\sy*(0.0000)})
	--({\sx*(1.7700)},{\sy*(0.0000)})
	--({\sx*(1.7800)},{\sy*(0.0000)})
	--({\sx*(1.7900)},{\sy*(0.0000)})
	--({\sx*(1.8000)},{\sy*(-0.0000)})
	--({\sx*(1.8100)},{\sy*(-0.0000)})
	--({\sx*(1.8200)},{\sy*(-0.0000)})
	--({\sx*(1.8300)},{\sy*(-0.0000)})
	--({\sx*(1.8400)},{\sy*(-0.0000)})
	--({\sx*(1.8500)},{\sy*(-0.0000)})
	--({\sx*(1.8600)},{\sy*(-0.0000)})
	--({\sx*(1.8700)},{\sy*(-0.0000)})
	--({\sx*(1.8800)},{\sy*(-0.0000)})
	--({\sx*(1.8900)},{\sy*(-0.0000)})
	--({\sx*(1.9000)},{\sy*(-0.0000)})
	--({\sx*(1.9100)},{\sy*(-0.0000)})
	--({\sx*(1.9200)},{\sy*(-0.0000)})
	--({\sx*(1.9300)},{\sy*(-0.0000)})
	--({\sx*(1.9400)},{\sy*(-0.0000)})
	--({\sx*(1.9500)},{\sy*(-0.0000)})
	--({\sx*(1.9600)},{\sy*(-0.0000)})
	--({\sx*(1.9700)},{\sy*(-0.0000)})
	--({\sx*(1.9800)},{\sy*(-0.0000)})
	--({\sx*(1.9900)},{\sy*(-0.0000)})
	--({\sx*(2.0000)},{\sy*(-0.0000)})
	--({\sx*(2.0100)},{\sy*(-0.0000)})
	--({\sx*(2.0200)},{\sy*(-0.0000)})
	--({\sx*(2.0300)},{\sy*(-0.0000)})
	--({\sx*(2.0400)},{\sy*(-0.0000)})
	--({\sx*(2.0500)},{\sy*(-0.0000)})
	--({\sx*(2.0600)},{\sy*(-0.0000)})
	--({\sx*(2.0700)},{\sy*(-0.0000)})
	--({\sx*(2.0800)},{\sy*(-0.0000)})
	--({\sx*(2.0900)},{\sy*(-0.0000)})
	--({\sx*(2.1000)},{\sy*(-0.0000)})
	--({\sx*(2.1100)},{\sy*(-0.0000)})
	--({\sx*(2.1200)},{\sy*(-0.0000)})
	--({\sx*(2.1300)},{\sy*(-0.0000)})
	--({\sx*(2.1400)},{\sy*(-0.0000)})
	--({\sx*(2.1500)},{\sy*(0.0000)})
	--({\sx*(2.1600)},{\sy*(0.0000)})
	--({\sx*(2.1700)},{\sy*(0.0000)})
	--({\sx*(2.1800)},{\sy*(0.0000)})
	--({\sx*(2.1900)},{\sy*(0.0000)})
	--({\sx*(2.2000)},{\sy*(0.0000)})
	--({\sx*(2.2100)},{\sy*(0.0000)})
	--({\sx*(2.2200)},{\sy*(0.0000)})
	--({\sx*(2.2300)},{\sy*(0.0000)})
	--({\sx*(2.2400)},{\sy*(0.0000)})
	--({\sx*(2.2500)},{\sy*(0.0000)})
	--({\sx*(2.2600)},{\sy*(0.0000)})
	--({\sx*(2.2700)},{\sy*(0.0000)})
	--({\sx*(2.2800)},{\sy*(0.0000)})
	--({\sx*(2.2900)},{\sy*(0.0000)})
	--({\sx*(2.3000)},{\sy*(0.0000)})
	--({\sx*(2.3100)},{\sy*(0.0000)})
	--({\sx*(2.3200)},{\sy*(0.0000)})
	--({\sx*(2.3300)},{\sy*(0.0000)})
	--({\sx*(2.3400)},{\sy*(0.0000)})
	--({\sx*(2.3500)},{\sy*(0.0000)})
	--({\sx*(2.3600)},{\sy*(0.0000)})
	--({\sx*(2.3700)},{\sy*(0.0000)})
	--({\sx*(2.3800)},{\sy*(0.0000)})
	--({\sx*(2.3900)},{\sy*(0.0000)})
	--({\sx*(2.4000)},{\sy*(0.0000)})
	--({\sx*(2.4100)},{\sy*(0.0000)})
	--({\sx*(2.4200)},{\sy*(0.0000)})
	--({\sx*(2.4300)},{\sy*(0.0000)})
	--({\sx*(2.4400)},{\sy*(0.0000)})
	--({\sx*(2.4500)},{\sy*(0.0000)})
	--({\sx*(2.4600)},{\sy*(0.0000)})
	--({\sx*(2.4700)},{\sy*(0.0000)})
	--({\sx*(2.4800)},{\sy*(0.0000)})
	--({\sx*(2.4900)},{\sy*(0.0000)})
	--({\sx*(2.5000)},{\sy*(-0.0000)})
	--({\sx*(2.5100)},{\sy*(-0.0000)})
	--({\sx*(2.5200)},{\sy*(-0.0000)})
	--({\sx*(2.5300)},{\sy*(-0.0000)})
	--({\sx*(2.5400)},{\sy*(-0.0000)})
	--({\sx*(2.5500)},{\sy*(-0.0000)})
	--({\sx*(2.5600)},{\sy*(-0.0000)})
	--({\sx*(2.5700)},{\sy*(-0.0000)})
	--({\sx*(2.5800)},{\sy*(-0.0000)})
	--({\sx*(2.5900)},{\sy*(-0.0000)})
	--({\sx*(2.6000)},{\sy*(-0.0000)})
	--({\sx*(2.6100)},{\sy*(-0.0000)})
	--({\sx*(2.6200)},{\sy*(-0.0000)})
	--({\sx*(2.6300)},{\sy*(-0.0000)})
	--({\sx*(2.6400)},{\sy*(-0.0000)})
	--({\sx*(2.6500)},{\sy*(-0.0000)})
	--({\sx*(2.6600)},{\sy*(-0.0000)})
	--({\sx*(2.6700)},{\sy*(-0.0000)})
	--({\sx*(2.6800)},{\sy*(-0.0000)})
	--({\sx*(2.6900)},{\sy*(-0.0000)})
	--({\sx*(2.7000)},{\sy*(-0.0000)})
	--({\sx*(2.7100)},{\sy*(-0.0000)})
	--({\sx*(2.7200)},{\sy*(-0.0000)})
	--({\sx*(2.7300)},{\sy*(-0.0000)})
	--({\sx*(2.7400)},{\sy*(-0.0000)})
	--({\sx*(2.7500)},{\sy*(-0.0000)})
	--({\sx*(2.7600)},{\sy*(-0.0000)})
	--({\sx*(2.7700)},{\sy*(-0.0000)})
	--({\sx*(2.7800)},{\sy*(-0.0000)})
	--({\sx*(2.7900)},{\sy*(-0.0000)})
	--({\sx*(2.8000)},{\sy*(-0.0000)})
	--({\sx*(2.8100)},{\sy*(-0.0000)})
	--({\sx*(2.8200)},{\sy*(-0.0000)})
	--({\sx*(2.8300)},{\sy*(-0.0000)})
	--({\sx*(2.8400)},{\sy*(-0.0000)})
	--({\sx*(2.8500)},{\sy*(-0.0000)})
	--({\sx*(2.8600)},{\sy*(0.0000)})
	--({\sx*(2.8700)},{\sy*(0.0000)})
	--({\sx*(2.8800)},{\sy*(0.0000)})
	--({\sx*(2.8900)},{\sy*(0.0000)})
	--({\sx*(2.9000)},{\sy*(0.0000)})
	--({\sx*(2.9100)},{\sy*(0.0000)})
	--({\sx*(2.9200)},{\sy*(0.0000)})
	--({\sx*(2.9300)},{\sy*(0.0000)})
	--({\sx*(2.9400)},{\sy*(0.0000)})
	--({\sx*(2.9500)},{\sy*(0.0000)})
	--({\sx*(2.9600)},{\sy*(0.0000)})
	--({\sx*(2.9700)},{\sy*(0.0000)})
	--({\sx*(2.9800)},{\sy*(0.0000)})
	--({\sx*(2.9900)},{\sy*(0.0000)})
	--({\sx*(3.0000)},{\sy*(0.0000)})
	--({\sx*(3.0100)},{\sy*(0.0000)})
	--({\sx*(3.0200)},{\sy*(0.0000)})
	--({\sx*(3.0300)},{\sy*(0.0000)})
	--({\sx*(3.0400)},{\sy*(0.0000)})
	--({\sx*(3.0500)},{\sy*(0.0000)})
	--({\sx*(3.0600)},{\sy*(0.0000)})
	--({\sx*(3.0700)},{\sy*(0.0000)})
	--({\sx*(3.0800)},{\sy*(0.0000)})
	--({\sx*(3.0900)},{\sy*(0.0000)})
	--({\sx*(3.1000)},{\sy*(0.0000)})
	--({\sx*(3.1100)},{\sy*(0.0000)})
	--({\sx*(3.1200)},{\sy*(0.0000)})
	--({\sx*(3.1300)},{\sy*(0.0000)})
	--({\sx*(3.1400)},{\sy*(0.0000)})
	--({\sx*(3.1500)},{\sy*(0.0000)})
	--({\sx*(3.1600)},{\sy*(0.0000)})
	--({\sx*(3.1700)},{\sy*(0.0000)})
	--({\sx*(3.1800)},{\sy*(0.0000)})
	--({\sx*(3.1900)},{\sy*(0.0000)})
	--({\sx*(3.2000)},{\sy*(0.0000)})
	--({\sx*(3.2100)},{\sy*(-0.0000)})
	--({\sx*(3.2200)},{\sy*(-0.0000)})
	--({\sx*(3.2300)},{\sy*(-0.0000)})
	--({\sx*(3.2400)},{\sy*(-0.0000)})
	--({\sx*(3.2500)},{\sy*(-0.0000)})
	--({\sx*(3.2600)},{\sy*(-0.0000)})
	--({\sx*(3.2700)},{\sy*(-0.0000)})
	--({\sx*(3.2800)},{\sy*(-0.0000)})
	--({\sx*(3.2900)},{\sy*(-0.0000)})
	--({\sx*(3.3000)},{\sy*(-0.0000)})
	--({\sx*(3.3100)},{\sy*(-0.0000)})
	--({\sx*(3.3200)},{\sy*(-0.0000)})
	--({\sx*(3.3300)},{\sy*(-0.0000)})
	--({\sx*(3.3400)},{\sy*(-0.0000)})
	--({\sx*(3.3500)},{\sy*(-0.0000)})
	--({\sx*(3.3600)},{\sy*(-0.0000)})
	--({\sx*(3.3700)},{\sy*(-0.0000)})
	--({\sx*(3.3800)},{\sy*(-0.0000)})
	--({\sx*(3.3900)},{\sy*(-0.0000)})
	--({\sx*(3.4000)},{\sy*(-0.0000)})
	--({\sx*(3.4100)},{\sy*(-0.0000)})
	--({\sx*(3.4200)},{\sy*(-0.0000)})
	--({\sx*(3.4300)},{\sy*(-0.0000)})
	--({\sx*(3.4400)},{\sy*(-0.0000)})
	--({\sx*(3.4500)},{\sy*(-0.0000)})
	--({\sx*(3.4600)},{\sy*(-0.0000)})
	--({\sx*(3.4700)},{\sy*(-0.0000)})
	--({\sx*(3.4800)},{\sy*(-0.0000)})
	--({\sx*(3.4900)},{\sy*(-0.0000)})
	--({\sx*(3.5000)},{\sy*(-0.0000)})
	--({\sx*(3.5100)},{\sy*(-0.0000)})
	--({\sx*(3.5200)},{\sy*(-0.0000)})
	--({\sx*(3.5300)},{\sy*(-0.0000)})
	--({\sx*(3.5400)},{\sy*(0.0000)})
	--({\sx*(3.5500)},{\sy*(0.0000)})
	--({\sx*(3.5600)},{\sy*(0.0000)})
	--({\sx*(3.5700)},{\sy*(0.0000)})
	--({\sx*(3.5800)},{\sy*(0.0000)})
	--({\sx*(3.5900)},{\sy*(0.0000)})
	--({\sx*(3.6000)},{\sy*(0.0000)})
	--({\sx*(3.6100)},{\sy*(0.0000)})
	--({\sx*(3.6200)},{\sy*(0.0000)})
	--({\sx*(3.6300)},{\sy*(0.0000)})
	--({\sx*(3.6400)},{\sy*(0.0000)})
	--({\sx*(3.6500)},{\sy*(0.0000)})
	--({\sx*(3.6600)},{\sy*(0.0000)})
	--({\sx*(3.6700)},{\sy*(0.0000)})
	--({\sx*(3.6800)},{\sy*(0.0000)})
	--({\sx*(3.6900)},{\sy*(0.0000)})
	--({\sx*(3.7000)},{\sy*(0.0000)})
	--({\sx*(3.7100)},{\sy*(0.0000)})
	--({\sx*(3.7200)},{\sy*(0.0000)})
	--({\sx*(3.7300)},{\sy*(0.0000)})
	--({\sx*(3.7400)},{\sy*(0.0000)})
	--({\sx*(3.7500)},{\sy*(0.0000)})
	--({\sx*(3.7600)},{\sy*(0.0000)})
	--({\sx*(3.7700)},{\sy*(0.0000)})
	--({\sx*(3.7800)},{\sy*(0.0000)})
	--({\sx*(3.7900)},{\sy*(0.0000)})
	--({\sx*(3.8000)},{\sy*(0.0000)})
	--({\sx*(3.8100)},{\sy*(0.0000)})
	--({\sx*(3.8200)},{\sy*(0.0000)})
	--({\sx*(3.8300)},{\sy*(0.0000)})
	--({\sx*(3.8400)},{\sy*(0.0000)})
	--({\sx*(3.8500)},{\sy*(0.0000)})
	--({\sx*(3.8600)},{\sy*(-0.0000)})
	--({\sx*(3.8700)},{\sy*(-0.0000)})
	--({\sx*(3.8800)},{\sy*(-0.0000)})
	--({\sx*(3.8900)},{\sy*(-0.0000)})
	--({\sx*(3.9000)},{\sy*(-0.0000)})
	--({\sx*(3.9100)},{\sy*(-0.0000)})
	--({\sx*(3.9200)},{\sy*(-0.0000)})
	--({\sx*(3.9300)},{\sy*(-0.0000)})
	--({\sx*(3.9400)},{\sy*(-0.0000)})
	--({\sx*(3.9500)},{\sy*(-0.0000)})
	--({\sx*(3.9600)},{\sy*(-0.0000)})
	--({\sx*(3.9700)},{\sy*(-0.0000)})
	--({\sx*(3.9800)},{\sy*(-0.0000)})
	--({\sx*(3.9900)},{\sy*(-0.0000)})
	--({\sx*(4.0000)},{\sy*(-0.0000)})
	--({\sx*(4.0100)},{\sy*(-0.0000)})
	--({\sx*(4.0200)},{\sy*(-0.0000)})
	--({\sx*(4.0300)},{\sy*(-0.0000)})
	--({\sx*(4.0400)},{\sy*(-0.0000)})
	--({\sx*(4.0500)},{\sy*(-0.0000)})
	--({\sx*(4.0600)},{\sy*(-0.0000)})
	--({\sx*(4.0700)},{\sy*(-0.0000)})
	--({\sx*(4.0800)},{\sy*(-0.0000)})
	--({\sx*(4.0900)},{\sy*(-0.0000)})
	--({\sx*(4.1000)},{\sy*(-0.0000)})
	--({\sx*(4.1100)},{\sy*(-0.0000)})
	--({\sx*(4.1200)},{\sy*(-0.0000)})
	--({\sx*(4.1300)},{\sy*(-0.0000)})
	--({\sx*(4.1400)},{\sy*(0.0000)})
	--({\sx*(4.1500)},{\sy*(0.0000)})
	--({\sx*(4.1600)},{\sy*(0.0000)})
	--({\sx*(4.1700)},{\sy*(0.0000)})
	--({\sx*(4.1800)},{\sy*(0.0000)})
	--({\sx*(4.1900)},{\sy*(0.0000)})
	--({\sx*(4.2000)},{\sy*(0.0000)})
	--({\sx*(4.2100)},{\sy*(0.0000)})
	--({\sx*(4.2200)},{\sy*(0.0000)})
	--({\sx*(4.2300)},{\sy*(0.0000)})
	--({\sx*(4.2400)},{\sy*(0.0000)})
	--({\sx*(4.2500)},{\sy*(0.0000)})
	--({\sx*(4.2600)},{\sy*(0.0000)})
	--({\sx*(4.2700)},{\sy*(0.0000)})
	--({\sx*(4.2800)},{\sy*(0.0000)})
	--({\sx*(4.2900)},{\sy*(0.0000)})
	--({\sx*(4.3000)},{\sy*(0.0000)})
	--({\sx*(4.3100)},{\sy*(0.0000)})
	--({\sx*(4.3200)},{\sy*(0.0000)})
	--({\sx*(4.3300)},{\sy*(0.0000)})
	--({\sx*(4.3400)},{\sy*(0.0000)})
	--({\sx*(4.3500)},{\sy*(0.0000)})
	--({\sx*(4.3600)},{\sy*(0.0000)})
	--({\sx*(4.3700)},{\sy*(0.0000)})
	--({\sx*(4.3800)},{\sy*(0.0000)})
	--({\sx*(4.3900)},{\sy*(-0.0000)})
	--({\sx*(4.4000)},{\sy*(-0.0000)})
	--({\sx*(4.4100)},{\sy*(-0.0000)})
	--({\sx*(4.4200)},{\sy*(-0.0000)})
	--({\sx*(4.4300)},{\sy*(-0.0000)})
	--({\sx*(4.4400)},{\sy*(-0.0000)})
	--({\sx*(4.4500)},{\sy*(-0.0000)})
	--({\sx*(4.4600)},{\sy*(-0.0000)})
	--({\sx*(4.4700)},{\sy*(-0.0000)})
	--({\sx*(4.4800)},{\sy*(-0.0000)})
	--({\sx*(4.4900)},{\sy*(-0.0000)})
	--({\sx*(4.5000)},{\sy*(-0.0000)})
	--({\sx*(4.5100)},{\sy*(-0.0000)})
	--({\sx*(4.5200)},{\sy*(-0.0000)})
	--({\sx*(4.5300)},{\sy*(-0.0000)})
	--({\sx*(4.5400)},{\sy*(-0.0000)})
	--({\sx*(4.5500)},{\sy*(-0.0000)})
	--({\sx*(4.5600)},{\sy*(-0.0000)})
	--({\sx*(4.5700)},{\sy*(-0.0000)})
	--({\sx*(4.5800)},{\sy*(-0.0000)})
	--({\sx*(4.5900)},{\sy*(-0.0000)})
	--({\sx*(4.6000)},{\sy*(-0.0000)})
	--({\sx*(4.6100)},{\sy*(0.0000)})
	--({\sx*(4.6200)},{\sy*(0.0000)})
	--({\sx*(4.6300)},{\sy*(0.0000)})
	--({\sx*(4.6400)},{\sy*(0.0000)})
	--({\sx*(4.6500)},{\sy*(0.0000)})
	--({\sx*(4.6600)},{\sy*(0.0000)})
	--({\sx*(4.6700)},{\sy*(0.0000)})
	--({\sx*(4.6800)},{\sy*(0.0000)})
	--({\sx*(4.6900)},{\sy*(0.0000)})
	--({\sx*(4.7000)},{\sy*(0.0000)})
	--({\sx*(4.7100)},{\sy*(0.0000)})
	--({\sx*(4.7200)},{\sy*(0.0000)})
	--({\sx*(4.7300)},{\sy*(0.0000)})
	--({\sx*(4.7400)},{\sy*(0.0000)})
	--({\sx*(4.7500)},{\sy*(0.0000)})
	--({\sx*(4.7600)},{\sy*(0.0000)})
	--({\sx*(4.7700)},{\sy*(0.0000)})
	--({\sx*(4.7800)},{\sy*(-0.0000)})
	--({\sx*(4.7900)},{\sy*(-0.0000)})
	--({\sx*(4.8000)},{\sy*(-0.0000)})
	--({\sx*(4.8100)},{\sy*(-0.0000)})
	--({\sx*(4.8200)},{\sy*(-0.0000)})
	--({\sx*(4.8300)},{\sy*(-0.0000)})
	--({\sx*(4.8400)},{\sy*(-0.0000)})
	--({\sx*(4.8500)},{\sy*(-0.0000)})
	--({\sx*(4.8600)},{\sy*(-0.0000)})
	--({\sx*(4.8700)},{\sy*(-0.0000)})
	--({\sx*(4.8800)},{\sy*(-0.0000)})
	--({\sx*(4.8900)},{\sy*(-0.0000)})
	--({\sx*(4.9000)},{\sy*(0.0000)})
	--({\sx*(4.9100)},{\sy*(0.0000)})
	--({\sx*(4.9200)},{\sy*(0.0000)})
	--({\sx*(4.9300)},{\sy*(0.0000)})
	--({\sx*(4.9400)},{\sy*(0.0000)})
	--({\sx*(4.9500)},{\sy*(0.0000)})
	--({\sx*(4.9600)},{\sy*(0.0000)})
	--({\sx*(4.9700)},{\sy*(0.0000)})
	--({\sx*(4.9800)},{\sy*(-0.0000)})
	--({\sx*(4.9900)},{\sy*(-0.0000)})
	--({\sx*(5.0000)},{\sy*(0.0000)});
}
\def\xwertel{
\fill[color=red] (0.0000,0) circle[radius={0.07/\skala}];
\fill[color=white] (0.0000,0) circle[radius={0.05/\skala}];
\fill[color=red] (0.0214,0) circle[radius={0.07/\skala}];
\fill[color=white] (0.0214,0) circle[radius={0.05/\skala}];
\fill[color=red] (0.0852,0) circle[radius={0.07/\skala}];
\fill[color=white] (0.0852,0) circle[radius={0.05/\skala}];
\fill[color=red] (0.1903,0) circle[radius={0.07/\skala}];
\fill[color=white] (0.1903,0) circle[radius={0.05/\skala}];
\fill[color=red] (0.3349,0) circle[radius={0.07/\skala}];
\fill[color=white] (0.3349,0) circle[radius={0.05/\skala}];
\fill[color=red] (0.5166,0) circle[radius={0.07/\skala}];
\fill[color=white] (0.5166,0) circle[radius={0.05/\skala}];
\fill[color=red] (0.7322,0) circle[radius={0.07/\skala}];
\fill[color=white] (0.7322,0) circle[radius={0.05/\skala}];
\fill[color=red] (0.9781,0) circle[radius={0.07/\skala}];
\fill[color=white] (0.9781,0) circle[radius={0.05/\skala}];
\fill[color=red] (1.2500,0) circle[radius={0.07/\skala}];
\fill[color=white] (1.2500,0) circle[radius={0.05/\skala}];
\fill[color=red] (1.5433,0) circle[radius={0.07/\skala}];
\fill[color=white] (1.5433,0) circle[radius={0.05/\skala}];
\fill[color=red] (1.8530,0) circle[radius={0.07/\skala}];
\fill[color=white] (1.8530,0) circle[radius={0.05/\skala}];
\fill[color=red] (2.1737,0) circle[radius={0.07/\skala}];
\fill[color=white] (2.1737,0) circle[radius={0.05/\skala}];
\fill[color=red] (2.5000,0) circle[radius={0.07/\skala}];
\fill[color=white] (2.5000,0) circle[radius={0.05/\skala}];
\fill[color=red] (2.8263,0) circle[radius={0.07/\skala}];
\fill[color=white] (2.8263,0) circle[radius={0.05/\skala}];
\fill[color=red] (3.1470,0) circle[radius={0.07/\skala}];
\fill[color=white] (3.1470,0) circle[radius={0.05/\skala}];
\fill[color=red] (3.4567,0) circle[radius={0.07/\skala}];
\fill[color=white] (3.4567,0) circle[radius={0.05/\skala}];
\fill[color=red] (3.7500,0) circle[radius={0.07/\skala}];
\fill[color=white] (3.7500,0) circle[radius={0.05/\skala}];
\fill[color=red] (4.0219,0) circle[radius={0.07/\skala}];
\fill[color=white] (4.0219,0) circle[radius={0.05/\skala}];
\fill[color=red] (4.2678,0) circle[radius={0.07/\skala}];
\fill[color=white] (4.2678,0) circle[radius={0.05/\skala}];
\fill[color=red] (4.4834,0) circle[radius={0.07/\skala}];
\fill[color=white] (4.4834,0) circle[radius={0.05/\skala}];
\fill[color=red] (4.6651,0) circle[radius={0.07/\skala}];
\fill[color=white] (4.6651,0) circle[radius={0.05/\skala}];
\fill[color=red] (4.8097,0) circle[radius={0.07/\skala}];
\fill[color=white] (4.8097,0) circle[radius={0.05/\skala}];
\fill[color=red] (4.9148,0) circle[radius={0.07/\skala}];
\fill[color=white] (4.9148,0) circle[radius={0.05/\skala}];
\fill[color=red] (4.9786,0) circle[radius={0.07/\skala}];
\fill[color=white] (4.9786,0) circle[radius={0.05/\skala}];
\fill[color=red] (5.0000,0) circle[radius={0.07/\skala}];
\fill[color=white] (5.0000,0) circle[radius={0.05/\skala}];
}
\def\punktel{24}
\def\maxfehlerl{1.947\cdot 10^{-12}}
\def\fehlerl{
\draw[color=red,line width=1.4pt,line join=round] ({\sx*(0.000)},{\sy*(0.0000)})
	--({\sx*(0.0100)},{\sy*(-0.0938)})
	--({\sx*(0.0200)},{\sy*(-0.0162)})
	--({\sx*(0.0300)},{\sy*(0.1065)})
	--({\sx*(0.0400)},{\sy*(0.2051)})
	--({\sx*(0.0500)},{\sy*(0.2489)})
	--({\sx*(0.0600)},{\sy*(0.2322)})
	--({\sx*(0.0700)},{\sy*(0.1642)})
	--({\sx*(0.0800)},{\sy*(0.0611)})
	--({\sx*(0.0900)},{\sy*(-0.0582)})
	--({\sx*(0.1000)},{\sy*(-0.1767)})
	--({\sx*(0.1100)},{\sy*(-0.2793)})
	--({\sx*(0.1200)},{\sy*(-0.3550)})
	--({\sx*(0.1300)},{\sy*(-0.3971)})
	--({\sx*(0.1400)},{\sy*(-0.4034)})
	--({\sx*(0.1500)},{\sy*(-0.3744)})
	--({\sx*(0.1600)},{\sy*(-0.3134)})
	--({\sx*(0.1700)},{\sy*(-0.2264)})
	--({\sx*(0.1800)},{\sy*(-0.1205)})
	--({\sx*(0.1900)},{\sy*(-0.0037)})
	--({\sx*(0.2000)},{\sy*(0.1166)})
	--({\sx*(0.2100)},{\sy*(0.2326)})
	--({\sx*(0.2200)},{\sy*(0.3373)})
	--({\sx*(0.2300)},{\sy*(0.4254)})
	--({\sx*(0.2400)},{\sy*(0.4923)})
	--({\sx*(0.2500)},{\sy*(0.5356)})
	--({\sx*(0.2600)},{\sy*(0.5527)})
	--({\sx*(0.2700)},{\sy*(0.5439)})
	--({\sx*(0.2800)},{\sy*(0.5099)})
	--({\sx*(0.2900)},{\sy*(0.4526)})
	--({\sx*(0.3000)},{\sy*(0.3754)})
	--({\sx*(0.3100)},{\sy*(0.2808)})
	--({\sx*(0.3200)},{\sy*(0.1739)})
	--({\sx*(0.3300)},{\sy*(0.0586)})
	--({\sx*(0.3400)},{\sy*(-0.0604)})
	--({\sx*(0.3500)},{\sy*(-0.1786)})
	--({\sx*(0.3600)},{\sy*(-0.2921)})
	--({\sx*(0.3700)},{\sy*(-0.3966)})
	--({\sx*(0.3800)},{\sy*(-0.4889)})
	--({\sx*(0.3900)},{\sy*(-0.5656)})
	--({\sx*(0.4000)},{\sy*(-0.6253)})
	--({\sx*(0.4100)},{\sy*(-0.6654)})
	--({\sx*(0.4200)},{\sy*(-0.6852)})
	--({\sx*(0.4300)},{\sy*(-0.6846)})
	--({\sx*(0.4400)},{\sy*(-0.6639)})
	--({\sx*(0.4500)},{\sy*(-0.6235)})
	--({\sx*(0.4600)},{\sy*(-0.5650)})
	--({\sx*(0.4700)},{\sy*(-0.4903)})
	--({\sx*(0.4800)},{\sy*(-0.4017)})
	--({\sx*(0.4900)},{\sy*(-0.3015)})
	--({\sx*(0.5000)},{\sy*(-0.1925)})
	--({\sx*(0.5100)},{\sy*(-0.0775)})
	--({\sx*(0.5200)},{\sy*(0.0399)})
	--({\sx*(0.5300)},{\sy*(0.1573)})
	--({\sx*(0.5400)},{\sy*(0.2718)})
	--({\sx*(0.5500)},{\sy*(0.3805)})
	--({\sx*(0.5600)},{\sy*(0.4806)})
	--({\sx*(0.5700)},{\sy*(0.5703)})
	--({\sx*(0.5800)},{\sy*(0.6475)})
	--({\sx*(0.5900)},{\sy*(0.7106)})
	--({\sx*(0.6000)},{\sy*(0.7580)})
	--({\sx*(0.6100)},{\sy*(0.7891)})
	--({\sx*(0.6200)},{\sy*(0.8033)})
	--({\sx*(0.6300)},{\sy*(0.8004)})
	--({\sx*(0.6400)},{\sy*(0.7805)})
	--({\sx*(0.6500)},{\sy*(0.7444)})
	--({\sx*(0.6600)},{\sy*(0.6928)})
	--({\sx*(0.6700)},{\sy*(0.6270)})
	--({\sx*(0.6800)},{\sy*(0.5481)})
	--({\sx*(0.6900)},{\sy*(0.4584)})
	--({\sx*(0.7000)},{\sy*(0.3591)})
	--({\sx*(0.7100)},{\sy*(0.2527)})
	--({\sx*(0.7200)},{\sy*(0.1408)})
	--({\sx*(0.7300)},{\sy*(0.0257)})
	--({\sx*(0.7400)},{\sy*(-0.0900)})
	--({\sx*(0.7500)},{\sy*(-0.2048)})
	--({\sx*(0.7600)},{\sy*(-0.3161)})
	--({\sx*(0.7700)},{\sy*(-0.4221)})
	--({\sx*(0.7800)},{\sy*(-0.5211)})
	--({\sx*(0.7900)},{\sy*(-0.6111)})
	--({\sx*(0.8000)},{\sy*(-0.6908)})
	--({\sx*(0.8100)},{\sy*(-0.7589)})
	--({\sx*(0.8200)},{\sy*(-0.8145)})
	--({\sx*(0.8300)},{\sy*(-0.8561)})
	--({\sx*(0.8400)},{\sy*(-0.8835)})
	--({\sx*(0.8500)},{\sy*(-0.8967)})
	--({\sx*(0.8600)},{\sy*(-0.8951)})
	--({\sx*(0.8700)},{\sy*(-0.8789)})
	--({\sx*(0.8800)},{\sy*(-0.8483)})
	--({\sx*(0.8900)},{\sy*(-0.8044)})
	--({\sx*(0.9000)},{\sy*(-0.7474)})
	--({\sx*(0.9100)},{\sy*(-0.6789)})
	--({\sx*(0.9200)},{\sy*(-0.5994)})
	--({\sx*(0.9300)},{\sy*(-0.5106)})
	--({\sx*(0.9400)},{\sy*(-0.4138)})
	--({\sx*(0.9500)},{\sy*(-0.3107)})
	--({\sx*(0.9600)},{\sy*(-0.2025)})
	--({\sx*(0.9700)},{\sy*(-0.0912)})
	--({\sx*(0.9800)},{\sy*(0.0216)})
	--({\sx*(0.9900)},{\sy*(0.1343)})
	--({\sx*(1.0000)},{\sy*(0.2453)})
	--({\sx*(1.0100)},{\sy*(0.3529)})
	--({\sx*(1.0200)},{\sy*(0.4555)})
	--({\sx*(1.0300)},{\sy*(0.5516)})
	--({\sx*(1.0400)},{\sy*(0.6402)})
	--({\sx*(1.0500)},{\sy*(0.7200)})
	--({\sx*(1.0600)},{\sy*(0.7897)})
	--({\sx*(1.0700)},{\sy*(0.8486)})
	--({\sx*(1.0800)},{\sy*(0.8960)})
	--({\sx*(1.0900)},{\sy*(0.9311)})
	--({\sx*(1.1000)},{\sy*(0.9536)})
	--({\sx*(1.1100)},{\sy*(0.9631)})
	--({\sx*(1.1200)},{\sy*(0.9601)})
	--({\sx*(1.1300)},{\sy*(0.9442)})
	--({\sx*(1.1400)},{\sy*(0.9156)})
	--({\sx*(1.1500)},{\sy*(0.8752)})
	--({\sx*(1.1600)},{\sy*(0.8232)})
	--({\sx*(1.1700)},{\sy*(0.7606)})
	--({\sx*(1.1800)},{\sy*(0.6884)})
	--({\sx*(1.1900)},{\sy*(0.6070)})
	--({\sx*(1.2000)},{\sy*(0.5179)})
	--({\sx*(1.2100)},{\sy*(0.4224)})
	--({\sx*(1.2200)},{\sy*(0.3216)})
	--({\sx*(1.2300)},{\sy*(0.2166)})
	--({\sx*(1.2400)},{\sy*(0.1089)})
	--({\sx*(1.2500)},{\sy*(0.0000)})
	--({\sx*(1.2600)},{\sy*(-0.1090)})
	--({\sx*(1.2700)},{\sy*(-0.2166)})
	--({\sx*(1.2800)},{\sy*(-0.3216)})
	--({\sx*(1.2900)},{\sy*(-0.4225)})
	--({\sx*(1.3000)},{\sy*(-0.5183)})
	--({\sx*(1.3100)},{\sy*(-0.6079)})
	--({\sx*(1.3200)},{\sy*(-0.6903)})
	--({\sx*(1.3300)},{\sy*(-0.7643)})
	--({\sx*(1.3400)},{\sy*(-0.8292)})
	--({\sx*(1.3500)},{\sy*(-0.8844)})
	--({\sx*(1.3600)},{\sy*(-0.9292)})
	--({\sx*(1.3700)},{\sy*(-0.9631)})
	--({\sx*(1.3800)},{\sy*(-0.9858)})
	--({\sx*(1.3900)},{\sy*(-0.9970)})
	--({\sx*(1.4000)},{\sy*(-0.9969)})
	--({\sx*(1.4100)},{\sy*(-0.9853)})
	--({\sx*(1.4200)},{\sy*(-0.9625)})
	--({\sx*(1.4300)},{\sy*(-0.9286)})
	--({\sx*(1.4400)},{\sy*(-0.8845)})
	--({\sx*(1.4500)},{\sy*(-0.8305)})
	--({\sx*(1.4600)},{\sy*(-0.7671)})
	--({\sx*(1.4700)},{\sy*(-0.6953)})
	--({\sx*(1.4800)},{\sy*(-0.6158)})
	--({\sx*(1.4900)},{\sy*(-0.5297)})
	--({\sx*(1.5000)},{\sy*(-0.4378)})
	--({\sx*(1.5100)},{\sy*(-0.3412)})
	--({\sx*(1.5200)},{\sy*(-0.2410)})
	--({\sx*(1.5300)},{\sy*(-0.1385)})
	--({\sx*(1.5400)},{\sy*(-0.0344)})
	--({\sx*(1.5500)},{\sy*(0.0698)})
	--({\sx*(1.5600)},{\sy*(0.1731)})
	--({\sx*(1.5700)},{\sy*(0.2743)})
	--({\sx*(1.5800)},{\sy*(0.3725)})
	--({\sx*(1.5900)},{\sy*(0.4664)})
	--({\sx*(1.6000)},{\sy*(0.5552)})
	--({\sx*(1.6100)},{\sy*(0.6380)})
	--({\sx*(1.6200)},{\sy*(0.7139)})
	--({\sx*(1.6300)},{\sy*(0.7821)})
	--({\sx*(1.6400)},{\sy*(0.8419)})
	--({\sx*(1.6500)},{\sy*(0.8928)})
	--({\sx*(1.6600)},{\sy*(0.9344)})
	--({\sx*(1.6700)},{\sy*(0.9661)})
	--({\sx*(1.6800)},{\sy*(0.9877)})
	--({\sx*(1.6900)},{\sy*(0.9990)})
	--({\sx*(1.7000)},{\sy*(1.0000)})
	--({\sx*(1.7100)},{\sy*(0.9908)})
	--({\sx*(1.7200)},{\sy*(0.9714)})
	--({\sx*(1.7300)},{\sy*(0.9421)})
	--({\sx*(1.7400)},{\sy*(0.9032)})
	--({\sx*(1.7500)},{\sy*(0.8553)})
	--({\sx*(1.7600)},{\sy*(0.7990)})
	--({\sx*(1.7700)},{\sy*(0.7345)})
	--({\sx*(1.7800)},{\sy*(0.6629)})
	--({\sx*(1.7900)},{\sy*(0.5848)})
	--({\sx*(1.8000)},{\sy*(0.5011)})
	--({\sx*(1.8100)},{\sy*(0.4125)})
	--({\sx*(1.8200)},{\sy*(0.3200)})
	--({\sx*(1.8300)},{\sy*(0.2247)})
	--({\sx*(1.8400)},{\sy*(0.1274)})
	--({\sx*(1.8500)},{\sy*(0.0291)})
	--({\sx*(1.8600)},{\sy*(-0.0692)})
	--({\sx*(1.8700)},{\sy*(-0.1665)})
	--({\sx*(1.8800)},{\sy*(-0.2619)})
	--({\sx*(1.8900)},{\sy*(-0.3545)})
	--({\sx*(1.9000)},{\sy*(-0.4433)})
	--({\sx*(1.9100)},{\sy*(-0.5274)})
	--({\sx*(1.9200)},{\sy*(-0.6061)})
	--({\sx*(1.9300)},{\sy*(-0.6787)})
	--({\sx*(1.9400)},{\sy*(-0.7445)})
	--({\sx*(1.9500)},{\sy*(-0.8028)})
	--({\sx*(1.9600)},{\sy*(-0.8531)})
	--({\sx*(1.9700)},{\sy*(-0.8950)})
	--({\sx*(1.9800)},{\sy*(-0.9281)})
	--({\sx*(1.9900)},{\sy*(-0.9522)})
	--({\sx*(2.0000)},{\sy*(-0.9670)})
	--({\sx*(2.0100)},{\sy*(-0.9725)})
	--({\sx*(2.0200)},{\sy*(-0.9686)})
	--({\sx*(2.0300)},{\sy*(-0.9555)})
	--({\sx*(2.0400)},{\sy*(-0.9333)})
	--({\sx*(2.0500)},{\sy*(-0.9022)})
	--({\sx*(2.0600)},{\sy*(-0.8627)})
	--({\sx*(2.0700)},{\sy*(-0.8151)})
	--({\sx*(2.0800)},{\sy*(-0.7600)})
	--({\sx*(2.0900)},{\sy*(-0.6978)})
	--({\sx*(2.1000)},{\sy*(-0.6293)})
	--({\sx*(2.1100)},{\sy*(-0.5550)})
	--({\sx*(2.1200)},{\sy*(-0.4759)})
	--({\sx*(2.1300)},{\sy*(-0.3925)})
	--({\sx*(2.1400)},{\sy*(-0.3058)})
	--({\sx*(2.1500)},{\sy*(-0.2165)})
	--({\sx*(2.1600)},{\sy*(-0.1256)})
	--({\sx*(2.1700)},{\sy*(-0.0338)})
	--({\sx*(2.1800)},{\sy*(0.0578)})
	--({\sx*(2.1900)},{\sy*(0.1486)})
	--({\sx*(2.2000)},{\sy*(0.2376)})
	--({\sx*(2.2100)},{\sy*(0.3240)})
	--({\sx*(2.2200)},{\sy*(0.4070)})
	--({\sx*(2.2300)},{\sy*(0.4858)})
	--({\sx*(2.2400)},{\sy*(0.5598)})
	--({\sx*(2.2500)},{\sy*(0.6282)})
	--({\sx*(2.2600)},{\sy*(0.6905)})
	--({\sx*(2.2700)},{\sy*(0.7461)})
	--({\sx*(2.2800)},{\sy*(0.7945)})
	--({\sx*(2.2900)},{\sy*(0.8352)})
	--({\sx*(2.3000)},{\sy*(0.8681)})
	--({\sx*(2.3100)},{\sy*(0.8928)})
	--({\sx*(2.3200)},{\sy*(0.9091)})
	--({\sx*(2.3300)},{\sy*(0.9168)})
	--({\sx*(2.3400)},{\sy*(0.9160)})
	--({\sx*(2.3500)},{\sy*(0.9068)})
	--({\sx*(2.3600)},{\sy*(0.8892)})
	--({\sx*(2.3700)},{\sy*(0.8635)})
	--({\sx*(2.3800)},{\sy*(0.8299)})
	--({\sx*(2.3900)},{\sy*(0.7888)})
	--({\sx*(2.4000)},{\sy*(0.7406)})
	--({\sx*(2.4100)},{\sy*(0.6858)})
	--({\sx*(2.4200)},{\sy*(0.6250)})
	--({\sx*(2.4300)},{\sy*(0.5587)})
	--({\sx*(2.4400)},{\sy*(0.4876)})
	--({\sx*(2.4500)},{\sy*(0.4123)})
	--({\sx*(2.4600)},{\sy*(0.3337)})
	--({\sx*(2.4700)},{\sy*(0.2523)})
	--({\sx*(2.4800)},{\sy*(0.1691)})
	--({\sx*(2.4900)},{\sy*(0.0847)})
	--({\sx*(2.5000)},{\sy*(0.0000)})
	--({\sx*(2.5100)},{\sy*(-0.0843)})
	--({\sx*(2.5200)},{\sy*(-0.1673)})
	--({\sx*(2.5300)},{\sy*(-0.2484)})
	--({\sx*(2.5400)},{\sy*(-0.3267)})
	--({\sx*(2.5500)},{\sy*(-0.4016)})
	--({\sx*(2.5600)},{\sy*(-0.4724)})
	--({\sx*(2.5700)},{\sy*(-0.5384)})
	--({\sx*(2.5800)},{\sy*(-0.5991)})
	--({\sx*(2.5900)},{\sy*(-0.6540)})
	--({\sx*(2.6000)},{\sy*(-0.7025)})
	--({\sx*(2.6100)},{\sy*(-0.7443)})
	--({\sx*(2.6200)},{\sy*(-0.7789)})
	--({\sx*(2.6300)},{\sy*(-0.8061)})
	--({\sx*(2.6400)},{\sy*(-0.8258)})
	--({\sx*(2.6500)},{\sy*(-0.8377)})
	--({\sx*(2.6600)},{\sy*(-0.8418)})
	--({\sx*(2.6700)},{\sy*(-0.8381)})
	--({\sx*(2.6800)},{\sy*(-0.8266)})
	--({\sx*(2.6900)},{\sy*(-0.8075)})
	--({\sx*(2.7000)},{\sy*(-0.7811)})
	--({\sx*(2.7100)},{\sy*(-0.7476)})
	--({\sx*(2.7200)},{\sy*(-0.7073)})
	--({\sx*(2.7300)},{\sy*(-0.6608)})
	--({\sx*(2.7400)},{\sy*(-0.6083)})
	--({\sx*(2.7500)},{\sy*(-0.5505)})
	--({\sx*(2.7600)},{\sy*(-0.4880)})
	--({\sx*(2.7700)},{\sy*(-0.4213)})
	--({\sx*(2.7800)},{\sy*(-0.3511)})
	--({\sx*(2.7900)},{\sy*(-0.2780)})
	--({\sx*(2.8000)},{\sy*(-0.2028)})
	--({\sx*(2.8100)},{\sy*(-0.1262)})
	--({\sx*(2.8200)},{\sy*(-0.0489)})
	--({\sx*(2.8300)},{\sy*(0.0284)})
	--({\sx*(2.8400)},{\sy*(0.1050)})
	--({\sx*(2.8500)},{\sy*(0.1800)})
	--({\sx*(2.8600)},{\sy*(0.2529)})
	--({\sx*(2.8700)},{\sy*(0.3229)})
	--({\sx*(2.8800)},{\sy*(0.3894)})
	--({\sx*(2.8900)},{\sy*(0.4518)})
	--({\sx*(2.9000)},{\sy*(0.5096)})
	--({\sx*(2.9100)},{\sy*(0.5621)})
	--({\sx*(2.9200)},{\sy*(0.6090)})
	--({\sx*(2.9300)},{\sy*(0.6497)})
	--({\sx*(2.9400)},{\sy*(0.6840)})
	--({\sx*(2.9500)},{\sy*(0.7117)})
	--({\sx*(2.9600)},{\sy*(0.7323)})
	--({\sx*(2.9700)},{\sy*(0.7458)})
	--({\sx*(2.9800)},{\sy*(0.7521)})
	--({\sx*(2.9900)},{\sy*(0.7511)})
	--({\sx*(3.0000)},{\sy*(0.7430)})
	--({\sx*(3.0100)},{\sy*(0.7278)})
	--({\sx*(3.0200)},{\sy*(0.7057)})
	--({\sx*(3.0300)},{\sy*(0.6769)})
	--({\sx*(3.0400)},{\sy*(0.6418)})
	--({\sx*(3.0500)},{\sy*(0.6008)})
	--({\sx*(3.0600)},{\sy*(0.5543)})
	--({\sx*(3.0700)},{\sy*(0.5027)})
	--({\sx*(3.0800)},{\sy*(0.4466)})
	--({\sx*(3.0900)},{\sy*(0.3865)})
	--({\sx*(3.1000)},{\sy*(0.3232)})
	--({\sx*(3.1100)},{\sy*(0.2571)})
	--({\sx*(3.1200)},{\sy*(0.1890)})
	--({\sx*(3.1300)},{\sy*(0.1195)})
	--({\sx*(3.1400)},{\sy*(0.0494)})
	--({\sx*(3.1500)},{\sy*(-0.0206)})
	--({\sx*(3.1600)},{\sy*(-0.0900)})
	--({\sx*(3.1700)},{\sy*(-0.1579)})
	--({\sx*(3.1800)},{\sy*(-0.2238)})
	--({\sx*(3.1900)},{\sy*(-0.2869)})
	--({\sx*(3.2000)},{\sy*(-0.3467)})
	--({\sx*(3.2100)},{\sy*(-0.4025)})
	--({\sx*(3.2200)},{\sy*(-0.4539)})
	--({\sx*(3.2300)},{\sy*(-0.5003)})
	--({\sx*(3.2400)},{\sy*(-0.5414)})
	--({\sx*(3.2500)},{\sy*(-0.5766)})
	--({\sx*(3.2600)},{\sy*(-0.6057)})
	--({\sx*(3.2700)},{\sy*(-0.6285)})
	--({\sx*(3.2800)},{\sy*(-0.6446)})
	--({\sx*(3.2900)},{\sy*(-0.6541)})
	--({\sx*(3.3000)},{\sy*(-0.6568)})
	--({\sx*(3.3100)},{\sy*(-0.6527)})
	--({\sx*(3.3200)},{\sy*(-0.6420)})
	--({\sx*(3.3300)},{\sy*(-0.6247)})
	--({\sx*(3.3400)},{\sy*(-0.6011)})
	--({\sx*(3.3500)},{\sy*(-0.5714)})
	--({\sx*(3.3600)},{\sy*(-0.5360)})
	--({\sx*(3.3700)},{\sy*(-0.4953)})
	--({\sx*(3.3800)},{\sy*(-0.4498)})
	--({\sx*(3.3900)},{\sy*(-0.3999)})
	--({\sx*(3.4000)},{\sy*(-0.3462)})
	--({\sx*(3.4100)},{\sy*(-0.2893)})
	--({\sx*(3.4200)},{\sy*(-0.2299)})
	--({\sx*(3.4300)},{\sy*(-0.1684)})
	--({\sx*(3.4400)},{\sy*(-0.1057)})
	--({\sx*(3.4500)},{\sy*(-0.0425)})
	--({\sx*(3.4600)},{\sy*(0.0208)})
	--({\sx*(3.4700)},{\sy*(0.0832)})
	--({\sx*(3.4800)},{\sy*(0.1442)})
	--({\sx*(3.4900)},{\sy*(0.2030)})
	--({\sx*(3.5000)},{\sy*(0.2592)})
	--({\sx*(3.5100)},{\sy*(0.3120)})
	--({\sx*(3.5200)},{\sy*(0.3609)})
	--({\sx*(3.5300)},{\sy*(0.4054)})
	--({\sx*(3.5400)},{\sy*(0.4449)})
	--({\sx*(3.5500)},{\sy*(0.4792)})
	--({\sx*(3.5600)},{\sy*(0.5078)})
	--({\sx*(3.5700)},{\sy*(0.5304)})
	--({\sx*(3.5800)},{\sy*(0.5469)})
	--({\sx*(3.5900)},{\sy*(0.5570)})
	--({\sx*(3.6000)},{\sy*(0.5606)})
	--({\sx*(3.6100)},{\sy*(0.5579)})
	--({\sx*(3.6200)},{\sy*(0.5488)})
	--({\sx*(3.6300)},{\sy*(0.5334)})
	--({\sx*(3.6400)},{\sy*(0.5120)})
	--({\sx*(3.6500)},{\sy*(0.4848)})
	--({\sx*(3.6600)},{\sy*(0.4523)})
	--({\sx*(3.6700)},{\sy*(0.4147)})
	--({\sx*(3.6800)},{\sy*(0.3726)})
	--({\sx*(3.6900)},{\sy*(0.3265)})
	--({\sx*(3.7000)},{\sy*(0.2770)})
	--({\sx*(3.7100)},{\sy*(0.2246)})
	--({\sx*(3.7200)},{\sy*(0.1700)})
	--({\sx*(3.7300)},{\sy*(0.1139)})
	--({\sx*(3.7400)},{\sy*(0.0570)})
	--({\sx*(3.7500)},{\sy*(0.0000)})
	--({\sx*(3.7600)},{\sy*(-0.0565)})
	--({\sx*(3.7700)},{\sy*(-0.1117)})
	--({\sx*(3.7800)},{\sy*(-0.1649)})
	--({\sx*(3.7900)},{\sy*(-0.2156)})
	--({\sx*(3.8000)},{\sy*(-0.2630)})
	--({\sx*(3.8100)},{\sy*(-0.3067)})
	--({\sx*(3.8200)},{\sy*(-0.3460)})
	--({\sx*(3.8300)},{\sy*(-0.3804)})
	--({\sx*(3.8400)},{\sy*(-0.4097)})
	--({\sx*(3.8500)},{\sy*(-0.4333)})
	--({\sx*(3.8600)},{\sy*(-0.4510)})
	--({\sx*(3.8700)},{\sy*(-0.4627)})
	--({\sx*(3.8800)},{\sy*(-0.4681)})
	--({\sx*(3.8900)},{\sy*(-0.4673)})
	--({\sx*(3.9000)},{\sy*(-0.4602)})
	--({\sx*(3.9100)},{\sy*(-0.4471)})
	--({\sx*(3.9200)},{\sy*(-0.4281)})
	--({\sx*(3.9300)},{\sy*(-0.4034)})
	--({\sx*(3.9400)},{\sy*(-0.3735)})
	--({\sx*(3.9500)},{\sy*(-0.3388)})
	--({\sx*(3.9600)},{\sy*(-0.2998)})
	--({\sx*(3.9700)},{\sy*(-0.2570)})
	--({\sx*(3.9800)},{\sy*(-0.2111)})
	--({\sx*(3.9900)},{\sy*(-0.1627)})
	--({\sx*(4.0000)},{\sy*(-0.1125)})
	--({\sx*(4.0100)},{\sy*(-0.0613)})
	--({\sx*(4.0200)},{\sy*(-0.0098)})
	--({\sx*(4.0300)},{\sy*(0.0413)})
	--({\sx*(4.0400)},{\sy*(0.0911)})
	--({\sx*(4.0500)},{\sy*(0.1391)})
	--({\sx*(4.0600)},{\sy*(0.1843)})
	--({\sx*(4.0700)},{\sy*(0.2263)})
	--({\sx*(4.0800)},{\sy*(0.2643)})
	--({\sx*(4.0900)},{\sy*(0.2978)})
	--({\sx*(4.1000)},{\sy*(0.3263)})
	--({\sx*(4.1100)},{\sy*(0.3494)})
	--({\sx*(4.1200)},{\sy*(0.3667)})
	--({\sx*(4.1300)},{\sy*(0.3779)})
	--({\sx*(4.1400)},{\sy*(0.3830)})
	--({\sx*(4.1500)},{\sy*(0.3818)})
	--({\sx*(4.1600)},{\sy*(0.3744)})
	--({\sx*(4.1700)},{\sy*(0.3609)})
	--({\sx*(4.1800)},{\sy*(0.3416)})
	--({\sx*(4.1900)},{\sy*(0.3168)})
	--({\sx*(4.2000)},{\sy*(0.2869)})
	--({\sx*(4.2100)},{\sy*(0.2525)})
	--({\sx*(4.2200)},{\sy*(0.2142)})
	--({\sx*(4.2300)},{\sy*(0.1727)})
	--({\sx*(4.2400)},{\sy*(0.1286)})
	--({\sx*(4.2500)},{\sy*(0.0829)})
	--({\sx*(4.2600)},{\sy*(0.0363)})
	--({\sx*(4.2700)},{\sy*(-0.0104)})
	--({\sx*(4.2800)},{\sy*(-0.0562)})
	--({\sx*(4.2900)},{\sy*(-0.1003)})
	--({\sx*(4.3000)},{\sy*(-0.1419)})
	--({\sx*(4.3100)},{\sy*(-0.1803)})
	--({\sx*(4.3200)},{\sy*(-0.2145)})
	--({\sx*(4.3300)},{\sy*(-0.2441)})
	--({\sx*(4.3400)},{\sy*(-0.2685)})
	--({\sx*(4.3500)},{\sy*(-0.2870)})
	--({\sx*(4.3600)},{\sy*(-0.2995)})
	--({\sx*(4.3700)},{\sy*(-0.3056)})
	--({\sx*(4.3800)},{\sy*(-0.3052)})
	--({\sx*(4.3900)},{\sy*(-0.2983)})
	--({\sx*(4.4000)},{\sy*(-0.2852)})
	--({\sx*(4.4100)},{\sy*(-0.2660)})
	--({\sx*(4.4200)},{\sy*(-0.2412)})
	--({\sx*(4.4300)},{\sy*(-0.2114)})
	--({\sx*(4.4400)},{\sy*(-0.1773)})
	--({\sx*(4.4500)},{\sy*(-0.1396)})
	--({\sx*(4.4600)},{\sy*(-0.0993)})
	--({\sx*(4.4700)},{\sy*(-0.0572)})
	--({\sx*(4.4800)},{\sy*(-0.0144)})
	--({\sx*(4.4900)},{\sy*(0.0280)})
	--({\sx*(4.5000)},{\sy*(0.0689)})
	--({\sx*(4.5100)},{\sy*(0.1074)})
	--({\sx*(4.5200)},{\sy*(0.1424)})
	--({\sx*(4.5300)},{\sy*(0.1731)})
	--({\sx*(4.5400)},{\sy*(0.1985)})
	--({\sx*(4.5500)},{\sy*(0.2179)})
	--({\sx*(4.5600)},{\sy*(0.2309)})
	--({\sx*(4.5700)},{\sy*(0.2370)})
	--({\sx*(4.5800)},{\sy*(0.2360)})
	--({\sx*(4.5900)},{\sy*(0.2280)})
	--({\sx*(4.6000)},{\sy*(0.2132)})
	--({\sx*(4.6100)},{\sy*(0.1920)})
	--({\sx*(4.6200)},{\sy*(0.1651)})
	--({\sx*(4.6300)},{\sy*(0.1333)})
	--({\sx*(4.6400)},{\sy*(0.0977)})
	--({\sx*(4.6500)},{\sy*(0.0595)})
	--({\sx*(4.6600)},{\sy*(0.0200)})
	--({\sx*(4.6700)},{\sy*(-0.0193)})
	--({\sx*(4.6800)},{\sy*(-0.0570)})
	--({\sx*(4.6900)},{\sy*(-0.0917)})
	--({\sx*(4.7000)},{\sy*(-0.1218)})
	--({\sx*(4.7100)},{\sy*(-0.1463)})
	--({\sx*(4.7200)},{\sy*(-0.1640)})
	--({\sx*(4.7300)},{\sy*(-0.1741)})
	--({\sx*(4.7400)},{\sy*(-0.1761)})
	--({\sx*(4.7500)},{\sy*(-0.1698)})
	--({\sx*(4.7600)},{\sy*(-0.1554)})
	--({\sx*(4.7700)},{\sy*(-0.1336)})
	--({\sx*(4.7800)},{\sy*(-0.1054)})
	--({\sx*(4.7900)},{\sy*(-0.0723)})
	--({\sx*(4.8000)},{\sy*(-0.0361)})
	--({\sx*(4.8100)},{\sy*(0.0011)})
	--({\sx*(4.8200)},{\sy*(0.0370)})
	--({\sx*(4.8300)},{\sy*(0.0691)})
	--({\sx*(4.8400)},{\sy*(0.0951)})
	--({\sx*(4.8500)},{\sy*(0.1131)})
	--({\sx*(4.8600)},{\sy*(0.1213)})
	--({\sx*(4.8700)},{\sy*(0.1189)})
	--({\sx*(4.8800)},{\sy*(0.1057)})
	--({\sx*(4.8900)},{\sy*(0.0828)})
	--({\sx*(4.9000)},{\sy*(0.0521)})
	--({\sx*(4.9100)},{\sy*(0.0171)})
	--({\sx*(4.9200)},{\sy*(-0.0178)})
	--({\sx*(4.9300)},{\sy*(-0.0477)})
	--({\sx*(4.9400)},{\sy*(-0.0671)})
	--({\sx*(4.9500)},{\sy*(-0.0716)})
	--({\sx*(4.9600)},{\sy*(-0.0587)})
	--({\sx*(4.9700)},{\sy*(-0.0303)})
	--({\sx*(4.9800)},{\sy*(0.0047)})
	--({\sx*(4.9900)},{\sy*(0.0266)})
	--({\sx*(5.0000)},{\sy*(0.0000)});
}
\def\relfehlerl{
\draw[color=blue,line width=1.4pt,line join=round] ({\sx*(0.000)},{\sy*(0.0000)})
	--({\sx*(0.0100)},{\sy*(-0.0000)})
	--({\sx*(0.0200)},{\sy*(-0.0000)})
	--({\sx*(0.0300)},{\sy*(0.0000)})
	--({\sx*(0.0400)},{\sy*(0.0000)})
	--({\sx*(0.0500)},{\sy*(0.0000)})
	--({\sx*(0.0600)},{\sy*(0.0000)})
	--({\sx*(0.0700)},{\sy*(0.0000)})
	--({\sx*(0.0800)},{\sy*(0.0000)})
	--({\sx*(0.0900)},{\sy*(-0.0000)})
	--({\sx*(0.1000)},{\sy*(-0.0000)})
	--({\sx*(0.1100)},{\sy*(-0.0000)})
	--({\sx*(0.1200)},{\sy*(-0.0000)})
	--({\sx*(0.1300)},{\sy*(-0.0000)})
	--({\sx*(0.1400)},{\sy*(-0.0000)})
	--({\sx*(0.1500)},{\sy*(-0.0000)})
	--({\sx*(0.1600)},{\sy*(-0.0000)})
	--({\sx*(0.1700)},{\sy*(-0.0000)})
	--({\sx*(0.1800)},{\sy*(-0.0000)})
	--({\sx*(0.1900)},{\sy*(-0.0000)})
	--({\sx*(0.2000)},{\sy*(0.0000)})
	--({\sx*(0.2100)},{\sy*(0.0000)})
	--({\sx*(0.2200)},{\sy*(0.0000)})
	--({\sx*(0.2300)},{\sy*(0.0000)})
	--({\sx*(0.2400)},{\sy*(0.0000)})
	--({\sx*(0.2500)},{\sy*(0.0000)})
	--({\sx*(0.2600)},{\sy*(0.0000)})
	--({\sx*(0.2700)},{\sy*(0.0000)})
	--({\sx*(0.2800)},{\sy*(0.0000)})
	--({\sx*(0.2900)},{\sy*(0.0000)})
	--({\sx*(0.3000)},{\sy*(0.0000)})
	--({\sx*(0.3100)},{\sy*(0.0000)})
	--({\sx*(0.3200)},{\sy*(0.0000)})
	--({\sx*(0.3300)},{\sy*(0.0000)})
	--({\sx*(0.3400)},{\sy*(-0.0000)})
	--({\sx*(0.3500)},{\sy*(-0.0000)})
	--({\sx*(0.3600)},{\sy*(-0.0000)})
	--({\sx*(0.3700)},{\sy*(-0.0000)})
	--({\sx*(0.3800)},{\sy*(-0.0000)})
	--({\sx*(0.3900)},{\sy*(-0.0000)})
	--({\sx*(0.4000)},{\sy*(-0.0000)})
	--({\sx*(0.4100)},{\sy*(-0.0000)})
	--({\sx*(0.4200)},{\sy*(-0.0000)})
	--({\sx*(0.4300)},{\sy*(-0.0000)})
	--({\sx*(0.4400)},{\sy*(-0.0000)})
	--({\sx*(0.4500)},{\sy*(-0.0000)})
	--({\sx*(0.4600)},{\sy*(-0.0000)})
	--({\sx*(0.4700)},{\sy*(-0.0000)})
	--({\sx*(0.4800)},{\sy*(-0.0000)})
	--({\sx*(0.4900)},{\sy*(-0.0000)})
	--({\sx*(0.5000)},{\sy*(-0.0000)})
	--({\sx*(0.5100)},{\sy*(-0.0000)})
	--({\sx*(0.5200)},{\sy*(0.0000)})
	--({\sx*(0.5300)},{\sy*(0.0000)})
	--({\sx*(0.5400)},{\sy*(0.0000)})
	--({\sx*(0.5500)},{\sy*(0.0000)})
	--({\sx*(0.5600)},{\sy*(0.0000)})
	--({\sx*(0.5700)},{\sy*(0.0000)})
	--({\sx*(0.5800)},{\sy*(0.0000)})
	--({\sx*(0.5900)},{\sy*(0.0000)})
	--({\sx*(0.6000)},{\sy*(0.0000)})
	--({\sx*(0.6100)},{\sy*(0.0000)})
	--({\sx*(0.6200)},{\sy*(0.0000)})
	--({\sx*(0.6300)},{\sy*(0.0000)})
	--({\sx*(0.6400)},{\sy*(0.0000)})
	--({\sx*(0.6500)},{\sy*(0.0000)})
	--({\sx*(0.6600)},{\sy*(0.0000)})
	--({\sx*(0.6700)},{\sy*(0.0000)})
	--({\sx*(0.6800)},{\sy*(0.0000)})
	--({\sx*(0.6900)},{\sy*(0.0000)})
	--({\sx*(0.7000)},{\sy*(0.0000)})
	--({\sx*(0.7100)},{\sy*(0.0000)})
	--({\sx*(0.7200)},{\sy*(0.0000)})
	--({\sx*(0.7300)},{\sy*(0.0000)})
	--({\sx*(0.7400)},{\sy*(-0.0000)})
	--({\sx*(0.7500)},{\sy*(-0.0000)})
	--({\sx*(0.7600)},{\sy*(-0.0000)})
	--({\sx*(0.7700)},{\sy*(-0.0000)})
	--({\sx*(0.7800)},{\sy*(-0.0000)})
	--({\sx*(0.7900)},{\sy*(-0.0000)})
	--({\sx*(0.8000)},{\sy*(-0.0000)})
	--({\sx*(0.8100)},{\sy*(-0.0000)})
	--({\sx*(0.8200)},{\sy*(-0.0000)})
	--({\sx*(0.8300)},{\sy*(-0.0000)})
	--({\sx*(0.8400)},{\sy*(-0.0000)})
	--({\sx*(0.8500)},{\sy*(-0.0000)})
	--({\sx*(0.8600)},{\sy*(-0.0000)})
	--({\sx*(0.8700)},{\sy*(-0.0000)})
	--({\sx*(0.8800)},{\sy*(-0.0000)})
	--({\sx*(0.8900)},{\sy*(-0.0000)})
	--({\sx*(0.9000)},{\sy*(-0.0000)})
	--({\sx*(0.9100)},{\sy*(-0.0000)})
	--({\sx*(0.9200)},{\sy*(-0.0000)})
	--({\sx*(0.9300)},{\sy*(-0.0000)})
	--({\sx*(0.9400)},{\sy*(-0.0000)})
	--({\sx*(0.9500)},{\sy*(-0.0000)})
	--({\sx*(0.9600)},{\sy*(-0.0000)})
	--({\sx*(0.9700)},{\sy*(-0.0000)})
	--({\sx*(0.9800)},{\sy*(0.0000)})
	--({\sx*(0.9900)},{\sy*(0.0000)})
	--({\sx*(1.0000)},{\sy*(0.0000)})
	--({\sx*(1.0100)},{\sy*(0.0000)})
	--({\sx*(1.0200)},{\sy*(0.0000)})
	--({\sx*(1.0300)},{\sy*(0.0000)})
	--({\sx*(1.0400)},{\sy*(0.0000)})
	--({\sx*(1.0500)},{\sy*(0.0000)})
	--({\sx*(1.0600)},{\sy*(0.0000)})
	--({\sx*(1.0700)},{\sy*(0.0000)})
	--({\sx*(1.0800)},{\sy*(0.0000)})
	--({\sx*(1.0900)},{\sy*(0.0000)})
	--({\sx*(1.1000)},{\sy*(0.0000)})
	--({\sx*(1.1100)},{\sy*(0.0000)})
	--({\sx*(1.1200)},{\sy*(0.0000)})
	--({\sx*(1.1300)},{\sy*(0.0000)})
	--({\sx*(1.1400)},{\sy*(0.0000)})
	--({\sx*(1.1500)},{\sy*(0.0000)})
	--({\sx*(1.1600)},{\sy*(0.0000)})
	--({\sx*(1.1700)},{\sy*(0.0000)})
	--({\sx*(1.1800)},{\sy*(0.0000)})
	--({\sx*(1.1900)},{\sy*(0.0000)})
	--({\sx*(1.2000)},{\sy*(0.0000)})
	--({\sx*(1.2100)},{\sy*(0.0000)})
	--({\sx*(1.2200)},{\sy*(0.0000)})
	--({\sx*(1.2300)},{\sy*(0.0000)})
	--({\sx*(1.2400)},{\sy*(0.0000)})
	--({\sx*(1.2500)},{\sy*(0.0000)})
	--({\sx*(1.2600)},{\sy*(-0.0000)})
	--({\sx*(1.2700)},{\sy*(-0.0000)})
	--({\sx*(1.2800)},{\sy*(-0.0000)})
	--({\sx*(1.2900)},{\sy*(-0.0000)})
	--({\sx*(1.3000)},{\sy*(-0.0000)})
	--({\sx*(1.3100)},{\sy*(-0.0000)})
	--({\sx*(1.3200)},{\sy*(-0.0000)})
	--({\sx*(1.3300)},{\sy*(-0.0000)})
	--({\sx*(1.3400)},{\sy*(-0.0000)})
	--({\sx*(1.3500)},{\sy*(-0.0000)})
	--({\sx*(1.3600)},{\sy*(-0.0000)})
	--({\sx*(1.3700)},{\sy*(-0.0000)})
	--({\sx*(1.3800)},{\sy*(-0.0000)})
	--({\sx*(1.3900)},{\sy*(-0.0000)})
	--({\sx*(1.4000)},{\sy*(-0.0000)})
	--({\sx*(1.4100)},{\sy*(-0.0000)})
	--({\sx*(1.4200)},{\sy*(-0.0000)})
	--({\sx*(1.4300)},{\sy*(-0.0000)})
	--({\sx*(1.4400)},{\sy*(-0.0000)})
	--({\sx*(1.4500)},{\sy*(-0.0000)})
	--({\sx*(1.4600)},{\sy*(-0.0000)})
	--({\sx*(1.4700)},{\sy*(-0.0000)})
	--({\sx*(1.4800)},{\sy*(-0.0000)})
	--({\sx*(1.4900)},{\sy*(-0.0000)})
	--({\sx*(1.5000)},{\sy*(-0.0000)})
	--({\sx*(1.5100)},{\sy*(-0.0000)})
	--({\sx*(1.5200)},{\sy*(-0.0000)})
	--({\sx*(1.5300)},{\sy*(-0.0000)})
	--({\sx*(1.5400)},{\sy*(-0.0000)})
	--({\sx*(1.5500)},{\sy*(0.0000)})
	--({\sx*(1.5600)},{\sy*(0.0000)})
	--({\sx*(1.5700)},{\sy*(0.0000)})
	--({\sx*(1.5800)},{\sy*(0.0000)})
	--({\sx*(1.5900)},{\sy*(0.0000)})
	--({\sx*(1.6000)},{\sy*(0.0000)})
	--({\sx*(1.6100)},{\sy*(0.0000)})
	--({\sx*(1.6200)},{\sy*(0.0000)})
	--({\sx*(1.6300)},{\sy*(0.0000)})
	--({\sx*(1.6400)},{\sy*(0.0000)})
	--({\sx*(1.6500)},{\sy*(0.0000)})
	--({\sx*(1.6600)},{\sy*(0.0000)})
	--({\sx*(1.6700)},{\sy*(0.0000)})
	--({\sx*(1.6800)},{\sy*(0.0000)})
	--({\sx*(1.6900)},{\sy*(0.0000)})
	--({\sx*(1.7000)},{\sy*(0.0000)})
	--({\sx*(1.7100)},{\sy*(0.0000)})
	--({\sx*(1.7200)},{\sy*(0.0000)})
	--({\sx*(1.7300)},{\sy*(0.0000)})
	--({\sx*(1.7400)},{\sy*(0.0000)})
	--({\sx*(1.7500)},{\sy*(0.0000)})
	--({\sx*(1.7600)},{\sy*(0.0000)})
	--({\sx*(1.7700)},{\sy*(0.0000)})
	--({\sx*(1.7800)},{\sy*(0.0000)})
	--({\sx*(1.7900)},{\sy*(0.0000)})
	--({\sx*(1.8000)},{\sy*(0.0000)})
	--({\sx*(1.8100)},{\sy*(0.0000)})
	--({\sx*(1.8200)},{\sy*(0.0000)})
	--({\sx*(1.8300)},{\sy*(0.0000)})
	--({\sx*(1.8400)},{\sy*(0.0000)})
	--({\sx*(1.8500)},{\sy*(0.0000)})
	--({\sx*(1.8600)},{\sy*(-0.0000)})
	--({\sx*(1.8700)},{\sy*(-0.0000)})
	--({\sx*(1.8800)},{\sy*(-0.0000)})
	--({\sx*(1.8900)},{\sy*(-0.0000)})
	--({\sx*(1.9000)},{\sy*(-0.0000)})
	--({\sx*(1.9100)},{\sy*(-0.0000)})
	--({\sx*(1.9200)},{\sy*(-0.0000)})
	--({\sx*(1.9300)},{\sy*(-0.0000)})
	--({\sx*(1.9400)},{\sy*(-0.0000)})
	--({\sx*(1.9500)},{\sy*(-0.0000)})
	--({\sx*(1.9600)},{\sy*(-0.0000)})
	--({\sx*(1.9700)},{\sy*(-0.0000)})
	--({\sx*(1.9800)},{\sy*(-0.0000)})
	--({\sx*(1.9900)},{\sy*(-0.0000)})
	--({\sx*(2.0000)},{\sy*(-0.0000)})
	--({\sx*(2.0100)},{\sy*(-0.0000)})
	--({\sx*(2.0200)},{\sy*(-0.0000)})
	--({\sx*(2.0300)},{\sy*(-0.0000)})
	--({\sx*(2.0400)},{\sy*(-0.0000)})
	--({\sx*(2.0500)},{\sy*(-0.0000)})
	--({\sx*(2.0600)},{\sy*(-0.0000)})
	--({\sx*(2.0700)},{\sy*(-0.0000)})
	--({\sx*(2.0800)},{\sy*(-0.0000)})
	--({\sx*(2.0900)},{\sy*(-0.0000)})
	--({\sx*(2.1000)},{\sy*(-0.0000)})
	--({\sx*(2.1100)},{\sy*(-0.0000)})
	--({\sx*(2.1200)},{\sy*(-0.0000)})
	--({\sx*(2.1300)},{\sy*(-0.0000)})
	--({\sx*(2.1400)},{\sy*(-0.0000)})
	--({\sx*(2.1500)},{\sy*(-0.0000)})
	--({\sx*(2.1600)},{\sy*(-0.0000)})
	--({\sx*(2.1700)},{\sy*(-0.0000)})
	--({\sx*(2.1800)},{\sy*(0.0000)})
	--({\sx*(2.1900)},{\sy*(0.0000)})
	--({\sx*(2.2000)},{\sy*(0.0000)})
	--({\sx*(2.2100)},{\sy*(0.0000)})
	--({\sx*(2.2200)},{\sy*(0.0000)})
	--({\sx*(2.2300)},{\sy*(0.0000)})
	--({\sx*(2.2400)},{\sy*(0.0000)})
	--({\sx*(2.2500)},{\sy*(0.0000)})
	--({\sx*(2.2600)},{\sy*(0.0000)})
	--({\sx*(2.2700)},{\sy*(0.0000)})
	--({\sx*(2.2800)},{\sy*(0.0000)})
	--({\sx*(2.2900)},{\sy*(0.0000)})
	--({\sx*(2.3000)},{\sy*(0.0000)})
	--({\sx*(2.3100)},{\sy*(0.0000)})
	--({\sx*(2.3200)},{\sy*(0.0000)})
	--({\sx*(2.3300)},{\sy*(0.0000)})
	--({\sx*(2.3400)},{\sy*(0.0000)})
	--({\sx*(2.3500)},{\sy*(0.0000)})
	--({\sx*(2.3600)},{\sy*(0.0000)})
	--({\sx*(2.3700)},{\sy*(0.0000)})
	--({\sx*(2.3800)},{\sy*(0.0000)})
	--({\sx*(2.3900)},{\sy*(0.0000)})
	--({\sx*(2.4000)},{\sy*(0.0000)})
	--({\sx*(2.4100)},{\sy*(0.0000)})
	--({\sx*(2.4200)},{\sy*(0.0000)})
	--({\sx*(2.4300)},{\sy*(0.0000)})
	--({\sx*(2.4400)},{\sy*(0.0000)})
	--({\sx*(2.4500)},{\sy*(0.0000)})
	--({\sx*(2.4600)},{\sy*(0.0000)})
	--({\sx*(2.4700)},{\sy*(0.0000)})
	--({\sx*(2.4800)},{\sy*(0.0000)})
	--({\sx*(2.4900)},{\sy*(0.0000)})
	--({\sx*(2.5000)},{\sy*(0.0000)})
	--({\sx*(2.5100)},{\sy*(-0.0000)})
	--({\sx*(2.5200)},{\sy*(-0.0000)})
	--({\sx*(2.5300)},{\sy*(-0.0000)})
	--({\sx*(2.5400)},{\sy*(-0.0000)})
	--({\sx*(2.5500)},{\sy*(-0.0000)})
	--({\sx*(2.5600)},{\sy*(-0.0000)})
	--({\sx*(2.5700)},{\sy*(-0.0000)})
	--({\sx*(2.5800)},{\sy*(-0.0000)})
	--({\sx*(2.5900)},{\sy*(-0.0000)})
	--({\sx*(2.6000)},{\sy*(-0.0000)})
	--({\sx*(2.6100)},{\sy*(-0.0000)})
	--({\sx*(2.6200)},{\sy*(-0.0000)})
	--({\sx*(2.6300)},{\sy*(-0.0000)})
	--({\sx*(2.6400)},{\sy*(-0.0000)})
	--({\sx*(2.6500)},{\sy*(-0.0000)})
	--({\sx*(2.6600)},{\sy*(-0.0000)})
	--({\sx*(2.6700)},{\sy*(-0.0000)})
	--({\sx*(2.6800)},{\sy*(-0.0000)})
	--({\sx*(2.6900)},{\sy*(-0.0000)})
	--({\sx*(2.7000)},{\sy*(-0.0000)})
	--({\sx*(2.7100)},{\sy*(-0.0000)})
	--({\sx*(2.7200)},{\sy*(-0.0000)})
	--({\sx*(2.7300)},{\sy*(-0.0000)})
	--({\sx*(2.7400)},{\sy*(-0.0000)})
	--({\sx*(2.7500)},{\sy*(-0.0000)})
	--({\sx*(2.7600)},{\sy*(-0.0000)})
	--({\sx*(2.7700)},{\sy*(-0.0000)})
	--({\sx*(2.7800)},{\sy*(-0.0000)})
	--({\sx*(2.7900)},{\sy*(-0.0000)})
	--({\sx*(2.8000)},{\sy*(-0.0000)})
	--({\sx*(2.8100)},{\sy*(-0.0000)})
	--({\sx*(2.8200)},{\sy*(-0.0000)})
	--({\sx*(2.8300)},{\sy*(0.0000)})
	--({\sx*(2.8400)},{\sy*(0.0000)})
	--({\sx*(2.8500)},{\sy*(0.0000)})
	--({\sx*(2.8600)},{\sy*(0.0000)})
	--({\sx*(2.8700)},{\sy*(0.0000)})
	--({\sx*(2.8800)},{\sy*(0.0000)})
	--({\sx*(2.8900)},{\sy*(0.0000)})
	--({\sx*(2.9000)},{\sy*(0.0000)})
	--({\sx*(2.9100)},{\sy*(0.0000)})
	--({\sx*(2.9200)},{\sy*(0.0000)})
	--({\sx*(2.9300)},{\sy*(0.0000)})
	--({\sx*(2.9400)},{\sy*(0.0000)})
	--({\sx*(2.9500)},{\sy*(0.0000)})
	--({\sx*(2.9600)},{\sy*(0.0000)})
	--({\sx*(2.9700)},{\sy*(0.0000)})
	--({\sx*(2.9800)},{\sy*(0.0000)})
	--({\sx*(2.9900)},{\sy*(0.0000)})
	--({\sx*(3.0000)},{\sy*(0.0000)})
	--({\sx*(3.0100)},{\sy*(0.0000)})
	--({\sx*(3.0200)},{\sy*(0.0000)})
	--({\sx*(3.0300)},{\sy*(0.0000)})
	--({\sx*(3.0400)},{\sy*(0.0000)})
	--({\sx*(3.0500)},{\sy*(0.0000)})
	--({\sx*(3.0600)},{\sy*(0.0000)})
	--({\sx*(3.0700)},{\sy*(0.0000)})
	--({\sx*(3.0800)},{\sy*(0.0000)})
	--({\sx*(3.0900)},{\sy*(0.0000)})
	--({\sx*(3.1000)},{\sy*(0.0000)})
	--({\sx*(3.1100)},{\sy*(0.0000)})
	--({\sx*(3.1200)},{\sy*(0.0000)})
	--({\sx*(3.1300)},{\sy*(0.0000)})
	--({\sx*(3.1400)},{\sy*(0.0000)})
	--({\sx*(3.1500)},{\sy*(-0.0000)})
	--({\sx*(3.1600)},{\sy*(-0.0000)})
	--({\sx*(3.1700)},{\sy*(-0.0000)})
	--({\sx*(3.1800)},{\sy*(-0.0000)})
	--({\sx*(3.1900)},{\sy*(-0.0000)})
	--({\sx*(3.2000)},{\sy*(-0.0000)})
	--({\sx*(3.2100)},{\sy*(-0.0000)})
	--({\sx*(3.2200)},{\sy*(-0.0000)})
	--({\sx*(3.2300)},{\sy*(-0.0000)})
	--({\sx*(3.2400)},{\sy*(-0.0000)})
	--({\sx*(3.2500)},{\sy*(-0.0000)})
	--({\sx*(3.2600)},{\sy*(-0.0000)})
	--({\sx*(3.2700)},{\sy*(-0.0000)})
	--({\sx*(3.2800)},{\sy*(-0.0000)})
	--({\sx*(3.2900)},{\sy*(-0.0000)})
	--({\sx*(3.3000)},{\sy*(-0.0000)})
	--({\sx*(3.3100)},{\sy*(-0.0000)})
	--({\sx*(3.3200)},{\sy*(-0.0000)})
	--({\sx*(3.3300)},{\sy*(-0.0000)})
	--({\sx*(3.3400)},{\sy*(-0.0000)})
	--({\sx*(3.3500)},{\sy*(-0.0000)})
	--({\sx*(3.3600)},{\sy*(-0.0000)})
	--({\sx*(3.3700)},{\sy*(-0.0000)})
	--({\sx*(3.3800)},{\sy*(-0.0000)})
	--({\sx*(3.3900)},{\sy*(-0.0000)})
	--({\sx*(3.4000)},{\sy*(-0.0000)})
	--({\sx*(3.4100)},{\sy*(-0.0000)})
	--({\sx*(3.4200)},{\sy*(-0.0000)})
	--({\sx*(3.4300)},{\sy*(-0.0000)})
	--({\sx*(3.4400)},{\sy*(-0.0000)})
	--({\sx*(3.4500)},{\sy*(-0.0000)})
	--({\sx*(3.4600)},{\sy*(0.0000)})
	--({\sx*(3.4700)},{\sy*(0.0000)})
	--({\sx*(3.4800)},{\sy*(0.0000)})
	--({\sx*(3.4900)},{\sy*(0.0000)})
	--({\sx*(3.5000)},{\sy*(0.0000)})
	--({\sx*(3.5100)},{\sy*(0.0000)})
	--({\sx*(3.5200)},{\sy*(0.0000)})
	--({\sx*(3.5300)},{\sy*(0.0000)})
	--({\sx*(3.5400)},{\sy*(0.0000)})
	--({\sx*(3.5500)},{\sy*(0.0000)})
	--({\sx*(3.5600)},{\sy*(0.0000)})
	--({\sx*(3.5700)},{\sy*(0.0000)})
	--({\sx*(3.5800)},{\sy*(0.0000)})
	--({\sx*(3.5900)},{\sy*(0.0000)})
	--({\sx*(3.6000)},{\sy*(0.0000)})
	--({\sx*(3.6100)},{\sy*(0.0000)})
	--({\sx*(3.6200)},{\sy*(0.0000)})
	--({\sx*(3.6300)},{\sy*(0.0000)})
	--({\sx*(3.6400)},{\sy*(0.0000)})
	--({\sx*(3.6500)},{\sy*(0.0000)})
	--({\sx*(3.6600)},{\sy*(0.0000)})
	--({\sx*(3.6700)},{\sy*(0.0000)})
	--({\sx*(3.6800)},{\sy*(0.0000)})
	--({\sx*(3.6900)},{\sy*(0.0000)})
	--({\sx*(3.7000)},{\sy*(0.0000)})
	--({\sx*(3.7100)},{\sy*(0.0000)})
	--({\sx*(3.7200)},{\sy*(0.0000)})
	--({\sx*(3.7300)},{\sy*(0.0000)})
	--({\sx*(3.7400)},{\sy*(0.0000)})
	--({\sx*(3.7500)},{\sy*(0.0000)})
	--({\sx*(3.7600)},{\sy*(-0.0000)})
	--({\sx*(3.7700)},{\sy*(-0.0000)})
	--({\sx*(3.7800)},{\sy*(-0.0000)})
	--({\sx*(3.7900)},{\sy*(-0.0000)})
	--({\sx*(3.8000)},{\sy*(-0.0000)})
	--({\sx*(3.8100)},{\sy*(-0.0000)})
	--({\sx*(3.8200)},{\sy*(-0.0000)})
	--({\sx*(3.8300)},{\sy*(-0.0000)})
	--({\sx*(3.8400)},{\sy*(-0.0000)})
	--({\sx*(3.8500)},{\sy*(-0.0000)})
	--({\sx*(3.8600)},{\sy*(-0.0000)})
	--({\sx*(3.8700)},{\sy*(-0.0000)})
	--({\sx*(3.8800)},{\sy*(-0.0000)})
	--({\sx*(3.8900)},{\sy*(-0.0000)})
	--({\sx*(3.9000)},{\sy*(-0.0000)})
	--({\sx*(3.9100)},{\sy*(-0.0000)})
	--({\sx*(3.9200)},{\sy*(-0.0000)})
	--({\sx*(3.9300)},{\sy*(-0.0000)})
	--({\sx*(3.9400)},{\sy*(-0.0000)})
	--({\sx*(3.9500)},{\sy*(-0.0000)})
	--({\sx*(3.9600)},{\sy*(-0.0000)})
	--({\sx*(3.9700)},{\sy*(-0.0000)})
	--({\sx*(3.9800)},{\sy*(-0.0000)})
	--({\sx*(3.9900)},{\sy*(-0.0000)})
	--({\sx*(4.0000)},{\sy*(-0.0000)})
	--({\sx*(4.0100)},{\sy*(-0.0000)})
	--({\sx*(4.0200)},{\sy*(-0.0000)})
	--({\sx*(4.0300)},{\sy*(0.0000)})
	--({\sx*(4.0400)},{\sy*(0.0000)})
	--({\sx*(4.0500)},{\sy*(0.0000)})
	--({\sx*(4.0600)},{\sy*(0.0000)})
	--({\sx*(4.0700)},{\sy*(0.0000)})
	--({\sx*(4.0800)},{\sy*(0.0000)})
	--({\sx*(4.0900)},{\sy*(0.0000)})
	--({\sx*(4.1000)},{\sy*(0.0000)})
	--({\sx*(4.1100)},{\sy*(0.0000)})
	--({\sx*(4.1200)},{\sy*(0.0000)})
	--({\sx*(4.1300)},{\sy*(0.0000)})
	--({\sx*(4.1400)},{\sy*(0.0000)})
	--({\sx*(4.1500)},{\sy*(0.0000)})
	--({\sx*(4.1600)},{\sy*(0.0000)})
	--({\sx*(4.1700)},{\sy*(0.0000)})
	--({\sx*(4.1800)},{\sy*(0.0000)})
	--({\sx*(4.1900)},{\sy*(0.0000)})
	--({\sx*(4.2000)},{\sy*(0.0000)})
	--({\sx*(4.2100)},{\sy*(0.0000)})
	--({\sx*(4.2200)},{\sy*(0.0000)})
	--({\sx*(4.2300)},{\sy*(0.0000)})
	--({\sx*(4.2400)},{\sy*(0.0000)})
	--({\sx*(4.2500)},{\sy*(0.0000)})
	--({\sx*(4.2600)},{\sy*(0.0000)})
	--({\sx*(4.2700)},{\sy*(-0.0000)})
	--({\sx*(4.2800)},{\sy*(-0.0000)})
	--({\sx*(4.2900)},{\sy*(-0.0000)})
	--({\sx*(4.3000)},{\sy*(-0.0000)})
	--({\sx*(4.3100)},{\sy*(-0.0000)})
	--({\sx*(4.3200)},{\sy*(-0.0000)})
	--({\sx*(4.3300)},{\sy*(-0.0000)})
	--({\sx*(4.3400)},{\sy*(-0.0000)})
	--({\sx*(4.3500)},{\sy*(-0.0000)})
	--({\sx*(4.3600)},{\sy*(-0.0000)})
	--({\sx*(4.3700)},{\sy*(-0.0000)})
	--({\sx*(4.3800)},{\sy*(-0.0000)})
	--({\sx*(4.3900)},{\sy*(-0.0000)})
	--({\sx*(4.4000)},{\sy*(-0.0000)})
	--({\sx*(4.4100)},{\sy*(-0.0000)})
	--({\sx*(4.4200)},{\sy*(-0.0000)})
	--({\sx*(4.4300)},{\sy*(-0.0000)})
	--({\sx*(4.4400)},{\sy*(-0.0000)})
	--({\sx*(4.4500)},{\sy*(-0.0000)})
	--({\sx*(4.4600)},{\sy*(-0.0000)})
	--({\sx*(4.4700)},{\sy*(-0.0000)})
	--({\sx*(4.4800)},{\sy*(-0.0000)})
	--({\sx*(4.4900)},{\sy*(0.0000)})
	--({\sx*(4.5000)},{\sy*(0.0000)})
	--({\sx*(4.5100)},{\sy*(0.0000)})
	--({\sx*(4.5200)},{\sy*(0.0000)})
	--({\sx*(4.5300)},{\sy*(0.0000)})
	--({\sx*(4.5400)},{\sy*(0.0000)})
	--({\sx*(4.5500)},{\sy*(0.0000)})
	--({\sx*(4.5600)},{\sy*(0.0000)})
	--({\sx*(4.5700)},{\sy*(0.0000)})
	--({\sx*(4.5800)},{\sy*(0.0000)})
	--({\sx*(4.5900)},{\sy*(0.0000)})
	--({\sx*(4.6000)},{\sy*(0.0000)})
	--({\sx*(4.6100)},{\sy*(0.0000)})
	--({\sx*(4.6200)},{\sy*(0.0000)})
	--({\sx*(4.6300)},{\sy*(0.0000)})
	--({\sx*(4.6400)},{\sy*(0.0000)})
	--({\sx*(4.6500)},{\sy*(0.0000)})
	--({\sx*(4.6600)},{\sy*(0.0000)})
	--({\sx*(4.6700)},{\sy*(-0.0000)})
	--({\sx*(4.6800)},{\sy*(-0.0000)})
	--({\sx*(4.6900)},{\sy*(-0.0000)})
	--({\sx*(4.7000)},{\sy*(-0.0000)})
	--({\sx*(4.7100)},{\sy*(-0.0000)})
	--({\sx*(4.7200)},{\sy*(-0.0000)})
	--({\sx*(4.7300)},{\sy*(-0.0000)})
	--({\sx*(4.7400)},{\sy*(-0.0000)})
	--({\sx*(4.7500)},{\sy*(-0.0000)})
	--({\sx*(4.7600)},{\sy*(-0.0000)})
	--({\sx*(4.7700)},{\sy*(-0.0000)})
	--({\sx*(4.7800)},{\sy*(-0.0000)})
	--({\sx*(4.7900)},{\sy*(-0.0000)})
	--({\sx*(4.8000)},{\sy*(-0.0000)})
	--({\sx*(4.8100)},{\sy*(0.0000)})
	--({\sx*(4.8200)},{\sy*(0.0000)})
	--({\sx*(4.8300)},{\sy*(0.0000)})
	--({\sx*(4.8400)},{\sy*(0.0000)})
	--({\sx*(4.8500)},{\sy*(0.0000)})
	--({\sx*(4.8600)},{\sy*(0.0000)})
	--({\sx*(4.8700)},{\sy*(0.0000)})
	--({\sx*(4.8800)},{\sy*(0.0000)})
	--({\sx*(4.8900)},{\sy*(0.0000)})
	--({\sx*(4.9000)},{\sy*(0.0000)})
	--({\sx*(4.9100)},{\sy*(0.0000)})
	--({\sx*(4.9200)},{\sy*(-0.0000)})
	--({\sx*(4.9300)},{\sy*(-0.0000)})
	--({\sx*(4.9400)},{\sy*(-0.0000)})
	--({\sx*(4.9500)},{\sy*(-0.0000)})
	--({\sx*(4.9600)},{\sy*(-0.0000)})
	--({\sx*(4.9700)},{\sy*(-0.0000)})
	--({\sx*(4.9800)},{\sy*(0.0000)})
	--({\sx*(4.9900)},{\sy*(0.0000)})
	--({\sx*(5.0000)},{\sy*(0.0000)});
}
\def\xwertem{
\fill[color=red] (0.0000,0) circle[radius={0.07/\skala}];
\fill[color=white] (0.0000,0) circle[radius={0.05/\skala}];
\fill[color=red] (0.0182,0) circle[radius={0.07/\skala}];
\fill[color=white] (0.0182,0) circle[radius={0.05/\skala}];
\fill[color=red] (0.0726,0) circle[radius={0.07/\skala}];
\fill[color=white] (0.0726,0) circle[radius={0.05/\skala}];
\fill[color=red] (0.1625,0) circle[radius={0.07/\skala}];
\fill[color=white] (0.1625,0) circle[radius={0.05/\skala}];
\fill[color=red] (0.2864,0) circle[radius={0.07/\skala}];
\fill[color=white] (0.2864,0) circle[radius={0.05/\skala}];
\fill[color=red] (0.4425,0) circle[radius={0.07/\skala}];
\fill[color=white] (0.4425,0) circle[radius={0.05/\skala}];
\fill[color=red] (0.6287,0) circle[radius={0.07/\skala}];
\fill[color=white] (0.6287,0) circle[radius={0.05/\skala}];
\fill[color=red] (0.8422,0) circle[radius={0.07/\skala}];
\fill[color=white] (0.8422,0) circle[radius={0.05/\skala}];
\fill[color=red] (1.0798,0) circle[radius={0.07/\skala}];
\fill[color=white] (1.0798,0) circle[radius={0.05/\skala}];
\fill[color=red] (1.3382,0) circle[radius={0.07/\skala}];
\fill[color=white] (1.3382,0) circle[radius={0.05/\skala}];
\fill[color=red] (1.6135,0) circle[radius={0.07/\skala}];
\fill[color=white] (1.6135,0) circle[radius={0.05/\skala}];
\fill[color=red] (1.9017,0) circle[radius={0.07/\skala}];
\fill[color=white] (1.9017,0) circle[radius={0.05/\skala}];
\fill[color=red] (2.1987,0) circle[radius={0.07/\skala}];
\fill[color=white] (2.1987,0) circle[radius={0.05/\skala}];
\fill[color=red] (2.5000,0) circle[radius={0.07/\skala}];
\fill[color=white] (2.5000,0) circle[radius={0.05/\skala}];
\fill[color=red] (2.8013,0) circle[radius={0.07/\skala}];
\fill[color=white] (2.8013,0) circle[radius={0.05/\skala}];
\fill[color=red] (3.0983,0) circle[radius={0.07/\skala}];
\fill[color=white] (3.0983,0) circle[radius={0.05/\skala}];
\fill[color=red] (3.3865,0) circle[radius={0.07/\skala}];
\fill[color=white] (3.3865,0) circle[radius={0.05/\skala}];
\fill[color=red] (3.6618,0) circle[radius={0.07/\skala}];
\fill[color=white] (3.6618,0) circle[radius={0.05/\skala}];
\fill[color=red] (3.9202,0) circle[radius={0.07/\skala}];
\fill[color=white] (3.9202,0) circle[radius={0.05/\skala}];
\fill[color=red] (4.1578,0) circle[radius={0.07/\skala}];
\fill[color=white] (4.1578,0) circle[radius={0.05/\skala}];
\fill[color=red] (4.3713,0) circle[radius={0.07/\skala}];
\fill[color=white] (4.3713,0) circle[radius={0.05/\skala}];
\fill[color=red] (4.5575,0) circle[radius={0.07/\skala}];
\fill[color=white] (4.5575,0) circle[radius={0.05/\skala}];
\fill[color=red] (4.7136,0) circle[radius={0.07/\skala}];
\fill[color=white] (4.7136,0) circle[radius={0.05/\skala}];
\fill[color=red] (4.8375,0) circle[radius={0.07/\skala}];
\fill[color=white] (4.8375,0) circle[radius={0.05/\skala}];
\fill[color=red] (4.9274,0) circle[radius={0.07/\skala}];
\fill[color=white] (4.9274,0) circle[radius={0.05/\skala}];
\fill[color=red] (4.9818,0) circle[radius={0.07/\skala}];
\fill[color=white] (4.9818,0) circle[radius={0.05/\skala}];
\fill[color=red] (5.0000,0) circle[radius={0.07/\skala}];
\fill[color=white] (5.0000,0) circle[radius={0.05/\skala}];
}
\def\punktem{26}
\def\maxfehlerm{6.040\cdot 10^{-14}}
\def\fehlerm{
\draw[color=red,line width=1.4pt,line join=round] ({\sx*(0.000)},{\sy*(0.0000)})
	--({\sx*(0.0100)},{\sy*(0.1066)})
	--({\sx*(0.0200)},{\sy*(-0.0303)})
	--({\sx*(0.0300)},{\sy*(-0.1930)})
	--({\sx*(0.0400)},{\sy*(-0.2895)})
	--({\sx*(0.0500)},{\sy*(-0.2794)})
	--({\sx*(0.0600)},{\sy*(-0.1857)})
	--({\sx*(0.0700)},{\sy*(-0.0432)})
	--({\sx*(0.0800)},{\sy*(0.1222)})
	--({\sx*(0.0900)},{\sy*(0.2803)})
	--({\sx*(0.1000)},{\sy*(0.3925)})
	--({\sx*(0.1100)},{\sy*(0.4596)})
	--({\sx*(0.1200)},{\sy*(0.4669)})
	--({\sx*(0.1300)},{\sy*(0.4246)})
	--({\sx*(0.1400)},{\sy*(0.3235)})
	--({\sx*(0.1500)},{\sy*(0.1893)})
	--({\sx*(0.1600)},{\sy*(0.0386)})
	--({\sx*(0.1700)},{\sy*(-0.1222)})
	--({\sx*(0.1800)},{\sy*(-0.2840)})
	--({\sx*(0.1900)},{\sy*(-0.4219)})
	--({\sx*(0.2000)},{\sy*(-0.5285)})
	--({\sx*(0.2100)},{\sy*(-0.6011)})
	--({\sx*(0.2200)},{\sy*(-0.6443)})
	--({\sx*(0.2300)},{\sy*(-0.6314)})
	--({\sx*(0.2400)},{\sy*(-0.5855)})
	--({\sx*(0.2500)},{\sy*(-0.5037)})
	--({\sx*(0.2600)},{\sy*(-0.3888)})
	--({\sx*(0.2700)},{\sy*(-0.2564)})
	--({\sx*(0.2800)},{\sy*(-0.1029)})
	--({\sx*(0.2900)},{\sy*(0.0478)})
	--({\sx*(0.3000)},{\sy*(0.2142)})
	--({\sx*(0.3100)},{\sy*(0.3649)})
	--({\sx*(0.3200)},{\sy*(0.4972)})
	--({\sx*(0.3300)},{\sy*(0.6057)})
	--({\sx*(0.3400)},{\sy*(0.6967)})
	--({\sx*(0.3500)},{\sy*(0.7482)})
	--({\sx*(0.3600)},{\sy*(0.7711)})
	--({\sx*(0.3700)},{\sy*(0.7638)})
	--({\sx*(0.3800)},{\sy*(0.7390)})
	--({\sx*(0.3900)},{\sy*(0.6664)})
	--({\sx*(0.4000)},{\sy*(0.5790)})
	--({\sx*(0.4100)},{\sy*(0.4586)})
	--({\sx*(0.4200)},{\sy*(0.3382)})
	--({\sx*(0.4300)},{\sy*(0.1912)})
	--({\sx*(0.4400)},{\sy*(0.0423)})
	--({\sx*(0.4500)},{\sy*(-0.1176)})
	--({\sx*(0.4600)},{\sy*(-0.2629)})
	--({\sx*(0.4700)},{\sy*(-0.4053)})
	--({\sx*(0.4800)},{\sy*(-0.5377)})
	--({\sx*(0.4900)},{\sy*(-0.6544)})
	--({\sx*(0.5000)},{\sy*(-0.7436)})
	--({\sx*(0.5100)},{\sy*(-0.8153)})
	--({\sx*(0.5200)},{\sy*(-0.8585)})
	--({\sx*(0.5300)},{\sy*(-0.8906)})
	--({\sx*(0.5400)},{\sy*(-0.8860)})
	--({\sx*(0.5500)},{\sy*(-0.8539)})
	--({\sx*(0.5600)},{\sy*(-0.8143)})
	--({\sx*(0.5700)},{\sy*(-0.7353)})
	--({\sx*(0.5800)},{\sy*(-0.6434)})
	--({\sx*(0.5900)},{\sy*(-0.5239)})
	--({\sx*(0.6000)},{\sy*(-0.3980)})
	--({\sx*(0.6100)},{\sy*(-0.2711)})
	--({\sx*(0.6200)},{\sy*(-0.1213)})
	--({\sx*(0.6300)},{\sy*(0.0211)})
	--({\sx*(0.6400)},{\sy*(0.1636)})
	--({\sx*(0.6500)},{\sy*(0.3079)})
	--({\sx*(0.6600)},{\sy*(0.4412)})
	--({\sx*(0.6700)},{\sy*(0.5726)})
	--({\sx*(0.6800)},{\sy*(0.6783)})
	--({\sx*(0.6900)},{\sy*(0.7730)})
	--({\sx*(0.7000)},{\sy*(0.8465)})
	--({\sx*(0.7100)},{\sy*(0.9108)})
	--({\sx*(0.7200)},{\sy*(0.9485)})
	--({\sx*(0.7300)},{\sy*(0.9614)})
	--({\sx*(0.7400)},{\sy*(0.9586)})
	--({\sx*(0.7500)},{\sy*(0.9393)})
	--({\sx*(0.7600)},{\sy*(0.8998)})
	--({\sx*(0.7700)},{\sy*(0.8336)})
	--({\sx*(0.7800)},{\sy*(0.7518)})
	--({\sx*(0.7900)},{\sy*(0.6562)})
	--({\sx*(0.8000)},{\sy*(0.5542)})
	--({\sx*(0.8100)},{\sy*(0.4320)})
	--({\sx*(0.8200)},{\sy*(0.2996)})
	--({\sx*(0.8300)},{\sy*(0.1691)})
	--({\sx*(0.8400)},{\sy*(0.0386)})
	--({\sx*(0.8500)},{\sy*(-0.1094)})
	--({\sx*(0.8600)},{\sy*(-0.2381)})
	--({\sx*(0.8700)},{\sy*(-0.3676)})
	--({\sx*(0.8800)},{\sy*(-0.4862)})
	--({\sx*(0.8900)},{\sy*(-0.6094)})
	--({\sx*(0.9000)},{\sy*(-0.7077)})
	--({\sx*(0.9100)},{\sy*(-0.8024)})
	--({\sx*(0.9200)},{\sy*(-0.8704)})
	--({\sx*(0.9300)},{\sy*(-0.9283)})
	--({\sx*(0.9400)},{\sy*(-0.9678)})
	--({\sx*(0.9500)},{\sy*(-0.9963)})
	--({\sx*(0.9600)},{\sy*(-1.0000)})
	--({\sx*(0.9700)},{\sy*(-0.9890)})
	--({\sx*(0.9800)},{\sy*(-0.9596)})
	--({\sx*(0.9900)},{\sy*(-0.9210)})
	--({\sx*(1.0000)},{\sy*(-0.8566)})
	--({\sx*(1.0100)},{\sy*(-0.7771)})
	--({\sx*(1.0200)},{\sy*(-0.6935)})
	--({\sx*(1.0300)},{\sy*(-0.5974)})
	--({\sx*(1.0400)},{\sy*(-0.4839)})
	--({\sx*(1.0500)},{\sy*(-0.3658)})
	--({\sx*(1.0600)},{\sy*(-0.2486)})
	--({\sx*(1.0700)},{\sy*(-0.1245)})
	--({\sx*(1.0800)},{\sy*(0.0028)})
	--({\sx*(1.0900)},{\sy*(0.1337)})
	--({\sx*(1.1000)},{\sy*(0.2518)})
	--({\sx*(1.1100)},{\sy*(0.3718)})
	--({\sx*(1.1200)},{\sy*(0.4890)})
	--({\sx*(1.1300)},{\sy*(0.5956)})
	--({\sx*(1.1400)},{\sy*(0.6870)})
	--({\sx*(1.1500)},{\sy*(0.7716)})
	--({\sx*(1.1600)},{\sy*(0.8433)})
	--({\sx*(1.1700)},{\sy*(0.9049)})
	--({\sx*(1.1800)},{\sy*(0.9536)})
	--({\sx*(1.1900)},{\sy*(0.9812)})
	--({\sx*(1.2000)},{\sy*(0.9963)})
	--({\sx*(1.2100)},{\sy*(0.9991)})
	--({\sx*(1.2200)},{\sy*(0.9899)})
	--({\sx*(1.2300)},{\sy*(0.9577)})
	--({\sx*(1.2400)},{\sy*(0.9159)})
	--({\sx*(1.2500)},{\sy*(0.8635)})
	--({\sx*(1.2600)},{\sy*(0.7964)})
	--({\sx*(1.2700)},{\sy*(0.7178)})
	--({\sx*(1.2800)},{\sy*(0.6268)})
	--({\sx*(1.2900)},{\sy*(0.5340)})
	--({\sx*(1.3000)},{\sy*(0.4315)})
	--({\sx*(1.3100)},{\sy*(0.3212)})
	--({\sx*(1.3200)},{\sy*(0.2073)})
	--({\sx*(1.3300)},{\sy*(0.0947)})
	--({\sx*(1.3400)},{\sy*(-0.0188)})
	--({\sx*(1.3500)},{\sy*(-0.1365)})
	--({\sx*(1.3600)},{\sy*(-0.2486)})
	--({\sx*(1.3700)},{\sy*(-0.3539)})
	--({\sx*(1.3800)},{\sy*(-0.4568)})
	--({\sx*(1.3900)},{\sy*(-0.5538)})
	--({\sx*(1.4000)},{\sy*(-0.6434)})
	--({\sx*(1.4100)},{\sy*(-0.7197)})
	--({\sx*(1.4200)},{\sy*(-0.7886)})
	--({\sx*(1.4300)},{\sy*(-0.8447)})
	--({\sx*(1.4400)},{\sy*(-0.8906)})
	--({\sx*(1.4500)},{\sy*(-0.9283)})
	--({\sx*(1.4600)},{\sy*(-0.9458)})
	--({\sx*(1.4700)},{\sy*(-0.9554)})
	--({\sx*(1.4800)},{\sy*(-0.9540)})
	--({\sx*(1.4900)},{\sy*(-0.9370)})
	--({\sx*(1.5000)},{\sy*(-0.9085)})
	--({\sx*(1.5100)},{\sy*(-0.8681)})
	--({\sx*(1.5200)},{\sy*(-0.8194)})
	--({\sx*(1.5300)},{\sy*(-0.7601)})
	--({\sx*(1.5400)},{\sy*(-0.6889)})
	--({\sx*(1.5500)},{\sy*(-0.6124)})
	--({\sx*(1.5600)},{\sy*(-0.5262)})
	--({\sx*(1.5700)},{\sy*(-0.4375)})
	--({\sx*(1.5800)},{\sy*(-0.3403)})
	--({\sx*(1.5900)},{\sy*(-0.2390)})
	--({\sx*(1.6000)},{\sy*(-0.1397)})
	--({\sx*(1.6100)},{\sy*(-0.0372)})
	--({\sx*(1.6200)},{\sy*(0.0671)})
	--({\sx*(1.6300)},{\sy*(0.1673)})
	--({\sx*(1.6400)},{\sy*(0.2629)})
	--({\sx*(1.6500)},{\sy*(0.3550)})
	--({\sx*(1.6600)},{\sy*(0.4453)})
	--({\sx*(1.6700)},{\sy*(0.5271)})
	--({\sx*(1.6800)},{\sy*(0.6039)})
	--({\sx*(1.6900)},{\sy*(0.6693)})
	--({\sx*(1.7000)},{\sy*(0.7270)})
	--({\sx*(1.7100)},{\sy*(0.7783)})
	--({\sx*(1.7200)},{\sy*(0.8192)})
	--({\sx*(1.7300)},{\sy*(0.8474)})
	--({\sx*(1.7400)},{\sy*(0.8637)})
	--({\sx*(1.7500)},{\sy*(0.8743)})
	--({\sx*(1.7600)},{\sy*(0.8736)})
	--({\sx*(1.7700)},{\sy*(0.8591)})
	--({\sx*(1.7800)},{\sy*(0.8348)})
	--({\sx*(1.7900)},{\sy*(0.8047)})
	--({\sx*(1.8000)},{\sy*(0.7640)})
	--({\sx*(1.8100)},{\sy*(0.7142)})
	--({\sx*(1.8200)},{\sy*(0.6551)})
	--({\sx*(1.8300)},{\sy*(0.5910)})
	--({\sx*(1.8400)},{\sy*(0.5211)})
	--({\sx*(1.8500)},{\sy*(0.4437)})
	--({\sx*(1.8600)},{\sy*(0.3624)})
	--({\sx*(1.8700)},{\sy*(0.2783)})
	--({\sx*(1.8800)},{\sy*(0.1926)})
	--({\sx*(1.8900)},{\sy*(0.1029)})
	--({\sx*(1.9000)},{\sy*(0.0149)})
	--({\sx*(1.9100)},{\sy*(-0.0719)})
	--({\sx*(1.9200)},{\sy*(-0.1585)})
	--({\sx*(1.9300)},{\sy*(-0.2410)})
	--({\sx*(1.9400)},{\sy*(-0.3213)})
	--({\sx*(1.9500)},{\sy*(-0.3975)})
	--({\sx*(1.9600)},{\sy*(-0.4671)})
	--({\sx*(1.9700)},{\sy*(-0.5294)})
	--({\sx*(1.9800)},{\sy*(-0.5875)})
	--({\sx*(1.9900)},{\sy*(-0.6382)})
	--({\sx*(2.0000)},{\sy*(-0.6791)})
	--({\sx*(2.0100)},{\sy*(-0.7136)})
	--({\sx*(2.0200)},{\sy*(-0.7393)})
	--({\sx*(2.0300)},{\sy*(-0.7564)})
	--({\sx*(2.0400)},{\sy*(-0.7642)})
	--({\sx*(2.0500)},{\sy*(-0.7640)})
	--({\sx*(2.0600)},{\sy*(-0.7552)})
	--({\sx*(2.0700)},{\sy*(-0.7386)})
	--({\sx*(2.0800)},{\sy*(-0.7124)})
	--({\sx*(2.0900)},{\sy*(-0.6798)})
	--({\sx*(2.1000)},{\sy*(-0.6396)})
	--({\sx*(2.1100)},{\sy*(-0.5932)})
	--({\sx*(2.1200)},{\sy*(-0.5403)})
	--({\sx*(2.1300)},{\sy*(-0.4831)})
	--({\sx*(2.1400)},{\sy*(-0.4193)})
	--({\sx*(2.1500)},{\sy*(-0.3529)})
	--({\sx*(2.1600)},{\sy*(-0.2832)})
	--({\sx*(2.1700)},{\sy*(-0.2113)})
	--({\sx*(2.1800)},{\sy*(-0.1378)})
	--({\sx*(2.1900)},{\sy*(-0.0641)})
	--({\sx*(2.2000)},{\sy*(0.0098)})
	--({\sx*(2.2100)},{\sy*(0.0827)})
	--({\sx*(2.2200)},{\sy*(0.1536)})
	--({\sx*(2.2300)},{\sy*(0.2212)})
	--({\sx*(2.2400)},{\sy*(0.2864)})
	--({\sx*(2.2500)},{\sy*(0.3477)})
	--({\sx*(2.2600)},{\sy*(0.4039)})
	--({\sx*(2.2700)},{\sy*(0.4549)})
	--({\sx*(2.2800)},{\sy*(0.5009)})
	--({\sx*(2.2900)},{\sy*(0.5398)})
	--({\sx*(2.3000)},{\sy*(0.5728)})
	--({\sx*(2.3100)},{\sy*(0.5991)})
	--({\sx*(2.3200)},{\sy*(0.6185)})
	--({\sx*(2.3300)},{\sy*(0.6307)})
	--({\sx*(2.3400)},{\sy*(0.6361)})
	--({\sx*(2.3500)},{\sy*(0.6348)})
	--({\sx*(2.3600)},{\sy*(0.6255)})
	--({\sx*(2.3700)},{\sy*(0.6104)})
	--({\sx*(2.3800)},{\sy*(0.5890)})
	--({\sx*(2.3900)},{\sy*(0.5614)})
	--({\sx*(2.4000)},{\sy*(0.5276)})
	--({\sx*(2.4100)},{\sy*(0.4897)})
	--({\sx*(2.4200)},{\sy*(0.4454)})
	--({\sx*(2.4300)},{\sy*(0.3981)})
	--({\sx*(2.4400)},{\sy*(0.3472)})
	--({\sx*(2.4500)},{\sy*(0.2926)})
	--({\sx*(2.4600)},{\sy*(0.2361)})
	--({\sx*(2.4700)},{\sy*(0.1779)})
	--({\sx*(2.4800)},{\sy*(0.1190)})
	--({\sx*(2.4900)},{\sy*(0.0594)})
	--({\sx*(2.5000)},{\sy*(-0.0002)})
	--({\sx*(2.5100)},{\sy*(-0.0581)})
	--({\sx*(2.5200)},{\sy*(-0.1151)})
	--({\sx*(2.5300)},{\sy*(-0.1695)})
	--({\sx*(2.5400)},{\sy*(-0.2216)})
	--({\sx*(2.5500)},{\sy*(-0.2698)})
	--({\sx*(2.5600)},{\sy*(-0.3148)})
	--({\sx*(2.5700)},{\sy*(-0.3554)})
	--({\sx*(2.5800)},{\sy*(-0.3911)})
	--({\sx*(2.5900)},{\sy*(-0.4224)})
	--({\sx*(2.6000)},{\sy*(-0.4486)})
	--({\sx*(2.6100)},{\sy*(-0.4696)})
	--({\sx*(2.6200)},{\sy*(-0.4847)})
	--({\sx*(2.6300)},{\sy*(-0.4938)})
	--({\sx*(2.6400)},{\sy*(-0.4983)})
	--({\sx*(2.6500)},{\sy*(-0.4971)})
	--({\sx*(2.6600)},{\sy*(-0.4901)})
	--({\sx*(2.6700)},{\sy*(-0.4783)})
	--({\sx*(2.6800)},{\sy*(-0.4615)})
	--({\sx*(2.6900)},{\sy*(-0.4398)})
	--({\sx*(2.7000)},{\sy*(-0.4130)})
	--({\sx*(2.7100)},{\sy*(-0.3831)})
	--({\sx*(2.7200)},{\sy*(-0.3493)})
	--({\sx*(2.7300)},{\sy*(-0.3122)})
	--({\sx*(2.7400)},{\sy*(-0.2724)})
	--({\sx*(2.7500)},{\sy*(-0.2308)})
	--({\sx*(2.7600)},{\sy*(-0.1871)})
	--({\sx*(2.7700)},{\sy*(-0.1423)})
	--({\sx*(2.7800)},{\sy*(-0.0968)})
	--({\sx*(2.7900)},{\sy*(-0.0512)})
	--({\sx*(2.8000)},{\sy*(-0.0059)})
	--({\sx*(2.8100)},{\sy*(0.0383)})
	--({\sx*(2.8200)},{\sy*(0.0814)})
	--({\sx*(2.8300)},{\sy*(0.1227)})
	--({\sx*(2.8400)},{\sy*(0.1615)})
	--({\sx*(2.8500)},{\sy*(0.1981)})
	--({\sx*(2.8600)},{\sy*(0.2312)})
	--({\sx*(2.8700)},{\sy*(0.2615)})
	--({\sx*(2.8800)},{\sy*(0.2874)})
	--({\sx*(2.8900)},{\sy*(0.3107)})
	--({\sx*(2.9000)},{\sy*(0.3289)})
	--({\sx*(2.9100)},{\sy*(0.3436)})
	--({\sx*(2.9200)},{\sy*(0.3540)})
	--({\sx*(2.9300)},{\sy*(0.3600)})
	--({\sx*(2.9400)},{\sy*(0.3624)})
	--({\sx*(2.9500)},{\sy*(0.3601)})
	--({\sx*(2.9600)},{\sy*(0.3536)})
	--({\sx*(2.9700)},{\sy*(0.3435)})
	--({\sx*(2.9800)},{\sy*(0.3300)})
	--({\sx*(2.9900)},{\sy*(0.3128)})
	--({\sx*(3.0000)},{\sy*(0.2928)})
	--({\sx*(3.0100)},{\sy*(0.2696)})
	--({\sx*(3.0200)},{\sy*(0.2439)})
	--({\sx*(3.0300)},{\sy*(0.2159)})
	--({\sx*(3.0400)},{\sy*(0.1864)})
	--({\sx*(3.0500)},{\sy*(0.1558)})
	--({\sx*(3.0600)},{\sy*(0.1239)})
	--({\sx*(3.0700)},{\sy*(0.0914)})
	--({\sx*(3.0800)},{\sy*(0.0588)})
	--({\sx*(3.0900)},{\sy*(0.0264)})
	--({\sx*(3.1000)},{\sy*(-0.0054)})
	--({\sx*(3.1100)},{\sy*(-0.0363)})
	--({\sx*(3.1200)},{\sy*(-0.0659)})
	--({\sx*(3.1300)},{\sy*(-0.0938)})
	--({\sx*(3.1400)},{\sy*(-0.1198)})
	--({\sx*(3.1500)},{\sy*(-0.1436)})
	--({\sx*(3.1600)},{\sy*(-0.1651)})
	--({\sx*(3.1700)},{\sy*(-0.1840)})
	--({\sx*(3.1800)},{\sy*(-0.1997)})
	--({\sx*(3.1900)},{\sy*(-0.2128)})
	--({\sx*(3.2000)},{\sy*(-0.2234)})
	--({\sx*(3.2100)},{\sy*(-0.2303)})
	--({\sx*(3.2200)},{\sy*(-0.2345)})
	--({\sx*(3.2300)},{\sy*(-0.2361)})
	--({\sx*(3.2400)},{\sy*(-0.2342)})
	--({\sx*(3.2500)},{\sy*(-0.2298)})
	--({\sx*(3.2600)},{\sy*(-0.2226)})
	--({\sx*(3.2700)},{\sy*(-0.2129)})
	--({\sx*(3.2800)},{\sy*(-0.2008)})
	--({\sx*(3.2900)},{\sy*(-0.1871)})
	--({\sx*(3.3000)},{\sy*(-0.1714)})
	--({\sx*(3.3100)},{\sy*(-0.1540)})
	--({\sx*(3.3200)},{\sy*(-0.1355)})
	--({\sx*(3.3300)},{\sy*(-0.1158)})
	--({\sx*(3.3400)},{\sy*(-0.0954)})
	--({\sx*(3.3500)},{\sy*(-0.0748)})
	--({\sx*(3.3600)},{\sy*(-0.0540)})
	--({\sx*(3.3700)},{\sy*(-0.0332)})
	--({\sx*(3.3800)},{\sy*(-0.0129)})
	--({\sx*(3.3900)},{\sy*(0.0067)})
	--({\sx*(3.4000)},{\sy*(0.0254)})
	--({\sx*(3.4100)},{\sy*(0.0430)})
	--({\sx*(3.4200)},{\sy*(0.0592)})
	--({\sx*(3.4300)},{\sy*(0.0739)})
	--({\sx*(3.4400)},{\sy*(0.0870)})
	--({\sx*(3.4500)},{\sy*(0.0984)})
	--({\sx*(3.4600)},{\sy*(0.1077)})
	--({\sx*(3.4700)},{\sy*(0.1152)})
	--({\sx*(3.4800)},{\sy*(0.1209)})
	--({\sx*(3.4900)},{\sy*(0.1245)})
	--({\sx*(3.5000)},{\sy*(0.1264)})
	--({\sx*(3.5100)},{\sy*(0.1264)})
	--({\sx*(3.5200)},{\sy*(0.1245)})
	--({\sx*(3.5300)},{\sy*(0.1212)})
	--({\sx*(3.5400)},{\sy*(0.1161)})
	--({\sx*(3.5500)},{\sy*(0.1099)})
	--({\sx*(3.5600)},{\sy*(0.1023)})
	--({\sx*(3.5700)},{\sy*(0.0936)})
	--({\sx*(3.5800)},{\sy*(0.0839)})
	--({\sx*(3.5900)},{\sy*(0.0741)})
	--({\sx*(3.6000)},{\sy*(0.0636)})
	--({\sx*(3.6100)},{\sy*(0.0526)})
	--({\sx*(3.6200)},{\sy*(0.0420)})
	--({\sx*(3.6300)},{\sy*(0.0312)})
	--({\sx*(3.6400)},{\sy*(0.0209)})
	--({\sx*(3.6500)},{\sy*(0.0109)})
	--({\sx*(3.6600)},{\sy*(0.0016)})
	--({\sx*(3.6700)},{\sy*(-0.0070)})
	--({\sx*(3.6800)},{\sy*(-0.0147)})
	--({\sx*(3.6900)},{\sy*(-0.0215)})
	--({\sx*(3.7000)},{\sy*(-0.0273)})
	--({\sx*(3.7100)},{\sy*(-0.0320)})
	--({\sx*(3.7200)},{\sy*(-0.0355)})
	--({\sx*(3.7300)},{\sy*(-0.0382)})
	--({\sx*(3.7400)},{\sy*(-0.0395)})
	--({\sx*(3.7500)},{\sy*(-0.0401)})
	--({\sx*(3.7600)},{\sy*(-0.0398)})
	--({\sx*(3.7700)},{\sy*(-0.0387)})
	--({\sx*(3.7800)},{\sy*(-0.0367)})
	--({\sx*(3.7900)},{\sy*(-0.0339)})
	--({\sx*(3.8000)},{\sy*(-0.0311)})
	--({\sx*(3.8100)},{\sy*(-0.0275)})
	--({\sx*(3.8200)},{\sy*(-0.0235)})
	--({\sx*(3.8300)},{\sy*(-0.0197)})
	--({\sx*(3.8400)},{\sy*(-0.0164)})
	--({\sx*(3.8500)},{\sy*(-0.0127)})
	--({\sx*(3.8600)},{\sy*(-0.0092)})
	--({\sx*(3.8700)},{\sy*(-0.0060)})
	--({\sx*(3.8800)},{\sy*(-0.0037)})
	--({\sx*(3.8900)},{\sy*(-0.0019)})
	--({\sx*(3.9000)},{\sy*(-0.0006)})
	--({\sx*(3.9100)},{\sy*(0.0001)})
	--({\sx*(3.9200)},{\sy*(0.0000)})
	--({\sx*(3.9300)},{\sy*(-0.0007)})
	--({\sx*(3.9400)},{\sy*(-0.0022)})
	--({\sx*(3.9500)},{\sy*(-0.0042)})
	--({\sx*(3.9600)},{\sy*(-0.0067)})
	--({\sx*(3.9700)},{\sy*(-0.0099)})
	--({\sx*(3.9800)},{\sy*(-0.0131)})
	--({\sx*(3.9900)},{\sy*(-0.0167)})
	--({\sx*(4.0000)},{\sy*(-0.0207)})
	--({\sx*(4.0100)},{\sy*(-0.0242)})
	--({\sx*(4.0200)},{\sy*(-0.0278)})
	--({\sx*(4.0300)},{\sy*(-0.0310)})
	--({\sx*(4.0400)},{\sy*(-0.0336)})
	--({\sx*(4.0500)},{\sy*(-0.0360)})
	--({\sx*(4.0600)},{\sy*(-0.0374)})
	--({\sx*(4.0700)},{\sy*(-0.0378)})
	--({\sx*(4.0800)},{\sy*(-0.0375)})
	--({\sx*(4.0900)},{\sy*(-0.0361)})
	--({\sx*(4.1000)},{\sy*(-0.0337)})
	--({\sx*(4.1100)},{\sy*(-0.0302)})
	--({\sx*(4.1200)},{\sy*(-0.0257)})
	--({\sx*(4.1300)},{\sy*(-0.0201)})
	--({\sx*(4.1400)},{\sy*(-0.0136)})
	--({\sx*(4.1500)},{\sy*(-0.0062)})
	--({\sx*(4.1600)},{\sy*(0.0018)})
	--({\sx*(4.1700)},{\sy*(0.0104)})
	--({\sx*(4.1800)},{\sy*(0.0195)})
	--({\sx*(4.1900)},{\sy*(0.0286)})
	--({\sx*(4.2000)},{\sy*(0.0377)})
	--({\sx*(4.2100)},{\sy*(0.0464)})
	--({\sx*(4.2200)},{\sy*(0.0547)})
	--({\sx*(4.2300)},{\sy*(0.0621)})
	--({\sx*(4.2400)},{\sy*(0.0685)})
	--({\sx*(4.2500)},{\sy*(0.0737)})
	--({\sx*(4.2600)},{\sy*(0.0774)})
	--({\sx*(4.2700)},{\sy*(0.0795)})
	--({\sx*(4.2800)},{\sy*(0.0798)})
	--({\sx*(4.2900)},{\sy*(0.0781)})
	--({\sx*(4.3000)},{\sy*(0.0746)})
	--({\sx*(4.3100)},{\sy*(0.0693)})
	--({\sx*(4.3200)},{\sy*(0.0618)})
	--({\sx*(4.3300)},{\sy*(0.0528)})
	--({\sx*(4.3400)},{\sy*(0.0419)})
	--({\sx*(4.3500)},{\sy*(0.0297)})
	--({\sx*(4.3600)},{\sy*(0.0162)})
	--({\sx*(4.3700)},{\sy*(0.0019)})
	--({\sx*(4.3800)},{\sy*(-0.0130)})
	--({\sx*(4.3900)},{\sy*(-0.0279)})
	--({\sx*(4.4000)},{\sy*(-0.0427)})
	--({\sx*(4.4100)},{\sy*(-0.0567)})
	--({\sx*(4.4200)},{\sy*(-0.0695)})
	--({\sx*(4.4300)},{\sy*(-0.0810)})
	--({\sx*(4.4400)},{\sy*(-0.0903)})
	--({\sx*(4.4500)},{\sy*(-0.0974)})
	--({\sx*(4.4600)},{\sy*(-0.1019)})
	--({\sx*(4.4700)},{\sy*(-0.1036)})
	--({\sx*(4.4800)},{\sy*(-0.1026)})
	--({\sx*(4.4900)},{\sy*(-0.0980)})
	--({\sx*(4.5000)},{\sy*(-0.0904)})
	--({\sx*(4.5100)},{\sy*(-0.0800)})
	--({\sx*(4.5200)},{\sy*(-0.0667)})
	--({\sx*(4.5300)},{\sy*(-0.0512)})
	--({\sx*(4.5400)},{\sy*(-0.0337)})
	--({\sx*(4.5500)},{\sy*(-0.0147)})
	--({\sx*(4.5600)},{\sy*(0.0051)})
	--({\sx*(4.5700)},{\sy*(0.0250)})
	--({\sx*(4.5800)},{\sy*(0.0443)})
	--({\sx*(4.5900)},{\sy*(0.0624)})
	--({\sx*(4.6000)},{\sy*(0.0785)})
	--({\sx*(4.6100)},{\sy*(0.0917)})
	--({\sx*(4.6200)},{\sy*(0.1017)})
	--({\sx*(4.6300)},{\sy*(0.1080)})
	--({\sx*(4.6400)},{\sy*(0.1100)})
	--({\sx*(4.6500)},{\sy*(0.1073)})
	--({\sx*(4.6600)},{\sy*(0.1000)})
	--({\sx*(4.6700)},{\sy*(0.0884)})
	--({\sx*(4.6800)},{\sy*(0.0728)})
	--({\sx*(4.6900)},{\sy*(0.0538)})
	--({\sx*(4.7000)},{\sy*(0.0321)})
	--({\sx*(4.7100)},{\sy*(0.0087)})
	--({\sx*(4.7200)},{\sy*(-0.0152)})
	--({\sx*(4.7300)},{\sy*(-0.0384)})
	--({\sx*(4.7400)},{\sy*(-0.0597)})
	--({\sx*(4.7500)},{\sy*(-0.0777)})
	--({\sx*(4.7600)},{\sy*(-0.0911)})
	--({\sx*(4.7700)},{\sy*(-0.0993)})
	--({\sx*(4.7800)},{\sy*(-0.1010)})
	--({\sx*(4.7900)},{\sy*(-0.0964)})
	--({\sx*(4.8000)},{\sy*(-0.0851)})
	--({\sx*(4.8100)},{\sy*(-0.0679)})
	--({\sx*(4.8200)},{\sy*(-0.0458)})
	--({\sx*(4.8300)},{\sy*(-0.0203)})
	--({\sx*(4.8400)},{\sy*(0.0066)})
	--({\sx*(4.8500)},{\sy*(0.0326)})
	--({\sx*(4.8600)},{\sy*(0.0551)})
	--({\sx*(4.8700)},{\sy*(0.0717)})
	--({\sx*(4.8800)},{\sy*(0.0802)})
	--({\sx*(4.8900)},{\sy*(0.0792)})
	--({\sx*(4.9000)},{\sy*(0.0682)})
	--({\sx*(4.9100)},{\sy*(0.0481)})
	--({\sx*(4.9200)},{\sy*(0.0214)})
	--({\sx*(4.9300)},{\sy*(-0.0076)})
	--({\sx*(4.9400)},{\sy*(-0.0335)})
	--({\sx*(4.9500)},{\sy*(-0.0499)})
	--({\sx*(4.9600)},{\sy*(-0.0511)})
	--({\sx*(4.9700)},{\sy*(-0.0347)})
	--({\sx*(4.9800)},{\sy*(-0.0055)})
	--({\sx*(4.9900)},{\sy*(0.0191)})
	--({\sx*(5.0000)},{\sy*(0.0000)});
}
\def\relfehlerm{
\draw[color=blue,line width=1.4pt,line join=round] ({\sx*(0.000)},{\sy*(0.0000)})
	--({\sx*(0.0100)},{\sy*(0.0000)})
	--({\sx*(0.0200)},{\sy*(-0.0000)})
	--({\sx*(0.0300)},{\sy*(-0.0000)})
	--({\sx*(0.0400)},{\sy*(-0.0000)})
	--({\sx*(0.0500)},{\sy*(-0.0000)})
	--({\sx*(0.0600)},{\sy*(-0.0000)})
	--({\sx*(0.0700)},{\sy*(-0.0000)})
	--({\sx*(0.0800)},{\sy*(0.0000)})
	--({\sx*(0.0900)},{\sy*(0.0000)})
	--({\sx*(0.1000)},{\sy*(0.0000)})
	--({\sx*(0.1100)},{\sy*(0.0000)})
	--({\sx*(0.1200)},{\sy*(0.0000)})
	--({\sx*(0.1300)},{\sy*(0.0000)})
	--({\sx*(0.1400)},{\sy*(0.0000)})
	--({\sx*(0.1500)},{\sy*(0.0000)})
	--({\sx*(0.1600)},{\sy*(0.0000)})
	--({\sx*(0.1700)},{\sy*(-0.0000)})
	--({\sx*(0.1800)},{\sy*(-0.0000)})
	--({\sx*(0.1900)},{\sy*(-0.0000)})
	--({\sx*(0.2000)},{\sy*(-0.0000)})
	--({\sx*(0.2100)},{\sy*(-0.0000)})
	--({\sx*(0.2200)},{\sy*(-0.0000)})
	--({\sx*(0.2300)},{\sy*(-0.0000)})
	--({\sx*(0.2400)},{\sy*(-0.0000)})
	--({\sx*(0.2500)},{\sy*(-0.0000)})
	--({\sx*(0.2600)},{\sy*(-0.0000)})
	--({\sx*(0.2700)},{\sy*(-0.0000)})
	--({\sx*(0.2800)},{\sy*(-0.0000)})
	--({\sx*(0.2900)},{\sy*(0.0000)})
	--({\sx*(0.3000)},{\sy*(0.0000)})
	--({\sx*(0.3100)},{\sy*(0.0000)})
	--({\sx*(0.3200)},{\sy*(0.0000)})
	--({\sx*(0.3300)},{\sy*(0.0000)})
	--({\sx*(0.3400)},{\sy*(0.0000)})
	--({\sx*(0.3500)},{\sy*(0.0000)})
	--({\sx*(0.3600)},{\sy*(0.0000)})
	--({\sx*(0.3700)},{\sy*(0.0000)})
	--({\sx*(0.3800)},{\sy*(0.0000)})
	--({\sx*(0.3900)},{\sy*(0.0000)})
	--({\sx*(0.4000)},{\sy*(0.0000)})
	--({\sx*(0.4100)},{\sy*(0.0000)})
	--({\sx*(0.4200)},{\sy*(0.0000)})
	--({\sx*(0.4300)},{\sy*(0.0000)})
	--({\sx*(0.4400)},{\sy*(0.0000)})
	--({\sx*(0.4500)},{\sy*(-0.0000)})
	--({\sx*(0.4600)},{\sy*(-0.0000)})
	--({\sx*(0.4700)},{\sy*(-0.0000)})
	--({\sx*(0.4800)},{\sy*(-0.0000)})
	--({\sx*(0.4900)},{\sy*(-0.0000)})
	--({\sx*(0.5000)},{\sy*(-0.0000)})
	--({\sx*(0.5100)},{\sy*(-0.0000)})
	--({\sx*(0.5200)},{\sy*(-0.0000)})
	--({\sx*(0.5300)},{\sy*(-0.0000)})
	--({\sx*(0.5400)},{\sy*(-0.0000)})
	--({\sx*(0.5500)},{\sy*(-0.0000)})
	--({\sx*(0.5600)},{\sy*(-0.0000)})
	--({\sx*(0.5700)},{\sy*(-0.0000)})
	--({\sx*(0.5800)},{\sy*(-0.0000)})
	--({\sx*(0.5900)},{\sy*(-0.0000)})
	--({\sx*(0.6000)},{\sy*(-0.0000)})
	--({\sx*(0.6100)},{\sy*(-0.0000)})
	--({\sx*(0.6200)},{\sy*(-0.0000)})
	--({\sx*(0.6300)},{\sy*(0.0000)})
	--({\sx*(0.6400)},{\sy*(0.0000)})
	--({\sx*(0.6500)},{\sy*(0.0000)})
	--({\sx*(0.6600)},{\sy*(0.0000)})
	--({\sx*(0.6700)},{\sy*(0.0000)})
	--({\sx*(0.6800)},{\sy*(0.0000)})
	--({\sx*(0.6900)},{\sy*(0.0000)})
	--({\sx*(0.7000)},{\sy*(0.0000)})
	--({\sx*(0.7100)},{\sy*(0.0000)})
	--({\sx*(0.7200)},{\sy*(0.0000)})
	--({\sx*(0.7300)},{\sy*(0.0000)})
	--({\sx*(0.7400)},{\sy*(0.0000)})
	--({\sx*(0.7500)},{\sy*(0.0000)})
	--({\sx*(0.7600)},{\sy*(0.0000)})
	--({\sx*(0.7700)},{\sy*(0.0000)})
	--({\sx*(0.7800)},{\sy*(0.0000)})
	--({\sx*(0.7900)},{\sy*(0.0000)})
	--({\sx*(0.8000)},{\sy*(0.0000)})
	--({\sx*(0.8100)},{\sy*(0.0000)})
	--({\sx*(0.8200)},{\sy*(0.0000)})
	--({\sx*(0.8300)},{\sy*(0.0000)})
	--({\sx*(0.8400)},{\sy*(0.0000)})
	--({\sx*(0.8500)},{\sy*(-0.0000)})
	--({\sx*(0.8600)},{\sy*(-0.0000)})
	--({\sx*(0.8700)},{\sy*(-0.0000)})
	--({\sx*(0.8800)},{\sy*(-0.0000)})
	--({\sx*(0.8900)},{\sy*(-0.0000)})
	--({\sx*(0.9000)},{\sy*(-0.0000)})
	--({\sx*(0.9100)},{\sy*(-0.0000)})
	--({\sx*(0.9200)},{\sy*(-0.0000)})
	--({\sx*(0.9300)},{\sy*(-0.0000)})
	--({\sx*(0.9400)},{\sy*(-0.0000)})
	--({\sx*(0.9500)},{\sy*(-0.0000)})
	--({\sx*(0.9600)},{\sy*(-0.0000)})
	--({\sx*(0.9700)},{\sy*(-0.0000)})
	--({\sx*(0.9800)},{\sy*(-0.0000)})
	--({\sx*(0.9900)},{\sy*(-0.0000)})
	--({\sx*(1.0000)},{\sy*(-0.0000)})
	--({\sx*(1.0100)},{\sy*(-0.0000)})
	--({\sx*(1.0200)},{\sy*(-0.0000)})
	--({\sx*(1.0300)},{\sy*(-0.0000)})
	--({\sx*(1.0400)},{\sy*(-0.0000)})
	--({\sx*(1.0500)},{\sy*(-0.0000)})
	--({\sx*(1.0600)},{\sy*(-0.0000)})
	--({\sx*(1.0700)},{\sy*(-0.0000)})
	--({\sx*(1.0800)},{\sy*(0.0000)})
	--({\sx*(1.0900)},{\sy*(0.0000)})
	--({\sx*(1.1000)},{\sy*(0.0000)})
	--({\sx*(1.1100)},{\sy*(0.0000)})
	--({\sx*(1.1200)},{\sy*(0.0000)})
	--({\sx*(1.1300)},{\sy*(0.0000)})
	--({\sx*(1.1400)},{\sy*(0.0000)})
	--({\sx*(1.1500)},{\sy*(0.0000)})
	--({\sx*(1.1600)},{\sy*(0.0000)})
	--({\sx*(1.1700)},{\sy*(0.0000)})
	--({\sx*(1.1800)},{\sy*(0.0000)})
	--({\sx*(1.1900)},{\sy*(0.0000)})
	--({\sx*(1.2000)},{\sy*(0.0000)})
	--({\sx*(1.2100)},{\sy*(0.0000)})
	--({\sx*(1.2200)},{\sy*(0.0000)})
	--({\sx*(1.2300)},{\sy*(0.0000)})
	--({\sx*(1.2400)},{\sy*(0.0000)})
	--({\sx*(1.2500)},{\sy*(0.0000)})
	--({\sx*(1.2600)},{\sy*(0.0000)})
	--({\sx*(1.2700)},{\sy*(0.0000)})
	--({\sx*(1.2800)},{\sy*(0.0000)})
	--({\sx*(1.2900)},{\sy*(0.0000)})
	--({\sx*(1.3000)},{\sy*(0.0000)})
	--({\sx*(1.3100)},{\sy*(0.0000)})
	--({\sx*(1.3200)},{\sy*(0.0000)})
	--({\sx*(1.3300)},{\sy*(0.0000)})
	--({\sx*(1.3400)},{\sy*(-0.0000)})
	--({\sx*(1.3500)},{\sy*(-0.0000)})
	--({\sx*(1.3600)},{\sy*(-0.0000)})
	--({\sx*(1.3700)},{\sy*(-0.0000)})
	--({\sx*(1.3800)},{\sy*(-0.0000)})
	--({\sx*(1.3900)},{\sy*(-0.0000)})
	--({\sx*(1.4000)},{\sy*(-0.0000)})
	--({\sx*(1.4100)},{\sy*(-0.0000)})
	--({\sx*(1.4200)},{\sy*(-0.0000)})
	--({\sx*(1.4300)},{\sy*(-0.0000)})
	--({\sx*(1.4400)},{\sy*(-0.0000)})
	--({\sx*(1.4500)},{\sy*(-0.0000)})
	--({\sx*(1.4600)},{\sy*(-0.0000)})
	--({\sx*(1.4700)},{\sy*(-0.0000)})
	--({\sx*(1.4800)},{\sy*(-0.0000)})
	--({\sx*(1.4900)},{\sy*(-0.0000)})
	--({\sx*(1.5000)},{\sy*(-0.0000)})
	--({\sx*(1.5100)},{\sy*(-0.0000)})
	--({\sx*(1.5200)},{\sy*(-0.0000)})
	--({\sx*(1.5300)},{\sy*(-0.0000)})
	--({\sx*(1.5400)},{\sy*(-0.0000)})
	--({\sx*(1.5500)},{\sy*(-0.0000)})
	--({\sx*(1.5600)},{\sy*(-0.0000)})
	--({\sx*(1.5700)},{\sy*(-0.0000)})
	--({\sx*(1.5800)},{\sy*(-0.0000)})
	--({\sx*(1.5900)},{\sy*(-0.0000)})
	--({\sx*(1.6000)},{\sy*(-0.0000)})
	--({\sx*(1.6100)},{\sy*(-0.0000)})
	--({\sx*(1.6200)},{\sy*(0.0000)})
	--({\sx*(1.6300)},{\sy*(0.0000)})
	--({\sx*(1.6400)},{\sy*(0.0000)})
	--({\sx*(1.6500)},{\sy*(0.0000)})
	--({\sx*(1.6600)},{\sy*(0.0000)})
	--({\sx*(1.6700)},{\sy*(0.0000)})
	--({\sx*(1.6800)},{\sy*(0.0000)})
	--({\sx*(1.6900)},{\sy*(0.0000)})
	--({\sx*(1.7000)},{\sy*(0.0000)})
	--({\sx*(1.7100)},{\sy*(0.0000)})
	--({\sx*(1.7200)},{\sy*(0.0000)})
	--({\sx*(1.7300)},{\sy*(0.0000)})
	--({\sx*(1.7400)},{\sy*(0.0000)})
	--({\sx*(1.7500)},{\sy*(0.0000)})
	--({\sx*(1.7600)},{\sy*(0.0000)})
	--({\sx*(1.7700)},{\sy*(0.0000)})
	--({\sx*(1.7800)},{\sy*(0.0000)})
	--({\sx*(1.7900)},{\sy*(0.0000)})
	--({\sx*(1.8000)},{\sy*(0.0000)})
	--({\sx*(1.8100)},{\sy*(0.0000)})
	--({\sx*(1.8200)},{\sy*(0.0000)})
	--({\sx*(1.8300)},{\sy*(0.0000)})
	--({\sx*(1.8400)},{\sy*(0.0000)})
	--({\sx*(1.8500)},{\sy*(0.0000)})
	--({\sx*(1.8600)},{\sy*(0.0000)})
	--({\sx*(1.8700)},{\sy*(0.0000)})
	--({\sx*(1.8800)},{\sy*(0.0000)})
	--({\sx*(1.8900)},{\sy*(0.0000)})
	--({\sx*(1.9000)},{\sy*(0.0000)})
	--({\sx*(1.9100)},{\sy*(-0.0000)})
	--({\sx*(1.9200)},{\sy*(-0.0000)})
	--({\sx*(1.9300)},{\sy*(-0.0000)})
	--({\sx*(1.9400)},{\sy*(-0.0000)})
	--({\sx*(1.9500)},{\sy*(-0.0000)})
	--({\sx*(1.9600)},{\sy*(-0.0000)})
	--({\sx*(1.9700)},{\sy*(-0.0000)})
	--({\sx*(1.9800)},{\sy*(-0.0000)})
	--({\sx*(1.9900)},{\sy*(-0.0000)})
	--({\sx*(2.0000)},{\sy*(-0.0000)})
	--({\sx*(2.0100)},{\sy*(-0.0000)})
	--({\sx*(2.0200)},{\sy*(-0.0000)})
	--({\sx*(2.0300)},{\sy*(-0.0000)})
	--({\sx*(2.0400)},{\sy*(-0.0000)})
	--({\sx*(2.0500)},{\sy*(-0.0000)})
	--({\sx*(2.0600)},{\sy*(-0.0000)})
	--({\sx*(2.0700)},{\sy*(-0.0000)})
	--({\sx*(2.0800)},{\sy*(-0.0000)})
	--({\sx*(2.0900)},{\sy*(-0.0000)})
	--({\sx*(2.1000)},{\sy*(-0.0000)})
	--({\sx*(2.1100)},{\sy*(-0.0000)})
	--({\sx*(2.1200)},{\sy*(-0.0000)})
	--({\sx*(2.1300)},{\sy*(-0.0000)})
	--({\sx*(2.1400)},{\sy*(-0.0000)})
	--({\sx*(2.1500)},{\sy*(-0.0000)})
	--({\sx*(2.1600)},{\sy*(-0.0000)})
	--({\sx*(2.1700)},{\sy*(-0.0000)})
	--({\sx*(2.1800)},{\sy*(-0.0000)})
	--({\sx*(2.1900)},{\sy*(-0.0000)})
	--({\sx*(2.2000)},{\sy*(0.0000)})
	--({\sx*(2.2100)},{\sy*(0.0000)})
	--({\sx*(2.2200)},{\sy*(0.0000)})
	--({\sx*(2.2300)},{\sy*(0.0000)})
	--({\sx*(2.2400)},{\sy*(0.0000)})
	--({\sx*(2.2500)},{\sy*(0.0000)})
	--({\sx*(2.2600)},{\sy*(0.0000)})
	--({\sx*(2.2700)},{\sy*(0.0000)})
	--({\sx*(2.2800)},{\sy*(0.0000)})
	--({\sx*(2.2900)},{\sy*(0.0000)})
	--({\sx*(2.3000)},{\sy*(0.0000)})
	--({\sx*(2.3100)},{\sy*(0.0000)})
	--({\sx*(2.3200)},{\sy*(0.0000)})
	--({\sx*(2.3300)},{\sy*(0.0000)})
	--({\sx*(2.3400)},{\sy*(0.0000)})
	--({\sx*(2.3500)},{\sy*(0.0000)})
	--({\sx*(2.3600)},{\sy*(0.0000)})
	--({\sx*(2.3700)},{\sy*(0.0000)})
	--({\sx*(2.3800)},{\sy*(0.0000)})
	--({\sx*(2.3900)},{\sy*(0.0000)})
	--({\sx*(2.4000)},{\sy*(0.0000)})
	--({\sx*(2.4100)},{\sy*(0.0000)})
	--({\sx*(2.4200)},{\sy*(0.0000)})
	--({\sx*(2.4300)},{\sy*(0.0000)})
	--({\sx*(2.4400)},{\sy*(0.0000)})
	--({\sx*(2.4500)},{\sy*(0.0000)})
	--({\sx*(2.4600)},{\sy*(0.0000)})
	--({\sx*(2.4700)},{\sy*(0.0000)})
	--({\sx*(2.4800)},{\sy*(0.0000)})
	--({\sx*(2.4900)},{\sy*(0.0000)})
	--({\sx*(2.5000)},{\sy*(-0.0000)})
	--({\sx*(2.5100)},{\sy*(-0.0000)})
	--({\sx*(2.5200)},{\sy*(-0.0000)})
	--({\sx*(2.5300)},{\sy*(-0.0000)})
	--({\sx*(2.5400)},{\sy*(-0.0000)})
	--({\sx*(2.5500)},{\sy*(-0.0000)})
	--({\sx*(2.5600)},{\sy*(-0.0000)})
	--({\sx*(2.5700)},{\sy*(-0.0000)})
	--({\sx*(2.5800)},{\sy*(-0.0000)})
	--({\sx*(2.5900)},{\sy*(-0.0000)})
	--({\sx*(2.6000)},{\sy*(-0.0000)})
	--({\sx*(2.6100)},{\sy*(-0.0000)})
	--({\sx*(2.6200)},{\sy*(-0.0000)})
	--({\sx*(2.6300)},{\sy*(-0.0000)})
	--({\sx*(2.6400)},{\sy*(-0.0000)})
	--({\sx*(2.6500)},{\sy*(-0.0000)})
	--({\sx*(2.6600)},{\sy*(-0.0000)})
	--({\sx*(2.6700)},{\sy*(-0.0000)})
	--({\sx*(2.6800)},{\sy*(-0.0000)})
	--({\sx*(2.6900)},{\sy*(-0.0000)})
	--({\sx*(2.7000)},{\sy*(-0.0000)})
	--({\sx*(2.7100)},{\sy*(-0.0000)})
	--({\sx*(2.7200)},{\sy*(-0.0000)})
	--({\sx*(2.7300)},{\sy*(-0.0000)})
	--({\sx*(2.7400)},{\sy*(-0.0000)})
	--({\sx*(2.7500)},{\sy*(-0.0000)})
	--({\sx*(2.7600)},{\sy*(-0.0000)})
	--({\sx*(2.7700)},{\sy*(-0.0000)})
	--({\sx*(2.7800)},{\sy*(-0.0000)})
	--({\sx*(2.7900)},{\sy*(-0.0000)})
	--({\sx*(2.8000)},{\sy*(-0.0000)})
	--({\sx*(2.8100)},{\sy*(0.0000)})
	--({\sx*(2.8200)},{\sy*(0.0000)})
	--({\sx*(2.8300)},{\sy*(0.0000)})
	--({\sx*(2.8400)},{\sy*(0.0000)})
	--({\sx*(2.8500)},{\sy*(0.0000)})
	--({\sx*(2.8600)},{\sy*(0.0000)})
	--({\sx*(2.8700)},{\sy*(0.0000)})
	--({\sx*(2.8800)},{\sy*(0.0000)})
	--({\sx*(2.8900)},{\sy*(0.0000)})
	--({\sx*(2.9000)},{\sy*(0.0000)})
	--({\sx*(2.9100)},{\sy*(0.0000)})
	--({\sx*(2.9200)},{\sy*(0.0000)})
	--({\sx*(2.9300)},{\sy*(0.0000)})
	--({\sx*(2.9400)},{\sy*(0.0000)})
	--({\sx*(2.9500)},{\sy*(0.0000)})
	--({\sx*(2.9600)},{\sy*(0.0000)})
	--({\sx*(2.9700)},{\sy*(0.0000)})
	--({\sx*(2.9800)},{\sy*(0.0000)})
	--({\sx*(2.9900)},{\sy*(0.0000)})
	--({\sx*(3.0000)},{\sy*(0.0000)})
	--({\sx*(3.0100)},{\sy*(0.0000)})
	--({\sx*(3.0200)},{\sy*(0.0000)})
	--({\sx*(3.0300)},{\sy*(0.0000)})
	--({\sx*(3.0400)},{\sy*(0.0000)})
	--({\sx*(3.0500)},{\sy*(0.0000)})
	--({\sx*(3.0600)},{\sy*(0.0000)})
	--({\sx*(3.0700)},{\sy*(0.0000)})
	--({\sx*(3.0800)},{\sy*(0.0000)})
	--({\sx*(3.0900)},{\sy*(0.0000)})
	--({\sx*(3.1000)},{\sy*(-0.0000)})
	--({\sx*(3.1100)},{\sy*(-0.0000)})
	--({\sx*(3.1200)},{\sy*(-0.0000)})
	--({\sx*(3.1300)},{\sy*(-0.0000)})
	--({\sx*(3.1400)},{\sy*(-0.0000)})
	--({\sx*(3.1500)},{\sy*(-0.0000)})
	--({\sx*(3.1600)},{\sy*(-0.0000)})
	--({\sx*(3.1700)},{\sy*(-0.0000)})
	--({\sx*(3.1800)},{\sy*(-0.0000)})
	--({\sx*(3.1900)},{\sy*(-0.0000)})
	--({\sx*(3.2000)},{\sy*(-0.0000)})
	--({\sx*(3.2100)},{\sy*(-0.0000)})
	--({\sx*(3.2200)},{\sy*(-0.0000)})
	--({\sx*(3.2300)},{\sy*(-0.0000)})
	--({\sx*(3.2400)},{\sy*(-0.0000)})
	--({\sx*(3.2500)},{\sy*(-0.0000)})
	--({\sx*(3.2600)},{\sy*(-0.0000)})
	--({\sx*(3.2700)},{\sy*(-0.0000)})
	--({\sx*(3.2800)},{\sy*(-0.0000)})
	--({\sx*(3.2900)},{\sy*(-0.0000)})
	--({\sx*(3.3000)},{\sy*(-0.0000)})
	--({\sx*(3.3100)},{\sy*(-0.0000)})
	--({\sx*(3.3200)},{\sy*(-0.0000)})
	--({\sx*(3.3300)},{\sy*(-0.0000)})
	--({\sx*(3.3400)},{\sy*(-0.0000)})
	--({\sx*(3.3500)},{\sy*(-0.0000)})
	--({\sx*(3.3600)},{\sy*(-0.0000)})
	--({\sx*(3.3700)},{\sy*(-0.0000)})
	--({\sx*(3.3800)},{\sy*(-0.0000)})
	--({\sx*(3.3900)},{\sy*(0.0000)})
	--({\sx*(3.4000)},{\sy*(0.0000)})
	--({\sx*(3.4100)},{\sy*(0.0000)})
	--({\sx*(3.4200)},{\sy*(0.0000)})
	--({\sx*(3.4300)},{\sy*(0.0000)})
	--({\sx*(3.4400)},{\sy*(0.0000)})
	--({\sx*(3.4500)},{\sy*(0.0000)})
	--({\sx*(3.4600)},{\sy*(0.0000)})
	--({\sx*(3.4700)},{\sy*(0.0000)})
	--({\sx*(3.4800)},{\sy*(0.0000)})
	--({\sx*(3.4900)},{\sy*(0.0000)})
	--({\sx*(3.5000)},{\sy*(0.0000)})
	--({\sx*(3.5100)},{\sy*(0.0000)})
	--({\sx*(3.5200)},{\sy*(0.0000)})
	--({\sx*(3.5300)},{\sy*(0.0000)})
	--({\sx*(3.5400)},{\sy*(0.0000)})
	--({\sx*(3.5500)},{\sy*(0.0000)})
	--({\sx*(3.5600)},{\sy*(0.0000)})
	--({\sx*(3.5700)},{\sy*(0.0000)})
	--({\sx*(3.5800)},{\sy*(0.0000)})
	--({\sx*(3.5900)},{\sy*(0.0000)})
	--({\sx*(3.6000)},{\sy*(0.0000)})
	--({\sx*(3.6100)},{\sy*(0.0000)})
	--({\sx*(3.6200)},{\sy*(0.0000)})
	--({\sx*(3.6300)},{\sy*(0.0000)})
	--({\sx*(3.6400)},{\sy*(0.0000)})
	--({\sx*(3.6500)},{\sy*(0.0000)})
	--({\sx*(3.6600)},{\sy*(0.0000)})
	--({\sx*(3.6700)},{\sy*(-0.0000)})
	--({\sx*(3.6800)},{\sy*(-0.0000)})
	--({\sx*(3.6900)},{\sy*(-0.0000)})
	--({\sx*(3.7000)},{\sy*(-0.0000)})
	--({\sx*(3.7100)},{\sy*(-0.0000)})
	--({\sx*(3.7200)},{\sy*(-0.0000)})
	--({\sx*(3.7300)},{\sy*(-0.0000)})
	--({\sx*(3.7400)},{\sy*(-0.0000)})
	--({\sx*(3.7500)},{\sy*(-0.0000)})
	--({\sx*(3.7600)},{\sy*(-0.0000)})
	--({\sx*(3.7700)},{\sy*(-0.0000)})
	--({\sx*(3.7800)},{\sy*(-0.0000)})
	--({\sx*(3.7900)},{\sy*(-0.0000)})
	--({\sx*(3.8000)},{\sy*(-0.0000)})
	--({\sx*(3.8100)},{\sy*(-0.0000)})
	--({\sx*(3.8200)},{\sy*(-0.0000)})
	--({\sx*(3.8300)},{\sy*(-0.0000)})
	--({\sx*(3.8400)},{\sy*(-0.0000)})
	--({\sx*(3.8500)},{\sy*(-0.0000)})
	--({\sx*(3.8600)},{\sy*(-0.0000)})
	--({\sx*(3.8700)},{\sy*(-0.0000)})
	--({\sx*(3.8800)},{\sy*(-0.0000)})
	--({\sx*(3.8900)},{\sy*(-0.0000)})
	--({\sx*(3.9000)},{\sy*(-0.0000)})
	--({\sx*(3.9100)},{\sy*(0.0000)})
	--({\sx*(3.9200)},{\sy*(0.0000)})
	--({\sx*(3.9300)},{\sy*(-0.0000)})
	--({\sx*(3.9400)},{\sy*(-0.0000)})
	--({\sx*(3.9500)},{\sy*(-0.0000)})
	--({\sx*(3.9600)},{\sy*(-0.0000)})
	--({\sx*(3.9700)},{\sy*(-0.0000)})
	--({\sx*(3.9800)},{\sy*(-0.0000)})
	--({\sx*(3.9900)},{\sy*(-0.0000)})
	--({\sx*(4.0000)},{\sy*(-0.0000)})
	--({\sx*(4.0100)},{\sy*(-0.0000)})
	--({\sx*(4.0200)},{\sy*(-0.0000)})
	--({\sx*(4.0300)},{\sy*(-0.0000)})
	--({\sx*(4.0400)},{\sy*(-0.0000)})
	--({\sx*(4.0500)},{\sy*(-0.0000)})
	--({\sx*(4.0600)},{\sy*(-0.0000)})
	--({\sx*(4.0700)},{\sy*(-0.0000)})
	--({\sx*(4.0800)},{\sy*(-0.0000)})
	--({\sx*(4.0900)},{\sy*(-0.0000)})
	--({\sx*(4.1000)},{\sy*(-0.0000)})
	--({\sx*(4.1100)},{\sy*(-0.0000)})
	--({\sx*(4.1200)},{\sy*(-0.0000)})
	--({\sx*(4.1300)},{\sy*(-0.0000)})
	--({\sx*(4.1400)},{\sy*(-0.0000)})
	--({\sx*(4.1500)},{\sy*(-0.0000)})
	--({\sx*(4.1600)},{\sy*(0.0000)})
	--({\sx*(4.1700)},{\sy*(0.0000)})
	--({\sx*(4.1800)},{\sy*(0.0000)})
	--({\sx*(4.1900)},{\sy*(0.0000)})
	--({\sx*(4.2000)},{\sy*(0.0000)})
	--({\sx*(4.2100)},{\sy*(0.0000)})
	--({\sx*(4.2200)},{\sy*(0.0000)})
	--({\sx*(4.2300)},{\sy*(0.0000)})
	--({\sx*(4.2400)},{\sy*(0.0000)})
	--({\sx*(4.2500)},{\sy*(0.0000)})
	--({\sx*(4.2600)},{\sy*(0.0000)})
	--({\sx*(4.2700)},{\sy*(0.0000)})
	--({\sx*(4.2800)},{\sy*(0.0000)})
	--({\sx*(4.2900)},{\sy*(0.0000)})
	--({\sx*(4.3000)},{\sy*(0.0000)})
	--({\sx*(4.3100)},{\sy*(0.0000)})
	--({\sx*(4.3200)},{\sy*(0.0000)})
	--({\sx*(4.3300)},{\sy*(0.0000)})
	--({\sx*(4.3400)},{\sy*(0.0000)})
	--({\sx*(4.3500)},{\sy*(0.0000)})
	--({\sx*(4.3600)},{\sy*(0.0000)})
	--({\sx*(4.3700)},{\sy*(0.0000)})
	--({\sx*(4.3800)},{\sy*(-0.0000)})
	--({\sx*(4.3900)},{\sy*(-0.0000)})
	--({\sx*(4.4000)},{\sy*(-0.0000)})
	--({\sx*(4.4100)},{\sy*(-0.0000)})
	--({\sx*(4.4200)},{\sy*(-0.0000)})
	--({\sx*(4.4300)},{\sy*(-0.0000)})
	--({\sx*(4.4400)},{\sy*(-0.0000)})
	--({\sx*(4.4500)},{\sy*(-0.0000)})
	--({\sx*(4.4600)},{\sy*(-0.0000)})
	--({\sx*(4.4700)},{\sy*(-0.0000)})
	--({\sx*(4.4800)},{\sy*(-0.0000)})
	--({\sx*(4.4900)},{\sy*(-0.0000)})
	--({\sx*(4.5000)},{\sy*(-0.0000)})
	--({\sx*(4.5100)},{\sy*(-0.0000)})
	--({\sx*(4.5200)},{\sy*(-0.0000)})
	--({\sx*(4.5300)},{\sy*(-0.0000)})
	--({\sx*(4.5400)},{\sy*(-0.0000)})
	--({\sx*(4.5500)},{\sy*(-0.0000)})
	--({\sx*(4.5600)},{\sy*(0.0000)})
	--({\sx*(4.5700)},{\sy*(0.0000)})
	--({\sx*(4.5800)},{\sy*(0.0000)})
	--({\sx*(4.5900)},{\sy*(0.0000)})
	--({\sx*(4.6000)},{\sy*(0.0000)})
	--({\sx*(4.6100)},{\sy*(0.0000)})
	--({\sx*(4.6200)},{\sy*(0.0000)})
	--({\sx*(4.6300)},{\sy*(0.0000)})
	--({\sx*(4.6400)},{\sy*(0.0000)})
	--({\sx*(4.6500)},{\sy*(0.0000)})
	--({\sx*(4.6600)},{\sy*(0.0000)})
	--({\sx*(4.6700)},{\sy*(0.0000)})
	--({\sx*(4.6800)},{\sy*(0.0000)})
	--({\sx*(4.6900)},{\sy*(0.0000)})
	--({\sx*(4.7000)},{\sy*(0.0000)})
	--({\sx*(4.7100)},{\sy*(0.0000)})
	--({\sx*(4.7200)},{\sy*(-0.0000)})
	--({\sx*(4.7300)},{\sy*(-0.0000)})
	--({\sx*(4.7400)},{\sy*(-0.0000)})
	--({\sx*(4.7500)},{\sy*(-0.0000)})
	--({\sx*(4.7600)},{\sy*(-0.0000)})
	--({\sx*(4.7700)},{\sy*(-0.0000)})
	--({\sx*(4.7800)},{\sy*(-0.0000)})
	--({\sx*(4.7900)},{\sy*(-0.0000)})
	--({\sx*(4.8000)},{\sy*(-0.0000)})
	--({\sx*(4.8100)},{\sy*(-0.0000)})
	--({\sx*(4.8200)},{\sy*(-0.0000)})
	--({\sx*(4.8300)},{\sy*(-0.0000)})
	--({\sx*(4.8400)},{\sy*(0.0000)})
	--({\sx*(4.8500)},{\sy*(0.0000)})
	--({\sx*(4.8600)},{\sy*(0.0000)})
	--({\sx*(4.8700)},{\sy*(0.0000)})
	--({\sx*(4.8800)},{\sy*(0.0000)})
	--({\sx*(4.8900)},{\sy*(0.0000)})
	--({\sx*(4.9000)},{\sy*(0.0000)})
	--({\sx*(4.9100)},{\sy*(0.0000)})
	--({\sx*(4.9200)},{\sy*(0.0000)})
	--({\sx*(4.9300)},{\sy*(-0.0000)})
	--({\sx*(4.9400)},{\sy*(-0.0000)})
	--({\sx*(4.9500)},{\sy*(-0.0000)})
	--({\sx*(4.9600)},{\sy*(-0.0000)})
	--({\sx*(4.9700)},{\sy*(-0.0000)})
	--({\sx*(4.9800)},{\sy*(-0.0000)})
	--({\sx*(4.9900)},{\sy*(0.0000)})
	--({\sx*(5.0000)},{\sy*(0.0000)});
}
\def\xwerten{
\fill[color=red] (0.0000,0) circle[radius={0.07/\skala}];
\fill[color=white] (0.0000,0) circle[radius={0.05/\skala}];
\fill[color=red] (0.0157,0) circle[radius={0.07/\skala}];
\fill[color=white] (0.0157,0) circle[radius={0.05/\skala}];
\fill[color=red] (0.0627,0) circle[radius={0.07/\skala}];
\fill[color=white] (0.0627,0) circle[radius={0.05/\skala}];
\fill[color=red] (0.1403,0) circle[radius={0.07/\skala}];
\fill[color=white] (0.1403,0) circle[radius={0.05/\skala}];
\fill[color=red] (0.2476,0) circle[radius={0.07/\skala}];
\fill[color=white] (0.2476,0) circle[radius={0.05/\skala}];
\fill[color=red] (0.3832,0) circle[radius={0.07/\skala}];
\fill[color=white] (0.3832,0) circle[radius={0.05/\skala}];
\fill[color=red] (0.5454,0) circle[radius={0.07/\skala}];
\fill[color=white] (0.5454,0) circle[radius={0.05/\skala}];
\fill[color=red] (0.7322,0) circle[radius={0.07/\skala}];
\fill[color=white] (0.7322,0) circle[radius={0.05/\skala}];
\fill[color=red] (0.9413,0) circle[radius={0.07/\skala}];
\fill[color=white] (0.9413,0) circle[radius={0.05/\skala}];
\fill[color=red] (1.1699,0) circle[radius={0.07/\skala}];
\fill[color=white] (1.1699,0) circle[radius={0.05/\skala}];
\fill[color=red] (1.4153,0) circle[radius={0.07/\skala}];
\fill[color=white] (1.4153,0) circle[radius={0.05/\skala}];
\fill[color=red] (1.6743,0) circle[radius={0.07/\skala}];
\fill[color=white] (1.6743,0) circle[radius={0.05/\skala}];
\fill[color=red] (1.9437,0) circle[radius={0.07/\skala}];
\fill[color=white] (1.9437,0) circle[radius={0.05/\skala}];
\fill[color=red] (2.2201,0) circle[radius={0.07/\skala}];
\fill[color=white] (2.2201,0) circle[radius={0.05/\skala}];
\fill[color=red] (2.5000,0) circle[radius={0.07/\skala}];
\fill[color=white] (2.5000,0) circle[radius={0.05/\skala}];
\fill[color=red] (2.7799,0) circle[radius={0.07/\skala}];
\fill[color=white] (2.7799,0) circle[radius={0.05/\skala}];
\fill[color=red] (3.0563,0) circle[radius={0.07/\skala}];
\fill[color=white] (3.0563,0) circle[radius={0.05/\skala}];
\fill[color=red] (3.3257,0) circle[radius={0.07/\skala}];
\fill[color=white] (3.3257,0) circle[radius={0.05/\skala}];
\fill[color=red] (3.5847,0) circle[radius={0.07/\skala}];
\fill[color=white] (3.5847,0) circle[radius={0.05/\skala}];
\fill[color=red] (3.8301,0) circle[radius={0.07/\skala}];
\fill[color=white] (3.8301,0) circle[radius={0.05/\skala}];
\fill[color=red] (4.0587,0) circle[radius={0.07/\skala}];
\fill[color=white] (4.0587,0) circle[radius={0.05/\skala}];
\fill[color=red] (4.2678,0) circle[radius={0.07/\skala}];
\fill[color=white] (4.2678,0) circle[radius={0.05/\skala}];
\fill[color=red] (4.4546,0) circle[radius={0.07/\skala}];
\fill[color=white] (4.4546,0) circle[radius={0.05/\skala}];
\fill[color=red] (4.6168,0) circle[radius={0.07/\skala}];
\fill[color=white] (4.6168,0) circle[radius={0.05/\skala}];
\fill[color=red] (4.7524,0) circle[radius={0.07/\skala}];
\fill[color=white] (4.7524,0) circle[radius={0.05/\skala}];
\fill[color=red] (4.8597,0) circle[radius={0.07/\skala}];
\fill[color=white] (4.8597,0) circle[radius={0.05/\skala}];
\fill[color=red] (4.9373,0) circle[radius={0.07/\skala}];
\fill[color=white] (4.9373,0) circle[radius={0.05/\skala}];
\fill[color=red] (4.9843,0) circle[radius={0.07/\skala}];
\fill[color=white] (4.9843,0) circle[radius={0.05/\skala}];
\fill[color=red] (5.0000,0) circle[radius={0.07/\skala}];
\fill[color=white] (5.0000,0) circle[radius={0.05/\skala}];
}
\def\punkten{28}
\def\maxfehlern{2.715\cdot 10^{-15}}
\def\fehlern{
\draw[color=red,line width=1.4pt,line join=round] ({\sx*(0.000)},{\sy*(0.0000)})
	--({\sx*(0.0100)},{\sy*(0.0000)})
	--({\sx*(0.0200)},{\sy*(0.0409)})
	--({\sx*(0.0300)},{\sy*(-0.1227)})
	--({\sx*(0.0400)},{\sy*(0.1840)})
	--({\sx*(0.0500)},{\sy*(0.3680)})
	--({\sx*(0.0600)},{\sy*(0.0613)})
	--({\sx*(0.0700)},{\sy*(0.1022)})
	--({\sx*(0.0800)},{\sy*(-0.0818)})
	--({\sx*(0.0900)},{\sy*(-0.0409)})
	--({\sx*(0.1000)},{\sy*(-0.1227)})
	--({\sx*(0.1100)},{\sy*(-0.2453)})
	--({\sx*(0.1200)},{\sy*(-0.3066)})
	--({\sx*(0.1300)},{\sy*(0.0204)})
	--({\sx*(0.1400)},{\sy*(-0.1431)})
	--({\sx*(0.1500)},{\sy*(-0.1635)})
	--({\sx*(0.1600)},{\sy*(-0.0204)})
	--({\sx*(0.1700)},{\sy*(0.0409)})
	--({\sx*(0.1800)},{\sy*(0.0818)})
	--({\sx*(0.1900)},{\sy*(0.1022)})
	--({\sx*(0.2000)},{\sy*(0.1022)})
	--({\sx*(0.2100)},{\sy*(0.1431)})
	--({\sx*(0.2200)},{\sy*(0.1431)})
	--({\sx*(0.2300)},{\sy*(0.0204)})
	--({\sx*(0.2400)},{\sy*(-0.1431)})
	--({\sx*(0.2500)},{\sy*(-0.0613)})
	--({\sx*(0.2600)},{\sy*(0.1431)})
	--({\sx*(0.2700)},{\sy*(0.0613)})
	--({\sx*(0.2800)},{\sy*(-0.2658)})
	--({\sx*(0.2900)},{\sy*(-0.1840)})
	--({\sx*(0.3000)},{\sy*(-0.1022)})
	--({\sx*(0.3100)},{\sy*(-0.0409)})
	--({\sx*(0.3200)},{\sy*(-0.2249)})
	--({\sx*(0.3300)},{\sy*(-0.2044)})
	--({\sx*(0.3400)},{\sy*(0.0204)})
	--({\sx*(0.3500)},{\sy*(0.0204)})
	--({\sx*(0.3600)},{\sy*(0.0204)})
	--({\sx*(0.3700)},{\sy*(-0.1022)})
	--({\sx*(0.3800)},{\sy*(0.1022)})
	--({\sx*(0.3900)},{\sy*(0.2044)})
	--({\sx*(0.4000)},{\sy*(0.2453)})
	--({\sx*(0.4100)},{\sy*(0.1022)})
	--({\sx*(0.4200)},{\sy*(0.3271)})
	--({\sx*(0.4300)},{\sy*(0.3680)})
	--({\sx*(0.4400)},{\sy*(0.2453)})
	--({\sx*(0.4500)},{\sy*(0.3066)})
	--({\sx*(0.4600)},{\sy*(0.3271)})
	--({\sx*(0.4700)},{\sy*(0.3066)})
	--({\sx*(0.4800)},{\sy*(0.3475)})
	--({\sx*(0.4900)},{\sy*(0.1431)})
	--({\sx*(0.5000)},{\sy*(0.1840)})
	--({\sx*(0.5100)},{\sy*(0.2658)})
	--({\sx*(0.5200)},{\sy*(0.1635)})
	--({\sx*(0.5300)},{\sy*(0.0000)})
	--({\sx*(0.5400)},{\sy*(0.1431)})
	--({\sx*(0.5500)},{\sy*(0.0409)})
	--({\sx*(0.5600)},{\sy*(0.0409)})
	--({\sx*(0.5700)},{\sy*(-0.1840)})
	--({\sx*(0.5800)},{\sy*(-0.3475)})
	--({\sx*(0.5900)},{\sy*(-0.1227)})
	--({\sx*(0.6000)},{\sy*(-0.2453)})
	--({\sx*(0.6100)},{\sy*(-0.2658)})
	--({\sx*(0.6200)},{\sy*(-0.3066)})
	--({\sx*(0.6300)},{\sy*(-0.2453)})
	--({\sx*(0.6400)},{\sy*(-0.2453)})
	--({\sx*(0.6500)},{\sy*(-0.3271)})
	--({\sx*(0.6600)},{\sy*(-0.3884)})
	--({\sx*(0.6700)},{\sy*(-0.2044)})
	--({\sx*(0.6800)},{\sy*(-0.2658)})
	--({\sx*(0.6900)},{\sy*(-0.2453)})
	--({\sx*(0.7000)},{\sy*(-0.3271)})
	--({\sx*(0.7100)},{\sy*(-0.0613)})
	--({\sx*(0.7200)},{\sy*(-0.0204)})
	--({\sx*(0.7300)},{\sy*(-0.1635)})
	--({\sx*(0.7400)},{\sy*(-0.1635)})
	--({\sx*(0.7500)},{\sy*(-0.1431)})
	--({\sx*(0.7600)},{\sy*(0.0409)})
	--({\sx*(0.7700)},{\sy*(0.0409)})
	--({\sx*(0.7800)},{\sy*(-0.0409)})
	--({\sx*(0.7900)},{\sy*(0.0818)})
	--({\sx*(0.8000)},{\sy*(0.2044)})
	--({\sx*(0.8100)},{\sy*(0.0204)})
	--({\sx*(0.8200)},{\sy*(0.0000)})
	--({\sx*(0.8300)},{\sy*(0.0409)})
	--({\sx*(0.8400)},{\sy*(0.1635)})
	--({\sx*(0.8500)},{\sy*(0.0818)})
	--({\sx*(0.8600)},{\sy*(0.0818)})
	--({\sx*(0.8700)},{\sy*(0.0818)})
	--({\sx*(0.8800)},{\sy*(0.0613)})
	--({\sx*(0.8900)},{\sy*(0.0204)})
	--({\sx*(0.9000)},{\sy*(0.0409)})
	--({\sx*(0.9100)},{\sy*(0.0409)})
	--({\sx*(0.9200)},{\sy*(0.0818)})
	--({\sx*(0.9300)},{\sy*(-0.0409)})
	--({\sx*(0.9400)},{\sy*(0.0409)})
	--({\sx*(0.9500)},{\sy*(-0.1431)})
	--({\sx*(0.9600)},{\sy*(-0.0204)})
	--({\sx*(0.9700)},{\sy*(0.0818)})
	--({\sx*(0.9800)},{\sy*(-0.0511)})
	--({\sx*(0.9900)},{\sy*(-0.1124)})
	--({\sx*(1.0000)},{\sy*(-0.0102)})
	--({\sx*(1.0100)},{\sy*(0.0000)})
	--({\sx*(1.0200)},{\sy*(-0.1124)})
	--({\sx*(1.0300)},{\sy*(-0.2044)})
	--({\sx*(1.0400)},{\sy*(-0.0102)})
	--({\sx*(1.0500)},{\sy*(0.0716)})
	--({\sx*(1.0600)},{\sy*(-0.1124)})
	--({\sx*(1.0700)},{\sy*(-0.1431)})
	--({\sx*(1.0800)},{\sy*(-0.0204)})
	--({\sx*(1.0900)},{\sy*(0.0818)})
	--({\sx*(1.1000)},{\sy*(-0.0920)})
	--({\sx*(1.1100)},{\sy*(-0.1022)})
	--({\sx*(1.1200)},{\sy*(-0.0511)})
	--({\sx*(1.1300)},{\sy*(0.0204)})
	--({\sx*(1.1400)},{\sy*(-0.0920)})
	--({\sx*(1.1500)},{\sy*(-0.1124)})
	--({\sx*(1.1600)},{\sy*(-0.1227)})
	--({\sx*(1.1700)},{\sy*(-0.0409)})
	--({\sx*(1.1800)},{\sy*(-0.0102)})
	--({\sx*(1.1900)},{\sy*(-0.1124)})
	--({\sx*(1.2000)},{\sy*(-0.1840)})
	--({\sx*(1.2100)},{\sy*(-0.0716)})
	--({\sx*(1.2200)},{\sy*(0.0102)})
	--({\sx*(1.2300)},{\sy*(-0.1124)})
	--({\sx*(1.2400)},{\sy*(-0.1431)})
	--({\sx*(1.2500)},{\sy*(-0.0920)})
	--({\sx*(1.2600)},{\sy*(-0.0716)})
	--({\sx*(1.2700)},{\sy*(-0.1124)})
	--({\sx*(1.2800)},{\sy*(-0.1840)})
	--({\sx*(1.2900)},{\sy*(-0.1431)})
	--({\sx*(1.3000)},{\sy*(-0.0716)})
	--({\sx*(1.3100)},{\sy*(-0.1022)})
	--({\sx*(1.3200)},{\sy*(-0.1431)})
	--({\sx*(1.3300)},{\sy*(-0.1227)})
	--({\sx*(1.3400)},{\sy*(-0.0511)})
	--({\sx*(1.3500)},{\sy*(-0.1124)})
	--({\sx*(1.3600)},{\sy*(-0.1022)})
	--({\sx*(1.3700)},{\sy*(-0.0716)})
	--({\sx*(1.3800)},{\sy*(-0.0409)})
	--({\sx*(1.3900)},{\sy*(-0.0307)})
	--({\sx*(1.4000)},{\sy*(-0.0716)})
	--({\sx*(1.4100)},{\sy*(-0.0716)})
	--({\sx*(1.4200)},{\sy*(0.0102)})
	--({\sx*(1.4300)},{\sy*(0.0613)})
	--({\sx*(1.4400)},{\sy*(0.0613)})
	--({\sx*(1.4500)},{\sy*(0.0613)})
	--({\sx*(1.4600)},{\sy*(0.0613)})
	--({\sx*(1.4700)},{\sy*(0.1124)})
	--({\sx*(1.4800)},{\sy*(0.1227)})
	--({\sx*(1.4900)},{\sy*(0.1022)})
	--({\sx*(1.5000)},{\sy*(0.1635)})
	--({\sx*(1.5100)},{\sy*(0.2453)})
	--({\sx*(1.5200)},{\sy*(0.2044)})
	--({\sx*(1.5300)},{\sy*(0.1635)})
	--({\sx*(1.5400)},{\sy*(0.2198)})
	--({\sx*(1.5500)},{\sy*(0.2709)})
	--({\sx*(1.5600)},{\sy*(0.2300)})
	--({\sx*(1.5700)},{\sy*(0.1635)})
	--({\sx*(1.5800)},{\sy*(0.1993)})
	--({\sx*(1.5900)},{\sy*(0.1993)})
	--({\sx*(1.6000)},{\sy*(0.1687)})
	--({\sx*(1.6100)},{\sy*(0.1124)})
	--({\sx*(1.6200)},{\sy*(0.1431)})
	--({\sx*(1.6300)},{\sy*(0.1584)})
	--({\sx*(1.6400)},{\sy*(0.1124)})
	--({\sx*(1.6500)},{\sy*(0.0460)})
	--({\sx*(1.6600)},{\sy*(0.0460)})
	--({\sx*(1.6700)},{\sy*(0.0256)})
	--({\sx*(1.6800)},{\sy*(-0.0204)})
	--({\sx*(1.6900)},{\sy*(-0.0562)})
	--({\sx*(1.7000)},{\sy*(-0.1227)})
	--({\sx*(1.7100)},{\sy*(-0.1227)})
	--({\sx*(1.7200)},{\sy*(-0.1329)})
	--({\sx*(1.7300)},{\sy*(-0.1789)})
	--({\sx*(1.7400)},{\sy*(-0.2095)})
	--({\sx*(1.7500)},{\sy*(-0.2453)})
	--({\sx*(1.7600)},{\sy*(-0.2249)})
	--({\sx*(1.7700)},{\sy*(-0.2760)})
	--({\sx*(1.7800)},{\sy*(-0.3169)})
	--({\sx*(1.7900)},{\sy*(-0.3526)})
	--({\sx*(1.8000)},{\sy*(-0.3169)})
	--({\sx*(1.8100)},{\sy*(-0.3680)})
	--({\sx*(1.8200)},{\sy*(-0.3578)})
	--({\sx*(1.8300)},{\sy*(-0.3731)})
	--({\sx*(1.8400)},{\sy*(-0.3015)})
	--({\sx*(1.8500)},{\sy*(-0.3629)})
	--({\sx*(1.8600)},{\sy*(-0.3373)})
	--({\sx*(1.8700)},{\sy*(-0.2811)})
	--({\sx*(1.8800)},{\sy*(-0.2300)})
	--({\sx*(1.8900)},{\sy*(-0.2402)})
	--({\sx*(1.9000)},{\sy*(-0.2249)})
	--({\sx*(1.9100)},{\sy*(-0.1584)})
	--({\sx*(1.9200)},{\sy*(-0.1022)})
	--({\sx*(1.9300)},{\sy*(-0.0383)})
	--({\sx*(1.9400)},{\sy*(0.0026)})
	--({\sx*(1.9500)},{\sy*(0.0204)})
	--({\sx*(1.9600)},{\sy*(0.0716)})
	--({\sx*(1.9700)},{\sy*(0.1457)})
	--({\sx*(1.9800)},{\sy*(0.1738)})
	--({\sx*(1.9900)},{\sy*(0.2249)})
	--({\sx*(2.0000)},{\sy*(0.2734)})
	--({\sx*(2.0100)},{\sy*(0.3143)})
	--({\sx*(2.0200)},{\sy*(0.3424)})
	--({\sx*(2.0300)},{\sy*(0.3935)})
	--({\sx*(2.0400)},{\sy*(0.4140)})
	--({\sx*(2.0500)},{\sy*(0.4574)})
	--({\sx*(2.0600)},{\sy*(0.4702)})
	--({\sx*(2.0700)},{\sy*(0.4830)})
	--({\sx*(2.0800)},{\sy*(0.4906)})
	--({\sx*(2.0900)},{\sy*(0.5111)})
	--({\sx*(2.1000)},{\sy*(0.4855)})
	--({\sx*(2.1100)},{\sy*(0.4906)})
	--({\sx*(2.1200)},{\sy*(0.4753)})
	--({\sx*(2.1300)},{\sy*(0.4268)})
	--({\sx*(2.1400)},{\sy*(0.4012)})
	--({\sx*(2.1500)},{\sy*(0.3782)})
	--({\sx*(2.1600)},{\sy*(0.3245)})
	--({\sx*(2.1700)},{\sy*(0.2836)})
	--({\sx*(2.1800)},{\sy*(0.2325)})
	--({\sx*(2.1900)},{\sy*(0.1840)})
	--({\sx*(2.2000)},{\sy*(0.1252)})
	--({\sx*(2.2100)},{\sy*(0.0664)})
	--({\sx*(2.2200)},{\sy*(-0.0051)})
	--({\sx*(2.2300)},{\sy*(-0.0767)})
	--({\sx*(2.2400)},{\sy*(-0.1354)})
	--({\sx*(2.2500)},{\sy*(-0.1993)})
	--({\sx*(2.2600)},{\sy*(-0.2594)})
	--({\sx*(2.2700)},{\sy*(-0.3348)})
	--({\sx*(2.2800)},{\sy*(-0.3731)})
	--({\sx*(2.2900)},{\sy*(-0.4383)})
	--({\sx*(2.3000)},{\sy*(-0.4727)})
	--({\sx*(2.3100)},{\sy*(-0.5136)})
	--({\sx*(2.3200)},{\sy*(-0.5622)})
	--({\sx*(2.3300)},{\sy*(-0.5814)})
	--({\sx*(2.3400)},{\sy*(-0.6095)})
	--({\sx*(2.3500)},{\sy*(-0.6159)})
	--({\sx*(2.3600)},{\sy*(-0.6363)})
	--({\sx*(2.3700)},{\sy*(-0.6491)})
	--({\sx*(2.3800)},{\sy*(-0.6389)})
	--({\sx*(2.3900)},{\sy*(-0.6120)})
	--({\sx*(2.4000)},{\sy*(-0.5916)})
	--({\sx*(2.4100)},{\sy*(-0.5635)})
	--({\sx*(2.4200)},{\sy*(-0.5175)})
	--({\sx*(2.4300)},{\sy*(-0.4817)})
	--({\sx*(2.4400)},{\sy*(-0.4216)})
	--({\sx*(2.4500)},{\sy*(-0.3565)})
	--({\sx*(2.4600)},{\sy*(-0.2977)})
	--({\sx*(2.4700)},{\sy*(-0.2236)})
	--({\sx*(2.4800)},{\sy*(-0.1482)})
	--({\sx*(2.4900)},{\sy*(-0.0779)})
	--({\sx*(2.5000)},{\sy*(0.0000)})
	--({\sx*(2.5100)},{\sy*(0.0805)})
	--({\sx*(2.5200)},{\sy*(0.1610)})
	--({\sx*(2.5300)},{\sy*(0.2313)})
	--({\sx*(2.5400)},{\sy*(0.3066)})
	--({\sx*(2.5500)},{\sy*(0.3846)})
	--({\sx*(2.5600)},{\sy*(0.4593)})
	--({\sx*(2.5700)},{\sy*(0.5239)})
	--({\sx*(2.5800)},{\sy*(0.5769)})
	--({\sx*(2.5900)},{\sy*(0.6254)})
	--({\sx*(2.6000)},{\sy*(0.6746)})
	--({\sx*(2.6100)},{\sy*(0.7066)})
	--({\sx*(2.6200)},{\sy*(0.7334)})
	--({\sx*(2.6300)},{\sy*(0.7532)})
	--({\sx*(2.6400)},{\sy*(0.7647)})
	--({\sx*(2.6500)},{\sy*(0.7609)})
	--({\sx*(2.6600)},{\sy*(0.7558)})
	--({\sx*(2.6700)},{\sy*(0.7315)})
	--({\sx*(2.6800)},{\sy*(0.7091)})
	--({\sx*(2.6900)},{\sy*(0.6580)})
	--({\sx*(2.7000)},{\sy*(0.6254)})
	--({\sx*(2.7100)},{\sy*(0.5526)})
	--({\sx*(2.7200)},{\sy*(0.4887)})
	--({\sx*(2.7300)},{\sy*(0.4261)})
	--({\sx*(2.7400)},{\sy*(0.3469)})
	--({\sx*(2.7500)},{\sy*(0.2664)})
	--({\sx*(2.7600)},{\sy*(0.1795)})
	--({\sx*(2.7700)},{\sy*(0.0882)})
	--({\sx*(2.7800)},{\sy*(0.0000)})
	--({\sx*(2.7900)},{\sy*(-0.0926)})
	--({\sx*(2.8000)},{\sy*(-0.1814)})
	--({\sx*(2.8100)},{\sy*(-0.2814)})
	--({\sx*(2.8200)},{\sy*(-0.3696)})
	--({\sx*(2.8300)},{\sy*(-0.4462)})
	--({\sx*(2.8400)},{\sy*(-0.5264)})
	--({\sx*(2.8500)},{\sy*(-0.5951)})
	--({\sx*(2.8600)},{\sy*(-0.6618)})
	--({\sx*(2.8700)},{\sy*(-0.7213)})
	--({\sx*(2.8800)},{\sy*(-0.7669)})
	--({\sx*(2.8900)},{\sy*(-0.8062)})
	--({\sx*(2.9000)},{\sy*(-0.8446)})
	--({\sx*(2.9100)},{\sy*(-0.8618)})
	--({\sx*(2.9200)},{\sy*(-0.8637)})
	--({\sx*(2.9300)},{\sy*(-0.8663)})
	--({\sx*(2.9400)},{\sy*(-0.8417)})
	--({\sx*(2.9500)},{\sy*(-0.8203)})
	--({\sx*(2.9600)},{\sy*(-0.7807)})
	--({\sx*(2.9700)},{\sy*(-0.7382)})
	--({\sx*(2.9800)},{\sy*(-0.6705)})
	--({\sx*(2.9900)},{\sy*(-0.6044)})
	--({\sx*(3.0000)},{\sy*(-0.5395)})
	--({\sx*(3.0100)},{\sy*(-0.4513)})
	--({\sx*(3.0200)},{\sy*(-0.3590)})
	--({\sx*(3.0300)},{\sy*(-0.2670)})
	--({\sx*(3.0400)},{\sy*(-0.1671)})
	--({\sx*(3.0500)},{\sy*(-0.0658)})
	--({\sx*(3.0600)},{\sy*(0.0379)})
	--({\sx*(3.0700)},{\sy*(0.1423)})
	--({\sx*(3.0800)},{\sy*(0.2453)})
	--({\sx*(3.0900)},{\sy*(0.3471)})
	--({\sx*(3.1000)},{\sy*(0.4464)})
	--({\sx*(3.1100)},{\sy*(0.5277)})
	--({\sx*(3.1200)},{\sy*(0.6144)})
	--({\sx*(3.1300)},{\sy*(0.6927)})
	--({\sx*(3.1400)},{\sy*(0.7629)})
	--({\sx*(3.1500)},{\sy*(0.8216)})
	--({\sx*(3.1600)},{\sy*(0.8637)})
	--({\sx*(3.1700)},{\sy*(0.9003)})
	--({\sx*(3.1800)},{\sy*(0.9243)})
	--({\sx*(3.1900)},{\sy*(0.9372)})
	--({\sx*(3.2000)},{\sy*(0.9364)})
	--({\sx*(3.2100)},{\sy*(0.9150)})
	--({\sx*(3.2200)},{\sy*(0.9013)})
	--({\sx*(3.2300)},{\sy*(0.8428)})
	--({\sx*(3.2400)},{\sy*(0.8037)})
	--({\sx*(3.2500)},{\sy*(0.7401)})
	--({\sx*(3.2600)},{\sy*(0.6662)})
	--({\sx*(3.2700)},{\sy*(0.5865)})
	--({\sx*(3.2800)},{\sy*(0.4929)})
	--({\sx*(3.2900)},{\sy*(0.3929)})
	--({\sx*(3.3000)},{\sy*(0.2879)})
	--({\sx*(3.3100)},{\sy*(0.1785)})
	--({\sx*(3.3200)},{\sy*(0.0649)})
	--({\sx*(3.3300)},{\sy*(-0.0494)})
	--({\sx*(3.3400)},{\sy*(-0.1634)})
	--({\sx*(3.3500)},{\sy*(-0.2756)})
	--({\sx*(3.3600)},{\sy*(-0.3880)})
	--({\sx*(3.3700)},{\sy*(-0.4914)})
	--({\sx*(3.3800)},{\sy*(-0.5850)})
	--({\sx*(3.3900)},{\sy*(-0.6766)})
	--({\sx*(3.4000)},{\sy*(-0.7565)})
	--({\sx*(3.4100)},{\sy*(-0.8272)})
	--({\sx*(3.4200)},{\sy*(-0.8845)})
	--({\sx*(3.4300)},{\sy*(-0.9306)})
	--({\sx*(3.4400)},{\sy*(-0.9594)})
	--({\sx*(3.4500)},{\sy*(-0.9817)})
	--({\sx*(3.4600)},{\sy*(-0.9782)})
	--({\sx*(3.4700)},{\sy*(-0.9750)})
	--({\sx*(3.4800)},{\sy*(-0.9489)})
	--({\sx*(3.4900)},{\sy*(-0.9089)})
	--({\sx*(3.5000)},{\sy*(-0.8626)})
	--({\sx*(3.5100)},{\sy*(-0.7906)})
	--({\sx*(3.5200)},{\sy*(-0.7080)})
	--({\sx*(3.5300)},{\sy*(-0.6190)})
	--({\sx*(3.5400)},{\sy*(-0.5221)})
	--({\sx*(3.5500)},{\sy*(-0.4115)})
	--({\sx*(3.5600)},{\sy*(-0.2969)})
	--({\sx*(3.5700)},{\sy*(-0.1810)})
	--({\sx*(3.5800)},{\sy*(-0.0583)})
	--({\sx*(3.5900)},{\sy*(0.0656)})
	--({\sx*(3.6000)},{\sy*(0.1890)})
	--({\sx*(3.6100)},{\sy*(0.3086)})
	--({\sx*(3.6200)},{\sy*(0.4238)})
	--({\sx*(3.6300)},{\sy*(0.5348)})
	--({\sx*(3.6400)},{\sy*(0.6385)})
	--({\sx*(3.6500)},{\sy*(0.7232)})
	--({\sx*(3.6600)},{\sy*(0.8080)})
	--({\sx*(3.6700)},{\sy*(0.8721)})
	--({\sx*(3.6800)},{\sy*(0.9291)})
	--({\sx*(3.6900)},{\sy*(0.9710)})
	--({\sx*(3.7000)},{\sy*(0.9848)})
	--({\sx*(3.7100)},{\sy*(1.0000)})
	--({\sx*(3.7200)},{\sy*(0.9833)})
	--({\sx*(3.7300)},{\sy*(0.9603)})
	--({\sx*(3.7400)},{\sy*(0.9156)})
	--({\sx*(3.7500)},{\sy*(0.8607)})
	--({\sx*(3.7600)},{\sy*(0.7864)})
	--({\sx*(3.7700)},{\sy*(0.7069)})
	--({\sx*(3.7800)},{\sy*(0.6106)})
	--({\sx*(3.7900)},{\sy*(0.4994)})
	--({\sx*(3.8000)},{\sy*(0.3804)})
	--({\sx*(3.8100)},{\sy*(0.2594)})
	--({\sx*(3.8200)},{\sy*(0.1318)})
	--({\sx*(3.8300)},{\sy*(0.0011)})
	--({\sx*(3.8400)},{\sy*(-0.1298)})
	--({\sx*(3.8500)},{\sy*(-0.2580)})
	--({\sx*(3.8600)},{\sy*(-0.3815)})
	--({\sx*(3.8700)},{\sy*(-0.5025)})
	--({\sx*(3.8800)},{\sy*(-0.6049)})
	--({\sx*(3.8900)},{\sy*(-0.7061)})
	--({\sx*(3.9000)},{\sy*(-0.7903)})
	--({\sx*(3.9100)},{\sy*(-0.8620)})
	--({\sx*(3.9200)},{\sy*(-0.9226)})
	--({\sx*(3.9300)},{\sy*(-0.9543)})
	--({\sx*(3.9400)},{\sy*(-0.9729)})
	--({\sx*(3.9500)},{\sy*(-0.9745)})
	--({\sx*(3.9600)},{\sy*(-0.9551)})
	--({\sx*(3.9700)},{\sy*(-0.9194)})
	--({\sx*(3.9800)},{\sy*(-0.8702)})
	--({\sx*(3.9900)},{\sy*(-0.7967)})
	--({\sx*(4.0000)},{\sy*(-0.7161)})
	--({\sx*(4.0100)},{\sy*(-0.6191)})
	--({\sx*(4.0200)},{\sy*(-0.5055)})
	--({\sx*(4.0300)},{\sy*(-0.3832)})
	--({\sx*(4.0400)},{\sy*(-0.2525)})
	--({\sx*(4.0500)},{\sy*(-0.1192)})
	--({\sx*(4.0600)},{\sy*(0.0175)})
	--({\sx*(4.0700)},{\sy*(0.1527)})
	--({\sx*(4.0800)},{\sy*(0.2869)})
	--({\sx*(4.0900)},{\sy*(0.4152)})
	--({\sx*(4.1000)},{\sy*(0.5301)})
	--({\sx*(4.1100)},{\sy*(0.6365)})
	--({\sx*(4.1200)},{\sy*(0.7297)})
	--({\sx*(4.1300)},{\sy*(0.8069)})
	--({\sx*(4.1400)},{\sy*(0.8645)})
	--({\sx*(4.1500)},{\sy*(0.9062)})
	--({\sx*(4.1600)},{\sy*(0.9280)})
	--({\sx*(4.1700)},{\sy*(0.9305)})
	--({\sx*(4.1800)},{\sy*(0.9025)})
	--({\sx*(4.1900)},{\sy*(0.8616)})
	--({\sx*(4.2000)},{\sy*(0.7997)})
	--({\sx*(4.2100)},{\sy*(0.7199)})
	--({\sx*(4.2200)},{\sy*(0.6189)})
	--({\sx*(4.2300)},{\sy*(0.5075)})
	--({\sx*(4.2400)},{\sy*(0.3842)})
	--({\sx*(4.2500)},{\sy*(0.2489)})
	--({\sx*(4.2600)},{\sy*(0.1102)})
	--({\sx*(4.2700)},{\sy*(-0.0317)})
	--({\sx*(4.2800)},{\sy*(-0.1725)})
	--({\sx*(4.2900)},{\sy*(-0.3079)})
	--({\sx*(4.3000)},{\sy*(-0.4337)})
	--({\sx*(4.3100)},{\sy*(-0.5531)})
	--({\sx*(4.3200)},{\sy*(-0.6489)})
	--({\sx*(4.3300)},{\sy*(-0.7323)})
	--({\sx*(4.3400)},{\sy*(-0.7986)})
	--({\sx*(4.3500)},{\sy*(-0.8406)})
	--({\sx*(4.3600)},{\sy*(-0.8564)})
	--({\sx*(4.3700)},{\sy*(-0.8461)})
	--({\sx*(4.3800)},{\sy*(-0.8200)})
	--({\sx*(4.3900)},{\sy*(-0.7624)})
	--({\sx*(4.4000)},{\sy*(-0.6857)})
	--({\sx*(4.4100)},{\sy*(-0.5910)})
	--({\sx*(4.4200)},{\sy*(-0.4781)})
	--({\sx*(4.4300)},{\sy*(-0.3481)})
	--({\sx*(4.4400)},{\sy*(-0.2100)})
	--({\sx*(4.4500)},{\sy*(-0.0664)})
	--({\sx*(4.4600)},{\sy*(0.0786)})
	--({\sx*(4.4700)},{\sy*(0.2209)})
	--({\sx*(4.4800)},{\sy*(0.3550)})
	--({\sx*(4.4900)},{\sy*(0.4743)})
	--({\sx*(4.5000)},{\sy*(0.5785)})
	--({\sx*(4.5100)},{\sy*(0.6617)})
	--({\sx*(4.5200)},{\sy*(0.7227)})
	--({\sx*(4.5300)},{\sy*(0.7516)})
	--({\sx*(4.5400)},{\sy*(0.7600)})
	--({\sx*(4.5500)},{\sy*(0.7298)})
	--({\sx*(4.5600)},{\sy*(0.6833)})
	--({\sx*(4.5700)},{\sy*(0.6058)})
	--({\sx*(4.5800)},{\sy*(0.5045)})
	--({\sx*(4.5900)},{\sy*(0.3819)})
	--({\sx*(4.6000)},{\sy*(0.2467)})
	--({\sx*(4.6100)},{\sy*(0.1014)})
	--({\sx*(4.6200)},{\sy*(-0.0472)})
	--({\sx*(4.6300)},{\sy*(-0.1911)})
	--({\sx*(4.6400)},{\sy*(-0.3260)})
	--({\sx*(4.6500)},{\sy*(-0.4446)})
	--({\sx*(4.6600)},{\sy*(-0.5383)})
	--({\sx*(4.6700)},{\sy*(-0.6090)})
	--({\sx*(4.6800)},{\sy*(-0.6430)})
	--({\sx*(4.6900)},{\sy*(-0.6447)})
	--({\sx*(4.7000)},{\sy*(-0.6090)})
	--({\sx*(4.7100)},{\sy*(-0.5453)})
	--({\sx*(4.7200)},{\sy*(-0.4465)})
	--({\sx*(4.7300)},{\sy*(-0.3246)})
	--({\sx*(4.7400)},{\sy*(-0.1852)})
	--({\sx*(4.7500)},{\sy*(-0.0365)})
	--({\sx*(4.7600)},{\sy*(0.1130)})
	--({\sx*(4.7700)},{\sy*(0.2515)})
	--({\sx*(4.7800)},{\sy*(0.3684)})
	--({\sx*(4.7900)},{\sy*(0.4578)})
	--({\sx*(4.8000)},{\sy*(0.5070)})
	--({\sx*(4.8100)},{\sy*(0.5166)})
	--({\sx*(4.8200)},{\sy*(0.4823)})
	--({\sx*(4.8300)},{\sy*(0.4024)})
	--({\sx*(4.8400)},{\sy*(0.2887)})
	--({\sx*(4.8500)},{\sy*(0.1473)})
	--({\sx*(4.8600)},{\sy*(-0.0044)})
	--({\sx*(4.8700)},{\sy*(-0.1500)})
	--({\sx*(4.8800)},{\sy*(-0.2717)})
	--({\sx*(4.8900)},{\sy*(-0.3535)})
	--({\sx*(4.9000)},{\sy*(-0.3787)})
	--({\sx*(4.9100)},{\sy*(-0.3441)})
	--({\sx*(4.9200)},{\sy*(-0.2494)})
	--({\sx*(4.9300)},{\sy*(-0.1121)})
	--({\sx*(4.9400)},{\sy*(0.0405)})
	--({\sx*(4.9500)},{\sy*(0.1702)})
	--({\sx*(4.9600)},{\sy*(0.2328)})
	--({\sx*(4.9700)},{\sy*(0.1968)})
	--({\sx*(4.9800)},{\sy*(0.0680)})
	--({\sx*(4.9900)},{\sy*(-0.0731)})
	--({\sx*(5.0000)},{\sy*(0.0000)});
}
\def\relfehlern{
\draw[color=blue,line width=1.4pt,line join=round] ({\sx*(0.000)},{\sy*(0.0000)})
	--({\sx*(0.0100)},{\sy*(0.0000)})
	--({\sx*(0.0200)},{\sy*(0.0000)})
	--({\sx*(0.0300)},{\sy*(-0.0000)})
	--({\sx*(0.0400)},{\sy*(0.0000)})
	--({\sx*(0.0500)},{\sy*(0.0000)})
	--({\sx*(0.0600)},{\sy*(0.0000)})
	--({\sx*(0.0700)},{\sy*(0.0000)})
	--({\sx*(0.0800)},{\sy*(-0.0000)})
	--({\sx*(0.0900)},{\sy*(-0.0000)})
	--({\sx*(0.1000)},{\sy*(-0.0000)})
	--({\sx*(0.1100)},{\sy*(-0.0000)})
	--({\sx*(0.1200)},{\sy*(-0.0000)})
	--({\sx*(0.1300)},{\sy*(0.0000)})
	--({\sx*(0.1400)},{\sy*(-0.0000)})
	--({\sx*(0.1500)},{\sy*(-0.0000)})
	--({\sx*(0.1600)},{\sy*(-0.0000)})
	--({\sx*(0.1700)},{\sy*(0.0000)})
	--({\sx*(0.1800)},{\sy*(0.0000)})
	--({\sx*(0.1900)},{\sy*(0.0000)})
	--({\sx*(0.2000)},{\sy*(0.0000)})
	--({\sx*(0.2100)},{\sy*(0.0000)})
	--({\sx*(0.2200)},{\sy*(0.0000)})
	--({\sx*(0.2300)},{\sy*(0.0000)})
	--({\sx*(0.2400)},{\sy*(-0.0000)})
	--({\sx*(0.2500)},{\sy*(-0.0000)})
	--({\sx*(0.2600)},{\sy*(0.0000)})
	--({\sx*(0.2700)},{\sy*(0.0000)})
	--({\sx*(0.2800)},{\sy*(-0.0000)})
	--({\sx*(0.2900)},{\sy*(-0.0000)})
	--({\sx*(0.3000)},{\sy*(-0.0000)})
	--({\sx*(0.3100)},{\sy*(-0.0000)})
	--({\sx*(0.3200)},{\sy*(-0.0000)})
	--({\sx*(0.3300)},{\sy*(-0.0000)})
	--({\sx*(0.3400)},{\sy*(0.0000)})
	--({\sx*(0.3500)},{\sy*(0.0000)})
	--({\sx*(0.3600)},{\sy*(0.0000)})
	--({\sx*(0.3700)},{\sy*(-0.0000)})
	--({\sx*(0.3800)},{\sy*(0.0000)})
	--({\sx*(0.3900)},{\sy*(0.0000)})
	--({\sx*(0.4000)},{\sy*(0.0000)})
	--({\sx*(0.4100)},{\sy*(0.0000)})
	--({\sx*(0.4200)},{\sy*(0.0000)})
	--({\sx*(0.4300)},{\sy*(0.0000)})
	--({\sx*(0.4400)},{\sy*(0.0000)})
	--({\sx*(0.4500)},{\sy*(0.0000)})
	--({\sx*(0.4600)},{\sy*(0.0000)})
	--({\sx*(0.4700)},{\sy*(0.0000)})
	--({\sx*(0.4800)},{\sy*(0.0000)})
	--({\sx*(0.4900)},{\sy*(0.0000)})
	--({\sx*(0.5000)},{\sy*(0.0000)})
	--({\sx*(0.5100)},{\sy*(0.0000)})
	--({\sx*(0.5200)},{\sy*(0.0000)})
	--({\sx*(0.5300)},{\sy*(0.0000)})
	--({\sx*(0.5400)},{\sy*(0.0000)})
	--({\sx*(0.5500)},{\sy*(0.0000)})
	--({\sx*(0.5600)},{\sy*(0.0000)})
	--({\sx*(0.5700)},{\sy*(-0.0000)})
	--({\sx*(0.5800)},{\sy*(-0.0000)})
	--({\sx*(0.5900)},{\sy*(-0.0000)})
	--({\sx*(0.6000)},{\sy*(-0.0000)})
	--({\sx*(0.6100)},{\sy*(-0.0000)})
	--({\sx*(0.6200)},{\sy*(-0.0000)})
	--({\sx*(0.6300)},{\sy*(-0.0000)})
	--({\sx*(0.6400)},{\sy*(-0.0000)})
	--({\sx*(0.6500)},{\sy*(-0.0000)})
	--({\sx*(0.6600)},{\sy*(-0.0000)})
	--({\sx*(0.6700)},{\sy*(-0.0000)})
	--({\sx*(0.6800)},{\sy*(-0.0000)})
	--({\sx*(0.6900)},{\sy*(-0.0000)})
	--({\sx*(0.7000)},{\sy*(-0.0000)})
	--({\sx*(0.7100)},{\sy*(-0.0000)})
	--({\sx*(0.7200)},{\sy*(-0.0000)})
	--({\sx*(0.7300)},{\sy*(-0.0000)})
	--({\sx*(0.7400)},{\sy*(-0.0000)})
	--({\sx*(0.7500)},{\sy*(-0.0000)})
	--({\sx*(0.7600)},{\sy*(0.0000)})
	--({\sx*(0.7700)},{\sy*(0.0000)})
	--({\sx*(0.7800)},{\sy*(-0.0000)})
	--({\sx*(0.7900)},{\sy*(0.0000)})
	--({\sx*(0.8000)},{\sy*(0.0000)})
	--({\sx*(0.8100)},{\sy*(0.0000)})
	--({\sx*(0.8200)},{\sy*(0.0000)})
	--({\sx*(0.8300)},{\sy*(0.0000)})
	--({\sx*(0.8400)},{\sy*(0.0000)})
	--({\sx*(0.8500)},{\sy*(0.0000)})
	--({\sx*(0.8600)},{\sy*(0.0000)})
	--({\sx*(0.8700)},{\sy*(0.0000)})
	--({\sx*(0.8800)},{\sy*(0.0000)})
	--({\sx*(0.8900)},{\sy*(0.0000)})
	--({\sx*(0.9000)},{\sy*(0.0000)})
	--({\sx*(0.9100)},{\sy*(0.0000)})
	--({\sx*(0.9200)},{\sy*(0.0000)})
	--({\sx*(0.9300)},{\sy*(-0.0000)})
	--({\sx*(0.9400)},{\sy*(0.0000)})
	--({\sx*(0.9500)},{\sy*(-0.0000)})
	--({\sx*(0.9600)},{\sy*(-0.0000)})
	--({\sx*(0.9700)},{\sy*(0.0000)})
	--({\sx*(0.9800)},{\sy*(-0.0000)})
	--({\sx*(0.9900)},{\sy*(-0.0000)})
	--({\sx*(1.0000)},{\sy*(-0.0000)})
	--({\sx*(1.0100)},{\sy*(0.0000)})
	--({\sx*(1.0200)},{\sy*(-0.0000)})
	--({\sx*(1.0300)},{\sy*(-0.0000)})
	--({\sx*(1.0400)},{\sy*(-0.0000)})
	--({\sx*(1.0500)},{\sy*(0.0000)})
	--({\sx*(1.0600)},{\sy*(-0.0000)})
	--({\sx*(1.0700)},{\sy*(-0.0000)})
	--({\sx*(1.0800)},{\sy*(-0.0000)})
	--({\sx*(1.0900)},{\sy*(0.0000)})
	--({\sx*(1.1000)},{\sy*(-0.0000)})
	--({\sx*(1.1100)},{\sy*(-0.0000)})
	--({\sx*(1.1200)},{\sy*(-0.0000)})
	--({\sx*(1.1300)},{\sy*(0.0000)})
	--({\sx*(1.1400)},{\sy*(-0.0000)})
	--({\sx*(1.1500)},{\sy*(-0.0000)})
	--({\sx*(1.1600)},{\sy*(-0.0000)})
	--({\sx*(1.1700)},{\sy*(-0.0000)})
	--({\sx*(1.1800)},{\sy*(-0.0000)})
	--({\sx*(1.1900)},{\sy*(-0.0000)})
	--({\sx*(1.2000)},{\sy*(-0.0000)})
	--({\sx*(1.2100)},{\sy*(-0.0000)})
	--({\sx*(1.2200)},{\sy*(0.0000)})
	--({\sx*(1.2300)},{\sy*(-0.0000)})
	--({\sx*(1.2400)},{\sy*(-0.0000)})
	--({\sx*(1.2500)},{\sy*(-0.0000)})
	--({\sx*(1.2600)},{\sy*(-0.0000)})
	--({\sx*(1.2700)},{\sy*(-0.0000)})
	--({\sx*(1.2800)},{\sy*(-0.0000)})
	--({\sx*(1.2900)},{\sy*(-0.0000)})
	--({\sx*(1.3000)},{\sy*(-0.0000)})
	--({\sx*(1.3100)},{\sy*(-0.0000)})
	--({\sx*(1.3200)},{\sy*(-0.0000)})
	--({\sx*(1.3300)},{\sy*(-0.0000)})
	--({\sx*(1.3400)},{\sy*(-0.0000)})
	--({\sx*(1.3500)},{\sy*(-0.0000)})
	--({\sx*(1.3600)},{\sy*(-0.0000)})
	--({\sx*(1.3700)},{\sy*(-0.0000)})
	--({\sx*(1.3800)},{\sy*(-0.0000)})
	--({\sx*(1.3900)},{\sy*(-0.0000)})
	--({\sx*(1.4000)},{\sy*(-0.0000)})
	--({\sx*(1.4100)},{\sy*(-0.0000)})
	--({\sx*(1.4200)},{\sy*(0.0000)})
	--({\sx*(1.4300)},{\sy*(0.0000)})
	--({\sx*(1.4400)},{\sy*(0.0000)})
	--({\sx*(1.4500)},{\sy*(0.0000)})
	--({\sx*(1.4600)},{\sy*(0.0000)})
	--({\sx*(1.4700)},{\sy*(0.0000)})
	--({\sx*(1.4800)},{\sy*(0.0000)})
	--({\sx*(1.4900)},{\sy*(0.0000)})
	--({\sx*(1.5000)},{\sy*(0.0000)})
	--({\sx*(1.5100)},{\sy*(0.0000)})
	--({\sx*(1.5200)},{\sy*(0.0000)})
	--({\sx*(1.5300)},{\sy*(0.0000)})
	--({\sx*(1.5400)},{\sy*(0.0000)})
	--({\sx*(1.5500)},{\sy*(0.0000)})
	--({\sx*(1.5600)},{\sy*(0.0000)})
	--({\sx*(1.5700)},{\sy*(0.0000)})
	--({\sx*(1.5800)},{\sy*(0.0000)})
	--({\sx*(1.5900)},{\sy*(0.0000)})
	--({\sx*(1.6000)},{\sy*(0.0000)})
	--({\sx*(1.6100)},{\sy*(0.0000)})
	--({\sx*(1.6200)},{\sy*(0.0000)})
	--({\sx*(1.6300)},{\sy*(0.0000)})
	--({\sx*(1.6400)},{\sy*(0.0000)})
	--({\sx*(1.6500)},{\sy*(0.0000)})
	--({\sx*(1.6600)},{\sy*(0.0000)})
	--({\sx*(1.6700)},{\sy*(0.0000)})
	--({\sx*(1.6800)},{\sy*(-0.0000)})
	--({\sx*(1.6900)},{\sy*(-0.0000)})
	--({\sx*(1.7000)},{\sy*(-0.0000)})
	--({\sx*(1.7100)},{\sy*(-0.0000)})
	--({\sx*(1.7200)},{\sy*(-0.0000)})
	--({\sx*(1.7300)},{\sy*(-0.0000)})
	--({\sx*(1.7400)},{\sy*(-0.0000)})
	--({\sx*(1.7500)},{\sy*(-0.0000)})
	--({\sx*(1.7600)},{\sy*(-0.0000)})
	--({\sx*(1.7700)},{\sy*(-0.0000)})
	--({\sx*(1.7800)},{\sy*(-0.0000)})
	--({\sx*(1.7900)},{\sy*(-0.0000)})
	--({\sx*(1.8000)},{\sy*(-0.0000)})
	--({\sx*(1.8100)},{\sy*(-0.0000)})
	--({\sx*(1.8200)},{\sy*(-0.0000)})
	--({\sx*(1.8300)},{\sy*(-0.0000)})
	--({\sx*(1.8400)},{\sy*(-0.0000)})
	--({\sx*(1.8500)},{\sy*(-0.0000)})
	--({\sx*(1.8600)},{\sy*(-0.0000)})
	--({\sx*(1.8700)},{\sy*(-0.0000)})
	--({\sx*(1.8800)},{\sy*(-0.0000)})
	--({\sx*(1.8900)},{\sy*(-0.0000)})
	--({\sx*(1.9000)},{\sy*(-0.0000)})
	--({\sx*(1.9100)},{\sy*(-0.0000)})
	--({\sx*(1.9200)},{\sy*(-0.0000)})
	--({\sx*(1.9300)},{\sy*(-0.0000)})
	--({\sx*(1.9400)},{\sy*(0.0000)})
	--({\sx*(1.9500)},{\sy*(0.0000)})
	--({\sx*(1.9600)},{\sy*(0.0000)})
	--({\sx*(1.9700)},{\sy*(0.0000)})
	--({\sx*(1.9800)},{\sy*(0.0000)})
	--({\sx*(1.9900)},{\sy*(0.0000)})
	--({\sx*(2.0000)},{\sy*(0.0000)})
	--({\sx*(2.0100)},{\sy*(0.0000)})
	--({\sx*(2.0200)},{\sy*(0.0000)})
	--({\sx*(2.0300)},{\sy*(0.0000)})
	--({\sx*(2.0400)},{\sy*(0.0000)})
	--({\sx*(2.0500)},{\sy*(0.0000)})
	--({\sx*(2.0600)},{\sy*(0.0000)})
	--({\sx*(2.0700)},{\sy*(0.0000)})
	--({\sx*(2.0800)},{\sy*(0.0000)})
	--({\sx*(2.0900)},{\sy*(0.0000)})
	--({\sx*(2.1000)},{\sy*(0.0000)})
	--({\sx*(2.1100)},{\sy*(0.0000)})
	--({\sx*(2.1200)},{\sy*(0.0000)})
	--({\sx*(2.1300)},{\sy*(0.0000)})
	--({\sx*(2.1400)},{\sy*(0.0000)})
	--({\sx*(2.1500)},{\sy*(0.0000)})
	--({\sx*(2.1600)},{\sy*(0.0000)})
	--({\sx*(2.1700)},{\sy*(0.0000)})
	--({\sx*(2.1800)},{\sy*(0.0000)})
	--({\sx*(2.1900)},{\sy*(0.0000)})
	--({\sx*(2.2000)},{\sy*(0.0000)})
	--({\sx*(2.2100)},{\sy*(0.0000)})
	--({\sx*(2.2200)},{\sy*(-0.0000)})
	--({\sx*(2.2300)},{\sy*(-0.0000)})
	--({\sx*(2.2400)},{\sy*(-0.0000)})
	--({\sx*(2.2500)},{\sy*(-0.0000)})
	--({\sx*(2.2600)},{\sy*(-0.0000)})
	--({\sx*(2.2700)},{\sy*(-0.0000)})
	--({\sx*(2.2800)},{\sy*(-0.0000)})
	--({\sx*(2.2900)},{\sy*(-0.0000)})
	--({\sx*(2.3000)},{\sy*(-0.0000)})
	--({\sx*(2.3100)},{\sy*(-0.0000)})
	--({\sx*(2.3200)},{\sy*(-0.0000)})
	--({\sx*(2.3300)},{\sy*(-0.0000)})
	--({\sx*(2.3400)},{\sy*(-0.0000)})
	--({\sx*(2.3500)},{\sy*(-0.0000)})
	--({\sx*(2.3600)},{\sy*(-0.0000)})
	--({\sx*(2.3700)},{\sy*(-0.0000)})
	--({\sx*(2.3800)},{\sy*(-0.0000)})
	--({\sx*(2.3900)},{\sy*(-0.0000)})
	--({\sx*(2.4000)},{\sy*(-0.0000)})
	--({\sx*(2.4100)},{\sy*(-0.0000)})
	--({\sx*(2.4200)},{\sy*(-0.0000)})
	--({\sx*(2.4300)},{\sy*(-0.0000)})
	--({\sx*(2.4400)},{\sy*(-0.0000)})
	--({\sx*(2.4500)},{\sy*(-0.0000)})
	--({\sx*(2.4600)},{\sy*(-0.0000)})
	--({\sx*(2.4700)},{\sy*(-0.0000)})
	--({\sx*(2.4800)},{\sy*(-0.0000)})
	--({\sx*(2.4900)},{\sy*(-0.0000)})
	--({\sx*(2.5000)},{\sy*(0.0000)})
	--({\sx*(2.5100)},{\sy*(0.0000)})
	--({\sx*(2.5200)},{\sy*(0.0000)})
	--({\sx*(2.5300)},{\sy*(0.0000)})
	--({\sx*(2.5400)},{\sy*(0.0000)})
	--({\sx*(2.5500)},{\sy*(0.0000)})
	--({\sx*(2.5600)},{\sy*(0.0000)})
	--({\sx*(2.5700)},{\sy*(0.0000)})
	--({\sx*(2.5800)},{\sy*(0.0000)})
	--({\sx*(2.5900)},{\sy*(0.0000)})
	--({\sx*(2.6000)},{\sy*(0.0000)})
	--({\sx*(2.6100)},{\sy*(0.0000)})
	--({\sx*(2.6200)},{\sy*(0.0000)})
	--({\sx*(2.6300)},{\sy*(0.0000)})
	--({\sx*(2.6400)},{\sy*(0.0000)})
	--({\sx*(2.6500)},{\sy*(0.0000)})
	--({\sx*(2.6600)},{\sy*(0.0000)})
	--({\sx*(2.6700)},{\sy*(0.0000)})
	--({\sx*(2.6800)},{\sy*(0.0000)})
	--({\sx*(2.6900)},{\sy*(0.0000)})
	--({\sx*(2.7000)},{\sy*(0.0000)})
	--({\sx*(2.7100)},{\sy*(0.0000)})
	--({\sx*(2.7200)},{\sy*(0.0000)})
	--({\sx*(2.7300)},{\sy*(0.0000)})
	--({\sx*(2.7400)},{\sy*(0.0000)})
	--({\sx*(2.7500)},{\sy*(0.0000)})
	--({\sx*(2.7600)},{\sy*(0.0000)})
	--({\sx*(2.7700)},{\sy*(0.0000)})
	--({\sx*(2.7800)},{\sy*(0.0000)})
	--({\sx*(2.7900)},{\sy*(-0.0000)})
	--({\sx*(2.8000)},{\sy*(-0.0000)})
	--({\sx*(2.8100)},{\sy*(-0.0000)})
	--({\sx*(2.8200)},{\sy*(-0.0000)})
	--({\sx*(2.8300)},{\sy*(-0.0000)})
	--({\sx*(2.8400)},{\sy*(-0.0000)})
	--({\sx*(2.8500)},{\sy*(-0.0000)})
	--({\sx*(2.8600)},{\sy*(-0.0000)})
	--({\sx*(2.8700)},{\sy*(-0.0000)})
	--({\sx*(2.8800)},{\sy*(-0.0000)})
	--({\sx*(2.8900)},{\sy*(-0.0000)})
	--({\sx*(2.9000)},{\sy*(-0.0000)})
	--({\sx*(2.9100)},{\sy*(-0.0000)})
	--({\sx*(2.9200)},{\sy*(-0.0000)})
	--({\sx*(2.9300)},{\sy*(-0.0000)})
	--({\sx*(2.9400)},{\sy*(-0.0000)})
	--({\sx*(2.9500)},{\sy*(-0.0000)})
	--({\sx*(2.9600)},{\sy*(-0.0000)})
	--({\sx*(2.9700)},{\sy*(-0.0000)})
	--({\sx*(2.9800)},{\sy*(-0.0000)})
	--({\sx*(2.9900)},{\sy*(-0.0000)})
	--({\sx*(3.0000)},{\sy*(-0.0000)})
	--({\sx*(3.0100)},{\sy*(-0.0000)})
	--({\sx*(3.0200)},{\sy*(-0.0000)})
	--({\sx*(3.0300)},{\sy*(-0.0000)})
	--({\sx*(3.0400)},{\sy*(-0.0000)})
	--({\sx*(3.0500)},{\sy*(-0.0000)})
	--({\sx*(3.0600)},{\sy*(0.0000)})
	--({\sx*(3.0700)},{\sy*(0.0000)})
	--({\sx*(3.0800)},{\sy*(0.0000)})
	--({\sx*(3.0900)},{\sy*(0.0000)})
	--({\sx*(3.1000)},{\sy*(0.0000)})
	--({\sx*(3.1100)},{\sy*(0.0000)})
	--({\sx*(3.1200)},{\sy*(0.0000)})
	--({\sx*(3.1300)},{\sy*(0.0000)})
	--({\sx*(3.1400)},{\sy*(0.0000)})
	--({\sx*(3.1500)},{\sy*(0.0000)})
	--({\sx*(3.1600)},{\sy*(0.0000)})
	--({\sx*(3.1700)},{\sy*(0.0000)})
	--({\sx*(3.1800)},{\sy*(0.0000)})
	--({\sx*(3.1900)},{\sy*(0.0000)})
	--({\sx*(3.2000)},{\sy*(0.0000)})
	--({\sx*(3.2100)},{\sy*(0.0000)})
	--({\sx*(3.2200)},{\sy*(0.0000)})
	--({\sx*(3.2300)},{\sy*(0.0000)})
	--({\sx*(3.2400)},{\sy*(0.0000)})
	--({\sx*(3.2500)},{\sy*(0.0000)})
	--({\sx*(3.2600)},{\sy*(0.0000)})
	--({\sx*(3.2700)},{\sy*(0.0000)})
	--({\sx*(3.2800)},{\sy*(0.0000)})
	--({\sx*(3.2900)},{\sy*(0.0000)})
	--({\sx*(3.3000)},{\sy*(0.0000)})
	--({\sx*(3.3100)},{\sy*(0.0000)})
	--({\sx*(3.3200)},{\sy*(0.0000)})
	--({\sx*(3.3300)},{\sy*(-0.0000)})
	--({\sx*(3.3400)},{\sy*(-0.0000)})
	--({\sx*(3.3500)},{\sy*(-0.0000)})
	--({\sx*(3.3600)},{\sy*(-0.0000)})
	--({\sx*(3.3700)},{\sy*(-0.0000)})
	--({\sx*(3.3800)},{\sy*(-0.0000)})
	--({\sx*(3.3900)},{\sy*(-0.0000)})
	--({\sx*(3.4000)},{\sy*(-0.0000)})
	--({\sx*(3.4100)},{\sy*(-0.0000)})
	--({\sx*(3.4200)},{\sy*(-0.0000)})
	--({\sx*(3.4300)},{\sy*(-0.0000)})
	--({\sx*(3.4400)},{\sy*(-0.0000)})
	--({\sx*(3.4500)},{\sy*(-0.0000)})
	--({\sx*(3.4600)},{\sy*(-0.0000)})
	--({\sx*(3.4700)},{\sy*(-0.0000)})
	--({\sx*(3.4800)},{\sy*(-0.0000)})
	--({\sx*(3.4900)},{\sy*(-0.0000)})
	--({\sx*(3.5000)},{\sy*(-0.0000)})
	--({\sx*(3.5100)},{\sy*(-0.0000)})
	--({\sx*(3.5200)},{\sy*(-0.0000)})
	--({\sx*(3.5300)},{\sy*(-0.0000)})
	--({\sx*(3.5400)},{\sy*(-0.0000)})
	--({\sx*(3.5500)},{\sy*(-0.0000)})
	--({\sx*(3.5600)},{\sy*(-0.0000)})
	--({\sx*(3.5700)},{\sy*(-0.0000)})
	--({\sx*(3.5800)},{\sy*(-0.0000)})
	--({\sx*(3.5900)},{\sy*(0.0000)})
	--({\sx*(3.6000)},{\sy*(0.0000)})
	--({\sx*(3.6100)},{\sy*(0.0000)})
	--({\sx*(3.6200)},{\sy*(0.0000)})
	--({\sx*(3.6300)},{\sy*(0.0000)})
	--({\sx*(3.6400)},{\sy*(0.0000)})
	--({\sx*(3.6500)},{\sy*(0.0000)})
	--({\sx*(3.6600)},{\sy*(0.0000)})
	--({\sx*(3.6700)},{\sy*(0.0000)})
	--({\sx*(3.6800)},{\sy*(0.0000)})
	--({\sx*(3.6900)},{\sy*(0.0000)})
	--({\sx*(3.7000)},{\sy*(0.0000)})
	--({\sx*(3.7100)},{\sy*(0.0000)})
	--({\sx*(3.7200)},{\sy*(0.0000)})
	--({\sx*(3.7300)},{\sy*(0.0000)})
	--({\sx*(3.7400)},{\sy*(0.0000)})
	--({\sx*(3.7500)},{\sy*(0.0000)})
	--({\sx*(3.7600)},{\sy*(0.0000)})
	--({\sx*(3.7700)},{\sy*(0.0000)})
	--({\sx*(3.7800)},{\sy*(0.0000)})
	--({\sx*(3.7900)},{\sy*(0.0000)})
	--({\sx*(3.8000)},{\sy*(0.0000)})
	--({\sx*(3.8100)},{\sy*(0.0000)})
	--({\sx*(3.8200)},{\sy*(0.0000)})
	--({\sx*(3.8300)},{\sy*(0.0000)})
	--({\sx*(3.8400)},{\sy*(-0.0000)})
	--({\sx*(3.8500)},{\sy*(-0.0000)})
	--({\sx*(3.8600)},{\sy*(-0.0000)})
	--({\sx*(3.8700)},{\sy*(-0.0000)})
	--({\sx*(3.8800)},{\sy*(-0.0000)})
	--({\sx*(3.8900)},{\sy*(-0.0000)})
	--({\sx*(3.9000)},{\sy*(-0.0000)})
	--({\sx*(3.9100)},{\sy*(-0.0000)})
	--({\sx*(3.9200)},{\sy*(-0.0000)})
	--({\sx*(3.9300)},{\sy*(-0.0000)})
	--({\sx*(3.9400)},{\sy*(-0.0000)})
	--({\sx*(3.9500)},{\sy*(-0.0000)})
	--({\sx*(3.9600)},{\sy*(-0.0000)})
	--({\sx*(3.9700)},{\sy*(-0.0000)})
	--({\sx*(3.9800)},{\sy*(-0.0000)})
	--({\sx*(3.9900)},{\sy*(-0.0000)})
	--({\sx*(4.0000)},{\sy*(-0.0000)})
	--({\sx*(4.0100)},{\sy*(-0.0000)})
	--({\sx*(4.0200)},{\sy*(-0.0000)})
	--({\sx*(4.0300)},{\sy*(-0.0000)})
	--({\sx*(4.0400)},{\sy*(-0.0000)})
	--({\sx*(4.0500)},{\sy*(-0.0000)})
	--({\sx*(4.0600)},{\sy*(0.0000)})
	--({\sx*(4.0700)},{\sy*(0.0000)})
	--({\sx*(4.0800)},{\sy*(0.0000)})
	--({\sx*(4.0900)},{\sy*(0.0000)})
	--({\sx*(4.1000)},{\sy*(0.0000)})
	--({\sx*(4.1100)},{\sy*(0.0000)})
	--({\sx*(4.1200)},{\sy*(0.0000)})
	--({\sx*(4.1300)},{\sy*(0.0000)})
	--({\sx*(4.1400)},{\sy*(0.0000)})
	--({\sx*(4.1500)},{\sy*(0.0000)})
	--({\sx*(4.1600)},{\sy*(0.0000)})
	--({\sx*(4.1700)},{\sy*(0.0000)})
	--({\sx*(4.1800)},{\sy*(0.0000)})
	--({\sx*(4.1900)},{\sy*(0.0000)})
	--({\sx*(4.2000)},{\sy*(0.0000)})
	--({\sx*(4.2100)},{\sy*(0.0000)})
	--({\sx*(4.2200)},{\sy*(0.0000)})
	--({\sx*(4.2300)},{\sy*(0.0000)})
	--({\sx*(4.2400)},{\sy*(0.0000)})
	--({\sx*(4.2500)},{\sy*(0.0000)})
	--({\sx*(4.2600)},{\sy*(0.0000)})
	--({\sx*(4.2700)},{\sy*(-0.0000)})
	--({\sx*(4.2800)},{\sy*(-0.0000)})
	--({\sx*(4.2900)},{\sy*(-0.0000)})
	--({\sx*(4.3000)},{\sy*(-0.0000)})
	--({\sx*(4.3100)},{\sy*(-0.0000)})
	--({\sx*(4.3200)},{\sy*(-0.0000)})
	--({\sx*(4.3300)},{\sy*(-0.0000)})
	--({\sx*(4.3400)},{\sy*(-0.0000)})
	--({\sx*(4.3500)},{\sy*(-0.0000)})
	--({\sx*(4.3600)},{\sy*(-0.0000)})
	--({\sx*(4.3700)},{\sy*(-0.0000)})
	--({\sx*(4.3800)},{\sy*(-0.0000)})
	--({\sx*(4.3900)},{\sy*(-0.0000)})
	--({\sx*(4.4000)},{\sy*(-0.0000)})
	--({\sx*(4.4100)},{\sy*(-0.0000)})
	--({\sx*(4.4200)},{\sy*(-0.0000)})
	--({\sx*(4.4300)},{\sy*(-0.0000)})
	--({\sx*(4.4400)},{\sy*(-0.0000)})
	--({\sx*(4.4500)},{\sy*(-0.0000)})
	--({\sx*(4.4600)},{\sy*(0.0000)})
	--({\sx*(4.4700)},{\sy*(0.0000)})
	--({\sx*(4.4800)},{\sy*(0.0000)})
	--({\sx*(4.4900)},{\sy*(0.0000)})
	--({\sx*(4.5000)},{\sy*(0.0000)})
	--({\sx*(4.5100)},{\sy*(0.0000)})
	--({\sx*(4.5200)},{\sy*(0.0000)})
	--({\sx*(4.5300)},{\sy*(0.0000)})
	--({\sx*(4.5400)},{\sy*(0.0000)})
	--({\sx*(4.5500)},{\sy*(0.0000)})
	--({\sx*(4.5600)},{\sy*(0.0000)})
	--({\sx*(4.5700)},{\sy*(0.0000)})
	--({\sx*(4.5800)},{\sy*(0.0000)})
	--({\sx*(4.5900)},{\sy*(0.0000)})
	--({\sx*(4.6000)},{\sy*(0.0000)})
	--({\sx*(4.6100)},{\sy*(0.0000)})
	--({\sx*(4.6200)},{\sy*(-0.0000)})
	--({\sx*(4.6300)},{\sy*(-0.0000)})
	--({\sx*(4.6400)},{\sy*(-0.0000)})
	--({\sx*(4.6500)},{\sy*(-0.0000)})
	--({\sx*(4.6600)},{\sy*(-0.0000)})
	--({\sx*(4.6700)},{\sy*(-0.0000)})
	--({\sx*(4.6800)},{\sy*(-0.0000)})
	--({\sx*(4.6900)},{\sy*(-0.0000)})
	--({\sx*(4.7000)},{\sy*(-0.0000)})
	--({\sx*(4.7100)},{\sy*(-0.0000)})
	--({\sx*(4.7200)},{\sy*(-0.0000)})
	--({\sx*(4.7300)},{\sy*(-0.0000)})
	--({\sx*(4.7400)},{\sy*(-0.0000)})
	--({\sx*(4.7500)},{\sy*(-0.0000)})
	--({\sx*(4.7600)},{\sy*(0.0000)})
	--({\sx*(4.7700)},{\sy*(0.0000)})
	--({\sx*(4.7800)},{\sy*(0.0000)})
	--({\sx*(4.7900)},{\sy*(0.0000)})
	--({\sx*(4.8000)},{\sy*(0.0000)})
	--({\sx*(4.8100)},{\sy*(0.0000)})
	--({\sx*(4.8200)},{\sy*(0.0000)})
	--({\sx*(4.8300)},{\sy*(0.0000)})
	--({\sx*(4.8400)},{\sy*(0.0000)})
	--({\sx*(4.8500)},{\sy*(0.0000)})
	--({\sx*(4.8600)},{\sy*(-0.0000)})
	--({\sx*(4.8700)},{\sy*(-0.0000)})
	--({\sx*(4.8800)},{\sy*(-0.0000)})
	--({\sx*(4.8900)},{\sy*(-0.0000)})
	--({\sx*(4.9000)},{\sy*(-0.0000)})
	--({\sx*(4.9100)},{\sy*(-0.0000)})
	--({\sx*(4.9200)},{\sy*(-0.0000)})
	--({\sx*(4.9300)},{\sy*(-0.0000)})
	--({\sx*(4.9400)},{\sy*(0.0000)})
	--({\sx*(4.9500)},{\sy*(0.0000)})
	--({\sx*(4.9600)},{\sy*(0.0000)})
	--({\sx*(4.9700)},{\sy*(0.0000)})
	--({\sx*(4.9800)},{\sy*(0.0000)})
	--({\sx*(4.9900)},{\sy*(-0.0000)})
	--({\sx*(5.0000)},{\sy*(0.0000)});
}
\def\xwerteo{
\fill[color=red] (0.0000,0) circle[radius={0.07/\skala}];
\fill[color=white] (0.0000,0) circle[radius={0.05/\skala}];
\fill[color=red] (0.0137,0) circle[radius={0.07/\skala}];
\fill[color=white] (0.0137,0) circle[radius={0.05/\skala}];
\fill[color=red] (0.0546,0) circle[radius={0.07/\skala}];
\fill[color=white] (0.0546,0) circle[radius={0.05/\skala}];
\fill[color=red] (0.1224,0) circle[radius={0.07/\skala}];
\fill[color=white] (0.1224,0) circle[radius={0.05/\skala}];
\fill[color=red] (0.2161,0) circle[radius={0.07/\skala}];
\fill[color=white] (0.2161,0) circle[radius={0.05/\skala}];
\fill[color=red] (0.3349,0) circle[radius={0.07/\skala}];
\fill[color=white] (0.3349,0) circle[radius={0.05/\skala}];
\fill[color=red] (0.4775,0) circle[radius={0.07/\skala}];
\fill[color=white] (0.4775,0) circle[radius={0.05/\skala}];
\fill[color=red] (0.6421,0) circle[radius={0.07/\skala}];
\fill[color=white] (0.6421,0) circle[radius={0.05/\skala}];
\fill[color=red] (0.8272,0) circle[radius={0.07/\skala}];
\fill[color=white] (0.8272,0) circle[radius={0.05/\skala}];
\fill[color=red] (1.0305,0) circle[radius={0.07/\skala}];
\fill[color=white] (1.0305,0) circle[radius={0.05/\skala}];
\fill[color=red] (1.2500,0) circle[radius={0.07/\skala}];
\fill[color=white] (1.2500,0) circle[radius={0.05/\skala}];
\fill[color=red] (1.4832,0) circle[radius={0.07/\skala}];
\fill[color=white] (1.4832,0) circle[radius={0.05/\skala}];
\fill[color=red] (1.7275,0) circle[radius={0.07/\skala}];
\fill[color=white] (1.7275,0) circle[radius={0.05/\skala}];
\fill[color=red] (1.9802,0) circle[radius={0.07/\skala}];
\fill[color=white] (1.9802,0) circle[radius={0.05/\skala}];
\fill[color=red] (2.2387,0) circle[radius={0.07/\skala}];
\fill[color=white] (2.2387,0) circle[radius={0.05/\skala}];
\fill[color=red] (2.5000,0) circle[radius={0.07/\skala}];
\fill[color=white] (2.5000,0) circle[radius={0.05/\skala}];
\fill[color=red] (2.7613,0) circle[radius={0.07/\skala}];
\fill[color=white] (2.7613,0) circle[radius={0.05/\skala}];
\fill[color=red] (3.0198,0) circle[radius={0.07/\skala}];
\fill[color=white] (3.0198,0) circle[radius={0.05/\skala}];
\fill[color=red] (3.2725,0) circle[radius={0.07/\skala}];
\fill[color=white] (3.2725,0) circle[radius={0.05/\skala}];
\fill[color=red] (3.5168,0) circle[radius={0.07/\skala}];
\fill[color=white] (3.5168,0) circle[radius={0.05/\skala}];
\fill[color=red] (3.7500,0) circle[radius={0.07/\skala}];
\fill[color=white] (3.7500,0) circle[radius={0.05/\skala}];
\fill[color=red] (3.9695,0) circle[radius={0.07/\skala}];
\fill[color=white] (3.9695,0) circle[radius={0.05/\skala}];
\fill[color=red] (4.1728,0) circle[radius={0.07/\skala}];
\fill[color=white] (4.1728,0) circle[radius={0.05/\skala}];
\fill[color=red] (4.3579,0) circle[radius={0.07/\skala}];
\fill[color=white] (4.3579,0) circle[radius={0.05/\skala}];
\fill[color=red] (4.5225,0) circle[radius={0.07/\skala}];
\fill[color=white] (4.5225,0) circle[radius={0.05/\skala}];
\fill[color=red] (4.6651,0) circle[radius={0.07/\skala}];
\fill[color=white] (4.6651,0) circle[radius={0.05/\skala}];
\fill[color=red] (4.7839,0) circle[radius={0.07/\skala}];
\fill[color=white] (4.7839,0) circle[radius={0.05/\skala}];
\fill[color=red] (4.8776,0) circle[radius={0.07/\skala}];
\fill[color=white] (4.8776,0) circle[radius={0.05/\skala}];
\fill[color=red] (4.9454,0) circle[radius={0.07/\skala}];
\fill[color=white] (4.9454,0) circle[radius={0.05/\skala}];
\fill[color=red] (4.9863,0) circle[radius={0.07/\skala}];
\fill[color=white] (4.9863,0) circle[radius={0.05/\skala}];
\fill[color=red] (5.0000,0) circle[radius={0.07/\skala}];
\fill[color=white] (5.0000,0) circle[radius={0.05/\skala}];
}
\def\punkteo{30}
\def\maxfehlero{6.106\cdot 10^{-16}}
\def\fehlero{
\draw[color=red,line width=1.4pt,line join=round] ({\sx*(0.000)},{\sy*(0.0000)})
	--({\sx*(0.0100)},{\sy*(1.0000)})
	--({\sx*(0.0200)},{\sy*(1.0000)})
	--({\sx*(0.0300)},{\sy*(0.5455)})
	--({\sx*(0.0400)},{\sy*(0.1818)})
	--({\sx*(0.0500)},{\sy*(0.4545)})
	--({\sx*(0.0600)},{\sy*(0.0000)})
	--({\sx*(0.0700)},{\sy*(-0.3636)})
	--({\sx*(0.0800)},{\sy*(-0.4545)})
	--({\sx*(0.0900)},{\sy*(0.2727)})
	--({\sx*(0.1000)},{\sy*(-0.3636)})
	--({\sx*(0.1100)},{\sy*(-0.5455)})
	--({\sx*(0.1200)},{\sy*(-1.0000)})
	--({\sx*(0.1300)},{\sy*(0.2727)})
	--({\sx*(0.1400)},{\sy*(-0.2727)})
	--({\sx*(0.1500)},{\sy*(-0.9091)})
	--({\sx*(0.1600)},{\sy*(-0.6364)})
	--({\sx*(0.1700)},{\sy*(-0.3636)})
	--({\sx*(0.1800)},{\sy*(-0.0909)})
	--({\sx*(0.1900)},{\sy*(-0.4545)})
	--({\sx*(0.2000)},{\sy*(-0.4545)})
	--({\sx*(0.2100)},{\sy*(0.0000)})
	--({\sx*(0.2200)},{\sy*(-0.2727)})
	--({\sx*(0.2300)},{\sy*(-0.4545)})
	--({\sx*(0.2400)},{\sy*(-1.0000)})
	--({\sx*(0.2500)},{\sy*(-0.6364)})
	--({\sx*(0.2600)},{\sy*(-0.5455)})
	--({\sx*(0.2700)},{\sy*(-0.0909)})
	--({\sx*(0.2800)},{\sy*(-0.4545)})
	--({\sx*(0.2900)},{\sy*(-0.6364)})
	--({\sx*(0.3000)},{\sy*(0.0000)})
	--({\sx*(0.3100)},{\sy*(-0.0909)})
	--({\sx*(0.3200)},{\sy*(0.0000)})
	--({\sx*(0.3300)},{\sy*(-0.0909)})
	--({\sx*(0.3400)},{\sy*(0.3636)})
	--({\sx*(0.3500)},{\sy*(0.6364)})
	--({\sx*(0.3600)},{\sy*(0.8182)})
	--({\sx*(0.3700)},{\sy*(0.0909)})
	--({\sx*(0.3800)},{\sy*(1.0000)})
	--({\sx*(0.3900)},{\sy*(0.2727)})
	--({\sx*(0.4000)},{\sy*(0.7273)})
	--({\sx*(0.4100)},{\sy*(0.1818)})
	--({\sx*(0.4200)},{\sy*(0.8182)})
	--({\sx*(0.4300)},{\sy*(0.9091)})
	--({\sx*(0.4400)},{\sy*(0.3636)})
	--({\sx*(0.4500)},{\sy*(0.0909)})
	--({\sx*(0.4600)},{\sy*(0.8182)})
	--({\sx*(0.4700)},{\sy*(0.6364)})
	--({\sx*(0.4800)},{\sy*(-0.3636)})
	--({\sx*(0.4900)},{\sy*(-0.4545)})
	--({\sx*(0.5000)},{\sy*(0.3636)})
	--({\sx*(0.5100)},{\sy*(0.4545)})
	--({\sx*(0.5200)},{\sy*(0.6364)})
	--({\sx*(0.5300)},{\sy*(-0.2727)})
	--({\sx*(0.5400)},{\sy*(-0.2727)})
	--({\sx*(0.5500)},{\sy*(0.0909)})
	--({\sx*(0.5600)},{\sy*(-0.7273)})
	--({\sx*(0.5700)},{\sy*(-0.3636)})
	--({\sx*(0.5800)},{\sy*(-0.4545)})
	--({\sx*(0.5900)},{\sy*(-0.0909)})
	--({\sx*(0.6000)},{\sy*(0.0000)})
	--({\sx*(0.6100)},{\sy*(-0.0909)})
	--({\sx*(0.6200)},{\sy*(-0.3636)})
	--({\sx*(0.6300)},{\sy*(-0.0909)})
	--({\sx*(0.6400)},{\sy*(-0.0909)})
	--({\sx*(0.6500)},{\sy*(-0.4545)})
	--({\sx*(0.6600)},{\sy*(-0.6364)})
	--({\sx*(0.6700)},{\sy*(0.0909)})
	--({\sx*(0.6800)},{\sy*(-0.1818)})
	--({\sx*(0.6900)},{\sy*(0.2727)})
	--({\sx*(0.7000)},{\sy*(-0.1818)})
	--({\sx*(0.7100)},{\sy*(0.7273)})
	--({\sx*(0.7200)},{\sy*(0.3636)})
	--({\sx*(0.7300)},{\sy*(0.0909)})
	--({\sx*(0.7400)},{\sy*(0.2727)})
	--({\sx*(0.7500)},{\sy*(-0.1818)})
	--({\sx*(0.7600)},{\sy*(0.3636)})
	--({\sx*(0.7700)},{\sy*(0.4545)})
	--({\sx*(0.7800)},{\sy*(-0.0909)})
	--({\sx*(0.7900)},{\sy*(0.8182)})
	--({\sx*(0.8000)},{\sy*(0.3636)})
	--({\sx*(0.8100)},{\sy*(0.0909)})
	--({\sx*(0.8200)},{\sy*(0.0909)})
	--({\sx*(0.8300)},{\sy*(0.3636)})
	--({\sx*(0.8400)},{\sy*(0.5455)})
	--({\sx*(0.8500)},{\sy*(-0.0909)})
	--({\sx*(0.8600)},{\sy*(-0.2727)})
	--({\sx*(0.8700)},{\sy*(-0.5455)})
	--({\sx*(0.8800)},{\sy*(0.2727)})
	--({\sx*(0.8900)},{\sy*(0.0000)})
	--({\sx*(0.9000)},{\sy*(0.0000)})
	--({\sx*(0.9100)},{\sy*(-0.7273)})
	--({\sx*(0.9200)},{\sy*(0.0000)})
	--({\sx*(0.9300)},{\sy*(-0.5455)})
	--({\sx*(0.9400)},{\sy*(0.0000)})
	--({\sx*(0.9500)},{\sy*(-0.5455)})
	--({\sx*(0.9600)},{\sy*(0.0000)})
	--({\sx*(0.9700)},{\sy*(0.2727)})
	--({\sx*(0.9800)},{\sy*(-0.2727)})
	--({\sx*(0.9900)},{\sy*(-0.5000)})
	--({\sx*(1.0000)},{\sy*(-0.0909)})
	--({\sx*(1.0100)},{\sy*(0.2727)})
	--({\sx*(1.0200)},{\sy*(-0.2727)})
	--({\sx*(1.0300)},{\sy*(-0.1818)})
	--({\sx*(1.0400)},{\sy*(0.0000)})
	--({\sx*(1.0500)},{\sy*(0.5455)})
	--({\sx*(1.0600)},{\sy*(-0.0909)})
	--({\sx*(1.0700)},{\sy*(0.0000)})
	--({\sx*(1.0800)},{\sy*(0.0909)})
	--({\sx*(1.0900)},{\sy*(0.3182)})
	--({\sx*(1.1000)},{\sy*(0.0455)})
	--({\sx*(1.1100)},{\sy*(0.0455)})
	--({\sx*(1.1200)},{\sy*(0.3182)})
	--({\sx*(1.1300)},{\sy*(0.5000)})
	--({\sx*(1.1400)},{\sy*(0.4091)})
	--({\sx*(1.1500)},{\sy*(-0.2727)})
	--({\sx*(1.1600)},{\sy*(0.0000)})
	--({\sx*(1.1700)},{\sy*(0.1364)})
	--({\sx*(1.1800)},{\sy*(0.4545)})
	--({\sx*(1.1900)},{\sy*(0.2727)})
	--({\sx*(1.2000)},{\sy*(-0.3636)})
	--({\sx*(1.2100)},{\sy*(0.2273)})
	--({\sx*(1.2200)},{\sy*(0.4091)})
	--({\sx*(1.2300)},{\sy*(0.0000)})
	--({\sx*(1.2400)},{\sy*(-0.3182)})
	--({\sx*(1.2500)},{\sy*(0.3636)})
	--({\sx*(1.2600)},{\sy*(0.0455)})
	--({\sx*(1.2700)},{\sy*(-0.0455)})
	--({\sx*(1.2800)},{\sy*(-0.4091)})
	--({\sx*(1.2900)},{\sy*(-0.4091)})
	--({\sx*(1.3000)},{\sy*(0.0909)})
	--({\sx*(1.3100)},{\sy*(-0.2727)})
	--({\sx*(1.3200)},{\sy*(-0.5455)})
	--({\sx*(1.3300)},{\sy*(-0.1364)})
	--({\sx*(1.3400)},{\sy*(-0.2273)})
	--({\sx*(1.3500)},{\sy*(-0.2727)})
	--({\sx*(1.3600)},{\sy*(-0.5455)})
	--({\sx*(1.3700)},{\sy*(-0.4091)})
	--({\sx*(1.3800)},{\sy*(-0.2727)})
	--({\sx*(1.3900)},{\sy*(-0.2273)})
	--({\sx*(1.4000)},{\sy*(-0.6364)})
	--({\sx*(1.4100)},{\sy*(-0.5455)})
	--({\sx*(1.4200)},{\sy*(-0.4091)})
	--({\sx*(1.4300)},{\sy*(-0.1364)})
	--({\sx*(1.4400)},{\sy*(-0.1818)})
	--({\sx*(1.4500)},{\sy*(-0.2273)})
	--({\sx*(1.4600)},{\sy*(-0.0909)})
	--({\sx*(1.4700)},{\sy*(0.0455)})
	--({\sx*(1.4800)},{\sy*(-0.0455)})
	--({\sx*(1.4900)},{\sy*(-0.1364)})
	--({\sx*(1.5000)},{\sy*(0.0000)})
	--({\sx*(1.5100)},{\sy*(0.2727)})
	--({\sx*(1.5200)},{\sy*(0.3182)})
	--({\sx*(1.5300)},{\sy*(-0.0227)})
	--({\sx*(1.5400)},{\sy*(0.2045)})
	--({\sx*(1.5500)},{\sy*(0.3864)})
	--({\sx*(1.5600)},{\sy*(0.3182)})
	--({\sx*(1.5700)},{\sy*(0.2045)})
	--({\sx*(1.5800)},{\sy*(0.4318)})
	--({\sx*(1.5900)},{\sy*(0.4545)})
	--({\sx*(1.6000)},{\sy*(0.2045)})
	--({\sx*(1.6100)},{\sy*(0.1818)})
	--({\sx*(1.6200)},{\sy*(0.4091)})
	--({\sx*(1.6300)},{\sy*(0.3864)})
	--({\sx*(1.6400)},{\sy*(0.4545)})
	--({\sx*(1.6500)},{\sy*(0.1818)})
	--({\sx*(1.6600)},{\sy*(0.3409)})
	--({\sx*(1.6700)},{\sy*(0.2955)})
	--({\sx*(1.6800)},{\sy*(0.4091)})
	--({\sx*(1.6900)},{\sy*(0.3636)})
	--({\sx*(1.7000)},{\sy*(0.0227)})
	--({\sx*(1.7100)},{\sy*(0.2955)})
	--({\sx*(1.7200)},{\sy*(0.3636)})
	--({\sx*(1.7300)},{\sy*(0.2273)})
	--({\sx*(1.7400)},{\sy*(-0.0455)})
	--({\sx*(1.7500)},{\sy*(-0.1591)})
	--({\sx*(1.7600)},{\sy*(0.1591)})
	--({\sx*(1.7700)},{\sy*(0.0682)})
	--({\sx*(1.7800)},{\sy*(-0.1818)})
	--({\sx*(1.7900)},{\sy*(-0.1364)})
	--({\sx*(1.8000)},{\sy*(-0.0682)})
	--({\sx*(1.8100)},{\sy*(-0.0909)})
	--({\sx*(1.8200)},{\sy*(-0.2500)})
	--({\sx*(1.8300)},{\sy*(-0.2273)})
	--({\sx*(1.8400)},{\sy*(-0.1364)})
	--({\sx*(1.8500)},{\sy*(-0.2273)})
	--({\sx*(1.8600)},{\sy*(-0.2727)})
	--({\sx*(1.8700)},{\sy*(-0.2273)})
	--({\sx*(1.8800)},{\sy*(-0.1591)})
	--({\sx*(1.8900)},{\sy*(-0.2955)})
	--({\sx*(1.9000)},{\sy*(-0.3864)})
	--({\sx*(1.9100)},{\sy*(-0.1818)})
	--({\sx*(1.9200)},{\sy*(-0.2273)})
	--({\sx*(1.9300)},{\sy*(-0.0227)})
	--({\sx*(1.9400)},{\sy*(-0.1136)})
	--({\sx*(1.9500)},{\sy*(-0.2045)})
	--({\sx*(1.9600)},{\sy*(-0.1023)})
	--({\sx*(1.9700)},{\sy*(0.0568)})
	--({\sx*(1.9800)},{\sy*(-0.0114)})
	--({\sx*(1.9900)},{\sy*(0.0227)})
	--({\sx*(2.0000)},{\sy*(0.0000)})
	--({\sx*(2.0100)},{\sy*(0.0909)})
	--({\sx*(2.0200)},{\sy*(0.1932)})
	--({\sx*(2.0300)},{\sy*(0.1477)})
	--({\sx*(2.0400)},{\sy*(0.1364)})
	--({\sx*(2.0500)},{\sy*(0.3182)})
	--({\sx*(2.0600)},{\sy*(0.2500)})
	--({\sx*(2.0700)},{\sy*(0.2841)})
	--({\sx*(2.0800)},{\sy*(0.2955)})
	--({\sx*(2.0900)},{\sy*(0.2841)})
	--({\sx*(2.1000)},{\sy*(0.2955)})
	--({\sx*(2.1100)},{\sy*(0.3977)})
	--({\sx*(2.1200)},{\sy*(0.3295)})
	--({\sx*(2.1300)},{\sy*(0.3750)})
	--({\sx*(2.1400)},{\sy*(0.2614)})
	--({\sx*(2.1500)},{\sy*(0.3182)})
	--({\sx*(2.1600)},{\sy*(0.3182)})
	--({\sx*(2.1700)},{\sy*(0.2386)})
	--({\sx*(2.1800)},{\sy*(0.1250)})
	--({\sx*(2.1900)},{\sy*(0.1932)})
	--({\sx*(2.2000)},{\sy*(0.1818)})
	--({\sx*(2.2100)},{\sy*(0.1250)})
	--({\sx*(2.2200)},{\sy*(0.0795)})
	--({\sx*(2.2300)},{\sy*(0.0682)})
	--({\sx*(2.2400)},{\sy*(-0.0341)})
	--({\sx*(2.2500)},{\sy*(-0.0455)})
	--({\sx*(2.2600)},{\sy*(-0.0568)})
	--({\sx*(2.2700)},{\sy*(-0.1023)})
	--({\sx*(2.2800)},{\sy*(-0.1875)})
	--({\sx*(2.2900)},{\sy*(-0.2273)})
	--({\sx*(2.3000)},{\sy*(-0.2784)})
	--({\sx*(2.3100)},{\sy*(-0.2557)})
	--({\sx*(2.3200)},{\sy*(-0.3011)})
	--({\sx*(2.3300)},{\sy*(-0.3636)})
	--({\sx*(2.3400)},{\sy*(-0.3239)})
	--({\sx*(2.3500)},{\sy*(-0.3636)})
	--({\sx*(2.3600)},{\sy*(-0.3580)})
	--({\sx*(2.3700)},{\sy*(-0.4205)})
	--({\sx*(2.3800)},{\sy*(-0.3580)})
	--({\sx*(2.3900)},{\sy*(-0.3523)})
	--({\sx*(2.4000)},{\sy*(-0.3807)})
	--({\sx*(2.4100)},{\sy*(-0.3523)})
	--({\sx*(2.4200)},{\sy*(-0.2898)})
	--({\sx*(2.4300)},{\sy*(-0.3068)})
	--({\sx*(2.4400)},{\sy*(-0.2670)})
	--({\sx*(2.4500)},{\sy*(-0.1875)})
	--({\sx*(2.4600)},{\sy*(-0.1932)})
	--({\sx*(2.4700)},{\sy*(-0.1534)})
	--({\sx*(2.4800)},{\sy*(-0.0909)})
	--({\sx*(2.4900)},{\sy*(-0.0568)})
	--({\sx*(2.5000)},{\sy*(0.0000)})
	--({\sx*(2.5100)},{\sy*(0.0625)})
	--({\sx*(2.5200)},{\sy*(0.0739)})
	--({\sx*(2.5300)},{\sy*(0.1193)})
	--({\sx*(2.5400)},{\sy*(0.1989)})
	--({\sx*(2.5500)},{\sy*(0.2244)})
	--({\sx*(2.5600)},{\sy*(0.2869)})
	--({\sx*(2.5700)},{\sy*(0.2614)})
	--({\sx*(2.5800)},{\sy*(0.3438)})
	--({\sx*(2.5900)},{\sy*(0.3608)})
	--({\sx*(2.6000)},{\sy*(0.3693)})
	--({\sx*(2.6100)},{\sy*(0.4034)})
	--({\sx*(2.6200)},{\sy*(0.3977)})
	--({\sx*(2.6300)},{\sy*(0.3949)})
	--({\sx*(2.6400)},{\sy*(0.3693)})
	--({\sx*(2.6500)},{\sy*(0.3693)})
	--({\sx*(2.6600)},{\sy*(0.3864)})
	--({\sx*(2.6700)},{\sy*(0.3267)})
	--({\sx*(2.6800)},{\sy*(0.3295)})
	--({\sx*(2.6900)},{\sy*(0.2869)})
	--({\sx*(2.7000)},{\sy*(0.2983)})
	--({\sx*(2.7100)},{\sy*(0.2216)})
	--({\sx*(2.7200)},{\sy*(0.2074)})
	--({\sx*(2.7300)},{\sy*(0.1534)})
	--({\sx*(2.7400)},{\sy*(0.0994)})
	--({\sx*(2.7500)},{\sy*(0.0455)})
	--({\sx*(2.7600)},{\sy*(0.0085)})
	--({\sx*(2.7700)},{\sy*(-0.0312)})
	--({\sx*(2.7800)},{\sy*(-0.0881)})
	--({\sx*(2.7900)},{\sy*(-0.1534)})
	--({\sx*(2.8000)},{\sy*(-0.1960)})
	--({\sx*(2.8100)},{\sy*(-0.2230)})
	--({\sx*(2.8200)},{\sy*(-0.2599)})
	--({\sx*(2.8300)},{\sy*(-0.3196)})
	--({\sx*(2.8400)},{\sy*(-0.3324)})
	--({\sx*(2.8500)},{\sy*(-0.3352)})
	--({\sx*(2.8600)},{\sy*(-0.3963)})
	--({\sx*(2.8700)},{\sy*(-0.3679)})
	--({\sx*(2.8800)},{\sy*(-0.4276)})
	--({\sx*(2.8900)},{\sy*(-0.3778)})
	--({\sx*(2.9000)},{\sy*(-0.4105)})
	--({\sx*(2.9100)},{\sy*(-0.4034)})
	--({\sx*(2.9200)},{\sy*(-0.3821)})
	--({\sx*(2.9300)},{\sy*(-0.3636)})
	--({\sx*(2.9400)},{\sy*(-0.3409)})
	--({\sx*(2.9500)},{\sy*(-0.2827)})
	--({\sx*(2.9600)},{\sy*(-0.3125)})
	--({\sx*(2.9700)},{\sy*(-0.2188)})
	--({\sx*(2.9800)},{\sy*(-0.2145)})
	--({\sx*(2.9900)},{\sy*(-0.1591)})
	--({\sx*(3.0000)},{\sy*(-0.0994)})
	--({\sx*(3.0100)},{\sy*(-0.0526)})
	--({\sx*(3.0200)},{\sy*(0.0028)})
	--({\sx*(3.0300)},{\sy*(0.0611)})
	--({\sx*(3.0400)},{\sy*(0.0966)})
	--({\sx*(3.0500)},{\sy*(0.1491)})
	--({\sx*(3.0600)},{\sy*(0.1996)})
	--({\sx*(3.0700)},{\sy*(0.2486)})
	--({\sx*(3.0800)},{\sy*(0.2919)})
	--({\sx*(3.0900)},{\sy*(0.3161)})
	--({\sx*(3.1000)},{\sy*(0.3480)})
	--({\sx*(3.1100)},{\sy*(0.3445)})
	--({\sx*(3.1200)},{\sy*(0.3828)})
	--({\sx*(3.1300)},{\sy*(0.4126)})
	--({\sx*(3.1400)},{\sy*(0.4077)})
	--({\sx*(3.1500)},{\sy*(0.4460)})
	--({\sx*(3.1600)},{\sy*(0.4119)})
	--({\sx*(3.1700)},{\sy*(0.3991)})
	--({\sx*(3.1800)},{\sy*(0.3565)})
	--({\sx*(3.1900)},{\sy*(0.3786)})
	--({\sx*(3.2000)},{\sy*(0.3089)})
	--({\sx*(3.2100)},{\sy*(0.2912)})
	--({\sx*(3.2200)},{\sy*(0.2585)})
	--({\sx*(3.2300)},{\sy*(0.2060)})
	--({\sx*(3.2400)},{\sy*(0.1655)})
	--({\sx*(3.2500)},{\sy*(0.1186)})
	--({\sx*(3.2600)},{\sy*(0.0561)})
	--({\sx*(3.2700)},{\sy*(0.0110)})
	--({\sx*(3.2800)},{\sy*(-0.0362)})
	--({\sx*(3.2900)},{\sy*(-0.0891)})
	--({\sx*(3.3000)},{\sy*(-0.1449)})
	--({\sx*(3.3100)},{\sy*(-0.1907)})
	--({\sx*(3.3200)},{\sy*(-0.2589)})
	--({\sx*(3.3300)},{\sy*(-0.2507)})
	--({\sx*(3.3400)},{\sy*(-0.2944)})
	--({\sx*(3.3500)},{\sy*(-0.3654)})
	--({\sx*(3.3600)},{\sy*(-0.3977)})
	--({\sx*(3.3700)},{\sy*(-0.4233)})
	--({\sx*(3.3800)},{\sy*(-0.3963)})
	--({\sx*(3.3900)},{\sy*(-0.4162)})
	--({\sx*(3.4000)},{\sy*(-0.4148)})
	--({\sx*(3.4100)},{\sy*(-0.3746)})
	--({\sx*(3.4200)},{\sy*(-0.4116)})
	--({\sx*(3.4300)},{\sy*(-0.3555)})
	--({\sx*(3.4400)},{\sy*(-0.3288)})
	--({\sx*(3.4500)},{\sy*(-0.2997)})
	--({\sx*(3.4600)},{\sy*(-0.2876)})
	--({\sx*(3.4700)},{\sy*(-0.2372)})
	--({\sx*(3.4800)},{\sy*(-0.1914)})
	--({\sx*(3.4900)},{\sy*(-0.1468)})
	--({\sx*(3.5000)},{\sy*(-0.0902)})
	--({\sx*(3.5100)},{\sy*(-0.0359)})
	--({\sx*(3.5200)},{\sy*(0.0165)})
	--({\sx*(3.5300)},{\sy*(0.0700)})
	--({\sx*(3.5400)},{\sy*(0.1319)})
	--({\sx*(3.5500)},{\sy*(0.1648)})
	--({\sx*(3.5600)},{\sy*(0.2189)})
	--({\sx*(3.5700)},{\sy*(0.2587)})
	--({\sx*(3.5800)},{\sy*(0.3139)})
	--({\sx*(3.5900)},{\sy*(0.3212)})
	--({\sx*(3.6000)},{\sy*(0.3816)})
	--({\sx*(3.6100)},{\sy*(0.3828)})
	--({\sx*(3.6200)},{\sy*(0.3796)})
	--({\sx*(3.6300)},{\sy*(0.4137)})
	--({\sx*(3.6400)},{\sy*(0.3983)})
	--({\sx*(3.6500)},{\sy*(0.3924)})
	--({\sx*(3.6600)},{\sy*(0.3588)})
	--({\sx*(3.6700)},{\sy*(0.3597)})
	--({\sx*(3.6800)},{\sy*(0.3070)})
	--({\sx*(3.6900)},{\sy*(0.3086)})
	--({\sx*(3.7000)},{\sy*(0.2491)})
	--({\sx*(3.7100)},{\sy*(0.1995)})
	--({\sx*(3.7200)},{\sy*(0.1538)})
	--({\sx*(3.7300)},{\sy*(0.1090)})
	--({\sx*(3.7400)},{\sy*(0.0513)})
	--({\sx*(3.7500)},{\sy*(0.0000)})
	--({\sx*(3.7600)},{\sy*(-0.0542)})
	--({\sx*(3.7700)},{\sy*(-0.1063)})
	--({\sx*(3.7800)},{\sy*(-0.1635)})
	--({\sx*(3.7900)},{\sy*(-0.2021)})
	--({\sx*(3.8000)},{\sy*(-0.2369)})
	--({\sx*(3.8100)},{\sy*(-0.2759)})
	--({\sx*(3.8200)},{\sy*(-0.3242)})
	--({\sx*(3.8300)},{\sy*(-0.3454)})
	--({\sx*(3.8400)},{\sy*(-0.3454)})
	--({\sx*(3.8500)},{\sy*(-0.3589)})
	--({\sx*(3.8600)},{\sy*(-0.3854)})
	--({\sx*(3.8700)},{\sy*(-0.3686)})
	--({\sx*(3.8800)},{\sy*(-0.3584)})
	--({\sx*(3.8900)},{\sy*(-0.3324)})
	--({\sx*(3.9000)},{\sy*(-0.3213)})
	--({\sx*(3.9100)},{\sy*(-0.2914)})
	--({\sx*(3.9200)},{\sy*(-0.2418)})
	--({\sx*(3.9300)},{\sy*(-0.2041)})
	--({\sx*(3.9400)},{\sy*(-0.1602)})
	--({\sx*(3.9500)},{\sy*(-0.1079)})
	--({\sx*(3.9600)},{\sy*(-0.0477)})
	--({\sx*(3.9700)},{\sy*(0.0032)})
	--({\sx*(3.9800)},{\sy*(0.0557)})
	--({\sx*(3.9900)},{\sy*(0.1189)})
	--({\sx*(4.0000)},{\sy*(0.1585)})
	--({\sx*(4.0100)},{\sy*(0.2018)})
	--({\sx*(4.0200)},{\sy*(0.2293)})
	--({\sx*(4.0300)},{\sy*(0.2795)})
	--({\sx*(4.0400)},{\sy*(0.3247)})
	--({\sx*(4.0500)},{\sy*(0.3325)})
	--({\sx*(4.0600)},{\sy*(0.3558)})
	--({\sx*(4.0700)},{\sy*(0.3485)})
	--({\sx*(4.0800)},{\sy*(0.3420)})
	--({\sx*(4.0900)},{\sy*(0.3404)})
	--({\sx*(4.1000)},{\sy*(0.3086)})
	--({\sx*(4.1100)},{\sy*(0.2968)})
	--({\sx*(4.1200)},{\sy*(0.2536)})
	--({\sx*(4.1300)},{\sy*(0.2302)})
	--({\sx*(4.1400)},{\sy*(0.1691)})
	--({\sx*(4.1500)},{\sy*(0.1229)})
	--({\sx*(4.1600)},{\sy*(0.0714)})
	--({\sx*(4.1700)},{\sy*(0.0158)})
	--({\sx*(4.1800)},{\sy*(-0.0396)})
	--({\sx*(4.1900)},{\sy*(-0.0934)})
	--({\sx*(4.2000)},{\sy*(-0.1354)})
	--({\sx*(4.2100)},{\sy*(-0.1906)})
	--({\sx*(4.2200)},{\sy*(-0.2335)})
	--({\sx*(4.2300)},{\sy*(-0.2667)})
	--({\sx*(4.2400)},{\sy*(-0.2913)})
	--({\sx*(4.2500)},{\sy*(-0.3149)})
	--({\sx*(4.2600)},{\sy*(-0.3107)})
	--({\sx*(4.2700)},{\sy*(-0.3259)})
	--({\sx*(4.2800)},{\sy*(-0.3321)})
	--({\sx*(4.2900)},{\sy*(-0.2993)})
	--({\sx*(4.3000)},{\sy*(-0.2683)})
	--({\sx*(4.3100)},{\sy*(-0.2278)})
	--({\sx*(4.3200)},{\sy*(-0.1959)})
	--({\sx*(4.3300)},{\sy*(-0.1575)})
	--({\sx*(4.3400)},{\sy*(-0.0957)})
	--({\sx*(4.3500)},{\sy*(-0.0449)})
	--({\sx*(4.3600)},{\sy*(0.0111)})
	--({\sx*(4.3700)},{\sy*(0.0646)})
	--({\sx*(4.3800)},{\sy*(0.1178)})
	--({\sx*(4.3900)},{\sy*(0.1658)})
	--({\sx*(4.4000)},{\sy*(0.2112)})
	--({\sx*(4.4100)},{\sy*(0.2398)})
	--({\sx*(4.4200)},{\sy*(0.2687)})
	--({\sx*(4.4300)},{\sy*(0.2854)})
	--({\sx*(4.4400)},{\sy*(0.2905)})
	--({\sx*(4.4500)},{\sy*(0.2760)})
	--({\sx*(4.4600)},{\sy*(0.2857)})
	--({\sx*(4.4700)},{\sy*(0.2424)})
	--({\sx*(4.4800)},{\sy*(0.2014)})
	--({\sx*(4.4900)},{\sy*(0.1677)})
	--({\sx*(4.5000)},{\sy*(0.1210)})
	--({\sx*(4.5100)},{\sy*(0.0679)})
	--({\sx*(4.5200)},{\sy*(0.0139)})
	--({\sx*(4.5300)},{\sy*(-0.0408)})
	--({\sx*(4.5400)},{\sy*(-0.0905)})
	--({\sx*(4.5500)},{\sy*(-0.1418)})
	--({\sx*(4.5600)},{\sy*(-0.1818)})
	--({\sx*(4.5700)},{\sy*(-0.2174)})
	--({\sx*(4.5800)},{\sy*(-0.2452)})
	--({\sx*(4.5900)},{\sy*(-0.2604)})
	--({\sx*(4.6000)},{\sy*(-0.2435)})
	--({\sx*(4.6100)},{\sy*(-0.2427)})
	--({\sx*(4.6200)},{\sy*(-0.2061)})
	--({\sx*(4.6300)},{\sy*(-0.1751)})
	--({\sx*(4.6400)},{\sy*(-0.1336)})
	--({\sx*(4.6500)},{\sy*(-0.0820)})
	--({\sx*(4.6600)},{\sy*(-0.0279)})
	--({\sx*(4.6700)},{\sy*(0.0278)})
	--({\sx*(4.6800)},{\sy*(0.0807)})
	--({\sx*(4.6900)},{\sy*(0.1245)})
	--({\sx*(4.7000)},{\sy*(0.1692)})
	--({\sx*(4.7100)},{\sy*(0.1891)})
	--({\sx*(4.7200)},{\sy*(0.2038)})
	--({\sx*(4.7300)},{\sy*(0.2091)})
	--({\sx*(4.7400)},{\sy*(0.1929)})
	--({\sx*(4.7500)},{\sy*(0.1638)})
	--({\sx*(4.7600)},{\sy*(0.1217)})
	--({\sx*(4.7700)},{\sy*(0.0788)})
	--({\sx*(4.7800)},{\sy*(0.0209)})
	--({\sx*(4.7900)},{\sy*(-0.0332)})
	--({\sx*(4.8000)},{\sy*(-0.0837)})
	--({\sx*(4.8100)},{\sy*(-0.1233)})
	--({\sx*(4.8200)},{\sy*(-0.1485)})
	--({\sx*(4.8300)},{\sy*(-0.1618)})
	--({\sx*(4.8400)},{\sy*(-0.1627)})
	--({\sx*(4.8500)},{\sy*(-0.1306)})
	--({\sx*(4.8600)},{\sy*(-0.0928)})
	--({\sx*(4.8700)},{\sy*(-0.0446)})
	--({\sx*(4.8800)},{\sy*(0.0129)})
	--({\sx*(4.8900)},{\sy*(0.0637)})
	--({\sx*(4.9000)},{\sy*(0.1008)})
	--({\sx*(4.9100)},{\sy*(0.1174)})
	--({\sx*(4.9200)},{\sy*(0.1109)})
	--({\sx*(4.9300)},{\sy*(0.0816)})
	--({\sx*(4.9400)},{\sy*(0.0301)})
	--({\sx*(4.9500)},{\sy*(-0.0237)})
	--({\sx*(4.9600)},{\sy*(-0.0651)})
	--({\sx*(4.9700)},{\sy*(-0.0691)})
	--({\sx*(4.9800)},{\sy*(-0.0341)})
	--({\sx*(4.9900)},{\sy*(0.0174)})
	--({\sx*(5.0000)},{\sy*(0.0000)});
}
\def\relfehlero{
\draw[color=blue,line width=1.4pt,line join=round] ({\sx*(0.000)},{\sy*(0.0000)})
	--({\sx*(0.0100)},{\sy*(0.0000)})
	--({\sx*(0.0200)},{\sy*(0.0000)})
	--({\sx*(0.0300)},{\sy*(0.0000)})
	--({\sx*(0.0400)},{\sy*(0.0000)})
	--({\sx*(0.0500)},{\sy*(0.0000)})
	--({\sx*(0.0600)},{\sy*(0.0000)})
	--({\sx*(0.0700)},{\sy*(-0.0000)})
	--({\sx*(0.0800)},{\sy*(-0.0000)})
	--({\sx*(0.0900)},{\sy*(0.0000)})
	--({\sx*(0.1000)},{\sy*(-0.0000)})
	--({\sx*(0.1100)},{\sy*(-0.0000)})
	--({\sx*(0.1200)},{\sy*(-0.0000)})
	--({\sx*(0.1300)},{\sy*(0.0000)})
	--({\sx*(0.1400)},{\sy*(-0.0000)})
	--({\sx*(0.1500)},{\sy*(-0.0000)})
	--({\sx*(0.1600)},{\sy*(-0.0000)})
	--({\sx*(0.1700)},{\sy*(-0.0000)})
	--({\sx*(0.1800)},{\sy*(-0.0000)})
	--({\sx*(0.1900)},{\sy*(-0.0000)})
	--({\sx*(0.2000)},{\sy*(-0.0000)})
	--({\sx*(0.2100)},{\sy*(0.0000)})
	--({\sx*(0.2200)},{\sy*(-0.0000)})
	--({\sx*(0.2300)},{\sy*(-0.0000)})
	--({\sx*(0.2400)},{\sy*(-0.0000)})
	--({\sx*(0.2500)},{\sy*(-0.0000)})
	--({\sx*(0.2600)},{\sy*(-0.0000)})
	--({\sx*(0.2700)},{\sy*(-0.0000)})
	--({\sx*(0.2800)},{\sy*(-0.0000)})
	--({\sx*(0.2900)},{\sy*(-0.0000)})
	--({\sx*(0.3000)},{\sy*(0.0000)})
	--({\sx*(0.3100)},{\sy*(-0.0000)})
	--({\sx*(0.3200)},{\sy*(0.0000)})
	--({\sx*(0.3300)},{\sy*(-0.0000)})
	--({\sx*(0.3400)},{\sy*(0.0000)})
	--({\sx*(0.3500)},{\sy*(0.0000)})
	--({\sx*(0.3600)},{\sy*(0.0000)})
	--({\sx*(0.3700)},{\sy*(0.0000)})
	--({\sx*(0.3800)},{\sy*(0.0000)})
	--({\sx*(0.3900)},{\sy*(0.0000)})
	--({\sx*(0.4000)},{\sy*(0.0000)})
	--({\sx*(0.4100)},{\sy*(0.0000)})
	--({\sx*(0.4200)},{\sy*(0.0000)})
	--({\sx*(0.4300)},{\sy*(0.0000)})
	--({\sx*(0.4400)},{\sy*(0.0000)})
	--({\sx*(0.4500)},{\sy*(0.0000)})
	--({\sx*(0.4600)},{\sy*(0.0000)})
	--({\sx*(0.4700)},{\sy*(0.0000)})
	--({\sx*(0.4800)},{\sy*(-0.0000)})
	--({\sx*(0.4900)},{\sy*(-0.0000)})
	--({\sx*(0.5000)},{\sy*(0.0000)})
	--({\sx*(0.5100)},{\sy*(0.0000)})
	--({\sx*(0.5200)},{\sy*(0.0000)})
	--({\sx*(0.5300)},{\sy*(-0.0000)})
	--({\sx*(0.5400)},{\sy*(-0.0000)})
	--({\sx*(0.5500)},{\sy*(0.0000)})
	--({\sx*(0.5600)},{\sy*(-0.0000)})
	--({\sx*(0.5700)},{\sy*(-0.0000)})
	--({\sx*(0.5800)},{\sy*(-0.0000)})
	--({\sx*(0.5900)},{\sy*(-0.0000)})
	--({\sx*(0.6000)},{\sy*(0.0000)})
	--({\sx*(0.6100)},{\sy*(-0.0000)})
	--({\sx*(0.6200)},{\sy*(-0.0000)})
	--({\sx*(0.6300)},{\sy*(-0.0000)})
	--({\sx*(0.6400)},{\sy*(-0.0000)})
	--({\sx*(0.6500)},{\sy*(-0.0000)})
	--({\sx*(0.6600)},{\sy*(-0.0000)})
	--({\sx*(0.6700)},{\sy*(0.0000)})
	--({\sx*(0.6800)},{\sy*(-0.0000)})
	--({\sx*(0.6900)},{\sy*(0.0000)})
	--({\sx*(0.7000)},{\sy*(-0.0000)})
	--({\sx*(0.7100)},{\sy*(0.0000)})
	--({\sx*(0.7200)},{\sy*(0.0000)})
	--({\sx*(0.7300)},{\sy*(0.0000)})
	--({\sx*(0.7400)},{\sy*(0.0000)})
	--({\sx*(0.7500)},{\sy*(-0.0000)})
	--({\sx*(0.7600)},{\sy*(0.0000)})
	--({\sx*(0.7700)},{\sy*(0.0000)})
	--({\sx*(0.7800)},{\sy*(-0.0000)})
	--({\sx*(0.7900)},{\sy*(0.0000)})
	--({\sx*(0.8000)},{\sy*(0.0000)})
	--({\sx*(0.8100)},{\sy*(0.0000)})
	--({\sx*(0.8200)},{\sy*(0.0000)})
	--({\sx*(0.8300)},{\sy*(0.0000)})
	--({\sx*(0.8400)},{\sy*(0.0000)})
	--({\sx*(0.8500)},{\sy*(-0.0000)})
	--({\sx*(0.8600)},{\sy*(-0.0000)})
	--({\sx*(0.8700)},{\sy*(-0.0000)})
	--({\sx*(0.8800)},{\sy*(0.0000)})
	--({\sx*(0.8900)},{\sy*(0.0000)})
	--({\sx*(0.9000)},{\sy*(0.0000)})
	--({\sx*(0.9100)},{\sy*(-0.0000)})
	--({\sx*(0.9200)},{\sy*(0.0000)})
	--({\sx*(0.9300)},{\sy*(-0.0000)})
	--({\sx*(0.9400)},{\sy*(0.0000)})
	--({\sx*(0.9500)},{\sy*(-0.0000)})
	--({\sx*(0.9600)},{\sy*(0.0000)})
	--({\sx*(0.9700)},{\sy*(0.0000)})
	--({\sx*(0.9800)},{\sy*(-0.0000)})
	--({\sx*(0.9900)},{\sy*(-0.0000)})
	--({\sx*(1.0000)},{\sy*(-0.0000)})
	--({\sx*(1.0100)},{\sy*(0.0000)})
	--({\sx*(1.0200)},{\sy*(-0.0000)})
	--({\sx*(1.0300)},{\sy*(-0.0000)})
	--({\sx*(1.0400)},{\sy*(0.0000)})
	--({\sx*(1.0500)},{\sy*(0.0000)})
	--({\sx*(1.0600)},{\sy*(-0.0000)})
	--({\sx*(1.0700)},{\sy*(0.0000)})
	--({\sx*(1.0800)},{\sy*(0.0000)})
	--({\sx*(1.0900)},{\sy*(0.0000)})
	--({\sx*(1.1000)},{\sy*(0.0000)})
	--({\sx*(1.1100)},{\sy*(0.0000)})
	--({\sx*(1.1200)},{\sy*(0.0000)})
	--({\sx*(1.1300)},{\sy*(0.0000)})
	--({\sx*(1.1400)},{\sy*(0.0000)})
	--({\sx*(1.1500)},{\sy*(-0.0000)})
	--({\sx*(1.1600)},{\sy*(0.0000)})
	--({\sx*(1.1700)},{\sy*(0.0000)})
	--({\sx*(1.1800)},{\sy*(0.0000)})
	--({\sx*(1.1900)},{\sy*(0.0000)})
	--({\sx*(1.2000)},{\sy*(-0.0000)})
	--({\sx*(1.2100)},{\sy*(0.0000)})
	--({\sx*(1.2200)},{\sy*(0.0000)})
	--({\sx*(1.2300)},{\sy*(0.0000)})
	--({\sx*(1.2400)},{\sy*(-0.0000)})
	--({\sx*(1.2500)},{\sy*(0.0000)})
	--({\sx*(1.2600)},{\sy*(0.0000)})
	--({\sx*(1.2700)},{\sy*(-0.0000)})
	--({\sx*(1.2800)},{\sy*(-0.0000)})
	--({\sx*(1.2900)},{\sy*(-0.0000)})
	--({\sx*(1.3000)},{\sy*(0.0000)})
	--({\sx*(1.3100)},{\sy*(-0.0000)})
	--({\sx*(1.3200)},{\sy*(-0.0000)})
	--({\sx*(1.3300)},{\sy*(-0.0000)})
	--({\sx*(1.3400)},{\sy*(-0.0000)})
	--({\sx*(1.3500)},{\sy*(-0.0000)})
	--({\sx*(1.3600)},{\sy*(-0.0000)})
	--({\sx*(1.3700)},{\sy*(-0.0000)})
	--({\sx*(1.3800)},{\sy*(-0.0000)})
	--({\sx*(1.3900)},{\sy*(-0.0000)})
	--({\sx*(1.4000)},{\sy*(-0.0000)})
	--({\sx*(1.4100)},{\sy*(-0.0000)})
	--({\sx*(1.4200)},{\sy*(-0.0000)})
	--({\sx*(1.4300)},{\sy*(-0.0000)})
	--({\sx*(1.4400)},{\sy*(-0.0000)})
	--({\sx*(1.4500)},{\sy*(-0.0000)})
	--({\sx*(1.4600)},{\sy*(-0.0000)})
	--({\sx*(1.4700)},{\sy*(0.0000)})
	--({\sx*(1.4800)},{\sy*(-0.0000)})
	--({\sx*(1.4900)},{\sy*(-0.0000)})
	--({\sx*(1.5000)},{\sy*(0.0000)})
	--({\sx*(1.5100)},{\sy*(0.0000)})
	--({\sx*(1.5200)},{\sy*(0.0000)})
	--({\sx*(1.5300)},{\sy*(-0.0000)})
	--({\sx*(1.5400)},{\sy*(0.0000)})
	--({\sx*(1.5500)},{\sy*(0.0000)})
	--({\sx*(1.5600)},{\sy*(0.0000)})
	--({\sx*(1.5700)},{\sy*(0.0000)})
	--({\sx*(1.5800)},{\sy*(0.0000)})
	--({\sx*(1.5900)},{\sy*(0.0000)})
	--({\sx*(1.6000)},{\sy*(0.0000)})
	--({\sx*(1.6100)},{\sy*(0.0000)})
	--({\sx*(1.6200)},{\sy*(0.0000)})
	--({\sx*(1.6300)},{\sy*(0.0000)})
	--({\sx*(1.6400)},{\sy*(0.0000)})
	--({\sx*(1.6500)},{\sy*(0.0000)})
	--({\sx*(1.6600)},{\sy*(0.0000)})
	--({\sx*(1.6700)},{\sy*(0.0000)})
	--({\sx*(1.6800)},{\sy*(0.0000)})
	--({\sx*(1.6900)},{\sy*(0.0000)})
	--({\sx*(1.7000)},{\sy*(0.0000)})
	--({\sx*(1.7100)},{\sy*(0.0000)})
	--({\sx*(1.7200)},{\sy*(0.0000)})
	--({\sx*(1.7300)},{\sy*(0.0000)})
	--({\sx*(1.7400)},{\sy*(-0.0000)})
	--({\sx*(1.7500)},{\sy*(-0.0000)})
	--({\sx*(1.7600)},{\sy*(0.0000)})
	--({\sx*(1.7700)},{\sy*(0.0000)})
	--({\sx*(1.7800)},{\sy*(-0.0000)})
	--({\sx*(1.7900)},{\sy*(-0.0000)})
	--({\sx*(1.8000)},{\sy*(-0.0000)})
	--({\sx*(1.8100)},{\sy*(-0.0000)})
	--({\sx*(1.8200)},{\sy*(-0.0000)})
	--({\sx*(1.8300)},{\sy*(-0.0000)})
	--({\sx*(1.8400)},{\sy*(-0.0000)})
	--({\sx*(1.8500)},{\sy*(-0.0000)})
	--({\sx*(1.8600)},{\sy*(-0.0000)})
	--({\sx*(1.8700)},{\sy*(-0.0000)})
	--({\sx*(1.8800)},{\sy*(-0.0000)})
	--({\sx*(1.8900)},{\sy*(-0.0000)})
	--({\sx*(1.9000)},{\sy*(-0.0000)})
	--({\sx*(1.9100)},{\sy*(-0.0000)})
	--({\sx*(1.9200)},{\sy*(-0.0000)})
	--({\sx*(1.9300)},{\sy*(-0.0000)})
	--({\sx*(1.9400)},{\sy*(-0.0000)})
	--({\sx*(1.9500)},{\sy*(-0.0000)})
	--({\sx*(1.9600)},{\sy*(-0.0000)})
	--({\sx*(1.9700)},{\sy*(0.0000)})
	--({\sx*(1.9800)},{\sy*(-0.0000)})
	--({\sx*(1.9900)},{\sy*(0.0000)})
	--({\sx*(2.0000)},{\sy*(0.0000)})
	--({\sx*(2.0100)},{\sy*(0.0000)})
	--({\sx*(2.0200)},{\sy*(0.0000)})
	--({\sx*(2.0300)},{\sy*(0.0000)})
	--({\sx*(2.0400)},{\sy*(0.0000)})
	--({\sx*(2.0500)},{\sy*(0.0000)})
	--({\sx*(2.0600)},{\sy*(0.0000)})
	--({\sx*(2.0700)},{\sy*(0.0000)})
	--({\sx*(2.0800)},{\sy*(0.0000)})
	--({\sx*(2.0900)},{\sy*(0.0000)})
	--({\sx*(2.1000)},{\sy*(0.0000)})
	--({\sx*(2.1100)},{\sy*(0.0000)})
	--({\sx*(2.1200)},{\sy*(0.0000)})
	--({\sx*(2.1300)},{\sy*(0.0000)})
	--({\sx*(2.1400)},{\sy*(0.0000)})
	--({\sx*(2.1500)},{\sy*(0.0000)})
	--({\sx*(2.1600)},{\sy*(0.0000)})
	--({\sx*(2.1700)},{\sy*(0.0000)})
	--({\sx*(2.1800)},{\sy*(0.0000)})
	--({\sx*(2.1900)},{\sy*(0.0000)})
	--({\sx*(2.2000)},{\sy*(0.0000)})
	--({\sx*(2.2100)},{\sy*(0.0000)})
	--({\sx*(2.2200)},{\sy*(0.0000)})
	--({\sx*(2.2300)},{\sy*(0.0000)})
	--({\sx*(2.2400)},{\sy*(-0.0000)})
	--({\sx*(2.2500)},{\sy*(-0.0000)})
	--({\sx*(2.2600)},{\sy*(-0.0000)})
	--({\sx*(2.2700)},{\sy*(-0.0000)})
	--({\sx*(2.2800)},{\sy*(-0.0000)})
	--({\sx*(2.2900)},{\sy*(-0.0000)})
	--({\sx*(2.3000)},{\sy*(-0.0000)})
	--({\sx*(2.3100)},{\sy*(-0.0000)})
	--({\sx*(2.3200)},{\sy*(-0.0000)})
	--({\sx*(2.3300)},{\sy*(-0.0000)})
	--({\sx*(2.3400)},{\sy*(-0.0000)})
	--({\sx*(2.3500)},{\sy*(-0.0000)})
	--({\sx*(2.3600)},{\sy*(-0.0000)})
	--({\sx*(2.3700)},{\sy*(-0.0000)})
	--({\sx*(2.3800)},{\sy*(-0.0000)})
	--({\sx*(2.3900)},{\sy*(-0.0000)})
	--({\sx*(2.4000)},{\sy*(-0.0000)})
	--({\sx*(2.4100)},{\sy*(-0.0000)})
	--({\sx*(2.4200)},{\sy*(-0.0000)})
	--({\sx*(2.4300)},{\sy*(-0.0000)})
	--({\sx*(2.4400)},{\sy*(-0.0000)})
	--({\sx*(2.4500)},{\sy*(-0.0000)})
	--({\sx*(2.4600)},{\sy*(-0.0000)})
	--({\sx*(2.4700)},{\sy*(-0.0000)})
	--({\sx*(2.4800)},{\sy*(-0.0000)})
	--({\sx*(2.4900)},{\sy*(-0.0000)})
	--({\sx*(2.5000)},{\sy*(0.0000)})
	--({\sx*(2.5100)},{\sy*(0.0000)})
	--({\sx*(2.5200)},{\sy*(0.0000)})
	--({\sx*(2.5300)},{\sy*(0.0000)})
	--({\sx*(2.5400)},{\sy*(0.0000)})
	--({\sx*(2.5500)},{\sy*(0.0000)})
	--({\sx*(2.5600)},{\sy*(0.0000)})
	--({\sx*(2.5700)},{\sy*(0.0000)})
	--({\sx*(2.5800)},{\sy*(0.0000)})
	--({\sx*(2.5900)},{\sy*(0.0000)})
	--({\sx*(2.6000)},{\sy*(0.0000)})
	--({\sx*(2.6100)},{\sy*(0.0000)})
	--({\sx*(2.6200)},{\sy*(0.0000)})
	--({\sx*(2.6300)},{\sy*(0.0000)})
	--({\sx*(2.6400)},{\sy*(0.0000)})
	--({\sx*(2.6500)},{\sy*(0.0000)})
	--({\sx*(2.6600)},{\sy*(0.0000)})
	--({\sx*(2.6700)},{\sy*(0.0000)})
	--({\sx*(2.6800)},{\sy*(0.0000)})
	--({\sx*(2.6900)},{\sy*(0.0000)})
	--({\sx*(2.7000)},{\sy*(0.0000)})
	--({\sx*(2.7100)},{\sy*(0.0000)})
	--({\sx*(2.7200)},{\sy*(0.0000)})
	--({\sx*(2.7300)},{\sy*(0.0000)})
	--({\sx*(2.7400)},{\sy*(0.0000)})
	--({\sx*(2.7500)},{\sy*(0.0000)})
	--({\sx*(2.7600)},{\sy*(0.0000)})
	--({\sx*(2.7700)},{\sy*(-0.0000)})
	--({\sx*(2.7800)},{\sy*(-0.0000)})
	--({\sx*(2.7900)},{\sy*(-0.0000)})
	--({\sx*(2.8000)},{\sy*(-0.0000)})
	--({\sx*(2.8100)},{\sy*(-0.0000)})
	--({\sx*(2.8200)},{\sy*(-0.0000)})
	--({\sx*(2.8300)},{\sy*(-0.0000)})
	--({\sx*(2.8400)},{\sy*(-0.0000)})
	--({\sx*(2.8500)},{\sy*(-0.0000)})
	--({\sx*(2.8600)},{\sy*(-0.0000)})
	--({\sx*(2.8700)},{\sy*(-0.0000)})
	--({\sx*(2.8800)},{\sy*(-0.0000)})
	--({\sx*(2.8900)},{\sy*(-0.0000)})
	--({\sx*(2.9000)},{\sy*(-0.0000)})
	--({\sx*(2.9100)},{\sy*(-0.0000)})
	--({\sx*(2.9200)},{\sy*(-0.0000)})
	--({\sx*(2.9300)},{\sy*(-0.0000)})
	--({\sx*(2.9400)},{\sy*(-0.0000)})
	--({\sx*(2.9500)},{\sy*(-0.0000)})
	--({\sx*(2.9600)},{\sy*(-0.0000)})
	--({\sx*(2.9700)},{\sy*(-0.0000)})
	--({\sx*(2.9800)},{\sy*(-0.0000)})
	--({\sx*(2.9900)},{\sy*(-0.0000)})
	--({\sx*(3.0000)},{\sy*(-0.0000)})
	--({\sx*(3.0100)},{\sy*(-0.0000)})
	--({\sx*(3.0200)},{\sy*(0.0000)})
	--({\sx*(3.0300)},{\sy*(0.0000)})
	--({\sx*(3.0400)},{\sy*(0.0000)})
	--({\sx*(3.0500)},{\sy*(0.0000)})
	--({\sx*(3.0600)},{\sy*(0.0000)})
	--({\sx*(3.0700)},{\sy*(0.0000)})
	--({\sx*(3.0800)},{\sy*(0.0000)})
	--({\sx*(3.0900)},{\sy*(0.0000)})
	--({\sx*(3.1000)},{\sy*(0.0000)})
	--({\sx*(3.1100)},{\sy*(0.0000)})
	--({\sx*(3.1200)},{\sy*(0.0000)})
	--({\sx*(3.1300)},{\sy*(0.0000)})
	--({\sx*(3.1400)},{\sy*(0.0000)})
	--({\sx*(3.1500)},{\sy*(0.0000)})
	--({\sx*(3.1600)},{\sy*(0.0000)})
	--({\sx*(3.1700)},{\sy*(0.0000)})
	--({\sx*(3.1800)},{\sy*(0.0000)})
	--({\sx*(3.1900)},{\sy*(0.0000)})
	--({\sx*(3.2000)},{\sy*(0.0000)})
	--({\sx*(3.2100)},{\sy*(0.0000)})
	--({\sx*(3.2200)},{\sy*(0.0000)})
	--({\sx*(3.2300)},{\sy*(0.0000)})
	--({\sx*(3.2400)},{\sy*(0.0000)})
	--({\sx*(3.2500)},{\sy*(0.0000)})
	--({\sx*(3.2600)},{\sy*(0.0000)})
	--({\sx*(3.2700)},{\sy*(0.0000)})
	--({\sx*(3.2800)},{\sy*(-0.0000)})
	--({\sx*(3.2900)},{\sy*(-0.0000)})
	--({\sx*(3.3000)},{\sy*(-0.0000)})
	--({\sx*(3.3100)},{\sy*(-0.0000)})
	--({\sx*(3.3200)},{\sy*(-0.0000)})
	--({\sx*(3.3300)},{\sy*(-0.0000)})
	--({\sx*(3.3400)},{\sy*(-0.0000)})
	--({\sx*(3.3500)},{\sy*(-0.0000)})
	--({\sx*(3.3600)},{\sy*(-0.0000)})
	--({\sx*(3.3700)},{\sy*(-0.0000)})
	--({\sx*(3.3800)},{\sy*(-0.0000)})
	--({\sx*(3.3900)},{\sy*(-0.0000)})
	--({\sx*(3.4000)},{\sy*(-0.0000)})
	--({\sx*(3.4100)},{\sy*(-0.0000)})
	--({\sx*(3.4200)},{\sy*(-0.0000)})
	--({\sx*(3.4300)},{\sy*(-0.0000)})
	--({\sx*(3.4400)},{\sy*(-0.0000)})
	--({\sx*(3.4500)},{\sy*(-0.0000)})
	--({\sx*(3.4600)},{\sy*(-0.0000)})
	--({\sx*(3.4700)},{\sy*(-0.0000)})
	--({\sx*(3.4800)},{\sy*(-0.0000)})
	--({\sx*(3.4900)},{\sy*(-0.0000)})
	--({\sx*(3.5000)},{\sy*(-0.0000)})
	--({\sx*(3.5100)},{\sy*(-0.0000)})
	--({\sx*(3.5200)},{\sy*(0.0000)})
	--({\sx*(3.5300)},{\sy*(0.0000)})
	--({\sx*(3.5400)},{\sy*(0.0000)})
	--({\sx*(3.5500)},{\sy*(0.0000)})
	--({\sx*(3.5600)},{\sy*(0.0000)})
	--({\sx*(3.5700)},{\sy*(0.0000)})
	--({\sx*(3.5800)},{\sy*(0.0000)})
	--({\sx*(3.5900)},{\sy*(0.0000)})
	--({\sx*(3.6000)},{\sy*(0.0000)})
	--({\sx*(3.6100)},{\sy*(0.0000)})
	--({\sx*(3.6200)},{\sy*(0.0000)})
	--({\sx*(3.6300)},{\sy*(0.0000)})
	--({\sx*(3.6400)},{\sy*(0.0000)})
	--({\sx*(3.6500)},{\sy*(0.0000)})
	--({\sx*(3.6600)},{\sy*(0.0000)})
	--({\sx*(3.6700)},{\sy*(0.0000)})
	--({\sx*(3.6800)},{\sy*(0.0000)})
	--({\sx*(3.6900)},{\sy*(0.0000)})
	--({\sx*(3.7000)},{\sy*(0.0000)})
	--({\sx*(3.7100)},{\sy*(0.0000)})
	--({\sx*(3.7200)},{\sy*(0.0000)})
	--({\sx*(3.7300)},{\sy*(0.0000)})
	--({\sx*(3.7400)},{\sy*(0.0000)})
	--({\sx*(3.7500)},{\sy*(0.0000)})
	--({\sx*(3.7600)},{\sy*(-0.0000)})
	--({\sx*(3.7700)},{\sy*(-0.0000)})
	--({\sx*(3.7800)},{\sy*(-0.0000)})
	--({\sx*(3.7900)},{\sy*(-0.0000)})
	--({\sx*(3.8000)},{\sy*(-0.0000)})
	--({\sx*(3.8100)},{\sy*(-0.0000)})
	--({\sx*(3.8200)},{\sy*(-0.0000)})
	--({\sx*(3.8300)},{\sy*(-0.0000)})
	--({\sx*(3.8400)},{\sy*(-0.0000)})
	--({\sx*(3.8500)},{\sy*(-0.0000)})
	--({\sx*(3.8600)},{\sy*(-0.0000)})
	--({\sx*(3.8700)},{\sy*(-0.0000)})
	--({\sx*(3.8800)},{\sy*(-0.0000)})
	--({\sx*(3.8900)},{\sy*(-0.0000)})
	--({\sx*(3.9000)},{\sy*(-0.0000)})
	--({\sx*(3.9100)},{\sy*(-0.0000)})
	--({\sx*(3.9200)},{\sy*(-0.0000)})
	--({\sx*(3.9300)},{\sy*(-0.0000)})
	--({\sx*(3.9400)},{\sy*(-0.0000)})
	--({\sx*(3.9500)},{\sy*(-0.0000)})
	--({\sx*(3.9600)},{\sy*(-0.0000)})
	--({\sx*(3.9700)},{\sy*(0.0000)})
	--({\sx*(3.9800)},{\sy*(0.0000)})
	--({\sx*(3.9900)},{\sy*(0.0000)})
	--({\sx*(4.0000)},{\sy*(0.0000)})
	--({\sx*(4.0100)},{\sy*(0.0000)})
	--({\sx*(4.0200)},{\sy*(0.0000)})
	--({\sx*(4.0300)},{\sy*(0.0000)})
	--({\sx*(4.0400)},{\sy*(0.0000)})
	--({\sx*(4.0500)},{\sy*(0.0000)})
	--({\sx*(4.0600)},{\sy*(0.0000)})
	--({\sx*(4.0700)},{\sy*(0.0000)})
	--({\sx*(4.0800)},{\sy*(0.0000)})
	--({\sx*(4.0900)},{\sy*(0.0000)})
	--({\sx*(4.1000)},{\sy*(0.0000)})
	--({\sx*(4.1100)},{\sy*(0.0000)})
	--({\sx*(4.1200)},{\sy*(0.0000)})
	--({\sx*(4.1300)},{\sy*(0.0000)})
	--({\sx*(4.1400)},{\sy*(0.0000)})
	--({\sx*(4.1500)},{\sy*(0.0000)})
	--({\sx*(4.1600)},{\sy*(0.0000)})
	--({\sx*(4.1700)},{\sy*(0.0000)})
	--({\sx*(4.1800)},{\sy*(-0.0000)})
	--({\sx*(4.1900)},{\sy*(-0.0000)})
	--({\sx*(4.2000)},{\sy*(-0.0000)})
	--({\sx*(4.2100)},{\sy*(-0.0000)})
	--({\sx*(4.2200)},{\sy*(-0.0000)})
	--({\sx*(4.2300)},{\sy*(-0.0000)})
	--({\sx*(4.2400)},{\sy*(-0.0000)})
	--({\sx*(4.2500)},{\sy*(-0.0000)})
	--({\sx*(4.2600)},{\sy*(-0.0000)})
	--({\sx*(4.2700)},{\sy*(-0.0000)})
	--({\sx*(4.2800)},{\sy*(-0.0000)})
	--({\sx*(4.2900)},{\sy*(-0.0000)})
	--({\sx*(4.3000)},{\sy*(-0.0000)})
	--({\sx*(4.3100)},{\sy*(-0.0000)})
	--({\sx*(4.3200)},{\sy*(-0.0000)})
	--({\sx*(4.3300)},{\sy*(-0.0000)})
	--({\sx*(4.3400)},{\sy*(-0.0000)})
	--({\sx*(4.3500)},{\sy*(-0.0000)})
	--({\sx*(4.3600)},{\sy*(0.0000)})
	--({\sx*(4.3700)},{\sy*(0.0000)})
	--({\sx*(4.3800)},{\sy*(0.0000)})
	--({\sx*(4.3900)},{\sy*(0.0000)})
	--({\sx*(4.4000)},{\sy*(0.0000)})
	--({\sx*(4.4100)},{\sy*(0.0000)})
	--({\sx*(4.4200)},{\sy*(0.0000)})
	--({\sx*(4.4300)},{\sy*(0.0000)})
	--({\sx*(4.4400)},{\sy*(0.0000)})
	--({\sx*(4.4500)},{\sy*(0.0000)})
	--({\sx*(4.4600)},{\sy*(0.0000)})
	--({\sx*(4.4700)},{\sy*(0.0000)})
	--({\sx*(4.4800)},{\sy*(0.0000)})
	--({\sx*(4.4900)},{\sy*(0.0000)})
	--({\sx*(4.5000)},{\sy*(0.0000)})
	--({\sx*(4.5100)},{\sy*(0.0000)})
	--({\sx*(4.5200)},{\sy*(0.0000)})
	--({\sx*(4.5300)},{\sy*(-0.0000)})
	--({\sx*(4.5400)},{\sy*(-0.0000)})
	--({\sx*(4.5500)},{\sy*(-0.0000)})
	--({\sx*(4.5600)},{\sy*(-0.0000)})
	--({\sx*(4.5700)},{\sy*(-0.0000)})
	--({\sx*(4.5800)},{\sy*(-0.0000)})
	--({\sx*(4.5900)},{\sy*(-0.0000)})
	--({\sx*(4.6000)},{\sy*(-0.0000)})
	--({\sx*(4.6100)},{\sy*(-0.0000)})
	--({\sx*(4.6200)},{\sy*(-0.0000)})
	--({\sx*(4.6300)},{\sy*(-0.0000)})
	--({\sx*(4.6400)},{\sy*(-0.0000)})
	--({\sx*(4.6500)},{\sy*(-0.0000)})
	--({\sx*(4.6600)},{\sy*(-0.0000)})
	--({\sx*(4.6700)},{\sy*(0.0000)})
	--({\sx*(4.6800)},{\sy*(0.0000)})
	--({\sx*(4.6900)},{\sy*(0.0000)})
	--({\sx*(4.7000)},{\sy*(0.0000)})
	--({\sx*(4.7100)},{\sy*(0.0000)})
	--({\sx*(4.7200)},{\sy*(0.0000)})
	--({\sx*(4.7300)},{\sy*(0.0000)})
	--({\sx*(4.7400)},{\sy*(0.0000)})
	--({\sx*(4.7500)},{\sy*(0.0000)})
	--({\sx*(4.7600)},{\sy*(0.0000)})
	--({\sx*(4.7700)},{\sy*(0.0000)})
	--({\sx*(4.7800)},{\sy*(0.0000)})
	--({\sx*(4.7900)},{\sy*(-0.0000)})
	--({\sx*(4.8000)},{\sy*(-0.0000)})
	--({\sx*(4.8100)},{\sy*(-0.0000)})
	--({\sx*(4.8200)},{\sy*(-0.0000)})
	--({\sx*(4.8300)},{\sy*(-0.0000)})
	--({\sx*(4.8400)},{\sy*(-0.0000)})
	--({\sx*(4.8500)},{\sy*(-0.0000)})
	--({\sx*(4.8600)},{\sy*(-0.0000)})
	--({\sx*(4.8700)},{\sy*(-0.0000)})
	--({\sx*(4.8800)},{\sy*(0.0000)})
	--({\sx*(4.8900)},{\sy*(0.0000)})
	--({\sx*(4.9000)},{\sy*(0.0000)})
	--({\sx*(4.9100)},{\sy*(0.0000)})
	--({\sx*(4.9200)},{\sy*(0.0000)})
	--({\sx*(4.9300)},{\sy*(0.0000)})
	--({\sx*(4.9400)},{\sy*(0.0000)})
	--({\sx*(4.9500)},{\sy*(-0.0000)})
	--({\sx*(4.9600)},{\sy*(-0.0000)})
	--({\sx*(4.9700)},{\sy*(-0.0000)})
	--({\sx*(4.9800)},{\sy*(-0.0000)})
	--({\sx*(4.9900)},{\sy*(0.0000)})
	--({\sx*(5.0000)},{\sy*(0.0000)});
}
\def\xwertep{
\fill[color=red] (0.0000,0) circle[radius={0.07/\skala}];
\fill[color=white] (0.0000,0) circle[radius={0.05/\skala}];
\fill[color=red] (0.0120,0) circle[radius={0.07/\skala}];
\fill[color=white] (0.0120,0) circle[radius={0.05/\skala}];
\fill[color=red] (0.0480,0) circle[radius={0.07/\skala}];
\fill[color=white] (0.0480,0) circle[radius={0.05/\skala}];
\fill[color=red] (0.1076,0) circle[radius={0.07/\skala}];
\fill[color=white] (0.1076,0) circle[radius={0.05/\skala}];
\fill[color=red] (0.1903,0) circle[radius={0.07/\skala}];
\fill[color=white] (0.1903,0) circle[radius={0.05/\skala}];
\fill[color=red] (0.2952,0) circle[radius={0.07/\skala}];
\fill[color=white] (0.2952,0) circle[radius={0.05/\skala}];
\fill[color=red] (0.4213,0) circle[radius={0.07/\skala}];
\fill[color=white] (0.4213,0) circle[radius={0.05/\skala}];
\fill[color=red] (0.5675,0) circle[radius={0.07/\skala}];
\fill[color=white] (0.5675,0) circle[radius={0.05/\skala}];
\fill[color=red] (0.7322,0) circle[radius={0.07/\skala}];
\fill[color=white] (0.7322,0) circle[radius={0.05/\skala}];
\fill[color=red] (0.9140,0) circle[radius={0.07/\skala}];
\fill[color=white] (0.9140,0) circle[radius={0.05/\skala}];
\fill[color=red] (1.1111,0) circle[radius={0.07/\skala}];
\fill[color=white] (1.1111,0) circle[radius={0.05/\skala}];
\fill[color=red] (1.3215,0) circle[radius={0.07/\skala}];
\fill[color=white] (1.3215,0) circle[radius={0.05/\skala}];
\fill[color=red] (1.5433,0) circle[radius={0.07/\skala}];
\fill[color=white] (1.5433,0) circle[radius={0.05/\skala}];
\fill[color=red] (1.7743,0) circle[radius={0.07/\skala}];
\fill[color=white] (1.7743,0) circle[radius={0.05/\skala}];
\fill[color=red] (2.0123,0) circle[radius={0.07/\skala}];
\fill[color=white] (2.0123,0) circle[radius={0.05/\skala}];
\fill[color=red] (2.2550,0) circle[radius={0.07/\skala}];
\fill[color=white] (2.2550,0) circle[radius={0.05/\skala}];
\fill[color=red] (2.5000,0) circle[radius={0.07/\skala}];
\fill[color=white] (2.5000,0) circle[radius={0.05/\skala}];
\fill[color=red] (2.7450,0) circle[radius={0.07/\skala}];
\fill[color=white] (2.7450,0) circle[radius={0.05/\skala}];
\fill[color=red] (2.9877,0) circle[radius={0.07/\skala}];
\fill[color=white] (2.9877,0) circle[radius={0.05/\skala}];
\fill[color=red] (3.2257,0) circle[radius={0.07/\skala}];
\fill[color=white] (3.2257,0) circle[radius={0.05/\skala}];
\fill[color=red] (3.4567,0) circle[radius={0.07/\skala}];
\fill[color=white] (3.4567,0) circle[radius={0.05/\skala}];
\fill[color=red] (3.6785,0) circle[radius={0.07/\skala}];
\fill[color=white] (3.6785,0) circle[radius={0.05/\skala}];
\fill[color=red] (3.8889,0) circle[radius={0.07/\skala}];
\fill[color=white] (3.8889,0) circle[radius={0.05/\skala}];
\fill[color=red] (4.0860,0) circle[radius={0.07/\skala}];
\fill[color=white] (4.0860,0) circle[radius={0.05/\skala}];
\fill[color=red] (4.2678,0) circle[radius={0.07/\skala}];
\fill[color=white] (4.2678,0) circle[radius={0.05/\skala}];
\fill[color=red] (4.4325,0) circle[radius={0.07/\skala}];
\fill[color=white] (4.4325,0) circle[radius={0.05/\skala}];
\fill[color=red] (4.5787,0) circle[radius={0.07/\skala}];
\fill[color=white] (4.5787,0) circle[radius={0.05/\skala}];
\fill[color=red] (4.7048,0) circle[radius={0.07/\skala}];
\fill[color=white] (4.7048,0) circle[radius={0.05/\skala}];
\fill[color=red] (4.8097,0) circle[radius={0.07/\skala}];
\fill[color=white] (4.8097,0) circle[radius={0.05/\skala}];
\fill[color=red] (4.8924,0) circle[radius={0.07/\skala}];
\fill[color=white] (4.8924,0) circle[radius={0.05/\skala}];
\fill[color=red] (4.9520,0) circle[radius={0.07/\skala}];
\fill[color=white] (4.9520,0) circle[radius={0.05/\skala}];
\fill[color=red] (4.9880,0) circle[radius={0.07/\skala}];
\fill[color=white] (4.9880,0) circle[radius={0.05/\skala}];
\fill[color=red] (5.0000,0) circle[radius={0.07/\skala}];
\fill[color=white] (5.0000,0) circle[radius={0.05/\skala}];
}
\def\punktep{32}
\def\maxfehlerp{6.661\cdot 10^{-16}}
\def\fehlerp{
\draw[color=red,line width=1.4pt,line join=round] ({\sx*(0.000)},{\sy*(0.0000)})
	--({\sx*(0.0100)},{\sy*(0.5000)})
	--({\sx*(0.0200)},{\sy*(0.2500)})
	--({\sx*(0.0300)},{\sy*(-0.8333)})
	--({\sx*(0.0400)},{\sy*(0.0000)})
	--({\sx*(0.0500)},{\sy*(0.8333)})
	--({\sx*(0.0600)},{\sy*(0.3333)})
	--({\sx*(0.0700)},{\sy*(0.2500)})
	--({\sx*(0.0800)},{\sy*(-0.0833)})
	--({\sx*(0.0900)},{\sy*(0.8333)})
	--({\sx*(0.1000)},{\sy*(1.0000)})
	--({\sx*(0.1100)},{\sy*(0.5000)})
	--({\sx*(0.1200)},{\sy*(-0.1667)})
	--({\sx*(0.1300)},{\sy*(0.7500)})
	--({\sx*(0.1400)},{\sy*(0.0000)})
	--({\sx*(0.1500)},{\sy*(0.4167)})
	--({\sx*(0.1600)},{\sy*(-0.3333)})
	--({\sx*(0.1700)},{\sy*(-0.1667)})
	--({\sx*(0.1800)},{\sy*(0.3333)})
	--({\sx*(0.1900)},{\sy*(-0.2500)})
	--({\sx*(0.2000)},{\sy*(-0.5000)})
	--({\sx*(0.2100)},{\sy*(0.0833)})
	--({\sx*(0.2200)},{\sy*(-0.3333)})
	--({\sx*(0.2300)},{\sy*(-0.8333)})
	--({\sx*(0.2400)},{\sy*(-0.6667)})
	--({\sx*(0.2500)},{\sy*(-0.0833)})
	--({\sx*(0.2600)},{\sy*(0.2500)})
	--({\sx*(0.2700)},{\sy*(-0.4167)})
	--({\sx*(0.2800)},{\sy*(-0.6667)})
	--({\sx*(0.2900)},{\sy*(-0.5000)})
	--({\sx*(0.3000)},{\sy*(0.0000)})
	--({\sx*(0.3100)},{\sy*(0.1667)})
	--({\sx*(0.3200)},{\sy*(-0.0833)})
	--({\sx*(0.3300)},{\sy*(-0.2500)})
	--({\sx*(0.3400)},{\sy*(0.5833)})
	--({\sx*(0.3500)},{\sy*(0.8333)})
	--({\sx*(0.3600)},{\sy*(0.0833)})
	--({\sx*(0.3700)},{\sy*(-0.2500)})
	--({\sx*(0.3800)},{\sy*(0.5833)})
	--({\sx*(0.3900)},{\sy*(0.4167)})
	--({\sx*(0.4000)},{\sy*(-0.6667)})
	--({\sx*(0.4100)},{\sy*(-0.1667)})
	--({\sx*(0.4200)},{\sy*(0.1667)})
	--({\sx*(0.4300)},{\sy*(0.6667)})
	--({\sx*(0.4400)},{\sy*(0.1667)})
	--({\sx*(0.4500)},{\sy*(-0.0833)})
	--({\sx*(0.4600)},{\sy*(0.6667)})
	--({\sx*(0.4700)},{\sy*(0.5833)})
	--({\sx*(0.4800)},{\sy*(0.3333)})
	--({\sx*(0.4900)},{\sy*(0.2500)})
	--({\sx*(0.5000)},{\sy*(0.4167)})
	--({\sx*(0.5100)},{\sy*(1.0000)})
	--({\sx*(0.5200)},{\sy*(0.1667)})
	--({\sx*(0.5300)},{\sy*(-0.5000)})
	--({\sx*(0.5400)},{\sy*(0.0000)})
	--({\sx*(0.5500)},{\sy*(0.1667)})
	--({\sx*(0.5600)},{\sy*(-0.1667)})
	--({\sx*(0.5700)},{\sy*(-0.1667)})
	--({\sx*(0.5800)},{\sy*(-0.1667)})
	--({\sx*(0.5900)},{\sy*(0.5000)})
	--({\sx*(0.6000)},{\sy*(-0.5000)})
	--({\sx*(0.6100)},{\sy*(0.1667)})
	--({\sx*(0.6200)},{\sy*(-0.5833)})
	--({\sx*(0.6300)},{\sy*(0.5000)})
	--({\sx*(0.6400)},{\sy*(-0.0833)})
	--({\sx*(0.6500)},{\sy*(0.0000)})
	--({\sx*(0.6600)},{\sy*(-0.4167)})
	--({\sx*(0.6700)},{\sy*(0.7500)})
	--({\sx*(0.6800)},{\sy*(-0.2500)})
	--({\sx*(0.6900)},{\sy*(0.0833)})
	--({\sx*(0.7000)},{\sy*(-0.5833)})
	--({\sx*(0.7100)},{\sy*(0.1667)})
	--({\sx*(0.7200)},{\sy*(0.0000)})
	--({\sx*(0.7300)},{\sy*(-0.3333)})
	--({\sx*(0.7400)},{\sy*(-0.4167)})
	--({\sx*(0.7500)},{\sy*(-0.1667)})
	--({\sx*(0.7600)},{\sy*(0.0833)})
	--({\sx*(0.7700)},{\sy*(-0.0833)})
	--({\sx*(0.7800)},{\sy*(-0.0833)})
	--({\sx*(0.7900)},{\sy*(0.1667)})
	--({\sx*(0.8000)},{\sy*(0.0000)})
	--({\sx*(0.8100)},{\sy*(-0.0833)})
	--({\sx*(0.8200)},{\sy*(-0.2500)})
	--({\sx*(0.8300)},{\sy*(-0.4167)})
	--({\sx*(0.8400)},{\sy*(0.1667)})
	--({\sx*(0.8500)},{\sy*(-0.0833)})
	--({\sx*(0.8600)},{\sy*(0.5000)})
	--({\sx*(0.8700)},{\sy*(-0.1667)})
	--({\sx*(0.8800)},{\sy*(0.3333)})
	--({\sx*(0.8900)},{\sy*(0.0833)})
	--({\sx*(0.9000)},{\sy*(0.3333)})
	--({\sx*(0.9100)},{\sy*(-0.2500)})
	--({\sx*(0.9200)},{\sy*(0.3333)})
	--({\sx*(0.9300)},{\sy*(0.2500)})
	--({\sx*(0.9400)},{\sy*(0.6667)})
	--({\sx*(0.9500)},{\sy*(0.0000)})
	--({\sx*(0.9600)},{\sy*(0.4167)})
	--({\sx*(0.9700)},{\sy*(0.2083)})
	--({\sx*(0.9800)},{\sy*(0.2083)})
	--({\sx*(0.9900)},{\sy*(-0.0417)})
	--({\sx*(1.0000)},{\sy*(0.3333)})
	--({\sx*(1.0100)},{\sy*(0.5000)})
	--({\sx*(1.0200)},{\sy*(0.0833)})
	--({\sx*(1.0300)},{\sy*(0.2500)})
	--({\sx*(1.0400)},{\sy*(0.1250)})
	--({\sx*(1.0500)},{\sy*(0.3333)})
	--({\sx*(1.0600)},{\sy*(-0.0833)})
	--({\sx*(1.0700)},{\sy*(-0.0417)})
	--({\sx*(1.0800)},{\sy*(-0.0417)})
	--({\sx*(1.0900)},{\sy*(0.3333)})
	--({\sx*(1.1000)},{\sy*(0.0000)})
	--({\sx*(1.1100)},{\sy*(-0.0417)})
	--({\sx*(1.1200)},{\sy*(0.2083)})
	--({\sx*(1.1300)},{\sy*(0.0417)})
	--({\sx*(1.1400)},{\sy*(-0.0417)})
	--({\sx*(1.1500)},{\sy*(-0.1250)})
	--({\sx*(1.1600)},{\sy*(-0.2917)})
	--({\sx*(1.1700)},{\sy*(-0.1250)})
	--({\sx*(1.1800)},{\sy*(-0.0417)})
	--({\sx*(1.1900)},{\sy*(0.1250)})
	--({\sx*(1.2000)},{\sy*(-0.1250)})
	--({\sx*(1.2100)},{\sy*(-0.3333)})
	--({\sx*(1.2200)},{\sy*(0.0000)})
	--({\sx*(1.2300)},{\sy*(0.1250)})
	--({\sx*(1.2400)},{\sy*(-0.3750)})
	--({\sx*(1.2500)},{\sy*(0.0000)})
	--({\sx*(1.2600)},{\sy*(0.1667)})
	--({\sx*(1.2700)},{\sy*(0.1250)})
	--({\sx*(1.2800)},{\sy*(-0.2083)})
	--({\sx*(1.2900)},{\sy*(-0.0833)})
	--({\sx*(1.3000)},{\sy*(-0.0417)})
	--({\sx*(1.3100)},{\sy*(-0.0417)})
	--({\sx*(1.3200)},{\sy*(-0.2083)})
	--({\sx*(1.3300)},{\sy*(-0.0417)})
	--({\sx*(1.3400)},{\sy*(0.1667)})
	--({\sx*(1.3500)},{\sy*(0.2917)})
	--({\sx*(1.3600)},{\sy*(0.0833)})
	--({\sx*(1.3700)},{\sy*(0.0833)})
	--({\sx*(1.3800)},{\sy*(0.2083)})
	--({\sx*(1.3900)},{\sy*(-0.0417)})
	--({\sx*(1.4000)},{\sy*(0.1250)})
	--({\sx*(1.4100)},{\sy*(0.0417)})
	--({\sx*(1.4200)},{\sy*(0.0417)})
	--({\sx*(1.4300)},{\sy*(0.5417)})
	--({\sx*(1.4400)},{\sy*(0.0417)})
	--({\sx*(1.4500)},{\sy*(0.0000)})
	--({\sx*(1.4600)},{\sy*(0.0833)})
	--({\sx*(1.4700)},{\sy*(0.2500)})
	--({\sx*(1.4800)},{\sy*(0.1250)})
	--({\sx*(1.4900)},{\sy*(-0.0417)})
	--({\sx*(1.5000)},{\sy*(0.1250)})
	--({\sx*(1.5100)},{\sy*(0.2083)})
	--({\sx*(1.5200)},{\sy*(0.0833)})
	--({\sx*(1.5300)},{\sy*(-0.0625)})
	--({\sx*(1.5400)},{\sy*(0.2500)})
	--({\sx*(1.5500)},{\sy*(0.2917)})
	--({\sx*(1.5600)},{\sy*(0.1667)})
	--({\sx*(1.5700)},{\sy*(-0.0625)})
	--({\sx*(1.5800)},{\sy*(0.1250)})
	--({\sx*(1.5900)},{\sy*(0.2292)})
	--({\sx*(1.6000)},{\sy*(0.2292)})
	--({\sx*(1.6100)},{\sy*(-0.1042)})
	--({\sx*(1.6200)},{\sy*(0.1250)})
	--({\sx*(1.6300)},{\sy*(0.1458)})
	--({\sx*(1.6400)},{\sy*(-0.1042)})
	--({\sx*(1.6500)},{\sy*(-0.0417)})
	--({\sx*(1.6600)},{\sy*(-0.0208)})
	--({\sx*(1.6700)},{\sy*(0.0000)})
	--({\sx*(1.6800)},{\sy*(0.2083)})
	--({\sx*(1.6900)},{\sy*(0.0208)})
	--({\sx*(1.7000)},{\sy*(-0.1458)})
	--({\sx*(1.7100)},{\sy*(-0.0208)})
	--({\sx*(1.7200)},{\sy*(0.0625)})
	--({\sx*(1.7300)},{\sy*(-0.0833)})
	--({\sx*(1.7400)},{\sy*(-0.1458)})
	--({\sx*(1.7500)},{\sy*(0.0417)})
	--({\sx*(1.7600)},{\sy*(0.0833)})
	--({\sx*(1.7700)},{\sy*(0.0208)})
	--({\sx*(1.7800)},{\sy*(-0.1250)})
	--({\sx*(1.7900)},{\sy*(0.0417)})
	--({\sx*(1.8000)},{\sy*(0.1250)})
	--({\sx*(1.8100)},{\sy*(-0.0833)})
	--({\sx*(1.8200)},{\sy*(-0.1042)})
	--({\sx*(1.8300)},{\sy*(0.0000)})
	--({\sx*(1.8400)},{\sy*(0.0208)})
	--({\sx*(1.8500)},{\sy*(-0.0208)})
	--({\sx*(1.8600)},{\sy*(-0.1667)})
	--({\sx*(1.8700)},{\sy*(-0.1042)})
	--({\sx*(1.8800)},{\sy*(0.0833)})
	--({\sx*(1.8900)},{\sy*(0.1042)})
	--({\sx*(1.9000)},{\sy*(-0.0208)})
	--({\sx*(1.9100)},{\sy*(0.0208)})
	--({\sx*(1.9200)},{\sy*(0.0208)})
	--({\sx*(1.9300)},{\sy*(0.0625)})
	--({\sx*(1.9400)},{\sy*(0.0521)})
	--({\sx*(1.9500)},{\sy*(-0.1146)})
	--({\sx*(1.9600)},{\sy*(0.0521)})
	--({\sx*(1.9700)},{\sy*(0.0729)})
	--({\sx*(1.9800)},{\sy*(-0.0208)})
	--({\sx*(1.9900)},{\sy*(-0.0938)})
	--({\sx*(2.0000)},{\sy*(0.0000)})
	--({\sx*(2.0100)},{\sy*(0.0104)})
	--({\sx*(2.0200)},{\sy*(-0.0729)})
	--({\sx*(2.0300)},{\sy*(-0.0729)})
	--({\sx*(2.0400)},{\sy*(0.0000)})
	--({\sx*(2.0500)},{\sy*(0.0521)})
	--({\sx*(2.0600)},{\sy*(0.0104)})
	--({\sx*(2.0700)},{\sy*(0.0312)})
	--({\sx*(2.0800)},{\sy*(-0.0417)})
	--({\sx*(2.0900)},{\sy*(-0.1042)})
	--({\sx*(2.1000)},{\sy*(0.0417)})
	--({\sx*(2.1100)},{\sy*(0.0312)})
	--({\sx*(2.1200)},{\sy*(-0.0417)})
	--({\sx*(2.1300)},{\sy*(-0.0625)})
	--({\sx*(2.1400)},{\sy*(0.0417)})
	--({\sx*(2.1500)},{\sy*(-0.0104)})
	--({\sx*(2.1600)},{\sy*(-0.0208)})
	--({\sx*(2.1700)},{\sy*(0.0000)})
	--({\sx*(2.1800)},{\sy*(-0.0417)})
	--({\sx*(2.1900)},{\sy*(0.0000)})
	--({\sx*(2.2000)},{\sy*(-0.0104)})
	--({\sx*(2.2100)},{\sy*(0.0104)})
	--({\sx*(2.2200)},{\sy*(0.0000)})
	--({\sx*(2.2300)},{\sy*(0.0208)})
	--({\sx*(2.2400)},{\sy*(-0.0208)})
	--({\sx*(2.2500)},{\sy*(0.0000)})
	--({\sx*(2.2600)},{\sy*(0.0052)})
	--({\sx*(2.2700)},{\sy*(0.0104)})
	--({\sx*(2.2800)},{\sy*(0.0417)})
	--({\sx*(2.2900)},{\sy*(0.0260)})
	--({\sx*(2.3000)},{\sy*(-0.0521)})
	--({\sx*(2.3100)},{\sy*(-0.0156)})
	--({\sx*(2.3200)},{\sy*(-0.0156)})
	--({\sx*(2.3300)},{\sy*(0.0052)})
	--({\sx*(2.3400)},{\sy*(-0.0052)})
	--({\sx*(2.3500)},{\sy*(0.0000)})
	--({\sx*(2.3600)},{\sy*(0.0208)})
	--({\sx*(2.3700)},{\sy*(0.0208)})
	--({\sx*(2.3800)},{\sy*(-0.0208)})
	--({\sx*(2.3900)},{\sy*(0.0521)})
	--({\sx*(2.4000)},{\sy*(-0.0625)})
	--({\sx*(2.4100)},{\sy*(0.0156)})
	--({\sx*(2.4200)},{\sy*(0.0208)})
	--({\sx*(2.4300)},{\sy*(0.0052)})
	--({\sx*(2.4400)},{\sy*(0.0312)})
	--({\sx*(2.4500)},{\sy*(0.0469)})
	--({\sx*(2.4600)},{\sy*(0.0156)})
	--({\sx*(2.4700)},{\sy*(0.0156)})
	--({\sx*(2.4800)},{\sy*(0.0365)})
	--({\sx*(2.4900)},{\sy*(0.0156)})
	--({\sx*(2.5000)},{\sy*(0.0000)})
	--({\sx*(2.5100)},{\sy*(0.0208)})
	--({\sx*(2.5200)},{\sy*(0.0208)})
	--({\sx*(2.5300)},{\sy*(-0.0469)})
	--({\sx*(2.5400)},{\sy*(-0.0208)})
	--({\sx*(2.5500)},{\sy*(0.0052)})
	--({\sx*(2.5600)},{\sy*(0.0260)})
	--({\sx*(2.5700)},{\sy*(0.0000)})
	--({\sx*(2.5800)},{\sy*(0.0469)})
	--({\sx*(2.5900)},{\sy*(-0.0208)})
	--({\sx*(2.6000)},{\sy*(0.0026)})
	--({\sx*(2.6100)},{\sy*(-0.0052)})
	--({\sx*(2.6200)},{\sy*(-0.0234)})
	--({\sx*(2.6300)},{\sy*(0.0339)})
	--({\sx*(2.6400)},{\sy*(0.0208)})
	--({\sx*(2.6500)},{\sy*(0.0156)})
	--({\sx*(2.6600)},{\sy*(-0.0182)})
	--({\sx*(2.6700)},{\sy*(-0.0026)})
	--({\sx*(2.6800)},{\sy*(-0.0052)})
	--({\sx*(2.6900)},{\sy*(-0.0260)})
	--({\sx*(2.7000)},{\sy*(-0.0078)})
	--({\sx*(2.7100)},{\sy*(0.0052)})
	--({\sx*(2.7200)},{\sy*(0.0000)})
	--({\sx*(2.7300)},{\sy*(-0.0052)})
	--({\sx*(2.7400)},{\sy*(-0.0104)})
	--({\sx*(2.7500)},{\sy*(0.0078)})
	--({\sx*(2.7600)},{\sy*(0.0078)})
	--({\sx*(2.7700)},{\sy*(0.0000)})
	--({\sx*(2.7800)},{\sy*(0.0104)})
	--({\sx*(2.7900)},{\sy*(-0.0130)})
	--({\sx*(2.8000)},{\sy*(-0.0130)})
	--({\sx*(2.8100)},{\sy*(-0.0052)})
	--({\sx*(2.8200)},{\sy*(0.0169)})
	--({\sx*(2.8300)},{\sy*(0.0326)})
	--({\sx*(2.8400)},{\sy*(-0.0117)})
	--({\sx*(2.8500)},{\sy*(0.0013)})
	--({\sx*(2.8600)},{\sy*(0.0234)})
	--({\sx*(2.8700)},{\sy*(-0.0026)})
	--({\sx*(2.8800)},{\sy*(-0.0182)})
	--({\sx*(2.8900)},{\sy*(-0.0065)})
	--({\sx*(2.9000)},{\sy*(0.0443)})
	--({\sx*(2.9100)},{\sy*(0.0234)})
	--({\sx*(2.9200)},{\sy*(0.0052)})
	--({\sx*(2.9300)},{\sy*(-0.0143)})
	--({\sx*(2.9400)},{\sy*(-0.0065)})
	--({\sx*(2.9500)},{\sy*(-0.0091)})
	--({\sx*(2.9600)},{\sy*(0.0013)})
	--({\sx*(2.9700)},{\sy*(-0.0026)})
	--({\sx*(2.9800)},{\sy*(0.0013)})
	--({\sx*(2.9900)},{\sy*(0.0052)})
	--({\sx*(3.0000)},{\sy*(0.0000)})
	--({\sx*(3.0100)},{\sy*(-0.0026)})
	--({\sx*(3.0200)},{\sy*(0.0039)})
	--({\sx*(3.0300)},{\sy*(0.0052)})
	--({\sx*(3.0400)},{\sy*(0.0078)})
	--({\sx*(3.0500)},{\sy*(0.0085)})
	--({\sx*(3.0600)},{\sy*(0.0091)})
	--({\sx*(3.0700)},{\sy*(-0.0020)})
	--({\sx*(3.0800)},{\sy*(-0.0143)})
	--({\sx*(3.0900)},{\sy*(0.0221)})
	--({\sx*(3.1000)},{\sy*(0.0352)})
	--({\sx*(3.1100)},{\sy*(-0.0358)})
	--({\sx*(3.1200)},{\sy*(-0.0221)})
	--({\sx*(3.1300)},{\sy*(-0.0156)})
	--({\sx*(3.1400)},{\sy*(-0.0260)})
	--({\sx*(3.1500)},{\sy*(0.0189)})
	--({\sx*(3.1600)},{\sy*(0.0065)})
	--({\sx*(3.1700)},{\sy*(0.0013)})
	--({\sx*(3.1800)},{\sy*(0.0163)})
	--({\sx*(3.1900)},{\sy*(0.0078)})
	--({\sx*(3.2000)},{\sy*(0.0052)})
	--({\sx*(3.2100)},{\sy*(-0.0072)})
	--({\sx*(3.2200)},{\sy*(-0.0007)})
	--({\sx*(3.2300)},{\sy*(-0.0013)})
	--({\sx*(3.2400)},{\sy*(-0.0013)})
	--({\sx*(3.2500)},{\sy*(0.0013)})
	--({\sx*(3.2600)},{\sy*(0.0039)})
	--({\sx*(3.2700)},{\sy*(-0.0042)})
	--({\sx*(3.2800)},{\sy*(-0.0091)})
	--({\sx*(3.2900)},{\sy*(0.0101)})
	--({\sx*(3.3000)},{\sy*(-0.0020)})
	--({\sx*(3.3100)},{\sy*(0.0241)})
	--({\sx*(3.3200)},{\sy*(0.0094)})
	--({\sx*(3.3300)},{\sy*(0.0146)})
	--({\sx*(3.3400)},{\sy*(-0.0117)})
	--({\sx*(3.3500)},{\sy*(-0.0059)})
	--({\sx*(3.3600)},{\sy*(0.0007)})
	--({\sx*(3.3700)},{\sy*(0.0046)})
	--({\sx*(3.3800)},{\sy*(-0.0033)})
	--({\sx*(3.3900)},{\sy*(-0.0039)})
	--({\sx*(3.4000)},{\sy*(-0.0205)})
	--({\sx*(3.4100)},{\sy*(-0.0065)})
	--({\sx*(3.4200)},{\sy*(-0.0120)})
	--({\sx*(3.4300)},{\sy*(0.0013)})
	--({\sx*(3.4400)},{\sy*(-0.0055)})
	--({\sx*(3.4500)},{\sy*(-0.0013)})
	--({\sx*(3.4600)},{\sy*(0.0010)})
	--({\sx*(3.4700)},{\sy*(0.0007)})
	--({\sx*(3.4800)},{\sy*(-0.0024)})
	--({\sx*(3.4900)},{\sy*(-0.0024)})
	--({\sx*(3.5000)},{\sy*(-0.0021)})
	--({\sx*(3.5100)},{\sy*(-0.0024)})
	--({\sx*(3.5200)},{\sy*(0.0241)})
	--({\sx*(3.5300)},{\sy*(0.0026)})
	--({\sx*(3.5400)},{\sy*(-0.0059)})
	--({\sx*(3.5500)},{\sy*(0.0054)})
	--({\sx*(3.5600)},{\sy*(0.0151)})
	--({\sx*(3.5700)},{\sy*(-0.0231)})
	--({\sx*(3.5800)},{\sy*(-0.0132)})
	--({\sx*(3.5900)},{\sy*(-0.0254)})
	--({\sx*(3.6000)},{\sy*(0.0138)})
	--({\sx*(3.6100)},{\sy*(0.0023)})
	--({\sx*(3.6200)},{\sy*(0.0015)})
	--({\sx*(3.6300)},{\sy*(-0.0028)})
	--({\sx*(3.6400)},{\sy*(-0.0049)})
	--({\sx*(3.6500)},{\sy*(0.0011)})
	--({\sx*(3.6600)},{\sy*(0.0008)})
	--({\sx*(3.6700)},{\sy*(-0.0009)})
	--({\sx*(3.6800)},{\sy*(0.0002)})
	--({\sx*(3.6900)},{\sy*(-0.0004)})
	--({\sx*(3.7000)},{\sy*(0.0045)})
	--({\sx*(3.7100)},{\sy*(-0.0112)})
	--({\sx*(3.7200)},{\sy*(-0.0059)})
	--({\sx*(3.7300)},{\sy*(-0.0101)})
	--({\sx*(3.7400)},{\sy*(0.0077)})
	--({\sx*(3.7500)},{\sy*(-0.0163)})
	--({\sx*(3.7600)},{\sy*(-0.0101)})
	--({\sx*(3.7700)},{\sy*(-0.0111)})
	--({\sx*(3.7800)},{\sy*(0.0015)})
	--({\sx*(3.7900)},{\sy*(-0.0085)})
	--({\sx*(3.8000)},{\sy*(0.0033)})
	--({\sx*(3.8100)},{\sy*(-0.0059)})
	--({\sx*(3.8200)},{\sy*(0.0175)})
	--({\sx*(3.8300)},{\sy*(-0.0123)})
	--({\sx*(3.8400)},{\sy*(0.0013)})
	--({\sx*(3.8500)},{\sy*(0.0086)})
	--({\sx*(3.8600)},{\sy*(-0.0015)})
	--({\sx*(3.8700)},{\sy*(-0.0034)})
	--({\sx*(3.8800)},{\sy*(-0.0013)})
	--({\sx*(3.8900)},{\sy*(0.0003)})
	--({\sx*(3.9000)},{\sy*(-0.0009)})
	--({\sx*(3.9100)},{\sy*(0.0020)})
	--({\sx*(3.9200)},{\sy*(-0.0045)})
	--({\sx*(3.9300)},{\sy*(0.0055)})
	--({\sx*(3.9400)},{\sy*(0.0171)})
	--({\sx*(3.9500)},{\sy*(-0.0199)})
	--({\sx*(3.9600)},{\sy*(-0.0235)})
	--({\sx*(3.9700)},{\sy*(-0.0086)})
	--({\sx*(3.9800)},{\sy*(-0.0000)})
	--({\sx*(3.9900)},{\sy*(0.0037)})
	--({\sx*(4.0000)},{\sy*(0.0237)})
	--({\sx*(4.0100)},{\sy*(-0.0120)})
	--({\sx*(4.0200)},{\sy*(0.0005)})
	--({\sx*(4.0300)},{\sy*(0.0040)})
	--({\sx*(4.0400)},{\sy*(0.0122)})
	--({\sx*(4.0500)},{\sy*(-0.0075)})
	--({\sx*(4.0600)},{\sy*(-0.0031)})
	--({\sx*(4.0700)},{\sy*(0.0046)})
	--({\sx*(4.0800)},{\sy*(0.0015)})
	--({\sx*(4.0900)},{\sy*(-0.0007)})
	--({\sx*(4.1000)},{\sy*(0.0018)})
	--({\sx*(4.1100)},{\sy*(-0.0030)})
	--({\sx*(4.1200)},{\sy*(-0.0017)})
	--({\sx*(4.1300)},{\sy*(-0.0115)})
	--({\sx*(4.1400)},{\sy*(-0.0057)})
	--({\sx*(4.1500)},{\sy*(-0.0073)})
	--({\sx*(4.1600)},{\sy*(0.0007)})
	--({\sx*(4.1700)},{\sy*(-0.0143)})
	--({\sx*(4.1800)},{\sy*(0.0050)})
	--({\sx*(4.1900)},{\sy*(-0.0067)})
	--({\sx*(4.2000)},{\sy*(-0.0062)})
	--({\sx*(4.2100)},{\sy*(-0.0030)})
	--({\sx*(4.2200)},{\sy*(0.0007)})
	--({\sx*(4.2300)},{\sy*(-0.0031)})
	--({\sx*(4.2400)},{\sy*(-0.0010)})
	--({\sx*(4.2500)},{\sy*(-0.0058)})
	--({\sx*(4.2600)},{\sy*(-0.0018)})
	--({\sx*(4.2700)},{\sy*(-0.0008)})
	--({\sx*(4.2800)},{\sy*(-0.0011)})
	--({\sx*(4.2900)},{\sy*(0.0021)})
	--({\sx*(4.3000)},{\sy*(0.0091)})
	--({\sx*(4.3100)},{\sy*(0.0003)})
	--({\sx*(4.3200)},{\sy*(0.0022)})
	--({\sx*(4.3300)},{\sy*(-0.0078)})
	--({\sx*(4.3400)},{\sy*(0.0104)})
	--({\sx*(4.3500)},{\sy*(0.0053)})
	--({\sx*(4.3600)},{\sy*(-0.0021)})
	--({\sx*(4.3700)},{\sy*(-0.0084)})
	--({\sx*(4.3800)},{\sy*(-0.0019)})
	--({\sx*(4.3900)},{\sy*(0.0043)})
	--({\sx*(4.4000)},{\sy*(0.0107)})
	--({\sx*(4.4100)},{\sy*(-0.0006)})
	--({\sx*(4.4200)},{\sy*(0.0009)})
	--({\sx*(4.4300)},{\sy*(0.0004)})
	--({\sx*(4.4400)},{\sy*(-0.0013)})
	--({\sx*(4.4500)},{\sy*(-0.0074)})
	--({\sx*(4.4600)},{\sy*(-0.0082)})
	--({\sx*(4.4700)},{\sy*(0.0043)})
	--({\sx*(4.4800)},{\sy*(-0.0095)})
	--({\sx*(4.4900)},{\sy*(-0.0056)})
	--({\sx*(4.5000)},{\sy*(-0.0106)})
	--({\sx*(4.5100)},{\sy*(-0.0089)})
	--({\sx*(4.5200)},{\sy*(-0.0031)})
	--({\sx*(4.5300)},{\sy*(0.0031)})
	--({\sx*(4.5400)},{\sy*(-0.0057)})
	--({\sx*(4.5500)},{\sy*(0.0007)})
	--({\sx*(4.5600)},{\sy*(-0.0025)})
	--({\sx*(4.5700)},{\sy*(0.0001)})
	--({\sx*(4.5800)},{\sy*(0.0002)})
	--({\sx*(4.5900)},{\sy*(0.0018)})
	--({\sx*(4.6000)},{\sy*(0.0042)})
	--({\sx*(4.6100)},{\sy*(0.0034)})
	--({\sx*(4.6200)},{\sy*(0.0145)})
	--({\sx*(4.6300)},{\sy*(-0.0026)})
	--({\sx*(4.6400)},{\sy*(0.0169)})
	--({\sx*(4.6500)},{\sy*(0.0173)})
	--({\sx*(4.6600)},{\sy*(0.0080)})
	--({\sx*(4.6700)},{\sy*(0.0027)})
	--({\sx*(4.6800)},{\sy*(0.0053)})
	--({\sx*(4.6900)},{\sy*(-0.0024)})
	--({\sx*(4.7000)},{\sy*(0.0002)})
	--({\sx*(4.7100)},{\sy*(-0.0017)})
	--({\sx*(4.7200)},{\sy*(-0.0044)})
	--({\sx*(4.7300)},{\sy*(-0.0061)})
	--({\sx*(4.7400)},{\sy*(-0.0002)})
	--({\sx*(4.7500)},{\sy*(0.0033)})
	--({\sx*(4.7600)},{\sy*(-0.0022)})
	--({\sx*(4.7700)},{\sy*(0.0037)})
	--({\sx*(4.7800)},{\sy*(-0.0048)})
	--({\sx*(4.7900)},{\sy*(-0.0042)})
	--({\sx*(4.8000)},{\sy*(-0.0020)})
	--({\sx*(4.8100)},{\sy*(-0.0001)})
	--({\sx*(4.8200)},{\sy*(0.0004)})
	--({\sx*(4.8300)},{\sy*(0.0032)})
	--({\sx*(4.8400)},{\sy*(0.0018)})
	--({\sx*(4.8500)},{\sy*(0.0031)})
	--({\sx*(4.8600)},{\sy*(0.0069)})
	--({\sx*(4.8700)},{\sy*(-0.0022)})
	--({\sx*(4.8800)},{\sy*(0.0013)})
	--({\sx*(4.8900)},{\sy*(-0.0002)})
	--({\sx*(4.9000)},{\sy*(0.0006)})
	--({\sx*(4.9100)},{\sy*(-0.0033)})
	--({\sx*(4.9200)},{\sy*(-0.0016)})
	--({\sx*(4.9300)},{\sy*(0.0005)})
	--({\sx*(4.9400)},{\sy*(-0.0057)})
	--({\sx*(4.9500)},{\sy*(-0.0009)})
	--({\sx*(4.9600)},{\sy*(0.0010)})
	--({\sx*(4.9700)},{\sy*(-0.0001)})
	--({\sx*(4.9800)},{\sy*(0.0006)})
	--({\sx*(4.9900)},{\sy*(-0.0004)})
	--({\sx*(5.0000)},{\sy*(0.0000)});
}
\def\relfehlerp{
\draw[color=blue,line width=1.4pt,line join=round] ({\sx*(0.000)},{\sy*(0.0000)})
	--({\sx*(0.0100)},{\sy*(0.0000)})
	--({\sx*(0.0200)},{\sy*(0.0000)})
	--({\sx*(0.0300)},{\sy*(-0.0000)})
	--({\sx*(0.0400)},{\sy*(0.0000)})
	--({\sx*(0.0500)},{\sy*(0.0000)})
	--({\sx*(0.0600)},{\sy*(0.0000)})
	--({\sx*(0.0700)},{\sy*(0.0000)})
	--({\sx*(0.0800)},{\sy*(-0.0000)})
	--({\sx*(0.0900)},{\sy*(0.0000)})
	--({\sx*(0.1000)},{\sy*(0.0000)})
	--({\sx*(0.1100)},{\sy*(0.0000)})
	--({\sx*(0.1200)},{\sy*(-0.0000)})
	--({\sx*(0.1300)},{\sy*(0.0000)})
	--({\sx*(0.1400)},{\sy*(0.0000)})
	--({\sx*(0.1500)},{\sy*(0.0000)})
	--({\sx*(0.1600)},{\sy*(-0.0000)})
	--({\sx*(0.1700)},{\sy*(-0.0000)})
	--({\sx*(0.1800)},{\sy*(0.0000)})
	--({\sx*(0.1900)},{\sy*(-0.0000)})
	--({\sx*(0.2000)},{\sy*(-0.0000)})
	--({\sx*(0.2100)},{\sy*(0.0000)})
	--({\sx*(0.2200)},{\sy*(-0.0000)})
	--({\sx*(0.2300)},{\sy*(-0.0000)})
	--({\sx*(0.2400)},{\sy*(-0.0000)})
	--({\sx*(0.2500)},{\sy*(-0.0000)})
	--({\sx*(0.2600)},{\sy*(0.0000)})
	--({\sx*(0.2700)},{\sy*(-0.0000)})
	--({\sx*(0.2800)},{\sy*(-0.0000)})
	--({\sx*(0.2900)},{\sy*(-0.0000)})
	--({\sx*(0.3000)},{\sy*(0.0000)})
	--({\sx*(0.3100)},{\sy*(0.0000)})
	--({\sx*(0.3200)},{\sy*(-0.0000)})
	--({\sx*(0.3300)},{\sy*(-0.0000)})
	--({\sx*(0.3400)},{\sy*(0.0000)})
	--({\sx*(0.3500)},{\sy*(0.0000)})
	--({\sx*(0.3600)},{\sy*(0.0000)})
	--({\sx*(0.3700)},{\sy*(-0.0000)})
	--({\sx*(0.3800)},{\sy*(0.0000)})
	--({\sx*(0.3900)},{\sy*(0.0000)})
	--({\sx*(0.4000)},{\sy*(-0.0000)})
	--({\sx*(0.4100)},{\sy*(-0.0000)})
	--({\sx*(0.4200)},{\sy*(0.0000)})
	--({\sx*(0.4300)},{\sy*(0.0000)})
	--({\sx*(0.4400)},{\sy*(0.0000)})
	--({\sx*(0.4500)},{\sy*(-0.0000)})
	--({\sx*(0.4600)},{\sy*(0.0000)})
	--({\sx*(0.4700)},{\sy*(0.0000)})
	--({\sx*(0.4800)},{\sy*(0.0000)})
	--({\sx*(0.4900)},{\sy*(0.0000)})
	--({\sx*(0.5000)},{\sy*(0.0000)})
	--({\sx*(0.5100)},{\sy*(0.0000)})
	--({\sx*(0.5200)},{\sy*(0.0000)})
	--({\sx*(0.5300)},{\sy*(-0.0000)})
	--({\sx*(0.5400)},{\sy*(0.0000)})
	--({\sx*(0.5500)},{\sy*(0.0000)})
	--({\sx*(0.5600)},{\sy*(-0.0000)})
	--({\sx*(0.5700)},{\sy*(-0.0000)})
	--({\sx*(0.5800)},{\sy*(-0.0000)})
	--({\sx*(0.5900)},{\sy*(0.0000)})
	--({\sx*(0.6000)},{\sy*(-0.0000)})
	--({\sx*(0.6100)},{\sy*(0.0000)})
	--({\sx*(0.6200)},{\sy*(-0.0000)})
	--({\sx*(0.6300)},{\sy*(0.0000)})
	--({\sx*(0.6400)},{\sy*(-0.0000)})
	--({\sx*(0.6500)},{\sy*(0.0000)})
	--({\sx*(0.6600)},{\sy*(-0.0000)})
	--({\sx*(0.6700)},{\sy*(0.0000)})
	--({\sx*(0.6800)},{\sy*(-0.0000)})
	--({\sx*(0.6900)},{\sy*(0.0000)})
	--({\sx*(0.7000)},{\sy*(-0.0000)})
	--({\sx*(0.7100)},{\sy*(0.0000)})
	--({\sx*(0.7200)},{\sy*(0.0000)})
	--({\sx*(0.7300)},{\sy*(-0.0000)})
	--({\sx*(0.7400)},{\sy*(-0.0000)})
	--({\sx*(0.7500)},{\sy*(-0.0000)})
	--({\sx*(0.7600)},{\sy*(0.0000)})
	--({\sx*(0.7700)},{\sy*(-0.0000)})
	--({\sx*(0.7800)},{\sy*(-0.0000)})
	--({\sx*(0.7900)},{\sy*(0.0000)})
	--({\sx*(0.8000)},{\sy*(0.0000)})
	--({\sx*(0.8100)},{\sy*(-0.0000)})
	--({\sx*(0.8200)},{\sy*(-0.0000)})
	--({\sx*(0.8300)},{\sy*(-0.0000)})
	--({\sx*(0.8400)},{\sy*(0.0000)})
	--({\sx*(0.8500)},{\sy*(-0.0000)})
	--({\sx*(0.8600)},{\sy*(0.0000)})
	--({\sx*(0.8700)},{\sy*(-0.0000)})
	--({\sx*(0.8800)},{\sy*(0.0000)})
	--({\sx*(0.8900)},{\sy*(0.0000)})
	--({\sx*(0.9000)},{\sy*(0.0000)})
	--({\sx*(0.9100)},{\sy*(-0.0000)})
	--({\sx*(0.9200)},{\sy*(0.0000)})
	--({\sx*(0.9300)},{\sy*(0.0000)})
	--({\sx*(0.9400)},{\sy*(0.0000)})
	--({\sx*(0.9500)},{\sy*(0.0000)})
	--({\sx*(0.9600)},{\sy*(0.0000)})
	--({\sx*(0.9700)},{\sy*(0.0000)})
	--({\sx*(0.9800)},{\sy*(0.0000)})
	--({\sx*(0.9900)},{\sy*(-0.0000)})
	--({\sx*(1.0000)},{\sy*(0.0000)})
	--({\sx*(1.0100)},{\sy*(0.0000)})
	--({\sx*(1.0200)},{\sy*(0.0000)})
	--({\sx*(1.0300)},{\sy*(0.0000)})
	--({\sx*(1.0400)},{\sy*(0.0000)})
	--({\sx*(1.0500)},{\sy*(0.0000)})
	--({\sx*(1.0600)},{\sy*(-0.0000)})
	--({\sx*(1.0700)},{\sy*(-0.0000)})
	--({\sx*(1.0800)},{\sy*(-0.0000)})
	--({\sx*(1.0900)},{\sy*(0.0000)})
	--({\sx*(1.1000)},{\sy*(0.0000)})
	--({\sx*(1.1100)},{\sy*(-0.0000)})
	--({\sx*(1.1200)},{\sy*(0.0000)})
	--({\sx*(1.1300)},{\sy*(0.0000)})
	--({\sx*(1.1400)},{\sy*(-0.0000)})
	--({\sx*(1.1500)},{\sy*(-0.0000)})
	--({\sx*(1.1600)},{\sy*(-0.0000)})
	--({\sx*(1.1700)},{\sy*(-0.0000)})
	--({\sx*(1.1800)},{\sy*(-0.0000)})
	--({\sx*(1.1900)},{\sy*(0.0000)})
	--({\sx*(1.2000)},{\sy*(-0.0000)})
	--({\sx*(1.2100)},{\sy*(-0.0000)})
	--({\sx*(1.2200)},{\sy*(0.0000)})
	--({\sx*(1.2300)},{\sy*(0.0000)})
	--({\sx*(1.2400)},{\sy*(-0.0000)})
	--({\sx*(1.2500)},{\sy*(0.0000)})
	--({\sx*(1.2600)},{\sy*(0.0000)})
	--({\sx*(1.2700)},{\sy*(0.0000)})
	--({\sx*(1.2800)},{\sy*(-0.0000)})
	--({\sx*(1.2900)},{\sy*(-0.0000)})
	--({\sx*(1.3000)},{\sy*(-0.0000)})
	--({\sx*(1.3100)},{\sy*(-0.0000)})
	--({\sx*(1.3200)},{\sy*(-0.0000)})
	--({\sx*(1.3300)},{\sy*(-0.0000)})
	--({\sx*(1.3400)},{\sy*(0.0000)})
	--({\sx*(1.3500)},{\sy*(0.0000)})
	--({\sx*(1.3600)},{\sy*(0.0000)})
	--({\sx*(1.3700)},{\sy*(0.0000)})
	--({\sx*(1.3800)},{\sy*(0.0000)})
	--({\sx*(1.3900)},{\sy*(-0.0000)})
	--({\sx*(1.4000)},{\sy*(0.0000)})
	--({\sx*(1.4100)},{\sy*(0.0000)})
	--({\sx*(1.4200)},{\sy*(0.0000)})
	--({\sx*(1.4300)},{\sy*(0.0000)})
	--({\sx*(1.4400)},{\sy*(0.0000)})
	--({\sx*(1.4500)},{\sy*(0.0000)})
	--({\sx*(1.4600)},{\sy*(0.0000)})
	--({\sx*(1.4700)},{\sy*(0.0000)})
	--({\sx*(1.4800)},{\sy*(0.0000)})
	--({\sx*(1.4900)},{\sy*(-0.0000)})
	--({\sx*(1.5000)},{\sy*(0.0000)})
	--({\sx*(1.5100)},{\sy*(0.0000)})
	--({\sx*(1.5200)},{\sy*(0.0000)})
	--({\sx*(1.5300)},{\sy*(-0.0000)})
	--({\sx*(1.5400)},{\sy*(0.0000)})
	--({\sx*(1.5500)},{\sy*(0.0000)})
	--({\sx*(1.5600)},{\sy*(0.0000)})
	--({\sx*(1.5700)},{\sy*(-0.0000)})
	--({\sx*(1.5800)},{\sy*(0.0000)})
	--({\sx*(1.5900)},{\sy*(0.0000)})
	--({\sx*(1.6000)},{\sy*(0.0000)})
	--({\sx*(1.6100)},{\sy*(-0.0000)})
	--({\sx*(1.6200)},{\sy*(0.0000)})
	--({\sx*(1.6300)},{\sy*(0.0000)})
	--({\sx*(1.6400)},{\sy*(-0.0000)})
	--({\sx*(1.6500)},{\sy*(-0.0000)})
	--({\sx*(1.6600)},{\sy*(-0.0000)})
	--({\sx*(1.6700)},{\sy*(0.0000)})
	--({\sx*(1.6800)},{\sy*(0.0000)})
	--({\sx*(1.6900)},{\sy*(0.0000)})
	--({\sx*(1.7000)},{\sy*(-0.0000)})
	--({\sx*(1.7100)},{\sy*(-0.0000)})
	--({\sx*(1.7200)},{\sy*(0.0000)})
	--({\sx*(1.7300)},{\sy*(-0.0000)})
	--({\sx*(1.7400)},{\sy*(-0.0000)})
	--({\sx*(1.7500)},{\sy*(0.0000)})
	--({\sx*(1.7600)},{\sy*(0.0000)})
	--({\sx*(1.7700)},{\sy*(0.0000)})
	--({\sx*(1.7800)},{\sy*(-0.0000)})
	--({\sx*(1.7900)},{\sy*(0.0000)})
	--({\sx*(1.8000)},{\sy*(0.0000)})
	--({\sx*(1.8100)},{\sy*(-0.0000)})
	--({\sx*(1.8200)},{\sy*(-0.0000)})
	--({\sx*(1.8300)},{\sy*(0.0000)})
	--({\sx*(1.8400)},{\sy*(0.0000)})
	--({\sx*(1.8500)},{\sy*(-0.0000)})
	--({\sx*(1.8600)},{\sy*(-0.0000)})
	--({\sx*(1.8700)},{\sy*(-0.0000)})
	--({\sx*(1.8800)},{\sy*(0.0000)})
	--({\sx*(1.8900)},{\sy*(0.0000)})
	--({\sx*(1.9000)},{\sy*(-0.0000)})
	--({\sx*(1.9100)},{\sy*(0.0000)})
	--({\sx*(1.9200)},{\sy*(0.0000)})
	--({\sx*(1.9300)},{\sy*(0.0000)})
	--({\sx*(1.9400)},{\sy*(0.0000)})
	--({\sx*(1.9500)},{\sy*(-0.0000)})
	--({\sx*(1.9600)},{\sy*(0.0000)})
	--({\sx*(1.9700)},{\sy*(0.0000)})
	--({\sx*(1.9800)},{\sy*(-0.0000)})
	--({\sx*(1.9900)},{\sy*(-0.0000)})
	--({\sx*(2.0000)},{\sy*(0.0000)})
	--({\sx*(2.0100)},{\sy*(0.0000)})
	--({\sx*(2.0200)},{\sy*(-0.0000)})
	--({\sx*(2.0300)},{\sy*(-0.0000)})
	--({\sx*(2.0400)},{\sy*(0.0000)})
	--({\sx*(2.0500)},{\sy*(0.0000)})
	--({\sx*(2.0600)},{\sy*(0.0000)})
	--({\sx*(2.0700)},{\sy*(0.0000)})
	--({\sx*(2.0800)},{\sy*(-0.0000)})
	--({\sx*(2.0900)},{\sy*(-0.0000)})
	--({\sx*(2.1000)},{\sy*(0.0000)})
	--({\sx*(2.1100)},{\sy*(0.0000)})
	--({\sx*(2.1200)},{\sy*(-0.0000)})
	--({\sx*(2.1300)},{\sy*(-0.0000)})
	--({\sx*(2.1400)},{\sy*(0.0000)})
	--({\sx*(2.1500)},{\sy*(-0.0000)})
	--({\sx*(2.1600)},{\sy*(-0.0000)})
	--({\sx*(2.1700)},{\sy*(0.0000)})
	--({\sx*(2.1800)},{\sy*(-0.0000)})
	--({\sx*(2.1900)},{\sy*(0.0000)})
	--({\sx*(2.2000)},{\sy*(-0.0000)})
	--({\sx*(2.2100)},{\sy*(0.0000)})
	--({\sx*(2.2200)},{\sy*(0.0000)})
	--({\sx*(2.2300)},{\sy*(0.0000)})
	--({\sx*(2.2400)},{\sy*(-0.0000)})
	--({\sx*(2.2500)},{\sy*(0.0000)})
	--({\sx*(2.2600)},{\sy*(0.0000)})
	--({\sx*(2.2700)},{\sy*(0.0000)})
	--({\sx*(2.2800)},{\sy*(0.0000)})
	--({\sx*(2.2900)},{\sy*(0.0000)})
	--({\sx*(2.3000)},{\sy*(-0.0000)})
	--({\sx*(2.3100)},{\sy*(-0.0000)})
	--({\sx*(2.3200)},{\sy*(-0.0000)})
	--({\sx*(2.3300)},{\sy*(0.0000)})
	--({\sx*(2.3400)},{\sy*(-0.0000)})
	--({\sx*(2.3500)},{\sy*(0.0000)})
	--({\sx*(2.3600)},{\sy*(0.0000)})
	--({\sx*(2.3700)},{\sy*(0.0000)})
	--({\sx*(2.3800)},{\sy*(-0.0000)})
	--({\sx*(2.3900)},{\sy*(0.0000)})
	--({\sx*(2.4000)},{\sy*(-0.0000)})
	--({\sx*(2.4100)},{\sy*(0.0000)})
	--({\sx*(2.4200)},{\sy*(0.0000)})
	--({\sx*(2.4300)},{\sy*(0.0000)})
	--({\sx*(2.4400)},{\sy*(0.0000)})
	--({\sx*(2.4500)},{\sy*(0.0000)})
	--({\sx*(2.4600)},{\sy*(0.0000)})
	--({\sx*(2.4700)},{\sy*(0.0000)})
	--({\sx*(2.4800)},{\sy*(0.0000)})
	--({\sx*(2.4900)},{\sy*(0.0000)})
	--({\sx*(2.5000)},{\sy*(0.0000)})
	--({\sx*(2.5100)},{\sy*(0.0000)})
	--({\sx*(2.5200)},{\sy*(0.0000)})
	--({\sx*(2.5300)},{\sy*(-0.0000)})
	--({\sx*(2.5400)},{\sy*(-0.0000)})
	--({\sx*(2.5500)},{\sy*(0.0000)})
	--({\sx*(2.5600)},{\sy*(0.0000)})
	--({\sx*(2.5700)},{\sy*(0.0000)})
	--({\sx*(2.5800)},{\sy*(0.0000)})
	--({\sx*(2.5900)},{\sy*(-0.0000)})
	--({\sx*(2.6000)},{\sy*(0.0000)})
	--({\sx*(2.6100)},{\sy*(-0.0000)})
	--({\sx*(2.6200)},{\sy*(-0.0000)})
	--({\sx*(2.6300)},{\sy*(0.0000)})
	--({\sx*(2.6400)},{\sy*(0.0000)})
	--({\sx*(2.6500)},{\sy*(0.0000)})
	--({\sx*(2.6600)},{\sy*(-0.0000)})
	--({\sx*(2.6700)},{\sy*(-0.0000)})
	--({\sx*(2.6800)},{\sy*(-0.0000)})
	--({\sx*(2.6900)},{\sy*(-0.0000)})
	--({\sx*(2.7000)},{\sy*(-0.0000)})
	--({\sx*(2.7100)},{\sy*(0.0000)})
	--({\sx*(2.7200)},{\sy*(0.0000)})
	--({\sx*(2.7300)},{\sy*(-0.0000)})
	--({\sx*(2.7400)},{\sy*(-0.0000)})
	--({\sx*(2.7500)},{\sy*(0.0000)})
	--({\sx*(2.7600)},{\sy*(0.0000)})
	--({\sx*(2.7700)},{\sy*(0.0000)})
	--({\sx*(2.7800)},{\sy*(0.0000)})
	--({\sx*(2.7900)},{\sy*(-0.0000)})
	--({\sx*(2.8000)},{\sy*(-0.0000)})
	--({\sx*(2.8100)},{\sy*(-0.0000)})
	--({\sx*(2.8200)},{\sy*(0.0000)})
	--({\sx*(2.8300)},{\sy*(0.0000)})
	--({\sx*(2.8400)},{\sy*(-0.0000)})
	--({\sx*(2.8500)},{\sy*(0.0000)})
	--({\sx*(2.8600)},{\sy*(0.0000)})
	--({\sx*(2.8700)},{\sy*(-0.0000)})
	--({\sx*(2.8800)},{\sy*(-0.0000)})
	--({\sx*(2.8900)},{\sy*(-0.0000)})
	--({\sx*(2.9000)},{\sy*(0.0000)})
	--({\sx*(2.9100)},{\sy*(0.0000)})
	--({\sx*(2.9200)},{\sy*(0.0000)})
	--({\sx*(2.9300)},{\sy*(-0.0000)})
	--({\sx*(2.9400)},{\sy*(-0.0000)})
	--({\sx*(2.9500)},{\sy*(-0.0000)})
	--({\sx*(2.9600)},{\sy*(0.0000)})
	--({\sx*(2.9700)},{\sy*(-0.0000)})
	--({\sx*(2.9800)},{\sy*(0.0000)})
	--({\sx*(2.9900)},{\sy*(0.0000)})
	--({\sx*(3.0000)},{\sy*(0.0000)})
	--({\sx*(3.0100)},{\sy*(-0.0000)})
	--({\sx*(3.0200)},{\sy*(0.0000)})
	--({\sx*(3.0300)},{\sy*(0.0000)})
	--({\sx*(3.0400)},{\sy*(0.0000)})
	--({\sx*(3.0500)},{\sy*(0.0000)})
	--({\sx*(3.0600)},{\sy*(0.0000)})
	--({\sx*(3.0700)},{\sy*(-0.0000)})
	--({\sx*(3.0800)},{\sy*(-0.0000)})
	--({\sx*(3.0900)},{\sy*(0.0000)})
	--({\sx*(3.1000)},{\sy*(0.0000)})
	--({\sx*(3.1100)},{\sy*(-0.0000)})
	--({\sx*(3.1200)},{\sy*(-0.0000)})
	--({\sx*(3.1300)},{\sy*(-0.0000)})
	--({\sx*(3.1400)},{\sy*(-0.0000)})
	--({\sx*(3.1500)},{\sy*(0.0000)})
	--({\sx*(3.1600)},{\sy*(0.0000)})
	--({\sx*(3.1700)},{\sy*(0.0000)})
	--({\sx*(3.1800)},{\sy*(0.0000)})
	--({\sx*(3.1900)},{\sy*(0.0000)})
	--({\sx*(3.2000)},{\sy*(0.0000)})
	--({\sx*(3.2100)},{\sy*(-0.0000)})
	--({\sx*(3.2200)},{\sy*(-0.0000)})
	--({\sx*(3.2300)},{\sy*(-0.0000)})
	--({\sx*(3.2400)},{\sy*(-0.0000)})
	--({\sx*(3.2500)},{\sy*(0.0000)})
	--({\sx*(3.2600)},{\sy*(0.0000)})
	--({\sx*(3.2700)},{\sy*(-0.0000)})
	--({\sx*(3.2800)},{\sy*(-0.0000)})
	--({\sx*(3.2900)},{\sy*(0.0000)})
	--({\sx*(3.3000)},{\sy*(-0.0000)})
	--({\sx*(3.3100)},{\sy*(0.0000)})
	--({\sx*(3.3200)},{\sy*(0.0000)})
	--({\sx*(3.3300)},{\sy*(0.0000)})
	--({\sx*(3.3400)},{\sy*(-0.0000)})
	--({\sx*(3.3500)},{\sy*(-0.0000)})
	--({\sx*(3.3600)},{\sy*(0.0000)})
	--({\sx*(3.3700)},{\sy*(0.0000)})
	--({\sx*(3.3800)},{\sy*(-0.0000)})
	--({\sx*(3.3900)},{\sy*(-0.0000)})
	--({\sx*(3.4000)},{\sy*(-0.0000)})
	--({\sx*(3.4100)},{\sy*(-0.0000)})
	--({\sx*(3.4200)},{\sy*(-0.0000)})
	--({\sx*(3.4300)},{\sy*(0.0000)})
	--({\sx*(3.4400)},{\sy*(-0.0000)})
	--({\sx*(3.4500)},{\sy*(-0.0000)})
	--({\sx*(3.4600)},{\sy*(0.0000)})
	--({\sx*(3.4700)},{\sy*(0.0000)})
	--({\sx*(3.4800)},{\sy*(-0.0000)})
	--({\sx*(3.4900)},{\sy*(-0.0000)})
	--({\sx*(3.5000)},{\sy*(-0.0000)})
	--({\sx*(3.5100)},{\sy*(-0.0000)})
	--({\sx*(3.5200)},{\sy*(0.0000)})
	--({\sx*(3.5300)},{\sy*(0.0000)})
	--({\sx*(3.5400)},{\sy*(-0.0000)})
	--({\sx*(3.5500)},{\sy*(0.0000)})
	--({\sx*(3.5600)},{\sy*(0.0000)})
	--({\sx*(3.5700)},{\sy*(-0.0000)})
	--({\sx*(3.5800)},{\sy*(-0.0000)})
	--({\sx*(3.5900)},{\sy*(-0.0000)})
	--({\sx*(3.6000)},{\sy*(0.0000)})
	--({\sx*(3.6100)},{\sy*(0.0000)})
	--({\sx*(3.6200)},{\sy*(0.0000)})
	--({\sx*(3.6300)},{\sy*(-0.0000)})
	--({\sx*(3.6400)},{\sy*(-0.0000)})
	--({\sx*(3.6500)},{\sy*(0.0000)})
	--({\sx*(3.6600)},{\sy*(0.0000)})
	--({\sx*(3.6700)},{\sy*(-0.0000)})
	--({\sx*(3.6800)},{\sy*(0.0000)})
	--({\sx*(3.6900)},{\sy*(-0.0000)})
	--({\sx*(3.7000)},{\sy*(0.0000)})
	--({\sx*(3.7100)},{\sy*(-0.0000)})
	--({\sx*(3.7200)},{\sy*(-0.0000)})
	--({\sx*(3.7300)},{\sy*(-0.0000)})
	--({\sx*(3.7400)},{\sy*(0.0000)})
	--({\sx*(3.7500)},{\sy*(-0.0000)})
	--({\sx*(3.7600)},{\sy*(-0.0000)})
	--({\sx*(3.7700)},{\sy*(-0.0000)})
	--({\sx*(3.7800)},{\sy*(0.0000)})
	--({\sx*(3.7900)},{\sy*(-0.0000)})
	--({\sx*(3.8000)},{\sy*(0.0000)})
	--({\sx*(3.8100)},{\sy*(-0.0000)})
	--({\sx*(3.8200)},{\sy*(0.0000)})
	--({\sx*(3.8300)},{\sy*(-0.0000)})
	--({\sx*(3.8400)},{\sy*(0.0000)})
	--({\sx*(3.8500)},{\sy*(0.0000)})
	--({\sx*(3.8600)},{\sy*(-0.0000)})
	--({\sx*(3.8700)},{\sy*(-0.0000)})
	--({\sx*(3.8800)},{\sy*(-0.0000)})
	--({\sx*(3.8900)},{\sy*(0.0000)})
	--({\sx*(3.9000)},{\sy*(-0.0000)})
	--({\sx*(3.9100)},{\sy*(0.0000)})
	--({\sx*(3.9200)},{\sy*(-0.0000)})
	--({\sx*(3.9300)},{\sy*(0.0000)})
	--({\sx*(3.9400)},{\sy*(0.0000)})
	--({\sx*(3.9500)},{\sy*(-0.0000)})
	--({\sx*(3.9600)},{\sy*(-0.0000)})
	--({\sx*(3.9700)},{\sy*(-0.0000)})
	--({\sx*(3.9800)},{\sy*(-0.0000)})
	--({\sx*(3.9900)},{\sy*(0.0000)})
	--({\sx*(4.0000)},{\sy*(0.0000)})
	--({\sx*(4.0100)},{\sy*(-0.0000)})
	--({\sx*(4.0200)},{\sy*(0.0000)})
	--({\sx*(4.0300)},{\sy*(0.0000)})
	--({\sx*(4.0400)},{\sy*(0.0000)})
	--({\sx*(4.0500)},{\sy*(-0.0000)})
	--({\sx*(4.0600)},{\sy*(-0.0000)})
	--({\sx*(4.0700)},{\sy*(0.0000)})
	--({\sx*(4.0800)},{\sy*(0.0000)})
	--({\sx*(4.0900)},{\sy*(-0.0000)})
	--({\sx*(4.1000)},{\sy*(0.0000)})
	--({\sx*(4.1100)},{\sy*(-0.0000)})
	--({\sx*(4.1200)},{\sy*(-0.0000)})
	--({\sx*(4.1300)},{\sy*(-0.0000)})
	--({\sx*(4.1400)},{\sy*(-0.0000)})
	--({\sx*(4.1500)},{\sy*(-0.0000)})
	--({\sx*(4.1600)},{\sy*(0.0000)})
	--({\sx*(4.1700)},{\sy*(-0.0000)})
	--({\sx*(4.1800)},{\sy*(0.0000)})
	--({\sx*(4.1900)},{\sy*(-0.0000)})
	--({\sx*(4.2000)},{\sy*(-0.0000)})
	--({\sx*(4.2100)},{\sy*(-0.0000)})
	--({\sx*(4.2200)},{\sy*(0.0000)})
	--({\sx*(4.2300)},{\sy*(-0.0000)})
	--({\sx*(4.2400)},{\sy*(-0.0000)})
	--({\sx*(4.2500)},{\sy*(-0.0000)})
	--({\sx*(4.2600)},{\sy*(-0.0000)})
	--({\sx*(4.2700)},{\sy*(-0.0000)})
	--({\sx*(4.2800)},{\sy*(-0.0000)})
	--({\sx*(4.2900)},{\sy*(0.0000)})
	--({\sx*(4.3000)},{\sy*(0.0000)})
	--({\sx*(4.3100)},{\sy*(0.0000)})
	--({\sx*(4.3200)},{\sy*(0.0000)})
	--({\sx*(4.3300)},{\sy*(-0.0000)})
	--({\sx*(4.3400)},{\sy*(0.0000)})
	--({\sx*(4.3500)},{\sy*(0.0000)})
	--({\sx*(4.3600)},{\sy*(-0.0000)})
	--({\sx*(4.3700)},{\sy*(-0.0000)})
	--({\sx*(4.3800)},{\sy*(-0.0000)})
	--({\sx*(4.3900)},{\sy*(0.0000)})
	--({\sx*(4.4000)},{\sy*(0.0000)})
	--({\sx*(4.4100)},{\sy*(-0.0000)})
	--({\sx*(4.4200)},{\sy*(0.0000)})
	--({\sx*(4.4300)},{\sy*(0.0000)})
	--({\sx*(4.4400)},{\sy*(-0.0000)})
	--({\sx*(4.4500)},{\sy*(-0.0000)})
	--({\sx*(4.4600)},{\sy*(-0.0000)})
	--({\sx*(4.4700)},{\sy*(0.0000)})
	--({\sx*(4.4800)},{\sy*(-0.0000)})
	--({\sx*(4.4900)},{\sy*(-0.0000)})
	--({\sx*(4.5000)},{\sy*(-0.0000)})
	--({\sx*(4.5100)},{\sy*(-0.0000)})
	--({\sx*(4.5200)},{\sy*(-0.0000)})
	--({\sx*(4.5300)},{\sy*(0.0000)})
	--({\sx*(4.5400)},{\sy*(-0.0000)})
	--({\sx*(4.5500)},{\sy*(0.0000)})
	--({\sx*(4.5600)},{\sy*(-0.0000)})
	--({\sx*(4.5700)},{\sy*(0.0000)})
	--({\sx*(4.5800)},{\sy*(0.0000)})
	--({\sx*(4.5900)},{\sy*(0.0000)})
	--({\sx*(4.6000)},{\sy*(0.0000)})
	--({\sx*(4.6100)},{\sy*(0.0000)})
	--({\sx*(4.6200)},{\sy*(0.0000)})
	--({\sx*(4.6300)},{\sy*(-0.0000)})
	--({\sx*(4.6400)},{\sy*(0.0000)})
	--({\sx*(4.6500)},{\sy*(0.0000)})
	--({\sx*(4.6600)},{\sy*(0.0000)})
	--({\sx*(4.6700)},{\sy*(0.0000)})
	--({\sx*(4.6800)},{\sy*(0.0000)})
	--({\sx*(4.6900)},{\sy*(-0.0000)})
	--({\sx*(4.7000)},{\sy*(0.0000)})
	--({\sx*(4.7100)},{\sy*(-0.0000)})
	--({\sx*(4.7200)},{\sy*(-0.0000)})
	--({\sx*(4.7300)},{\sy*(-0.0000)})
	--({\sx*(4.7400)},{\sy*(-0.0000)})
	--({\sx*(4.7500)},{\sy*(0.0000)})
	--({\sx*(4.7600)},{\sy*(-0.0000)})
	--({\sx*(4.7700)},{\sy*(0.0000)})
	--({\sx*(4.7800)},{\sy*(-0.0000)})
	--({\sx*(4.7900)},{\sy*(-0.0000)})
	--({\sx*(4.8000)},{\sy*(-0.0000)})
	--({\sx*(4.8100)},{\sy*(-0.0000)})
	--({\sx*(4.8200)},{\sy*(0.0000)})
	--({\sx*(4.8300)},{\sy*(0.0000)})
	--({\sx*(4.8400)},{\sy*(0.0000)})
	--({\sx*(4.8500)},{\sy*(0.0000)})
	--({\sx*(4.8600)},{\sy*(0.0000)})
	--({\sx*(4.8700)},{\sy*(-0.0000)})
	--({\sx*(4.8800)},{\sy*(0.0000)})
	--({\sx*(4.8900)},{\sy*(-0.0000)})
	--({\sx*(4.9000)},{\sy*(0.0000)})
	--({\sx*(4.9100)},{\sy*(-0.0000)})
	--({\sx*(4.9200)},{\sy*(-0.0000)})
	--({\sx*(4.9300)},{\sy*(0.0000)})
	--({\sx*(4.9400)},{\sy*(-0.0000)})
	--({\sx*(4.9500)},{\sy*(-0.0000)})
	--({\sx*(4.9600)},{\sy*(0.0000)})
	--({\sx*(4.9700)},{\sy*(-0.0000)})
	--({\sx*(4.9800)},{\sy*(0.0000)})
	--({\sx*(4.9900)},{\sy*(-0.0000)})
	--({\sx*(5.0000)},{\sy*(0.0000)});
}
\def\xwerteq{
\fill[color=red] (0.0000,0) circle[radius={0.07/\skala}];
\fill[color=white] (0.0000,0) circle[radius={0.05/\skala}];
\fill[color=red] (0.0107,0) circle[radius={0.07/\skala}];
\fill[color=white] (0.0107,0) circle[radius={0.05/\skala}];
\fill[color=red] (0.0426,0) circle[radius={0.07/\skala}];
\fill[color=white] (0.0426,0) circle[radius={0.05/\skala}];
\fill[color=red] (0.0954,0) circle[radius={0.07/\skala}];
\fill[color=white] (0.0954,0) circle[radius={0.05/\skala}];
\fill[color=red] (0.1688,0) circle[radius={0.07/\skala}];
\fill[color=white] (0.1688,0) circle[radius={0.05/\skala}];
\fill[color=red] (0.2621,0) circle[radius={0.07/\skala}];
\fill[color=white] (0.2621,0) circle[radius={0.05/\skala}];
\fill[color=red] (0.3745,0) circle[radius={0.07/\skala}];
\fill[color=white] (0.3745,0) circle[radius={0.05/\skala}];
\fill[color=red] (0.5050,0) circle[radius={0.07/\skala}];
\fill[color=white] (0.5050,0) circle[radius={0.05/\skala}];
\fill[color=red] (0.6525,0) circle[radius={0.07/\skala}];
\fill[color=white] (0.6525,0) circle[radius={0.05/\skala}];
\fill[color=red] (0.8158,0) circle[radius={0.07/\skala}];
\fill[color=white] (0.8158,0) circle[radius={0.05/\skala}];
\fill[color=red] (0.9934,0) circle[radius={0.07/\skala}];
\fill[color=white] (0.9934,0) circle[radius={0.05/\skala}];
\fill[color=red] (1.1839,0) circle[radius={0.07/\skala}];
\fill[color=white] (1.1839,0) circle[radius={0.05/\skala}];
\fill[color=red] (1.3857,0) circle[radius={0.07/\skala}];
\fill[color=white] (1.3857,0) circle[radius={0.05/\skala}];
\fill[color=red] (1.5969,0) circle[radius={0.07/\skala}];
\fill[color=white] (1.5969,0) circle[radius={0.05/\skala}];
\fill[color=red] (1.8158,0) circle[radius={0.07/\skala}];
\fill[color=white] (1.8158,0) circle[radius={0.05/\skala}];
\fill[color=red] (2.0406,0) circle[radius={0.07/\skala}];
\fill[color=white] (2.0406,0) circle[radius={0.05/\skala}];
\fill[color=red] (2.2693,0) circle[radius={0.07/\skala}];
\fill[color=white] (2.2693,0) circle[radius={0.05/\skala}];
\fill[color=red] (2.5000,0) circle[radius={0.07/\skala}];
\fill[color=white] (2.5000,0) circle[radius={0.05/\skala}];
\fill[color=red] (2.7307,0) circle[radius={0.07/\skala}];
\fill[color=white] (2.7307,0) circle[radius={0.05/\skala}];
\fill[color=red] (2.9594,0) circle[radius={0.07/\skala}];
\fill[color=white] (2.9594,0) circle[radius={0.05/\skala}];
\fill[color=red] (3.1842,0) circle[radius={0.07/\skala}];
\fill[color=white] (3.1842,0) circle[radius={0.05/\skala}];
\fill[color=red] (3.4031,0) circle[radius={0.07/\skala}];
\fill[color=white] (3.4031,0) circle[radius={0.05/\skala}];
\fill[color=red] (3.6143,0) circle[radius={0.07/\skala}];
\fill[color=white] (3.6143,0) circle[radius={0.05/\skala}];
\fill[color=red] (3.8161,0) circle[radius={0.07/\skala}];
\fill[color=white] (3.8161,0) circle[radius={0.05/\skala}];
\fill[color=red] (4.0066,0) circle[radius={0.07/\skala}];
\fill[color=white] (4.0066,0) circle[radius={0.05/\skala}];
\fill[color=red] (4.1842,0) circle[radius={0.07/\skala}];
\fill[color=white] (4.1842,0) circle[radius={0.05/\skala}];
\fill[color=red] (4.3475,0) circle[radius={0.07/\skala}];
\fill[color=white] (4.3475,0) circle[radius={0.05/\skala}];
\fill[color=red] (4.4950,0) circle[radius={0.07/\skala}];
\fill[color=white] (4.4950,0) circle[radius={0.05/\skala}];
\fill[color=red] (4.6255,0) circle[radius={0.07/\skala}];
\fill[color=white] (4.6255,0) circle[radius={0.05/\skala}];
\fill[color=red] (4.7379,0) circle[radius={0.07/\skala}];
\fill[color=white] (4.7379,0) circle[radius={0.05/\skala}];
\fill[color=red] (4.8312,0) circle[radius={0.07/\skala}];
\fill[color=white] (4.8312,0) circle[radius={0.05/\skala}];
\fill[color=red] (4.9046,0) circle[radius={0.07/\skala}];
\fill[color=white] (4.9046,0) circle[radius={0.05/\skala}];
\fill[color=red] (4.9574,0) circle[radius={0.07/\skala}];
\fill[color=white] (4.9574,0) circle[radius={0.05/\skala}];
\fill[color=red] (4.9893,0) circle[radius={0.07/\skala}];
\fill[color=white] (4.9893,0) circle[radius={0.05/\skala}];
\fill[color=red] (5.0000,0) circle[radius={0.07/\skala}];
\fill[color=white] (5.0000,0) circle[radius={0.05/\skala}];
}
\def\punkteq{34}
\def\maxfehlerq{7.216\cdot 10^{-16}}
\def\fehlerq{
\draw[color=red,line width=1.4pt,line join=round] ({\sx*(0.000)},{\sy*(0.0000)})
	--({\sx*(0.0100)},{\sy*(0.2308)})
	--({\sx*(0.0200)},{\sy*(0.3077)})
	--({\sx*(0.0300)},{\sy*(-0.2308)})
	--({\sx*(0.0400)},{\sy*(-0.6154)})
	--({\sx*(0.0500)},{\sy*(0.5385)})
	--({\sx*(0.0600)},{\sy*(0.0000)})
	--({\sx*(0.0700)},{\sy*(-0.3077)})
	--({\sx*(0.0800)},{\sy*(-0.6154)})
	--({\sx*(0.0900)},{\sy*(0.0769)})
	--({\sx*(0.1000)},{\sy*(0.1538)})
	--({\sx*(0.1100)},{\sy*(-0.3077)})
	--({\sx*(0.1200)},{\sy*(-0.6923)})
	--({\sx*(0.1300)},{\sy*(0.8462)})
	--({\sx*(0.1400)},{\sy*(0.0000)})
	--({\sx*(0.1500)},{\sy*(-0.7692)})
	--({\sx*(0.1600)},{\sy*(-0.3846)})
	--({\sx*(0.1700)},{\sy*(0.3846)})
	--({\sx*(0.1800)},{\sy*(0.5385)})
	--({\sx*(0.1900)},{\sy*(0.1538)})
	--({\sx*(0.2000)},{\sy*(-0.0769)})
	--({\sx*(0.2100)},{\sy*(0.5385)})
	--({\sx*(0.2200)},{\sy*(0.2308)})
	--({\sx*(0.2300)},{\sy*(0.2308)})
	--({\sx*(0.2400)},{\sy*(0.9231)})
	--({\sx*(0.2500)},{\sy*(0.0769)})
	--({\sx*(0.2600)},{\sy*(0.6923)})
	--({\sx*(0.2700)},{\sy*(0.1538)})
	--({\sx*(0.2800)},{\sy*(0.6923)})
	--({\sx*(0.2900)},{\sy*(0.2308)})
	--({\sx*(0.3000)},{\sy*(0.7692)})
	--({\sx*(0.3100)},{\sy*(1.0000)})
	--({\sx*(0.3200)},{\sy*(0.1538)})
	--({\sx*(0.3300)},{\sy*(0.1538)})
	--({\sx*(0.3400)},{\sy*(0.6923)})
	--({\sx*(0.3500)},{\sy*(0.1538)})
	--({\sx*(0.3600)},{\sy*(-0.0769)})
	--({\sx*(0.3700)},{\sy*(-0.3846)})
	--({\sx*(0.3800)},{\sy*(0.3846)})
	--({\sx*(0.3900)},{\sy*(0.1538)})
	--({\sx*(0.4000)},{\sy*(-0.2308)})
	--({\sx*(0.4100)},{\sy*(-0.5385)})
	--({\sx*(0.4200)},{\sy*(0.2308)})
	--({\sx*(0.4300)},{\sy*(-0.3846)})
	--({\sx*(0.4400)},{\sy*(-0.4615)})
	--({\sx*(0.4500)},{\sy*(-0.4615)})
	--({\sx*(0.4600)},{\sy*(-0.2308)})
	--({\sx*(0.4700)},{\sy*(-0.2308)})
	--({\sx*(0.4800)},{\sy*(-0.4615)})
	--({\sx*(0.4900)},{\sy*(-0.6923)})
	--({\sx*(0.5000)},{\sy*(-0.5385)})
	--({\sx*(0.5100)},{\sy*(-0.1538)})
	--({\sx*(0.5200)},{\sy*(-0.2308)})
	--({\sx*(0.5300)},{\sy*(-0.6923)})
	--({\sx*(0.5400)},{\sy*(-0.3077)})
	--({\sx*(0.5500)},{\sy*(-0.2308)})
	--({\sx*(0.5600)},{\sy*(-0.3077)})
	--({\sx*(0.5700)},{\sy*(-0.6154)})
	--({\sx*(0.5800)},{\sy*(-0.9231)})
	--({\sx*(0.5900)},{\sy*(0.0769)})
	--({\sx*(0.6000)},{\sy*(-0.3846)})
	--({\sx*(0.6100)},{\sy*(-0.4615)})
	--({\sx*(0.6200)},{\sy*(-1.0000)})
	--({\sx*(0.6300)},{\sy*(-0.0769)})
	--({\sx*(0.6400)},{\sy*(0.0000)})
	--({\sx*(0.6500)},{\sy*(0.0000)})
	--({\sx*(0.6600)},{\sy*(-0.8462)})
	--({\sx*(0.6700)},{\sy*(0.1538)})
	--({\sx*(0.6800)},{\sy*(-0.6923)})
	--({\sx*(0.6900)},{\sy*(-0.0769)})
	--({\sx*(0.7000)},{\sy*(-0.7692)})
	--({\sx*(0.7100)},{\sy*(0.0000)})
	--({\sx*(0.7200)},{\sy*(0.0000)})
	--({\sx*(0.7300)},{\sy*(-0.2308)})
	--({\sx*(0.7400)},{\sy*(0.0769)})
	--({\sx*(0.7500)},{\sy*(-0.2308)})
	--({\sx*(0.7600)},{\sy*(0.2308)})
	--({\sx*(0.7700)},{\sy*(0.0769)})
	--({\sx*(0.7800)},{\sy*(-0.2308)})
	--({\sx*(0.7900)},{\sy*(-0.0769)})
	--({\sx*(0.8000)},{\sy*(0.2308)})
	--({\sx*(0.8100)},{\sy*(0.2308)})
	--({\sx*(0.8200)},{\sy*(-0.2308)})
	--({\sx*(0.8300)},{\sy*(0.0000)})
	--({\sx*(0.8400)},{\sy*(0.4615)})
	--({\sx*(0.8500)},{\sy*(-0.2308)})
	--({\sx*(0.8600)},{\sy*(-0.1538)})
	--({\sx*(0.8700)},{\sy*(-0.3077)})
	--({\sx*(0.8800)},{\sy*(0.2308)})
	--({\sx*(0.8900)},{\sy*(0.1538)})
	--({\sx*(0.9000)},{\sy*(0.4615)})
	--({\sx*(0.9100)},{\sy*(-0.3846)})
	--({\sx*(0.9200)},{\sy*(0.4615)})
	--({\sx*(0.9300)},{\sy*(0.1538)})
	--({\sx*(0.9400)},{\sy*(0.3846)})
	--({\sx*(0.9500)},{\sy*(-0.3077)})
	--({\sx*(0.9600)},{\sy*(0.3846)})
	--({\sx*(0.9700)},{\sy*(0.3846)})
	--({\sx*(0.9800)},{\sy*(0.5000)})
	--({\sx*(0.9900)},{\sy*(0.3462)})
	--({\sx*(1.0000)},{\sy*(0.5769)})
	--({\sx*(1.0100)},{\sy*(0.3846)})
	--({\sx*(1.0200)},{\sy*(0.5385)})
	--({\sx*(1.0300)},{\sy*(-0.0385)})
	--({\sx*(1.0400)},{\sy*(0.3077)})
	--({\sx*(1.0500)},{\sy*(0.5385)})
	--({\sx*(1.0600)},{\sy*(0.3462)})
	--({\sx*(1.0700)},{\sy*(-0.1538)})
	--({\sx*(1.0800)},{\sy*(0.1923)})
	--({\sx*(1.0900)},{\sy*(0.4231)})
	--({\sx*(1.1000)},{\sy*(-0.0769)})
	--({\sx*(1.1100)},{\sy*(0.1538)})
	--({\sx*(1.1200)},{\sy*(0.2308)})
	--({\sx*(1.1300)},{\sy*(0.2692)})
	--({\sx*(1.1400)},{\sy*(0.1538)})
	--({\sx*(1.1500)},{\sy*(-0.0769)})
	--({\sx*(1.1600)},{\sy*(-0.1538)})
	--({\sx*(1.1700)},{\sy*(0.0769)})
	--({\sx*(1.1800)},{\sy*(0.2692)})
	--({\sx*(1.1900)},{\sy*(-0.2308)})
	--({\sx*(1.2000)},{\sy*(-0.1923)})
	--({\sx*(1.2100)},{\sy*(0.0769)})
	--({\sx*(1.2200)},{\sy*(0.1154)})
	--({\sx*(1.2300)},{\sy*(-0.0385)})
	--({\sx*(1.2400)},{\sy*(-0.1154)})
	--({\sx*(1.2500)},{\sy*(-0.0769)})
	--({\sx*(1.2600)},{\sy*(-0.0385)})
	--({\sx*(1.2700)},{\sy*(-0.1538)})
	--({\sx*(1.2800)},{\sy*(-0.0385)})
	--({\sx*(1.2900)},{\sy*(0.0769)})
	--({\sx*(1.3000)},{\sy*(0.0385)})
	--({\sx*(1.3100)},{\sy*(0.0000)})
	--({\sx*(1.3200)},{\sy*(-0.1923)})
	--({\sx*(1.3300)},{\sy*(-0.0385)})
	--({\sx*(1.3400)},{\sy*(0.0769)})
	--({\sx*(1.3500)},{\sy*(-0.0385)})
	--({\sx*(1.3600)},{\sy*(-0.2692)})
	--({\sx*(1.3700)},{\sy*(0.1154)})
	--({\sx*(1.3800)},{\sy*(0.1154)})
	--({\sx*(1.3900)},{\sy*(0.0385)})
	--({\sx*(1.4000)},{\sy*(-0.0385)})
	--({\sx*(1.4100)},{\sy*(-0.1923)})
	--({\sx*(1.4200)},{\sy*(0.0769)})
	--({\sx*(1.4300)},{\sy*(0.2308)})
	--({\sx*(1.4400)},{\sy*(0.0000)})
	--({\sx*(1.4500)},{\sy*(-0.2308)})
	--({\sx*(1.4600)},{\sy*(-0.1154)})
	--({\sx*(1.4700)},{\sy*(0.1154)})
	--({\sx*(1.4800)},{\sy*(0.0769)})
	--({\sx*(1.4900)},{\sy*(-0.0385)})
	--({\sx*(1.5000)},{\sy*(0.1538)})
	--({\sx*(1.5100)},{\sy*(0.2308)})
	--({\sx*(1.5200)},{\sy*(0.1154)})
	--({\sx*(1.5300)},{\sy*(-0.0192)})
	--({\sx*(1.5400)},{\sy*(0.0192)})
	--({\sx*(1.5500)},{\sy*(0.1731)})
	--({\sx*(1.5600)},{\sy*(0.0577)})
	--({\sx*(1.5700)},{\sy*(-0.0769)})
	--({\sx*(1.5800)},{\sy*(0.0769)})
	--({\sx*(1.5900)},{\sy*(0.0577)})
	--({\sx*(1.6000)},{\sy*(0.0192)})
	--({\sx*(1.6100)},{\sy*(-0.1346)})
	--({\sx*(1.6200)},{\sy*(0.0385)})
	--({\sx*(1.6300)},{\sy*(0.0577)})
	--({\sx*(1.6400)},{\sy*(0.0962)})
	--({\sx*(1.6500)},{\sy*(-0.1923)})
	--({\sx*(1.6600)},{\sy*(-0.1346)})
	--({\sx*(1.6700)},{\sy*(-0.2115)})
	--({\sx*(1.6800)},{\sy*(0.0000)})
	--({\sx*(1.6900)},{\sy*(-0.0385)})
	--({\sx*(1.7000)},{\sy*(-0.0577)})
	--({\sx*(1.7100)},{\sy*(-0.0769)})
	--({\sx*(1.7200)},{\sy*(-0.0192)})
	--({\sx*(1.7300)},{\sy*(-0.0769)})
	--({\sx*(1.7400)},{\sy*(-0.0577)})
	--({\sx*(1.7500)},{\sy*(-0.0192)})
	--({\sx*(1.7600)},{\sy*(0.1538)})
	--({\sx*(1.7700)},{\sy*(-0.0192)})
	--({\sx*(1.7800)},{\sy*(-0.1346)})
	--({\sx*(1.7900)},{\sy*(0.0385)})
	--({\sx*(1.8000)},{\sy*(0.0385)})
	--({\sx*(1.8100)},{\sy*(-0.0192)})
	--({\sx*(1.8200)},{\sy*(-0.1731)})
	--({\sx*(1.8300)},{\sy*(-0.0385)})
	--({\sx*(1.8400)},{\sy*(0.0962)})
	--({\sx*(1.8500)},{\sy*(0.0000)})
	--({\sx*(1.8600)},{\sy*(-0.0769)})
	--({\sx*(1.8700)},{\sy*(-0.0192)})
	--({\sx*(1.8800)},{\sy*(0.0769)})
	--({\sx*(1.8900)},{\sy*(0.0385)})
	--({\sx*(1.9000)},{\sy*(-0.1154)})
	--({\sx*(1.9100)},{\sy*(0.0577)})
	--({\sx*(1.9200)},{\sy*(0.0192)})
	--({\sx*(1.9300)},{\sy*(0.0962)})
	--({\sx*(1.9400)},{\sy*(0.0962)})
	--({\sx*(1.9500)},{\sy*(-0.0673)})
	--({\sx*(1.9600)},{\sy*(-0.0288)})
	--({\sx*(1.9700)},{\sy*(0.0865)})
	--({\sx*(1.9800)},{\sy*(0.0673)})
	--({\sx*(1.9900)},{\sy*(0.0000)})
	--({\sx*(2.0000)},{\sy*(0.0096)})
	--({\sx*(2.0100)},{\sy*(0.0192)})
	--({\sx*(2.0200)},{\sy*(0.0481)})
	--({\sx*(2.0300)},{\sy*(0.0000)})
	--({\sx*(2.0400)},{\sy*(-0.0481)})
	--({\sx*(2.0500)},{\sy*(0.0577)})
	--({\sx*(2.0600)},{\sy*(-0.0288)})
	--({\sx*(2.0700)},{\sy*(0.0096)})
	--({\sx*(2.0800)},{\sy*(-0.0288)})
	--({\sx*(2.0900)},{\sy*(0.0385)})
	--({\sx*(2.1000)},{\sy*(-0.0192)})
	--({\sx*(2.1100)},{\sy*(0.0000)})
	--({\sx*(2.1200)},{\sy*(0.0577)})
	--({\sx*(2.1300)},{\sy*(-0.1154)})
	--({\sx*(2.1400)},{\sy*(-0.0865)})
	--({\sx*(2.1500)},{\sy*(-0.0769)})
	--({\sx*(2.1600)},{\sy*(0.0481)})
	--({\sx*(2.1700)},{\sy*(-0.0577)})
	--({\sx*(2.1800)},{\sy*(-0.0865)})
	--({\sx*(2.1900)},{\sy*(-0.0192)})
	--({\sx*(2.2000)},{\sy*(-0.0096)})
	--({\sx*(2.2100)},{\sy*(-0.0385)})
	--({\sx*(2.2200)},{\sy*(-0.0481)})
	--({\sx*(2.2300)},{\sy*(-0.0192)})
	--({\sx*(2.2400)},{\sy*(-0.0288)})
	--({\sx*(2.2500)},{\sy*(-0.0385)})
	--({\sx*(2.2600)},{\sy*(0.0144)})
	--({\sx*(2.2700)},{\sy*(0.0240)})
	--({\sx*(2.2800)},{\sy*(0.0240)})
	--({\sx*(2.2900)},{\sy*(0.0144)})
	--({\sx*(2.3000)},{\sy*(0.0192)})
	--({\sx*(2.3100)},{\sy*(0.0337)})
	--({\sx*(2.3200)},{\sy*(0.0337)})
	--({\sx*(2.3300)},{\sy*(0.0048)})
	--({\sx*(2.3400)},{\sy*(-0.0288)})
	--({\sx*(2.3500)},{\sy*(0.0048)})
	--({\sx*(2.3600)},{\sy*(-0.0192)})
	--({\sx*(2.3700)},{\sy*(-0.0096)})
	--({\sx*(2.3800)},{\sy*(0.0385)})
	--({\sx*(2.3900)},{\sy*(-0.0288)})
	--({\sx*(2.4000)},{\sy*(0.0240)})
	--({\sx*(2.4100)},{\sy*(0.0385)})
	--({\sx*(2.4200)},{\sy*(-0.0240)})
	--({\sx*(2.4300)},{\sy*(-0.0144)})
	--({\sx*(2.4400)},{\sy*(0.0192)})
	--({\sx*(2.4500)},{\sy*(0.0337)})
	--({\sx*(2.4600)},{\sy*(0.0144)})
	--({\sx*(2.4700)},{\sy*(0.0288)})
	--({\sx*(2.4800)},{\sy*(-0.0096)})
	--({\sx*(2.4900)},{\sy*(0.0048)})
	--({\sx*(2.5000)},{\sy*(0.0000)})
	--({\sx*(2.5100)},{\sy*(-0.0337)})
	--({\sx*(2.5200)},{\sy*(-0.0096)})
	--({\sx*(2.5300)},{\sy*(-0.0096)})
	--({\sx*(2.5400)},{\sy*(-0.0096)})
	--({\sx*(2.5500)},{\sy*(0.0024)})
	--({\sx*(2.5600)},{\sy*(-0.0096)})
	--({\sx*(2.5700)},{\sy*(-0.0361)})
	--({\sx*(2.5800)},{\sy*(-0.0457)})
	--({\sx*(2.5900)},{\sy*(-0.0337)})
	--({\sx*(2.6000)},{\sy*(0.0240)})
	--({\sx*(2.6100)},{\sy*(0.0312)})
	--({\sx*(2.6200)},{\sy*(0.0048)})
	--({\sx*(2.6300)},{\sy*(-0.0385)})
	--({\sx*(2.6400)},{\sy*(-0.0168)})
	--({\sx*(2.6500)},{\sy*(-0.0120)})
	--({\sx*(2.6600)},{\sy*(-0.0337)})
	--({\sx*(2.6700)},{\sy*(0.0024)})
	--({\sx*(2.6800)},{\sy*(0.0096)})
	--({\sx*(2.6900)},{\sy*(0.0000)})
	--({\sx*(2.7000)},{\sy*(0.0024)})
	--({\sx*(2.7100)},{\sy*(-0.0024)})
	--({\sx*(2.7200)},{\sy*(-0.0048)})
	--({\sx*(2.7300)},{\sy*(-0.0072)})
	--({\sx*(2.7400)},{\sy*(-0.0048)})
	--({\sx*(2.7500)},{\sy*(0.0024)})
	--({\sx*(2.7600)},{\sy*(-0.0072)})
	--({\sx*(2.7700)},{\sy*(0.0144)})
	--({\sx*(2.7800)},{\sy*(0.0192)})
	--({\sx*(2.7900)},{\sy*(0.0288)})
	--({\sx*(2.8000)},{\sy*(0.0144)})
	--({\sx*(2.8100)},{\sy*(0.0048)})
	--({\sx*(2.8200)},{\sy*(0.0168)})
	--({\sx*(2.8300)},{\sy*(-0.0048)})
	--({\sx*(2.8400)},{\sy*(-0.0132)})
	--({\sx*(2.8500)},{\sy*(0.0228)})
	--({\sx*(2.8600)},{\sy*(-0.0337)})
	--({\sx*(2.8700)},{\sy*(0.0180)})
	--({\sx*(2.8800)},{\sy*(0.0288)})
	--({\sx*(2.8900)},{\sy*(0.0036)})
	--({\sx*(2.9000)},{\sy*(-0.0060)})
	--({\sx*(2.9100)},{\sy*(0.0072)})
	--({\sx*(2.9200)},{\sy*(0.0132)})
	--({\sx*(2.9300)},{\sy*(0.0012)})
	--({\sx*(2.9400)},{\sy*(-0.0084)})
	--({\sx*(2.9500)},{\sy*(0.0024)})
	--({\sx*(2.9600)},{\sy*(-0.0084)})
	--({\sx*(2.9700)},{\sy*(0.0144)})
	--({\sx*(2.9800)},{\sy*(0.0012)})
	--({\sx*(2.9900)},{\sy*(0.0036)})
	--({\sx*(3.0000)},{\sy*(0.0192)})
	--({\sx*(3.0100)},{\sy*(-0.0096)})
	--({\sx*(3.0200)},{\sy*(-0.0024)})
	--({\sx*(3.0300)},{\sy*(-0.0240)})
	--({\sx*(3.0400)},{\sy*(0.0012)})
	--({\sx*(3.0500)},{\sy*(0.0144)})
	--({\sx*(3.0600)},{\sy*(-0.0312)})
	--({\sx*(3.0700)},{\sy*(-0.0024)})
	--({\sx*(3.0800)},{\sy*(-0.0210)})
	--({\sx*(3.0900)},{\sy*(-0.0012)})
	--({\sx*(3.1000)},{\sy*(0.0102)})
	--({\sx*(3.1100)},{\sy*(-0.0174)})
	--({\sx*(3.1200)},{\sy*(-0.0096)})
	--({\sx*(3.1300)},{\sy*(-0.0048)})
	--({\sx*(3.1400)},{\sy*(-0.0168)})
	--({\sx*(3.1500)},{\sy*(-0.0102)})
	--({\sx*(3.1600)},{\sy*(0.0078)})
	--({\sx*(3.1700)},{\sy*(0.0030)})
	--({\sx*(3.1800)},{\sy*(-0.0018)})
	--({\sx*(3.1900)},{\sy*(0.0036)})
	--({\sx*(3.2000)},{\sy*(0.0042)})
	--({\sx*(3.2100)},{\sy*(-0.0054)})
	--({\sx*(3.2200)},{\sy*(0.0078)})
	--({\sx*(3.2300)},{\sy*(-0.0048)})
	--({\sx*(3.2400)},{\sy*(0.0343)})
	--({\sx*(3.2500)},{\sy*(-0.0030)})
	--({\sx*(3.2600)},{\sy*(0.0222)})
	--({\sx*(3.2700)},{\sy*(-0.0048)})
	--({\sx*(3.2800)},{\sy*(-0.0084)})
	--({\sx*(3.2900)},{\sy*(0.0367)})
	--({\sx*(3.3000)},{\sy*(0.0153)})
	--({\sx*(3.3100)},{\sy*(0.0328)})
	--({\sx*(3.3200)},{\sy*(0.0228)})
	--({\sx*(3.3300)},{\sy*(-0.0114)})
	--({\sx*(3.3400)},{\sy*(0.0015)})
	--({\sx*(3.3500)},{\sy*(0.0048)})
	--({\sx*(3.3600)},{\sy*(0.0075)})
	--({\sx*(3.3700)},{\sy*(0.0006)})
	--({\sx*(3.3800)},{\sy*(0.0042)})
	--({\sx*(3.3900)},{\sy*(-0.0006)})
	--({\sx*(3.4000)},{\sy*(-0.0003)})
	--({\sx*(3.4100)},{\sy*(-0.0015)})
	--({\sx*(3.4200)},{\sy*(-0.0039)})
	--({\sx*(3.4300)},{\sy*(0.0075)})
	--({\sx*(3.4400)},{\sy*(0.0063)})
	--({\sx*(3.4500)},{\sy*(-0.0027)})
	--({\sx*(3.4600)},{\sy*(0.0039)})
	--({\sx*(3.4700)},{\sy*(0.0005)})
	--({\sx*(3.4800)},{\sy*(0.0090)})
	--({\sx*(3.4900)},{\sy*(-0.0251)})
	--({\sx*(3.5000)},{\sy*(0.0032)})
	--({\sx*(3.5100)},{\sy*(-0.0312)})
	--({\sx*(3.5200)},{\sy*(-0.0267)})
	--({\sx*(3.5300)},{\sy*(-0.0340)})
	--({\sx*(3.5400)},{\sy*(0.0041)})
	--({\sx*(3.5500)},{\sy*(0.0078)})
	--({\sx*(3.5600)},{\sy*(-0.0042)})
	--({\sx*(3.5700)},{\sy*(-0.0084)})
	--({\sx*(3.5800)},{\sy*(0.0024)})
	--({\sx*(3.5900)},{\sy*(0.0035)})
	--({\sx*(3.6000)},{\sy*(-0.0026)})
	--({\sx*(3.6100)},{\sy*(-0.0012)})
	--({\sx*(3.6200)},{\sy*(0.0023)})
	--({\sx*(3.6300)},{\sy*(0.0008)})
	--({\sx*(3.6400)},{\sy*(-0.0011)})
	--({\sx*(3.6500)},{\sy*(0.0036)})
	--({\sx*(3.6600)},{\sy*(-0.0066)})
	--({\sx*(3.6700)},{\sy*(0.0052)})
	--({\sx*(3.6800)},{\sy*(-0.0077)})
	--({\sx*(3.6900)},{\sy*(0.0092)})
	--({\sx*(3.7000)},{\sy*(0.0029)})
	--({\sx*(3.7100)},{\sy*(0.0140)})
	--({\sx*(3.7200)},{\sy*(-0.0152)})
	--({\sx*(3.7300)},{\sy*(-0.0056)})
	--({\sx*(3.7400)},{\sy*(0.0094)})
	--({\sx*(3.7500)},{\sy*(0.0230)})
	--({\sx*(3.7600)},{\sy*(0.0135)})
	--({\sx*(3.7700)},{\sy*(0.0056)})
	--({\sx*(3.7800)},{\sy*(0.0059)})
	--({\sx*(3.7900)},{\sy*(-0.0069)})
	--({\sx*(3.8000)},{\sy*(0.0021)})
	--({\sx*(3.8100)},{\sy*(0.0011)})
	--({\sx*(3.8200)},{\sy*(-0.0001)})
	--({\sx*(3.8300)},{\sy*(-0.0001)})
	--({\sx*(3.8400)},{\sy*(0.0060)})
	--({\sx*(3.8500)},{\sy*(0.0000)})
	--({\sx*(3.8600)},{\sy*(-0.0025)})
	--({\sx*(3.8700)},{\sy*(0.0092)})
	--({\sx*(3.8800)},{\sy*(-0.0063)})
	--({\sx*(3.8900)},{\sy*(0.0042)})
	--({\sx*(3.9000)},{\sy*(-0.0017)})
	--({\sx*(3.9100)},{\sy*(0.0029)})
	--({\sx*(3.9200)},{\sy*(-0.0119)})
	--({\sx*(3.9300)},{\sy*(-0.0172)})
	--({\sx*(3.9400)},{\sy*(-0.0104)})
	--({\sx*(3.9500)},{\sy*(0.0009)})
	--({\sx*(3.9600)},{\sy*(-0.0123)})
	--({\sx*(3.9700)},{\sy*(0.0013)})
	--({\sx*(3.9800)},{\sy*(-0.0083)})
	--({\sx*(3.9900)},{\sy*(0.0005)})
	--({\sx*(4.0000)},{\sy*(0.0013)})
	--({\sx*(4.0100)},{\sy*(0.0004)})
	--({\sx*(4.0200)},{\sy*(-0.0002)})
	--({\sx*(4.0300)},{\sy*(0.0091)})
	--({\sx*(4.0400)},{\sy*(0.0065)})
	--({\sx*(4.0500)},{\sy*(0.0147)})
	--({\sx*(4.0600)},{\sy*(0.0082)})
	--({\sx*(4.0700)},{\sy*(-0.0028)})
	--({\sx*(4.0800)},{\sy*(0.0180)})
	--({\sx*(4.0900)},{\sy*(-0.0032)})
	--({\sx*(4.1000)},{\sy*(0.0038)})
	--({\sx*(4.1100)},{\sy*(0.0145)})
	--({\sx*(4.1200)},{\sy*(-0.0057)})
	--({\sx*(4.1300)},{\sy*(0.0064)})
	--({\sx*(4.1400)},{\sy*(0.0086)})
	--({\sx*(4.1500)},{\sy*(0.0081)})
	--({\sx*(4.1600)},{\sy*(-0.0011)})
	--({\sx*(4.1700)},{\sy*(0.0011)})
	--({\sx*(4.1800)},{\sy*(-0.0002)})
	--({\sx*(4.1900)},{\sy*(-0.0011)})
	--({\sx*(4.2000)},{\sy*(-0.0051)})
	--({\sx*(4.2100)},{\sy*(-0.0044)})
	--({\sx*(4.2200)},{\sy*(-0.0019)})
	--({\sx*(4.2300)},{\sy*(-0.0097)})
	--({\sx*(4.2400)},{\sy*(-0.0005)})
	--({\sx*(4.2500)},{\sy*(-0.0086)})
	--({\sx*(4.2600)},{\sy*(-0.0174)})
	--({\sx*(4.2700)},{\sy*(-0.0089)})
	--({\sx*(4.2800)},{\sy*(-0.0065)})
	--({\sx*(4.2900)},{\sy*(-0.0073)})
	--({\sx*(4.3000)},{\sy*(0.0011)})
	--({\sx*(4.3100)},{\sy*(-0.0061)})
	--({\sx*(4.3200)},{\sy*(0.0013)})
	--({\sx*(4.3300)},{\sy*(-0.0056)})
	--({\sx*(4.3400)},{\sy*(-0.0010)})
	--({\sx*(4.3500)},{\sy*(0.0000)})
	--({\sx*(4.3600)},{\sy*(0.0003)})
	--({\sx*(4.3700)},{\sy*(0.0075)})
	--({\sx*(4.3800)},{\sy*(0.0044)})
	--({\sx*(4.3900)},{\sy*(-0.0046)})
	--({\sx*(4.4000)},{\sy*(-0.0036)})
	--({\sx*(4.4100)},{\sy*(0.0024)})
	--({\sx*(4.4200)},{\sy*(0.0047)})
	--({\sx*(4.4300)},{\sy*(0.0089)})
	--({\sx*(4.4400)},{\sy*(0.0084)})
	--({\sx*(4.4500)},{\sy*(0.0041)})
	--({\sx*(4.4600)},{\sy*(-0.0027)})
	--({\sx*(4.4700)},{\sy*(0.0018)})
	--({\sx*(4.4800)},{\sy*(0.0058)})
	--({\sx*(4.4900)},{\sy*(0.0003)})
	--({\sx*(4.5000)},{\sy*(0.0001)})
	--({\sx*(4.5100)},{\sy*(-0.0031)})
	--({\sx*(4.5200)},{\sy*(0.0025)})
	--({\sx*(4.5300)},{\sy*(-0.0093)})
	--({\sx*(4.5400)},{\sy*(-0.0073)})
	--({\sx*(4.5500)},{\sy*(0.0030)})
	--({\sx*(4.5600)},{\sy*(-0.0004)})
	--({\sx*(4.5700)},{\sy*(0.0008)})
	--({\sx*(4.5800)},{\sy*(0.0004)})
	--({\sx*(4.5900)},{\sy*(-0.0078)})
	--({\sx*(4.6000)},{\sy*(-0.0078)})
	--({\sx*(4.6100)},{\sy*(-0.0002)})
	--({\sx*(4.6200)},{\sy*(-0.0006)})
	--({\sx*(4.6300)},{\sy*(0.0008)})
	--({\sx*(4.6400)},{\sy*(0.0003)})
	--({\sx*(4.6500)},{\sy*(0.0022)})
	--({\sx*(4.6600)},{\sy*(0.0072)})
	--({\sx*(4.6700)},{\sy*(0.0056)})
	--({\sx*(4.6800)},{\sy*(0.0117)})
	--({\sx*(4.6900)},{\sy*(0.0095)})
	--({\sx*(4.7000)},{\sy*(0.0060)})
	--({\sx*(4.7100)},{\sy*(-0.0017)})
	--({\sx*(4.7200)},{\sy*(0.0034)})
	--({\sx*(4.7300)},{\sy*(0.0003)})
	--({\sx*(4.7400)},{\sy*(-0.0001)})
	--({\sx*(4.7500)},{\sy*(0.0001)})
	--({\sx*(4.7600)},{\sy*(-0.0020)})
	--({\sx*(4.7700)},{\sy*(-0.0008)})
	--({\sx*(4.7800)},{\sy*(-0.0097)})
	--({\sx*(4.7900)},{\sy*(0.0052)})
	--({\sx*(4.8000)},{\sy*(0.0011)})
	--({\sx*(4.8100)},{\sy*(-0.0033)})
	--({\sx*(4.8200)},{\sy*(0.0018)})
	--({\sx*(4.8300)},{\sy*(-0.0000)})
	--({\sx*(4.8400)},{\sy*(-0.0007)})
	--({\sx*(4.8500)},{\sy*(0.0043)})
	--({\sx*(4.8600)},{\sy*(-0.0081)})
	--({\sx*(4.8700)},{\sy*(-0.0005)})
	--({\sx*(4.8800)},{\sy*(-0.0016)})
	--({\sx*(4.8900)},{\sy*(-0.0008)})
	--({\sx*(4.9000)},{\sy*(0.0002)})
	--({\sx*(4.9100)},{\sy*(-0.0005)})
	--({\sx*(4.9200)},{\sy*(0.0027)})
	--({\sx*(4.9300)},{\sy*(-0.0015)})
	--({\sx*(4.9400)},{\sy*(0.0005)})
	--({\sx*(4.9500)},{\sy*(0.0018)})
	--({\sx*(4.9600)},{\sy*(-0.0003)})
	--({\sx*(4.9700)},{\sy*(0.0018)})
	--({\sx*(4.9800)},{\sy*(-0.0014)})
	--({\sx*(4.9900)},{\sy*(0.0001)})
	--({\sx*(5.0000)},{\sy*(0.0000)});
}
\def\relfehlerq{
\draw[color=blue,line width=1.4pt,line join=round] ({\sx*(0.000)},{\sy*(0.0000)})
	--({\sx*(0.0100)},{\sy*(0.0000)})
	--({\sx*(0.0200)},{\sy*(0.0000)})
	--({\sx*(0.0300)},{\sy*(-0.0000)})
	--({\sx*(0.0400)},{\sy*(-0.0000)})
	--({\sx*(0.0500)},{\sy*(0.0000)})
	--({\sx*(0.0600)},{\sy*(0.0000)})
	--({\sx*(0.0700)},{\sy*(-0.0000)})
	--({\sx*(0.0800)},{\sy*(-0.0000)})
	--({\sx*(0.0900)},{\sy*(0.0000)})
	--({\sx*(0.1000)},{\sy*(0.0000)})
	--({\sx*(0.1100)},{\sy*(-0.0000)})
	--({\sx*(0.1200)},{\sy*(-0.0000)})
	--({\sx*(0.1300)},{\sy*(0.0000)})
	--({\sx*(0.1400)},{\sy*(0.0000)})
	--({\sx*(0.1500)},{\sy*(-0.0000)})
	--({\sx*(0.1600)},{\sy*(-0.0000)})
	--({\sx*(0.1700)},{\sy*(0.0000)})
	--({\sx*(0.1800)},{\sy*(0.0000)})
	--({\sx*(0.1900)},{\sy*(0.0000)})
	--({\sx*(0.2000)},{\sy*(-0.0000)})
	--({\sx*(0.2100)},{\sy*(0.0000)})
	--({\sx*(0.2200)},{\sy*(0.0000)})
	--({\sx*(0.2300)},{\sy*(0.0000)})
	--({\sx*(0.2400)},{\sy*(0.0000)})
	--({\sx*(0.2500)},{\sy*(0.0000)})
	--({\sx*(0.2600)},{\sy*(0.0000)})
	--({\sx*(0.2700)},{\sy*(0.0000)})
	--({\sx*(0.2800)},{\sy*(0.0000)})
	--({\sx*(0.2900)},{\sy*(0.0000)})
	--({\sx*(0.3000)},{\sy*(0.0000)})
	--({\sx*(0.3100)},{\sy*(0.0000)})
	--({\sx*(0.3200)},{\sy*(0.0000)})
	--({\sx*(0.3300)},{\sy*(0.0000)})
	--({\sx*(0.3400)},{\sy*(0.0000)})
	--({\sx*(0.3500)},{\sy*(0.0000)})
	--({\sx*(0.3600)},{\sy*(-0.0000)})
	--({\sx*(0.3700)},{\sy*(-0.0000)})
	--({\sx*(0.3800)},{\sy*(0.0000)})
	--({\sx*(0.3900)},{\sy*(0.0000)})
	--({\sx*(0.4000)},{\sy*(-0.0000)})
	--({\sx*(0.4100)},{\sy*(-0.0000)})
	--({\sx*(0.4200)},{\sy*(0.0000)})
	--({\sx*(0.4300)},{\sy*(-0.0000)})
	--({\sx*(0.4400)},{\sy*(-0.0000)})
	--({\sx*(0.4500)},{\sy*(-0.0000)})
	--({\sx*(0.4600)},{\sy*(-0.0000)})
	--({\sx*(0.4700)},{\sy*(-0.0000)})
	--({\sx*(0.4800)},{\sy*(-0.0000)})
	--({\sx*(0.4900)},{\sy*(-0.0000)})
	--({\sx*(0.5000)},{\sy*(-0.0000)})
	--({\sx*(0.5100)},{\sy*(-0.0000)})
	--({\sx*(0.5200)},{\sy*(-0.0000)})
	--({\sx*(0.5300)},{\sy*(-0.0000)})
	--({\sx*(0.5400)},{\sy*(-0.0000)})
	--({\sx*(0.5500)},{\sy*(-0.0000)})
	--({\sx*(0.5600)},{\sy*(-0.0000)})
	--({\sx*(0.5700)},{\sy*(-0.0000)})
	--({\sx*(0.5800)},{\sy*(-0.0000)})
	--({\sx*(0.5900)},{\sy*(0.0000)})
	--({\sx*(0.6000)},{\sy*(-0.0000)})
	--({\sx*(0.6100)},{\sy*(-0.0000)})
	--({\sx*(0.6200)},{\sy*(-0.0000)})
	--({\sx*(0.6300)},{\sy*(-0.0000)})
	--({\sx*(0.6400)},{\sy*(0.0000)})
	--({\sx*(0.6500)},{\sy*(0.0000)})
	--({\sx*(0.6600)},{\sy*(-0.0000)})
	--({\sx*(0.6700)},{\sy*(0.0000)})
	--({\sx*(0.6800)},{\sy*(-0.0000)})
	--({\sx*(0.6900)},{\sy*(-0.0000)})
	--({\sx*(0.7000)},{\sy*(-0.0000)})
	--({\sx*(0.7100)},{\sy*(0.0000)})
	--({\sx*(0.7200)},{\sy*(0.0000)})
	--({\sx*(0.7300)},{\sy*(-0.0000)})
	--({\sx*(0.7400)},{\sy*(0.0000)})
	--({\sx*(0.7500)},{\sy*(-0.0000)})
	--({\sx*(0.7600)},{\sy*(0.0000)})
	--({\sx*(0.7700)},{\sy*(0.0000)})
	--({\sx*(0.7800)},{\sy*(-0.0000)})
	--({\sx*(0.7900)},{\sy*(-0.0000)})
	--({\sx*(0.8000)},{\sy*(0.0000)})
	--({\sx*(0.8100)},{\sy*(0.0000)})
	--({\sx*(0.8200)},{\sy*(-0.0000)})
	--({\sx*(0.8300)},{\sy*(0.0000)})
	--({\sx*(0.8400)},{\sy*(0.0000)})
	--({\sx*(0.8500)},{\sy*(-0.0000)})
	--({\sx*(0.8600)},{\sy*(-0.0000)})
	--({\sx*(0.8700)},{\sy*(-0.0000)})
	--({\sx*(0.8800)},{\sy*(0.0000)})
	--({\sx*(0.8900)},{\sy*(0.0000)})
	--({\sx*(0.9000)},{\sy*(0.0000)})
	--({\sx*(0.9100)},{\sy*(-0.0000)})
	--({\sx*(0.9200)},{\sy*(0.0000)})
	--({\sx*(0.9300)},{\sy*(0.0000)})
	--({\sx*(0.9400)},{\sy*(0.0000)})
	--({\sx*(0.9500)},{\sy*(-0.0000)})
	--({\sx*(0.9600)},{\sy*(0.0000)})
	--({\sx*(0.9700)},{\sy*(0.0000)})
	--({\sx*(0.9800)},{\sy*(0.0000)})
	--({\sx*(0.9900)},{\sy*(0.0000)})
	--({\sx*(1.0000)},{\sy*(0.0000)})
	--({\sx*(1.0100)},{\sy*(0.0000)})
	--({\sx*(1.0200)},{\sy*(0.0000)})
	--({\sx*(1.0300)},{\sy*(-0.0000)})
	--({\sx*(1.0400)},{\sy*(0.0000)})
	--({\sx*(1.0500)},{\sy*(0.0000)})
	--({\sx*(1.0600)},{\sy*(0.0000)})
	--({\sx*(1.0700)},{\sy*(-0.0000)})
	--({\sx*(1.0800)},{\sy*(0.0000)})
	--({\sx*(1.0900)},{\sy*(0.0000)})
	--({\sx*(1.1000)},{\sy*(-0.0000)})
	--({\sx*(1.1100)},{\sy*(0.0000)})
	--({\sx*(1.1200)},{\sy*(0.0000)})
	--({\sx*(1.1300)},{\sy*(0.0000)})
	--({\sx*(1.1400)},{\sy*(0.0000)})
	--({\sx*(1.1500)},{\sy*(-0.0000)})
	--({\sx*(1.1600)},{\sy*(-0.0000)})
	--({\sx*(1.1700)},{\sy*(0.0000)})
	--({\sx*(1.1800)},{\sy*(0.0000)})
	--({\sx*(1.1900)},{\sy*(-0.0000)})
	--({\sx*(1.2000)},{\sy*(-0.0000)})
	--({\sx*(1.2100)},{\sy*(0.0000)})
	--({\sx*(1.2200)},{\sy*(0.0000)})
	--({\sx*(1.2300)},{\sy*(-0.0000)})
	--({\sx*(1.2400)},{\sy*(-0.0000)})
	--({\sx*(1.2500)},{\sy*(-0.0000)})
	--({\sx*(1.2600)},{\sy*(-0.0000)})
	--({\sx*(1.2700)},{\sy*(-0.0000)})
	--({\sx*(1.2800)},{\sy*(-0.0000)})
	--({\sx*(1.2900)},{\sy*(0.0000)})
	--({\sx*(1.3000)},{\sy*(0.0000)})
	--({\sx*(1.3100)},{\sy*(0.0000)})
	--({\sx*(1.3200)},{\sy*(-0.0000)})
	--({\sx*(1.3300)},{\sy*(-0.0000)})
	--({\sx*(1.3400)},{\sy*(0.0000)})
	--({\sx*(1.3500)},{\sy*(-0.0000)})
	--({\sx*(1.3600)},{\sy*(-0.0000)})
	--({\sx*(1.3700)},{\sy*(0.0000)})
	--({\sx*(1.3800)},{\sy*(0.0000)})
	--({\sx*(1.3900)},{\sy*(0.0000)})
	--({\sx*(1.4000)},{\sy*(-0.0000)})
	--({\sx*(1.4100)},{\sy*(-0.0000)})
	--({\sx*(1.4200)},{\sy*(0.0000)})
	--({\sx*(1.4300)},{\sy*(0.0000)})
	--({\sx*(1.4400)},{\sy*(0.0000)})
	--({\sx*(1.4500)},{\sy*(-0.0000)})
	--({\sx*(1.4600)},{\sy*(-0.0000)})
	--({\sx*(1.4700)},{\sy*(0.0000)})
	--({\sx*(1.4800)},{\sy*(0.0000)})
	--({\sx*(1.4900)},{\sy*(-0.0000)})
	--({\sx*(1.5000)},{\sy*(0.0000)})
	--({\sx*(1.5100)},{\sy*(0.0000)})
	--({\sx*(1.5200)},{\sy*(0.0000)})
	--({\sx*(1.5300)},{\sy*(-0.0000)})
	--({\sx*(1.5400)},{\sy*(0.0000)})
	--({\sx*(1.5500)},{\sy*(0.0000)})
	--({\sx*(1.5600)},{\sy*(0.0000)})
	--({\sx*(1.5700)},{\sy*(-0.0000)})
	--({\sx*(1.5800)},{\sy*(0.0000)})
	--({\sx*(1.5900)},{\sy*(0.0000)})
	--({\sx*(1.6000)},{\sy*(0.0000)})
	--({\sx*(1.6100)},{\sy*(-0.0000)})
	--({\sx*(1.6200)},{\sy*(0.0000)})
	--({\sx*(1.6300)},{\sy*(0.0000)})
	--({\sx*(1.6400)},{\sy*(0.0000)})
	--({\sx*(1.6500)},{\sy*(-0.0000)})
	--({\sx*(1.6600)},{\sy*(-0.0000)})
	--({\sx*(1.6700)},{\sy*(-0.0000)})
	--({\sx*(1.6800)},{\sy*(0.0000)})
	--({\sx*(1.6900)},{\sy*(-0.0000)})
	--({\sx*(1.7000)},{\sy*(-0.0000)})
	--({\sx*(1.7100)},{\sy*(-0.0000)})
	--({\sx*(1.7200)},{\sy*(-0.0000)})
	--({\sx*(1.7300)},{\sy*(-0.0000)})
	--({\sx*(1.7400)},{\sy*(-0.0000)})
	--({\sx*(1.7500)},{\sy*(-0.0000)})
	--({\sx*(1.7600)},{\sy*(0.0000)})
	--({\sx*(1.7700)},{\sy*(-0.0000)})
	--({\sx*(1.7800)},{\sy*(-0.0000)})
	--({\sx*(1.7900)},{\sy*(0.0000)})
	--({\sx*(1.8000)},{\sy*(0.0000)})
	--({\sx*(1.8100)},{\sy*(-0.0000)})
	--({\sx*(1.8200)},{\sy*(-0.0000)})
	--({\sx*(1.8300)},{\sy*(-0.0000)})
	--({\sx*(1.8400)},{\sy*(0.0000)})
	--({\sx*(1.8500)},{\sy*(0.0000)})
	--({\sx*(1.8600)},{\sy*(-0.0000)})
	--({\sx*(1.8700)},{\sy*(-0.0000)})
	--({\sx*(1.8800)},{\sy*(0.0000)})
	--({\sx*(1.8900)},{\sy*(0.0000)})
	--({\sx*(1.9000)},{\sy*(-0.0000)})
	--({\sx*(1.9100)},{\sy*(0.0000)})
	--({\sx*(1.9200)},{\sy*(0.0000)})
	--({\sx*(1.9300)},{\sy*(0.0000)})
	--({\sx*(1.9400)},{\sy*(0.0000)})
	--({\sx*(1.9500)},{\sy*(-0.0000)})
	--({\sx*(1.9600)},{\sy*(-0.0000)})
	--({\sx*(1.9700)},{\sy*(0.0000)})
	--({\sx*(1.9800)},{\sy*(0.0000)})
	--({\sx*(1.9900)},{\sy*(0.0000)})
	--({\sx*(2.0000)},{\sy*(0.0000)})
	--({\sx*(2.0100)},{\sy*(0.0000)})
	--({\sx*(2.0200)},{\sy*(0.0000)})
	--({\sx*(2.0300)},{\sy*(0.0000)})
	--({\sx*(2.0400)},{\sy*(-0.0000)})
	--({\sx*(2.0500)},{\sy*(0.0000)})
	--({\sx*(2.0600)},{\sy*(-0.0000)})
	--({\sx*(2.0700)},{\sy*(0.0000)})
	--({\sx*(2.0800)},{\sy*(-0.0000)})
	--({\sx*(2.0900)},{\sy*(0.0000)})
	--({\sx*(2.1000)},{\sy*(-0.0000)})
	--({\sx*(2.1100)},{\sy*(0.0000)})
	--({\sx*(2.1200)},{\sy*(0.0000)})
	--({\sx*(2.1300)},{\sy*(-0.0000)})
	--({\sx*(2.1400)},{\sy*(-0.0000)})
	--({\sx*(2.1500)},{\sy*(-0.0000)})
	--({\sx*(2.1600)},{\sy*(0.0000)})
	--({\sx*(2.1700)},{\sy*(-0.0000)})
	--({\sx*(2.1800)},{\sy*(-0.0000)})
	--({\sx*(2.1900)},{\sy*(-0.0000)})
	--({\sx*(2.2000)},{\sy*(-0.0000)})
	--({\sx*(2.2100)},{\sy*(-0.0000)})
	--({\sx*(2.2200)},{\sy*(-0.0000)})
	--({\sx*(2.2300)},{\sy*(-0.0000)})
	--({\sx*(2.2400)},{\sy*(-0.0000)})
	--({\sx*(2.2500)},{\sy*(-0.0000)})
	--({\sx*(2.2600)},{\sy*(0.0000)})
	--({\sx*(2.2700)},{\sy*(0.0000)})
	--({\sx*(2.2800)},{\sy*(0.0000)})
	--({\sx*(2.2900)},{\sy*(0.0000)})
	--({\sx*(2.3000)},{\sy*(0.0000)})
	--({\sx*(2.3100)},{\sy*(0.0000)})
	--({\sx*(2.3200)},{\sy*(0.0000)})
	--({\sx*(2.3300)},{\sy*(0.0000)})
	--({\sx*(2.3400)},{\sy*(-0.0000)})
	--({\sx*(2.3500)},{\sy*(0.0000)})
	--({\sx*(2.3600)},{\sy*(-0.0000)})
	--({\sx*(2.3700)},{\sy*(-0.0000)})
	--({\sx*(2.3800)},{\sy*(0.0000)})
	--({\sx*(2.3900)},{\sy*(-0.0000)})
	--({\sx*(2.4000)},{\sy*(0.0000)})
	--({\sx*(2.4100)},{\sy*(0.0000)})
	--({\sx*(2.4200)},{\sy*(-0.0000)})
	--({\sx*(2.4300)},{\sy*(-0.0000)})
	--({\sx*(2.4400)},{\sy*(0.0000)})
	--({\sx*(2.4500)},{\sy*(0.0000)})
	--({\sx*(2.4600)},{\sy*(0.0000)})
	--({\sx*(2.4700)},{\sy*(0.0000)})
	--({\sx*(2.4800)},{\sy*(-0.0000)})
	--({\sx*(2.4900)},{\sy*(0.0000)})
	--({\sx*(2.5000)},{\sy*(0.0000)})
	--({\sx*(2.5100)},{\sy*(-0.0000)})
	--({\sx*(2.5200)},{\sy*(-0.0000)})
	--({\sx*(2.5300)},{\sy*(-0.0000)})
	--({\sx*(2.5400)},{\sy*(-0.0000)})
	--({\sx*(2.5500)},{\sy*(0.0000)})
	--({\sx*(2.5600)},{\sy*(-0.0000)})
	--({\sx*(2.5700)},{\sy*(-0.0000)})
	--({\sx*(2.5800)},{\sy*(-0.0000)})
	--({\sx*(2.5900)},{\sy*(-0.0000)})
	--({\sx*(2.6000)},{\sy*(0.0000)})
	--({\sx*(2.6100)},{\sy*(0.0000)})
	--({\sx*(2.6200)},{\sy*(0.0000)})
	--({\sx*(2.6300)},{\sy*(-0.0000)})
	--({\sx*(2.6400)},{\sy*(-0.0000)})
	--({\sx*(2.6500)},{\sy*(-0.0000)})
	--({\sx*(2.6600)},{\sy*(-0.0000)})
	--({\sx*(2.6700)},{\sy*(0.0000)})
	--({\sx*(2.6800)},{\sy*(0.0000)})
	--({\sx*(2.6900)},{\sy*(0.0000)})
	--({\sx*(2.7000)},{\sy*(0.0000)})
	--({\sx*(2.7100)},{\sy*(-0.0000)})
	--({\sx*(2.7200)},{\sy*(-0.0000)})
	--({\sx*(2.7300)},{\sy*(-0.0000)})
	--({\sx*(2.7400)},{\sy*(-0.0000)})
	--({\sx*(2.7500)},{\sy*(0.0000)})
	--({\sx*(2.7600)},{\sy*(-0.0000)})
	--({\sx*(2.7700)},{\sy*(0.0000)})
	--({\sx*(2.7800)},{\sy*(0.0000)})
	--({\sx*(2.7900)},{\sy*(0.0000)})
	--({\sx*(2.8000)},{\sy*(0.0000)})
	--({\sx*(2.8100)},{\sy*(0.0000)})
	--({\sx*(2.8200)},{\sy*(0.0000)})
	--({\sx*(2.8300)},{\sy*(-0.0000)})
	--({\sx*(2.8400)},{\sy*(-0.0000)})
	--({\sx*(2.8500)},{\sy*(0.0000)})
	--({\sx*(2.8600)},{\sy*(-0.0000)})
	--({\sx*(2.8700)},{\sy*(0.0000)})
	--({\sx*(2.8800)},{\sy*(0.0000)})
	--({\sx*(2.8900)},{\sy*(0.0000)})
	--({\sx*(2.9000)},{\sy*(-0.0000)})
	--({\sx*(2.9100)},{\sy*(0.0000)})
	--({\sx*(2.9200)},{\sy*(0.0000)})
	--({\sx*(2.9300)},{\sy*(0.0000)})
	--({\sx*(2.9400)},{\sy*(-0.0000)})
	--({\sx*(2.9500)},{\sy*(0.0000)})
	--({\sx*(2.9600)},{\sy*(-0.0000)})
	--({\sx*(2.9700)},{\sy*(0.0000)})
	--({\sx*(2.9800)},{\sy*(0.0000)})
	--({\sx*(2.9900)},{\sy*(0.0000)})
	--({\sx*(3.0000)},{\sy*(0.0000)})
	--({\sx*(3.0100)},{\sy*(-0.0000)})
	--({\sx*(3.0200)},{\sy*(-0.0000)})
	--({\sx*(3.0300)},{\sy*(-0.0000)})
	--({\sx*(3.0400)},{\sy*(0.0000)})
	--({\sx*(3.0500)},{\sy*(0.0000)})
	--({\sx*(3.0600)},{\sy*(-0.0000)})
	--({\sx*(3.0700)},{\sy*(-0.0000)})
	--({\sx*(3.0800)},{\sy*(-0.0000)})
	--({\sx*(3.0900)},{\sy*(-0.0000)})
	--({\sx*(3.1000)},{\sy*(0.0000)})
	--({\sx*(3.1100)},{\sy*(-0.0000)})
	--({\sx*(3.1200)},{\sy*(-0.0000)})
	--({\sx*(3.1300)},{\sy*(-0.0000)})
	--({\sx*(3.1400)},{\sy*(-0.0000)})
	--({\sx*(3.1500)},{\sy*(-0.0000)})
	--({\sx*(3.1600)},{\sy*(0.0000)})
	--({\sx*(3.1700)},{\sy*(0.0000)})
	--({\sx*(3.1800)},{\sy*(-0.0000)})
	--({\sx*(3.1900)},{\sy*(0.0000)})
	--({\sx*(3.2000)},{\sy*(0.0000)})
	--({\sx*(3.2100)},{\sy*(-0.0000)})
	--({\sx*(3.2200)},{\sy*(0.0000)})
	--({\sx*(3.2300)},{\sy*(-0.0000)})
	--({\sx*(3.2400)},{\sy*(0.0000)})
	--({\sx*(3.2500)},{\sy*(-0.0000)})
	--({\sx*(3.2600)},{\sy*(0.0000)})
	--({\sx*(3.2700)},{\sy*(-0.0000)})
	--({\sx*(3.2800)},{\sy*(-0.0000)})
	--({\sx*(3.2900)},{\sy*(0.0000)})
	--({\sx*(3.3000)},{\sy*(0.0000)})
	--({\sx*(3.3100)},{\sy*(0.0000)})
	--({\sx*(3.3200)},{\sy*(0.0000)})
	--({\sx*(3.3300)},{\sy*(-0.0000)})
	--({\sx*(3.3400)},{\sy*(0.0000)})
	--({\sx*(3.3500)},{\sy*(0.0000)})
	--({\sx*(3.3600)},{\sy*(0.0000)})
	--({\sx*(3.3700)},{\sy*(0.0000)})
	--({\sx*(3.3800)},{\sy*(0.0000)})
	--({\sx*(3.3900)},{\sy*(-0.0000)})
	--({\sx*(3.4000)},{\sy*(-0.0000)})
	--({\sx*(3.4100)},{\sy*(-0.0000)})
	--({\sx*(3.4200)},{\sy*(-0.0000)})
	--({\sx*(3.4300)},{\sy*(0.0000)})
	--({\sx*(3.4400)},{\sy*(0.0000)})
	--({\sx*(3.4500)},{\sy*(-0.0000)})
	--({\sx*(3.4600)},{\sy*(0.0000)})
	--({\sx*(3.4700)},{\sy*(0.0000)})
	--({\sx*(3.4800)},{\sy*(0.0000)})
	--({\sx*(3.4900)},{\sy*(-0.0000)})
	--({\sx*(3.5000)},{\sy*(0.0000)})
	--({\sx*(3.5100)},{\sy*(-0.0000)})
	--({\sx*(3.5200)},{\sy*(-0.0000)})
	--({\sx*(3.5300)},{\sy*(-0.0000)})
	--({\sx*(3.5400)},{\sy*(0.0000)})
	--({\sx*(3.5500)},{\sy*(0.0000)})
	--({\sx*(3.5600)},{\sy*(-0.0000)})
	--({\sx*(3.5700)},{\sy*(-0.0000)})
	--({\sx*(3.5800)},{\sy*(0.0000)})
	--({\sx*(3.5900)},{\sy*(0.0000)})
	--({\sx*(3.6000)},{\sy*(-0.0000)})
	--({\sx*(3.6100)},{\sy*(-0.0000)})
	--({\sx*(3.6200)},{\sy*(0.0000)})
	--({\sx*(3.6300)},{\sy*(0.0000)})
	--({\sx*(3.6400)},{\sy*(-0.0000)})
	--({\sx*(3.6500)},{\sy*(0.0000)})
	--({\sx*(3.6600)},{\sy*(-0.0000)})
	--({\sx*(3.6700)},{\sy*(0.0000)})
	--({\sx*(3.6800)},{\sy*(-0.0000)})
	--({\sx*(3.6900)},{\sy*(0.0000)})
	--({\sx*(3.7000)},{\sy*(0.0000)})
	--({\sx*(3.7100)},{\sy*(0.0000)})
	--({\sx*(3.7200)},{\sy*(-0.0000)})
	--({\sx*(3.7300)},{\sy*(-0.0000)})
	--({\sx*(3.7400)},{\sy*(0.0000)})
	--({\sx*(3.7500)},{\sy*(0.0000)})
	--({\sx*(3.7600)},{\sy*(0.0000)})
	--({\sx*(3.7700)},{\sy*(0.0000)})
	--({\sx*(3.7800)},{\sy*(0.0000)})
	--({\sx*(3.7900)},{\sy*(-0.0000)})
	--({\sx*(3.8000)},{\sy*(0.0000)})
	--({\sx*(3.8100)},{\sy*(0.0000)})
	--({\sx*(3.8200)},{\sy*(-0.0000)})
	--({\sx*(3.8300)},{\sy*(-0.0000)})
	--({\sx*(3.8400)},{\sy*(0.0000)})
	--({\sx*(3.8500)},{\sy*(0.0000)})
	--({\sx*(3.8600)},{\sy*(-0.0000)})
	--({\sx*(3.8700)},{\sy*(0.0000)})
	--({\sx*(3.8800)},{\sy*(-0.0000)})
	--({\sx*(3.8900)},{\sy*(0.0000)})
	--({\sx*(3.9000)},{\sy*(-0.0000)})
	--({\sx*(3.9100)},{\sy*(0.0000)})
	--({\sx*(3.9200)},{\sy*(-0.0000)})
	--({\sx*(3.9300)},{\sy*(-0.0000)})
	--({\sx*(3.9400)},{\sy*(-0.0000)})
	--({\sx*(3.9500)},{\sy*(0.0000)})
	--({\sx*(3.9600)},{\sy*(-0.0000)})
	--({\sx*(3.9700)},{\sy*(0.0000)})
	--({\sx*(3.9800)},{\sy*(-0.0000)})
	--({\sx*(3.9900)},{\sy*(0.0000)})
	--({\sx*(4.0000)},{\sy*(0.0000)})
	--({\sx*(4.0100)},{\sy*(0.0000)})
	--({\sx*(4.0200)},{\sy*(-0.0000)})
	--({\sx*(4.0300)},{\sy*(0.0000)})
	--({\sx*(4.0400)},{\sy*(0.0000)})
	--({\sx*(4.0500)},{\sy*(0.0000)})
	--({\sx*(4.0600)},{\sy*(0.0000)})
	--({\sx*(4.0700)},{\sy*(-0.0000)})
	--({\sx*(4.0800)},{\sy*(0.0000)})
	--({\sx*(4.0900)},{\sy*(-0.0000)})
	--({\sx*(4.1000)},{\sy*(0.0000)})
	--({\sx*(4.1100)},{\sy*(0.0000)})
	--({\sx*(4.1200)},{\sy*(-0.0000)})
	--({\sx*(4.1300)},{\sy*(0.0000)})
	--({\sx*(4.1400)},{\sy*(0.0000)})
	--({\sx*(4.1500)},{\sy*(0.0000)})
	--({\sx*(4.1600)},{\sy*(-0.0000)})
	--({\sx*(4.1700)},{\sy*(0.0000)})
	--({\sx*(4.1800)},{\sy*(-0.0000)})
	--({\sx*(4.1900)},{\sy*(-0.0000)})
	--({\sx*(4.2000)},{\sy*(-0.0000)})
	--({\sx*(4.2100)},{\sy*(-0.0000)})
	--({\sx*(4.2200)},{\sy*(-0.0000)})
	--({\sx*(4.2300)},{\sy*(-0.0000)})
	--({\sx*(4.2400)},{\sy*(-0.0000)})
	--({\sx*(4.2500)},{\sy*(-0.0000)})
	--({\sx*(4.2600)},{\sy*(-0.0000)})
	--({\sx*(4.2700)},{\sy*(-0.0000)})
	--({\sx*(4.2800)},{\sy*(-0.0000)})
	--({\sx*(4.2900)},{\sy*(-0.0000)})
	--({\sx*(4.3000)},{\sy*(0.0000)})
	--({\sx*(4.3100)},{\sy*(-0.0000)})
	--({\sx*(4.3200)},{\sy*(0.0000)})
	--({\sx*(4.3300)},{\sy*(-0.0000)})
	--({\sx*(4.3400)},{\sy*(-0.0000)})
	--({\sx*(4.3500)},{\sy*(0.0000)})
	--({\sx*(4.3600)},{\sy*(0.0000)})
	--({\sx*(4.3700)},{\sy*(0.0000)})
	--({\sx*(4.3800)},{\sy*(0.0000)})
	--({\sx*(4.3900)},{\sy*(-0.0000)})
	--({\sx*(4.4000)},{\sy*(-0.0000)})
	--({\sx*(4.4100)},{\sy*(0.0000)})
	--({\sx*(4.4200)},{\sy*(0.0000)})
	--({\sx*(4.4300)},{\sy*(0.0000)})
	--({\sx*(4.4400)},{\sy*(0.0000)})
	--({\sx*(4.4500)},{\sy*(0.0000)})
	--({\sx*(4.4600)},{\sy*(-0.0000)})
	--({\sx*(4.4700)},{\sy*(0.0000)})
	--({\sx*(4.4800)},{\sy*(0.0000)})
	--({\sx*(4.4900)},{\sy*(0.0000)})
	--({\sx*(4.5000)},{\sy*(0.0000)})
	--({\sx*(4.5100)},{\sy*(-0.0000)})
	--({\sx*(4.5200)},{\sy*(0.0000)})
	--({\sx*(4.5300)},{\sy*(-0.0000)})
	--({\sx*(4.5400)},{\sy*(-0.0000)})
	--({\sx*(4.5500)},{\sy*(0.0000)})
	--({\sx*(4.5600)},{\sy*(-0.0000)})
	--({\sx*(4.5700)},{\sy*(0.0000)})
	--({\sx*(4.5800)},{\sy*(0.0000)})
	--({\sx*(4.5900)},{\sy*(-0.0000)})
	--({\sx*(4.6000)},{\sy*(-0.0000)})
	--({\sx*(4.6100)},{\sy*(-0.0000)})
	--({\sx*(4.6200)},{\sy*(-0.0000)})
	--({\sx*(4.6300)},{\sy*(0.0000)})
	--({\sx*(4.6400)},{\sy*(0.0000)})
	--({\sx*(4.6500)},{\sy*(0.0000)})
	--({\sx*(4.6600)},{\sy*(0.0000)})
	--({\sx*(4.6700)},{\sy*(0.0000)})
	--({\sx*(4.6800)},{\sy*(0.0000)})
	--({\sx*(4.6900)},{\sy*(0.0000)})
	--({\sx*(4.7000)},{\sy*(0.0000)})
	--({\sx*(4.7100)},{\sy*(-0.0000)})
	--({\sx*(4.7200)},{\sy*(0.0000)})
	--({\sx*(4.7300)},{\sy*(0.0000)})
	--({\sx*(4.7400)},{\sy*(-0.0000)})
	--({\sx*(4.7500)},{\sy*(0.0000)})
	--({\sx*(4.7600)},{\sy*(-0.0000)})
	--({\sx*(4.7700)},{\sy*(-0.0000)})
	--({\sx*(4.7800)},{\sy*(-0.0000)})
	--({\sx*(4.7900)},{\sy*(0.0000)})
	--({\sx*(4.8000)},{\sy*(0.0000)})
	--({\sx*(4.8100)},{\sy*(-0.0000)})
	--({\sx*(4.8200)},{\sy*(0.0000)})
	--({\sx*(4.8300)},{\sy*(-0.0000)})
	--({\sx*(4.8400)},{\sy*(-0.0000)})
	--({\sx*(4.8500)},{\sy*(0.0000)})
	--({\sx*(4.8600)},{\sy*(-0.0000)})
	--({\sx*(4.8700)},{\sy*(-0.0000)})
	--({\sx*(4.8800)},{\sy*(-0.0000)})
	--({\sx*(4.8900)},{\sy*(-0.0000)})
	--({\sx*(4.9000)},{\sy*(0.0000)})
	--({\sx*(4.9100)},{\sy*(-0.0000)})
	--({\sx*(4.9200)},{\sy*(0.0000)})
	--({\sx*(4.9300)},{\sy*(-0.0000)})
	--({\sx*(4.9400)},{\sy*(0.0000)})
	--({\sx*(4.9500)},{\sy*(0.0000)})
	--({\sx*(4.9600)},{\sy*(-0.0000)})
	--({\sx*(4.9700)},{\sy*(0.0000)})
	--({\sx*(4.9800)},{\sy*(-0.0000)})
	--({\sx*(4.9900)},{\sy*(0.0000)})
	--({\sx*(5.0000)},{\sy*(0.0000)});
}

\begin{tikzpicture}[>=latex,thick,scale=\skala]

\def\sx{1}
\def\sy{0.9}

\def\plot#1{
	\csname fehler#1\endcsname
	\csname xwerte#1\endcsname
	\draw ({-0.1/\skala},\sy)--({0.1/\skala},\sy);
	\node at ({-0.1/\skala},\sy) [left] {$\csname maxfehler#1\endcsname$};
	\draw ({-0.1/\skala},-\sy)--({0.1/\skala},-\sy);
	\node at ({-0.1/\skala},-\sy) [left] {$-\csname maxfehler#1\endcsname$};
	\draw[->] ({-0.1/\skala},0)--(5.2,0) coordinate[label={$x$}];
	\draw[->] (0,{-1.1*\sy})--(0,{1.2*\sy}) coordinate[label={right:$y$}];
	\node at ({-0.2/\skala},0) [left] {$n=\csname punkte#1\endcsname$};
	\foreach \x in {1,...,5}{
		\draw (\x,{-0.1/\skala})--(\x,{0.1/\skala});
		\node at (\x,{-0.1*\skala}) [below] {$\x$};
	}
}

\begin{scope}
\plot{a}
\end{scope}

\begin{scope}[yshift={-2.5cm*\sy}]
\plot{h}
\end{scope}

\begin{scope}[yshift={-5cm*\sy}]
\plot{o}
\end{scope}

\begin{scope}[yshift={-7.5cm*\sy}]
\plot{q}
\end{scope}

\end{tikzpicture}
\end{document}

