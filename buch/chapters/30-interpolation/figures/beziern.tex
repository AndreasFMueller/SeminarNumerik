%
% beziern.tex -- Bezier-Kurve höherer Ordnung
%
% (c) 2020 Prof Dr Andreas Müller, Hochschule Rapperswil
%
\documentclass[tikz]{standalone}
\usepackage{amsmath}
\usepackage{times}
\usepackage{txfonts}
\usepackage{pgfplots}
\usepackage{csvsimple}
\usetikzlibrary{arrows,intersections,math}
\begin{document}
\def\skala{1}
\begin{tikzpicture}[>=latex,thick,scale=\skala]

\coordinate (A) at (0,0);
\coordinate (B) at (2,5);
\coordinate (C) at (5,7);
\coordinate (D) at (8,2);
\coordinate (E) at (9,-1);
\coordinate (F) at (12,3);

\draw[color=gray!20,line width=1.4pt] (A) -- (B) -- (C) -- (D) -- (E) -- (F);

\draw[color=blue,line width=1.4pt]
	plot[domain=0:1,samples=100]
	({1*(1-\x)*(1-\x)*(1-\x)*(1-\x)*(1-\x)*0+5*(1-\x)*(1-\x)*(1-\x)*(1-\x)*\x*2+10*(1-\x)*(1-\x)*(1-\x)*\x*\x*5+10*(1-\x)*(1-\x)*\x*\x*\x*8+5*(1-\x)*\x*\x*\x*\x*9+1*\x*\x*\x*\x*\x*12},{1*(1-\x)*(1-\x)*(1-\x)*(1-\x)*(1-\x)*0+5*(1-\x)*(1-\x)*(1-\x)*(1-\x)*\x*5+10*(1-\x)*(1-\x)*(1-\x)*\x*\x*7+10*(1-\x)*(1-\x)*\x*\x*\x*2+5*(1-\x)*\x*\x*\x*\x*(-1)+1*\x*\x*\x*\x*\x*3});

\fill[color=red] (A) circle[radius=0.08];
\fill[color=red] (B) circle[radius=0.08];
\fill[color=red] (C) circle[radius=0.08];
\fill[color=red] (D) circle[radius=0.08];
\fill[color=red] (E) circle[radius=0.08];
\fill[color=red] (F) circle[radius=0.08];

\node at (A) [left] {$P_0$};
\node at (B) [above left] {$P_1$};
\node at (C) [above] {$P_2$};
\node at (D) [below left] {$P_3$};
\node at (E) [left] {$P_4$};
\node at (F) [above left] {$P_5$};

\end{tikzpicture}
\end{document}

