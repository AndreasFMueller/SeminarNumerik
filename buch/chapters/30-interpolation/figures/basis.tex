%
% basis.tex -- Basis-Bilder für Lagrange Interpolationspolynom
%
% (c) 2020 Prof Dr Andreas Müller, Hochschule Rapperswil
%
\documentclass[tikz]{standalone}
\usepackage{amsmath}
\usepackage{times}
\usepackage{txfonts}
\usepackage{pgfplots}
\usepackage{csvsimple}
\usetikzlibrary{arrows,intersections,math}
\begin{document}
\def\skala{1.95}
\def\yskala{0.5}
\def\basiszero{
\draw[line width=1.4pt,color=red] (-0.5000,{2.9326*\yskala})
	--(-0.4762,{2.8016*\yskala})
	--(-0.4524,{2.6752*\yskala})
	--(-0.4286,{2.5533*\yskala})
	--(-0.4048,{2.4357*\yskala})
	--(-0.3810,{2.3225*\yskala})
	--(-0.3571,{2.2133*\yskala})
	--(-0.3333,{2.1082*\yskala})
	--(-0.3095,{2.0070*\yskala})
	--(-0.2857,{1.9096*\yskala})
	--(-0.2619,{1.8159*\yskala})
	--(-0.2381,{1.7258*\yskala})
	--(-0.2143,{1.6391*\yskala})
	--(-0.1905,{1.5559*\yskala})
	--(-0.1667,{1.4759*\yskala})
	--(-0.1429,{1.3991*\yskala})
	--(-0.1190,{1.3254*\yskala})
	--(-0.0952,{1.2547*\yskala})
	--(-0.0714,{1.1869*\yskala})
	--(-0.0476,{1.1219*\yskala})
	--(-0.0238,{1.0596*\yskala})
	--(0.0000,{1.0000*\yskala})
	--(0.0238,{0.9429*\yskala})
	--(0.0476,{0.8883*\yskala})
	--(0.0714,{0.8361*\yskala})
	--(0.0952,{0.7863*\yskala})
	--(0.1190,{0.7386*\yskala})
	--(0.1429,{0.6932*\yskala})
	--(0.1667,{0.6498*\yskala})
	--(0.1905,{0.6084*\yskala})
	--(0.2143,{0.5690*\yskala})
	--(0.2381,{0.5315*\yskala})
	--(0.2619,{0.4958*\yskala})
	--(0.2857,{0.4619*\yskala})
	--(0.3095,{0.4296*\yskala})
	--(0.3333,{0.3990*\yskala})
	--(0.3571,{0.3700*\yskala})
	--(0.3810,{0.3425*\yskala})
	--(0.4048,{0.3164*\yskala})
	--(0.4286,{0.2917*\yskala})
	--(0.4524,{0.2684*\yskala})
	--(0.4762,{0.2464*\yskala})
	--(0.5000,{0.2256*\yskala})
	--(0.5238,{0.2060*\yskala})
	--(0.5476,{0.1876*\yskala})
	--(0.5714,{0.1702*\yskala})
	--(0.5952,{0.1539*\yskala})
	--(0.6190,{0.1387*\yskala})
	--(0.6429,{0.1243*\yskala})
	--(0.6667,{0.1110*\yskala})
	--(0.6905,{0.0985*\yskala})
	--(0.7143,{0.0868*\yskala})
	--(0.7381,{0.0760*\yskala})
	--(0.7619,{0.0659*\yskala})
	--(0.7857,{0.0565*\yskala})
	--(0.8095,{0.0479*\yskala})
	--(0.8333,{0.0399*\yskala})
	--(0.8571,{0.0325*\yskala})
	--(0.8810,{0.0258*\yskala})
	--(0.9048,{0.0196*\yskala})
	--(0.9286,{0.0140*\yskala})
	--(0.9524,{0.0088*\yskala})
	--(0.9762,{0.0042*\yskala})
	--(1.0000,{-0.0000*\yskala})
	--(1.0238,{-0.0038*\yskala})
	--(1.0476,{-0.0071*\yskala})
	--(1.0714,{-0.0101*\yskala})
	--(1.0952,{-0.0127*\yskala})
	--(1.1190,{-0.0150*\yskala})
	--(1.1429,{-0.0169*\yskala})
	--(1.1667,{-0.0186*\yskala})
	--(1.1905,{-0.0199*\yskala})
	--(1.2143,{-0.0211*\yskala})
	--(1.2381,{-0.0220*\yskala})
	--(1.2619,{-0.0226*\yskala})
	--(1.2857,{-0.0231*\yskala})
	--(1.3095,{-0.0234*\yskala})
	--(1.3333,{-0.0235*\yskala})
	--(1.3571,{-0.0234*\yskala})
	--(1.3810,{-0.0232*\yskala})
	--(1.4048,{-0.0229*\yskala})
	--(1.4286,{-0.0224*\yskala})
	--(1.4524,{-0.0219*\yskala})
	--(1.4762,{-0.0212*\yskala})
	--(1.5000,{-0.0205*\yskala})
	--(1.5238,{-0.0197*\yskala})
	--(1.5476,{-0.0188*\yskala})
	--(1.5714,{-0.0179*\yskala})
	--(1.5952,{-0.0170*\yskala})
	--(1.6190,{-0.0160*\yskala})
	--(1.6429,{-0.0149*\yskala})
	--(1.6667,{-0.0139*\yskala})
	--(1.6905,{-0.0128*\yskala})
	--(1.7143,{-0.0117*\yskala})
	--(1.7381,{-0.0107*\yskala})
	--(1.7619,{-0.0096*\yskala})
	--(1.7857,{-0.0085*\yskala})
	--(1.8095,{-0.0075*\yskala})
	--(1.8333,{-0.0064*\yskala})
	--(1.8571,{-0.0054*\yskala})
	--(1.8810,{-0.0044*\yskala})
	--(1.9048,{-0.0035*\yskala})
	--(1.9286,{-0.0026*\yskala})
	--(1.9524,{-0.0017*\yskala})
	--(1.9762,{-0.0008*\yskala})
	--(2.0000,{0.0000*\yskala})
	--(2.0238,{0.0008*\yskala})
	--(2.0476,{0.0015*\yskala})
	--(2.0714,{0.0022*\yskala})
	--(2.0952,{0.0028*\yskala})
	--(2.1190,{0.0034*\yskala})
	--(2.1429,{0.0040*\yskala})
	--(2.1667,{0.0045*\yskala})
	--(2.1905,{0.0049*\yskala})
	--(2.2143,{0.0053*\yskala})
	--(2.2381,{0.0057*\yskala})
	--(2.2619,{0.0060*\yskala})
	--(2.2857,{0.0063*\yskala})
	--(2.3095,{0.0065*\yskala})
	--(2.3333,{0.0067*\yskala})
	--(2.3571,{0.0068*\yskala})
	--(2.3810,{0.0069*\yskala})
	--(2.4048,{0.0070*\yskala})
	--(2.4286,{0.0070*\yskala})
	--(2.4524,{0.0070*\yskala})
	--(2.4762,{0.0069*\yskala})
	--(2.5000,{0.0068*\yskala})
	--(2.5238,{0.0067*\yskala})
	--(2.5476,{0.0065*\yskala})
	--(2.5714,{0.0064*\yskala})
	--(2.5952,{0.0061*\yskala})
	--(2.6190,{0.0059*\yskala})
	--(2.6429,{0.0056*\yskala})
	--(2.6667,{0.0053*\yskala})
	--(2.6905,{0.0050*\yskala})
	--(2.7143,{0.0047*\yskala})
	--(2.7381,{0.0043*\yskala})
	--(2.7619,{0.0040*\yskala})
	--(2.7857,{0.0036*\yskala})
	--(2.8095,{0.0032*\yskala})
	--(2.8333,{0.0028*\yskala})
	--(2.8571,{0.0024*\yskala})
	--(2.8810,{0.0020*\yskala})
	--(2.9048,{0.0016*\yskala})
	--(2.9286,{0.0012*\yskala})
	--(2.9524,{0.0008*\yskala})
	--(2.9762,{0.0004*\yskala})
	--(3.0000,{-0.0000*\yskala})
	--(3.0238,{-0.0004*\yskala})
	--(3.0476,{-0.0008*\yskala})
	--(3.0714,{-0.0012*\yskala})
	--(3.0952,{-0.0015*\yskala})
	--(3.1190,{-0.0019*\yskala})
	--(3.1429,{-0.0022*\yskala})
	--(3.1667,{-0.0025*\yskala})
	--(3.1905,{-0.0028*\yskala})
	--(3.2143,{-0.0031*\yskala})
	--(3.2381,{-0.0034*\yskala})
	--(3.2619,{-0.0036*\yskala})
	--(3.2857,{-0.0039*\yskala})
	--(3.3095,{-0.0041*\yskala})
	--(3.3333,{-0.0043*\yskala})
	--(3.3571,{-0.0044*\yskala})
	--(3.3810,{-0.0046*\yskala})
	--(3.4048,{-0.0047*\yskala})
	--(3.4286,{-0.0048*\yskala})
	--(3.4524,{-0.0048*\yskala})
	--(3.4762,{-0.0049*\yskala})
	--(3.5000,{-0.0049*\yskala})
	--(3.5238,{-0.0049*\yskala})
	--(3.5476,{-0.0048*\yskala})
	--(3.5714,{-0.0048*\yskala})
	--(3.5952,{-0.0047*\yskala})
	--(3.6190,{-0.0046*\yskala})
	--(3.6429,{-0.0044*\yskala})
	--(3.6667,{-0.0043*\yskala})
	--(3.6905,{-0.0041*\yskala})
	--(3.7143,{-0.0039*\yskala})
	--(3.7381,{-0.0036*\yskala})
	--(3.7619,{-0.0034*\yskala})
	--(3.7857,{-0.0031*\yskala})
	--(3.8095,{-0.0028*\yskala})
	--(3.8333,{-0.0025*\yskala})
	--(3.8571,{-0.0022*\yskala})
	--(3.8810,{-0.0019*\yskala})
	--(3.9048,{-0.0015*\yskala})
	--(3.9286,{-0.0012*\yskala})
	--(3.9524,{-0.0008*\yskala})
	--(3.9762,{-0.0004*\yskala})
	--(4.0000,{0.0000*\yskala})
	--(4.0238,{0.0004*\yskala})
	--(4.0476,{0.0008*\yskala})
	--(4.0714,{0.0012*\yskala})
	--(4.0952,{0.0016*\yskala})
	--(4.1190,{0.0020*\yskala})
	--(4.1429,{0.0024*\yskala})
	--(4.1667,{0.0028*\yskala})
	--(4.1905,{0.0032*\yskala})
	--(4.2143,{0.0036*\yskala})
	--(4.2381,{0.0040*\yskala})
	--(4.2619,{0.0043*\yskala})
	--(4.2857,{0.0047*\yskala})
	--(4.3095,{0.0050*\yskala})
	--(4.3333,{0.0053*\yskala})
	--(4.3571,{0.0056*\yskala})
	--(4.3810,{0.0059*\yskala})
	--(4.4048,{0.0061*\yskala})
	--(4.4286,{0.0064*\yskala})
	--(4.4524,{0.0065*\yskala})
	--(4.4762,{0.0067*\yskala})
	--(4.5000,{0.0068*\yskala})
	--(4.5238,{0.0069*\yskala})
	--(4.5476,{0.0070*\yskala})
	--(4.5714,{0.0070*\yskala})
	--(4.5952,{0.0070*\yskala})
	--(4.6190,{0.0069*\yskala})
	--(4.6429,{0.0068*\yskala})
	--(4.6667,{0.0067*\yskala})
	--(4.6905,{0.0065*\yskala})
	--(4.7143,{0.0063*\yskala})
	--(4.7381,{0.0060*\yskala})
	--(4.7619,{0.0057*\yskala})
	--(4.7857,{0.0053*\yskala})
	--(4.8095,{0.0049*\yskala})
	--(4.8333,{0.0045*\yskala})
	--(4.8571,{0.0040*\yskala})
	--(4.8810,{0.0034*\yskala})
	--(4.9048,{0.0028*\yskala})
	--(4.9286,{0.0022*\yskala})
	--(4.9524,{0.0015*\yskala})
	--(4.9762,{0.0008*\yskala})
	--(5.0000,{-0.0000*\yskala})
	--(5.0238,{-0.0008*\yskala})
	--(5.0476,{-0.0017*\yskala})
	--(5.0714,{-0.0026*\yskala})
	--(5.0952,{-0.0035*\yskala})
	--(5.1190,{-0.0044*\yskala})
	--(5.1429,{-0.0054*\yskala})
	--(5.1667,{-0.0064*\yskala})
	--(5.1905,{-0.0075*\yskala})
	--(5.2143,{-0.0085*\yskala})
	--(5.2381,{-0.0096*\yskala})
	--(5.2619,{-0.0107*\yskala})
	--(5.2857,{-0.0117*\yskala})
	--(5.3095,{-0.0128*\yskala})
	--(5.3333,{-0.0139*\yskala})
	--(5.3571,{-0.0149*\yskala})
	--(5.3810,{-0.0160*\yskala})
	--(5.4048,{-0.0170*\yskala})
	--(5.4286,{-0.0179*\yskala})
	--(5.4524,{-0.0188*\yskala})
	--(5.4762,{-0.0197*\yskala})
	--(5.5000,{-0.0205*\yskala})
	--(5.5238,{-0.0212*\yskala})
	--(5.5476,{-0.0219*\yskala})
	--(5.5714,{-0.0224*\yskala})
	--(5.5952,{-0.0229*\yskala})
	--(5.6190,{-0.0232*\yskala})
	--(5.6429,{-0.0234*\yskala})
	--(5.6667,{-0.0235*\yskala})
	--(5.6905,{-0.0234*\yskala})
	--(5.7143,{-0.0231*\yskala})
	--(5.7381,{-0.0226*\yskala})
	--(5.7619,{-0.0220*\yskala})
	--(5.7857,{-0.0211*\yskala})
	--(5.8095,{-0.0199*\yskala})
	--(5.8333,{-0.0186*\yskala})
	--(5.8571,{-0.0169*\yskala})
	--(5.8810,{-0.0150*\yskala})
	--(5.9048,{-0.0127*\yskala})
	--(5.9286,{-0.0101*\yskala})
	--(5.9524,{-0.0071*\yskala})
	--(5.9762,{-0.0038*\yskala})
	--(6.0000,{0.0000*\yskala})
	--(6.0238,{0.0042*\yskala})
	--(6.0476,{0.0088*\yskala})
	--(6.0714,{0.0140*\yskala})
	--(6.0952,{0.0196*\yskala})
	--(6.1190,{0.0258*\yskala})
	--(6.1429,{0.0325*\yskala})
	--(6.1667,{0.0399*\yskala})
	--(6.1905,{0.0479*\yskala})
	--(6.2143,{0.0565*\yskala})
	--(6.2381,{0.0659*\yskala})
	--(6.2619,{0.0760*\yskala})
	--(6.2857,{0.0868*\yskala})
	--(6.3095,{0.0985*\yskala})
	--(6.3333,{0.1110*\yskala})
	--(6.3571,{0.1243*\yskala})
	--(6.3810,{0.1387*\yskala})
	--(6.4048,{0.1539*\yskala})
	--(6.4286,{0.1702*\yskala})
	--(6.4524,{0.1876*\yskala})
	--(6.4762,{0.2060*\yskala})
	--(6.5000,{0.2256*\yskala});
}
\def\basisone{
\draw[line width=1.4pt,color=red] (-0.5000,{-5.8652*\yskala})
	--(-0.4762,{-5.4224*\yskala})
	--(-0.4524,{-4.9995*\yskala})
	--(-0.4286,{-4.5959*\yskala})
	--(-0.4048,{-4.2109*\yskala})
	--(-0.3810,{-3.8441*\yskala})
	--(-0.3571,{-3.4947*\yskala})
	--(-0.3333,{-3.1623*\yskala})
	--(-0.3095,{-2.8463*\yskala})
	--(-0.2857,{-2.5461*\yskala})
	--(-0.2619,{-2.2613*\yskala})
	--(-0.2381,{-1.9913*\yskala})
	--(-0.2143,{-1.7356*\yskala})
	--(-0.1905,{-1.4936*\yskala})
	--(-0.1667,{-1.2651*\yskala})
	--(-0.1429,{-1.0493*\yskala})
	--(-0.1190,{-0.8460*\yskala})
	--(-0.0952,{-0.6546*\yskala})
	--(-0.0714,{-0.4748*\yskala})
	--(-0.0476,{-0.3060*\yskala})
	--(-0.0238,{-0.1479*\yskala})
	--(0.0000,{0.0000*\yskala})
	--(0.0238,{0.1380*\yskala})
	--(0.0476,{0.2665*\yskala})
	--(0.0714,{0.3859*\yskala})
	--(0.0952,{0.4966*\yskala})
	--(0.1190,{0.5989*\yskala})
	--(0.1429,{0.6932*\yskala})
	--(0.1667,{0.7797*\yskala})
	--(0.1905,{0.8590*\yskala})
	--(0.2143,{0.9311*\yskala})
	--(0.2381,{0.9966*\yskala})
	--(0.2619,{1.0556*\yskala})
	--(0.2857,{1.1085*\yskala})
	--(0.3095,{1.1556*\yskala})
	--(0.3333,{1.1971*\yskala})
	--(0.3571,{1.2333*\yskala})
	--(0.3810,{1.2645*\yskala})
	--(0.4048,{1.2908*\yskala})
	--(0.4286,{1.3127*\yskala})
	--(0.4524,{1.3303*\yskala})
	--(0.4762,{1.3438*\yskala})
	--(0.5000,{1.3535*\yskala})
	--(0.5238,{1.3596*\yskala})
	--(0.5476,{1.3623*\yskala})
	--(0.5714,{1.3617*\yskala})
	--(0.5952,{1.3582*\yskala})
	--(0.6190,{1.3519*\yskala})
	--(0.6429,{1.3429*\yskala})
	--(0.6667,{1.3315*\yskala})
	--(0.6905,{1.3178*\yskala})
	--(0.7143,{1.3020*\yskala})
	--(0.7381,{1.2843*\yskala})
	--(0.7619,{1.2647*\yskala})
	--(0.7857,{1.2435*\yskala})
	--(0.8095,{1.2208*\yskala})
	--(0.8333,{1.1967*\yskala})
	--(0.8571,{1.1713*\yskala})
	--(0.8810,{1.1449*\yskala})
	--(0.9048,{1.1175*\yskala})
	--(0.9286,{1.0891*\yskala})
	--(0.9524,{1.0601*\yskala})
	--(0.9762,{1.0303*\yskala})
	--(1.0000,{1.0000*\yskala})
	--(1.0238,{0.9692*\yskala})
	--(1.0476,{0.9381*\yskala})
	--(1.0714,{0.9067*\yskala})
	--(1.0952,{0.8750*\yskala})
	--(1.1190,{0.8433*\yskala})
	--(1.1429,{0.8115*\yskala})
	--(1.1667,{0.7797*\yskala})
	--(1.1905,{0.7481*\yskala})
	--(1.2143,{0.7166*\yskala})
	--(1.2381,{0.6853*\yskala})
	--(1.2619,{0.6542*\yskala})
	--(1.2857,{0.6235*\yskala})
	--(1.3095,{0.5932*\yskala})
	--(1.3333,{0.5633*\yskala})
	--(1.3571,{0.5339*\yskala})
	--(1.3810,{0.5050*\yskala})
	--(1.4048,{0.4766*\yskala})
	--(1.4286,{0.4488*\yskala})
	--(1.4524,{0.4216*\yskala})
	--(1.4762,{0.3950*\yskala})
	--(1.5000,{0.3691*\yskala})
	--(1.5238,{0.3439*\yskala})
	--(1.5476,{0.3194*\yskala})
	--(1.5714,{0.2956*\yskala})
	--(1.5952,{0.2726*\yskala})
	--(1.6190,{0.2503*\yskala})
	--(1.6429,{0.2288*\yskala})
	--(1.6667,{0.2080*\yskala})
	--(1.6905,{0.1881*\yskala})
	--(1.7143,{0.1689*\yskala})
	--(1.7381,{0.1505*\yskala})
	--(1.7619,{0.1329*\yskala})
	--(1.7857,{0.1161*\yskala})
	--(1.8095,{0.1001*\yskala})
	--(1.8333,{0.0849*\yskala})
	--(1.8571,{0.0705*\yskala})
	--(1.8810,{0.0568*\yskala})
	--(1.9048,{0.0440*\yskala})
	--(1.9286,{0.0319*\yskala})
	--(1.9524,{0.0205*\yskala})
	--(1.9762,{0.0099*\yskala})
	--(2.0000,{-0.0000*\yskala})
	--(2.0238,{-0.0092*\yskala})
	--(2.0476,{-0.0176*\yskala})
	--(2.0714,{-0.0254*\yskala})
	--(2.0952,{-0.0325*\yskala})
	--(2.1190,{-0.0389*\yskala})
	--(2.1429,{-0.0448*\yskala})
	--(2.1667,{-0.0499*\yskala})
	--(2.1905,{-0.0545*\yskala})
	--(2.2143,{-0.0585*\yskala})
	--(2.2381,{-0.0619*\yskala})
	--(2.2619,{-0.0648*\yskala})
	--(2.2857,{-0.0672*\yskala})
	--(2.3095,{-0.0690*\yskala})
	--(2.3333,{-0.0704*\yskala})
	--(2.3571,{-0.0713*\yskala})
	--(2.3810,{-0.0718*\yskala})
	--(2.4048,{-0.0719*\yskala})
	--(2.4286,{-0.0715*\yskala})
	--(2.4524,{-0.0708*\yskala})
	--(2.4762,{-0.0698*\yskala})
	--(2.5000,{-0.0684*\yskala})
	--(2.5238,{-0.0667*\yskala})
	--(2.5476,{-0.0647*\yskala})
	--(2.5714,{-0.0624*\yskala})
	--(2.5952,{-0.0599*\yskala})
	--(2.6190,{-0.0572*\yskala})
	--(2.6429,{-0.0543*\yskala})
	--(2.6667,{-0.0512*\yskala})
	--(2.6905,{-0.0480*\yskala})
	--(2.7143,{-0.0446*\yskala})
	--(2.7381,{-0.0411*\yskala})
	--(2.7619,{-0.0375*\yskala})
	--(2.7857,{-0.0338*\yskala})
	--(2.8095,{-0.0300*\yskala})
	--(2.8333,{-0.0262*\yskala})
	--(2.8571,{-0.0224*\yskala})
	--(2.8810,{-0.0186*\yskala})
	--(2.9048,{-0.0148*\yskala})
	--(2.9286,{-0.0110*\yskala})
	--(2.9524,{-0.0073*\yskala})
	--(2.9762,{-0.0036*\yskala})
	--(3.0000,{0.0000*\yskala})
	--(3.0238,{0.0035*\yskala})
	--(3.0476,{0.0070*\yskala})
	--(3.0714,{0.0103*\yskala})
	--(3.0952,{0.0135*\yskala})
	--(3.1190,{0.0165*\yskala})
	--(3.1429,{0.0194*\yskala})
	--(3.1667,{0.0222*\yskala})
	--(3.1905,{0.0248*\yskala})
	--(3.2143,{0.0272*\yskala})
	--(3.2381,{0.0295*\yskala})
	--(3.2619,{0.0316*\yskala})
	--(3.2857,{0.0334*\yskala})
	--(3.3095,{0.0351*\yskala})
	--(3.3333,{0.0366*\yskala})
	--(3.3571,{0.0378*\yskala})
	--(3.3810,{0.0389*\yskala})
	--(3.4048,{0.0398*\yskala})
	--(3.4286,{0.0404*\yskala})
	--(3.4524,{0.0408*\yskala})
	--(3.4762,{0.0410*\yskala})
	--(3.5000,{0.0410*\yskala})
	--(3.5238,{0.0408*\yskala})
	--(3.5476,{0.0404*\yskala})
	--(3.5714,{0.0397*\yskala})
	--(3.5952,{0.0389*\yskala})
	--(3.6190,{0.0379*\yskala})
	--(3.6429,{0.0366*\yskala})
	--(3.6667,{0.0352*\yskala})
	--(3.6905,{0.0336*\yskala})
	--(3.7143,{0.0318*\yskala})
	--(3.7381,{0.0299*\yskala})
	--(3.7619,{0.0278*\yskala})
	--(3.7857,{0.0255*\yskala})
	--(3.8095,{0.0231*\yskala})
	--(3.8333,{0.0206*\yskala})
	--(3.8571,{0.0179*\yskala})
	--(3.8810,{0.0151*\yskala})
	--(3.9048,{0.0123*\yskala})
	--(3.9286,{0.0093*\yskala})
	--(3.9524,{0.0063*\yskala})
	--(3.9762,{0.0032*\yskala})
	--(4.0000,{-0.0000*\yskala})
	--(4.0238,{-0.0032*\yskala})
	--(4.0476,{-0.0064*\yskala})
	--(4.0714,{-0.0096*\yskala})
	--(4.0952,{-0.0129*\yskala})
	--(4.1190,{-0.0161*\yskala})
	--(4.1429,{-0.0192*\yskala})
	--(4.1667,{-0.0223*\yskala})
	--(4.1905,{-0.0254*\yskala})
	--(4.2143,{-0.0284*\yskala})
	--(4.2381,{-0.0313*\yskala})
	--(4.2619,{-0.0341*\yskala})
	--(4.2857,{-0.0367*\yskala})
	--(4.3095,{-0.0392*\yskala})
	--(4.3333,{-0.0416*\yskala})
	--(4.3571,{-0.0438*\yskala})
	--(4.3810,{-0.0458*\yskala})
	--(4.4048,{-0.0477*\yskala})
	--(4.4286,{-0.0493*\yskala})
	--(4.4524,{-0.0507*\yskala})
	--(4.4762,{-0.0518*\yskala})
	--(4.5000,{-0.0527*\yskala})
	--(4.5238,{-0.0534*\yskala})
	--(4.5476,{-0.0538*\yskala})
	--(4.5714,{-0.0539*\yskala})
	--(4.5952,{-0.0537*\yskala})
	--(4.6190,{-0.0532*\yskala})
	--(4.6429,{-0.0523*\yskala})
	--(4.6667,{-0.0512*\yskala})
	--(4.6905,{-0.0498*\yskala})
	--(4.7143,{-0.0480*\yskala})
	--(4.7381,{-0.0458*\yskala})
	--(4.7619,{-0.0434*\yskala})
	--(4.7857,{-0.0406*\yskala})
	--(4.8095,{-0.0374*\yskala})
	--(4.8333,{-0.0339*\yskala})
	--(4.8571,{-0.0301*\yskala})
	--(4.8810,{-0.0259*\yskala})
	--(4.9048,{-0.0213*\yskala})
	--(4.9286,{-0.0165*\yskala})
	--(4.9524,{-0.0113*\yskala})
	--(4.9762,{-0.0058*\yskala})
	--(5.0000,{0.0000*\yskala})
	--(5.0238,{0.0061*\yskala})
	--(5.0476,{0.0125*\yskala})
	--(5.0714,{0.0191*\yskala})
	--(5.0952,{0.0260*\yskala})
	--(5.1190,{0.0331*\yskala})
	--(5.1429,{0.0404*\yskala})
	--(5.1667,{0.0479*\yskala})
	--(5.1905,{0.0555*\yskala})
	--(5.2143,{0.0632*\yskala})
	--(5.2381,{0.0711*\yskala})
	--(5.2619,{0.0789*\yskala})
	--(5.2857,{0.0868*\yskala})
	--(5.3095,{0.0946*\yskala})
	--(5.3333,{0.1024*\yskala})
	--(5.3571,{0.1101*\yskala})
	--(5.3810,{0.1176*\yskala})
	--(5.4048,{0.1248*\yskala})
	--(5.4286,{0.1318*\yskala})
	--(5.4524,{0.1384*\yskala})
	--(5.4762,{0.1446*\yskala})
	--(5.5000,{0.1504*\yskala})
	--(5.5238,{0.1556*\yskala})
	--(5.5476,{0.1602*\yskala})
	--(5.5714,{0.1641*\yskala})
	--(5.5952,{0.1672*\yskala})
	--(5.6190,{0.1695*\yskala})
	--(5.6429,{0.1708*\yskala})
	--(5.6667,{0.1710*\yskala})
	--(5.6905,{0.1701*\yskala})
	--(5.7143,{0.1680*\yskala})
	--(5.7381,{0.1644*\yskala})
	--(5.7619,{0.1595*\yskala})
	--(5.7857,{0.1529*\yskala})
	--(5.8095,{0.1446*\yskala})
	--(5.8333,{0.1344*\yskala})
	--(5.8571,{0.1223*\yskala})
	--(5.8810,{0.1081*\yskala})
	--(5.9048,{0.0916*\yskala})
	--(5.9286,{0.0727*\yskala})
	--(5.9524,{0.0513*\yskala})
	--(5.9762,{0.0271*\yskala})
	--(6.0000,{-0.0000*\yskala})
	--(6.0238,{-0.0301*\yskala})
	--(6.0476,{-0.0635*\yskala})
	--(6.0714,{-0.1003*\yskala})
	--(6.0952,{-0.1407*\yskala})
	--(6.1190,{-0.1849*\yskala})
	--(6.1429,{-0.2332*\yskala})
	--(6.1667,{-0.2857*\yskala})
	--(6.1905,{-0.3426*\yskala})
	--(6.2143,{-0.4042*\yskala})
	--(6.2381,{-0.4707*\yskala})
	--(6.2619,{-0.5423*\yskala})
	--(6.2857,{-0.6193*\yskala})
	--(6.3095,{-0.7020*\yskala})
	--(6.3333,{-0.7906*\yskala})
	--(6.3571,{-0.8853*\yskala})
	--(6.3810,{-0.9865*\yskala})
	--(6.4048,{-1.0945*\yskala})
	--(6.4286,{-1.2094*\yskala})
	--(6.4524,{-1.3317*\yskala})
	--(6.4762,{-1.4617*\yskala})
	--(6.5000,{-1.5996*\yskala});
}
\def\basistwo{
\draw[line width=1.4pt,color=red] (-0.5000,{8.7979*\yskala})
	--(-0.4762,{8.0815*\yskala})
	--(-0.4524,{7.4022*\yskala})
	--(-0.4286,{6.7587*\yskala})
	--(-0.4048,{6.1496*\yskala})
	--(-0.3810,{5.5739*\yskala})
	--(-0.3571,{5.0303*\yskala})
	--(-0.3333,{4.5176*\yskala})
	--(-0.3095,{4.0347*\yskala})
	--(-0.2857,{3.5805*\yskala})
	--(-0.2619,{3.1539*\yskala})
	--(-0.2381,{2.7539*\yskala})
	--(-0.2143,{2.3794*\yskala})
	--(-0.1905,{2.0294*\yskala})
	--(-0.1667,{1.7030*\yskala})
	--(-0.1429,{1.3991*\yskala})
	--(-0.1190,{1.1169*\yskala})
	--(-0.0952,{0.8555*\yskala})
	--(-0.0714,{0.6139*\yskala})
	--(-0.0476,{0.3914*\yskala})
	--(-0.0238,{0.1870*\yskala})
	--(0.0000,{-0.0000*\yskala})
	--(0.0238,{-0.1704*\yskala})
	--(0.0476,{-0.3250*\yskala})
	--(0.0714,{-0.4645*\yskala})
	--(0.0952,{-0.5897*\yskala})
	--(0.1190,{-0.7012*\yskala})
	--(0.1429,{-0.7998*\yskala})
	--(0.1667,{-0.8861*\yskala})
	--(0.1905,{-0.9607*\yskala})
	--(0.2143,{-1.0242*\yskala})
	--(0.2381,{-1.0774*\yskala})
	--(0.2619,{-1.1207*\yskala})
	--(0.2857,{-1.1547*\yskala})
	--(0.3095,{-1.1800*\yskala})
	--(0.3333,{-1.1971*\yskala})
	--(0.3571,{-1.2065*\yskala})
	--(0.3810,{-1.2087*\yskala})
	--(0.4048,{-1.2041*\yskala})
	--(0.4286,{-1.1934*\yskala})
	--(0.4524,{-1.1768*\yskala})
	--(0.4762,{-1.1548*\yskala})
	--(0.5000,{-1.1279*\yskala})
	--(0.5238,{-1.0964*\yskala})
	--(0.5476,{-1.0608*\yskala})
	--(0.5714,{-1.0213*\yskala})
	--(0.5952,{-0.9784*\yskala})
	--(0.6190,{-0.9323*\yskala})
	--(0.6429,{-0.8835*\yskala})
	--(0.6667,{-0.8322*\yskala})
	--(0.6905,{-0.7787*\yskala})
	--(0.7143,{-0.7233*\yskala})
	--(0.7381,{-0.6664*\yskala})
	--(0.7619,{-0.6080*\yskala})
	--(0.7857,{-0.5486*\yskala})
	--(0.8095,{-0.4883*\yskala})
	--(0.8333,{-0.4274*\yskala})
	--(0.8571,{-0.3660*\yskala})
	--(0.8810,{-0.3045*\yskala})
	--(0.9048,{-0.2429*\yskala})
	--(0.9286,{-0.1815*\yskala})
	--(0.9524,{-0.1205*\yskala})
	--(0.9762,{-0.0599*\yskala})
	--(1.0000,{0.0000*\yskala})
	--(1.0238,{0.0591*\yskala})
	--(1.0476,{0.1173*\yskala})
	--(1.0714,{0.1744*\yskala})
	--(1.0952,{0.2303*\yskala})
	--(1.1190,{0.2849*\yskala})
	--(1.1429,{0.3381*\yskala})
	--(1.1667,{0.3899*\yskala})
	--(1.1905,{0.4400*\yskala})
	--(1.2143,{0.4886*\yskala})
	--(1.2381,{0.5354*\yskala})
	--(1.2619,{0.5804*\yskala})
	--(1.2857,{0.6235*\yskala})
	--(1.3095,{0.6648*\yskala})
	--(1.3333,{0.7042*\yskala})
	--(1.3571,{0.7415*\yskala})
	--(1.3810,{0.7769*\yskala})
	--(1.4048,{0.8102*\yskala})
	--(1.4286,{0.8415*\yskala})
	--(1.4524,{0.8707*\yskala})
	--(1.4762,{0.8978*\yskala})
	--(1.5000,{0.9229*\yskala})
	--(1.5238,{0.9458*\yskala})
	--(1.5476,{0.9667*\yskala})
	--(1.5714,{0.9855*\yskala})
	--(1.5952,{1.0022*\yskala})
	--(1.6190,{1.0169*\yskala})
	--(1.6429,{1.0296*\yskala})
	--(1.6667,{1.0402*\yskala})
	--(1.6905,{1.0489*\yskala})
	--(1.7143,{1.0557*\yskala})
	--(1.7381,{1.0605*\yskala})
	--(1.7619,{1.0635*\yskala})
	--(1.7857,{1.0646*\yskala})
	--(1.8095,{1.0640*\yskala})
	--(1.8333,{1.0616*\yskala})
	--(1.8571,{1.0575*\yskala})
	--(1.8810,{1.0517*\yskala})
	--(1.9048,{1.0444*\yskala})
	--(1.9286,{1.0355*\yskala})
	--(1.9524,{1.0251*\yskala})
	--(1.9762,{1.0132*\yskala})
	--(2.0000,{1.0000*\yskala})
	--(2.0238,{0.9855*\yskala})
	--(2.0476,{0.9697*\yskala})
	--(2.0714,{0.9527*\yskala})
	--(2.0952,{0.9345*\yskala})
	--(2.1190,{0.9153*\yskala})
	--(2.1429,{0.8950*\yskala})
	--(2.1667,{0.8738*\yskala})
	--(2.1905,{0.8518*\yskala})
	--(2.2143,{0.8288*\yskala})
	--(2.2381,{0.8052*\yskala})
	--(2.2619,{0.7808*\yskala})
	--(2.2857,{0.7558*\yskala})
	--(2.3095,{0.7302*\yskala})
	--(2.3333,{0.7042*\yskala})
	--(2.3571,{0.6776*\yskala})
	--(2.3810,{0.6507*\yskala})
	--(2.4048,{0.6235*\yskala})
	--(2.4286,{0.5961*\yskala})
	--(2.4524,{0.5684*\yskala})
	--(2.4762,{0.5406*\yskala})
	--(2.5000,{0.5127*\yskala})
	--(2.5238,{0.4848*\yskala})
	--(2.5476,{0.4569*\yskala})
	--(2.5714,{0.4292*\yskala})
	--(2.5952,{0.4015*\yskala})
	--(2.6190,{0.3741*\yskala})
	--(2.6429,{0.3469*\yskala})
	--(2.6667,{0.3201*\yskala})
	--(2.6905,{0.2936*\yskala})
	--(2.7143,{0.2674*\yskala})
	--(2.7381,{0.2418*\yskala})
	--(2.7619,{0.2166*\yskala})
	--(2.7857,{0.1919*\yskala})
	--(2.8095,{0.1678*\yskala})
	--(2.8333,{0.1444*\yskala})
	--(2.8571,{0.1215*\yskala})
	--(2.8810,{0.0994*\yskala})
	--(2.9048,{0.0780*\yskala})
	--(2.9286,{0.0573*\yskala})
	--(2.9524,{0.0374*\yskala})
	--(2.9762,{0.0183*\yskala})
	--(3.0000,{-0.0000*\yskala})
	--(3.0238,{-0.0174*\yskala})
	--(3.0476,{-0.0340*\yskala})
	--(3.0714,{-0.0497*\yskala})
	--(3.0952,{-0.0644*\yskala})
	--(3.1190,{-0.0783*\yskala})
	--(3.1429,{-0.0912*\yskala})
	--(3.1667,{-0.1031*\yskala})
	--(3.1905,{-0.1141*\yskala})
	--(3.2143,{-0.1242*\yskala})
	--(3.2381,{-0.1333*\yskala})
	--(3.2619,{-0.1414*\yskala})
	--(3.2857,{-0.1486*\yskala})
	--(3.3095,{-0.1548*\yskala})
	--(3.3333,{-0.1600*\yskala})
	--(3.3571,{-0.1643*\yskala})
	--(3.3810,{-0.1677*\yskala})
	--(3.4048,{-0.1701*\yskala})
	--(3.4286,{-0.1717*\yskala})
	--(3.4524,{-0.1723*\yskala})
	--(3.4762,{-0.1720*\yskala})
	--(3.5000,{-0.1709*\yskala})
	--(3.5238,{-0.1689*\yskala})
	--(3.5476,{-0.1661*\yskala})
	--(3.5714,{-0.1626*\yskala})
	--(3.5952,{-0.1582*\yskala})
	--(3.6190,{-0.1531*\yskala})
	--(3.6429,{-0.1473*\yskala})
	--(3.6667,{-0.1408*\yskala})
	--(3.6905,{-0.1337*\yskala})
	--(3.7143,{-0.1260*\yskala})
	--(3.7381,{-0.1177*\yskala})
	--(3.7619,{-0.1088*\yskala})
	--(3.7857,{-0.0995*\yskala})
	--(3.8095,{-0.0897*\yskala})
	--(3.8333,{-0.0794*\yskala})
	--(3.8571,{-0.0688*\yskala})
	--(3.8810,{-0.0579*\yskala})
	--(3.9048,{-0.0467*\yskala})
	--(3.9286,{-0.0353*\yskala})
	--(3.9524,{-0.0237*\yskala})
	--(3.9762,{-0.0119*\yskala})
	--(4.0000,{0.0000*\yskala})
	--(4.0238,{0.0119*\yskala})
	--(4.0476,{0.0238*\yskala})
	--(4.0714,{0.0357*\yskala})
	--(4.0952,{0.0475*\yskala})
	--(4.1190,{0.0591*\yskala})
	--(4.1429,{0.0705*\yskala})
	--(4.1667,{0.0817*\yskala})
	--(4.1905,{0.0925*\yskala})
	--(4.2143,{0.1030*\yskala})
	--(4.2381,{0.1131*\yskala})
	--(4.2619,{0.1228*\yskala})
	--(4.2857,{0.1320*\yskala})
	--(4.3095,{0.1406*\yskala})
	--(4.3333,{0.1486*\yskala})
	--(4.3571,{0.1560*\yskala})
	--(4.3810,{0.1627*\yskala})
	--(4.4048,{0.1687*\yskala})
	--(4.4286,{0.1739*\yskala})
	--(4.4524,{0.1783*\yskala})
	--(4.4762,{0.1819*\yskala})
	--(4.5000,{0.1846*\yskala})
	--(4.5238,{0.1863*\yskala})
	--(4.5476,{0.1872*\yskala})
	--(4.5714,{0.1870*\yskala})
	--(4.5952,{0.1858*\yskala})
	--(4.6190,{0.1836*\yskala})
	--(4.6429,{0.1804*\yskala})
	--(4.6667,{0.1760*\yskala})
	--(4.6905,{0.1706*\yskala})
	--(4.7143,{0.1641*\yskala})
	--(4.7381,{0.1564*\yskala})
	--(4.7619,{0.1477*\yskala})
	--(4.7857,{0.1378*\yskala})
	--(4.8095,{0.1268*\yskala})
	--(4.8333,{0.1147*\yskala})
	--(4.8571,{0.1014*\yskala})
	--(4.8810,{0.0871*\yskala})
	--(4.9048,{0.0717*\yskala})
	--(4.9286,{0.0553*\yskala})
	--(4.9524,{0.0378*\yskala})
	--(4.9762,{0.0194*\yskala})
	--(5.0000,{-0.0000*\yskala})
	--(5.0238,{-0.0203*\yskala})
	--(5.0476,{-0.0414*\yskala})
	--(5.0714,{-0.0633*\yskala})
	--(5.0952,{-0.0860*\yskala})
	--(5.1190,{-0.1092*\yskala})
	--(5.1429,{-0.1331*\yskala})
	--(5.1667,{-0.1575*\yskala})
	--(5.1905,{-0.1822*\yskala})
	--(5.2143,{-0.2072*\yskala})
	--(5.2381,{-0.2325*\yskala})
	--(5.2619,{-0.2578*\yskala})
	--(5.2857,{-0.2830*\yskala})
	--(5.3095,{-0.3081*\yskala})
	--(5.3333,{-0.3329*\yskala})
	--(5.3571,{-0.3572*\yskala})
	--(5.3810,{-0.3808*\yskala})
	--(5.4048,{-0.4037*\yskala})
	--(5.4286,{-0.4255*\yskala})
	--(5.4524,{-0.4463*\yskala})
	--(5.4762,{-0.4656*\yskala})
	--(5.5000,{-0.4834*\yskala})
	--(5.5238,{-0.4994*\yskala})
	--(5.5476,{-0.5134*\yskala})
	--(5.5714,{-0.5251*\yskala})
	--(5.5952,{-0.5343*\yskala})
	--(5.6190,{-0.5407*\yskala})
	--(5.6429,{-0.5441*\yskala})
	--(5.6667,{-0.5441*\yskala})
	--(5.6905,{-0.5405*\yskala})
	--(5.7143,{-0.5329*\yskala})
	--(5.7381,{-0.5211*\yskala})
	--(5.7619,{-0.5046*\yskala})
	--(5.7857,{-0.4831*\yskala})
	--(5.8095,{-0.4563*\yskala})
	--(5.8333,{-0.4238*\yskala})
	--(5.8571,{-0.3851*\yskala})
	--(5.8810,{-0.3399*\yskala})
	--(5.9048,{-0.2877*\yskala})
	--(5.9286,{-0.2280*\yskala})
	--(5.9524,{-0.1605*\yskala})
	--(5.9762,{-0.0847*\yskala})
	--(6.0000,{0.0000*\yskala})
	--(6.0238,{0.0940*\yskala})
	--(6.0476,{0.1980*\yskala})
	--(6.0714,{0.3123*\yskala})
	--(6.0952,{0.4377*\yskala})
	--(6.1190,{0.5746*\yskala})
	--(6.1429,{0.7237*\yskala})
	--(6.1667,{0.8855*\yskala})
	--(6.1905,{1.0608*\yskala})
	--(6.2143,{1.2502*\yskala})
	--(6.2381,{1.4543*\yskala})
	--(6.2619,{1.6739*\yskala})
	--(6.2857,{1.9096*\yskala})
	--(6.3095,{2.1623*\yskala})
	--(6.3333,{2.4326*\yskala})
	--(6.3571,{2.7213*\yskala})
	--(6.3810,{3.0293*\yskala})
	--(6.4048,{3.3574*\yskala})
	--(6.4286,{3.7064*\yskala})
	--(6.4524,{4.0771*\yskala})
	--(6.4762,{4.4706*\yskala})
	--(6.5000,{4.8877*\yskala});
}
\def\basisthree{
\draw[line width=1.4pt,color=red] (-0.5000,{-8.3789*\yskala})
	--(-0.4762,{-7.6756*\yskala})
	--(-0.4524,{-7.0108*\yskala})
	--(-0.4286,{-6.3832*\yskala})
	--(-0.4048,{-5.7913*\yskala})
	--(-0.3810,{-5.2337*\yskala})
	--(-0.3571,{-4.7092*\yskala})
	--(-0.3333,{-4.2164*\yskala})
	--(-0.3095,{-3.7541*\yskala})
	--(-0.2857,{-3.3211*\yskala})
	--(-0.2619,{-2.9161*\yskala})
	--(-0.2381,{-2.5379*\yskala})
	--(-0.2143,{-2.1855*\yskala})
	--(-0.1905,{-1.8578*\yskala})
	--(-0.1667,{-1.5536*\yskala})
	--(-0.1429,{-1.2719*\yskala})
	--(-0.1190,{-1.0118*\yskala})
	--(-0.0952,{-0.7721*\yskala})
	--(-0.0714,{-0.5520*\yskala})
	--(-0.0476,{-0.3506*\yskala})
	--(-0.0238,{-0.1669*\yskala})
	--(0.0000,{0.0000*\yskala})
	--(0.0238,{0.1509*\yskala})
	--(0.0476,{0.2866*\yskala})
	--(0.0714,{0.4079*\yskala})
	--(0.0952,{0.5156*\yskala})
	--(0.1190,{0.6104*\yskala})
	--(0.1429,{0.6932*\yskala})
	--(0.1667,{0.7644*\yskala})
	--(0.1905,{0.8250*\yskala})
	--(0.2143,{0.8754*\yskala})
	--(0.2381,{0.9164*\yskala})
	--(0.2619,{0.9485*\yskala})
	--(0.2857,{0.9724*\yskala})
	--(0.3095,{0.9885*\yskala})
	--(0.3333,{0.9976*\yskala})
	--(0.3571,{1.0000*\yskala})
	--(0.3810,{0.9962*\yskala})
	--(0.4048,{0.9869*\yskala})
	--(0.4286,{0.9724*\yskala})
	--(0.4524,{0.9532*\yskala})
	--(0.4762,{0.9297*\yskala})
	--(0.5000,{0.9023*\yskala})
	--(0.5238,{0.8715*\yskala})
	--(0.5476,{0.8376*\yskala})
	--(0.5714,{0.8010*\yskala})
	--(0.5952,{0.7620*\yskala})
	--(0.6190,{0.7210*\yskala})
	--(0.6429,{0.6782*\yskala})
	--(0.6667,{0.6340*\yskala})
	--(0.6905,{0.5887*\yskala})
	--(0.7143,{0.5425*\yskala})
	--(0.7381,{0.4957*\yskala})
	--(0.7619,{0.4485*\yskala})
	--(0.7857,{0.4011*\yskala})
	--(0.8095,{0.3538*\yskala})
	--(0.8333,{0.3068*\yskala})
	--(0.8571,{0.2603*\yskala})
	--(0.8810,{0.2144*\yskala})
	--(0.9048,{0.1693*\yskala})
	--(0.9286,{0.1252*\yskala})
	--(0.9524,{0.0822*\yskala})
	--(0.9762,{0.0404*\yskala})
	--(1.0000,{-0.0000*\yskala})
	--(1.0238,{-0.0389*\yskala})
	--(1.0476,{-0.0763*\yskala})
	--(1.0714,{-0.1119*\yskala})
	--(1.0952,{-0.1458*\yskala})
	--(1.1190,{-0.1779*\yskala})
	--(1.1429,{-0.2081*\yskala})
	--(1.1667,{-0.2363*\yskala})
	--(1.1905,{-0.2625*\yskala})
	--(1.2143,{-0.2866*\yskala})
	--(1.2381,{-0.3087*\yskala})
	--(1.2619,{-0.3286*\yskala})
	--(1.2857,{-0.3464*\yskala})
	--(1.3095,{-0.3621*\yskala})
	--(1.3333,{-0.3756*\yskala})
	--(1.3571,{-0.3869*\yskala})
	--(1.3810,{-0.3961*\yskala})
	--(1.4048,{-0.4031*\yskala})
	--(1.4286,{-0.4080*\yskala})
	--(1.4524,{-0.4108*\yskala})
	--(1.4762,{-0.4115*\yskala})
	--(1.5000,{-0.4102*\yskala})
	--(1.5238,{-0.4068*\yskala})
	--(1.5476,{-0.4015*\yskala})
	--(1.5714,{-0.3942*\yskala})
	--(1.5952,{-0.3850*\yskala})
	--(1.6190,{-0.3740*\yskala})
	--(1.6429,{-0.3613*\yskala})
	--(1.6667,{-0.3467*\yskala})
	--(1.6905,{-0.3306*\yskala})
	--(1.7143,{-0.3128*\yskala})
	--(1.7381,{-0.2935*\yskala})
	--(1.7619,{-0.2727*\yskala})
	--(1.7857,{-0.2505*\yskala})
	--(1.8095,{-0.2270*\yskala})
	--(1.8333,{-0.2022*\yskala})
	--(1.8571,{-0.1762*\yskala})
	--(1.8810,{-0.1492*\yskala})
	--(1.9048,{-0.1211*\yskala})
	--(1.9286,{-0.0920*\yskala})
	--(1.9524,{-0.0621*\yskala})
	--(1.9762,{-0.0314*\yskala})
	--(2.0000,{0.0000*\yskala})
	--(2.0238,{0.0320*\yskala})
	--(2.0476,{0.0646*\yskala})
	--(2.0714,{0.0977*\yskala})
	--(2.0952,{0.1312*\yskala})
	--(2.1190,{0.1649*\yskala})
	--(2.1429,{0.1989*\yskala})
	--(2.1667,{0.2330*\yskala})
	--(2.1905,{0.2672*\yskala})
	--(2.2143,{0.3014*\yskala})
	--(2.2381,{0.3355*\yskala})
	--(2.2619,{0.3694*\yskala})
	--(2.2857,{0.4031*\yskala})
	--(2.3095,{0.4365*\yskala})
	--(2.3333,{0.4694*\yskala})
	--(2.3571,{0.5020*\yskala})
	--(2.3810,{0.5339*\yskala})
	--(2.4048,{0.5653*\yskala})
	--(2.4286,{0.5961*\yskala})
	--(2.4524,{0.6260*\yskala})
	--(2.4762,{0.6552*\yskala})
	--(2.5000,{0.6836*\yskala})
	--(2.5238,{0.7110*\yskala})
	--(2.5476,{0.7375*\yskala})
	--(2.5714,{0.7629*\yskala})
	--(2.5952,{0.7873*\yskala})
	--(2.6190,{0.8106*\yskala})
	--(2.6429,{0.8327*\yskala})
	--(2.6667,{0.8535*\yskala})
	--(2.6905,{0.8731*\yskala})
	--(2.7143,{0.8915*\yskala})
	--(2.7381,{0.9085*\yskala})
	--(2.7619,{0.9241*\yskala})
	--(2.7857,{0.9383*\yskala})
	--(2.8095,{0.9511*\yskala})
	--(2.8333,{0.9625*\yskala})
	--(2.8571,{0.9724*\yskala})
	--(2.8810,{0.9808*\yskala})
	--(2.9048,{0.9877*\yskala})
	--(2.9286,{0.9931*\yskala})
	--(2.9524,{0.9969*\yskala})
	--(2.9762,{0.9992*\yskala})
	--(3.0000,{1.0000*\yskala})
	--(3.0238,{0.9992*\yskala})
	--(3.0476,{0.9969*\yskala})
	--(3.0714,{0.9931*\yskala})
	--(3.0952,{0.9877*\yskala})
	--(3.1190,{0.9808*\yskala})
	--(3.1429,{0.9724*\yskala})
	--(3.1667,{0.9625*\yskala})
	--(3.1905,{0.9511*\yskala})
	--(3.2143,{0.9383*\yskala})
	--(3.2381,{0.9241*\yskala})
	--(3.2619,{0.9085*\yskala})
	--(3.2857,{0.8915*\yskala})
	--(3.3095,{0.8731*\yskala})
	--(3.3333,{0.8535*\yskala})
	--(3.3571,{0.8327*\yskala})
	--(3.3810,{0.8106*\yskala})
	--(3.4048,{0.7873*\yskala})
	--(3.4286,{0.7629*\yskala})
	--(3.4524,{0.7375*\yskala})
	--(3.4762,{0.7110*\yskala})
	--(3.5000,{0.6836*\yskala})
	--(3.5238,{0.6552*\yskala})
	--(3.5476,{0.6260*\yskala})
	--(3.5714,{0.5961*\yskala})
	--(3.5952,{0.5653*\yskala})
	--(3.6190,{0.5339*\yskala})
	--(3.6429,{0.5020*\yskala})
	--(3.6667,{0.4694*\yskala})
	--(3.6905,{0.4365*\yskala})
	--(3.7143,{0.4031*\yskala})
	--(3.7381,{0.3694*\yskala})
	--(3.7619,{0.3355*\yskala})
	--(3.7857,{0.3014*\yskala})
	--(3.8095,{0.2672*\yskala})
	--(3.8333,{0.2330*\yskala})
	--(3.8571,{0.1989*\yskala})
	--(3.8810,{0.1649*\yskala})
	--(3.9048,{0.1312*\yskala})
	--(3.9286,{0.0977*\yskala})
	--(3.9524,{0.0646*\yskala})
	--(3.9762,{0.0320*\yskala})
	--(4.0000,{-0.0000*\yskala})
	--(4.0238,{-0.0314*\yskala})
	--(4.0476,{-0.0621*\yskala})
	--(4.0714,{-0.0920*\yskala})
	--(4.0952,{-0.1211*\yskala})
	--(4.1190,{-0.1492*\yskala})
	--(4.1429,{-0.1762*\yskala})
	--(4.1667,{-0.2022*\yskala})
	--(4.1905,{-0.2270*\yskala})
	--(4.2143,{-0.2505*\yskala})
	--(4.2381,{-0.2727*\yskala})
	--(4.2619,{-0.2935*\yskala})
	--(4.2857,{-0.3128*\yskala})
	--(4.3095,{-0.3306*\yskala})
	--(4.3333,{-0.3467*\yskala})
	--(4.3571,{-0.3613*\yskala})
	--(4.3810,{-0.3740*\yskala})
	--(4.4048,{-0.3850*\yskala})
	--(4.4286,{-0.3942*\yskala})
	--(4.4524,{-0.4015*\yskala})
	--(4.4762,{-0.4068*\yskala})
	--(4.5000,{-0.4102*\yskala})
	--(4.5238,{-0.4115*\yskala})
	--(4.5476,{-0.4108*\yskala})
	--(4.5714,{-0.4080*\yskala})
	--(4.5952,{-0.4031*\yskala})
	--(4.6190,{-0.3961*\yskala})
	--(4.6429,{-0.3869*\yskala})
	--(4.6667,{-0.3756*\yskala})
	--(4.6905,{-0.3621*\yskala})
	--(4.7143,{-0.3464*\yskala})
	--(4.7381,{-0.3286*\yskala})
	--(4.7619,{-0.3087*\yskala})
	--(4.7857,{-0.2866*\yskala})
	--(4.8095,{-0.2625*\yskala})
	--(4.8333,{-0.2363*\yskala})
	--(4.8571,{-0.2081*\yskala})
	--(4.8810,{-0.1779*\yskala})
	--(4.9048,{-0.1458*\yskala})
	--(4.9286,{-0.1119*\yskala})
	--(4.9524,{-0.0763*\yskala})
	--(4.9762,{-0.0389*\yskala})
	--(5.0000,{0.0000*\yskala})
	--(5.0238,{0.0404*\yskala})
	--(5.0476,{0.0822*\yskala})
	--(5.0714,{0.1252*\yskala})
	--(5.0952,{0.1693*\yskala})
	--(5.1190,{0.2144*\yskala})
	--(5.1429,{0.2603*\yskala})
	--(5.1667,{0.3068*\yskala})
	--(5.1905,{0.3538*\yskala})
	--(5.2143,{0.4011*\yskala})
	--(5.2381,{0.4485*\yskala})
	--(5.2619,{0.4957*\yskala})
	--(5.2857,{0.5425*\yskala})
	--(5.3095,{0.5887*\yskala})
	--(5.3333,{0.6340*\yskala})
	--(5.3571,{0.6782*\yskala})
	--(5.3810,{0.7210*\yskala})
	--(5.4048,{0.7620*\yskala})
	--(5.4286,{0.8010*\yskala})
	--(5.4524,{0.8376*\yskala})
	--(5.4762,{0.8715*\yskala})
	--(5.5000,{0.9023*\yskala})
	--(5.5238,{0.9297*\yskala})
	--(5.5476,{0.9532*\yskala})
	--(5.5714,{0.9724*\yskala})
	--(5.5952,{0.9869*\yskala})
	--(5.6190,{0.9962*\yskala})
	--(5.6429,{1.0000*\yskala})
	--(5.6667,{0.9976*\yskala})
	--(5.6905,{0.9885*\yskala})
	--(5.7143,{0.9724*\yskala})
	--(5.7381,{0.9485*\yskala})
	--(5.7619,{0.9164*\yskala})
	--(5.7857,{0.8754*\yskala})
	--(5.8095,{0.8250*\yskala})
	--(5.8333,{0.7644*\yskala})
	--(5.8571,{0.6932*\yskala})
	--(5.8810,{0.6104*\yskala})
	--(5.9048,{0.5156*\yskala})
	--(5.9286,{0.4079*\yskala})
	--(5.9524,{0.2866*\yskala})
	--(5.9762,{0.1509*\yskala})
	--(6.0000,{-0.0000*\yskala})
	--(6.0238,{-0.1669*\yskala})
	--(6.0476,{-0.3506*\yskala})
	--(6.0714,{-0.5520*\yskala})
	--(6.0952,{-0.7721*\yskala})
	--(6.1190,{-1.0118*\yskala})
	--(6.1429,{-1.2719*\yskala})
	--(6.1667,{-1.5536*\yskala})
	--(6.1905,{-1.8578*\yskala})
	--(6.2143,{-2.1855*\yskala})
	--(6.2381,{-2.5379*\yskala})
	--(6.2619,{-2.9161*\yskala})
	--(6.2857,{-3.3211*\yskala})
	--(6.3095,{-3.7541*\yskala})
	--(6.3333,{-4.2164*\yskala})
	--(6.3571,{-4.7092*\yskala})
	--(6.3810,{-5.2337*\yskala})
	--(6.4048,{-5.7913*\yskala})
	--(6.4286,{-6.3832*\yskala})
	--(6.4524,{-7.0108*\yskala})
	--(6.4762,{-7.6756*\yskala})
	--(6.5000,{-8.3789*\yskala});
}
\def\basisfour{
\draw[line width=1.4pt,color=red] (-0.5000,{4.8877*\yskala})
	--(-0.4762,{4.4706*\yskala})
	--(-0.4524,{4.0771*\yskala})
	--(-0.4286,{3.7064*\yskala})
	--(-0.4048,{3.3574*\yskala})
	--(-0.3810,{3.0293*\yskala})
	--(-0.3571,{2.7213*\yskala})
	--(-0.3333,{2.4326*\yskala})
	--(-0.3095,{2.1623*\yskala})
	--(-0.2857,{1.9096*\yskala})
	--(-0.2619,{1.6739*\yskala})
	--(-0.2381,{1.4543*\yskala})
	--(-0.2143,{1.2502*\yskala})
	--(-0.1905,{1.0608*\yskala})
	--(-0.1667,{0.8855*\yskala})
	--(-0.1429,{0.7237*\yskala})
	--(-0.1190,{0.5746*\yskala})
	--(-0.0952,{0.4377*\yskala})
	--(-0.0714,{0.3123*\yskala})
	--(-0.0476,{0.1980*\yskala})
	--(-0.0238,{0.0940*\yskala})
	--(0.0000,{-0.0000*\yskala})
	--(0.0238,{-0.0847*\yskala})
	--(0.0476,{-0.1605*\yskala})
	--(0.0714,{-0.2280*\yskala})
	--(0.0952,{-0.2877*\yskala})
	--(0.1190,{-0.3399*\yskala})
	--(0.1429,{-0.3851*\yskala})
	--(0.1667,{-0.4238*\yskala})
	--(0.1905,{-0.4563*\yskala})
	--(0.2143,{-0.4831*\yskala})
	--(0.2381,{-0.5046*\yskala})
	--(0.2619,{-0.5211*\yskala})
	--(0.2857,{-0.5329*\yskala})
	--(0.3095,{-0.5405*\yskala})
	--(0.3333,{-0.5441*\yskala})
	--(0.3571,{-0.5441*\yskala})
	--(0.3810,{-0.5407*\yskala})
	--(0.4048,{-0.5343*\yskala})
	--(0.4286,{-0.5251*\yskala})
	--(0.4524,{-0.5134*\yskala})
	--(0.4762,{-0.4994*\yskala})
	--(0.5000,{-0.4834*\yskala})
	--(0.5238,{-0.4656*\yskala})
	--(0.5476,{-0.4463*\yskala})
	--(0.5714,{-0.4255*\yskala})
	--(0.5952,{-0.4037*\yskala})
	--(0.6190,{-0.3808*\yskala})
	--(0.6429,{-0.3572*\yskala})
	--(0.6667,{-0.3329*\yskala})
	--(0.6905,{-0.3081*\yskala})
	--(0.7143,{-0.2830*\yskala})
	--(0.7381,{-0.2578*\yskala})
	--(0.7619,{-0.2325*\yskala})
	--(0.7857,{-0.2072*\yskala})
	--(0.8095,{-0.1822*\yskala})
	--(0.8333,{-0.1575*\yskala})
	--(0.8571,{-0.1331*\yskala})
	--(0.8810,{-0.1092*\yskala})
	--(0.9048,{-0.0860*\yskala})
	--(0.9286,{-0.0633*\yskala})
	--(0.9524,{-0.0414*\yskala})
	--(0.9762,{-0.0203*\yskala})
	--(1.0000,{0.0000*\yskala})
	--(1.0238,{0.0194*\yskala})
	--(1.0476,{0.0378*\yskala})
	--(1.0714,{0.0553*\yskala})
	--(1.0952,{0.0717*\yskala})
	--(1.1190,{0.0871*\yskala})
	--(1.1429,{0.1014*\yskala})
	--(1.1667,{0.1147*\yskala})
	--(1.1905,{0.1268*\yskala})
	--(1.2143,{0.1378*\yskala})
	--(1.2381,{0.1477*\yskala})
	--(1.2619,{0.1564*\yskala})
	--(1.2857,{0.1641*\yskala})
	--(1.3095,{0.1706*\yskala})
	--(1.3333,{0.1760*\yskala})
	--(1.3571,{0.1804*\yskala})
	--(1.3810,{0.1836*\yskala})
	--(1.4048,{0.1858*\yskala})
	--(1.4286,{0.1870*\yskala})
	--(1.4524,{0.1872*\yskala})
	--(1.4762,{0.1863*\yskala})
	--(1.5000,{0.1846*\yskala})
	--(1.5238,{0.1819*\yskala})
	--(1.5476,{0.1783*\yskala})
	--(1.5714,{0.1739*\yskala})
	--(1.5952,{0.1687*\yskala})
	--(1.6190,{0.1627*\yskala})
	--(1.6429,{0.1560*\yskala})
	--(1.6667,{0.1486*\yskala})
	--(1.6905,{0.1406*\yskala})
	--(1.7143,{0.1320*\yskala})
	--(1.7381,{0.1228*\yskala})
	--(1.7619,{0.1131*\yskala})
	--(1.7857,{0.1030*\yskala})
	--(1.8095,{0.0925*\yskala})
	--(1.8333,{0.0817*\yskala})
	--(1.8571,{0.0705*\yskala})
	--(1.8810,{0.0591*\yskala})
	--(1.9048,{0.0475*\yskala})
	--(1.9286,{0.0357*\yskala})
	--(1.9524,{0.0238*\yskala})
	--(1.9762,{0.0119*\yskala})
	--(2.0000,{-0.0000*\yskala})
	--(2.0238,{-0.0119*\yskala})
	--(2.0476,{-0.0237*\yskala})
	--(2.0714,{-0.0353*\yskala})
	--(2.0952,{-0.0467*\yskala})
	--(2.1190,{-0.0579*\yskala})
	--(2.1429,{-0.0688*\yskala})
	--(2.1667,{-0.0794*\yskala})
	--(2.1905,{-0.0897*\yskala})
	--(2.2143,{-0.0995*\yskala})
	--(2.2381,{-0.1088*\yskala})
	--(2.2619,{-0.1177*\yskala})
	--(2.2857,{-0.1260*\yskala})
	--(2.3095,{-0.1337*\yskala})
	--(2.3333,{-0.1408*\yskala})
	--(2.3571,{-0.1473*\yskala})
	--(2.3810,{-0.1531*\yskala})
	--(2.4048,{-0.1582*\yskala})
	--(2.4286,{-0.1626*\yskala})
	--(2.4524,{-0.1661*\yskala})
	--(2.4762,{-0.1689*\yskala})
	--(2.5000,{-0.1709*\yskala})
	--(2.5238,{-0.1720*\yskala})
	--(2.5476,{-0.1723*\yskala})
	--(2.5714,{-0.1717*\yskala})
	--(2.5952,{-0.1701*\yskala})
	--(2.6190,{-0.1677*\yskala})
	--(2.6429,{-0.1643*\yskala})
	--(2.6667,{-0.1600*\yskala})
	--(2.6905,{-0.1548*\yskala})
	--(2.7143,{-0.1486*\yskala})
	--(2.7381,{-0.1414*\yskala})
	--(2.7619,{-0.1333*\yskala})
	--(2.7857,{-0.1242*\yskala})
	--(2.8095,{-0.1141*\yskala})
	--(2.8333,{-0.1031*\yskala})
	--(2.8571,{-0.0912*\yskala})
	--(2.8810,{-0.0783*\yskala})
	--(2.9048,{-0.0644*\yskala})
	--(2.9286,{-0.0497*\yskala})
	--(2.9524,{-0.0340*\yskala})
	--(2.9762,{-0.0174*\yskala})
	--(3.0000,{0.0000*\yskala})
	--(3.0238,{0.0183*\yskala})
	--(3.0476,{0.0374*\yskala})
	--(3.0714,{0.0573*\yskala})
	--(3.0952,{0.0780*\yskala})
	--(3.1190,{0.0994*\yskala})
	--(3.1429,{0.1215*\yskala})
	--(3.1667,{0.1444*\yskala})
	--(3.1905,{0.1678*\yskala})
	--(3.2143,{0.1919*\yskala})
	--(3.2381,{0.2166*\yskala})
	--(3.2619,{0.2418*\yskala})
	--(3.2857,{0.2674*\yskala})
	--(3.3095,{0.2936*\yskala})
	--(3.3333,{0.3201*\yskala})
	--(3.3571,{0.3469*\yskala})
	--(3.3810,{0.3741*\yskala})
	--(3.4048,{0.4015*\yskala})
	--(3.4286,{0.4292*\yskala})
	--(3.4524,{0.4569*\yskala})
	--(3.4762,{0.4848*\yskala})
	--(3.5000,{0.5127*\yskala})
	--(3.5238,{0.5406*\yskala})
	--(3.5476,{0.5684*\yskala})
	--(3.5714,{0.5961*\yskala})
	--(3.5952,{0.6235*\yskala})
	--(3.6190,{0.6507*\yskala})
	--(3.6429,{0.6776*\yskala})
	--(3.6667,{0.7042*\yskala})
	--(3.6905,{0.7302*\yskala})
	--(3.7143,{0.7558*\yskala})
	--(3.7381,{0.7808*\yskala})
	--(3.7619,{0.8052*\yskala})
	--(3.7857,{0.8288*\yskala})
	--(3.8095,{0.8518*\yskala})
	--(3.8333,{0.8738*\yskala})
	--(3.8571,{0.8950*\yskala})
	--(3.8810,{0.9153*\yskala})
	--(3.9048,{0.9345*\yskala})
	--(3.9286,{0.9527*\yskala})
	--(3.9524,{0.9697*\yskala})
	--(3.9762,{0.9855*\yskala})
	--(4.0000,{1.0000*\yskala})
	--(4.0238,{1.0132*\yskala})
	--(4.0476,{1.0251*\yskala})
	--(4.0714,{1.0355*\yskala})
	--(4.0952,{1.0444*\yskala})
	--(4.1190,{1.0517*\yskala})
	--(4.1429,{1.0575*\yskala})
	--(4.1667,{1.0616*\yskala})
	--(4.1905,{1.0640*\yskala})
	--(4.2143,{1.0646*\yskala})
	--(4.2381,{1.0635*\yskala})
	--(4.2619,{1.0605*\yskala})
	--(4.2857,{1.0557*\yskala})
	--(4.3095,{1.0489*\yskala})
	--(4.3333,{1.0402*\yskala})
	--(4.3571,{1.0296*\yskala})
	--(4.3810,{1.0169*\yskala})
	--(4.4048,{1.0022*\yskala})
	--(4.4286,{0.9855*\yskala})
	--(4.4524,{0.9667*\yskala})
	--(4.4762,{0.9458*\yskala})
	--(4.5000,{0.9229*\yskala})
	--(4.5238,{0.8978*\yskala})
	--(4.5476,{0.8707*\yskala})
	--(4.5714,{0.8415*\yskala})
	--(4.5952,{0.8102*\yskala})
	--(4.6190,{0.7769*\yskala})
	--(4.6429,{0.7415*\yskala})
	--(4.6667,{0.7042*\yskala})
	--(4.6905,{0.6648*\yskala})
	--(4.7143,{0.6235*\yskala})
	--(4.7381,{0.5804*\yskala})
	--(4.7619,{0.5354*\yskala})
	--(4.7857,{0.4886*\yskala})
	--(4.8095,{0.4400*\yskala})
	--(4.8333,{0.3899*\yskala})
	--(4.8571,{0.3381*\yskala})
	--(4.8810,{0.2849*\yskala})
	--(4.9048,{0.2303*\yskala})
	--(4.9286,{0.1744*\yskala})
	--(4.9524,{0.1173*\yskala})
	--(4.9762,{0.0591*\yskala})
	--(5.0000,{-0.0000*\yskala})
	--(5.0238,{-0.0599*\yskala})
	--(5.0476,{-0.1205*\yskala})
	--(5.0714,{-0.1815*\yskala})
	--(5.0952,{-0.2429*\yskala})
	--(5.1190,{-0.3045*\yskala})
	--(5.1429,{-0.3660*\yskala})
	--(5.1667,{-0.4274*\yskala})
	--(5.1905,{-0.4883*\yskala})
	--(5.2143,{-0.5486*\yskala})
	--(5.2381,{-0.6080*\yskala})
	--(5.2619,{-0.6664*\yskala})
	--(5.2857,{-0.7233*\yskala})
	--(5.3095,{-0.7787*\yskala})
	--(5.3333,{-0.8322*\yskala})
	--(5.3571,{-0.8835*\yskala})
	--(5.3810,{-0.9323*\yskala})
	--(5.4048,{-0.9784*\yskala})
	--(5.4286,{-1.0213*\yskala})
	--(5.4524,{-1.0608*\yskala})
	--(5.4762,{-1.0964*\yskala})
	--(5.5000,{-1.1279*\yskala})
	--(5.5238,{-1.1548*\yskala})
	--(5.5476,{-1.1768*\yskala})
	--(5.5714,{-1.1934*\yskala})
	--(5.5952,{-1.2041*\yskala})
	--(5.6190,{-1.2087*\yskala})
	--(5.6429,{-1.2065*\yskala})
	--(5.6667,{-1.1971*\yskala})
	--(5.6905,{-1.1800*\yskala})
	--(5.7143,{-1.1547*\yskala})
	--(5.7381,{-1.1207*\yskala})
	--(5.7619,{-1.0774*\yskala})
	--(5.7857,{-1.0242*\yskala})
	--(5.8095,{-0.9607*\yskala})
	--(5.8333,{-0.8861*\yskala})
	--(5.8571,{-0.7998*\yskala})
	--(5.8810,{-0.7012*\yskala})
	--(5.9048,{-0.5897*\yskala})
	--(5.9286,{-0.4645*\yskala})
	--(5.9524,{-0.3250*\yskala})
	--(5.9762,{-0.1704*\yskala})
	--(6.0000,{0.0000*\yskala})
	--(6.0238,{0.1870*\yskala})
	--(6.0476,{0.3914*\yskala})
	--(6.0714,{0.6139*\yskala})
	--(6.0952,{0.8555*\yskala})
	--(6.1190,{1.1169*\yskala})
	--(6.1429,{1.3991*\yskala})
	--(6.1667,{1.7030*\yskala})
	--(6.1905,{2.0294*\yskala})
	--(6.2143,{2.3794*\yskala})
	--(6.2381,{2.7539*\yskala})
	--(6.2619,{3.1539*\yskala})
	--(6.2857,{3.5805*\yskala})
	--(6.3095,{4.0347*\yskala})
	--(6.3333,{4.5176*\yskala})
	--(6.3571,{5.0303*\yskala})
	--(6.3810,{5.5739*\yskala})
	--(6.4048,{6.1496*\yskala})
	--(6.4286,{6.7587*\yskala})
	--(6.4524,{7.4022*\yskala})
	--(6.4762,{8.0815*\yskala})
	--(6.5000,{8.7979*\yskala});
}
\def\basisfive{
\draw[line width=1.4pt,color=red] (-0.5000,{-1.5996*\yskala})
	--(-0.4762,{-1.4617*\yskala})
	--(-0.4524,{-1.3317*\yskala})
	--(-0.4286,{-1.2094*\yskala})
	--(-0.4048,{-1.0945*\yskala})
	--(-0.3810,{-0.9865*\yskala})
	--(-0.3571,{-0.8853*\yskala})
	--(-0.3333,{-0.7906*\yskala})
	--(-0.3095,{-0.7020*\yskala})
	--(-0.2857,{-0.6193*\yskala})
	--(-0.2619,{-0.5423*\yskala})
	--(-0.2381,{-0.4707*\yskala})
	--(-0.2143,{-0.4042*\yskala})
	--(-0.1905,{-0.3426*\yskala})
	--(-0.1667,{-0.2857*\yskala})
	--(-0.1429,{-0.2332*\yskala})
	--(-0.1190,{-0.1849*\yskala})
	--(-0.0952,{-0.1407*\yskala})
	--(-0.0714,{-0.1003*\yskala})
	--(-0.0476,{-0.0635*\yskala})
	--(-0.0238,{-0.0301*\yskala})
	--(0.0000,{0.0000*\yskala})
	--(0.0238,{0.0271*\yskala})
	--(0.0476,{0.0513*\yskala})
	--(0.0714,{0.0727*\yskala})
	--(0.0952,{0.0916*\yskala})
	--(0.1190,{0.1081*\yskala})
	--(0.1429,{0.1223*\yskala})
	--(0.1667,{0.1344*\yskala})
	--(0.1905,{0.1446*\yskala})
	--(0.2143,{0.1529*\yskala})
	--(0.2381,{0.1595*\yskala})
	--(0.2619,{0.1644*\yskala})
	--(0.2857,{0.1680*\yskala})
	--(0.3095,{0.1701*\yskala})
	--(0.3333,{0.1710*\yskala})
	--(0.3571,{0.1708*\yskala})
	--(0.3810,{0.1695*\yskala})
	--(0.4048,{0.1672*\yskala})
	--(0.4286,{0.1641*\yskala})
	--(0.4524,{0.1602*\yskala})
	--(0.4762,{0.1556*\yskala})
	--(0.5000,{0.1504*\yskala})
	--(0.5238,{0.1446*\yskala})
	--(0.5476,{0.1384*\yskala})
	--(0.5714,{0.1318*\yskala})
	--(0.5952,{0.1248*\yskala})
	--(0.6190,{0.1176*\yskala})
	--(0.6429,{0.1101*\yskala})
	--(0.6667,{0.1024*\yskala})
	--(0.6905,{0.0946*\yskala})
	--(0.7143,{0.0868*\yskala})
	--(0.7381,{0.0789*\yskala})
	--(0.7619,{0.0711*\yskala})
	--(0.7857,{0.0632*\yskala})
	--(0.8095,{0.0555*\yskala})
	--(0.8333,{0.0479*\yskala})
	--(0.8571,{0.0404*\yskala})
	--(0.8810,{0.0331*\yskala})
	--(0.9048,{0.0260*\yskala})
	--(0.9286,{0.0191*\yskala})
	--(0.9524,{0.0125*\yskala})
	--(0.9762,{0.0061*\yskala})
	--(1.0000,{-0.0000*\yskala})
	--(1.0238,{-0.0058*\yskala})
	--(1.0476,{-0.0113*\yskala})
	--(1.0714,{-0.0165*\yskala})
	--(1.0952,{-0.0213*\yskala})
	--(1.1190,{-0.0259*\yskala})
	--(1.1429,{-0.0301*\yskala})
	--(1.1667,{-0.0339*\yskala})
	--(1.1905,{-0.0374*\yskala})
	--(1.2143,{-0.0406*\yskala})
	--(1.2381,{-0.0434*\yskala})
	--(1.2619,{-0.0458*\yskala})
	--(1.2857,{-0.0480*\yskala})
	--(1.3095,{-0.0498*\yskala})
	--(1.3333,{-0.0512*\yskala})
	--(1.3571,{-0.0523*\yskala})
	--(1.3810,{-0.0532*\yskala})
	--(1.4048,{-0.0537*\yskala})
	--(1.4286,{-0.0539*\yskala})
	--(1.4524,{-0.0538*\yskala})
	--(1.4762,{-0.0534*\yskala})
	--(1.5000,{-0.0527*\yskala})
	--(1.5238,{-0.0518*\yskala})
	--(1.5476,{-0.0507*\yskala})
	--(1.5714,{-0.0493*\yskala})
	--(1.5952,{-0.0477*\yskala})
	--(1.6190,{-0.0458*\yskala})
	--(1.6429,{-0.0438*\yskala})
	--(1.6667,{-0.0416*\yskala})
	--(1.6905,{-0.0392*\yskala})
	--(1.7143,{-0.0367*\yskala})
	--(1.7381,{-0.0341*\yskala})
	--(1.7619,{-0.0313*\yskala})
	--(1.7857,{-0.0284*\yskala})
	--(1.8095,{-0.0254*\yskala})
	--(1.8333,{-0.0223*\yskala})
	--(1.8571,{-0.0192*\yskala})
	--(1.8810,{-0.0161*\yskala})
	--(1.9048,{-0.0129*\yskala})
	--(1.9286,{-0.0096*\yskala})
	--(1.9524,{-0.0064*\yskala})
	--(1.9762,{-0.0032*\yskala})
	--(2.0000,{0.0000*\yskala})
	--(2.0238,{0.0032*\yskala})
	--(2.0476,{0.0063*\yskala})
	--(2.0714,{0.0093*\yskala})
	--(2.0952,{0.0123*\yskala})
	--(2.1190,{0.0151*\yskala})
	--(2.1429,{0.0179*\yskala})
	--(2.1667,{0.0206*\yskala})
	--(2.1905,{0.0231*\yskala})
	--(2.2143,{0.0255*\yskala})
	--(2.2381,{0.0278*\yskala})
	--(2.2619,{0.0299*\yskala})
	--(2.2857,{0.0318*\yskala})
	--(2.3095,{0.0336*\yskala})
	--(2.3333,{0.0352*\yskala})
	--(2.3571,{0.0366*\yskala})
	--(2.3810,{0.0379*\yskala})
	--(2.4048,{0.0389*\yskala})
	--(2.4286,{0.0397*\yskala})
	--(2.4524,{0.0404*\yskala})
	--(2.4762,{0.0408*\yskala})
	--(2.5000,{0.0410*\yskala})
	--(2.5238,{0.0410*\yskala})
	--(2.5476,{0.0408*\yskala})
	--(2.5714,{0.0404*\yskala})
	--(2.5952,{0.0398*\yskala})
	--(2.6190,{0.0389*\yskala})
	--(2.6429,{0.0378*\yskala})
	--(2.6667,{0.0366*\yskala})
	--(2.6905,{0.0351*\yskala})
	--(2.7143,{0.0334*\yskala})
	--(2.7381,{0.0316*\yskala})
	--(2.7619,{0.0295*\yskala})
	--(2.7857,{0.0272*\yskala})
	--(2.8095,{0.0248*\yskala})
	--(2.8333,{0.0222*\yskala})
	--(2.8571,{0.0194*\yskala})
	--(2.8810,{0.0165*\yskala})
	--(2.9048,{0.0135*\yskala})
	--(2.9286,{0.0103*\yskala})
	--(2.9524,{0.0070*\yskala})
	--(2.9762,{0.0035*\yskala})
	--(3.0000,{-0.0000*\yskala})
	--(3.0238,{-0.0036*\yskala})
	--(3.0476,{-0.0073*\yskala})
	--(3.0714,{-0.0110*\yskala})
	--(3.0952,{-0.0148*\yskala})
	--(3.1190,{-0.0186*\yskala})
	--(3.1429,{-0.0224*\yskala})
	--(3.1667,{-0.0262*\yskala})
	--(3.1905,{-0.0300*\yskala})
	--(3.2143,{-0.0338*\yskala})
	--(3.2381,{-0.0375*\yskala})
	--(3.2619,{-0.0411*\yskala})
	--(3.2857,{-0.0446*\yskala})
	--(3.3095,{-0.0480*\yskala})
	--(3.3333,{-0.0512*\yskala})
	--(3.3571,{-0.0543*\yskala})
	--(3.3810,{-0.0572*\yskala})
	--(3.4048,{-0.0599*\yskala})
	--(3.4286,{-0.0624*\yskala})
	--(3.4524,{-0.0647*\yskala})
	--(3.4762,{-0.0667*\yskala})
	--(3.5000,{-0.0684*\yskala})
	--(3.5238,{-0.0698*\yskala})
	--(3.5476,{-0.0708*\yskala})
	--(3.5714,{-0.0715*\yskala})
	--(3.5952,{-0.0719*\yskala})
	--(3.6190,{-0.0718*\yskala})
	--(3.6429,{-0.0713*\yskala})
	--(3.6667,{-0.0704*\yskala})
	--(3.6905,{-0.0690*\yskala})
	--(3.7143,{-0.0672*\yskala})
	--(3.7381,{-0.0648*\yskala})
	--(3.7619,{-0.0619*\yskala})
	--(3.7857,{-0.0585*\yskala})
	--(3.8095,{-0.0545*\yskala})
	--(3.8333,{-0.0499*\yskala})
	--(3.8571,{-0.0448*\yskala})
	--(3.8810,{-0.0389*\yskala})
	--(3.9048,{-0.0325*\yskala})
	--(3.9286,{-0.0254*\yskala})
	--(3.9524,{-0.0176*\yskala})
	--(3.9762,{-0.0092*\yskala})
	--(4.0000,{0.0000*\yskala})
	--(4.0238,{0.0099*\yskala})
	--(4.0476,{0.0205*\yskala})
	--(4.0714,{0.0319*\yskala})
	--(4.0952,{0.0440*\yskala})
	--(4.1190,{0.0568*\yskala})
	--(4.1429,{0.0705*\yskala})
	--(4.1667,{0.0849*\yskala})
	--(4.1905,{0.1001*\yskala})
	--(4.2143,{0.1161*\yskala})
	--(4.2381,{0.1329*\yskala})
	--(4.2619,{0.1505*\yskala})
	--(4.2857,{0.1689*\yskala})
	--(4.3095,{0.1881*\yskala})
	--(4.3333,{0.2080*\yskala})
	--(4.3571,{0.2288*\yskala})
	--(4.3810,{0.2503*\yskala})
	--(4.4048,{0.2726*\yskala})
	--(4.4286,{0.2956*\yskala})
	--(4.4524,{0.3194*\yskala})
	--(4.4762,{0.3439*\yskala})
	--(4.5000,{0.3691*\yskala})
	--(4.5238,{0.3950*\yskala})
	--(4.5476,{0.4216*\yskala})
	--(4.5714,{0.4488*\yskala})
	--(4.5952,{0.4766*\yskala})
	--(4.6190,{0.5050*\yskala})
	--(4.6429,{0.5339*\yskala})
	--(4.6667,{0.5633*\yskala})
	--(4.6905,{0.5932*\yskala})
	--(4.7143,{0.6235*\yskala})
	--(4.7381,{0.6542*\yskala})
	--(4.7619,{0.6853*\yskala})
	--(4.7857,{0.7166*\yskala})
	--(4.8095,{0.7481*\yskala})
	--(4.8333,{0.7797*\yskala})
	--(4.8571,{0.8115*\yskala})
	--(4.8810,{0.8433*\yskala})
	--(4.9048,{0.8750*\yskala})
	--(4.9286,{0.9067*\yskala})
	--(4.9524,{0.9381*\yskala})
	--(4.9762,{0.9692*\yskala})
	--(5.0000,{1.0000*\yskala})
	--(5.0238,{1.0303*\yskala})
	--(5.0476,{1.0601*\yskala})
	--(5.0714,{1.0891*\yskala})
	--(5.0952,{1.1175*\yskala})
	--(5.1190,{1.1449*\yskala})
	--(5.1429,{1.1713*\yskala})
	--(5.1667,{1.1967*\yskala})
	--(5.1905,{1.2208*\yskala})
	--(5.2143,{1.2435*\yskala})
	--(5.2381,{1.2647*\yskala})
	--(5.2619,{1.2843*\yskala})
	--(5.2857,{1.3020*\yskala})
	--(5.3095,{1.3178*\yskala})
	--(5.3333,{1.3315*\yskala})
	--(5.3571,{1.3429*\yskala})
	--(5.3810,{1.3519*\yskala})
	--(5.4048,{1.3582*\yskala})
	--(5.4286,{1.3617*\yskala})
	--(5.4524,{1.3623*\yskala})
	--(5.4762,{1.3596*\yskala})
	--(5.5000,{1.3535*\yskala})
	--(5.5238,{1.3438*\yskala})
	--(5.5476,{1.3303*\yskala})
	--(5.5714,{1.3127*\yskala})
	--(5.5952,{1.2908*\yskala})
	--(5.6190,{1.2645*\yskala})
	--(5.6429,{1.2333*\yskala})
	--(5.6667,{1.1971*\yskala})
	--(5.6905,{1.1556*\yskala})
	--(5.7143,{1.1085*\yskala})
	--(5.7381,{1.0556*\yskala})
	--(5.7619,{0.9966*\yskala})
	--(5.7857,{0.9311*\yskala})
	--(5.8095,{0.8590*\yskala})
	--(5.8333,{0.7797*\yskala})
	--(5.8571,{0.6932*\yskala})
	--(5.8810,{0.5989*\yskala})
	--(5.9048,{0.4966*\yskala})
	--(5.9286,{0.3859*\yskala})
	--(5.9524,{0.2665*\yskala})
	--(5.9762,{0.1380*\yskala})
	--(6.0000,{-0.0000*\yskala})
	--(6.0238,{-0.1479*\yskala})
	--(6.0476,{-0.3060*\yskala})
	--(6.0714,{-0.4748*\yskala})
	--(6.0952,{-0.6546*\yskala})
	--(6.1190,{-0.8460*\yskala})
	--(6.1429,{-1.0493*\yskala})
	--(6.1667,{-1.2651*\yskala})
	--(6.1905,{-1.4936*\yskala})
	--(6.2143,{-1.7356*\yskala})
	--(6.2381,{-1.9913*\yskala})
	--(6.2619,{-2.2613*\yskala})
	--(6.2857,{-2.5461*\yskala})
	--(6.3095,{-2.8463*\yskala})
	--(6.3333,{-3.1623*\yskala})
	--(6.3571,{-3.4947*\yskala})
	--(6.3810,{-3.8441*\yskala})
	--(6.4048,{-4.2109*\yskala})
	--(6.4286,{-4.5959*\yskala})
	--(6.4524,{-4.9995*\yskala})
	--(6.4762,{-5.4224*\yskala})
	--(6.5000,{-5.8652*\yskala});
}
\def\basissix{
\draw[line width=1.4pt,color=red] (-0.5000,{0.2256*\yskala})
	--(-0.4762,{0.2060*\yskala})
	--(-0.4524,{0.1876*\yskala})
	--(-0.4286,{0.1702*\yskala})
	--(-0.4048,{0.1539*\yskala})
	--(-0.3810,{0.1387*\yskala})
	--(-0.3571,{0.1243*\yskala})
	--(-0.3333,{0.1110*\yskala})
	--(-0.3095,{0.0985*\yskala})
	--(-0.2857,{0.0868*\yskala})
	--(-0.2619,{0.0760*\yskala})
	--(-0.2381,{0.0659*\yskala})
	--(-0.2143,{0.0565*\yskala})
	--(-0.1905,{0.0479*\yskala})
	--(-0.1667,{0.0399*\yskala})
	--(-0.1429,{0.0325*\yskala})
	--(-0.1190,{0.0258*\yskala})
	--(-0.0952,{0.0196*\yskala})
	--(-0.0714,{0.0140*\yskala})
	--(-0.0476,{0.0088*\yskala})
	--(-0.0238,{0.0042*\yskala})
	--(0.0000,{-0.0000*\yskala})
	--(0.0238,{-0.0038*\yskala})
	--(0.0476,{-0.0071*\yskala})
	--(0.0714,{-0.0101*\yskala})
	--(0.0952,{-0.0127*\yskala})
	--(0.1190,{-0.0150*\yskala})
	--(0.1429,{-0.0169*\yskala})
	--(0.1667,{-0.0186*\yskala})
	--(0.1905,{-0.0199*\yskala})
	--(0.2143,{-0.0211*\yskala})
	--(0.2381,{-0.0220*\yskala})
	--(0.2619,{-0.0226*\yskala})
	--(0.2857,{-0.0231*\yskala})
	--(0.3095,{-0.0234*\yskala})
	--(0.3333,{-0.0235*\yskala})
	--(0.3571,{-0.0234*\yskala})
	--(0.3810,{-0.0232*\yskala})
	--(0.4048,{-0.0229*\yskala})
	--(0.4286,{-0.0224*\yskala})
	--(0.4524,{-0.0219*\yskala})
	--(0.4762,{-0.0212*\yskala})
	--(0.5000,{-0.0205*\yskala})
	--(0.5238,{-0.0197*\yskala})
	--(0.5476,{-0.0188*\yskala})
	--(0.5714,{-0.0179*\yskala})
	--(0.5952,{-0.0170*\yskala})
	--(0.6190,{-0.0160*\yskala})
	--(0.6429,{-0.0149*\yskala})
	--(0.6667,{-0.0139*\yskala})
	--(0.6905,{-0.0128*\yskala})
	--(0.7143,{-0.0117*\yskala})
	--(0.7381,{-0.0107*\yskala})
	--(0.7619,{-0.0096*\yskala})
	--(0.7857,{-0.0085*\yskala})
	--(0.8095,{-0.0075*\yskala})
	--(0.8333,{-0.0064*\yskala})
	--(0.8571,{-0.0054*\yskala})
	--(0.8810,{-0.0044*\yskala})
	--(0.9048,{-0.0035*\yskala})
	--(0.9286,{-0.0026*\yskala})
	--(0.9524,{-0.0017*\yskala})
	--(0.9762,{-0.0008*\yskala})
	--(1.0000,{0.0000*\yskala})
	--(1.0238,{0.0008*\yskala})
	--(1.0476,{0.0015*\yskala})
	--(1.0714,{0.0022*\yskala})
	--(1.0952,{0.0028*\yskala})
	--(1.1190,{0.0034*\yskala})
	--(1.1429,{0.0040*\yskala})
	--(1.1667,{0.0045*\yskala})
	--(1.1905,{0.0049*\yskala})
	--(1.2143,{0.0053*\yskala})
	--(1.2381,{0.0057*\yskala})
	--(1.2619,{0.0060*\yskala})
	--(1.2857,{0.0063*\yskala})
	--(1.3095,{0.0065*\yskala})
	--(1.3333,{0.0067*\yskala})
	--(1.3571,{0.0068*\yskala})
	--(1.3810,{0.0069*\yskala})
	--(1.4048,{0.0070*\yskala})
	--(1.4286,{0.0070*\yskala})
	--(1.4524,{0.0070*\yskala})
	--(1.4762,{0.0069*\yskala})
	--(1.5000,{0.0068*\yskala})
	--(1.5238,{0.0067*\yskala})
	--(1.5476,{0.0065*\yskala})
	--(1.5714,{0.0064*\yskala})
	--(1.5952,{0.0061*\yskala})
	--(1.6190,{0.0059*\yskala})
	--(1.6429,{0.0056*\yskala})
	--(1.6667,{0.0053*\yskala})
	--(1.6905,{0.0050*\yskala})
	--(1.7143,{0.0047*\yskala})
	--(1.7381,{0.0043*\yskala})
	--(1.7619,{0.0040*\yskala})
	--(1.7857,{0.0036*\yskala})
	--(1.8095,{0.0032*\yskala})
	--(1.8333,{0.0028*\yskala})
	--(1.8571,{0.0024*\yskala})
	--(1.8810,{0.0020*\yskala})
	--(1.9048,{0.0016*\yskala})
	--(1.9286,{0.0012*\yskala})
	--(1.9524,{0.0008*\yskala})
	--(1.9762,{0.0004*\yskala})
	--(2.0000,{-0.0000*\yskala})
	--(2.0238,{-0.0004*\yskala})
	--(2.0476,{-0.0008*\yskala})
	--(2.0714,{-0.0012*\yskala})
	--(2.0952,{-0.0015*\yskala})
	--(2.1190,{-0.0019*\yskala})
	--(2.1429,{-0.0022*\yskala})
	--(2.1667,{-0.0025*\yskala})
	--(2.1905,{-0.0028*\yskala})
	--(2.2143,{-0.0031*\yskala})
	--(2.2381,{-0.0034*\yskala})
	--(2.2619,{-0.0036*\yskala})
	--(2.2857,{-0.0039*\yskala})
	--(2.3095,{-0.0041*\yskala})
	--(2.3333,{-0.0043*\yskala})
	--(2.3571,{-0.0044*\yskala})
	--(2.3810,{-0.0046*\yskala})
	--(2.4048,{-0.0047*\yskala})
	--(2.4286,{-0.0048*\yskala})
	--(2.4524,{-0.0048*\yskala})
	--(2.4762,{-0.0049*\yskala})
	--(2.5000,{-0.0049*\yskala})
	--(2.5238,{-0.0049*\yskala})
	--(2.5476,{-0.0048*\yskala})
	--(2.5714,{-0.0048*\yskala})
	--(2.5952,{-0.0047*\yskala})
	--(2.6190,{-0.0046*\yskala})
	--(2.6429,{-0.0044*\yskala})
	--(2.6667,{-0.0043*\yskala})
	--(2.6905,{-0.0041*\yskala})
	--(2.7143,{-0.0039*\yskala})
	--(2.7381,{-0.0036*\yskala})
	--(2.7619,{-0.0034*\yskala})
	--(2.7857,{-0.0031*\yskala})
	--(2.8095,{-0.0028*\yskala})
	--(2.8333,{-0.0025*\yskala})
	--(2.8571,{-0.0022*\yskala})
	--(2.8810,{-0.0019*\yskala})
	--(2.9048,{-0.0015*\yskala})
	--(2.9286,{-0.0012*\yskala})
	--(2.9524,{-0.0008*\yskala})
	--(2.9762,{-0.0004*\yskala})
	--(3.0000,{0.0000*\yskala})
	--(3.0238,{0.0004*\yskala})
	--(3.0476,{0.0008*\yskala})
	--(3.0714,{0.0012*\yskala})
	--(3.0952,{0.0016*\yskala})
	--(3.1190,{0.0020*\yskala})
	--(3.1429,{0.0024*\yskala})
	--(3.1667,{0.0028*\yskala})
	--(3.1905,{0.0032*\yskala})
	--(3.2143,{0.0036*\yskala})
	--(3.2381,{0.0040*\yskala})
	--(3.2619,{0.0043*\yskala})
	--(3.2857,{0.0047*\yskala})
	--(3.3095,{0.0050*\yskala})
	--(3.3333,{0.0053*\yskala})
	--(3.3571,{0.0056*\yskala})
	--(3.3810,{0.0059*\yskala})
	--(3.4048,{0.0061*\yskala})
	--(3.4286,{0.0064*\yskala})
	--(3.4524,{0.0065*\yskala})
	--(3.4762,{0.0067*\yskala})
	--(3.5000,{0.0068*\yskala})
	--(3.5238,{0.0069*\yskala})
	--(3.5476,{0.0070*\yskala})
	--(3.5714,{0.0070*\yskala})
	--(3.5952,{0.0070*\yskala})
	--(3.6190,{0.0069*\yskala})
	--(3.6429,{0.0068*\yskala})
	--(3.6667,{0.0067*\yskala})
	--(3.6905,{0.0065*\yskala})
	--(3.7143,{0.0063*\yskala})
	--(3.7381,{0.0060*\yskala})
	--(3.7619,{0.0057*\yskala})
	--(3.7857,{0.0053*\yskala})
	--(3.8095,{0.0049*\yskala})
	--(3.8333,{0.0045*\yskala})
	--(3.8571,{0.0040*\yskala})
	--(3.8810,{0.0034*\yskala})
	--(3.9048,{0.0028*\yskala})
	--(3.9286,{0.0022*\yskala})
	--(3.9524,{0.0015*\yskala})
	--(3.9762,{0.0008*\yskala})
	--(4.0000,{-0.0000*\yskala})
	--(4.0238,{-0.0008*\yskala})
	--(4.0476,{-0.0017*\yskala})
	--(4.0714,{-0.0026*\yskala})
	--(4.0952,{-0.0035*\yskala})
	--(4.1190,{-0.0044*\yskala})
	--(4.1429,{-0.0054*\yskala})
	--(4.1667,{-0.0064*\yskala})
	--(4.1905,{-0.0075*\yskala})
	--(4.2143,{-0.0085*\yskala})
	--(4.2381,{-0.0096*\yskala})
	--(4.2619,{-0.0107*\yskala})
	--(4.2857,{-0.0117*\yskala})
	--(4.3095,{-0.0128*\yskala})
	--(4.3333,{-0.0139*\yskala})
	--(4.3571,{-0.0149*\yskala})
	--(4.3810,{-0.0160*\yskala})
	--(4.4048,{-0.0170*\yskala})
	--(4.4286,{-0.0179*\yskala})
	--(4.4524,{-0.0188*\yskala})
	--(4.4762,{-0.0197*\yskala})
	--(4.5000,{-0.0205*\yskala})
	--(4.5238,{-0.0212*\yskala})
	--(4.5476,{-0.0219*\yskala})
	--(4.5714,{-0.0224*\yskala})
	--(4.5952,{-0.0229*\yskala})
	--(4.6190,{-0.0232*\yskala})
	--(4.6429,{-0.0234*\yskala})
	--(4.6667,{-0.0235*\yskala})
	--(4.6905,{-0.0234*\yskala})
	--(4.7143,{-0.0231*\yskala})
	--(4.7381,{-0.0226*\yskala})
	--(4.7619,{-0.0220*\yskala})
	--(4.7857,{-0.0211*\yskala})
	--(4.8095,{-0.0199*\yskala})
	--(4.8333,{-0.0186*\yskala})
	--(4.8571,{-0.0169*\yskala})
	--(4.8810,{-0.0150*\yskala})
	--(4.9048,{-0.0127*\yskala})
	--(4.9286,{-0.0101*\yskala})
	--(4.9524,{-0.0071*\yskala})
	--(4.9762,{-0.0038*\yskala})
	--(5.0000,{0.0000*\yskala})
	--(5.0238,{0.0042*\yskala})
	--(5.0476,{0.0088*\yskala})
	--(5.0714,{0.0140*\yskala})
	--(5.0952,{0.0196*\yskala})
	--(5.1190,{0.0258*\yskala})
	--(5.1429,{0.0325*\yskala})
	--(5.1667,{0.0399*\yskala})
	--(5.1905,{0.0479*\yskala})
	--(5.2143,{0.0565*\yskala})
	--(5.2381,{0.0659*\yskala})
	--(5.2619,{0.0760*\yskala})
	--(5.2857,{0.0868*\yskala})
	--(5.3095,{0.0985*\yskala})
	--(5.3333,{0.1110*\yskala})
	--(5.3571,{0.1243*\yskala})
	--(5.3810,{0.1387*\yskala})
	--(5.4048,{0.1539*\yskala})
	--(5.4286,{0.1702*\yskala})
	--(5.4524,{0.1876*\yskala})
	--(5.4762,{0.2060*\yskala})
	--(5.5000,{0.2256*\yskala})
	--(5.5238,{0.2464*\yskala})
	--(5.5476,{0.2684*\yskala})
	--(5.5714,{0.2917*\yskala})
	--(5.5952,{0.3164*\yskala})
	--(5.6190,{0.3425*\yskala})
	--(5.6429,{0.3700*\yskala})
	--(5.6667,{0.3990*\yskala})
	--(5.6905,{0.4296*\yskala})
	--(5.7143,{0.4619*\yskala})
	--(5.7381,{0.4958*\yskala})
	--(5.7619,{0.5315*\yskala})
	--(5.7857,{0.5690*\yskala})
	--(5.8095,{0.6084*\yskala})
	--(5.8333,{0.6498*\yskala})
	--(5.8571,{0.6932*\yskala})
	--(5.8810,{0.7386*\yskala})
	--(5.9048,{0.7863*\yskala})
	--(5.9286,{0.8361*\yskala})
	--(5.9524,{0.8883*\yskala})
	--(5.9762,{0.9429*\yskala})
	--(6.0000,{1.0000*\yskala})
	--(6.0238,{1.0596*\yskala})
	--(6.0476,{1.1219*\yskala})
	--(6.0714,{1.1869*\yskala})
	--(6.0952,{1.2547*\yskala})
	--(6.1190,{1.3254*\yskala})
	--(6.1429,{1.3991*\yskala})
	--(6.1667,{1.4759*\yskala})
	--(6.1905,{1.5559*\yskala})
	--(6.2143,{1.6391*\yskala})
	--(6.2381,{1.7258*\yskala})
	--(6.2619,{1.8159*\yskala})
	--(6.2857,{1.9096*\yskala})
	--(6.3095,{2.0070*\yskala})
	--(6.3333,{2.1082*\yskala})
	--(6.3571,{2.2133*\yskala})
	--(6.3810,{2.3225*\yskala})
	--(6.4048,{2.4357*\yskala})
	--(6.4286,{2.5533*\yskala})
	--(6.4524,{2.6752*\yskala})
	--(6.4762,{2.8016*\yskala})
	--(6.5000,{2.9326*\yskala});
}

\begin{tikzpicture}[>=latex,thick,scale=\skala]

\def\bild#1#2#3{
\begin{scope}[yshift=#1]
\begin{scope}
\clip (-0.4,-0.4) rectangle (6.5,0.75);
\expandafter\expandafter\csname#2\endcsname
\end{scope}
\fill[color=red] (#3,\yskala) circle[radius={0.08/\skala}];
\draw[->] (-0.4,0)--(6.5,0) coordinate[label={$x$}];
\draw[->] (0,-0.35)--(0,0.85) coordinate[label={right:$y$}];
\draw ({-0.1/\skala},\yskala)--({0.1/\skala},\yskala);
\node at ({-0.1/\skala},\yskala) [left] {$1$};
\node at (3,\yskala) [above] {$j=#3$};
\foreach \x in {1,...,6}{
	\draw (\x,{-0.1/\skala})--(\x,{0.1/\skala});
	\node at (\x,{-0.1/\skala}) [below] {$\x$};
}
\node at ({-0.1/\skala},{-0.1/\skala}) [below left] {$0$};
\end{scope}
}

\bild{0cm}{basiszero}{0}
\bild{-1.4cm}{basisone}{1}
\bild{-2.8cm}{basistwo}{2}
\bild{-4.2cm}{basisthree}{3}
\bild{-5.6cm}{basisfour}{4}
\bild{-7.0cm}{basisfive}{5}
\bild{-8.4cm}{basissix}{6}

\end{tikzpicture}
\end{document}

