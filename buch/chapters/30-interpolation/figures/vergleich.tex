%
% vergleich.tex -- Vergleich von Tschebyscheff-Stützstelle und äquidistanten
%
% (c) 2020 Prof Dr Andreas Müller, Hochschule Rapperswil
%
\documentclass[tikz]{standalone}
\usepackage{amsmath}
\usepackage{times}
\usepackage{txfonts}
\usepackage{pgfplots}
\usepackage{csvsimple}
\usetikzlibrary{arrows,intersections,math}
\begin{document}
\def\skala{1}
\begin{tikzpicture}[>=latex,thick,scale=\skala]

\begin{scope}[yshift=9.0cm]
\begin{scope}
\clip (-6,-6) rectangle (6,6);

%\draw[color=red,line width=1.4pt]
%	plot[domain=-1:1,samples=600]
%		({6*\x},{2*(\x+1)*(\x-1)*2});
%
%\draw[color=red,line width=1.4pt]
%	plot[domain=-1:1,samples=600]
%		({6*\x},{2*(\x+1)*\x*(\x-1)*4});

%\draw[color=red,line width=1.4pt]
%	plot[domain=-1:1,samples=600]
%		({6*\x},{2*(\x+1)*(\x+0.33333)*(\x-0.33333)*(\x-1)*8});

%\draw[color=red,line width=1.4pt]
%	plot[domain=-1:1,samples=600]
%		({6*\x},{2*(\x+1)*(\x+0.5)*\x*(\x-0.5)*(\x-1)*16});

\draw[color=red,line width=1.4pt]
	plot[domain=-1:1,samples=600]
		({6*\x},{2*(\x+1)*(\x+0.6)*(\x+0.2)*(\x-0.2)*(\x-0.6)*(\x-1)*32});

\draw[color=red,line width=1.4pt]
	plot[domain=-1:1,samples=600]
		({6*\x},{2*(\x+1)*(\x+0.66666)*(\x+0.3333)*\x*((\x-0.33333)*(\x-0.66666)*(\x-1)*64});

\draw[color=red,line width=1.4pt]
	plot[domain=-1:1,samples=600]
		({6*\x},{2*(\x+1)*(\x+0.72429)*(\x+0.42857)*(\x+0.14286)*(\x-0.14286)*(\x-0.42857)*(\x-0.72429)*(\x-1)*128});

\draw[color=red,line width=1.4pt]
	plot[domain=-1:1,samples=600]
		({6*\x},{2*(\x+1)*(\x+0.75)*(\x+0.5)*(\x+0.25)*\x*(\x-0.25)*(\x-0.5)*(\x-0.75)*(\x-1)*256});

\draw[color=red,line width=1.4pt]
	plot[domain=-1:1,samples=600]
		({6*\x},{2*(\x+1)*(\x+0.77777)*(\x+0.55555)*(\x+0.33333)*(\x+0.1111)*(\x-0.11111)*(\x-0.33333)*(\x-0.55555)*(\x-0.77777)*(\x-1)*512});

\end{scope}

\draw[->] (-6.1,0)--(6.3,0) coordinate[label={$x$}];
\draw[->] (0,-6.1)--(0,6.3) coordinate[label={right:$y$}];
\draw (-6,-0.1)--(-6,0.1);
\node at (-6,-0.1) [below left] {$-1$};
\draw (6,-0.1)--(6,0.1);
\node at (6,-0.1) [below right] {$1$};

\draw (-0.1,-2)--(0.1,-2);
\draw (-0.1,-4)--(0.1,-4);
\draw (-0.1,-6)--(0.1,-6);
\node at (-0.1,-2) [left] {$-1$};
\node at (-0.1,-4) [left] {$-2$};
\node at (-0.1,-6) [left] {$-3$};
\draw (-0.1,2)--(0.1,2);
\draw (-0.1,4)--(0.1,4);
\draw (-0.1,6)--(0.1,6);
\node at (-0.1,2) [left] {$1$};
\node at (-0.1,4) [left] {$2$};
\node at (-0.1,6) [left] {$3$};

\end{scope}

\begin{scope}

\draw[color=red,line width=1.4pt]
	plot[domain=-1:1,samples=600]
		({6*\x},{2*(2*\x*\x-1)});

\draw[color=red,line width=1.4pt]
	plot[domain=-1:1,samples=600]
		({6*\x},{2*((4*\x*\x-3)*\x)});

\draw[color=red,line width=1.4pt]
	plot[domain=-1:1,samples=600]
		({6*\x},{2*(((8*\x*\x-8)*\x*\x)+1)});


\draw[color=red,line width=1.4pt]
	plot[domain=-1:1,samples=600]
		({6*\x},{2*(((16*\x*\x-20)*\x*\x+5)*\x)});

\draw[color=red,line width=1.4pt]
	plot[domain=-1:1,samples=600]
		({6*\x},{2*((((32*\x*\x-48)*\x*\x+18)*\x*\x)-1)});

\draw[color=red,line width=1.4pt]
	plot[domain=-1:1,samples=600]
		({6*\x},{2*((((64*\x*\x-112)*\x*\x+56)*\x*\x-7)*\x)});

\draw[color=red,line width=1.4pt]
	plot[domain=-1:1,samples=600]
		({6*\x},{2*((((128*\x*\x-256)*\x*\x+160)*\x*\x-32)*\x*\x+1)});

\draw[->] (-6.1,0)--(6.3,0) coordinate[label={$x$}];
\draw[->] (0,-2.1)--(0,2.3) coordinate[label={right:$y$}];
\draw (-6,-0.1)--(-6,0.1);
\node at (-6,-0.1) [below left] {$-1$};
\draw (6,-0.1)--(6,0.1);
\node at (6,-0.1) [below right] {$1$};
\draw (-0.1,-2)--(0.1,-2);
\node at (-0.1,-2) [left] {$-1$};
\draw (-0.1,2)--(0.1,2);
\node at (-0.1,2) [left] {$1$};

\end{scope}

\end{tikzpicture}
\end{document}

