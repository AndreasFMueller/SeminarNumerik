%
% h.tex -- Hermite-Polynome für zwei Stützstellen
%
% (c) 2020 Prof Dr Andreas Müller, Hochschule Rapperswil
%
\documentclass[tikz]{standalone}
\usepackage{amsmath}
\usepackage{times}
\usepackage{txfonts}
\usepackage{pgfplots}
\usepackage{csvsimple}
\usetikzlibrary{arrows,intersections,math}
\begin{document}
\def\skala{5}
\begin{tikzpicture}[>=latex,thick,scale=\skala]

\begin{scope}[yshift=0.5cm]
\draw[line width=1.5pt,color=red]
	plot[domain=0:1,samples=100] ({\x},{(1+2*\x)*(1-\x)*(1-\x)});
\draw[->] ({-0.1/\skala},0)--(1.1,0) coordinate[label={$x$}];
\draw[->] (0,{-0.1/\skala})--(0,1.1) coordinate[label={right:$y$}];
\draw (1,{0.1/\skala})--(1,{-0.1/\skala});
\node at (1,{-0.1/\skala}) [below] {$1$};
\node at (0,{-0.1/\skala}) [below] {$0$};
\node at (0.5,0.9) {$H_0(x)$};
\draw ({-0.1/\skala},1)--({0.1/\skala},1);
\node at ({-0.1/\skala},1) [left] {$1$};
\end{scope}

\begin{scope}[yshift=0.5cm,xshift=1.2cm]
\draw[line width=1.5pt,color=red]
	plot[domain=0:1,samples=100] ({\x},{(3-2*\x)*\x*\x});
\draw[->] ({-0.1/\skala},0)--(1.1,0) coordinate[label={$x$}];
\draw[->] (0,{-0.1/\skala})--(0,1.1) coordinate[label={right:$y$}];
\draw (1,{0.1/\skala})--(1,{-0.1/\skala});
\node at (1,{-0.1/\skala}) [below] {$1$};
\node at (0,{-0.1/\skala}) [below] {$0$};
\node at (0.5,0.9) {$H_1(x)$};
\draw ({-0.1/\skala},1)--({0.1/\skala},1);
\node at ({-0.1/\skala},1) [left] {$1$};
\end{scope}

\begin{scope}
\draw[line width=1.5pt,color=red]
	plot[domain=0:1,samples=100] ({\x},{\x*(1-\x)*(1-\x)});
\draw[->] ({-0.1/\skala},0)--(1.1,0) coordinate[label={$x$}];
\draw[->] (0,{-0.2})--(0,0.35) coordinate[label={right:$y$}];
\draw (1,{0.1/\skala})--(1,{-0.1/\skala});
\node at (1,{-0.1/\skala}) [below] {$1$};
%\node at (0,{-0.1/\skala}) [below] {$0$};
\node at (0.5,0.25) {$H_0^1(x)$};
\end{scope}

\begin{scope}[xshift=1.2cm]
\draw[line width=1.5pt,color=red]
	plot[domain=0:1,samples=100] ({\x},{-\x*\x*(1-\x)});
\draw[->] ({-0.1/\skala},0)--(1.1,0) coordinate[label={$x$}];
\draw[->] (0,{-0.2})--(0,0.35) coordinate[label={right:$y$}];
\draw (1,{0.1/\skala})--(1,{-0.1/\skala});
\node at (1,{-0.1/\skala}) [below] {$1$};
%\node at (0,{-0.1/\skala}) [below] {$0$};
\node at (0.5,0.25) {$H_1^1(x)$};
\end{scope}

\end{tikzpicture}
\end{document}

