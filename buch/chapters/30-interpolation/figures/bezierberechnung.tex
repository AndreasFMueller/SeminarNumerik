%
% bezierberechnung.tex -- Berechnung der Punkte einer Bézier-Kurve
%
% (c) 2020 Prof Dr Andreas Müller, Hochschule Rapperswil
%
\documentclass[tikz]{standalone}
\usepackage{amsmath}
\usepackage{times}
\usepackage{txfonts}
\usepackage{pgfplots}
\usepackage{csvsimple}
\usetikzlibrary{arrows,intersections,math}
\begin{document}
\def\skala{1}
\begin{tikzpicture}[>=latex,thick,scale=\skala]

\definecolor{darkgreen}{rgb}{0,0.6,0}
\def\sx{2}
\pgfmathparse{\sx*sqrt(3)/2}
\xdef\sy{\pgfmathresult}

\def\pfeil#1#2{
	\draw[->,shorten >= 0.2cm,shorten <= 0.1cm]
		({#1},{#2}) -- ({#1+0.5*\sx},{#2-\sy+0.4*\sy});
	\node at ({#1+0.125*\sx},{#2-0.25*\sy}) [above right] {$1-t$};
}
\def\qfeil#1#2{
	\draw[->,shorten >= 0.2cm,shorten <= 0.1cm]
		({#1},{#2}) -- ({#1-0.5*\sx},{#2-\sy+0.4*\sy});
	\node at ({#1-0.125*\sx},{#2-0.25*\sy}) [above left] {$t$};
}

\def\summe#1#2{
	\pfeil{(#1+0.5*#2-3)*\sx}{-#2*\sy}
	\qfeil{(1+#1+0.5*#2-3)*\sx}{-#2*\sy}
	\draw ({#1+0.5*#2-3+0.5)*\sx},{(-0.6-#2)*\sy}) circle[radius=0.2];
	\draw
		({#1+0.5*#2-3+0.5)*\sx-0.14},{(-0.6-#2)*\sy})
		--
		({#1+0.5*#2-3+0.5)*\sx+0.14},{(-0.6-#2)*\sy});
	\draw
		({#1+0.5*#2-3+0.5)*\sx},{(-0.6-#2)*\sy-0.14})
		--
		({#1+0.5*#2-3+0.5)*\sx},{(-0.6-#2)*\sy+0.14});
	\draw[->,shorten <= 0.2cm,shorten >= 0.1cm]
		({#1+0.5*#2-3+0.5)*\sx},{(-0.6-#2)*\sy})
		--
		({#1+0.5*#2-3+0.5)*\sx},{(-1-#2)*\sy});
}

\foreach \x in {0,...,6}{
	\node at ({\sx*(\x-3)},0) [above] {$P_{\x}$};
	\fill[color=red] ({2*(\x-3)},0) circle[radius=0.08];
}

\foreach \x in {0,...,5}{
	\summe{\x}{0}
}

\foreach \x in {0,...,4}{
	\summe{\x}{1}
}

\foreach \x in {0,...,3}{
	\summe{\x}{2}
}

\foreach \x in {0,...,2}{
	\summe{\x}{3}
}

\foreach \x in {0,...,1}{
	\summe{\x}{4}
}
\summe{0}{5}

\foreach \x in {0,...,5}{
	\node at ({1+2*(\x-3)},{-1*\sy}) [above left] {$p_{\x,1}(t)$};
	\fill[color=darkgreen] ({1+2*(\x-3)},{-1*\sy}) circle[radius=0.08];
}

\foreach \x in {0,...,4}{
	\node at ({2+2*(\x-3)},{-2*\sy}) [above left] {$p_{\x,2}(t)$};
	\fill[color=blue] ({2+2*(\x-3)},{-2*\sy}) circle[radius=0.08];
}

\foreach \x in {0,...,3}{
	\fill ({3+2*(\x-3)},{-3*\sy}) circle[radius=0.08];
}

\foreach \x in {0,...,2}{
	\fill ({4+2*(\x-3)},{-4*\sy}) circle[radius=0.08];
}

\foreach \x in {0,1}{
	\fill ({5+2*(\x-3)},{-5*\sy}) circle[radius=0.08];
}

\fill[color=orange] ({0},{-6*\sy}) circle[radius=0.08];
\node at (0,{-6*\sy}) [below] {$p_{0,6}(t)$};

\end{tikzpicture}
\end{document}

