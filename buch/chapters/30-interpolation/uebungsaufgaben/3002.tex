Wir möchten das Lagrange- und das Hermite-Interpolationspolynom
für die Funktion $f(x)=\sin \frac{\pi}2x$ im Intervall $[0,1]$ 
vergleichen.
\index{Hermite-Interpolationspolynom}%
\index{Lagrange-Interpolationspolynom}%
Die Abbildung zeigt, dass die Polynome sehr nahe beieinander liegen.
\begin{center}
\begin{tikzpicture}[>=latex,thick,scale=4]
\draw[color=red,line width=1.4pt]
	plot[domain=0:1,samples=100] ({\x},{sin(90*\x)});
\draw[->] (-0.025,0) -- (1.2,0) coordinate[label={$x$}];
\draw[->] (0,-0.025) -- (0,1.2) coordinate[label={right:$y$}];
\draw[line width=0.7pt] (1,-0.025)--(1,0.025);
\draw[line width=0.7pt] (-0.025,1)--(0.025,1);
\draw[color=blue,line width=1pt]
	plot[domain=0:1,samples=100]
		({\x},{-0.42920357*\x*\x*\x+(-0.14159)*\x*\x+1.57079*\x});
%(%i15) q:combine(expand(q))
%            7/2  3        3    7/2  2    5/2  2       2    5/2
%        (- 3    x ) + 45 x  + 3    x  + 3    x  - 63 x  - 3    x + 22 x
%(%o15)  ---------------------------------------------------------------
%                                       4

\draw[color=darkgreen,line width=1pt]
	plot[domain=0:1,samples=100]
		({\x},{(((45-27*sqrt(3))*\x+(36*sqrt(3)-63))*\x+(22-9*sqrt(3)))*\x/4});



\begin{scope}[xshift=1.4cm]
\draw[->] (-0.025,0) -- (1.2,0) coordinate[label={$x$}];
\draw[->] (0,-0.25) -- (0,1.2) coordinate[label={right:$\text{Fehler}\times 100$}];
\draw[color=blue,line width=0.7pt]
	plot[domain=0:1,samples=300]
		({\x},{100*(sin(90*\x)+0.42920357*\x*\x*\x-(-0.14159)*\x*\x-1.57079*\x)});
\draw[color=darkgreen,line width=1pt]
	plot[domain=0:1,samples=300]
		({\x},{100*(sin(90*\x)-(((45-27*sqrt(3))*\x+(36*sqrt(3)-63))*\x+(22-9*sqrt(3)))*\x/4)});
\draw[line width=0.7pt] (1,-0.025)--(1,0.025);
\draw[line width=0.7pt] (-0.025,1)--(0.025,1);
\node[color=blue] at (0.7,0.9) [right] {$f(x)-h(x)$};
\node[color=darkgreen] at (0.5,-0.05) [below] {$f(x)-p(x)$};
\end{scope}
\end{tikzpicture}
\end{center}

\begin{teilaufgaben}
\item
Finden Sie das Hermite-Interpolationspolynom $h(x)$ von $f(x)$
für die Stützstellen $0$ und $1$.
\item
Finden Sie das Lagrange-Interpolationspolynom $p(x)$ von $f(x)$
für die Stützstellen $0$, $\frac13$, $\frac23$ und $1$.
\item 
Berechnen Sie die Funktionswerte $f(\frac12)$, $p(\frac12)$ und $h(\frac12)$.
\end{teilaufgaben}

\begin{loesung}
\begin{teilaufgaben}
\item
Das Hermite-Interpolationspolynom kann aus den Funktions- und
Ableitungswerten an den Intervallenden berechnet werden:
\begin{equation}
\begin{aligned}
 f(0) &=  0           &&\qquad &  f(1) &= 1 \\
f'(0) &= \frac{\pi}2  &&\qquad & f'(1) &= 0
\end{aligned}
\end{equation}
Daraus kann man das Interpolationspolynom ablesen:
\begin{align*}
h(x)
=
\frac{\pi}{2}H_0^1(x) + H_1(x)
&=
\frac{\pi}2 (x^3-2x^2+x) -2x^3+3x^2.
\\
&=
\biggl(\frac{\pi}2-2\biggr) x^3
+(3-\pi)x^2
+
\frac{\pi}2x
\end{align*}
\item
Für das Lagrange-Interplationspolynom geht man aus vom 
Polynom
\[
l(x) = x(x-{\textstyle\frac13})(x-{\textstyle\frac23})(x-1)
\]
und konstruiert daraus die Polynome
%(%i12) expand(subst(1/3,h,l0))
%                                3
%                             9 x        2   11 x
%(%o12)                    (- ----) + 9 x  - ---- + 1
%                              2              2
%(%i13) expand(subst(1/3,h,l1))
%                                  3       2
%                              27 x    45 x
%(%o13)                        ----- - ----- + 9 x
%                                2       2
%(%i14) expand(subst(1/3,h,l2))
%                                   3
%                               27 x         2   9 x
%(%o14)                      (- -----) + 18 x  - ---
%                                 2               2
%(%i15) expand(subst(1/3,h,l3))
%                                   3      2
%                                9 x    9 x
%(%o15)                          ---- - ---- + x
%                                 2      2
\begin{align*}
l_0(x) &= \frac{ (x-\frac13)(x-\frac23)(x-1)}{-\frac13(-\frac23)(-1)}
&&=-\frac92x^3+9x^2-\frac{11}2x+1
\\
l_1(x) &= \frac{x           (x-\frac23)(x-1)}{\frac13(-\frac13)(-\frac23)}
&&=\frac{27}2x^3-\frac{45}{2}x^2+9x
\\
l_2(x) &=-\frac{x(x-\frac13)           (x-1)}{\frac23\frac13(-\frac13)}
&&=-\frac{27}2x^3+18x^2-\frac92x
\\
l_3(x) &= \frac{x(x-\frac13)(x-\frac23)     }{\frac23\frac13}
&&=\frac92x^3-\frac92x^2+x
\end{align*}
Das Lagrange-Interpolationspolynom ist dann
\[
p(x)
=
\sum_{k=0}^3f_k l_k(x)
=
\sin\frac{\pi}6 l_1(x)
+
\sin\frac{\pi}3 l_2(x)
+
l_3(x)
=
\frac12l_1(x)
+
\frac{\sqrt{3}}2 l_2(x)
+
l_3(x)
\]
\item
Die Funktionswerte in $x=\frac12$ sind
\begin{align*}
f({\textstyle\frac12}) &= \frac{\sqrt{2}}2 = 0.70710678
\\
p({\textstyle\frac12}) &= 0.69843726
&
f({\textstyle\frac12}) - p({\textstyle\frac12}) 
&=
0.00866952
\\
h({\textstyle\frac12}) &= 0.69634954
&
f({\textstyle\frac12}) - h({\textstyle\frac12}) 
&=
0.01075724
\end{align*}
Diese Werte und die Abbildung oben zeigen, dass das
Lagrange-Interpolationspolynome etwas genauer ist.
Interessant am Hermite-Interpolationspolynome ist hingegen, dass
der Fehler im ganzen Intervall $>0$ ist, was man mit Hilfe der
Fehlerformel auch direkt nachweisen kann.
\qedhere
\end{teilaufgaben}
\end{loesung}

