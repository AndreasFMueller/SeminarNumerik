%
% 3001plot.tex -- Plot des Interpolationspolynoms
%
% (c) 2020 Prof Dr Andreas Müller, Hochschule Rapperswil
%
\documentclass[tikz]{standalone}
\usepackage{amsmath}
\usepackage{times}
\usepackage{txfonts}
\usepackage{pgfplots}
\usepackage{csvsimple}
\usetikzlibrary{arrows,intersections,math}
\begin{document}
\def\skala{1}
\begin{tikzpicture}[>=latex,thick,scale=\skala]

\pgfmathparse{13/(21*(3.14159/2))}
\xdef\xskala{\pgfmathresult}
%\def\xskala{0.5}
\def\yskala{1}

\draw[color=blue,line width=1.6pt]
	plot[domain={-10*3.14159/2}:{10*3.14159/2},samples=200]
		({\xskala*\x},{\yskala*sin(180*\x/3.14159)});
\foreach \k in {-10,...,10}{
	\pgfmathparse{\k*(3.14159/2)}
	\xdef\x{\pgfmathresult}
	\fill[color=blue] ({\xskala*\x},{\yskala*sin(180*\x/3.14159)})
		circle[radius=0.08];
}

\begin{scope}
\clip ({-0.2+-10*(3.14159/2)*\xskala},-2) rectangle ({0.2+10*(3.14159/2)*\xskala},2);
\draw[color=red,line width=1.2pt]
	({\xskala*-16.4934},{\yskala*132.0828})
	--({\xskala*-16.4148},{\yskala*102.7398})
	--({\xskala*-16.3363},{\yskala*78.7590})
	--({\xskala*-16.2577},{\yskala*59.2992})
	--({\xskala*-16.1792},{\yskala*43.6345})
	--({\xskala*-16.1007},{\yskala*31.1404})
	--({\xskala*-16.0221},{\yskala*21.2811})
	--({\xskala*-15.9436},{\yskala*13.5985})
	--({\xskala*-15.8650},{\yskala*7.7028})
	--({\xskala*-15.7865},{\yskala*3.2631})
	--({\xskala*-15.7080},{\yskala*-0.0000})
	--({\xskala*-15.6294},{\yskala*-2.3213})
	--({\xskala*-15.5509},{\yskala*-3.8970})
	--({\xskala*-15.4723},{\yskala*-4.8905})
	--({\xskala*-15.3938},{\yskala*-5.4365})
	--({\xskala*-15.3153},{\yskala*-5.6455})
	--({\xskala*-15.2367},{\yskala*-5.6073})
	--({\xskala*-15.1582},{\yskala*-5.3943})
	--({\xskala*-15.0796},{\yskala*-5.0641})
	--({\xskala*-15.0011},{\yskala*-4.6618})
	--({\xskala*-14.9226},{\yskala*-4.2223})
	--({\xskala*-14.8440},{\yskala*-3.7721})
	--({\xskala*-14.7655},{\yskala*-3.3306})
	--({\xskala*-14.6869},{\yskala*-2.9113})
	--({\xskala*-14.6084},{\yskala*-2.5235})
	--({\xskala*-14.5299},{\yskala*-2.1727})
	--({\xskala*-14.4513},{\yskala*-1.8614})
	--({\xskala*-14.3728},{\yskala*-1.5902})
	--({\xskala*-14.2942},{\yskala*-1.3579})
	--({\xskala*-14.2157},{\yskala*-1.1622})
	--({\xskala*-14.1372},{\yskala*-1.0000})
	--({\xskala*-14.0586},{\yskala*-0.8676})
	--({\xskala*-13.9801},{\yskala*-0.7613})
	--({\xskala*-13.9015},{\yskala*-0.6770})
	--({\xskala*-13.8230},{\yskala*-0.6109})
	--({\xskala*-13.7445},{\yskala*-0.5594})
	--({\xskala*-13.6659},{\yskala*-0.5189})
	--({\xskala*-13.5874},{\yskala*-0.4865})
	--({\xskala*-13.5088},{\yskala*-0.4593})
	--({\xskala*-13.4303},{\yskala*-0.4350})
	--({\xskala*-13.3518},{\yskala*-0.4115})
	--({\xskala*-13.2732},{\yskala*-0.3871})
	--({\xskala*-13.1947},{\yskala*-0.3605})
	--({\xskala*-13.1161},{\yskala*-0.3307})
	--({\xskala*-13.0376},{\yskala*-0.2969})
	--({\xskala*-12.9591},{\yskala*-0.2587})
	--({\xskala*-12.8805},{\yskala*-0.2158})
	--({\xskala*-12.8020},{\yskala*-0.1683})
	--({\xskala*-12.7235},{\yskala*-0.1162})
	--({\xskala*-12.6449},{\yskala*-0.0600})
	--({\xskala*-12.5664},{\yskala*0.0000})
	--({\xskala*-12.4878},{\yskala*0.0632})
	--({\xskala*-12.4093},{\yskala*0.1290})
	--({\xskala*-12.3308},{\yskala*0.1967})
	--({\xskala*-12.2522},{\yskala*0.2656})
	--({\xskala*-12.1737},{\yskala*0.3350})
	--({\xskala*-12.0951},{\yskala*0.4041})
	--({\xskala*-12.0166},{\yskala*0.4723})
	--({\xskala*-11.9381},{\yskala*0.5387})
	--({\xskala*-11.8595},{\yskala*0.6027})
	--({\xskala*-11.7810},{\yskala*0.6636})
	--({\xskala*-11.7024},{\yskala*0.7210})
	--({\xskala*-11.6239},{\yskala*0.7741})
	--({\xskala*-11.5454},{\yskala*0.8225})
	--({\xskala*-11.4668},{\yskala*0.8658})
	--({\xskala*-11.3883},{\yskala*0.9036})
	--({\xskala*-11.3097},{\yskala*0.9355})
	--({\xskala*-11.2312},{\yskala*0.9613})
	--({\xskala*-11.1527},{\yskala*0.9807})
	--({\xskala*-11.0741},{\yskala*0.9936})
	--({\xskala*-10.9956},{\yskala*1.0000})
	--({\xskala*-10.9170},{\yskala*0.9997})
	--({\xskala*-10.8385},{\yskala*0.9928})
	--({\xskala*-10.7600},{\yskala*0.9794})
	--({\xskala*-10.6814},{\yskala*0.9595})
	--({\xskala*-10.6029},{\yskala*0.9333})
	--({\xskala*-10.5243},{\yskala*0.9011})
	--({\xskala*-10.4458},{\yskala*0.8630})
	--({\xskala*-10.3673},{\yskala*0.8193})
	--({\xskala*-10.2887},{\yskala*0.7704})
	--({\xskala*-10.2102},{\yskala*0.7165})
	--({\xskala*-10.1316},{\yskala*0.6581})
	--({\xskala*-10.0531},{\yskala*0.5956})
	--({\xskala*-9.9746},{\yskala*0.5293})
	--({\xskala*-9.8960},{\yskala*0.4598})
	--({\xskala*-9.8175},{\yskala*0.3874})
	--({\xskala*-9.7389},{\yskala*0.3127})
	--({\xskala*-9.6604},{\yskala*0.2361})
	--({\xskala*-9.5819},{\yskala*0.1582})
	--({\xskala*-9.5033},{\yskala*0.0793})
	--({\xskala*-9.4248},{\yskala*-0.0000})
	--({\xskala*-9.3462},{\yskala*-0.0792})
	--({\xskala*-9.2677},{\yskala*-0.1578})
	--({\xskala*-9.1892},{\yskala*-0.2353})
	--({\xskala*-9.1106},{\yskala*-0.3113})
	--({\xskala*-9.0321},{\yskala*-0.3853})
	--({\xskala*-8.9535},{\yskala*-0.4568})
	--({\xskala*-8.8750},{\yskala*-0.5254})
	--({\xskala*-8.7965},{\yskala*-0.5907})
	--({\xskala*-8.7179},{\yskala*-0.6523})
	--({\xskala*-8.6394},{\yskala*-0.7099})
	--({\xskala*-8.5608},{\yskala*-0.7630})
	--({\xskala*-8.4823},{\yskala*-0.8114})
	--({\xskala*-8.4038},{\yskala*-0.8547})
	--({\xskala*-8.3252},{\yskala*-0.8928})
	--({\xskala*-8.2467},{\yskala*-0.9254})
	--({\xskala*-8.1681},{\yskala*-0.9522})
	--({\xskala*-8.0896},{\yskala*-0.9732})
	--({\xskala*-8.0111},{\yskala*-0.9882})
	--({\xskala*-7.9325},{\yskala*-0.9972})
	--({\xskala*-7.8540},{\yskala*-1.0000})
	--({\xskala*-7.7754},{\yskala*-0.9967})
	--({\xskala*-7.6969},{\yskala*-0.9872})
	--({\xskala*-7.6184},{\yskala*-0.9717})
	--({\xskala*-7.5398},{\yskala*-0.9503})
	--({\xskala*-7.4613},{\yskala*-0.9230})
	--({\xskala*-7.3827},{\yskala*-0.8900})
	--({\xskala*-7.3042},{\yskala*-0.8516})
	--({\xskala*-7.2257},{\yskala*-0.8079})
	--({\xskala*-7.1471},{\yskala*-0.7593})
	--({\xskala*-7.0686},{\yskala*-0.7061})
	--({\xskala*-6.9900},{\yskala*-0.6485})
	--({\xskala*-6.9115},{\yskala*-0.5869})
	--({\xskala*-6.8330},{\yskala*-0.5217})
	--({\xskala*-6.7544},{\yskala*-0.4533})
	--({\xskala*-6.6759},{\yskala*-0.3821})
	--({\xskala*-6.5973},{\yskala*-0.3085})
	--({\xskala*-6.5188},{\yskala*-0.2331})
	--({\xskala*-6.4403},{\yskala*-0.1562})
	--({\xskala*-6.3617},{\yskala*-0.0783})
	--({\xskala*-6.2832},{\yskala*0.0000})
	--({\xskala*-6.2046},{\yskala*0.0784})
	--({\xskala*-6.1261},{\yskala*0.1562})
	--({\xskala*-6.0476},{\yskala*0.2332})
	--({\xskala*-5.9690},{\yskala*0.3087})
	--({\xskala*-5.8905},{\yskala*0.3823})
	--({\xskala*-5.8119},{\yskala*0.4535})
	--({\xskala*-5.7334},{\yskala*0.5220})
	--({\xskala*-5.6549},{\yskala*0.5873})
	--({\xskala*-5.5763},{\yskala*0.6489})
	--({\xskala*-5.4978},{\yskala*0.7066})
	--({\xskala*-5.4192},{\yskala*0.7599})
	--({\xskala*-5.3407},{\yskala*0.8086})
	--({\xskala*-5.2622},{\yskala*0.8522})
	--({\xskala*-5.1836},{\yskala*0.8907})
	--({\xskala*-5.1051},{\yskala*0.9236})
	--({\xskala*-5.0265},{\yskala*0.9508})
	--({\xskala*-4.9480},{\yskala*0.9722})
	--({\xskala*-4.8695},{\yskala*0.9876})
	--({\xskala*-4.7909},{\yskala*0.9969})
	--({\xskala*-4.7124},{\yskala*1.0000})
	--({\xskala*-4.6338},{\yskala*0.9970})
	--({\xskala*-4.5553},{\yskala*0.9878})
	--({\xskala*-4.4768},{\yskala*0.9725})
	--({\xskala*-4.3982},{\yskala*0.9512})
	--({\xskala*-4.3197},{\yskala*0.9241})
	--({\xskala*-4.2412},{\yskala*0.8913})
	--({\xskala*-4.1626},{\yskala*0.8529})
	--({\xskala*-4.0841},{\yskala*0.8093})
	--({\xskala*-4.0055},{\yskala*0.7607})
	--({\xskala*-3.9270},{\yskala*0.7074})
	--({\xskala*-3.8485},{\yskala*0.6497})
	--({\xskala*-3.7699},{\yskala*0.5880})
	--({\xskala*-3.6914},{\yskala*0.5227})
	--({\xskala*-3.6128},{\yskala*0.4542})
	--({\xskala*-3.5343},{\yskala*0.3829})
	--({\xskala*-3.4558},{\yskala*0.3092})
	--({\xskala*-3.3772},{\yskala*0.2336})
	--({\xskala*-3.2987},{\yskala*0.1565})
	--({\xskala*-3.2201},{\yskala*0.0785})
	--({\xskala*-3.1416},{\yskala*-0.0000})
	--({\xskala*-3.0631},{\yskala*-0.0785})
	--({\xskala*-2.9845},{\yskala*-0.1565})
	--({\xskala*-2.9060},{\yskala*-0.2335})
	--({\xskala*-2.8274},{\yskala*-0.3091})
	--({\xskala*-2.7489},{\yskala*-0.3828})
	--({\xskala*-2.6704},{\yskala*-0.4542})
	--({\xskala*-2.5918},{\yskala*-0.5227})
	--({\xskala*-2.5133},{\yskala*-0.5880})
	--({\xskala*-2.4347},{\yskala*-0.6496})
	--({\xskala*-2.3562},{\yskala*-0.7073})
	--({\xskala*-2.2777},{\yskala*-0.7606})
	--({\xskala*-2.1991},{\yskala*-0.8092})
	--({\xskala*-2.1206},{\yskala*-0.8528})
	--({\xskala*-2.0420},{\yskala*-0.8912})
	--({\xskala*-1.9635},{\yskala*-0.9240})
	--({\xskala*-1.8850},{\yskala*-0.9512})
	--({\xskala*-1.8064},{\yskala*-0.9725})
	--({\xskala*-1.7279},{\yskala*-0.9877})
	--({\xskala*-1.6493},{\yskala*-0.9969})
	--({\xskala*-1.5708},{\yskala*-1.0000})
	--({\xskala*-1.4923},{\yskala*-0.9969})
	--({\xskala*-1.4137},{\yskala*-0.9876})
	--({\xskala*-1.3352},{\yskala*-0.9723})
	--({\xskala*-1.2566},{\yskala*-0.9510})
	--({\xskala*-1.1781},{\yskala*-0.9238})
	--({\xskala*-1.0996},{\yskala*-0.8909})
	--({\xskala*-1.0210},{\yskala*-0.8525})
	--({\xskala*-0.9425},{\yskala*-0.8089})
	--({\xskala*-0.8639},{\yskala*-0.7602})
	--({\xskala*-0.7854},{\yskala*-0.7069})
	--({\xskala*-0.7069},{\yskala*-0.6493})
	--({\xskala*-0.6283},{\yskala*-0.5876})
	--({\xskala*-0.5498},{\yskala*-0.5224})
	--({\xskala*-0.4712},{\yskala*-0.4539})
	--({\xskala*-0.3927},{\yskala*-0.3826})
	--({\xskala*-0.3142},{\yskala*-0.3089})
	--({\xskala*-0.2356},{\yskala*-0.2334})
	--({\xskala*-0.1571},{\yskala*-0.1564})
	--({\xskala*-0.0785},{\yskala*-0.0784})
	--({\xskala*0.0000},{\yskala*0.0000})
	--({\xskala*0.0785},{\yskala*0.0784})
	--({\xskala*0.1571},{\yskala*0.1564})
	--({\xskala*0.2356},{\yskala*0.2334})
	--({\xskala*0.3142},{\yskala*0.3089})
	--({\xskala*0.3927},{\yskala*0.3826})
	--({\xskala*0.4712},{\yskala*0.4539})
	--({\xskala*0.5498},{\yskala*0.5224})
	--({\xskala*0.6283},{\yskala*0.5876})
	--({\xskala*0.7069},{\yskala*0.6493})
	--({\xskala*0.7854},{\yskala*0.7069})
	--({\xskala*0.8639},{\yskala*0.7602})
	--({\xskala*0.9425},{\yskala*0.8089})
	--({\xskala*1.0210},{\yskala*0.8525})
	--({\xskala*1.0996},{\yskala*0.8909})
	--({\xskala*1.1781},{\yskala*0.9238})
	--({\xskala*1.2566},{\yskala*0.9510})
	--({\xskala*1.3352},{\yskala*0.9723})
	--({\xskala*1.4137},{\yskala*0.9876})
	--({\xskala*1.4923},{\yskala*0.9969})
	--({\xskala*1.5708},{\yskala*1.0000})
	--({\xskala*1.6493},{\yskala*0.9969})
	--({\xskala*1.7279},{\yskala*0.9877})
	--({\xskala*1.8064},{\yskala*0.9725})
	--({\xskala*1.8850},{\yskala*0.9512})
	--({\xskala*1.9635},{\yskala*0.9240})
	--({\xskala*2.0420},{\yskala*0.8912})
	--({\xskala*2.1206},{\yskala*0.8528})
	--({\xskala*2.1991},{\yskala*0.8092})
	--({\xskala*2.2777},{\yskala*0.7606})
	--({\xskala*2.3562},{\yskala*0.7073})
	--({\xskala*2.4347},{\yskala*0.6496})
	--({\xskala*2.5133},{\yskala*0.5880})
	--({\xskala*2.5918},{\yskala*0.5227})
	--({\xskala*2.6704},{\yskala*0.4542})
	--({\xskala*2.7489},{\yskala*0.3828})
	--({\xskala*2.8274},{\yskala*0.3091})
	--({\xskala*2.9060},{\yskala*0.2335})
	--({\xskala*2.9845},{\yskala*0.1565})
	--({\xskala*3.0631},{\yskala*0.0785})
	--({\xskala*3.1416},{\yskala*0.0000})
	--({\xskala*3.2201},{\yskala*-0.0785})
	--({\xskala*3.2987},{\yskala*-0.1565})
	--({\xskala*3.3772},{\yskala*-0.2336})
	--({\xskala*3.4558},{\yskala*-0.3092})
	--({\xskala*3.5343},{\yskala*-0.3829})
	--({\xskala*3.6128},{\yskala*-0.4542})
	--({\xskala*3.6914},{\yskala*-0.5227})
	--({\xskala*3.7699},{\yskala*-0.5880})
	--({\xskala*3.8485},{\yskala*-0.6497})
	--({\xskala*3.9270},{\yskala*-0.7074})
	--({\xskala*4.0055},{\yskala*-0.7607})
	--({\xskala*4.0841},{\yskala*-0.8093})
	--({\xskala*4.1626},{\yskala*-0.8529})
	--({\xskala*4.2412},{\yskala*-0.8913})
	--({\xskala*4.3197},{\yskala*-0.9241})
	--({\xskala*4.3982},{\yskala*-0.9512})
	--({\xskala*4.4768},{\yskala*-0.9725})
	--({\xskala*4.5553},{\yskala*-0.9878})
	--({\xskala*4.6338},{\yskala*-0.9970})
	--({\xskala*4.7124},{\yskala*-1.0000})
	--({\xskala*4.7909},{\yskala*-0.9969})
	--({\xskala*4.8695},{\yskala*-0.9876})
	--({\xskala*4.9480},{\yskala*-0.9722})
	--({\xskala*5.0265},{\yskala*-0.9508})
	--({\xskala*5.1051},{\yskala*-0.9236})
	--({\xskala*5.1836},{\yskala*-0.8907})
	--({\xskala*5.2622},{\yskala*-0.8522})
	--({\xskala*5.3407},{\yskala*-0.8086})
	--({\xskala*5.4192},{\yskala*-0.7599})
	--({\xskala*5.4978},{\yskala*-0.7066})
	--({\xskala*5.5763},{\yskala*-0.6489})
	--({\xskala*5.6549},{\yskala*-0.5873})
	--({\xskala*5.7334},{\yskala*-0.5220})
	--({\xskala*5.8119},{\yskala*-0.4535})
	--({\xskala*5.8905},{\yskala*-0.3823})
	--({\xskala*5.9690},{\yskala*-0.3087})
	--({\xskala*6.0476},{\yskala*-0.2332})
	--({\xskala*6.1261},{\yskala*-0.1562})
	--({\xskala*6.2046},{\yskala*-0.0784})
	--({\xskala*6.2832},{\yskala*-0.0000})
	--({\xskala*6.3617},{\yskala*0.0783})
	--({\xskala*6.4403},{\yskala*0.1562})
	--({\xskala*6.5188},{\yskala*0.2331})
	--({\xskala*6.5973},{\yskala*0.3085})
	--({\xskala*6.6759},{\yskala*0.3821})
	--({\xskala*6.7544},{\yskala*0.4533})
	--({\xskala*6.8330},{\yskala*0.5217})
	--({\xskala*6.9115},{\yskala*0.5869})
	--({\xskala*6.9900},{\yskala*0.6485})
	--({\xskala*7.0686},{\yskala*0.7061})
	--({\xskala*7.1471},{\yskala*0.7593})
	--({\xskala*7.2257},{\yskala*0.8079})
	--({\xskala*7.3042},{\yskala*0.8516})
	--({\xskala*7.3827},{\yskala*0.8900})
	--({\xskala*7.4613},{\yskala*0.9230})
	--({\xskala*7.5398},{\yskala*0.9503})
	--({\xskala*7.6184},{\yskala*0.9717})
	--({\xskala*7.6969},{\yskala*0.9872})
	--({\xskala*7.7754},{\yskala*0.9967})
	--({\xskala*7.8540},{\yskala*1.0000})
	--({\xskala*7.9325},{\yskala*0.9972})
	--({\xskala*8.0111},{\yskala*0.9882})
	--({\xskala*8.0896},{\yskala*0.9732})
	--({\xskala*8.1681},{\yskala*0.9522})
	--({\xskala*8.2467},{\yskala*0.9254})
	--({\xskala*8.3252},{\yskala*0.8928})
	--({\xskala*8.4038},{\yskala*0.8547})
	--({\xskala*8.4823},{\yskala*0.8114})
	--({\xskala*8.5608},{\yskala*0.7630})
	--({\xskala*8.6394},{\yskala*0.7099})
	--({\xskala*8.7179},{\yskala*0.6523})
	--({\xskala*8.7965},{\yskala*0.5907})
	--({\xskala*8.8750},{\yskala*0.5254})
	--({\xskala*8.9535},{\yskala*0.4568})
	--({\xskala*9.0321},{\yskala*0.3853})
	--({\xskala*9.1106},{\yskala*0.3113})
	--({\xskala*9.1892},{\yskala*0.2353})
	--({\xskala*9.2677},{\yskala*0.1578})
	--({\xskala*9.3462},{\yskala*0.0792})
	--({\xskala*9.4248},{\yskala*0.0000})
	--({\xskala*9.5033},{\yskala*-0.0793})
	--({\xskala*9.5819},{\yskala*-0.1582})
	--({\xskala*9.6604},{\yskala*-0.2361})
	--({\xskala*9.7389},{\yskala*-0.3127})
	--({\xskala*9.8175},{\yskala*-0.3874})
	--({\xskala*9.8960},{\yskala*-0.4598})
	--({\xskala*9.9746},{\yskala*-0.5293})
	--({\xskala*10.0531},{\yskala*-0.5956})
	--({\xskala*10.1316},{\yskala*-0.6581})
	--({\xskala*10.2102},{\yskala*-0.7165})
	--({\xskala*10.2887},{\yskala*-0.7704})
	--({\xskala*10.3673},{\yskala*-0.8193})
	--({\xskala*10.4458},{\yskala*-0.8630})
	--({\xskala*10.5243},{\yskala*-0.9011})
	--({\xskala*10.6029},{\yskala*-0.9333})
	--({\xskala*10.6814},{\yskala*-0.9595})
	--({\xskala*10.7600},{\yskala*-0.9794})
	--({\xskala*10.8385},{\yskala*-0.9928})
	--({\xskala*10.9170},{\yskala*-0.9997})
	--({\xskala*10.9956},{\yskala*-1.0000})
	--({\xskala*11.0741},{\yskala*-0.9936})
	--({\xskala*11.1527},{\yskala*-0.9807})
	--({\xskala*11.2312},{\yskala*-0.9613})
	--({\xskala*11.3097},{\yskala*-0.9355})
	--({\xskala*11.3883},{\yskala*-0.9036})
	--({\xskala*11.4668},{\yskala*-0.8658})
	--({\xskala*11.5454},{\yskala*-0.8225})
	--({\xskala*11.6239},{\yskala*-0.7741})
	--({\xskala*11.7024},{\yskala*-0.7210})
	--({\xskala*11.7810},{\yskala*-0.6636})
	--({\xskala*11.8595},{\yskala*-0.6027})
	--({\xskala*11.9381},{\yskala*-0.5387})
	--({\xskala*12.0166},{\yskala*-0.4723})
	--({\xskala*12.0951},{\yskala*-0.4041})
	--({\xskala*12.1737},{\yskala*-0.3350})
	--({\xskala*12.2522},{\yskala*-0.2656})
	--({\xskala*12.3308},{\yskala*-0.1967})
	--({\xskala*12.4093},{\yskala*-0.1290})
	--({\xskala*12.4878},{\yskala*-0.0632})
	--({\xskala*12.5664},{\yskala*-0.0000})
	--({\xskala*12.6449},{\yskala*0.0600})
	--({\xskala*12.7235},{\yskala*0.1162})
	--({\xskala*12.8020},{\yskala*0.1683})
	--({\xskala*12.8805},{\yskala*0.2158})
	--({\xskala*12.9591},{\yskala*0.2587})
	--({\xskala*13.0376},{\yskala*0.2969})
	--({\xskala*13.1161},{\yskala*0.3307})
	--({\xskala*13.1947},{\yskala*0.3605})
	--({\xskala*13.2732},{\yskala*0.3871})
	--({\xskala*13.3518},{\yskala*0.4115})
	--({\xskala*13.4303},{\yskala*0.4350})
	--({\xskala*13.5088},{\yskala*0.4593})
	--({\xskala*13.5874},{\yskala*0.4865})
	--({\xskala*13.6659},{\yskala*0.5189})
	--({\xskala*13.7445},{\yskala*0.5594})
	--({\xskala*13.8230},{\yskala*0.6109})
	--({\xskala*13.9015},{\yskala*0.6770})
	--({\xskala*13.9801},{\yskala*0.7613})
	--({\xskala*14.0586},{\yskala*0.8676})
	--({\xskala*14.1372},{\yskala*1.0000})
	--({\xskala*14.2157},{\yskala*1.1622})
	--({\xskala*14.2942},{\yskala*1.3579})
	--({\xskala*14.3728},{\yskala*1.5902})
	--({\xskala*14.4513},{\yskala*1.8614})
	--({\xskala*14.5299},{\yskala*2.1727})
	--({\xskala*14.6084},{\yskala*2.5235})
	--({\xskala*14.6869},{\yskala*2.9113})
	--({\xskala*14.7655},{\yskala*3.3306})
	--({\xskala*14.8440},{\yskala*3.7721})
	--({\xskala*14.9226},{\yskala*4.2223})
	--({\xskala*15.0011},{\yskala*4.6618})
	--({\xskala*15.0796},{\yskala*5.0641})
	--({\xskala*15.1582},{\yskala*5.3943})
	--({\xskala*15.2367},{\yskala*5.6073})
	--({\xskala*15.3153},{\yskala*5.6455})
	--({\xskala*15.3938},{\yskala*5.4365})
	--({\xskala*15.4723},{\yskala*4.8905})
	--({\xskala*15.5509},{\yskala*3.8970})
	--({\xskala*15.6294},{\yskala*2.3213})
	--({\xskala*15.7080},{\yskala*0.0000})
	--({\xskala*15.7865},{\yskala*-3.2631})
	--({\xskala*15.8650},{\yskala*-7.7028})
	--({\xskala*15.9436},{\yskala*-13.5985})
	--({\xskala*16.0221},{\yskala*-21.2811})
	--({\xskala*16.1007},{\yskala*-31.1404})
	--({\xskala*16.1792},{\yskala*-43.6345})
	--({\xskala*16.2577},{\yskala*-59.2992})
	--({\xskala*16.3363},{\yskala*-78.7590})
	--({\xskala*16.4148},{\yskala*-102.7398})
	--({\xskala*16.4934},{\yskala*-132.0828});

\end{scope}

\draw[->] ({-0.1-\xskala*10*(3.14159/2)},0)--({0.3+\xskala*10*(3.14159/2)},0)
	coordinate[label={$x$}];
\draw[->] (0,-2.1)--(0,2.3) coordinate[label={right:$y$}];

\node at ({\xskala*3.14159},0) [below left] {$\pi\mathstrut$};
\node at ({\xskala*2*3.14159},0) [above left] {$2\pi\mathstrut$};
\node at ({\xskala*3*3.14159},0) [below left] {$3\pi\mathstrut$};
\node at ({\xskala*4*3.14159},0) [above left] {$4\pi\mathstrut$};
\node at ({\xskala*5*3.14159},0) [below left] {$5\pi\mathstrut$};

\node at ({-\xskala*3.14159},0) [above right] {$-\pi\mathstrut$};
\node at ({-\xskala*2*3.14159},0) [below right] {$-2\pi\mathstrut$};
\node at ({-\xskala*3*3.14159},0) [above right] {$-3\pi\mathstrut$};
\node at ({-\xskala*4*3.14159},0) [below right] {$-4\pi\mathstrut$};
\node at ({-\xskala*5*3.14159},0) [above right] {$-5\pi\mathstrut$};

\end{tikzpicture}
\end{document}

