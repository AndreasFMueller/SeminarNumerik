%
% 3003kreis.tex -- template for standalon tikz images
%
% (c) 2020 Prof Dr Andreas Müller, Hochschule Rapperswil
%
\documentclass[tikz]{standalone}
\usepackage{amsmath}
\usepackage{times}
\usepackage{txfonts}
\usepackage{pgfplots}
\usepackage{csvsimple}
\usetikzlibrary{arrows,intersections,math}
\begin{document}
\def\skala{1}
\begin{tikzpicture}[>=latex,thick,scale=\skala]

\pgfmathparse{4*(sqrt(2)-1)/3}
\xdef\a{\pgfmathresult}

\def\r{5}

\def\kreis#1#2#3{
	\draw[line width=#3,color=#2]
		plot[domain=0:1,samples=50]
			({\r*((1-\x)*(1-\x)*(1-\x)*0+3*(1-\x)*(1-\x)*\x*#1+3*(1-\x)*\x*\x*1+\x*\x*\x*1)},{\r*((1-\x)*(1-\x)*(1-\x)*1+3*(1-\x)*(1-\x)*\x*1+3*(1-\x)*\x*\x*#1+\x*\x*\x*0)});
}

\coordinate (A) at (\r,0);
\coordinate (B) at (\r,{\r*\a});
\coordinate (C) at ({\r*\a},\r);
\coordinate (D) at (0,\r);

\fill[color=red!10] (\r,0) arc (0:90:\r) -- (0,0) -- cycle;

\foreach \A in {0.1,0.2,...,1.2}{
	\kreis{\A}{blue!40}{0.7pt}
}
\kreis{0.55284}{blue}{1pt}

\draw[color=gray!40,line width=1.4pt] (A) -- (B) -- (C) -- (D);

\draw[->] (-0.1,0) -- ({\r+0.5},0) coordinate[label={$x$}];
\draw[->] (0,-0.1) -- (0,{\r+0.5}) coordinate[label={right:$y$}];

\node[color=blue!40] at ({0.55*\r},{0.55*\r}) [below left] {$a=0.1$};
\node[color=blue!40] at ({0.95*\r},{0.95*\r}) [above right] {$a=1.2$};

\fill[color=white,opacity=0.9] ({0.707*\r-1.1},{0.707*\r-0.3})
	rectangle ({0.707*\r+1.1},{0.707*\r+0.3});
\node[color=blue] at ({0.707*\r},{0.707*\r}) {$a=\frac43(\!\sqrt{2}-1)$};

\fill[color=red] (A) circle[radius=0.08];
\fill[color=red] (B) circle[radius=0.08];
\fill[color=red] (C) circle[radius=0.08];
\fill[color=red] (D) circle[radius=0.08];

\node at (B) [right] {$(1,a)$};
\node at (C) [above] {$(a,1)$};

\end{tikzpicture}
\end{document}

