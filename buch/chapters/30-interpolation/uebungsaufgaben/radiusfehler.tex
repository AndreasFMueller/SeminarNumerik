%
% abweichungen.tex -- 
%
% (c) 2020 Prof Dr Andreas Müller, Hochschule Rapperswil
%
\documentclass[tikz]{standalone}
\usepackage{amsmath}
\usepackage{times}
\usepackage{txfonts}
\usepackage{pgfplots}
\usepackage{csvsimple}
\usetikzlibrary{arrows,intersections,math}
\begin{document}
\def\skala{1}
\begin{tikzpicture}[>=latex,thick,scale=\skala]

\def\fehlerpfad#1#2{
\draw[line width=1.4pt,color=red] ({0.00000*#1},{0.00000*#2})
-- ({0.00500*#1},{0.00071*#2})
-- ({0.01000*#1},{0.00277*#2})
-- ({0.01500*#1},{0.00605*#2})
-- ({0.02000*#1},{0.01042*#2})
-- ({0.02500*#1},{0.01578*#2})
-- ({0.03000*#1},{0.02203*#2})
-- ({0.03500*#1},{0.02904*#2})
-- ({0.04000*#1},{0.03674*#2})
-- ({0.04500*#1},{0.04502*#2})
-- ({0.05000*#1},{0.05380*#2})
-- ({0.05500*#1},{0.06299*#2})
-- ({0.06000*#1},{0.07251*#2})
-- ({0.06500*#1},{0.08229*#2})
-- ({0.07000*#1},{0.09226*#2})
-- ({0.07500*#1},{0.10236*#2})
-- ({0.08000*#1},{0.11251*#2})
-- ({0.08500*#1},{0.12266*#2})
-- ({0.09000*#1},{0.13276*#2})
-- ({0.09500*#1},{0.14275*#2})
-- ({0.10000*#1},{0.15259*#2})
-- ({0.10500*#1},{0.16223*#2})
-- ({0.11000*#1},{0.17164*#2})
-- ({0.11500*#1},{0.18077*#2})
-- ({0.12000*#1},{0.18959*#2})
-- ({0.12500*#1},{0.19807*#2})
-- ({0.13000*#1},{0.20618*#2})
-- ({0.13500*#1},{0.21389*#2})
-- ({0.14000*#1},{0.22119*#2})
-- ({0.14500*#1},{0.22805*#2})
-- ({0.15000*#1},{0.23446*#2})
-- ({0.15500*#1},{0.24039*#2})
-- ({0.16000*#1},{0.24584*#2})
-- ({0.16500*#1},{0.25080*#2})
-- ({0.17000*#1},{0.25526*#2})
-- ({0.17500*#1},{0.25921*#2})
-- ({0.18000*#1},{0.26265*#2})
-- ({0.18500*#1},{0.26557*#2})
-- ({0.19000*#1},{0.26798*#2})
-- ({0.19500*#1},{0.26987*#2})
-- ({0.20000*#1},{0.27126*#2})
-- ({0.20500*#1},{0.27213*#2})
-- ({0.21000*#1},{0.27251*#2})
-- ({0.21500*#1},{0.27240*#2})
-- ({0.22000*#1},{0.27180*#2})
-- ({0.22500*#1},{0.27073*#2})
-- ({0.23000*#1},{0.26919*#2})
-- ({0.23500*#1},{0.26721*#2})
-- ({0.24000*#1},{0.26479*#2})
-- ({0.24500*#1},{0.26194*#2})
-- ({0.25000*#1},{0.25869*#2})
-- ({0.25500*#1},{0.25505*#2})
-- ({0.26000*#1},{0.25104*#2})
-- ({0.26500*#1},{0.24666*#2})
-- ({0.27000*#1},{0.24195*#2})
-- ({0.27500*#1},{0.23693*#2})
-- ({0.28000*#1},{0.23160*#2})
-- ({0.28500*#1},{0.22599*#2})
-- ({0.29000*#1},{0.22012*#2})
-- ({0.29500*#1},{0.21401*#2})
-- ({0.30000*#1},{0.20769*#2})
-- ({0.30500*#1},{0.20116*#2})
-- ({0.31000*#1},{0.19447*#2})
-- ({0.31500*#1},{0.18761*#2})
-- ({0.32000*#1},{0.18063*#2})
-- ({0.32500*#1},{0.17353*#2})
-- ({0.33000*#1},{0.16634*#2})
-- ({0.33500*#1},{0.15908*#2})
-- ({0.34000*#1},{0.15178*#2})
-- ({0.34500*#1},{0.14445*#2})
-- ({0.35000*#1},{0.13711*#2})
-- ({0.35500*#1},{0.12979*#2})
-- ({0.36000*#1},{0.12250*#2})
-- ({0.36500*#1},{0.11527*#2})
-- ({0.37000*#1},{0.10812*#2})
-- ({0.37500*#1},{0.10106*#2})
-- ({0.38000*#1},{0.09411*#2})
-- ({0.38500*#1},{0.08730*#2})
-- ({0.39000*#1},{0.08063*#2})
-- ({0.39500*#1},{0.07414*#2})
-- ({0.40000*#1},{0.06782*#2})
-- ({0.40500*#1},{0.06171*#2})
-- ({0.41000*#1},{0.05581*#2})
-- ({0.41500*#1},{0.05014*#2})
-- ({0.42000*#1},{0.04472*#2})
-- ({0.42500*#1},{0.03955*#2})
-- ({0.43000*#1},{0.03466*#2})
-- ({0.43500*#1},{0.03005*#2})
-- ({0.44000*#1},{0.02574*#2})
-- ({0.44500*#1},{0.02173*#2})
-- ({0.45000*#1},{0.01803*#2})
-- ({0.45500*#1},{0.01466*#2})
-- ({0.46000*#1},{0.01162*#2})
-- ({0.46500*#1},{0.00893*#2})
-- ({0.47000*#1},{0.00658*#2})
-- ({0.47500*#1},{0.00458*#2})
-- ({0.48000*#1},{0.00293*#2})
-- ({0.48500*#1},{0.00165*#2})
-- ({0.49000*#1},{0.00074*#2})
-- ({0.49500*#1},{0.00018*#2})
-- ({0.50000*#1},{0.00000*#2})
-- ({0.50500*#1},{0.00018*#2})
-- ({0.51000*#1},{0.00074*#2})
-- ({0.51500*#1},{0.00165*#2})
-- ({0.52000*#1},{0.00293*#2})
-- ({0.52500*#1},{0.00458*#2})
-- ({0.53000*#1},{0.00658*#2})
-- ({0.53500*#1},{0.00893*#2})
-- ({0.54000*#1},{0.01162*#2})
-- ({0.54500*#1},{0.01466*#2})
-- ({0.55000*#1},{0.01803*#2})
-- ({0.55500*#1},{0.02173*#2})
-- ({0.56000*#1},{0.02574*#2})
-- ({0.56500*#1},{0.03005*#2})
-- ({0.57000*#1},{0.03466*#2})
-- ({0.57500*#1},{0.03955*#2})
-- ({0.58000*#1},{0.04472*#2})
-- ({0.58500*#1},{0.05014*#2})
-- ({0.59000*#1},{0.05581*#2})
-- ({0.59500*#1},{0.06171*#2})
-- ({0.60000*#1},{0.06782*#2})
-- ({0.60500*#1},{0.07414*#2})
-- ({0.61000*#1},{0.08063*#2})
-- ({0.61500*#1},{0.08730*#2})
-- ({0.62000*#1},{0.09411*#2})
-- ({0.62500*#1},{0.10106*#2})
-- ({0.63000*#1},{0.10812*#2})
-- ({0.63500*#1},{0.11527*#2})
-- ({0.64000*#1},{0.12250*#2})
-- ({0.64500*#1},{0.12979*#2})
-- ({0.65000*#1},{0.13711*#2})
-- ({0.65500*#1},{0.14445*#2})
-- ({0.66000*#1},{0.15178*#2})
-- ({0.66500*#1},{0.15908*#2})
-- ({0.67000*#1},{0.16634*#2})
-- ({0.67500*#1},{0.17353*#2})
-- ({0.68000*#1},{0.18063*#2})
-- ({0.68500*#1},{0.18761*#2})
-- ({0.69000*#1},{0.19447*#2})
-- ({0.69500*#1},{0.20116*#2})
-- ({0.70000*#1},{0.20769*#2})
-- ({0.70500*#1},{0.21401*#2})
-- ({0.71000*#1},{0.22012*#2})
-- ({0.71500*#1},{0.22599*#2})
-- ({0.72000*#1},{0.23160*#2})
-- ({0.72500*#1},{0.23693*#2})
-- ({0.73000*#1},{0.24195*#2})
-- ({0.73500*#1},{0.24666*#2})
-- ({0.74000*#1},{0.25104*#2})
-- ({0.74500*#1},{0.25505*#2})
-- ({0.75000*#1},{0.25869*#2})
-- ({0.75500*#1},{0.26194*#2})
-- ({0.76000*#1},{0.26479*#2})
-- ({0.76500*#1},{0.26721*#2})
-- ({0.77000*#1},{0.26919*#2})
-- ({0.77500*#1},{0.27073*#2})
-- ({0.78000*#1},{0.27180*#2})
-- ({0.78500*#1},{0.27240*#2})
-- ({0.79000*#1},{0.27251*#2})
-- ({0.79500*#1},{0.27213*#2})
-- ({0.80000*#1},{0.27126*#2})
-- ({0.80500*#1},{0.26987*#2})
-- ({0.81000*#1},{0.26798*#2})
-- ({0.81500*#1},{0.26557*#2})
-- ({0.82000*#1},{0.26265*#2})
-- ({0.82500*#1},{0.25921*#2})
-- ({0.83000*#1},{0.25526*#2})
-- ({0.83500*#1},{0.25080*#2})
-- ({0.84000*#1},{0.24584*#2})
-- ({0.84500*#1},{0.24039*#2})
-- ({0.85000*#1},{0.23446*#2})
-- ({0.85500*#1},{0.22805*#2})
-- ({0.86000*#1},{0.22119*#2})
-- ({0.86500*#1},{0.21389*#2})
-- ({0.87000*#1},{0.20618*#2})
-- ({0.87500*#1},{0.19807*#2})
-- ({0.88000*#1},{0.18959*#2})
-- ({0.88500*#1},{0.18077*#2})
-- ({0.89000*#1},{0.17164*#2})
-- ({0.89500*#1},{0.16223*#2})
-- ({0.90000*#1},{0.15259*#2})
-- ({0.90500*#1},{0.14275*#2})
-- ({0.91000*#1},{0.13276*#2})
-- ({0.91500*#1},{0.12266*#2})
-- ({0.92000*#1},{0.11251*#2})
-- ({0.92500*#1},{0.10236*#2})
-- ({0.93000*#1},{0.09226*#2})
-- ({0.93500*#1},{0.08229*#2})
-- ({0.94000*#1},{0.07251*#2})
-- ({0.94500*#1},{0.06299*#2})
-- ({0.95000*#1},{0.05380*#2})
-- ({0.95500*#1},{0.04502*#2})
-- ({0.96000*#1},{0.03674*#2})
-- ({0.96500*#1},{0.02904*#2})
-- ({0.97000*#1},{0.02203*#2})
-- ({0.97500*#1},{0.01578*#2})
-- ({0.98000*#1},{0.01042*#2})
-- ({0.98500*#1},{0.00605*#2})
-- ({0.99000*#1},{0.00277*#2})
-- ({0.99500*#1},{0.00071*#2})
-- ({1.00000*#1},{0.00000*#2});
}

\def\kurvepfad#1#2{
\fill[color=red!20] ({0.00000*#1},{1.00000*#2})    arc (90:0:{#1}) -- ({1.00000*#1},{0.00000*#2})
-- ({1.00032*#1},{0.00828*#2})
-- ({1.00125*#1},{0.01656*#2})
-- ({1.00272*#1},{0.02486*#2})
-- ({1.00467*#1},{0.03318*#2})
-- ({1.00705*#1},{0.04155*#2})
-- ({1.00980*#1},{0.04996*#2})
-- ({1.01287*#1},{0.05843*#2})
-- ({1.01620*#1},{0.06696*#2})
-- ({1.01976*#1},{0.07556*#2})
-- ({1.02349*#1},{0.08422*#2})
-- ({1.02736*#1},{0.09296*#2})
-- ({1.03132*#1},{0.10177*#2})
-- ({1.03534*#1},{0.11065*#2})
-- ({1.03937*#1},{0.11960*#2})
-- ({1.04339*#1},{0.12862*#2})
-- ({1.04736*#1},{0.13770*#2})
-- ({1.05125*#1},{0.14684*#2})
-- ({1.05504*#1},{0.15604*#2})
-- ({1.05870*#1},{0.16529*#2})
-- ({1.06221*#1},{0.17458*#2})
-- ({1.06554*#1},{0.18391*#2})
-- ({1.06867*#1},{0.19328*#2})
-- ({1.07158*#1},{0.20267*#2})
-- ({1.07427*#1},{0.21207*#2})
-- ({1.07671*#1},{0.22149*#2})
-- ({1.07888*#1},{0.23092*#2})
-- ({1.08078*#1},{0.24033*#2})
-- ({1.08240*#1},{0.24974*#2})
-- ({1.08373*#1},{0.25912*#2})
-- ({1.08476*#1},{0.26848*#2})
-- ({1.08548*#1},{0.27781*#2})
-- ({1.08589*#1},{0.28709*#2})
-- ({1.08598*#1},{0.29632*#2})
-- ({1.08576*#1},{0.30549*#2})
-- ({1.08522*#1},{0.31460*#2})
-- ({1.08435*#1},{0.32364*#2})
-- ({1.08317*#1},{0.33260*#2})
-- ({1.08167*#1},{0.34147*#2})
-- ({1.07986*#1},{0.35026*#2})
-- ({1.07773*#1},{0.35895*#2})
-- ({1.07530*#1},{0.36754*#2})
-- ({1.07256*#1},{0.37602*#2})
-- ({1.06953*#1},{0.38439*#2})
-- ({1.06621*#1},{0.39265*#2})
-- ({1.06260*#1},{0.40079*#2})
-- ({1.05872*#1},{0.40880*#2})
-- ({1.05457*#1},{0.41669*#2})
-- ({1.05016*#1},{0.42446*#2})
-- ({1.04550*#1},{0.43209*#2})
-- ({1.04060*#1},{0.43959*#2})
-- ({1.03547*#1},{0.44696*#2})
-- ({1.03011*#1},{0.45420*#2})
-- ({1.02455*#1},{0.46130*#2})
-- ({1.01879*#1},{0.46826*#2})
-- ({1.01284*#1},{0.47509*#2})
-- ({1.00671*#1},{0.48179*#2})
-- ({1.00041*#1},{0.48836*#2})
-- ({0.99396*#1},{0.49479*#2})
-- ({0.98736*#1},{0.50110*#2})
-- ({0.98063*#1},{0.50728*#2})
-- ({0.97378*#1},{0.51334*#2})
-- ({0.96682*#1},{0.51927*#2})
-- ({0.95976*#1},{0.52509*#2})
-- ({0.95261*#1},{0.53079*#2})
-- ({0.94539*#1},{0.53639*#2})
-- ({0.93809*#1},{0.54187*#2})
-- ({0.93074*#1},{0.54726*#2})
-- ({0.92335*#1},{0.55255*#2})
-- ({0.91592*#1},{0.55775*#2})
-- ({0.90846*#1},{0.56287*#2})
-- ({0.90099*#1},{0.56790*#2})
-- ({0.89351*#1},{0.57287*#2})
-- ({0.88603*#1},{0.57776*#2})
-- ({0.87856*#1},{0.58259*#2})
-- ({0.87112*#1},{0.58736*#2})
-- ({0.86369*#1},{0.59209*#2})
-- ({0.85631*#1},{0.59677*#2})
-- ({0.84896*#1},{0.60141*#2})
-- ({0.84167*#1},{0.60602*#2})
-- ({0.83443*#1},{0.61061*#2})
-- ({0.82725*#1},{0.61519*#2})
-- ({0.82013*#1},{0.61975*#2})
-- ({0.81309*#1},{0.62431*#2})
-- ({0.80612*#1},{0.62887*#2})
-- ({0.79924*#1},{0.63344*#2})
-- ({0.79243*#1},{0.63803*#2})
-- ({0.78572*#1},{0.64264*#2})
-- ({0.77910*#1},{0.64727*#2})
-- ({0.77256*#1},{0.65195*#2})
-- ({0.76613*#1},{0.65666*#2})
-- ({0.75979*#1},{0.66141*#2})
-- ({0.75355*#1},{0.66622*#2})
-- ({0.74740*#1},{0.67109*#2})
-- ({0.74136*#1},{0.67601*#2})
-- ({0.73541*#1},{0.68100*#2})
-- ({0.72956*#1},{0.68607*#2})
-- ({0.72380*#1},{0.69121*#2})
-- ({0.71815*#1},{0.69642*#2})
-- ({0.71258*#1},{0.70172*#2})
-- ({0.70711*#1},{0.70711*#2})
-- ({0.70172*#1},{0.71258*#2})
-- ({0.69642*#1},{0.71815*#2})
-- ({0.69121*#1},{0.72380*#2})
-- ({0.68607*#1},{0.72956*#2})
-- ({0.68100*#1},{0.73541*#2})
-- ({0.67601*#1},{0.74136*#2})
-- ({0.67109*#1},{0.74740*#2})
-- ({0.66622*#1},{0.75355*#2})
-- ({0.66141*#1},{0.75979*#2})
-- ({0.65666*#1},{0.76613*#2})
-- ({0.65195*#1},{0.77256*#2})
-- ({0.64727*#1},{0.77910*#2})
-- ({0.64264*#1},{0.78572*#2})
-- ({0.63803*#1},{0.79243*#2})
-- ({0.63344*#1},{0.79924*#2})
-- ({0.62887*#1},{0.80612*#2})
-- ({0.62431*#1},{0.81309*#2})
-- ({0.61975*#1},{0.82013*#2})
-- ({0.61519*#1},{0.82725*#2})
-- ({0.61061*#1},{0.83443*#2})
-- ({0.60602*#1},{0.84167*#2})
-- ({0.60141*#1},{0.84896*#2})
-- ({0.59677*#1},{0.85631*#2})
-- ({0.59209*#1},{0.86369*#2})
-- ({0.58736*#1},{0.87112*#2})
-- ({0.58259*#1},{0.87856*#2})
-- ({0.57776*#1},{0.88603*#2})
-- ({0.57287*#1},{0.89351*#2})
-- ({0.56790*#1},{0.90099*#2})
-- ({0.56287*#1},{0.90846*#2})
-- ({0.55775*#1},{0.91592*#2})
-- ({0.55255*#1},{0.92335*#2})
-- ({0.54726*#1},{0.93074*#2})
-- ({0.54187*#1},{0.93809*#2})
-- ({0.53639*#1},{0.94539*#2})
-- ({0.53079*#1},{0.95261*#2})
-- ({0.52509*#1},{0.95976*#2})
-- ({0.51927*#1},{0.96682*#2})
-- ({0.51334*#1},{0.97378*#2})
-- ({0.50728*#1},{0.98063*#2})
-- ({0.50110*#1},{0.98736*#2})
-- ({0.49479*#1},{0.99396*#2})
-- ({0.48836*#1},{1.00041*#2})
-- ({0.48179*#1},{1.00671*#2})
-- ({0.47509*#1},{1.01284*#2})
-- ({0.46826*#1},{1.01879*#2})
-- ({0.46130*#1},{1.02455*#2})
-- ({0.45420*#1},{1.03011*#2})
-- ({0.44696*#1},{1.03547*#2})
-- ({0.43959*#1},{1.04060*#2})
-- ({0.43209*#1},{1.04550*#2})
-- ({0.42446*#1},{1.05016*#2})
-- ({0.41669*#1},{1.05457*#2})
-- ({0.40880*#1},{1.05872*#2})
-- ({0.40079*#1},{1.06260*#2})
-- ({0.39265*#1},{1.06621*#2})
-- ({0.38439*#1},{1.06953*#2})
-- ({0.37602*#1},{1.07256*#2})
-- ({0.36754*#1},{1.07530*#2})
-- ({0.35895*#1},{1.07773*#2})
-- ({0.35026*#1},{1.07986*#2})
-- ({0.34147*#1},{1.08167*#2})
-- ({0.33260*#1},{1.08317*#2})
-- ({0.32364*#1},{1.08435*#2})
-- ({0.31460*#1},{1.08522*#2})
-- ({0.30549*#1},{1.08576*#2})
-- ({0.29632*#1},{1.08598*#2})
-- ({0.28709*#1},{1.08589*#2})
-- ({0.27781*#1},{1.08548*#2})
-- ({0.26848*#1},{1.08476*#2})
-- ({0.25912*#1},{1.08373*#2})
-- ({0.24974*#1},{1.08240*#2})
-- ({0.24033*#1},{1.08078*#2})
-- ({0.23092*#1},{1.07888*#2})
-- ({0.22149*#1},{1.07671*#2})
-- ({0.21207*#1},{1.07427*#2})
-- ({0.20267*#1},{1.07158*#2})
-- ({0.19328*#1},{1.06867*#2})
-- ({0.18391*#1},{1.06554*#2})
-- ({0.17458*#1},{1.06221*#2})
-- ({0.16529*#1},{1.05870*#2})
-- ({0.15604*#1},{1.05504*#2})
-- ({0.14684*#1},{1.05125*#2})
-- ({0.13770*#1},{1.04736*#2})
-- ({0.12862*#1},{1.04339*#2})
-- ({0.11960*#1},{1.03937*#2})
-- ({0.11065*#1},{1.03534*#2})
-- ({0.10177*#1},{1.03132*#2})
-- ({0.09296*#1},{1.02736*#2})
-- ({0.08422*#1},{1.02349*#2})
-- ({0.07556*#1},{1.01976*#2})
-- ({0.06696*#1},{1.01620*#2})
-- ({0.05843*#1},{1.01287*#2})
-- ({0.04996*#1},{1.00980*#2})
-- ({0.04155*#1},{1.00705*#2})
-- ({0.03318*#1},{1.00467*#2})
-- ({0.02486*#1},{1.00272*#2})
-- ({0.01656*#1},{1.00125*#2})
-- ({0.00828*#1},{1.00032*#2})
-- ({0.00000*#1},{1.00000*#2})
--cycle;
\draw[line width=1.4pt,color=red] ({1.00000*#1},{0.00000*#2})
-- ({1.00032*#1},{0.00828*#2})
-- ({1.00125*#1},{0.01656*#2})
-- ({1.00272*#1},{0.02486*#2})
-- ({1.00467*#1},{0.03318*#2})
-- ({1.00705*#1},{0.04155*#2})
-- ({1.00980*#1},{0.04996*#2})
-- ({1.01287*#1},{0.05843*#2})
-- ({1.01620*#1},{0.06696*#2})
-- ({1.01976*#1},{0.07556*#2})
-- ({1.02349*#1},{0.08422*#2})
-- ({1.02736*#1},{0.09296*#2})
-- ({1.03132*#1},{0.10177*#2})
-- ({1.03534*#1},{0.11065*#2})
-- ({1.03937*#1},{0.11960*#2})
-- ({1.04339*#1},{0.12862*#2})
-- ({1.04736*#1},{0.13770*#2})
-- ({1.05125*#1},{0.14684*#2})
-- ({1.05504*#1},{0.15604*#2})
-- ({1.05870*#1},{0.16529*#2})
-- ({1.06221*#1},{0.17458*#2})
-- ({1.06554*#1},{0.18391*#2})
-- ({1.06867*#1},{0.19328*#2})
-- ({1.07158*#1},{0.20267*#2})
-- ({1.07427*#1},{0.21207*#2})
-- ({1.07671*#1},{0.22149*#2})
-- ({1.07888*#1},{0.23092*#2})
-- ({1.08078*#1},{0.24033*#2})
-- ({1.08240*#1},{0.24974*#2})
-- ({1.08373*#1},{0.25912*#2})
-- ({1.08476*#1},{0.26848*#2})
-- ({1.08548*#1},{0.27781*#2})
-- ({1.08589*#1},{0.28709*#2})
-- ({1.08598*#1},{0.29632*#2})
-- ({1.08576*#1},{0.30549*#2})
-- ({1.08522*#1},{0.31460*#2})
-- ({1.08435*#1},{0.32364*#2})
-- ({1.08317*#1},{0.33260*#2})
-- ({1.08167*#1},{0.34147*#2})
-- ({1.07986*#1},{0.35026*#2})
-- ({1.07773*#1},{0.35895*#2})
-- ({1.07530*#1},{0.36754*#2})
-- ({1.07256*#1},{0.37602*#2})
-- ({1.06953*#1},{0.38439*#2})
-- ({1.06621*#1},{0.39265*#2})
-- ({1.06260*#1},{0.40079*#2})
-- ({1.05872*#1},{0.40880*#2})
-- ({1.05457*#1},{0.41669*#2})
-- ({1.05016*#1},{0.42446*#2})
-- ({1.04550*#1},{0.43209*#2})
-- ({1.04060*#1},{0.43959*#2})
-- ({1.03547*#1},{0.44696*#2})
-- ({1.03011*#1},{0.45420*#2})
-- ({1.02455*#1},{0.46130*#2})
-- ({1.01879*#1},{0.46826*#2})
-- ({1.01284*#1},{0.47509*#2})
-- ({1.00671*#1},{0.48179*#2})
-- ({1.00041*#1},{0.48836*#2})
-- ({0.99396*#1},{0.49479*#2})
-- ({0.98736*#1},{0.50110*#2})
-- ({0.98063*#1},{0.50728*#2})
-- ({0.97378*#1},{0.51334*#2})
-- ({0.96682*#1},{0.51927*#2})
-- ({0.95976*#1},{0.52509*#2})
-- ({0.95261*#1},{0.53079*#2})
-- ({0.94539*#1},{0.53639*#2})
-- ({0.93809*#1},{0.54187*#2})
-- ({0.93074*#1},{0.54726*#2})
-- ({0.92335*#1},{0.55255*#2})
-- ({0.91592*#1},{0.55775*#2})
-- ({0.90846*#1},{0.56287*#2})
-- ({0.90099*#1},{0.56790*#2})
-- ({0.89351*#1},{0.57287*#2})
-- ({0.88603*#1},{0.57776*#2})
-- ({0.87856*#1},{0.58259*#2})
-- ({0.87112*#1},{0.58736*#2})
-- ({0.86369*#1},{0.59209*#2})
-- ({0.85631*#1},{0.59677*#2})
-- ({0.84896*#1},{0.60141*#2})
-- ({0.84167*#1},{0.60602*#2})
-- ({0.83443*#1},{0.61061*#2})
-- ({0.82725*#1},{0.61519*#2})
-- ({0.82013*#1},{0.61975*#2})
-- ({0.81309*#1},{0.62431*#2})
-- ({0.80612*#1},{0.62887*#2})
-- ({0.79924*#1},{0.63344*#2})
-- ({0.79243*#1},{0.63803*#2})
-- ({0.78572*#1},{0.64264*#2})
-- ({0.77910*#1},{0.64727*#2})
-- ({0.77256*#1},{0.65195*#2})
-- ({0.76613*#1},{0.65666*#2})
-- ({0.75979*#1},{0.66141*#2})
-- ({0.75355*#1},{0.66622*#2})
-- ({0.74740*#1},{0.67109*#2})
-- ({0.74136*#1},{0.67601*#2})
-- ({0.73541*#1},{0.68100*#2})
-- ({0.72956*#1},{0.68607*#2})
-- ({0.72380*#1},{0.69121*#2})
-- ({0.71815*#1},{0.69642*#2})
-- ({0.71258*#1},{0.70172*#2})
-- ({0.70711*#1},{0.70711*#2})
-- ({0.70172*#1},{0.71258*#2})
-- ({0.69642*#1},{0.71815*#2})
-- ({0.69121*#1},{0.72380*#2})
-- ({0.68607*#1},{0.72956*#2})
-- ({0.68100*#1},{0.73541*#2})
-- ({0.67601*#1},{0.74136*#2})
-- ({0.67109*#1},{0.74740*#2})
-- ({0.66622*#1},{0.75355*#2})
-- ({0.66141*#1},{0.75979*#2})
-- ({0.65666*#1},{0.76613*#2})
-- ({0.65195*#1},{0.77256*#2})
-- ({0.64727*#1},{0.77910*#2})
-- ({0.64264*#1},{0.78572*#2})
-- ({0.63803*#1},{0.79243*#2})
-- ({0.63344*#1},{0.79924*#2})
-- ({0.62887*#1},{0.80612*#2})
-- ({0.62431*#1},{0.81309*#2})
-- ({0.61975*#1},{0.82013*#2})
-- ({0.61519*#1},{0.82725*#2})
-- ({0.61061*#1},{0.83443*#2})
-- ({0.60602*#1},{0.84167*#2})
-- ({0.60141*#1},{0.84896*#2})
-- ({0.59677*#1},{0.85631*#2})
-- ({0.59209*#1},{0.86369*#2})
-- ({0.58736*#1},{0.87112*#2})
-- ({0.58259*#1},{0.87856*#2})
-- ({0.57776*#1},{0.88603*#2})
-- ({0.57287*#1},{0.89351*#2})
-- ({0.56790*#1},{0.90099*#2})
-- ({0.56287*#1},{0.90846*#2})
-- ({0.55775*#1},{0.91592*#2})
-- ({0.55255*#1},{0.92335*#2})
-- ({0.54726*#1},{0.93074*#2})
-- ({0.54187*#1},{0.93809*#2})
-- ({0.53639*#1},{0.94539*#2})
-- ({0.53079*#1},{0.95261*#2})
-- ({0.52509*#1},{0.95976*#2})
-- ({0.51927*#1},{0.96682*#2})
-- ({0.51334*#1},{0.97378*#2})
-- ({0.50728*#1},{0.98063*#2})
-- ({0.50110*#1},{0.98736*#2})
-- ({0.49479*#1},{0.99396*#2})
-- ({0.48836*#1},{1.00041*#2})
-- ({0.48179*#1},{1.00671*#2})
-- ({0.47509*#1},{1.01284*#2})
-- ({0.46826*#1},{1.01879*#2})
-- ({0.46130*#1},{1.02455*#2})
-- ({0.45420*#1},{1.03011*#2})
-- ({0.44696*#1},{1.03547*#2})
-- ({0.43959*#1},{1.04060*#2})
-- ({0.43209*#1},{1.04550*#2})
-- ({0.42446*#1},{1.05016*#2})
-- ({0.41669*#1},{1.05457*#2})
-- ({0.40880*#1},{1.05872*#2})
-- ({0.40079*#1},{1.06260*#2})
-- ({0.39265*#1},{1.06621*#2})
-- ({0.38439*#1},{1.06953*#2})
-- ({0.37602*#1},{1.07256*#2})
-- ({0.36754*#1},{1.07530*#2})
-- ({0.35895*#1},{1.07773*#2})
-- ({0.35026*#1},{1.07986*#2})
-- ({0.34147*#1},{1.08167*#2})
-- ({0.33260*#1},{1.08317*#2})
-- ({0.32364*#1},{1.08435*#2})
-- ({0.31460*#1},{1.08522*#2})
-- ({0.30549*#1},{1.08576*#2})
-- ({0.29632*#1},{1.08598*#2})
-- ({0.28709*#1},{1.08589*#2})
-- ({0.27781*#1},{1.08548*#2})
-- ({0.26848*#1},{1.08476*#2})
-- ({0.25912*#1},{1.08373*#2})
-- ({0.24974*#1},{1.08240*#2})
-- ({0.24033*#1},{1.08078*#2})
-- ({0.23092*#1},{1.07888*#2})
-- ({0.22149*#1},{1.07671*#2})
-- ({0.21207*#1},{1.07427*#2})
-- ({0.20267*#1},{1.07158*#2})
-- ({0.19328*#1},{1.06867*#2})
-- ({0.18391*#1},{1.06554*#2})
-- ({0.17458*#1},{1.06221*#2})
-- ({0.16529*#1},{1.05870*#2})
-- ({0.15604*#1},{1.05504*#2})
-- ({0.14684*#1},{1.05125*#2})
-- ({0.13770*#1},{1.04736*#2})
-- ({0.12862*#1},{1.04339*#2})
-- ({0.11960*#1},{1.03937*#2})
-- ({0.11065*#1},{1.03534*#2})
-- ({0.10177*#1},{1.03132*#2})
-- ({0.09296*#1},{1.02736*#2})
-- ({0.08422*#1},{1.02349*#2})
-- ({0.07556*#1},{1.01976*#2})
-- ({0.06696*#1},{1.01620*#2})
-- ({0.05843*#1},{1.01287*#2})
-- ({0.04996*#1},{1.00980*#2})
-- ({0.04155*#1},{1.00705*#2})
-- ({0.03318*#1},{1.00467*#2})
-- ({0.02486*#1},{1.00272*#2})
-- ({0.01656*#1},{1.00125*#2})
-- ({0.00828*#1},{1.00032*#2})
-- ({0.00000*#1},{1.00000*#2});
}


\def\s{15}

\begin{scope}[xshift=-6.5cm]
\fehlerpfad{5}{\s}
\draw[->] (-0.1,0)--(5.4,0) coordinate[label={$t$}];
\draw[->] (0,-0.1)--(0,5.4) coordinate[label={right:$|\vec{b}(t)|-1$}];
\draw (5,-0.1) -- (5,0.1);
\node at (5,-0.1) [below] {$1$};
\node at (0,-0.1) [below] {$0$};
\draw (-0.1,{\s*0.1})--(0.1,{\s*0.1});
\node at (-0.1,{\s*0.1}) [left] {$0.0001$};
\draw (-0.1,{\s*0.2})--(0.1,{\s*0.2});
\node at (-0.1,{\s*0.2}) [left] {$0.0002$};
\draw (-0.1,{\s*0.3})--(0.1,{\s*0.3});
\node at (-0.1,{\s*0.3}) [left] {$0.0003$};
\end{scope}

\begin{scope}[xshift=0cm]
\kurvepfad{5}{5}
\draw[->] (-0.1,0)--(5.4,0) coordinate[label={$x$}];
\draw[->] (0,-0.1)--(0,5.4) coordinate[label={right:$y$}];
\draw (5,-0.1)--(5,0.1);
\node at (5,-0.1) [below] {$1$};
\draw (-0.1,5)--(0.1,5);
\node at (-0.1,5) [left] {$1$};
\end{scope}

\end{tikzpicture}
\end{document}

