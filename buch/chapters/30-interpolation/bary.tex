%
% bary.tex
%
% (c) 2020 Prof Dr Andreas Müller, Hochschule Rapperswil
%
\section{Baryzentrische Formeln für Interpolationspolynome
\label{buch:section:baryzentrisch}}
\index{baryzentrische Formel}%
Die Interpolationspolynome von Lagrange und Hermite haben 
in der bis jetzt gezeigten Form das folgende grundlegende Problem.
Sie sind definiert über das Produkt
\[
l(x)
=
(x-x_0)(x-x_1)\dots (x-x_{n-1})(x-x_n).
\]
Ist die Zahl der Stützstellen gross und erstrecken sich die 
Stützstellen über einen grossen Bereich, dann sind einzelne Faktoren
$(x-x_i)$ immer gross, wie wir bei der Diskussion von Runges Phänomen
bereits diskutiert haben.
\index{Runges Phänomen}%
Zudem tritt bei der Berechnung eines Wertes in inmittelbarer 
Nähe der Stützstelle $x_j$ in dem Faktor $(x-x_j)$ Auslöschung auf.
Der grosse relative Fehler dieses Faktors wird durch die anderen Faktoren
zu einem grossen absoluten Fehler aufgeblasen.

Andererseits ist klar, dass sich das Interpolationspolynom vor allem
in der Nähe einer Stützstelle ändern sollte, wenn man den Wert an der
Stützstelle ändert.
Die anderen Stützstellen sollten also nur einen geringen Einfluss auf
den Wert des Interpolationspolynoms haben.
Dies geht aus der bisherigen Form des Interpolationspolynoms ebenfalls
nicht hervor.

Gesucht ist also eine Form des Interpolationspolynoms, welche einsichtig
macht, dass Änderungen von Stützwerten sich vor allem in der nähe der
betroffenen Stützstelle auswirken und die auch bei einer grossen Zahl
von Stützstellen stabil sind.

Früher wurde gezeigt, dass das Interpolationspolynom für Funktionswerte
$f_j$ an den Stützstellen $x_j$ durch die Linearkombination
\[
p(x) = \sum_{j=0}^n f_j l_j(x)
\]
gegeben ist.
Für die Polynome $l_j(x)$ wurde
\[
l_j(x)
=
\frac{
(x-x_0)(x-x_1)\cdots(\widehat{x-x_j}) \cdots (x-x_n)
}{
(x_j-x_0)(x_j-x_1)\cdots (\widehat{x_j-x_j})\cdots (x_j-x_n)
}
\]
gefunden.
Schreibt man
\[
w_j
=
\frac{1}{
\displaystyle\prod_{\scriptstyle k=1\atop \scriptstyle k\ne j}^n (x_j-x_k)
},
\]
dann kann man die Faktoren $l_j(x)$ auch als
\[
l_j(x)
=
\frac{l(x)}{(x-x_j)}\cdot w_j
\]
ausdrücken.
Damit wird das Interpolationspolynom jetzt
\begin{equation}
p(x)
=
l(x) \sum_{j=0}^n \frac{w_jf_j}{x-x_j}.
\label{buch:bary:px}
\end{equation}
Die Zahlen $w_j$ hängen nur von den Stützstellen ab, nicht von den
Funktionswerten $f_j$. 
Sie können also nach Festlegung der Stützstellen einmalig berechnet
werden und verursachen danach keinen weiteren Berechnungsaufwand.

Das Interpolationspolynom wird besonders einfach, wenn alle Funktionswerte
$f_j=1$ sind.
Da das konstante Polynom $p(x)=1$ genau diese Werte annimmt, muss
\[
1 = l(x) \sum_{j=0}^n \frac{w_j}{x-x_j}
\]
gelten.
Damit erhalten wir eine neue Darstellung für 
\begin{equation}
l(x)
=
\frac{1}{\displaystyle\sum_{j=0}^n \frac{w_j}{x-x_j}}.
\label{buch:bary:lx}
\end{equation}
In dieser Form wird vermieden, dass zur Berechnung von $l(x)$ eine
grosse Anzahl Produkte mit potentiell grossen Faktoren gebildet werden
muss.
Sorgen bereiten in der Produktdarstellung vor allem die Faktoren
$x-x_j$ für $x$ weit entfernt von $x_j$.
Stattdessen wird in \eqref{buch:bary:lx} eine Summe von Summanden gebildet,
die klein sind,
wenn $x$ weit von $x_j$ entfernt ist.

Die vorteilhafte Formulierung~\eqref{buch:bary:lx} kann nun dazu
verwendet werden, auch eine verbesserte Formulierung für das
Interpolationspolynom aufzustellen.
Dazu ersetzen wir den Faktor $l(x)$ in \eqref{buch:bary:px}
durch \eqref{buch:bary:lx} und erhalten
\begin{equation}
p(x)
=
\frac{\displaystyle \sum_{j=0}^n \frac{w_jf_j}{x-x_j}
}{
\displaystyle\sum_{j=0}^n \frac{w_j}{x-x_j}}.
\label{buch:bary:pfinal}
\end{equation}
Diese Form des Interpolationspolynoms ist ein gewichtetes Mittel 
der Werte $f_j$, gewichtet mit den Gewichten $w_j/(x-x_j)$.
\index{Mittel!gewichtet}%
Diese Gewichte sind klein für $x$ weit weg von $x_j$, die grössten
Gewichte haben die Funktionswerte $f_j$ nahe bei $x$.
\index{Gewicht}%
\index{$w_j$}%
Die Formel~\eqref{buch:bary:pfinal} ist daher eine numerisch besonders
vorteilhafte Form der Auswertung eines Interpolationspolynoms.






