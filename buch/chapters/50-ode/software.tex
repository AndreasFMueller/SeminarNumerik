%
% software.tex -- 
%
% (c) 2015 Prof Dr Andreas Mueller, Hochschule Rapperswil
%
\section{Software}
\rhead{Software}
Die im letzten Abschnitt entwickelten numerischen Verfahren zur Lösung
einer Differentialgleichung kommen ausschliesslich mit Auswertungen der
Funktion $f$ aus, die Ableitungen der Funktion $f$ müssen nicht bekannt
sein.
Es sollte also ein Leichtes sein, eine Softwarebibliothek zur
Verfügung zu stellen, mit der eine beliebige gewöhnliche
Differentialgleichung gelöst werden kann.
Als Input braucht es nur die Funktion $f$ und die Anfangsbedingungen.
\index{Softwarebibliothek}%

Als Beispiel wollen wir in diesem Abschnitt die Differentialgleichung
\[
y''+y=\sin \frac{x}{10},\qquad y(0)=y'(0)=0
\]
in verschiedenen Programmierumgebungen lösen.
Als erstes bringen wir die Differentialgleichung wieder in die Standardform
einer Vektordifferentialgleichung erste Ordnung:
\index{Vektordifferentialgleichung}%
\begin{equation}
Y=\begin{pmatrix} y_1\\y_2\end{pmatrix}
\qquad\Rightarrow\qquad
\frac{d}{dt}Y
=
\frac{d}{dx}\begin{pmatrix}y_1\\y_2\end{pmatrix}
=
\begin{pmatrix}
y_2\\
\displaystyle -y_1+\sin\frac{x}{10}
\end{pmatrix}
=
f(x,Y).
\label{buch:ode:beispieldgl}
\end{equation}
Ein numerisches Verfahren braucht also als Input eine Anfangsbedingung
sowie die Funktion $f$.
\index{Anfangsbedingung}%
Ausserdem muss es Möglichkeiten bereitstellen, wie man den Gang der
Rechnung beeinflussen kann, z.~B.~um die $x$-Werte anzugeben, für die
die $Y(x)$ ausgegeben werden sollen, oder um Genauigkeitsziele vorzugeben.

\subsection{Octave}
\index{Octave}%
In Octave steht eine einzige Funktion \texttt{lsode} zur Verfügung, welche
\index{lsode@{\texttt{lsode}}}
auf zuverlässige Art Differentialgleichungen löst.
Der Anwender muss eine Implementation der Funktion $f$ zur Verfügung
stellen, allerdings werden die Argument in der umgekehrten Reihenfolge
zu der erwarte, die wir in diesem Skript bisher verwendet haben.
Für die Beispieldifferentialgleichung \eqref{buch:ode:beispieldgl}
kann man sie zum Beispiel so definieren:
\lstinputlisting[style=Octave]{chapters/50-ode/examples/octave-dgl-f.m}

Beim Aufruf der Funktion \texttt{lsode} muss man den {\em Namen}
der Funktion, die Anfangsbedingung, sowie einen Vektoren mit $x$-Werten,
für die man die Lösung ausgegeben haben möchte, als Argumente
übergeben.
Der erste Wert im $x$-Vektor muss der $x$-Wert für die Anfangsbedingung
sein, in unserem Fall also $0$.
Um die Werte von $y(x)$ für $x$-Werte zu erhalten, die ganzzahlige Vielfache
$x_k=k\Delta x$ von $\Delta x$ sind, $1\le k \le 10$, muss man also die Befehle
\lstinputlisting[style=Octave]{chapters/50-ode/examples/octave-dgl-sol.m}
ausführen.
Als Rückgabewert erhält man eine Matrix, die in jeder Zeile die
Werte von $y(x)$ und $y'(x)$ zum entsprechenden Wert von $x$
aus dem \texttt{x}-Argument enthält.
Die Resultate sind zusammen mit den Werten der exakten
Lösung~\eqref{buch:ode:beispiel-loesung} 
in der Tabelle~\ref{buch:ode:octave-resultate} zusammengestellt.
Es ist gut erkennbar, wie der Fehler anfänglich langsam ansteigt,
dann aber unter Kontrolle bleibt.
Die Dokumentation der Funktion \texttt{lsode} beschreibt, wie man mit
Hilfe von Optionen ihr Verhalten und insbesondere die Grösse der
Fehler weiter beeinflussen kann.
\begin{table}
\centering
\begin{tabular}{|>{$}r<{$}|>{$}r<{$}|>{$}r<{$}|>{$}r<{$}|}
\hline
    x&  y_{\text{numerisch}}(x)&y_{\text{exakt}}(x) & \text{Fehler}\\
\hline
    0&  0.00000000&  0.00000000&  0.00000000\\
    1&  0.00158525&  0.00158528&  0.00000003\\
    2&  0.01090682&  0.01090678&  0.00000003\\
    3&  0.02858723&  0.02858716&  0.00000007\\
    4&  0.04756207&  0.04756212&  0.00000004\\
    5&  0.05957416&  0.05957437&  0.00000020\\
    6&  0.06276426&  0.06276444&  0.00000018\\
    7&  0.06337942&  0.06337932&  0.00000010\\
    8&  0.07002849&  0.07002811&  0.00000037\\
    9&  0.08576626&  0.08576594&  0.00000032\\
   10&  0.10528405&  0.10528416&  0.00000010\\
  100&  0.84661503&  0.84661930&  0.00000427\\
 1000& -0.55228836& -0.55234514&  0.00005678\\
 2000&  0.90392063&  0.90373523&  0.00018540\\
 3000& -0.99018339& -0.99032256&  0.00013917\\
 4000&  0.75185982&  0.75202340&  0.00016358\\
 5000& -0.25298074& -0.25252044&  0.00046030\\
 6000& -0.30093757& -0.30056348&  0.00037408\\
 7000&  0.76905079&  0.76889872&  0.00015207\\
 8000& -1.00327790& -1.00396748&  0.00068958\\
 9000&  0.88860437&  0.88793099&  0.00067338\\
10000& -0.50337856& -0.50335983&  0.00001873\\
\hline
\end{tabular}
\caption{Exakte und numerische Lösung der Beispieldifferentialgleichung
berechnet mit der Funktion \texttt{lsode} von Octave.
\label{buch:ode:octave-resultate}}
\end{table}

Die Schrittlänge muss in einem numerischen Verfahren nicht immer gleich
gross gewählt werden.
Falls die Ableitungen der gesuchte Funktion nur langsam ändern, können
bei gleicher Genauigkeit grössere Schritte verwendet werden, um
Rechnzeit einzusparen.
Die Funktion $\texttt{lsode}$ verwendet eine solche Schrittlängensteuerung.
Mehr Information dazu findet man im Kapitel~\ref{chapter:steps}.

\subsection{GNU Scientific Library}
\index{GNU scientific library}%
\index{GSL}%
Während Octave dem Benutzer die Wahl eines geeigneten Verfahrens abnimmt
und ihm überhaupt wenig Kontrolle über den Gang der Rechnung gibt,
kann ein C-Programmierer durch den Einsatz der GNU Scientific Library (GSL) die
\index{C}%
volle Kontrolle über alle Aspekte der Iteration erhalten.
Der Preis ist eine wesentlich höhere Komplexität.
Ziel dieses Abschnitts ist, ein einfaches Beispielprogramm zu
zeigen, welches als Basis eigener Programme dienen kann.
Es verwendet eine Runge-Kutta-Verfahren achter Ordnung.

Die Funktionen zum Lösen von gewöhnlichen Differentialgleichungen
der GSL haben alle das Präfix \texttt{gsl\_odeiv2\_}. 
Zunächst braucht es natürlich wieder eine Implementation der
Funktion $f$. 
Die GSL übergibt zwei Arrays, im einen findet die Funktion die aktuellen
$Y$-Werte, im anderen soll sie die Werte der Ableitung zurückgeben.
Für die Beispiel-Differentialgleichung \eqref{buch:ode:beispieldgl}
sieht der Code wie folgt aus:
\lstinputlisting[style=C]{chapters/50-ode/examples/dgl-f.c}
Der Parameter \texttt{params} dient dazu, der Funktion zusätzliche
Parameter zu übergeben.
In unserem Fall ist das nur die Zahl $\omega$.
Da \texttt{params} ein \texttt{void}-Pointer ist, kann eine beliebige
Struktur zur Parameterübergabe verwendet werden.

Die Differentialgleichung wird beschrieben durch eine Struktur vom Typ
\texttt{gsl\_odeiv2\_system}, welche ausser Zeigern auf die Funktion
und die Parameter-Struktur auch noch die Dimension der Vektoren enthält.
Es kann auch noch ein Funktionszeiger für eine Funktion übergeben
werden, die die Jacobi-Matrix berechnet, in unserem Beispiel wird dies
jedoch nicht benötigt.

Die eigentliche Berechnung wird von einer ``driver''-Funktion durchgeführt.
Diese sorgt im wesentlichen für die Wahl der Schrittweite, verwaltet
Datenstrukturen und ruft die Funktionen auf, die die einzelnen Schritte
durchführen.
Die Treiber-Funktion führt die einzelnen Schritte (im Sinne der
in Abschnitt~\ref{buch:section:ode:einschritt} besprochenen
Einschritt-Verfahren) mit
Hilfe der Schritt-Funktionen durch, von denen die Bibliothek eine
ganze Reihe bereitstellt.
Die Funktion \texttt{gsl\_odeiv2\_step\_rk4} ist das klassische
Runge-Kutta-Verfahren vierter Ordnung, welches in
Abschnitt~\ref{subsection:buch:ode:runge-kutta}
beschrieben wurde.
\index{Runge-Kutta-Verfahren}%
Im Beispielverfahren verwenden wir \texttt{gsl\_odeiv2\_step\_rk8pd},
das Runge-Kutta-Prince-Dormand-Verfahren achter Ordnung.
\index{Runge-Kutta-Prince-Dormand-Verfahren}
Für Aufgaben allgemeiner Art ebenfalls sehr gut geeignet ist das
Runge-Kutta-Fehlberg-Verfahren fünfter Ordnung mit dem Namen
\texttt{gsl\_odeiv2\_step\_rkf45}.
\index{Runge-Kutta-Fehlberg-Verfahren}%
Diese Datenstrukturen werden mit dem Code
\lstinputlisting[style=C]{chapters/50-ode/examples/dgl-init.c}
initialisiert.
Durch Austausch des zweiten Arguments der Driver-Allozierungs-Funktion
kann man leicht das Verfahren wechseln und so Zeitaufwand und Genauigkeit
für verschiedene Lösungsverfahren vergleichen.
\index{Zeitaufwand}%

Um die Rechnung durchzuführen, muss jetzt die Driver-Funktion so oft
angewendet werden, wie man Punkt der Lösungskurve ausgeben will.
Dazu dient die Funktion \texttt{gsl\_odeiv2\_driver\_apply}. 
An Argumenten braucht sie den eben initialisierten Driver, den aktuellen
$x$-Wert, den $x_{\text{next}}$-Wert, für den der nächste Punkt
ausgegeben werden soll, sowie einen Vektor, in dem der aktuelle Anfangswert
für $Y(x)$ übergeben und $Y(x_{\text{next}})$ zurückgegeben wird.
$x$ wird als Referenz übergeben, wenn die Funktion zurückkehrt,
findet man dort den neuen aktuellen Wert von $x$, also im Erfolgsfall
$x_{\text{next}}$.
In unserem Fall brauchen wir $X(x)$ für ganzzahlige $x$, die folgende
Schleife bewerkstelligt dies:
\lstinputlisting[style=C]{chapters/50-ode/examples/dgl-loop.c}

\begin{table}
\centering
\begin{tabular}{|>{$}r<{$}|>{$}r<{$}|>{$}r<{$}|>{$}r<{$}|}
\hline
    x&  y_{\text{numerisch}}(x)&y_{\text{exakt}}(x) & \text{Fehler}\\
\hline
    1&   0.01584477&   0.01584477&  -0.00000000\\
    2&   0.10882786&   0.10882787&  -0.00000000\\
    3&   0.28425071&   0.28425071&  -0.00000001\\
    4&   0.46979656&   0.46979656&  -0.00000000\\
    5&   0.58112927&   0.58112926&   0.00000001\\
    6&   0.59856974&   0.59856972&   0.00000002\\
    7&   0.58436266&   0.58436265&   0.00000001\\
    8&   0.62466692&   0.62466694&  -0.00000001\\
    9&   0.74961115&   0.74961117&  -0.00000003\\
   10&   0.90492231&   0.90492232&  -0.00000001\\
   20&   0.82626555&   0.82626556&  -0.00000000\\
   30&   0.24234669&   0.24234664&   0.00000005\\
   40&  -0.83971103&  -0.83971092&  -0.00000011\\
   50&  -0.94210772&  -0.94210787&   0.00000015\\
   60&  -0.25144906&  -0.25144893&  -0.00000013\\
   70&   0.58545210&   0.58545205&   0.00000005\\
   80&   1.09974465&   1.09974456&   0.00000009\\
   90&   0.32597838&   0.32597860&  -0.00000023\\
  100&  -0.49836792&  -0.49836823&   0.00000031\\
  200&   1.01037929&   1.01037877&   0.00000052\\
  300&  -0.89702589&  -0.89702630&   0.00000041\\
  400&   0.83859090&   0.83859101&  -0.00000010\\
  500&  -0.21777633&  -0.21777543&  -0.00000090\\
  600&  -0.31235410&  -0.31235239&  -0.00000171\\
  700&   0.72675906&   0.72676124&  -0.00000219\\
  800&  -1.09422992&  -1.09422790&  -0.00000202\\
  900&   0.80223761&   0.80223872&  -0.00000111\\
 1000&  -0.59500324&  -0.59500363&   0.00000040\\
 2000&  -0.97606658&  -0.97606187&  -0.00000471\\
 3000&  -1.03200392&  -1.03199479&  -0.00000914\\
 4000&  -0.79047800&  -0.79047372&  -0.00000428\\
 5000&  -0.37269297&  -0.37270218&   0.00000921\\
 6000&   0.08785158&   0.08783273&   0.00001886\\
 7000&   0.49827647&   0.49826463&   0.00001184\\
 8000&   0.80219750&   0.80220742&  -0.00000992\\
 9000&   0.94568799&   0.94571577&  -0.00002778\\
10000&   0.86607968&   0.86610200&  -0.00002232\\
\hline
\end{tabular}
\caption{Lösungen der Beispieldifferentialgleichung \eqref{buch:ode:beispieldgl}
mit Hilfe der GNU Scientific Library (GSL).
\label{buch:ode:gsl-resultate}}
\end{table}

Man kann die Funktion $f$ im Programm natürlich auch mit einem Zähler
ausstatten und damit herausfinden, wie viele Aufrufe der Funktion
für die numerische Lösung benötigt werden.
\index{Zähler}%
\index{Programm}%
Es stellt sich heraus, dass für das erste Intervall von $0$ bis $1$
die Funktion $f$ 131 mal aufgerufen wird, hier versucht die Bibliothek
die optimale Schrittweite $h$ zu bestimmen.
\index{Schrittweite}%
In allen folgenden Intervallen der Länge $1$ von $n$ bis $n+1$ werden nur
noch jeweils 13 Aufrufe der Funktion benötigt.
Verwendet man stattdessen das Runge-Kutta-Fehlberg-Verfahren,
werden pro Intervall 18 Auswertungen der Funktion $f$ benötigt,
und die Genauigkeit sinkt auf zwei Stellen nach dem Komma.
\index{Runge-Kutta-Fehlberg-Verfahren}%

