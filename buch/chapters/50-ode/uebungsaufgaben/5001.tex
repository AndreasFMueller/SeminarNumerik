In dieser Aufgabe soll ein implizites Verfahren für die Lösung der
Differentialgleichung
\begin{equation}
y'=f(x,y)
\label{5001:ode}
\end{equation}
gefunden werden.
Gesucht werden also die Funktionswerte $y_0,y_1,\dots,y_n$ zu den
$x$-Werten $x_0,x_1,\dots,x_n$, wobei wir typischerweise $x_k = x_0+kh$ 
wählen.
\begin{teilaufgaben}
\item
Leiten Sie aus der Differentialgleichung \eqref{5001:ode}
eine Näherungsgleichung für benachbarte Punkte $(x_k,y_k)$ und
$(x_{k+1},y_{k+1})$ ab, die beide Steigungen in den Punkten aber keine
weiteren Punkte verwendt.
\item
Wenden Sie dieses Verfahren auf die Gleichung $y'=f(x,y)=y$ an, und
drücken Sie $y_{k+1}$ durch $x_k$, $x_{k+1}=x_k+h$ und $y_{k}$ aus.
\item
Was erhält man, wenn man für die Differentialgleichung $y'=y$
als Schrittweite $1/n$ wählt und $n$ Schritte ausführt.
Wie hängt $y_n$ von $y_0$ ab?
Berechnen Sie den Grenzwert $n\to\infty$.
\item
Versuchen sie experimentell die Ordnung des Verfahrens zu bestimmen,
indem Sie die die Gleichung $y'=y'$ damit lösen und das Verhalten bei
Halbierung der Schrittweise untersuchen.
\end{teilaufgaben}

\begin{loesung}
\begin{teilaufgaben}
\item
Die Steigung zwischen den zwei Punkten $(x_k,y_k)$ und $(x_{k+1},y_{k+1})$
kann mit den beiden Steigungen $y_k' = f(x_k,y_k)$ und
$y_{k+1}'=f(x_{k+1},y_{k+1})$ verglichen werden.
Da keine dieser Steigungen wirklich für das ganze
Intervall $[x_k,x_{k+1}]$
repräsentativ sein kann, nehmen wir den Mittelwert, also
\begin{equation}
\frac{y_{k+1}-y_{k}}{x_{k+1}-x_{k}}
=
\frac12\bigl(f(x_k,y_k) + f(x_{k+1},y_{k+1})\bigr)
\end{equation}
Im Allgemeinen kann man diese Gleichung nicht nach $y_{k+1}$ auflösen.
\item
Setzt man die Differentialgleichung $f(x,y)=y$ ein, erhält man
\[
\frac{y_{k+1}-y_{k}}{x_{k+1}-x_{k}}
=
\frac{1}{h}(y_{k+1}-y_{k})
=
\frac12\bigl(y_k + y_{k+1}\bigr)
\]
was man nach $y_{k+1}$ auflösen kann:
\begin{equation}
2y_{k+1}-2y_{k} = hy_{k} + hy_{k+1}
\quad\Rightarrow\quad
(2-h)y_{k+1} = (2+h)y_k
\quad\Rightarrow\quad
y_{k+1} = \frac{2+h}{2-h}y_k.
\label{5001:verfahren}
\end{equation}
\item
Mit $h=\frac1h$ und $n$ Schritten erhält man
\[
y_n
=
y_0
\biggl(
\frac{2+\frac1n}{2-\frac1n}
\biggr)^n
=
y_0
\biggl(
\frac{1+\frac1{2n}}{1-\frac1{2n}}
\biggr)^n
=
y_0\biggl(1+\frac1{2n}\biggr)^n \biggl(1+\frac1{2n}+\frac1{(2n)^2}+\dots\biggr)^n
\simeq
y_0\biggl(1+\frac1n+o\biggl(\frac1{2n}\biggr)\biggr)^n
\to
y_0 e
\]
für $n\to \infty$.
Analog kann man für $n$ Schritte zwischen $0$ und $t$ die Formel
$y_n=y_0e^t$ ableiten.
\item
Die Formeln \eqref{5001:verfahren} erlauben jetzt auch, die
Konvergenzgeschwindigkeit zu berechnen.
Wir vergleichen dazu die Werte 
\[
\biggl(\frac{2+2^{-n}}{2-2^{-n}}\biggr)^{2^n}
\qquad
\text{mit}
\qquad
e.
\]
Wir erhalten die Werte in Tabelle~\ref{5001:tabelle}, die andeuten,
dass ein Verfahren erster Ordnung vorliegt.
\qedhere
\begin{table}
\centering
\begin{tabular}{|>{$}r<{$}|>{$}l<{$}|}
\hline
k&\text{Wert}\\
\hline
 1 & \underline{   2.7}777777777777781\\
 2 & \underline{   2.7}326114119117042\\
 3 & \underline{   2.7}218318928456022\\
 4 & \underline{   2.71}91673488624724\\
 5 & \underline{   2.718}5030842147762\\
 6 & \underline{   2.718}3371346324057\\
 7 & \underline{   2.7182}956545171231\\
 8 & \underline{   2.71828}52849433168\\
 9 & \underline{   2.71828}26925781960\\
10 & \underline{   2.71828}20444885607\\
11 & \underline{   2.7182818}824664454\\
12 & \underline{   2.7182818}419608967\\
13 & \underline{   2.7182818}318338904\\
14 & \underline{   2.71828182}93028337\\
15 & \underline{   2.718281828}6700021\\
16 & \underline{   2.718281828}5117856\\
17 & \underline{   2.718281828}5117865\\
18 & \underline{   2.71828182845}24526\\
19 & \underline{   2.71828182845}73971\\
20 & \underline{   2.71828182845}86330\\
\hline
\infty& 2.71828182845905\\
\hline
\end{tabular}
\caption{Näherungswerte für $y(1)$ für die Lösung $y(x)$ der
Differentialgleichung $y'=y$ mit Anfangswert $y(0)=1$.
Die korrekten Stellen sind unterstrichen.
Etwa jede dritte Iteration liefert eine zusätzliche Dezimalstelle
Genauigkeit, woraus man ablesen kann, dass ein Verfahren erster
Ordnung vorliegt.
\label{5001:tabelle}}
\end{table}
\end{teilaufgaben}
\end{loesung}


