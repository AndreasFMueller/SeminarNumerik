%
% fehler.tex -- 
%
% (c) 2015 Prof Dr Andreas Mueller, Hochschule Rapperswil
%
\section{Fehler-Entwicklung numerischer Lösungen}
\rhead{Fehler-Entwicklung}
\index{Fehler}%
Wir betrachten wieder die Differentialgleichung~\eqref{buch:ode:eulerdgl}
und versuchen, den Fehler eines Näherungsverfahrens zu bestimmen,
welches Schritte der Grösse $h$ durchführt, um den Wert $y(x)$
zu approximieren.

Das Euler-Verfahren verwendet Schritte der Form
\index{Euler-Verfahren}%
\[
y_{k+1}=y_{k\mathstrut} + hf(x_{k\mathstrut},y_{k\mathstrut}).
\]
In jedem einzelnen Schritt der Länge $\Delta x$
entsteht ein Fehler, dessen Grösse wir
aus der Taylor-Entwicklung
\index{Taylor-Reihe}%
\[
y(x+\Delta x)
=
y(x) + y'(x)\cdot \Delta x + R(x) \Delta x^2
\]
abschätzen können.
Die Funktion $R(x)$ ist beschränkt und beschreibt den verbleibenden
Fehler.
Um $y(x)$ zu approximieren, müssen $n=x/h$ Schritte der Schrittweite
$h$ durchgeführt werden, von denen jeder einen Fehler
von der Grössenordnung $R(x)h^2$ hat.
Der Gesamtfehler ist daher von der Grössenordnung
\[
y(x)-y_n=O\biggl(R(x)h^2\frac{x}h\biggr)=O(h),
\]
er ist also von erster Ordnung in $h$.
Um eine zusätzliche Stelle Genauigkeit zu erhalten, muss man also zehnmal
so viele Schritte von zehnmal kleinerer Länge durchführen.
Die höhere Zahl von Einzelschritt führt jedoch 
auch zusätzliche Rundungsfehler ein.

Könnte man den Fehler des Einzelschrittes wesentlich verkleinern, würde
auch die Abhängigkeit des Fehlers des Verfahrens von der Schrittweite
vorteilhafter.
Wäre der Fehler des Einzelschrittes $O(h^k)$ statt $O(h^2)$, dann
wäre der Gesamtfehler des Verfahrens nur noch $O(h^{k-1})$.
Für $k=3$ bedeutet dies, dass eine Halbierung der Schrittweite
zwar doppelt so viele Schritte braucht, aber auch, dass in jedem
Schritt nur ein Achtel des Fehlers auftritt.
Der Gesamtfehler ist also nur ein Viertel.
Mit zehnmal mehr Arbeit kann man also nicht nur eine Stelle an
Genauigkeit gewinnen, sondern gleich deren zwei.

Man nennt ein Verfahren, bei dem der Gesamt-Fehler von der Grössenordnung
$O(h^k)$ ist, von einem Verfahren $k$-ter Ordnung.
\index{Verfahren kter Ordnung@Verfahren $k$-ter Ordnung}%
Das Euler-Verfahren ist also ein Verfahren erster Ordnung oder ein
lineares Verfahren.
In der Praxis werden Verfahren bis zu vierter und fünfter Ordnung
verwendet, so dass eine zehnmal kleinere Schrittweite zu gleich
vier Stellen Genauigkeitsgewinn führen.
Das Ziel der kommenden Abschnitte muss daher sein, einfach
berechnebare Approximationen der Funktion $y(x)$ mit möglichst geringen
Einzelschrittfehlern zu finden.

