%
% grundlagen.tex -- Richtungsfeld und Lösungen
%
% (c) 2020 Prof Dr Andreas Müller, Hochschule Rapperswil
%
\documentclass[tikz]{standalone}
\usepackage{amsmath}
\usepackage{times}
\usepackage{txfonts}
\usepackage{pgfplots}
\usepackage{csvsimple}
\usetikzlibrary{arrows,intersections,math}
\begin{document}
\def\skala{2.7}
\begin{tikzpicture}[>=latex,thick,scale=\skala]

\def\kurve#1{
	\draw[color=red,line width=1.4pt]
		plot[domain=-1.125:3.625,samples=100]
			({\x},{1+\x+#1*exp(\x)});
}

\begin{scope}
\clip (-1.125,-0.625) rectangle (3.625,3.125);

\foreach \c in {-2,-1.8,...,2}{
	\kurve{\c}
}

\end{scope}

\def\d{0.1}
\foreach \x in {-1,-0.75,...,3.5}{
	\foreach \y in {-0.5,-0.25,...,3}{
		\draw[color=blue]
			({\x-\d*cos(atan(\y-\x))},{\y-\d*sin(atan(\y-\x))})
			--
			({\x+\d*cos(atan(\y-\x))},{\y+\d*sin(atan(\y-\x))});
	}
}

\draw[->] (-1.125,0)--(3.7,0) coordinate[label={$x$}];
\draw[->] (0,-0.625)--(0,3.3) coordinate[label={left:$y$}];

\end{tikzpicture}
\end{document}

