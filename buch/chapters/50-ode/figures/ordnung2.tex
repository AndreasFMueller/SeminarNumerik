%
% tikztemplate.tex -- template for standalon tikz images
%
% (c) 2020 Prof Dr Andreas Müller, Hochschule Rapperswil
%
\documentclass[tikz]{standalone}
\usepackage{amsmath}
\usepackage{times}
\usepackage{txfonts}
\usepackage{pgfplots}
\usepackage{csvsimple}
\usetikzlibrary{arrows,intersections,math}
\begin{document}
\def\skala{1}
\begin{tikzpicture}[>=latex,thick,scale=\skala]

\definecolor{darkgreen}{rgb}{0,0.6,0}
\def\h{0.2}
\xdef\xnull{1.0}
\xdef\ynull{2.7}

\def\richtungsfeld{
	\foreach \x in {0,0.5,...,6}{
		\foreach \y in {0,0.5,...,5.5}{
			\pgfmathparse{atan(0.25*(\y-\x))}
			\xdef\m{\pgfmathresult}
			\draw[color=blue] 
				({\x-\h*cos(\m)},{\y-\h*sin(\m)})
				--
				({\x+\h*cos(\m)},{\y+\h*sin(\m)});
		}
	}
}

\def\achsen{
	\draw[->] (-0.1,0) -- (6.5,0) coordinate[label={$x$}];
	\draw[->] (0,-0.1) -- (0,5.9) coordinate[label={right:$y$}];
}

\def\steigung#1#2#3{
	\pgfmathparse{atan(#3)}
	\xdef\winkel{\pgfmathresult}
	\draw[color=darkgreen,line width=2.4pt]
		({#1-3*\h*cos(\winkel)},{#2-3*\h*sin(\winkel)})
		--
		({#1+3*\h*cos(\winkel)},{#2+3*\h*sin(\winkel)});
}

\def\kurve{
	\begin{scope}
	\clip (-0.2,-0.1) rectangle (6.2,6.1);
	\draw[color=red,line width=1.4pt] plot[domain=-0.1:2,samples=100]
		({4*\x},{4*(\x+1+((\ynull/4)-(\xnull/4)-1)*exp(\x-(\xnull/4)))});
	\end{scope}
}

\def\labels{
	\fill[color=white,opacity=0.7]
		({\xnull-1.2},{\ynull+0.1}) rectangle ({\xnull-0.1},{\ynull+0.5});
	\node at (\xnull,\ynull) [above left] {$(x_0,y_0)$};

	\fill[color=white,opacity=0.7]
		({\xone-0.6},{\yone+0.2}) rectangle ({\xone+0.6},{\yone+0.6});
	\node at (\xone,{\yone+0.1}) [above] {$(x_1,y_1)$};

	\fill[color=white,opacity=0.7]
		({\xtwo+0.1},{\ytwo+0.1}) rectangle ({\xtwo+1.2},{\ytwo+0.5});
	\node at (\xtwo,\ytwo) [above right] {$(x_2,y_2)$};
}

\def\rk{
	\pgfmathparse{0.25*(\ynull-\xnull)}
	\xdef\mnull{\pgfmathresult}
	\pgfmathparse{\xnull+4}
	\xdef\xone{\pgfmathresult}
	\pgfmathparse{\ynull+4*\mnull}
	\xdef\yone{\pgfmathresult}
	\pgfmathparse{0.25*(\yone-\xone)}
	\xdef\mone{\pgfmathresult}
	\pgfmathparse{0.5*(\mnull+\mone)}
	\xdef\m{\pgfmathresult}
	\pgfmathparse{\xnull+4}
	\xdef\xtwo{\pgfmathresult}
	\pgfmathparse{\ynull+4*\m}
	\xdef\ytwo{\pgfmathresult}

	\steigung{\xnull}{\ynull}{\mnull}
	\steigung{\xone}{\yone}{\mone}

	\draw[color=orange,line width=1.4pt] (\xnull,\ynull) -- (\xtwo,\ytwo);
	\draw[color=gray,line width=1.4pt] (\xnull,\ynull) -- (\xone,\yone);
}

\def\euler{
	\pgfmathparse{0.25*(\ynull-\xnull)}
	\xdef\m{\pgfmathresult}
	\pgfmathparse{\xnull+2}
	\xdef\xone{\pgfmathresult}
	\pgfmathparse{\ynull+2*\m}
	\xdef\yone{\pgfmathresult}

	\pgfmathparse{0.25*(\yone-\xone)}
	\xdef\m{\pgfmathresult}

	\steigung{\xone}{\yone}{\m}

	\pgfmathparse{\xnull+4}
	\xdef\xtwo{\pgfmathresult}
	\pgfmathparse{\ynull+4*\m}
	\xdef\ytwo{\pgfmathresult}
	\draw[color=orange,line width=1.4pt] (\xnull,\ynull) -- (\xtwo,\ytwo);
	\draw[color=gray,line width=1.4pt] (\xnull,\ynull) -- (\xone,\yone);
}

\def\punkte{
	\fill[color=orange] (\xtwo,\ytwo) circle[radius=0.10];
	\fill[color=gray] (\xnull,\ynull) circle[radius=0.10];
	\fill[color=gray] (\xone,\yone) circle[radius=0.10];
	\fill[color=white] (\xnull,\ynull) circle[radius=0.06];
	\fill[color=white] (\xone,\yone) circle[radius=0.06];
	\fill[color=white] (\xtwo,\ytwo) circle[radius=0.06];
}

\begin{scope}[xshift=-3.5cm]
\kurve
\richtungsfeld
\euler
\punkte
\achsen
\labels
\end{scope}

\begin{scope}[xshift=3.5cm]
\kurve
\richtungsfeld
\rk
\punkte
\achsen
\labels
\end{scope}

\end{tikzpicture}
\end{document}

