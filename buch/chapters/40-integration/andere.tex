%
% andere.tex
%
% (c) 2020 Prof Dr Andreas Müller, Hochschule Rapperswil
%
\section{Weitere Integrationsverfahren
\label{buch:section:weitereintegrationsverfahren}}
\rhead{Weitere Integrationsverfahren}
Die bisher vorgestellten Verfahren haben vor allem den Mangel, dass
die Zahl der Funktionsauswertungen sehr gross wird, wenn eine
hohe Genauigkeit angestrebt wird.
Das Verfahren der {\em Gauss-Quadratur}, im Kapitel~\ref{chapter:quadratur}
\index{Gauss-Quadratur}%
vorgestellt, findet hochgenaue Integralwerte mit einer sehr geringen
Anzahl von Funktionsauswertungen.

Die Wahrscheinlichkeitsrechnung hat einen besonderen Bedarf nach
Integralauswertungen, wie zum Beispiel die Berechnung der 
Verteilungsfunktion der Normalverteilung.
In diesen Anwendungen gibt es jedoch häufig auch die Möglichkeit,
die gesuchte Wahrscheinlichkeit durch Simulation eines
Wahrscheinlichkeitsexperimentes zu erhalten.
Dies führt zu den sogenannten {\em Monte-Carlo-Methoden}.
\index{Monte-Carlo-Methode}%

