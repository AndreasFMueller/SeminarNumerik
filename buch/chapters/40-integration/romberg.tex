%
% romberg.tex
%
% (c) 2020 Prof Dr Andreas Müller, Hochschule Rapperswil
%
\section{Romberg-Algorithmus
\label{buch:section:romberg}}
\rhead{Romberg-Algorithmus}
\index{Romberg-Algorithmus}%
Die bisher betrachteten Methoden zur Berechnung von Integralen zeichnen
sich dadurch aus, dass die Fehler bei einer gegebenen Schrittweite 
sehr genau bekannt sind.
Es sollte daher möglich sein, die Ideen von
Abschnitt~\ref{buch:subsection:konvergenzbeschleunigung}
anzuwenden und die Konvergenz des Verfahrens zu beschleunigen, ohne
dass weitere Funktionswerte berechnet werden müssen.

\subsubsection{Elmination des Fehlers}
\index{Elimination}%
\index{Fehler}%
Wir kehren zurück zur Berechnung des Integrals
\index{Integral}%
\[
I=\int_a^b f(x)\,dx
\]
mit Hilfe der Trapezregel und bezeichnen wieder mit $T(h)$
die Approximation mit der Trapezregel mit Schrittweite $h$.
\index{Approximation}%
\index{Trapezregel}%
In \eqref{buch:equation:trapezfehler} haben wir den Fehler von $T(h)$
bestimmt.
Wir nehmen an, dass der Fehler vierter Ordnung im Vergleich zum Term
$Ch^2$ vernachlässigbar ist.
Wenn wir die Schrittweite immer wieder halbieren, erhalten wir immer bessere
Approximationen
\index{Schrittweite}%
\index{Halbierung}%
\begin{align*}
I&=T(h) + Ch^2
\\
&=T\biggl(\frac{h}2\biggr) + \frac{Ch^2}{4}.
\end{align*}
Nehmen wir für den Moment an, dass dies exakte Gleichungen sind, dann
können wir die Unbekannte $C$ eliminieren und nach der Unbekannten $I$
auflösen.
Dazu multiplizieren wir die zweite Gleichung mit $4$ und subtrahieren
die erste Gleichung.
Wir erhalten
\[
3I= 4T\biggl(\frac{h}2\biggr) - T(h)
\qquad\Rightarrow\qquad
I= \frac{4T(h/2) - T(h)}{3}.
\]
Natürlich ist dies auch noch nicht der exakte Wert des Integrals,
wir haben ja die Terme vierter Ordnung vernachlässigt.
Wir können aber daraus schliessen, dass 
\[
T^*(h/2) = \frac{4T(h/2)-T(h)}{3}
\]
eine Approximation für das Integral ist, deren Fehler nur noch
von vierter Ordnung $O(h^4)$ sein wird.

Da wir für die Approximationen $T^*(h)$ auch wieder eine Approximation
für die Fehler haben, können wir das gleiche Prinzip nochmals anwenden
und auch den Fehler vierter Ordnung eliminieren.
Multiplizieren wir die zweite Gleichung in
\begin{align*}
I&=T^*(h) + Ch^4 \\
 &=T^*(h/2) + C\frac{h^4}{16}
\end{align*}
mit $16$ und subtrahieren die erste, so erhalten wir
\[
15I = 16T^*(h/2)-T(h)
\qquad\Rightarrow\qquad
T^{**}(h/2) = I = \frac{16T^*(h/2)-T^*(h)}{15}.
\]
Dieses Verfahren der fortlaufenden Verbesserung der Konvergenz durch
Elimination des Fehlers ist bekannt als {\em Romberg-Algorithmus}.
\index{Romberg-Algorithmus}%
\index{Konvergenz}%

\subsubsection{Ein Beispiel}
Wir berechnen das Integral
\[
I = \int_0^{\pi} \sin x\,dx = 2
\]
mit Hilfe der Trapezregel und wenden das Beschleunigungsschema von Romberg
auf die erhaltenen Werte an.
\index{Konvergenzbeschleunigung}%
\index{Beschleunigung}%

\begin{table}
\def\u#1{\underline{#1}}
\centering
\renewcommand\arraystretch{1.15}
\begin{tabular}{|>{$}c<{$}|>{$}c<{$} >{$}c<{$} >{$}c<{$}|}
\hline
 k& T(\pi2^{-k})&T^*(\pi2^{-k})&T^{**}(\pi2^{-k})\\
\hline
 0&  0.\u{}0000000000000002&                        &                        \\
 1&  1.\u{}5707963267948966&  2.\u{0}943951023931953&                        \\
 2&  1.\u{}8961188979370398&  2.\u{00}45597549844207&  1.\u{99}85707318238357\\
 3&  1.\u{9}742316019455508&  2.\u{000}2691699483877&  1.\u{9999}831309459855\\
 4&  1.\u{99}35703437723393&  2.\u{0000}165910479355&  1.\u{999999}7524545718\\
 5&  1.\u{99}83933609701441&  2.\u{00000}10333694123&  1.\u{99999999}61908441\\
 6&  1.\u{999}5983886400366&  2.\u{0000000}645300009&  1.\u{9999999999}407068\\
 7&  1.\u{999}8996001842024&  2.\u{00000000}40322576&  1.\u{999999999999}0750\\
 8&  1.\u{9999}749002350549&  2.\u{000000000}2520060&  1.\u{9999999999999}891\\
 9&  1.\u{99999}37250705797&  2.\u{0000000000}157545&  2.\u{00000000000000}44\\
10&  1.\u{99999}84312683776&  2.\u{000000000000}9770&  1.\u{99999999999999}20\\
11&  1.\u{999999}6078171345&  2.\u{0000000000000}533&  1.\u{99999999999999}18\\
\hline
\end{tabular}
\caption{Berechung des Integrals $\int_0^\pi \sin x\,dx$ mit Hilfe
der Trapezregel (Spalte $T(\pi 2^{-k})$) und Beschleunigung der
Konvergenz mit Hilfe des Romberg-Algorithmus.
Für die Resultate auf der Zeile $k>0$ sind $2^k + 1$ Funktionsauswertungen
notwendig.
\label{buch:table:sinromberg}}
\end{table}

Die gefundenen Resultate sind in Tabelle~\ref{buch:table:sinromberg}
dargestellt.
Man kann erkennen, dass die Maschinengenauigkeit von 15 Nachkommastellen
mit der Trapezregel auch bei $k=11$  nicht erreicht wird.
Für die erste Romberg-Beschleunigung $T^*$ wird die Genauigkeit
bei $k=11$ erreicht, sie kann aber durch Erhöhung von $k$ nicht
mehr verbessert werden, da die aufsummierten Rundungsfehler
wichtiger werden.
In beiden Fällen muss der Integrand $2049$ mal ausgewertet werden.

Die zweite Romberg-Beschleunigung erreicht Maschinengenauigkeit bereits bei
$k=9$, als mit nur $513$ Funktionsauswertungen.
Die Romberg-Beschleunigung ist in diesem Beispiel also ausserordentlich
erfolgreich.

\subsubsection{Allgemeine Form des Romberg-Verfahrens}
In noch allgemeinerer Form kann man die Formeln des Romberg-Algorithmus
wie folgt schreiben.
Die Werte 
$T_{k,0} = T(2^{-k}h)$ mit $k=0,1,\dots$
sind die Werte der Trapezsumme mit $h=2^{-k}(b-a)$.
Die Beschleunigungen sind
\begin{align*}
T_{k,1} &= \frac{4T_{k,0}-T_{k-1,0}}{4-1} \\
T_{k,2} &= \frac{4^2T_{k,1}-T_{k-1,1}}{4^2-1}\\
\vdots\quad&\hspace{1cm} \vdots\\
T_{k,l} &= \frac{4^lT_{k,l-1}-T_{k-1,l-1}}{4^l-1}
\end{align*}
Dieses Verfahren funktioniert und beschleunigt die Konvergenz,
sofern die Ableitungen von $f(x)$ bis zur Ordnung $2l$ ausreichend glatt
und beschränkt sind.

\subsubsection{Wieviele Stellen Genauigkeit lassen sich erreichen?}
\index{Romberg-Verfahren}%
Das Romberg-Verfahren basiert darauf, dass man die Trapezsummen 
für sukzessiv halbiert Schrittlänge berechnet.
In jeder Iteration wird die Schrittweite also halbiert.
Die Fehler von $T_{k,l}$ sind von der Ordnung $h^{2l}$, also
\[
T_{k,l} = I + O(2^{-2(l+1)}).
\]
Auf jeder Zeile wird der Fehler also um den Faktor
$2^{-2l}=\frac1{4^l}$ kleiner,
dies entspricht einem Gewinn von zwei Binärstellen oder 
$2(l+1)\log_{10}2=0.60206(l+1)$ Dezimalstellen.
Man kann dies auch in der Tabelle~\ref{buch:table:sinromberg}
verfolgen.
In der ersten Spalte mit $l=0$ braucht es etwa 10 Zeilen,
um sechs Dezimalstellen
zu gewinnen, in der zweiten Spalte mit $l=1$ sind sechs Dezimalstellen
bereits nach fünf Zeilen erreicht, 12 Dezimalstellen nach 10 Zeilen.
In der dritten Spalte mit $l=2$ erreicht man 12 Stellen bereits ungefähr
nach fünf Zeilen.



