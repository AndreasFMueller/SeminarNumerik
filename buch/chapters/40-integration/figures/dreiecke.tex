%
% dreiecke.tex -- Beispiel
%
% (c) 2020 Prof Dr Andreas Müller, Hochschule Rapperswil
%
\documentclass[tikz]{standalone}
\usepackage{amsmath}
\usepackage{times}
\usepackage{txfonts}
\usepackage{pgfplots}
\usepackage{csvsimple}
\usetikzlibrary{arrows,intersections,math}
\begin{document}
\def\skala{1}
\begin{tikzpicture}[>=latex,thick,scale=\skala]

\fill[color=blue!10] (0,0) -- (5,0)
	-- (5,4.5) -- (4,4.5)
	-- (4,3.5) -- (3,3.5)
	-- (3,2.5) -- (2,2.5)
	-- (2,1.5) -- (1,1.5)
	-- (1,0.5) -- (0,0.5) -- cycle;

\fill[color=blue!40] (2,2) -- (3,3) -- (3,2.5) -- (2,2.5) -- cycle;

\draw[color=blue] (0.5,0) -- (0.5,0.5);
\draw[color=blue] (1.5,0) -- (1.5,1.5);
\draw[color=blue] (2.5,0) -- (2.5,2.5);
\draw[color=blue] (3.5,0) -- (3.5,3.5);
\draw[color=blue] (4.5,0) -- (4.5,4.5);

\node at (0.5,0) [below] {$\xi_0$};
\node at (1.5,0) [below] {$\xi_1$};
\node at (2.5,0) [below] {$\xi_2$};
\node at (3.5,0) [below] {$\xi_3$};
\node at (4.5,0) [below] {$\xi_4$};

\draw[line width=1.4pt] (0,0) -- (5,5);

\draw[->] (-0.1,0) -- (5.5,0) coordinate[label={$x$}];
\draw[->] (0,-0.1) -- (0,5.5) coordinate[label={right:$y$}];

\end{tikzpicture}
\end{document}

