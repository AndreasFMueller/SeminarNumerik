Berechnen Sie das Integral
\[
I = \int_0^1 \sin x\,dx
\]
der Funktion $f(x)=\sin x$ auf vier verschiedene Weisen:
\begin{teilaufgaben}
\item Exakt.
\item Mit Hilfe der Mittelpunktsformel mit genau zwei Teilintervallen.
\item Mit Hilfe der Trapezformel mit genau zwei Teilintervallen.
\item Mit Hilfe der Simpsonschen Formel
\[
\int_a^b f(x)\,dx
\simeq
I_S
=
\frac{b-a}{6} \bigl( f(a) + 4f(m) + f(b)\bigr)
\qquad
\text{mit $m=(a+b)/2$}
\]
(Aufgabe 4.1).
\end{teilaufgaben}

\begin{loesung}
\begin{teilaufgaben}
\item
Die exakte Berechnung des Integrals liefert
\[
I
=
\int_0^1 \sin x\,dx
=
\biggl[-\cos x\biggr]_0^1 
=
1-\cos 1
\simeq
0.45969769413186028260.
\]
\item
Mit der Mittelpunktsformel ist die Näherung
\[
I_{M}
=
\frac12 f({\textstyle\frac14}) + f({\textstyle\frac34})
=
0.46452135963892854816.
\]
\item
Mit der Trapezformel ist die Näherung
\[
I_T
=
\frac14\bigl(f(0)+2f({\textstyle\frac12}) + f(1)\bigr)
=
\frac14(0 + 2\sin{\textstyle\frac12} + \sin 1)
=
0.45008051550407562679.
\]
\item
Mit der Simpsonschen Regel findet man 
\[
I_S
=
\frac16\bigl(f(0) + 4f({\textstyle\frac12}) + f(1)\bigr)
=
\frac23\sin{\textstyle\frac12} + \frac16\sin 1
=
0.45986218987078475127.
\]
\end{teilaufgaben}
Die Mittelpunktsformel und die Trapezregel verwenden Interpolation
höchstens vom Grad 1 zur Approximation des Integrals.
Die Simpsonsche Formel verwendet ein quadratisches Approximationspolynom,
von dem man eine etwas höhere Genauigkeit erwarten kann.
Wir vergleichen die Resultate in der folgenden Tabelle
\begin{center}
\begin{tabular}{|l|>{$}c<{$}|>{$}l<{$}|}
\hline
Methode& &Wert\\
\hline
exakt              & I   &             0.45969769413186028260 \\
Mittelpunktsformel & I_M & 0.\underline{46}452135963892854816 \\
Trapezformel       & I_T & 0.\underline{45}008051550407562679 \\
Simpsonsche Regel  & I_S & 0.\underline{459}86218987078475127 \\
\hline
\end{tabular}
\end{center}
Die Simpsonsche Regel scheint also tatsächlich etwas besser zu sein.

Man kann mit Hilfe der Simpsonschen Regel ebenfalls eine Integrationsmethode
ähnliche wie die Trapezregel finden.
Dieser zusätzliche Aufwand lohnt sich aber meistens nicht, da sich mit
der Trapezregel und anschliessender Romberg-Beschleunigung die gleiche
Genauigkeit leichter erreichen lässt.
\end{loesung}

