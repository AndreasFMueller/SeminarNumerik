%
% normalverteilung.tex
%
% (c) 2020 Prof Dr Andreas Müller, Hochschule Rapeprswil
%
\subsection{Inverse der Normalverteilungsfunktion
\label{buch:subsection:inversenormal}}
Das Integral der Standardnormalverteilungsdichte
\index{Standardnormalverteilungsdichte}%
\index{Normalverteilung}%
\[
\Phi(x) = \int_{-\infty}^x e^{-t^2/2}\,dt
\]
kann nicht in geschlossener Form berechnet werden und erst recht
nicht invertiert werden.
Für die Anwendung wird jedoch die Umkehrfunktion benötigt, zu einem Wert
$p\in[0,1]$ ist dasjenige $x$ zu finden, für welches $F(x)=p$ gilt.
\index{Umkehrfunktion}%
Im Beispiel auf Seite~\pageref{buch:beispiel:erfc} wurde gezeigt,
wie die Fehlerfunktion
\index{Fehlerfunktion}%
\index{erfx@$\operatorname{erf}(x)$}%
\[
\operatorname{erf}(x) = \frac{2}{\sqrt{\pi}}\int_0^x e^{-t^2}\,dt
\]
dazu verwendet werden kann, die Normalverteilungsfunktion
\index{Verteilungsfunktion}%
\index{Normalverteilungsfunktion}%
\[
\Phi(x)
=
\frac12\biggl(1+\operatorname{erf}\biggl(\frac{x}{\sqrt{2}}\biggr)\biggr)
\]
zu berechnen.
In diesem Abschnitt soll untersucht werden, wie zu gegebenen Funktionswert
$p$ das zugehörige $x$ bestimmt werden kann.
Es soll also die Gleichung
\[
\Phi(x)=p
\qquad\Rightarrow\qquad
f(x)=\frac12+\frac12\operatorname{erf}\biggl(\frac{x}{\sqrt{2}}\biggr)-p=0
\]
gelöst werden.

\subsubsection{Sekantenverfahren}
\index{Sekantenverfahren}%
Die mit dem Sekantenverfahren gewonnen Iterationsfolge ist in der rechten
Spalte in Tabelle~\ref{buch:table:normalnewton} dargestellt.
Da die erste Ableitung von $f$ relativ langsam ändert, ist die Konvergenz
zwar zunächst offenbar nur linear, beschleunigt sich dann aber auf fast
quadratische Konvergenz, weil die Sekante die Tangente sehr gut approximiert,
sich das Sekantenverfahren in ihrem Konvergenzverhalten also dem
Newton-Verfahren anzunähern beginnt.
\index{Sekante}%
\index{Tangente}%
Wegen der unvermeidbaren Auslöschung bei der Berechnung der Sekantensteigung
wird das Verfahren dann aber instabil, die Iteration bricht ab.
\index{Auslöschung}%
Es können für den Typ \texttt{long double} nur etwa 18 signifikante
Stellen gefunden werden, während das Newton-Verfahren noch drei weitere
Stellen ermitteln kann.
\index{Newton-Verfahren}%
\index{long double@\texttt{long double}}%

\subsubsection{Newton-Verfahren}
Das Newton-Verfahren benötigt ausser dem Funktionswert auch noch die 
Ableitung
\index{Ableitung}%
\[
f'(x)
=
\frac{d}{dx}\frac{1}{\sqrt{\pi}}\int_0^{x/\sqrt{2}} e^{-t^2}\,dt
=
\frac{1}{\sqrt{2\pi}}e^{-x^2/2}.
\]
Damit wird die Iterationsformel für das Newton-Verfahren:
\index{Iterationsfolge}%
\begin{equation}
x_{n+1}
=
x_n - \sqrt{2\pi} e^{x_n^2/2}
\biggl(\frac12+\operatorname{erf}\biggl(\frac{x_n}{\!\sqrt{2}}\biggr)-p\biggr).
\end{equation}
Wie erwartet konvergiert die Iterationsfolge quadratisch für geeignete
Startwerte (siehe Tabelle~\ref{buch:table:normalnewton}).
Der Startwert $x_0=0$ funktioniert für jedes beliebige $p$.
\index{Startwert}%
Bei weiter von $0$ entfernten Starwerten läuft man Gefahr, dass die Iteration
zu betragsmässig grossen Werten $x$ springt, was dann zu einem Überlauf führt.

\begin{table}
\centering
\begin{tabular}{|>{$}r<{$}|>{$}r<{$}|>{$}r<{$}|}
\hline
 k &\textrm{$x_n$ nach Newton-Verfahren}&\textrm{$x_n$ nach Sekantenverfahren}\\
\hline
 0 &             0.00000000000000000000 &             0.00000000000000000000 \\
 1 & \underline{1}.12798272358394999736 & \underline{1}.00000000000000000000 \\
 2 & \underline{1.}50523868934241237280 & \underline{1}.31831529614238960517 \\
 3 & \underline{1.6}3077255164383121513 & \underline{1.}53246084663801306026 \\
 4 & \underline{1.644}69272791792957300 & \underline{1.6}2023672225157423321 \\
 5 & \underline{1.6448536}0566334308020 & \underline{1.64}272141855681471658 \\
 6 & \underline{1.644853626951472}02343 & \underline{1.6448}1100638619422225 \\
 7 & \underline{1.64485362695147239662} & \underline{1.644853}55229014184382 \\
 8 & \underline{1.64485362695147239662} & \underline{1.6448536269}4885539842 \\
 9 &                                    & \underline{1.64485362695147239}597 \\
\hline
\end{tabular}
\caption{Newton-Iteration und Sekantenverfahren zur Bestimmung der Inversen
der Verteilungsfunktion $\Phi(x)$ der Normalverteilung, berechnet mit dem Typ
\texttt{long double}.
Das Sekantenverfahren ist auch sehr schnell, da die Sekante bereits sehr gut
mit der vom Newton-Verfahren verwendeten Tangente übereinstimmt.
Wegen Auslöschung kann es allerdings nicht die volle Genauigkeit erreichen.
\label{buch:table:normalnewton}}
\end{table}





