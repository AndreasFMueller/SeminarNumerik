%
% normalverteilung.tex
%
% (c) 2020 Prof Dr Andreas Müller, Hochschule Rapeprswil
%
\subsection{Inverse der Normalverteilungsfunktion
\label{buch:subsection:inversenormal}}
Das Integral der Standardnormalverteilungsdichte
\[
\Phi(x) = \int_{-\infty}^x e^{-t^2/2}\,dt
\]
kann nicht in geschlossener Form berechnet werden und erst recht
nicht invertiert werden.
Für die Anwendung wird jedoch die Umkehrfunktion benötigt, zu einem Wert
$p\in[0,1]$ ist dasjenige $x$ zu finden, für welches $F(x)=p$ gilt.
Im Beispiel auf Seite~\pageref{buch:beispiel:erfc} wurde gezeigt,
wie die Fehlerfunktion
\[
\operatorname{erf}(x) = \frac{2}{\sqrt{\pi}}\int_0^x e^{-t^2}\,dt
\]
dazu verwendet werden kann
\[
\Phi(x) = \frac12+\operatorname{erf}(\sqrt{2}x)
\]
zu berechnen.
In diesem Abschnitt soll untersucht werden, wie zu gegebenen Funktionswert
das $x$ bestimmt werden kann.
Es soll also die Gleichung
\[
\Phi(x)=p
\qquad\Rightarrow\qquad
f(x)=\frac12+\operatorname{erf}(\sqrt{2}x)-p=0
\]
gelöst werden.

\subsubsection{Newton-Verfahren}
Das Newton-Verfahren benötigt ausser dem Funktionswert auch noch die 
Ableitung
\[
f'(x)
=
\frac{d}{dx}\frac{2}{\sqrt{\pi}}\int_0^x e^{-t^2}\,dt
=
e^{-x^2}.
\]
Damit wird die Iterationsformel für das Newton-Verfahren:
\begin{equation}
x_{n+1} = x_n - e^{x_n^2} \biggl(\frac12+\operatorname{erf}(x_n) -p \biggr).
\end{equation}





