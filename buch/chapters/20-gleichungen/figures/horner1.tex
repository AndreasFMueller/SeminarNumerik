%
% horner1.tex -- template for standalon tikz images
%
% (c) 2020 Prof Dr Andreas Müller, Hochschule Rapperswil
%
\documentclass[tikz]{standalone}
\usepackage{amsmath}
\usepackage{times}
\usepackage{txfonts}
\usepackage{pgfplots}
\usepackage{csvsimple}
\usetikzlibrary{arrows,intersections,math}
\begin{document}
\def\skala{1.2}
\begin{tikzpicture}[>=latex,thick,scale=\skala]

\draw[color=gray] (-0.5,-1.6)--(8.5,-1.6);

\fill[color=red!20] (7.7,-2.15) rectangle (8.3,-1.85);

\def\koef{-0.2}

\node at (0,\koef) {$a_n\mathstrut$};
\node at (1,\koef) {$a_{n-1}\mathstrut$};
\node at (2,\koef) {$a_{n-2}\mathstrut$};
\node at (3,\koef) {$a_{n-3}\mathstrut$};
\node at (4.5,\koef) {$\cdots\mathstrut$};
\node at (6,\koef) {$a_2\mathstrut$};
\node at (7,\koef) {$a_1\mathstrut$};
\node at (8,\koef) {$a_0\mathstrut$};

\node at (0,-2) {$a_n\mathstrut$};
\node at (1,-2) {$p_1\mathstrut$};
\node at (2,-2) {$p_2\mathstrut$};
\node at (3,-2) {$p_3\mathstrut$};
\node at (4,-2) {$\cdots\mathstrut$};
\node at (5,-2) {$p_{n-3}\mathstrut$};
\node at (6,-2) {$p_{n-2}\mathstrut$};
\node at (7,-2) {$p_{n-1}\mathstrut$};
\node at (8,-2) {$f(x)\mathstrut$};

\node at (1,-1) {$a_nx\mathstrut$};
\node at (2,-1) {$p_1x\mathstrut$};
\node at (3,-1) {$p_2x\mathstrut$};
\node at (4.5,-1) {$\cdots\mathstrut$};
\node at (6,-1) {$p_{n-3}x\mathstrut$};
\node at (7,-1) {$p_{n-2}x\mathstrut$};
\node at (8,-1) {$p_{n-1}x\mathstrut$};

\def\pfeil#1{
	\draw[->,shorten >= 0.3cm,shorten <=0.3cm] (#1,-2)--({#1+1},-1)
		node[midway,above left=-0.1cm] {$\cdot x$};
}
\pfeil{0}
\pfeil{1}
\pfeil{2}
\pfeil{5}
\pfeil{6}
\pfeil{7}

\def\pluszeichen#1{
	\node at (#1,-0.6) {$+$};
}
\pluszeichen{1}
\pluszeichen{2}
\pluszeichen{3}
\pluszeichen{6}
\pluszeichen{7}
\pluszeichen{8}

\def\gleich#1{
\draw[double equal sign distance,shorten >= 0.2cm,shorten <=0.2cm]
	(#1,-1)--(#1,-2);
}
\gleich{1}
\gleich{2}
\gleich{3}
\gleich{6}
\gleich{7}
\gleich{8}

\end{tikzpicture}
\end{document}

