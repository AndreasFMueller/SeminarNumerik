%
% newton.tex -- Newton-Verfahren
%
% (c) 2020 Prof Dr Andreas Müller, Hochschule Rapperswil
%
\documentclass[tikz]{standalone}
\usepackage{amsmath}
\usepackage{times}
\usepackage{txfonts}
\usepackage{pgfplots}
\usepackage{csvsimple}
\usetikzlibrary{arrows,intersections,math}
\begin{document}
\def\skala{1}
\begin{tikzpicture}[>=latex,thick,scale=\skala]

\def\f{4}

\begin{scope}
\clip (-6,-1.1) rectangle (3.1,5.1);
\draw[color=red,line width=1.4pt]
	plot[domain=-1.5:0.5,samples=100] ({\f*\x},{\f*(exp(\x)-0.5});
\end{scope}

\def\x{0.45}
\xdef\xzero{\x}
\def\schritt#1{
	\pgfmathparse{exp(\x)-0.5}
	\xdef\y{\pgfmathresult}
	\pgfmathparse{exp(\x)}
	\xdef\m{\pgfmathresult}
	\pgfmathparse{\x-\y/\m}
	\xdef\xneu{\pgfmathresult}
	
	\ifthenelse{#1=0}{
		\fill[color=gray!20]
			({\f*\x},{\f*\y})
			--
			({\f*\x},0)
			--
			({\f*\xneu},0)
			--cycle;
		\node at ({\f*\x},0) [below right] {$A$};
		\node at ({\f*\x},{\f*\y}) [right] {$B$};
		\node at ({\f*\xneu},0) [below right] {$C$};
	}{}
	\draw[color=gray] ({\f*\x},0)--({\f*\x},{\f*\y});
	\draw[color=gray] ({\f*(\x+0.1)},{\f*(\y+0.1*\m)})
				--({\f*(\xneu-0.1)},{\f*(0-0.1*\m)});
	\fill[color=blue] ({\f*\x},{\f*\y}) circle[radius=0.08];
	\draw[line width=0.2pt] ({\f*\x},0)--({\f*\x},{-0.8});
	\node at ({\f*\x},{-0.8}) [below] {$x_{#1}$};
	\xdef\x{\xneu}
}
\schritt{0}

\node at ({\f*0.5*(\x+\xzero)},{\f*0.5*\y}) [below right] {$f'(x_0)$};
\node at ({\f*\xzero},{0.5*\f*\y}) [right] {$f(x_0)$};

\schritt{1}
\schritt{2}
\schritt{3}

\draw[->] (-6.1,0) -- (3.3,0) coordinate[label={$x$}];
\draw[->] (0,-1.2) --(0,5.2) coordinate[label={$y$}];

\draw (-4,-0.1)--(-4,0.1);
\node at (-4,-0.1) [above] {$-1\mathstrut$};

\fill[color=red] ({\f*ln(0.5)},0) circle[radius=0.1];
\node[color=red] at ({\f*ln(0.5)},0) [above] {$x^*$};

\end{tikzpicture}
\end{document}

