%
% stufe.tex -- template for standalon tikz images
%
% (c) 2020 Prof Dr Andreas Müller, Hochschule Rapperswil
%
\documentclass[tikz]{standalone}
\usepackage{amsmath}
\usepackage{times}
\usepackage{txfonts}
\usepackage{pgfplots}
\usepackage{csvsimple}
\usetikzlibrary{arrows,intersections,math}
\begin{document}
\def\skala{1}
\begin{tikzpicture}[>=latex,thick,scale=\skala]

\pgfmathparse{3.5/0.3}
\xdef\m{\pgfmathresult}

\draw[color=red] (-0.1,-3)--(0.2,-3)--(0.5,0.5)--(6.5,0.5)--(7.5,3)--(9.1,3);
\fill[color=red] ({0.2+3/\m},0) circle[radius=0.08];
\node[color=red] at ({0.2+3/\m},0) [above left] {$x_0$};

\draw[color=blue] (-0.1,-3)--(1.5,-3)--(2.5,-0.5)--(8.5,-0.5)--(8.8,3)--(9.1,3);
\fill[color=blue] ({8.8-3/\m},0) circle[radius=0.08];
\node[color=blue] at ({8.8-3/\m},0) [above left] {$x_0$};

\draw[->] (-0.1,0)--(9.3,0) coordinate[label={$x$}];
\draw[->] (0,-3.1)--(0,3.3) coordinate[label={right:$y$}];

\foreach \x in {1,...,9}{
	\draw (\x,-0.1)--(\x,0.1);
	\node at (\x,-0.0) [below] {$\x$};
}
\foreach \y in {1,...,3}{
	\draw (-0.1,\y)--(0.1,\y);
	\node at (-0.1,\y) [left] {$\y$};
	\draw (-0.1,-\y)--(0.1,-\y);
	\node at (-0.1,-\y) [left] {$-\y$};
}
\node at (-0.1,0) [left] {$0$};

\end{tikzpicture}
\end{document}

