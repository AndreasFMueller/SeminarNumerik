%
% kondition.tex
%
% (c) 2020 Prof Dr Andreas Müller, Hochschule Rapperswil
%
\section{Kondition
\label{buch:section:kondition}}
Die Diskussion der iterativen Verfahren in
Abschnitt~\ref{buch:section:gaussseidel} hat gezeigt, dass
der Spektralradius aus Definition~\ref{buch:definition:spektralradius}
Auskunft gibt darüber, ob ein iteratives Verfahren konvergiert.
Insbesondere wurde behauptet, dass Spektralradius $\varrho(M)$ und
Gelfand-Radius $\pi(M)$ übereinstimmen.
In diesem Abschnitt sollen diese Kennzahl noch etwas vertieft
untersucht werden.

%
% Gelfand-Radius und Eigenwerte
%
\subsection{Gelfand-Radius und Eigenwerte
\label{buch:subsection:spektralradius}}
In Abschnitt~\ref{buch:subsection:konvergenzbedingung}
ist der Gelfand-Radius mit Hilfe eines Grenzwertes definiert worden.
Nur dieser Grenzwert ist in der Lage, über die Konvergenz eines 
Iterationsverfahrens Auskunft zu geben.
Der Grenzwert ist aber sehr mühsam zu berechnen.
Es wurde angedeutet, dass er mit dem Gelfand-Radius übereinstimmt
mit dem Spektralradius, dem Betrag des des betragsgrössten Eigenwertes.
Dies hat uns ein vergleichsweise einfach auszuwertendes Konvergenzkriterium
geliefert.
In diesem Abschnitt soll diese Identität zunächst an Spezialfällen
und später ganz allgemein gezeigt werden.

\subsubsection{Spezialfall: Diagonalisierbare Matrizen}
Ist eine Matrix $A$ diagonalisierbar, dann kann Sie durch eine Wahl
einer geeigneten Basis in Diagonalform
\[
A'
=
\begin{pmatrix}
\lambda_1&        0&\dots &0\\
0        &\lambda_2&\dots &0\\
\vdots   &         &\ddots&\vdots\\
0        &        0&\dots &\lambda_n
\end{pmatrix}
\]
gebracht werden, wobei die Eigenwerte $\lambda_i$  möglicherweise auch
komplex sein können.
Die Bezeichnungen sollen so gewählt sein, dass $\lambda_1$ der
betragsgrösste Eigenwert ist, dass also
\[
|\lambda_1| \ge |\lambda_2| \ge \dots \ge |\lambda_n.
\]
Wir nehmen für die folgende, einführende Diskussion ausserdem an, dass
sogar $|\lambda_1|>|\lambda_2|$ gilt.

Unter den genannten Voraussetzungen kann man jetzt den Gelfand-Radius
von $A$ berechnen.
Dazu muss man $|A^nv|$ für einen beliebigen Vektor $v$ und für
beliebiges $n$ berechnen.
Der Vektor $v$ lässt sich in der Eigenbasis von $A$ zerlegen, also
als Summe
\[
v = v_1+v_2+\dots+v_n
\]
schreiben, wobei $v_i$ Eigenvektoren zum Eigenwert $\lambda_i$ sind oder
Nullvektoren.
Die Anwendung von $A^k$ ergibt dann
\[
A^k v
=
A^k v_1 + A^k v_2 + \dots + A^k v_n
=
\lambda_1^k v_1 + \lambda_2^k v_2 + \dots + \lambda_n^k v_n.
\]
Für den Grenzwert braucht man die Norm von $A^kv$, also
\begin{align}
|A^kv|
&= |\lambda_1^k v_1 + \lambda_2^k v_2 + \dots + \lambda_3 v_3|
\notag
\\
\Rightarrow\qquad
\frac{|A^kv|}{\lambda_1^k}
&=
\biggl|
v_1 +
\biggl(\frac{\lambda_2}{\lambda_1}\biggr)^k v_2
+
\dots
+
\biggl(\frac{\lambda_n}{\lambda_1}\biggr)^k v_n
\biggr|.
\label{buch:spektralradius:eqn:eigenwerte}
\end{align}
Da alle Quotienten $|\lambda_i/\lambda_1|<1$ sind für $i\ge 2$,
konvergieren alle Terme auf der rechten Seite von
\eqref{buch:spektralradius:eqn:eigenwerte}
ausser dem ersten gegen $0$.
Folglich ist
\[
\lim_{k\to\infty} \frac{|A^kv|}{|\lambda_1|^k}
=
|v_1|
\qquad\Rightarrow\qquad
\lim_{k\to\infty} \frac{|A^kv|^\frac1k}{|\lambda_1|}
=
\lim_{k\to\infty}|v_1|^{\frac1k}
=
1.
\]
Dies gilt für alle Vektoren $v$, für die $v_1\ne 0$ ist.
Der maximale Wert dafür wird erreicht, wenn man für 
$v$ einen Eigenvektor der Länge $1$ zum Eigenwert $\lambda_1$ einsetzt,
dann ist $v=v_1$.
Es folgt dann
\[
\pi(A)
=
\lim_{k\to\infty} \| A^k\|^\frac1k
=
\lim_{k\to\infty} |A^kv|^\frac1k
=
|\lambda_1|
=
\varrho(A).
\]
Damit ist gezeigt, dass im Spezialfall einer diagonalisierbaren Matrix der
Gelfand-Radius tatsächlich der Betrag des betragsgrössten Eigenwertes ist.

\subsubsection{Blockmatrizen}
Wir betrachten jetzt eine $(n+m)\times(n+m)$-Blockmatrix der Form
\begin{equation}
A = \begin{pmatrix} B & 0 \\ 0 & C\end{pmatrix}
\label{buch:spektralradius:eqn:blockmatrix}
\end{equation}
mit einer $n\times n$-Matrix $B$ und einer $m\times m$-Matrix $C$.
Ihre Potenzen haben ebenfalls Blockform:
\[
A^k = \begin{pmatrix} B^k & 0 \\ 0 & C^k\end{pmatrix}.
\]
Ein Vektor $v$ kann in die zwei Summanden $v_1$ bestehen aus den
ersten $n$ Komponenten und $v_2$ bestehen aus den letzten $m$ 
Komponenten zerlegen.
Dann ist
\[
A^kv = B^kv_1 + C^kv_2.
\qquad\Rightarrow\qquad
|A^kv|
\le
|B^kv_1| + |C^kv_2|
\le 
\pi(B)^k |v_1| + \pi(C)^k |v_2|.
\]
Insbesondere haben wir das folgende Lemma gezeigt:

\begin{lemma}
\label{buch:spektralradius:lemma:diagonalbloecke}
Eine diagonale Blockmatrix $A$ \eqref{buch:spektralradius:eqn:blockmatrix}
Blöcken $B$ und $C$  hat Gelfand-Radius
\[
\pi(A) = \max ( \pi(B), \pi(C) )
\]
\end{lemma}

Selbstverständlich lässt sich das Lemma auf Blockmatrizen mit beliebig
vielen diagonalen Blöcken verallgemeinern.

Für Diagonalmatrizen der genannten Art sind aber auch die 
Eigenwerte leicht zu bestimmen.
Hat $B$ die Eigenwerte $\lambda_i^{(B)}$ mit $1\le i\le n$ und $C$ die
Eigenwerte $\lambda_j^{(C)}$ mit $1\le j\le m$, dann ist das charakteristische
Polynom der Blockmatrix $A$ natürlich
\[
\chi_A(\lambda) = \chi_B(\lambda)\chi_C(\lambda),
\]
woraus folgt, dass die Eigenwerte von $A$ die Vereinigung der Eigenwerte
von $B$ und $C$ sind.
Daher gilt auch für die Spektralradius die Formel
\[
\varrho(A) = \max(\varrho(B) , \varrho(C)).
\]

\subsubsection{Jordan-Blöcke}
Nicht jede Matrix ist diagonalisierbar, die bekanntesten Beispiele sind
die Matrizen
\begin{equation}
J_n(\lambda)
=
\begin{pmatrix}
\lambda &      1&       &       &       &       \\
        &\lambda&      1&       &       &       \\[-5pt]
        &       &\lambda&\ddots &       &       \\[-5pt]
        &       &       &\ddots &      1&       \\
        &       &       &       &\lambda&      1\\
        &       &       &       &       &\lambda
\end{pmatrix},
\label{buch:spektralradius:eqn:jordan}
\end{equation}
wobei $\lambda\in\mathbb C$ eine beliebige komplexe Zahl ist.
Wir nennen diese Matrizen {\em Jordan-Matrizen}.
Es ist klar, dass $J_n(\lambda)$ nur den $n$-fachen Eigenwert
$\lambda$ hat und dass der erste Standardbasisvektor ein
Eigenvektor zu diesem Eigenwert ist.

In der linearen Algebra lernt man, dass jede Matrix durch Wahl
einer geeigneten Basis als
Blockmatrix der Form
\[
A
=
\begin{pmatrix}
J_{n_1}(\lambda_1) &        0         & \dots & 0 \\
       0         & J_{n_2}(\lambda_2) & \dots & 0 \\[-4pt]
\vdots           &\vdots            &\ddots &\vdots \\
       0         &        0         & \dots &J_{n_l}(\lambda_l)
\end{pmatrix}
\]
geschrieben werden kann.
Die früheren Beobachtungen über den Spektralradius und den
Gelfand-Radius von Blockmatrizen zeigen uns daher, dass
nur gezeigt werden muss, dass nur die Gleichheit des Gelfand-Radius
und des Spektral-Radius von Jordan-Blöcken gezeigt werden muss.

\subsubsection{Iterationsfolgen}
\begin{satz}
\label{buch:spektralradius:satz:grenzwert}
Sei $A$ eine $n\times n$-Matrix mit Spektralradius $\varrho(A)$.
Dann ist $\varrho(A)<1$ genau dann, wenn
\[
\lim_{k\to\infty} A^k = 0.
\]
Ist andererseits $\varrho(A) > 1$, dann ist
\[
\lim_{k\to\infty} \|A^k\|=\infty.
\]
\end{satz}

\begin{proof}[Beweis]
Wie bereits angedeutet reicht es, diese Aussagen für einen einzelnen
Jordan-Block mit Eigenwert $\lambda$ zu beweisen.
Die $k$-te Potenz von $J_n(\lambda)$ ist
\[
J_n(\lambda)^k
=
\begin{pmatrix}
\lambda^k    & \binom{k}{1} \lambda^{k-1} & \binom{k}{2}\lambda^{k-2}&\dots&
\binom{k}{n-1}\lambda^{k-n+1}\\
      0      &\lambda^k & \binom{k}{1} \lambda^{k-1} & \dots &\binom{k}{n-2}\lambda^{k-n+2}\\
      0     &      0    & \lambda^k & \dots &\binom{k}{n-k+3}\lambda^{k-n+3}\\
\vdots      & \vdots    &               &\ddots & \\
     0      &      0    &      0        &\dots  &\lambda^k
\end{pmatrix}.
\]
Falls $|\lambda| < 1$ ist, gehen alle Potenzen von $\lambda$ exponentiell
schnell gegen $0$, während die Binomialkoeffizienten nur polynomiell
schnell anwachsen. 
In diesem Fall folgt also $J_n(\lambda)\to 0$.

Falls $|\lambda| >1$ divergieren bereits die Elemente auf der Diagonalen,
also ist $\|J_n(\lambda)^k\|\to\infty$ mit welcher Norm auch immer man
man die Matrix misst.
\end{proof}

Aus dem Beweis kann man noch mehr ablesen.
Für $\varrho(A)< 1$ ist die Norm $ \|A^k\| \le M \varrho(A)^k$ für eine
geeignete Konstante $M$,
für $\varrho(A) > 1$ gibt es eine Konstante $m$ mit
$\|A^k\| \ge m\varrho(A)^k$.

\subsubsection{Der Satz von Gelfand}
Der Satz von Gelfand ergibt sich jetzt als direkte Folge aus dem
Satz~\ref{buch:spektralradius:satz:grenzwert}.

\begin{satz}[Gelfand]
Für jede komplexe $n\times n$-Matrix $A$ gilt
\[
\pi(A)
=
\lim_{k\to\infty}\|A^k\|^\frac1k
=
\varrho(A).
\]
\end{satz}

\begin{proof}[Beweis]
Der Satz~\ref{buch:spektralradius:grenzwert} zeigt, dass der
Spektralradius ein scharfes Kriterium dafür ist, ob $\|A^k\|$ 
gegen 0 oder $\infty$ konvergiert.
Andererseits ändert ein Faktor $tA$ den Spektralradius ebenfalls um
den gleichen Faktor, also $\varrho(tA)=t\varrho(A)$.
Natürlich gilt auch
\[
\pi(tA)
=
\lim_{k\to\infty} \|t^kA^k\|^\frac1k
=
\lim_{k\to\infty} t\|A^k\|^\frac1k
=
t\lim_{k\to\infty} \|A^k\|^\frac1k
=
t\pi(A).
\]

Wir betrachten jetzt die Matrix
\[
A(\varepsilon) = \frac{A}{\varrho(A) + \varepsilon}.
\]
Der Spektralradius von $A(\varepsilon)$ ist
\[
\varrho(A(\varepsilon)) = \frac{\varrho(A)}{\varrho(A)+\varepsilon},
\]
er ist also $>1$ für negatives $\varepsilon$ und $<1$ für positives
$\varepsilon$.
Aus dem Satz~\ref{buch:spektralradius:grenzwert} liest man daher ab,
dass $\|A(\varepsilon)^k\|$ genau dann gegen $0$ konvergiert, wenn
$\varepsilon > 0$ ist und divergiert genau dann, wenn $\varepsilon< 0$ ist.

Aus der Bemerkung nach dem Beweis von
Satz~\ref{buch:spektralradius:grenzwert} schliesst man daher, dass 
es im Fall $\varepsilon > 0$ eine Konstante $M$ gibt mit
\begin{align*}
\|A(\varepsilon) ^k\|\le M\varrho(\varepsilon)^k
\quad&\Rightarrow\quad
\|A(\varepsilon) ^k\|^\frac1k\le M^\frac1k\varrho(\varepsilon)
\\
&\Rightarrow\quad
\pi(A) \le  \varrho(A(\varepsilon))
\underbrace{\lim_{k\to\infty} M^\frac1k}_{\displaystyle=1}
=
\varrho(A(\varepsilon))
=
\varrho(A)+\varepsilon.
\end{align*}
Dies gilt für beliebige $\varepsilon >0$, es folgt daher
$\pi(A) \le \varrho(A)$.

Andererseits gibt es für $\varepsilon <0$ eine Konstante $m$ mit
\begin{align*}
\|A(\varepsilon) ^k\|\ge m\varrho(\varepsilon)^k
\quad&\Rightarrow\quad
\|A(\varepsilon) ^k\|^\frac1k\ge m^\frac1k\varrho(\varepsilon)
\\
&\Rightarrow\quad
\pi(A) \ge  \varrho(A(\varepsilon))
\underbrace{\lim_{k\to\infty} m^\frac1k}_{\displaystyle=1}
=
\varrho(A(\varepsilon))
=
\varrho(A)+\varepsilon.
\end{align*}
Dies gilt für beliebige $\pi(A) \ge \varrho(A)$.
Zusammen mit $\pi(A) \le \varrho(A)$ folgt $\pi(A)=\varrho(A)$.
\end{proof}

%
% Konditionszahl
%
\subsection{Konditionszahl
\label{buch:subsection:konditionszahl}}
Der Spektralradius $\varrho(A)$ einer Matrix $A$ gibt darüber Auskunft,
ob ein auf $A$ basierendes Iterationsverfahren konvergiert.
Er sagt aber nicht einmal, ob die Matrix zum Beispiel regulär.
Doch auch diese Information lässt sich aus den Eigenwerten ablesen.
Die Matrix ist genau dann regulär, wenn alle Eigenwerte von $0$ verschieden
sind.
Ein Problem entsteht also dann, wenn einzelne Eigenwerte sehr klein sind.
Die Kombination besonders grosser und besonders kleiner Eigenwerte 
ist also ein Indikator, der auf mögliche numerische Probleme hinweisen kann.

Die Eigenwerte von $A^{-1}$ sind die Reziproken der Eigenwerte von $A$.
Ist $\lambda$ ein Eigenwert von $A$, dann ist $\lambda^{-1}$ ein Eigenwert
von $A^{-1}$.
Der betragskleinste Eigenwert von $A$ ist der betragsgrösste Eigenwert von
$A^{-1}$.
Numerische Probleme werden also dadurch angezeigt, dass $\varrho(A)$ gross
ist, oder $\varrho(A^{-1})$ klein.

\begin{definition}
\label{buch:konditionszahl:definition}
Die Konditionszahl einer Matrix $A$ ist der Quotient
\[
\kappa(A)
=
\frac{\varrho(A)}{\varrho(A^{-1})}
=
\frac{|\lambda_1|}{|\lambda_n|},
\]
wenn $\lambda_1$ der betragsgrösste und $\lambda_n$ der betragskleinste
Eigenwert von $A$ ist.
\end{definition}

Die Konditionszahl ist also immer $\ge 1$, dieser Wert wird zum Beispiel
für die Einheitsmatrix erreicht.
Schlechte Kondition tritt auf, wenn die Eigenwerte sehr grosse
Betragsunterschiede aufweisen.

\begin{beispiel}
Die Kahan-Matrix
\[
A
=
\begin{pmatrix}
1000&999\\
999&998
\end{pmatrix}
\]
besteht aus zwei fast gleichen Zeilen, sie ist als fast singulär,
die Determinante muss sehr klein sein.
Andererseits sind alle Einträge von der Grössenordnung $10^3$, man erwartet
also einen Spektralradius in der selben Grössenordnung.
Man erwartet daher eine Konditionszahl von weit über $10^3$.
Die numerische Rechnung ergibt für die Eigenwerte
\[
\lambda_1 = 1998.00050050375,
\qquad
\lambda_2 = -0.000500500375,
\]
was eine Konditionszahl von
\[
\kappa(A)
\approx
-3992006
\]
ergibt.
Diese einfache Matrix hat also sehr schlechte Kondition.

Die Determinante von $A$ ist
\[
\det A
=
1000\cdot 998 -999^2
=
-1.
\]
Damit kann man auch die Inverse algebraisch leicht angeben:
\[
A^{-1}
=
\frac{1}{\det(A)}
\begin{pmatrix} 998 & -999\\ -999 1000 \end{pmatrix}.
\]
Die numerische Rechnung offenbart aber die Schwierigkeiten, die die 
schlechte Kondition verursacht.
Das Produkt $AA^{-1}$ müsste die Einheitsmatrix ergeben, die numerische
Rechnung mit dem \texttt{double}-Typ ergibt jedoch
\[
\begin{pmatrix}
   1.0000000000{\color{red}74920} &-0.0000000000{\color{red}52182}\\
   0.00000000000{\color{red}3410} &\phantom{-}1.0000000000{\color{red}20236}
\end{pmatrix},
\]
der numerische Fehler ist also etwa um den Faktor $10^4$ grösser als
bei einer gut konditionierten Matrix erwartet werden kann.
\end{beispiel}


