%
% gausseidel.tex
%
% (c) 2020 Prof Dr Andreas Müller, Hochschule Rapperswil
%
\section{Gauss-Seidel Iteration
\label{buch:section:gaussseidel}}
Gegeben ist eine lineares Gleichungssystem von $n$ Gleichungen mit
$n$ Unbekannten, welches wir als
\begin{equation}
\begin{linsys}{5}
a_{11}x_1 &+& a_{12}x_2 &+& \dots  \hspace*{7pt}&+& a_{1n}x_n &=& b_1 \\
a_{21}x_1 &+& a_{22}x_2 &+& \dots  \hspace*{7pt}&+& a_{2n}x_n &=& b_2 \\
\vdots\hspace*{5pt}  & & \vdots\hspace*{5pt}  & & \ddots \hspace*{7pt}& & \vdots\hspace*{5pt}  & & \vdots\hspace*{5pt} \\
a_{21}x_1 &+& a_{n2}x_2 &+& \dots  \hspace*{7pt}&+& a_{nn}x_n &=& b_n
\end{linsys}
\label{buch:eqn:linalg:system}
\end{equation}
Abgekürzt wird das Gleichungssystem auch $Ax=b$ geschrieben, wobei $A=(a_{ij})$
die Koeffizientenmatrix ist, $x=(x_k)$ der Vektor der Unbekannten
und $b=(b_k)$ der Vektor der rechten Seiten.

%
% Iterative Lösung
%
\subsection{Iterative Lösung nach Gauss-Seidel
\label{buch:subsection:gauss-seidel}}
Jede der Gleichungen \eqref{buch:eqn:linalg:system} kann nach Variablen
aufgelöst werden, sofern der zugehörige Koeffizient von $0$ verschieden ist.
Gleichung $k$ in \eqref{buch:eqn:linalg:system} ist
\[
a_{k1}x_1 + a_{k2}x_2 + \dots + a_{kk}x_k + \dots a_{kn}x_n = b_k
\]
Aufgelöst nach $x_k$ ist dies
\[
x_k = \frac1{a_{kk}} (b_k - a_{k1}x_1 - a_{k2}x_2 - \dots - a_{kn}x_n),
\]
sofern $a_{kk}\ne 0$.
Diese Gleichung kann dazu verwendet werden, die Werte für die Unbekannten
zu verbessern.

Wir verwenden daher die Notation $x^{(m)}$ für die $m$-te Approximation
der Lösung.
Mit dieser Notation können wir die Iterations 

\begin{satz}[Gauss-Seidel-Iteration]
Unter geeigneten Voraussetzungen konvergiert die Folge $x^{(m)}$
definiert durch
\begin{equation}
x_k^{(m)}
=
b_k  - a_{k1}x_1^{(m)} - \dots - a_{k,k-1}x_{k-1}^{(m)}
- a_{k,k+1}x_{k+1}^{(m-1)} - \dots - a_{kn}x_n^{(m-1)}
\label{buch:eqn:gs:iteration}
\end{equation}
mit Startwert $x^{(0)}=0$
konvergiert gegen die Lösung $x$ des Gleichungssystems $Ax=b$.
\end{satz}

In Abschnitt~\ref{buch:subsection:konvergenzbedingung} werden die
Bedingungen genauer untersucht, die Konvergenz des Verfahrens gegen die
Lösung garantieren können.


\begin{beispiel}
Sei das Gleichungssystem gegeben durch
\begin{equation}
A=\begin{pmatrix}
2&1&1\\
1&3&1\\
1&1&4
\end{pmatrix}
\qquad\text{und}\qquad
b=\begin{pmatrix}
7\\6\\5
\end{pmatrix}.
\label{buch:eqn:gsbeispiel}
\end{equation}
Die Berechnung der Folge $x^{(m)}$ nach
\eqref{buch:eqn:gs:iteration}
liefert die Werte in Tabelle~\ref{buch:table:gaussseidelbeispiel}.
Die Konvergenz scheint linear zu sein.
\begin{table}
\centering
\begin{tabular}{|>{$}r<{$}|>{$}r<{$}>{$}r<{$}>{$}r<{$}|}
\hline
 m & x_1^{(m)} & x_2^{(m)} & x_3^{(m)} \\
\hline
 0 & 0.0000000             & 0.0000000             & 0.0000000             \\
 1 & 3.5000000             & 0.8333333             & 0.1666667             \\
 2 & 3.0000000             & 0.9444444             & 0.2638889             \\
 3 & \underline{2.8}958333 & \underline{0.94}67593 & \underline{0.2}893519 \\
 4 & \underline{2.88}19444 & \underline{0.94}29012 & \underline{0.29}37886 \\
 5 & \underline{2.88}16552 & \underline{0.941}5187 & \underline{0.294}2065 \\
 6 & \underline{2.882}1373 & \underline{0.941}2187 & \underline{0.2941}610 \\
 7 & \underline{2.8823}102 & \underline{0.94117}62 & \underline{0.2941}284 \\
 8 & \underline{2.8823}476 & \underline{0.94117}47 & \underline{0.29411}94 \\
 9 & \underline{2.882353}1 & \underline{0.94117}59 & \underline{0.294117}7 \\
10 & \underline{2.882353}1 & \underline{0.9411764} & \underline{0.2941176} \\
\hline
\infty&   2.8823529 & 0.9411764 & 0.2941176 \\
\hline
\end{tabular}
\caption{Lösung des Gleichungssystems mit Koeffizientenmatrix $A$ und
rechter Seite $b$ aus \eqref{buch:eqn:gsbeispiel} mit Hilfe des
Gauss-Seidel-Algorithmus.
In der letzten Zeile die exakten Resultate, erhalten mit dem
Gauss-Algorithmus.
\label{buch:table:gaussseidelbeispiel}}
\end{table}
\end{beispiel}


%
% Matrixformulierung
%
\subsection{Matrixformulierung
\label{buch:subsection:matrixformulierung}}
Die Iterationsformel~\eqref{buch:eqn:gs:iteration} verknüpft bei der
Berechnung von $x^{(m)}$ Komponenten von $x^{(m-1)}$ und $x^{(m)}$,
was es etwas schwieriger macht, die Iteration als Fixpunktiteration
der Form $x^{(m)} = Fx^{(m-1)}$ zu schreiben mit einer $n\times n$-Matrix $F$.
Um dies zu erreichen zerlegen wir die Matrix $A$ in drei Summanden
$A=L+D+U$, wobei $L$ eine untere Dreiecksmatrix mit Nullen auf der 
Diagonalen sein soll, $D$ eine Diagonalmatrix und $U$ eine obere
Dreiecksmatrix mit  Nullen auf der Diagonalen, also
\begin{align*}
L
&=
\begin{pmatrix}
0        &0        &0        &\dots   & 0         & 0      \\
a_{21}   &0        &0        &\dots   & 0         & 0      \\
a_{31}   &a_{3,2}  &0        &\dots   & 0         & 0      \\
\vdots   &\vdots   &\ddots   &\ddots  & \vdots    & \vdots \\
a_{n-1,1}&a_{n-1,2}&a_{n-1,3}&\dots   & 0         & 0      \\
a_{n1}   &a_{n2}   &a_{n3}   &\dots   & a_{n,n-1} & 0
\end{pmatrix},
\qquad
U
=
\begin{pmatrix}
0      & a_{12} & \dots  & a_{1,n-2} & a_{1,n-1}   & a_{1n} \\
0      & 0      & \dots  & a_{1,n-2} & a_{2,n-1}   & a_{2n} \\
\vdots & \vdots & \ddots & \ddots    &\vdots       & \vdots \\
0      & 0      & \dots  & 0         & a_{n-2,n-1} & a_{n-2,n} \\
0      & 0      & \dots  & 0         & 0           & a_{n-1,n} \\
0      & 0      & \dots  & 0         & 0           & 0
\end{pmatrix}
\\
D
&=
\operatorname{diag} (a_{11}, a_{22},\dots , a_{n-1,n-1}, a_{nn})
\end{align*}
Die Iterationsformel~\eqref{buch:eqn:gs:iteration} lässt sich
mit diesen Matrizen schreiben als
\[
Dx^{(m)} = b - Lx^{(m)} - Ux^{(m-1)}.
\]
Auflösen nach $x^{(m)}$ führt auf
\begin{equation}
x^{(m)} = (D+L)^{-1} ( b - Ux^{(m-1)} ).
\label{buch:eqn:gs:fixpunkt}
\end{equation}
Die Form \eqref{buch:eqn:gs:fixpunkt} für das Gauss-Seidel-Iterationsverfahren
ist jetzt die einer Fixpunkt-Iteration.

%
% Konvergenzbedingung
%
\subsection{Konvergenzbedingung
\label{buch:subsection:konvergenzbedingung}}
In Kapitel~\ref{chapter:berechnung} haben wir gelernt, dass eine
Fixpunktiteration konvergiert, wenn der Betrag der Ableitung $<1$ ist.
Hier liegt jedoch eine Matrix-Iteration mit der Abbildung
\[
F(x)
=
\underbrace{(D+L)^{-1} b}_{\displaystyle=c} - (D+L)^{-1}U x
=
c - (D+L)^{-1}Ux
\]
vor.
Die Ableitung ist daher ebenfalls eine Matrix, nämlich
\[
D_xF = (D+L)^{-1}U,
\]
und der Fehler der Iteration $m$ ist
\begin{equation}
\delta_m = (D+L)^{-1}U \delta_{m-1}.
\label{buch:gs:fehler}
\end{equation}
Konvergenz kann also nur vorliegen, wenn dieser Vektor im Laufe der
Iteration immer kleiner wird.
Dies ist zum Beispiel dann der Fall, wenn die {\em Norm} der Matrix
kleiner als $1$ ist:

\begin{definition}
Die {\em Norm} einer Matrix $M$ ist
\[
\|M\|
=
\max\{|Mx|\,|\, x\in\mathbb R^n\wedge |x|=1\}.
\]
Für einen Vektor $x\in\mathbb R^n$ gilt $|Mx| \le \|M\|\cdot |x|$.
\end{definition}

Die Bedingung \eqref{buch:gs:fehler} bedeutet jedoch nicht,
dass die Norm der Ableitung $<1$ sein muss, es genügt, wenn
genügend hohe Potenzen der Ableitung eine Norm $<1$ haben.

\begin{beispiel}
Die Matrix
\[
M=\begin{pmatrix}
0&2\\
\frac13&0
\end{pmatrix}
\]
hat Norm
\[
\|M\|
=
\max_{|x|=1} |Mx| 
=
\max_{t\in\mathbb R} \sqrt{2^2\cos^2 t +\frac1{3^2}\sin^2t} \ge 2.
\]
Da aber
\[
M^2 = \begin{pmatrix}
\frac{2}{3}&0\\
0&\frac{2}{3}
\end{pmatrix}
\qquad\Rightarrow\qquad \|M^2\|=\frac23
\]
ist, wird eine Iteration mit Ableitungsmatrix $M$ trotzdem
konvergieren, weil der Fehler nach jedem zweiten Schritt um den
Faktor $\frac23$ kleiner geworden ist.
\end{beispiel}

Dies führt uns auf das Konzept des Spektralradius

\begin{definition}
Der {\em Spektralradius} der Matrix $M$ ist
\[
\varrho(M)
=
\limsup_{n\to\infty} \|A^n\|^\frac1n.
\]
\end{definition}

Das Gauss-Seidel-Iterationsverfahren ist also genau dann für alle
Startwerte $x_0$ linear konvergent, wenn der Spektralradius
\[
\varrho\bigl( (L+D)^{-1}U \bigr) < 1
\]
ist.


