%
% bandmatrix.tex -- Bandmatrizen
%
% (c) 2020 Prof Dr Andreas Müller, Hochschule Rapperswil
%
\documentclass[tikz]{standalone}
\usepackage{amsmath}
\usepackage{times}
\usepackage{txfonts}
\usepackage{pgfplots}
\usepackage{csvsimple}
\usetikzlibrary{arrows,intersections,math}
\begin{document}
\def\skala{1}
\begin{tikzpicture}[>=latex,thick,scale=\skala]

\fill[color=gray!70] (-3,-3) rectangle (3,3);
\fill[color=gray!25] (-3,1.5) -- (1.5,-3) -- (-3,-3) -- cycle;
\draw (-3,1.5) -- (1.5,-3);
\fill[color=gray!25] (3,-1.5) -- (-1.5,3) -- (3,3) -- cycle;
\draw (3,-1.5) -- (-1.5,3);
\draw (-3,-3) rectangle (3,3);

\draw[<->] (-3,-3.3) -- (3,-3.3);
\draw[line width=0.3pt] (-3,-3) -- (-3,-3.4);
\draw[line width=0.3pt] (3,-3) -- (3,-3.4);

\node at (3.3,0) [right] {$n$};

\draw[<->] (3.3,-3) -- (3.3,3);
\draw[line width=0.3pt] (3,-3) -- (3.4,-3);
\draw[line width=0.3pt] (3,3) -- (3.4,3);

\node at (0,-3.3) [below] {$n$};

\draw[<->] (-3,3.3) -- (-1.5,3.3);
\draw[line width=0.3pt] (-3,3) -- (-3,3.4);
\draw[line width=0.3pt] (-1.5,3) -- (-1.5,3.4);
\node at (-2.25,3.3) [above] {$k$};

\draw[<->] (-3.3,3) -- (-3.3,1.5);
\draw[line width=0.3pt] (-3,3) -- (-3.4,3);
\draw[line width=0.3pt] (-3,1.5) -- (-3.4,1.5);
\node at (-3.3,2.25) [left] {$k$};

\node at (1.7,1.7) {\Huge $0$};
\node at (-1.7,-1.7) {\Huge $0$};

\end{tikzpicture}
\end{document}

