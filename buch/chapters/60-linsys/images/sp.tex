%
% sp.tex -- template for standalon tikz images
%
% (c) 2020 Prof Dr Andreas Müller, Hochschule Rapperswil
%
\documentclass[tikz]{standalone}
\usepackage{amsmath}
\usepackage{times}
\usepackage{txfonts}
\usepackage{pgfplots}
\usepackage{csvsimple}
\usetikzlibrary{arrows,intersections,math}
\begin{document}
\def\skala{2.33}
\input{sppaths.tex}
\begin{tikzpicture}[>=latex,thick,scale=\skala]

\xdef\sx{1}
\xdef\sy{1}

% add image content here
\begin{scope}
\begin{scope}
\clip (0,0) rectangle (3,2.1);
\richardson
\end{scope}
\draw[->] (-0.1,0) -- (3.15,0) coordinate[label={$\tau$}];
\draw[->] (0,-0.1) -- (0,2.3) coordinate[label={right:$\varrho(\frac1\tau A-E)$}];
\draw (1,{-0.1/\skala}) -- (1,{0.1/\skala});
\draw (2,{-0.1/\skala}) -- (2,{0.1/\skala});
\draw (3,{-0.1/\skala}) -- (3,{0.1/\skala});
\node at (1,{-0.1/\skala}) [below] {1};
\node at (2,{-0.1/\skala}) [below] {2};
\node at (3,{-0.1/\skala}) [below] {3};
\draw ({-0.1/\skala},1) -- ({0.1/\skala},1);
\draw ({-0.1/\skala},2) -- ({0.1/\skala},2);
\node at ({-0.1/\skala},1) [left] {1};
\node at ({-0.1/\skala},2) [left] {2};
\end{scope}

\xdef\sy{1}

\begin{scope}[xshift=3.5cm]
\begin{scope}
\clip (0,0) rectangle (2,2);
\sor
\end{scope}
\draw[->] (-0.1,0) -- (2.15,0) coordinate[label={$\omega$}];
\draw[->] (0,-0.1) -- (0,2.3) coordinate[label={right:$\varrho(B_\omega^{-1}C_\omega)$}];
\draw (1,{-0.1/\skala}) -- (1,{0.1/\skala});
\draw (2,{-0.1/\skala}) -- (2,{0.1/\skala});
\node at (1,{-0.1/\skala}) [below] {1};
\node at (2,{-0.1/\skala}) [below] {2};
\draw ({-0.1/\skala},1) -- ({0.1/\skala},1);
\draw ({-0.1/\skala},2) -- ({0.1/\skala},2);
\node at ({-0.1/\skala},1) [left] {1};
\node at ({-0.1/\skala},2) [left] {2};
\end{scope}

\end{tikzpicture}
\end{document}

