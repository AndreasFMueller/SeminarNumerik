%
% qr.tex
%
% (c) 2020 Prof Dr Andreas Müller, Hochschule Rapperswil
%
\section{QR-Zerlegung mit Spiegelungen
\label{buch:section:qr}}
\rhead{QR-Zerlegung mit Spiegelungen}
\index{QR-Zerlegung}%
Das Orthonormalisierungsproblem verlangt, dass zu gegebenen Vektoren
$a_1,\dots,a_n\in \mathbb R^l$ orthonormierte Vektoren $q_1,\dots ,q_n$
gefunden werden derart, dass
\begin{equation}
a_k
=
\langle q_1,\dots,q_k \rangle
\qquad
\forall 1\le k\le n.
\label{buch:eqn:qr:span}
\end{equation}
Die Bedingung \eqref{buch:eqn:qr:span} bedeutet, dass es Zahlen $r_{ik}$
mit $i\le k$ derart, dass
\begin{equation}
a_k = q_1 r_{11} +  q_2 r_{12} + \dots + q_k r_{1k}
\label{buch:eqn:qr:span2}
\end{equation}
Schreiben wir die Komponenten des Vektors $a_k$ als $l\times n$-Matrix
$A$ mit Einträgen $a_{ik}$ und analog für die Vektoren $q_k$, dann kann
\eqref{buch:eqn:qr:span2} in Matrixform geschrieben werden als
\[
A=QR, \qquad\text{mit}\quad
R
=
\begin{pmatrix}
r_{11}&r_{12}&\dots &r_{1n}\\
0     &r_{22}&\dots &r_{2n}\\
0     &0     &\dots &r_{3n}\\
\vdots&\vdots&\ddots&\vdots\\
\end{pmatrix}
\]
Darin ist $Q$ eine $l\times n$-Matrix und $R$ ist eine $n\times n$-Matrix.
Die Aufgabe, eine Menge von Vektoren $\{a_1,\dots,a_n\}$ zu orthonormieren,
ist also gleichbedeutend damit, für die Matrix $A$ eine Zerlegung $A=QR$
zu finden, wobei $Q$ orthogonal und $R$ eine obere Dreiecksmatrix sein soll.
\index{Dreiecksmatrix}%

%
% Gram-Schmidt-Orthonormalisierung
%
\subsection{Gram-Schmidt-Orthonormalisierung
\label{buch:subsection:gram-schmidt}}
Das einfachste und anschaulichste Orthonormalisierungsverfahren, der
Gram-Schmidt-Prozess verwendet die Formeln
\index{Gram-Schmidt-Prozess}%
\begin{align*}
q_1 & = \frac{a_1}{|a_1|}
\\
q_2 &= \frac{
a_2 - (q_1\cdot a_2) q_1
}{
|a_2 - (q_1\cdot a_2) q_1|
}
\\
q_3 &= \frac{
a_3 - (q_1\cdot a_3) q_1 - (q_2\cdot a_3) q_2
}{
|a_3 - (q_1\cdot a_3) q_1 - (q_2\cdot a_3) q_2|
}
\\
&\;\vdots\\
q_k
&=\frac{
a_k - (q_1\cdot a_k) q_1 -\dots- (q_{k-1}\cdot a_k) q_{k-1}
}{|
a_k - (q_1\cdot a_k) q_1 -\dots- (q_{k-1}\cdot a_k) q_{k-1}
|},
\end{align*}
um die orthonormierten Vektoren $q_k$ zu finden.
\index{orthonormiert}%
Wegen der Differenzen im Zähler und Nenner besteht die Gefahr von
Auslöschung, falls der Winkel zwischen $a_k$ und
$\langle a_1,\dots,a_{k-1}\rangle$ sehr klein ist.
\index{Auslöschung}%
Der Gram-Schmidt-Prozess ist daher in seiner Grundform nicht immer stabil.
In Kapitel~\ref{chapter:qr} wird an einem Simulationsbeispiel gezeigt,
wie sich die Instabilität des Gram-Schmidt-Algorithmus äusseren kann.

\subsection{Spiegelungen
\label{buch:subsection:spiegelungn}}
Mit dem Skalarprodukt kann zu jedem Paar $a$, $b$ von Vektoren 
gleicher Länge eine Matrix gefunden werden, welche eine Spiegelung
beschreibt, die $a$ auf $b$ abbildet, alle Vektoren orthogonal zu
$a$ und $b$ aber unverändert lässt.
\index{Skalarprodukt}%
\index{Spiegelung}%

Sei $n=(b-a)/|b-a|$ der Einheitsvektor in Richtung $b-a$.
Es ist $(a-b)\cdot (a+b) = |a|^2 - |b|^2$, also auch $n\cdot (a+b)=0$.
\index{Einheitsvektor}%

Die Abbildung
\[
s:
x\mapsto  x - 2 n(n\cdot x) 
\]
hält auf $n$ orthogonale Vektoren $x$ wegen $n\cdot x=0$ fest.
Für die Vektoren $a$ und $b$ ist $(b-a)\cdot a= -(b-a)\cdot b$, so
dass
\begin{align*}
a
&=
\underbrace{\frac12(a+b)}_{\displaystyle =u}
+
\underbrace{\frac12(a-b)}_{\displaystyle=v}
=
u+v
\\
b&= \frac12(a+b) - \frac12(a-b) = u - v
\end{align*}
gilt mit orthogonalen Vektoren $u=\frac12(a+b)$ und $v=\frac12(a-b)$.
Für die beiden Summanden gilt
\begin{align*}
s(u) &= u,\\
s(v) &= -v
\end{align*}
und daher
\begin{align*}
s(a)
&=
s(u+v)=s(u)+s(v)=u-v = b,
\\
s(b)
&=
s(u-v)=s(u)-s(v)=u+v = a.
\end{align*}
Die Abbildung $s$ vertauscht also die beiden Vektoren, sie ist eine
Spiegelung in der von $a$ und $b$ aufgespannten Ebene an einer
Geraden mit der Normalen $n$.

Der Vektor $n$ ermöglicht auch, die lineare Abbildung $s$ mit Hilfe
einer Matrix $s$ zu schreiben.
Das Skalarprodukt $n\cdot x$ ist das Matrixprodukt $n^t x$, der Vektor
$n(n\cdot x)$ als Matrixprodukt $nn^tx$ berechnet werden. 
Damit wird 
\[
s(x)
=
x - 2n(n\cdot x)
=
Ex - 2(nn^t) x
=
(E-2nn^t) x
\qquad\Rightarrow\qquad
S=E-2nn^t
\]
die Matrix der Abbildung $s$.
Wir haben damit den folgenden Satz bewiesen.

\begin{satz}
\label{buch:satz:sab}
Zu gegebenen Vektoren $a$ und $b$ gleicher, nicht verschwindender
Länge gibt es eine orthogonale Matrix
\[
S_{a,b} = E-2nn^t\qquad \text{mit}\quad n = (b-a)/|b-a|,
\]
die zu einer Spiegelung gehört, die $a$ auf $b$ abbildet und
umgekehrt  und ausserdem alle Vektoren senkrecht auf $a$ und $b$ fest lässt.
\end{satz}
\index{Spiegelungsmatrix}%

%
% QR-Zerlegung mit Spiegelungen
%
\subsection{QR-Zerlegung mit Spiegelungen
\label{buch:subsection:qrspiegelungen}}
Aus der QR-Zerlegung $A=QR$ erhält man durch Multiplikation mit $Q^t$
die Gleichung $Q^tA=R$.
Die Aufgabe, die QR-Zerlegung zu finden, ist also gleichbedeutend damit,
eine Matrix durch Multiplizieren mit orthogonalen Matrizen auf Dreiecksform
zu bringen.
\index{QR-Zerlegung}%
Das Produkt der orthogonalen Matrizen ist $Q^t$.
In diesem Abschnitt wird ein Algorithmus basierend auf den Matrizen
$S_{a,b}$ vorgestellt, der genau dies leistet.

Sei $a\in\mathbb R^l$ ein nicht verschwindender Vektor.
Dann ist der Vektor $b=(|a|,0,\dots,0)^t$ ein Vektor gleicher Länge.
Nach Satz~\ref{buch:satz:sab} ist die Matrix $S_{a,b}$ eine Spiegelung,
die $a$ auf $b$ abbildet.

Sei jetzt ein Vektor $a\in\mathbb R^l$ gegeben und eine Zahl $k< l$.
Wir zerlegen den Vektor $a$ in zwei Teile
\[
a = \begin{pmatrix}\tilde{a}\\\bar{a}\end{pmatrix}
\quad\text{mit}\quad
\tilde{a}=\begin{pmatrix}a_1\\\vdots\\a_{k-1}\end{pmatrix}
\bar{a}=\begin{pmatrix}a_k\\\vdots \\a_l\end{pmatrix}.
\]
Die Matrix $Q_{\bar{a}}$ bildet $\bar{a}$ auf einen Vektor ab, von dem
nur die erste Komponente von $0$ verschieden ist.
Daraus wird die Matrix
\begin{equation}
Q_a
=
\begin{pmatrix} E&0\\0&Q_{\bar{a}}\end{pmatrix},
\label{buch:eqn:Qa}
\end{equation}
die den Vektor
\[
a=\begin{pmatrix}
a_1\\\vdots\\ a_{k-1}\\a_{k}\\\vdots\\a_n
\end{pmatrix}
\qquad\text{auf}\qquad
Q_aa
=
\begin{pmatrix}a_1\\\vdots\\a_{k-1}\\ |\bar{a}|\\\vdots\\0\end{pmatrix}
\]
abbildet.
$Q_a$ lässt Vektoren $x$, für die $\bar{x}=0$ ist, unverändert.

Daraus lässt sich jetzt ein QR-Algorithmus konstruieren.

\begin{satz}[QR-Zerlegung mit Spiegelungen]
Gegeben ist die $l\times n$-Matrix $A$ mit $n\le l$.
Dann gibt es orthogonale Matrizen $Q_1,\dots,Q_n$ derart, dass in
jeder Matrix der Folge
\[
A_0 = A, \quad A_{i} = Q_{i}A_{i-1}\quad\text{für $0<i\le n$}
\]
die ersten $i$ Spalten die Form einer oberen Dreicksmatrix haben.
\index{Dreiecksmatrix}%
Mit $Q=Q_1^t\dots Q_n^t$ und $R=A_n=Q_n\dots Q_1A$ gilt $A=QR$.
\end{satz}

\begin{proof}[Beweis]
Wir müssen die Matrix $Q_i$ aus $A_{i-1}$ konstruieren.
Sei $a$ die $i$-te Matrix von $A_{i-1}$, dann setzen wir
$Q_i=Q_{a}$ mit der Matrix $Q_a$ nach \eqref{buch:eqn:Qa}.
Die Multiplikation mit $Q_i$ bringt die Spalte $i$ der Matrix 
$A_{i-1}$ in die für $A_{i}$ verlangte Form, ohne die 
Spalten mit Spaltenindex $<i$ zu verändern, da deren Komponenten dort,
wo $Q_a$ etwas bewirkt, alle verschwinden.
\end{proof}

\begin{beispiel}
Die Vektoren
\[
a_1 = \begin{pmatrix}1\\1\\1\end{pmatrix},
\qquad
a_2 = \begin{pmatrix}0\\1\\1\end{pmatrix},
\qquad
a_3 = \begin{pmatrix}0\\0\\1\end{pmatrix}
\]
sollen orthonormalisiert werden.
Die Matrix $Q_1$ spiegelt den Vektor 
\[
\begin{pmatrix}
1\\1\\1
\end{pmatrix}
\quad\text{auf}\quad
\begin{pmatrix}\sqrt{3}\\0\\0\end{pmatrix}
\]
ab.
Die zugehörige Matrix ist
\[
Q_1
=
\begin{pmatrix}
\frac{1}{\sqrt{3}} & \frac{1}{\sqrt{3}}          & \frac{1}{\sqrt{3}} \\
\frac{1}{\sqrt{3}} & \frac12-\frac{1}{2\sqrt{3}} & -\frac12-\frac{1}{2\sqrt{3}}\\
\frac{1}{\sqrt{3}} & -\frac12-\frac{1}{2\sqrt{3}}&\frac12-\frac1{2\sqrt{3}}
\end{pmatrix}
\qquad
A_1
=
Q_1A
=
\begin{pmatrix}
\sqrt{3} &  \frac{2}{\sqrt{3}} & \frac{1}{\sqrt{3}} \\
  0      & -\frac{1}{\sqrt{3}} & -\frac12-\frac{1}{2\sqrt{3}} \\
  0      & -\frac{1}{\sqrt{3}} & \frac12-\frac{1}{2\sqrt{3}}
\end{pmatrix}.
\]
Die zweite Matrix $Q_2$ ist
\[
Q_2
=
\begin{pmatrix}
1&0&0\\
0&-\frac{1}{\sqrt{2}} & -\frac{1}{\sqrt{2}} \\
0&-\frac{1}{\sqrt{2}} &  \frac{1}{\sqrt{2}}
\end{pmatrix},
\qquad
R
=
A_2 = Q_2A_1
=
\begin{pmatrix}
\sqrt{3} & \frac{2}{\sqrt{3}} & \frac{1}{\sqrt{3}} \\
    0    & \sqrt{\frac{2}{3}} & \frac{1}{\sqrt{6}} \\
    0    &         0          & \frac{1}{\sqrt{2}}
\end{pmatrix}
\]
Wegen $R=(Q_2Q_1)A$ folgt $A=(Q_2Q_1)^t R$ und damit
\[
Q=(Q_2Q_1)^t
=
\begin{pmatrix}
\frac{1}{\sqrt{3}} & - \sqrt{\frac{2}{3}} & 0\\
\frac{1}{\sqrt{3}} & \frac{1}{\sqrt{6}}   & - \frac{1}{\sqrt{2}} \\
\frac{1}{\sqrt{3}} & \frac{1}{\sqrt{6}}   & \frac{1}{\sqrt{2}}
\end{pmatrix}.
\]
Damit ist die QR-Zerlegung der Matrix $A$ gefunden.
Tatsächlich bilden die Spalten von $Q$ eine orthonormierte Basis
von $\mathbb R^3$, sie stimmt mit der Basis überein, die der
Gram-Schmidt-Prozess aus den Spalten der Matrix $A$ ableitet.
\index{Gram-Schmidt-Prozess}%
\end{beispiel}

Der Nachteil dieser Methode ist, dass die Matrizen $S_{a,b}$ typischerweise
sehr viele Einträge haben und dass damit die Matrixprodukte mit $S_{a,b}$
aufwendig zu berechnen sind.
Dies kann durch die Verwendung von Drehmatrizen, sogenannten Givens-Rotationen,
verbessert werden, wie dies in Kapitel~\ref{chapter:qr} durchgeführt wird.
\index{Drehmatrix}%
\index{Givens-Rotation}%

