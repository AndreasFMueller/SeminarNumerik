%
% jacobi.tex
%
% (c) 2020 Prof Dr Andreas Müller, Hochschule Rapperswil
%
\section{Diagonalisierung mit dem Jacobi-Verfahren
\label{buch:section:jacobi}}
\rhead{Diagonalisierung mit dem Jacobi-Verfahren}
\index{Diagonalisierung}%
Die Diskretisierung linearer partieller Differentialgleichungen wie
zum Beispiel
der Wellengleichung führen immer auf symmetrische Eigenwertprobleme,
also auf Gleichungen der From $Av=\lambda v$ mit $A=A^t$.
\index{Diskretisation}%
Aus der
linearen Algebra ist bekannt, dass es in diesem Fall eine orthogonale
Matrix $O$ gibt mit
\index{Matrix!orthogonal}%
\index{orthogonale Matrix}%
\[
\begin{pmatrix}
\lambda_1&\dots&0\\
\vdots&\ddots&\vdots\\
0&\dots&\lambda_n
\end{pmatrix}=O^tAO
\]
Die Matrix $O^tAO$ hat also die Basisvektoren $e_i=(0,\dots,1,\dots,0)^t$
als Eigenvektoren. $O$ bildet $e_i$ auf den $i$-ten Eigenvektor von $A$
ab, in der $i$-ten Spalten von $O$ steht der $i$-te Eigenvektor von $A$.
Findet man $O$, kann man daraus die Eigenvektoren ablesen.

Das Jacobi-Verfahren versucht, $O$ als Zusammensetzung von Drehungen
in jeweils zwei Dimensionen aufzubauen.
Gleichzeitig wird die
Matrix $A$ ``in place'' auf Diagonalform reduziert, so dass man dort
die Eigenwerte ablesen kann.
\index{Diagonalform}%
\index{Drehung}%

\subsection{Jacobi-Verfahren in zwei Dimensionen\label{jacobi2d}}
In zwei Dimensionen hat eine orthogonale Matrix $O$ immer die
Form
\[
O=\begin{pmatrix}\cos\alpha&\sin\alpha\\-\sin\alpha&\cos\alpha\end{pmatrix}.
\]
Die symmetrische Matrix
\[
A=\begin{pmatrix}a_{11}&a_{12}\\a_{21}&a_{22}\end{pmatrix}
\]
soll damit auf Diagonalform gebracht werden.
In
\[
O^tAO
=
\begin{pmatrix}
\cos\alpha&-\sin\alpha\\
\sin\alpha& \cos\alpha
\end{pmatrix}
\begin{pmatrix}
a_{11}&a_{12}\\
a_{21}&a_{22}
\end{pmatrix}
\begin{pmatrix}
 \cos\alpha&\sin\alpha\\
-\sin\alpha&\cos\alpha
\end{pmatrix}
\]
müssen die Elemente ausserhalb der Diagonalen zu $0$ werden, dann
stehen auf der Diagonalen die gesuchten Eigenwerte.
Durch Nachrechnen findet man für $\alpha$ die Bedingung
\begin{align*}
0&=
a_{11}\sin\alpha\cos\alpha +a_{12}(\cos^2\alpha-\sin^2\alpha)
-a_{22}\sin\alpha\cos\alpha
\\
&=
(a_{11}-a_{22})
\frac12\sin2\alpha+a_{12}\cos2\alpha\\
\cot2\alpha&=\frac{a_{22}-a_{11}}{2a_{12}}.
\end{align*}
Mit Hilfe der goniometrischen Beziehung
\[
\vartheta=\cot2\alpha = \frac{1-\tan^2\alpha}{2\tan\alpha}
\]
kann man sie als quadratische Gleichung für $\tan\alpha$ betrachten, nämlich
\[
\tan^2\alpha+2\vartheta\tan\alpha-1=0,
\]
welche die Lösungen
\[
\tan\alpha=\vartheta\pm\sqrt{\vartheta^2+1}
\]
hat. In der Matrix $O$ werden nur Sinus und Cosinus benötigt, diese
kann man durch algebraische Ausdrücke berechnen:
\begin{align}
\cos\alpha&=\frac1{\sqrt{1+\tan^2\alpha}}\label{cosalpha}\\
\sin\alpha&=\frac{\tan\alpha}{\sqrt{1+\tan^2\alpha}}\label{sinalpha}
\end{align}
Insbesondere sind keine numerisch aufwendigen trigonometrischen Operationen
notwendig, ausser der Wurzel in (\ref{cosalpha}) und (\ref{sinalpha})
sind alle Schritte mit den Grundoperationen durchführbar.

\subsection{Beliebige Dimension}
Für $n>2$ lässt sich die Reduktion auf Diagonalform nicht mehr in einem
Schritt durchführen. Der folgende Algorithmus führt jedoch zum Erfolg.
\begin{enumerate}
\item Initialisiere die Matrix $O$ als Einheitsmatrix: $O=I$
\item \label{loop} Für jedes Paar von Indizes $(p,q)$ mit $q>p$ führe
die folgenden zwei Schritte aus.
\item Finde eine Matrix
\[
O_{pq}
=
\begin{pmatrix}
 \cos\alpha&\sin\alpha\\
-\sin\alpha&\cos\alpha
\end{pmatrix}
\]
wie in Abschnitt \ref{jacobi2d}, welche die Teilmatrix
\[
A_{pq}=\begin{pmatrix}a_{pp}&a_{pq}\\a_{qp}&a_{qq}\end{pmatrix}
\]
auf Diagonalform bringt: $O_{pq}^tA_{pq}O_{pq}$.
\item Bilde die Matrix
\[
O'=\begin{pmatrix}
1 &\dots &0         &\dots &0          &      &0\\
  &\ddots&\vdots    &      &\vdots     &      &\vdots\\
0 &      &\cos\alpha&\dots &\sin\alpha&\dots &0\\
  &      &\vdots    &\ddots&\vdots     &      &\vdots\\
0 &      &-\sin\alpha&\dots & \cos\alpha&\dots &0\\
  &      &\vdots    &      & \vdots    &\ddots& \\
0 &\dots &0         &      & 0         &      &1
\end{pmatrix},
\]
wobei die trigonometrischen Funktionen in den Zeilen und Spalten mit
Indizes $p$ und $q$ stehen.
\item Setzte $A:=O'^tAO'$ und $O:=OO'$.
\item Wiederhole das Verfahren ab Schritt \ref{loop}, falls noch Indexpaare
$(p,q)$ mit $|a_{pq}|>\varepsilon$ vorkommen.
\end{enumerate}
Am Ende dieses Verfahrens steht in $A$ die Diagonalmatrix mit den
Eigenwerten, in $O$ steht die orthogonale Matrix, die $A$ auf
Diagonalform gebracht, sie enthält die Eigenvektoren in den
Spalten.

Dieses Verfahren ist nur auf symmetrische Matrizen anwendbar.
\index{symmetrisch}%
In Kapitel~\ref{chapter:francis} wird gezeigt, wie der
Francis-Algorithmus die Ideen der Abschnitte~\ref{buch:section:qr}
und \ref{buch:section:jacobi} zu einem funktionierenden
Eigenwert-Algorithmus kombiniert, der für beliebige Matrizen funktioniert.
\index{Francis-Algorithmus}



