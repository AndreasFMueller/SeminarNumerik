Betrachten Sie die Matrix $A$ und den Vektor $b$
\[
A=\begin{pmatrix}
   3&  0&  2 \\
   5&  1& -1 \\
   9&  1&  2
\end{pmatrix},
\qquad\text{und}\qquad
b
=
\begin{pmatrix}
11\\16\\31
\end{pmatrix}
\]
\begin{teilaufgaben}
\item
Finden Sie die Lösung der Gleichung $Ax=b$.
\item
Ist das Jacobi-Verfahren für diese Matrix konvergent?
\item
Ist das Gauss-Seidel-Verfahren für diese Matrix konvergent?
\item
Verifizieren Sie Ihre Resultate von b) und c) numerisch.
\end{teilaufgaben}


\begin{loesung}
\def\tabelleninhalt{
   & 0.0379 & 0.3621 & 0.0699 & 3.0001 & 2.0001 & 1.0001 \\
\hline
1 & 3.6201 & -2.0306 & 0.2249 & 2.9999 & 1.9996 & 0.9995 \\
2 & 3.5167 & -1.3588 & 0.3541 & 3.0003 & 1.9998 & 1.0005 \\
3 & 3.4306 & -0.7990 & 0.4617 & 2.9997 & 1.9988 & 0.9986 \\
4 & 3.3589 & -0.3325 & 0.5514 & 3.0009 & 2.0002 & 1.0021 \\
5 & 3.2990 & 0.0562 & 0.6262 & 2.9986 & 1.9974 & 0.9956 \\
6 & 3.2492 & 0.3802 & 0.6885 & 3.0029 & 2.0026 & 1.0076 \\
7 & 3.2077 & 0.6502 & 0.7404 & 2.9950 & 1.9930 & 0.9856 \\
8 & 3.1731 & 0.8751 & 0.7837 & 3.0096 & 2.0108 & 1.0262 \\
9 & 3.1442 & 1.0626 & 0.8197 & 2.9825 & 1.9782 & 0.9514 \\
10 & 3.1202 & 1.2188 & 0.8498 & 3.0324 & 2.0387 & 1.0896 \\
11 & 3.1001 & 1.3490 & 0.8748 & 2.9403 & 1.9274 & 0.8347 \\
12 & 3.0835 & 1.4575 & 0.8957 & 3.1102 & 2.1332 & 1.3050 \\
13 & 3.0695 & 1.5479 & 0.9131 & 2.7967 & 1.7540 & 0.4375 \\
14 & 3.0580 & 1.6233 & 0.9276 & 3.3750 & 2.4540 & 2.0379 \\
15 & 3.0483 & 1.6861 & 0.9396 & 2.3081 & 1.1629 & -0.9145 \\
16 & 3.0402 & 1.7384 & 0.9497 & 4.2763 & 3.5451 & 4.5322 \\
17 & 3.0335 & 1.7820 & 0.9581 & 0.6452 & -0.8494 & -5.5160 \\
18 & 3.0279 & 1.8183 & 0.9651 & 7.3440 & 7.2579 & 13.0212 \\
19 & 3.0233 & 1.8486 & 0.9709 & -5.0141 & -7.6987 & -21.1768 \\
20 & 3.0194 & 1.8738 & 0.9757 & 17.7846 & 19.8937 & 41.9129 \\
21 & 3.0162 & 1.8949 & 0.9798 & -24.2752 & -31.0099 & -74.4774 \\
22 & 3.0135 & 1.9124 & 0.9832 & 53.3183 & 62.8989 & 140.2436 \\
23 & 3.0112 & 1.9270 & 0.9860 & -89.8291 & -110.3477 & -255.8816 \\
24 & 3.0094 & 1.9392 & 0.9883 & 174.2544 & 209.2637 & 474.9046 \\
25 & 3.0078 & 1.9493 & 0.9902 & -312.9364 & -380.3673 & -873.2766 \\
26 & 3.0065 & 1.9577 & 0.9919 & 585.8511 & 707.4054 & 1613.8975 \\
27 & 3.0054 & 1.9648 & 0.9932 & -1072.2650 & -1299.3579 & -2974.5325 \\
28 & 3.0045 & 1.9707 & 0.9944 & 1986.6883 & 2402.7924 & 5490.3713 \\
29 & 3.0038 & 1.9755 & 0.9953 & -3656.5809 & -4427.0703 & -10125.9937 \\
30 & 3.0031 & 1.9796 & 0.9961 & 6754.3291 & 8172.9108 & 18683.6492 \\
}

\begin{teilaufgaben}
\item
Die Inverse von $A$ und die Lösung $x$ des Gleichungssystems $Ax=b$ sind
\[
A^{-1}
=
\begin{pmatrix}
   3&   2&  -2\\
 -19& -12&  13\\
  -4&  -3&   3
\end{pmatrix}
\qquad\Rightarrow\qquad
x=A^{-1}b=
\begin{pmatrix}
3\\2\\1
\end{pmatrix}.
\]
\item
Das Jacobi-Verfahren verwendet die Matrix
\[
B_{\text{Jacobi}}
=
D^{-1}(L+R)
=
\begin{pmatrix}
 3&  0&  0 \\
 0&  1&  0 \\
 0&  0&  2
\end{pmatrix}^{-1}
\begin{pmatrix}
 0&  0&  2 \\
 5&  0& -1 \\
 9&  1&  0
\end{pmatrix}
=
\begin{pmatrix}
   0    &   0    &\frac23\\
   5    &   0    & -1    \\
\frac92 &\frac12 &  0 
\end{pmatrix}
\]
Numerisch findet man die Eigenwerte
\[
\lambda_1 =    1.84484,
\quad
\lambda_2,3 =
  -0.92242 \pm 0.22927i.
\]
Der Spektralradius von $B_{\text{Jacobi}}$ ist daher
$\varrho(B_{\text{Jacobi}}) = 1.8448$, das Verfahren kann daher nicht
konvergieren.
\item
Das Gauss-Seidel-Verfahren verwendet Iteration der Matrix
\[
B_{\text{Gauss-Seidel}}
=
(L+D)^{-1}
=
\begin{pmatrix}
   3&  0&  0 \\
   5&  1&  0 \\
   9&  1&  2
\end{pmatrix}^{-1}
\begin{pmatrix}
   0&  0&  2 \\
   0&  0& -1 \\
   0&  0&  0
\end{pmatrix}
=
\begin{pmatrix}
0&0&\frac23\\
0&0&-\frac{13}3\\
0&0&-\frac56
\end{pmatrix}
\]
Diese Matrix hat den doppelten Eigenwert $0$ und den einfachen Eigenwert
$-\frac56$.
Der Spektralradius ist also 
$\varrho(B_{\text{Gauss-Seidel}}=\frac56<1$.
Das Gauss-Seidel-Verfahren konvergiert also.
\item
Die Resultate der numerischen Iteration
\begin{equation}
\begin{aligned}
x_{k+1} &= (L+D)^{-1}b - (L + D)^{-1}Rx_k&&\text{Gauss-Seidel}\\
y_{k+1} &= D^{-1}b - D^{-1}(L+R)y_k&&\text{Jacobi}
\end{aligned}
\label{6003:iteration}
\end{equation}
sind in Tabelle~\ref{6003:tabelle} zusammengestellt.
\qedhere
\end{teilaufgaben}
\begin{table}
\centering
\begin{tabular}{|>{$}r<{$}|>{$}r<{$}>{$}r<{$}>{$}r<{$}|>{$}r<{$}>{$}r<{$}>{$}r<{$}|}
\hline
 &\multicolumn{3}{c|}{Gauss-Seidel}&\multicolumn{3}{c|}{Jacobi}\\
k&x_{1k}&x_{2k}&x_{3k}&y_{1k}&y_{2k}&y_{3k}\\
\hline
\tabelleninhalt
\hline
\infty&3.0000 & 2.0000 & 1.0000 & \infty & \infty & \infty \\
\hline
\end{tabular}
\caption{Resultate der Iteration nach Gauss-Seidel und Jacobi mit Hilfe der
Iterationsformeln \eqref{6003:iteration}.
Im Falle des stabilen Gauss-Seidel-Verfahrens werden zufällige Startwerte
verwendet, im Jacobi-Verfahren liegen die Startwerte in unmittelbarer Nähe der
Lösung.
Trotzdem divergiert der Vektor im Jacobi-Verfahren ganz offensichtlich,
während das Gauss-Seidel-Verfahren konvergiert.
\label{6003:tabelle}}
\end{table}
\end{loesung}

