%
% instabilitaet.tex
%
% (c) 2020 Prof Dr Andreas Müller, Hochschule Rapperswil
%
\section{Numerische Instabilität
\label{buch:section:instabilitaet}}
\rhead{Numerische Instabilitöt}
\index{numerische Instabilität}%
\index{Instabilität}%
Die Untersuchungen zum Verhalten von Iterationsfolgen sind davon
ausgegangen, dass alle Rechnung ohne Fehler ausgeführt werden können.
\index{Iterationsfolge}%
Dies ist aber nicht realistisch, in den meisten numerischen Berechnungen
müssen Zwischenresultate mit der beschränkten verfügbaren Genauigkeit
der Gleitkommaformate gespeichert werden, die die Maschine zur Verfügung
stellt.
\index{Konvergenz}%
Es ist daher zu untersuchen, wie sich die Konvergenz eines Lösungsverfahrens
ändert, wenn es durch Rundungsfehler im Laufe der Rechnung gestört wird.
\index{Rundungsfehler}%

%
% Eine instabile Quadratwurzel
%
\subsection{Eine instabile Quadratwurzel
\label{buch:subsection:quadratwurzel}}
Die Quadratwurzel von $a$ ist ein Fixpunkt der Funktion
$f(x)=x^3/a$ denn $f(\!\sqrt{a})=a^{\frac32}/a=\!\sqrt{a}$.
\index{Quadratwurzel}%
\index{Wurzel}%
Das Konvergenzkriterium Satz~\ref{buch:satz:konvergenzkriterium} besagt,
dass Konvergenz garantiert ist, wenn der Betrag der Ableitung von $f'(x)$
am Fixpunkt kleiner als $1$ ist.
\index{Ableitung}%
Aber
\[
f'(x) = \frac{3x^2}{a}
\quad\Rightarrow\quad
f'(\!\sqrt{a}) = \frac{3\!\sqrt{a}^2}{a}=3,
\]
die Iterationsfolge wird also niemals konvergieren.
Im Gegenteil, der Fehler wird in jeder Iteration um den Faktor $3$
anwachsen.

Der \texttt{float}-Gleitkommatyp verwendet eine Mantisse von 23 bit,
das niederwertigste Bit hat also die Wertigkeit $2^{-23}$.
Eine Approximation von $\!\sqrt{a}$ wird daher, sofern sie nicht
exakt ist, beim Speichern als \texttt{float}-Zahl einen Fehler
von der Grössenordnung $2^{-24}$ aufnehmen.
\index{Approximation}%
Nach $k$ Iterationen wird dieser Fehler auf $3^k2^{-24}=2^{k\log_23-24}$
angewachsen sein.
\index{Iteration}%
Nach $k\ge 24/\log_23=15.142$ Iterationen ist also der Fehler von der
gleichen Grössenordnung wie das gesuchte Resultat.

\begin{table}
\centering
\renewcommand\arraystretch{1.15}
\begin{tabular}{|
>{$}r<{$}|
>{$}r<{$}|
>{$}r<{$}|
>{$}r<{$}|
>{$}r<{$}|
>{$}r<{$}|}
\hline
  & \texttt{10001} & \texttt{10010} & \texttt{10011} & \texttt{10100} & \texttt{10101} \\
  &   1.414213 &   1.414213 &   1.414214 &   1.414214 &   1.414214 \\
  & -2.384\cdot 10^{-7} & -1.192\cdot 10^{-7} &          0 & 1.192\cdot 10^{-7} & 2.384\cdot 10^{-7} \\
\hline
 1&   1.414213 &   1.414213 &   1.414213 &   1.414214 &   1.414214 \\
 2&   1.414211 &   1.414212 &   1.414213 &   1.414214 &   1.414216 \\
 3&   1.414207 &   1.414210 &   1.414212 &   1.414216 &   1.414220 \\
 4&   1.414193 &   1.414204 &   1.414210 &   1.414220 &   1.414232 \\
 5&   1.414152 &   1.414184 &   1.414204 &   1.414232 &   1.414268 \\
 6&   1.414029 &   1.414126 &   1.414184 &   1.414268 &   1.414377 \\
 7&   1.413660 &   1.413951 &   1.414126 &   1.414377 &   1.414705 \\
 8&   1.412553 &   1.413427 &   1.413951 &   1.414705 &   1.415687 \\
 9&   1.409239 &   1.411856 &   1.413427 &   1.415687 &   1.418640 \\
10&   1.399341 &   1.407152 &   1.411856 &   1.418640 &   1.427533 \\
11&   1.370064 &   1.393135 &   1.407152 &   1.427533 &   1.454550 \\
12&   1.285858 &   1.351915 &   1.393135 &   1.454550 &   1.538706 \\
13&   1.063039 &   1.235429 &   1.351915 &   1.538706 &   1.821534 \\
14&   0.600645 &   0.942808 &   1.235429 &   1.821534 &   3.021912 \\
15&   0.108348 &   0.419024 &   0.942808 &   3.021912 &  13.797982 \\
16&6.359\cdot 10^{-\phantom{0}4}&3.678\cdot 10^{-\phantom{0}2}&   0.419024 &  13.797982 &1.313\cdot10^{\phantom{0}3}\\
17&1.286\cdot 10^{-10}&2.489\cdot 10^{-\phantom{0}5}&3.678\cdot 10^{-\phantom{0}2}&1.313\cdot10^{\phantom{0}3}&1.132\cdot10^{\phantom{0}9}\\
18&1.063\cdot 10^{-30}&7.710\cdot 10^{-15}&2.489\cdot 10^{-5\phantom{0}}&1.132\cdot10^{\phantom{0}9}&7.271\cdot10^{26}\\
19&  0.000000  &2.298\cdot 10^{-43}&7.710\cdot 10^{-15}&7.271\cdot10^{26}&  \infty    \\
20&  0.000000  &  0.000000  &2.298\cdot 10^{-43}&  \infty    &  \infty    \\
\hline
\end{tabular}
\caption{Iterationsfolge der Funktion $f(x)=x^3/2$ für verschiedene Näherungen
des Fixpunktes $x^*=\!\sqrt{2}$.
Die Startwerte unterscheiden sich nur in den letzten zwei Bits ihrer
Darstellung als \texttt{float}-Gleitkommazahl, die letzten fünf Bits
der Mantisse sind in der ersten Kopfzeile dargestellt.
In der dritten Zeile stehen die Differenzen zur besten Approximation
von $\!\sqrt{2}$ durch eine \texttt{float}-Zahl.
\label{buch:table:sqrtinstabil}}
\end{table}

Wir illustrieren dies mit einer Rechnung\footnote{Das C++-Programm
hierzu ist \texttt{sqrt.cpp} im Verzechnis
\texttt{buch/chapters/experimente/sqrt} von \cite{buch:repo}.},
deren Resultate in Tabelle~\ref{buch:table:sqrtinstabil} zusammengestellt
sind.
\index{C++}%
Zunächst berechnen wir die Quadratwurzel $\!\sqrt{2}$ als \texttt{float}-Zahl.
Dann bilden wir die nächsten Nachbarn, die sich nur in den letzten
zwei Bits unterscheiden, die letzten fünf Bits der Mantisse dieser
Zahlen sind in der obersten Kopfzeile der Tabelle gezeigt.
In der zweiten Kopfzeile sieht man, dass diese Unterschiede so klein
sind, dass sie nur gerade die Rundung in der sechsten dezimalen
Nachkommastelle beinflussen können.
Der Unterschied dieser Startwerte zum Maschinen-Wert für $\!\sqrt{2}$
ist in der dritten Kopfzeile dargestellt.

Im unteren Teil der Tabelle wird dann ausgehend von jedem dieser Startwerte
die Iterationsfolge mit der Funktion $f(x)=x^3/2$ gebildet.
\index{Startwert}%
Spätestens nach der zweiten Iteration beginnen sich die Werte vom Fixpunkt zu
entfernen, nach 16 Iterationen sind die Abweichungen von der 
Grössenordnung $1$ wie in der Überschlagsrechnung weiter oben
vorhergesagt.

%
% Numerische Instabilität
%
\subsection{Numerische Instabilität
\label{buch:subsection:numerischeinstabilitaet}}
Von {\em numerischer Instabilität} spricht man, wenn ein Berechnungsverfahren
\index{Instabilität, numerische}
\index{numerische Instabilität}
allein wegen der unvermeidlichen Rundungsfehler keine sinnvollen
Resultate liefern kann.

\begin{beispiel}
Es soll das Integral
\[
I_n = \int_0^1 \frac{x^n}{x+a}\,dx
\]
für festes $a>1$ und für ganze Zahlen $n$ mit $1\le n\le 15$ berechnet
werden.
\index{Integral}%
Da im Intervall $[0,1]$ gilt $x^{n+1}>x^n$ ist $I_n$ eine monoton
abnehmende Folge.
Es ist auch klar, dass $\lim_{n\to\infty}I_n=0$.

Das Integral ist im Prinzip nicht schwierig zu berechnen, wenn man im
Integranden die Polynom-Division ausführt:
\index{Polynom-Division}%
\begin{align}
\frac{x^n}{x+a}
&=
x^{n-1} -ax^{n-2}+a^2x^{n-3}-\dots +(-1)^n xa^{n-2} +(-1)^{n-1}a^{n-1}
+(-1)^n\frac{a^n}{a+x}
\notag
\\
\int_0^1 \frac{x^n}{x+a}\,dx
&=
\frac1n
-\frac{a}{n-1}
+\frac{a^2}{n-2}
-\dots
+(-1)^n\frac{a^{n-2}}{2}
+(-1)^{n-1}a^{n-1}
+\log(1+a)-\log a.
\label{buch:numerik:instabil:summe}
\end{align}
Jeder Term ist ein elementares Integral.
\end{beispiel}

\begin{table}
\centering
\renewcommand\arraystretch{1.15}
\begin{tabular}{|>{$}r<{$}|>{$}r<{$}|>{$}r<{$}|}
\hline
 n&   I_n   & \text{Rückwärtsiteration}\\
\hline
 0&   0.095310179804324{\color{red}9          } & 0.09531017980432486\\
 1&   0.04689820195675{\color{red}07          } & 0.04689820195675140\\
 2&   0.0310179804324{\color{red}935          } & 0.03101798043248600\\
 3&   0.023153529008{\color{red}3985          } & 0.02315352900847329\\
 4&   0.01846470991{\color{red}60155          } & 0.01846470991526711\\
 5&   0.0153529008{\color{red}398455          } & 0.01535290084732894\\
 6&   0.013137658{\color{red}2682119          } & 0.01313765819337729\\
 7&   0.01148056{\color{red}01750236          } & 0.01148056092337000\\
 8&   0.01019439{\color{red}82497636          } & 0.01019439076629997\\
 9&   0.009167{\color{red}1286134746          } & 0.00916720344811137\\
10&   0.00832{\color{red}87138652537          } & 0.00832796551888631\\
11&   0.00762{\color{red}19522565537          } & 0.00762943572022779\\
12&   0.007{\color{red}1138107677959          } & 0.00703897613105546\\
13&   0.00{\color{red}57849692451174          } & 0.00653331561252233\\
14&   0.0{\color{red}135788789773972          } & 0.00609541530334817\\
15&  -0.0{\color{red}691221231073049          } & 0.00571251363318499\\
16&   0.{\color{red}753721231073049\phantom{0}} & 0.00537486366815009\\
17&  {\color{red}-7.47838878131872\phantom{00}} & 0.00507489273026384\\
18&  {\color{red}74.8394433687428\phantom{000}} & 0.00480662825291712\\
19&{\color{red}-748.341802108480\phantom{0000}} & 0.00456529641819719\\
20&{\color{red}7483.46802108480\phantom{00000}} & 0.00434703581802811\\
\hline
\end{tabular}
\caption{Instabile Iteration zur Berechnung der Integrale $I_n$ mit
$a=10$.
In jedem Schritt wird der Fehler mit $a$ multipliziert.
In der dritten Spalte die Resultate der stabilen Rückwärtsiteration.
\index{Rückwärtsiteration}%
\label{buch:table:Iiteration}}
\end{table}

Wenn $n$ gross ist, ist \eqref{buch:numerik:instabil:summe}
eine ziemlich aufwendig Rechnung.
Da man das Integral für alle $n$ haben will, bietet sich eine
Rekursionsformel an.
\index{Rekursionsformel}%
Es ist nämlich
\begin{align*}
I_n
&=
\int_0^1 \frac{x^n}{x+a}\,dx
=
\int_0^1 \frac{x^{n-1}(x+a-a)}{x+a}\,dx
=
\int_0^1 x^n - \frac{ax^{n-1}}{x+a}\,dx
\\
&=
\int_0^1 x^n\,dx - aI_{n-1}
=
\frac1{n+1} - a I_{n-1}.
\end{align*}
Wenn man also $I_0$ berechnet hat, dann kann man mit dieser 
Rekursionsformel alle anderen $I_n$ berechnen.
$I_0$ ist nicht schwierig zu berechnen, es ist
\[
I_0 = \int_0^1 \frac{1}{x+a}\,dx = \log(1+a) - \log a = \log\frac{1+a}a.
\]

Um die Empfindlichkeit der Rekursion auf Rundungsfehler zu untersuchen,
nehmen wir vereinfachend an, dass ausschliesslich bei der Berechnung von
$I_0$ ein unvermeidlicher Rundungsfehler der Grösse $\varepsilon$ passiert
und dass alle folgenden Operationen exakt ausgeführt werden können.
Da der Logarithmus eine transzendente Funktion ist, werden fast alle Werte
des Logrithmus bei Speichern als Gleitkommazahl gerundet werden müssen.
\index{Logarithmus}%
Wir müssen jetzt also die Rekursionsfolge $I^*_n$ ausgehend von
$I_0^* = I_0+\varepsilon$
bilden:
\[
\begin{aligned}
I_0^* &= I_0 + \varepsilon
\\
I_1^* &= \frac12 - a(I_0^*) = \frac12-aI_0 - a\varepsilon
      &&=I_1 - a \varepsilon
\\
I_2^* &= \frac13 - a(I_1^*) = \frac13-aI_1 + a^2\varepsilon
      &&=I_2 + a^2 \varepsilon
\\
\vdots\;&\quad\vdots\\
I_n^* &= \frac{1}{n+1} -aI_{n-1}^* = \frac{1}{n+1} - aI_{n-1} + (-a)^n\varepsilon
      &&= I_{n-1} + (-a)^n\varepsilon.
\end{aligned}
\]
In jedem Iterationsschritt wird der Fehler mit $a$ multipliziert.
In jedem Schritt gehen als $\log_2a$ Binärstellen Genaugikeit verloren.
Für $a=10$ bedeutet dies, dass in jedem Schritt eine Dezimalstelle
verloren geht.

Die Instabilität rührt daher, dass der Fehler in jedem Schritt mit
$a>1$ multipliziert wird.
Wir könnten versuchen, die Rekursion umzukehren, so dass durch $a$ 
dividiert wird, so könnte das Problem möglicherweise behoben werden.
Indem wir nach $I_{n-1}$ auflösen, erhalten wir die Rekursionsformel
\[
I_{n-1} = \frac1{a} \biggl(
\frac{1}{n+1} -I_n\biggr).
\]
\index{Rekursionsformel}%
In dieser Rekursion wird tatsächlich in jedem Schritt durch $a$ dividiert,
man darf daher davon ausgehen, dass in jeder Iteration 
$\log_2a$ Binärstellen Genauigkeit gewonnen werden.

Allerdings muss man für diese Iteration einen der Werte $I_n$ bereits
kennen.
Man weiss aber, dass $\lim_{n\to\infty} I_n=0$ ist, d.~h.~wenn man
$I_n$ durch $0$ ersetzt macht man einen Fehler exakt von der
Grössenordnung $I_n$.
Die Rekursion reduziert diesen Fehler in jedem Schritt um $\log_2a$
Binärstellen.
Da die Werte $I_n$ alle kleiner als $1$ sind, macht man also
niemals einen Fehler grösser als $1$ und nach $k$ Rückwärtsiterationsschritten
ist der Fehler kleiner als $a^k$.
\index{Rückwärtsiteration}%
Man kann also selbst ohne die Kenntnis eines Startwertes ausreichend
genaue Werte von $I_n$ bestimmen.
\index{Startwert}%
Dies ist in der dritten Spalte von Tabelle~\ref{buch:table:Iiteration}
durchgeführt.





