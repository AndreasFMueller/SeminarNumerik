Man kann zeigen, dass die durch
\begin{equation}
x_{n+1} = 2^{n+1} (\sqrt{1+2^{-n}x_n}-1)
\label{1002:eqn}
\end{equation}
definierte Folge für Startwerte $x_0>-1$ gegen $\log(x_0+1)$ konvergiert.
\begin{teilaufgaben}
\item
Warum tritt in der Rekursionsformel~\eqref{1002:eqn} Auslöschung auf?
\item
Formulieren Sie \eqref{1002:eqn} derart, dass keine Auslöschung auftritt.
\end{teilaufgaben}

\begin{hinweis}
Verwenden Sie die Halbwinkelformel
\begin{equation}
\tan\frac{\alpha}2 = \frac{\sqrt{1+\tan^2 \alpha } - 1}{\tan \alpha}.
\label{buch:1002:halbwinkel}
\end{equation}
\end{hinweis}

\begin{loesung}
\begin{teilaufgaben}
\item
Da die Folge $x_n$ konvergiert, geht $2^{-n}x_n$ gegen $0$ und damit
geht $\sqrt{1+2^{-n}x_n}$ gegen $1$.
Die Differenz mit $1$ führt dann zu Auslöschung.
\item
Im Zähler der rechten Seite der Halbwinkelformel
\eqref{buch:1002:halbwinkel} für den Tangens
kommt genau der Ausdruck vor, der für die
Auslöschung verantwortlich ist.
Um die Formel verwenden zu können muss
\[
\tan \alpha
=
\sqrt{\frac{x_n}{2^n}}
\qquad\Rightarrow\qquad
\alpha_n
=
\arctan
\sqrt{\frac{x_n}{2^n}}.
\]
Die Rekursionsformel besagt dann
\[
\frac{x_{n+1}}{2^{n+1}}
=
\tan\alpha
\cdot
\tan\frac{\alpha}2
=
\sqrt{\frac{x_n}{2^n}}
\tan\biggl(\frac12\arctan\sqrt{\frac{x_n}{2^n}}\biggr)
\]
oder
\begin{equation}
x_{n+1}
=
2^{n+1}
\sqrt{\frac{x_n}{2^n}}
\tan\biggl(\frac12\arctan\sqrt{\frac{x_n}{2^n}}\biggr).
\label{1002:stab}
\end{equation}
In dieser Form ist die Iteration ohne Auslöschung durchführbar, wie man in
Tabelle~\ref{1002:table} sehen kann.
Allerdings benötigt die Iteration jetzt zusätzlich die Auswertung
eines Arkustangens und eines Tangens, was den Rechenaufwand 
beträchtlich erhöht, selbst auf modernen Floatingpoint-Hardware, die
diese Operationen sehr schnell ausführen kann.
\qedhere
\end{teilaufgaben}
\begin{table}
\centering
\begin{tabular}{|>{$}r<{$}|>{$}r<{$}|>{$}r<{$}|}
\hline
n&\text{$x_n$ nach \eqref{1002:eqn}}&\text{$x_n$ nach \eqref{1002:stab}}\\
\hline
 0 &   0.5000000000             &   0.5000000000 \\
 1 &   0.\underline{4}494898319 &   0.\underline{4}494897127 \\
 2 &   0.\underline{4}267277718 &   0.\underline{4}267276525 \\
 3 &   0.\underline{4}159164429 &   0.\underline{4}159160256 \\
 4 &   0.\underline{4}106464386 &   0.\underline{4}106463492 \\
 5 &   0.\underline{40}80467224 &   0.\underline{40}80447853 \\
 6 &   0.\underline{40}67535400 &   0.\underline{40}67522287 \\
 7 &   0.\underline{40}61126709 &   0.\underline{40}61080217 \\
 8 &   0.\underline{405}7922363 &   0.\underline{405}7863951 \\
 9 &   0.\underline{405}6396484 &   0.\underline{405}6257606 \\
10 &   0.\underline{405}5175781 &   0.\underline{405}5454433 \\
11 &   0.\underline{405}5175781 &   0.\underline{405}5052698 \\
12 &   0.\underline{405}2734375 &   0.\underline{4054}851830 \\
13 &   0.\underline{405}2734375 &   0.\underline{4054}751098 \\
14 &   0.\underline{40}42968750 &   0.\underline{4054}700732 \\
15 &   0.\underline{40}23437500 &   0.\underline{40546}75400 \\
16 &   0.3984375000             &   0.\underline{40546}62883 \\
17 &   0.3906250000             &   0.\underline{405465}6327 \\
18 &   0.3750000000             &   0.\underline{405465}3049 \\
19 &   0.3750000000             &   0.\underline{405465}1558 \\
20 &   0.3750000000             &   0.\underline{4054650}664 \\
21 &   0.2500000000             &   0.\underline{4054650}068 \\
\hline
\infty&0.4054650962             &   0.4054650962 \\
\hline
\end{tabular}
\caption{Auslöschung in der Folge $x_{n}$ berechnet mit den zwei verschiedenen
Rekursionsformeln \eqref{1002:eqn} und \eqref{1002:stab} mit Hilfe von
\texttt{float}-Zahlen.
\label{1002:table}}
\end{table}
\end{loesung}


