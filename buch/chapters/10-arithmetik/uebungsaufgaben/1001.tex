Betrachten Sie die Funktion
\[
f(x) = \frac{1-\cos x}{x}.
\]
\index{$\cos(x)$}%
\begin{teilaufgaben}
\item Berechnen Sie $f(10^{-10})$.
\item Berechnen Sie $\displaystyle \lim_{x\to 0} f(x)$.
\item Was für ein Problem tritt bei der numerischen Berechnung von
$f(x)$ für kleine Werte von $x$ auf?
\item Schätzen Sie ab, wie klein  $x$ maximal werden darf, damit
die naive Berechnung von $f(x)$ gemäss obiger Formel mit
\texttt{float}- oder \texttt{double}-Zahlen einen vom in a) berechneten
Grenzwert verschiedenen Wert liefert.
\item Finden Sie eine Berechnungsformel für $f(x)$, die auch für kleine
Werte von $x$ funktioniert.
\end{teilaufgaben}

\begin{loesung}
\begin{teilaufgaben}
\item
Mit jedem Datentyp findet man $f(10^{-10})=0$.
\item
Der Grenzwert kann mit Hilfe der Regel von de l'Hospital berechnet werden:
\index{de l'Hospital}%
\[
\lim_{x\to 0}f(x)
=
\lim_{x\to 0}\frac{1-\cos x}{x}
=
\lim_{x\to 0}\frac{\frac{d}{dx}(1-\cos x)}{\frac{d}{dx} x}
=
\lim_{x\to 0} \frac{\sin x}{1} = 0.
\]
\item
Der Wert von $\cos x$ ist sehr nahe bei $1$, daher tritt Auslöschung auf.
\item
Die Taylor-Reihe für $\cos x$ ist
\index{Taylor-Reihe}
\[
\cos x = 1 -\frac1{2!}x^2 + \frac{1}{4!}x^4 - \frac{1}{6!}x^6 +\dots
\]
Der Funktionswert von $f(x)$ lässt sich nicht mehr von $0$ unterscheiden,
wenn der zweite Term der Reihe sich nicht mehr von $1$ unterscheiden lässt.
Dies ist genau der Wert für $\varepsilon$ für den verwendeten Datentyp,
den man der Tabelle~\ref{buch:table:limits} entnehmen kann.
\index{$\varepsilon$}%
Daraus leitet man ab:
\[
\frac12x^2 = \varepsilon
\qquad\Rightarrow\qquad
x=\sqrt{2\varepsilon}.
\]
Für \texttt{float} findet man $x\simeq 0.000488281$,
für \texttt{double} dagegen $x\simeq 2.10734\cdot 10^{-8}$.
\item
Die Halbwinkelformel für die $\sin$-Funktion liefert
\index{Halbwinkelformel}%
\[
\sin^2 \frac{x}2
=
\frac{1-\cos x}{2}
\qquad\Rightarrow\qquad
f(x)
=
\frac{1-\cos x}{x}
=
\frac{2\sin^2(x/2)}{x}
=
\frac{2}{x}\sin^2\frac{x}{2}
=:
g(x).
\]
Der Ausdruck $g(x)$ leidet nicht unter Auslöschung.
\index{Auslöschung}%
\qedhere
\end{teilaufgaben}
\begin{figure}
\centering
\includeagraphics[]{1001/numerisch.pdf}
\caption{Numerische Berechnung der Funktion $f(x)$ von Aufgabe 1.1 für
kleine Werte von $x$.
Die rote Kurve zeigt die Unzuverlässigkeit der Resultate infolge
Auslöschung.
Die blaue Kurve zeigt die Berechnung mit der verbesserten Funktion
$g(x)$, in der Auslöschung vermieden wird.
\index{Auslöschung}%
\label{buch:figure:1001}}
\end{figure}
Abbildung~\ref{buch:figure:1001} vergleicht die beiden Ausdrücke
$f(x)$ und $g(x)$ zur Berechnung der gegebenen Funktion.
\end{loesung}


