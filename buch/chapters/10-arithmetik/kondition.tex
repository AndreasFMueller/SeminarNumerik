%
% kondition.tex
%
% (c) 2020 Prof Dr Andreas Müller, Hochschule Rapperswil
%
\section{Kondition
\label{buch:section:kondition}}
\rhead{Kondition}
Die Beispiele zur numerischen Instabilität haben deutlich gemacht,
dass Instabilität dadurch entstehen kann, dass der Fehler im
Laufe der Rechnung grösser wird und dass diese Rechnung vielfach
wiederholt wird.

Man sagt, ein Problem sei schlecht konditioniert, wenn eine
kleine Änderung der Eingangsdaten eine grosse Änderung der Resultate
zur Folge hat.
Solche Problem verlangen, dass von Anfang an mit hoher Genauigkeit
gerechnet wird, und sie lassen sich schlecht iterieren, da sich
die Rundungsfehler mit der Zeit derart aufschaukeln werden, dass
man kein Vertrauen mehr in die gefunden Resultate haben kann.

Gut konditioniert Problem sind dagegen solche, bei denen kleine
Störungen der Eingangsdaten nur geringe Fehler in den Resultaten
zur Folge haben.
In einem gut konditionierten Problem werden sich während der
Rechnung auftretende Rundungsfehler kaum gravierend auswirken.





