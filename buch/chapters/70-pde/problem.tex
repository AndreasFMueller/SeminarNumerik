%
% problem.tex
%
% (c) 2020 Prof Dr Andreas Müller, Hochschule Rapperswil
%
\section{Problemstellung
\label{section:pde:problem}}
Gewöhnliche Differentialgleichungen werden durch eine einzige Funktion
$f\colon \mathbb R\times\mathbb R^n: (t,x)\mapsto f(t,x)$ beschrieben,
werden.
Eine Lösung der Differentialgleichung
\begin{equation}
\frac{dx}{dt} = f(t,x)
\label{pde:eqn:ode}
\end{equation}
mit der Anfangsbedingung
\[
x(t_0) = x_0
\]
ist eine Funktion $x(t)$ mit $x(t_0)=x_0$ derart, dass 
\[
\frac{dx(t)}{dt} = f(t, x(t)).
\]
Das Definitionsgebiet der Lösungsfunktion $x(t)$ ist ein Intervall
der Form $[t_0,a]$ mit $a>t_0$.

Für eine partielle Differentialgleichung ist die Situation wesentlich
komplizierter.
Zunächst gibt es Ableitungen der gesuchten Funktion $u(x_1,\dots,x_n)$
nach allen unabhängigen Variablen zu
berücksichtigen, so dass eine explizite Form der Differentialgleichung
wie in~\eqref{pde:eqn:ode} grundsätzlich nicht mehr möglich ist.
Das Definitionsgebiet ist eine fast beliebige Teilmenge
$\Omega\subset\mathbb R^n$ eines $n$-dimensionalen Raumes.
Insbesonderen kann das Definitionsgebiet sehr viel komplizierter sein
als im Falle einer gewöhnlichen Differentialgleichungen.
Die Form des Gebietes hat einen wesentlichen Einfluss auf die Lösungen
der Differentialgleichung.
Schliesslich wird es nicht mehr genügen, Werte in nur einem Randpunkt
des Gebietes zu kennen, wie das bei einer gewöhnlichen Differentialgleichung
der Fall war.
Vielmehr ist es eine nichttriviale Frage, auf welchem Teil des Randes
$\partial\Omega$ von $\Omega$ welche Funktions- oder Ableitungswerte
vorgegeben werden müssen, damit die Lösung der Differentialgleichung
eindeutig bestimmt ist.

Der Einfachheit halber betrachten wir in diesem Kapitel nur partielle
Differentialgleichungen für eine skalare Funktion $u=u(x_1,\dots,x_n)$.
In diesem Abschnitt geht es darum zu klären, wie genau ein solches
Problem gestellt werden muss.
In Abschnitt~\ref{subsection:pde:gebiet} studieren wir ein sinnvolles
Definitionsgebiet für die Differentialgleichung.
In Abschnitt~\ref{subsection:pde:klassifikation} klassifizieren wir
mögliche Differentialgleichungen bevor wir in
Abschnitt~\ref{subsection:pde:randbedingungen} Randbedingungen und
in Abschnitt~\ref{subsection:pde:loesung} Lösungen beschreiben.

\subsection{Gebiet und Rand
\label{subsection:pde:gebiet}}
Eine partielle Differentialgleichung sucht eine Funktion
\[
u\colon \Omega\to\mathbb R
\]
$u(x_1,\dots,x_n)$, für die Beziehungen zwischen den partiellen
Ableitungen nach den unabhängigen Variablen gelten.
Für alle Punkte einer Menge $\Omega\subset\mathbb R^n$ muss
als eine Gleichung ähnlich wie zum Beispiel
\[
\frac{\partial^2 u}{\partial x_1^2}
+
\dots
+
\frac{\partial^2 u}{\partial x_n^2}
=
0
\qquad
\forall \;
(x_1,\dots,x_n)\in \Omega
\]
gelten.
Diese Beobachtung schränkt die Art der Menge $\Omega$ bereits ein,
wie im Folgenden diskutiert werden soll.

Die Ableitung einer Funktion $u(x_1,\dots,x_n)$ in einem Punkt
$x_0=(x_{0,1},\dots,x_{0,n})$ ist eine lineare Funktion 
$Du(x_{0,1},\dots,x_{0,n}) $ derart, dass
\[
u(x_1,\dots,x_n) - u(x_{0,1},\dots,u_{0,n})
=
Du(x_{0,1},\dots,x_{0,n})\cdot (x-x_0) + o(|x-x_0|).
\]
Das Symbol $o(|x-x_0|)$ beschreibt eine Funktion, die schneller als
ihr Argument gegen $0$ geht, so dass für den Grenzwert des
Quotienten 
\[
\lim_{x\to x_0}\frac{o(|x-x_0|)}{|x-x_0|} = 0
\]
gilt.
Der Grenzwert bedeutet, dass es für jedes $\varepsilon>0$ eine Umgebung
$U_\delta(x_0)=\{x\;|\; |x-x_0|<\delta\}$ gibt derart, dass
\[
\bigl|
u(x)-u(x_0) - Du(x_0)\cdot (x-x_0)
\bigr|
< \varepsilon\cdot |x-x_0|
\qquad\forall x\in U_\delta(x_0)
\]
ist.
Dies ist nur sinnvoll, wenn die ganze Umgebung $U_{\delta}(x_0)\subset \Omega$
im Definitionsgebiet $\Omega$ der Differentialgleichung vorhanden ist.
Dies führt auf die folgende Definition.

\begin{definition}
Eine Menge $\Omega$ heisst {\em offen}, wenn mit jedem Punkt $x\in \Omega$
auch eine offene Umgebung $U_{\delta}(x)\subset\Omega$ darin enthalten ist.
Ein {\em Gebiet} ist eine offene Menge in $\mathbb R^n$.
\end{definition}

Gebiete sind also genau die sinnvollen Definitionsgebiete für eine partielle
Differentialgleichung.
Es reicht allerdings nicht, dass die Funktion $u$ auf $\Omega$ definiert
ist, da ja auch noch Randwerte erfüllt werden müssen.

\begin{definition}
Der {\em Abschluss} $\bar{\Omega}$ einer Menge $\Omega\subset\mathbb R^n$ ist
die Menge aller Punkte in $\mathbb R^n$, die Grenzwerte von Folgen in
$\Omega$ sind.
Das {\em Innere} $\mathring{A}$ einer Menge $A$ ist die Menge aller Punkte
$x$ derart, dass es eine Umgebung $U_\delta(x)$ gibt, die ganz in $A$
enthalten ist: $U_{\delta}(x)\subset A$.
\end{definition}

Alternativ kann man den Abschluss auch charakterisieren als die Menge
aller Punkte $x\in\mathbb R^n$, für die jede beliebige Umgebung
$U_\delta(x)$ auch Punkte von $\Omega$ enthält, also
$U_\delta(x)\cap \Omega\ne \emptyset$.
Für ein Gebiet ist natürlich $\mathring\Omega=\Omega$.

Die Lösung einer partiellen Differentialgleichung wird im Allgemeinen erst
festgelegt sein, wenn zusätzlich Werte auf Teilen des ``Randes'' des
Gebietes festgelegt worden sind.
Nur Punkte, die als Grenzwerte von Punkten in $\Omega$ erreicht werden
können, können zu diesem Zweck hinzugezogen werden.

\begin{definition}
Der {\em Rand} $\partial A$ einer Menge $A\subset\mathbb R^n$ ist
$\partial A=\bar{A}\setminus\mathring{A}$.
\end{definition}

Während also eine Differentialgleichung typischerweise auf einem
Gebiet $\Omega$ definiert ist, muss die gesuchte Lösungsfunktion
sogar auf auf dem Abschluss $\bar{\Omega}$ definiert sein.
Die Lösung wird im Allgemeinen erst dadurch festgelegt, dass zusätzlich
Werte oder Ableitungen auf Teilen des Randes $\partial\Omega$ 
vorgegeben werden.

\subsection{Klassifikation der partiellen Differentialgleichungen
\label{subsection:pde:klassifikation}}
Eine partielle Differentialgleichung beschreibt eine Beziehung 
zwischen den partiellen Ableitungen der Funktion $u(x_1,\dots,x_n)$.
Wie bei den gewöhnlichen Differentialgleichungen klassifizieren wir
auch partielle Differentialgleichungen nach der Ordnung.

\begin{definition}
Die {\em Ordnung} einer partiellen Differentialgleichung ist die
Ordnung der höchsten Ableitung, die in der Differentialgleichung
vorkommt.
\end{definition}

Die Beziehung zwischen den Ableitungen kann beschrieben werden durch
eine Funktion
\[
F(x_1,\dots,x_n,u,p_1,\dots,p_n,t_{11},t_{12},\dots,t_{nn},\dots)
\]
derart, dass nach der Substitution
\begin{align*}
u&\to u(x_1,\dots,x_n)
\\
p_i&\to \frac{\partial u}{\partial x_i}
\\
t_{ij} &Ò\to \frac{\partial^2 u}{\partial x_i\,\partial x_j}
\end{align*}
die Differentialgleichung
\[
F\biggl(
x_1,\dots,x_n,u(x_1,\dots,x_n),
\frac{\partial u}{\partial x_1},\dots,\frac{\partial u}{\partial x_n},
\frac{\partial u}{\partial x_1\,\partial x_1},
\frac{\partial u}{\partial x_1\,\partial x_2},\dots
\biggr)
=0
\]
entsteht.
Die Funktion $F$ kann also dazu verwendet werden, die verschiedenen
möglichen partiellen Differentialgleichungen zu klassifizieren.

\begin{beispiel}
Die Differentialgleichung
\[
\frac{\partial^2 u}{\partial x_1^2}
+
\frac{\partial^2 u}{\partial x_2^2}
=
0
\]
zweiter Ordnung wird beschrieben durch die Funktion
\[
F(x_1,x_2,u,p_1,p_2,t_{11},t_{12},t_{22})
=
t_{11} + t_{22}.
\]
Diese Funktion ist linear in $t_{11}$ und $t_{22}$.
\end{beispiel}

\begin{beispiel}
Die partielle Differentialgleichung erster Ordnung von Burgers
(Siehe auch Kapitel~\ref{chapter:burgers})
\[
\frac{\partial u}{\partial x_1}
+
u
\frac{\partial u}{\partial x_2}
=
0
\]
wird beschrieben durch die Funktion
\[
F(x_1,x_2,u,p_1,p_2)
=
p_1+up_2.
\]
Diese Funktion ist nicht linear in $u$ und $p_2$.
\end{beispiel}

\subsection{Randbedingungen
\label{subsection:pde:randbedingungen}}







