%
% explizitd.tex -- template for standalon tikz images
%
% (c) 2020 Prof Dr Andreas Müller, Hochschule Rapperswil
%
\documentclass[tikz]{standalone}
\usepackage{amsmath}
\usepackage{times}
\usepackage{txfonts}
\usepackage{pgfplots}
\usepackage{csvsimple}
\usetikzlibrary{arrows,intersections,math}
\begin{document}
\def\skala{1}
\begin{tikzpicture}[>=latex,thick,scale=\skala]

\def\sx{6}
\def\sy{1}
%
% explizitd.tex -- template for standalon tikz images
%
% (c) 2020 Prof Dr Andreas Müller, Hochschule Rapperswil
%
\documentclass[tikz]{standalone}
\usepackage{amsmath}
\usepackage{times}
\usepackage{txfonts}
\usepackage{pgfplots}
\usepackage{csvsimple}
\usetikzlibrary{arrows,intersections,math}
\begin{document}
\def\skala{1}
\begin{tikzpicture}[>=latex,thick,scale=\skala]

\def\sx{6}
\def\sy{1}
%
% explizitd.tex -- template for standalon tikz images
%
% (c) 2020 Prof Dr Andreas Müller, Hochschule Rapperswil
%
\documentclass[tikz]{standalone}
\usepackage{amsmath}
\usepackage{times}
\usepackage{txfonts}
\usepackage{pgfplots}
\usepackage{csvsimple}
\usetikzlibrary{arrows,intersections,math}
\begin{document}
\def\skala{1}
\begin{tikzpicture}[>=latex,thick,scale=\skala]

\def\sx{6}
\def\sy{1}
%
% explizitd.tex -- template for standalon tikz images
%
% (c) 2020 Prof Dr Andreas Müller, Hochschule Rapperswil
%
\documentclass[tikz]{standalone}
\usepackage{amsmath}
\usepackage{times}
\usepackage{txfonts}
\usepackage{pgfplots}
\usepackage{csvsimple}
\usetikzlibrary{arrows,intersections,math}
\begin{document}
\def\skala{1}
\begin{tikzpicture}[>=latex,thick,scale=\skala]

\def\sx{6}
\def\sy{1}
\input{explizitd.inc}

\draw[->] (-0.1,0) -- (12.6,0) coordinate[label={$c$}];
\draw[->] (0,-0.1) -- (0,7.5) coordinate[label={right:$\varrho(\tilde{A})$}];

\foreach \y in {1,...,7}{
	\draw (-0.1,\y)--(0.1,\y);
	\node at (-0.1,\y) [left] {$\y$};
}

\draw ({0.5*\sx},-0.1) -- ({0.5*\sx},0.1);
\draw ({1*\sx},-0.1) -- ({1*\sx},0.1);
\draw ({1.5*\sx},-0.1) -- ({1.5*\sx},0.1);
\draw ({2*\sx},-0.1) -- ({2*\sx},0.1);

\node at ({0*\sx},-0.1) [below] {$0$};
\node at ({0.5*\sx},-0.1) [below] {$\frac12$};
\node at ({1*\sx},-0.1) [below] {$1$};
\node at ({1.5*\sx},-0.1) [below] {$\frac{3}{2}$};
\node at ({2*\sx},-0.1) [below] {$2$};

\end{tikzpicture}
\end{document}



\draw[->] (-0.1,0) -- (12.6,0) coordinate[label={$c$}];
\draw[->] (0,-0.1) -- (0,7.5) coordinate[label={right:$\varrho(\tilde{A})$}];

\foreach \y in {1,...,7}{
	\draw (-0.1,\y)--(0.1,\y);
	\node at (-0.1,\y) [left] {$\y$};
}

\draw ({0.5*\sx},-0.1) -- ({0.5*\sx},0.1);
\draw ({1*\sx},-0.1) -- ({1*\sx},0.1);
\draw ({1.5*\sx},-0.1) -- ({1.5*\sx},0.1);
\draw ({2*\sx},-0.1) -- ({2*\sx},0.1);

\node at ({0*\sx},-0.1) [below] {$0$};
\node at ({0.5*\sx},-0.1) [below] {$\frac12$};
\node at ({1*\sx},-0.1) [below] {$1$};
\node at ({1.5*\sx},-0.1) [below] {$\frac{3}{2}$};
\node at ({2*\sx},-0.1) [below] {$2$};

\end{tikzpicture}
\end{document}



\draw[->] (-0.1,0) -- (12.6,0) coordinate[label={$c$}];
\draw[->] (0,-0.1) -- (0,7.5) coordinate[label={right:$\varrho(\tilde{A})$}];

\foreach \y in {1,...,7}{
	\draw (-0.1,\y)--(0.1,\y);
	\node at (-0.1,\y) [left] {$\y$};
}

\draw ({0.5*\sx},-0.1) -- ({0.5*\sx},0.1);
\draw ({1*\sx},-0.1) -- ({1*\sx},0.1);
\draw ({1.5*\sx},-0.1) -- ({1.5*\sx},0.1);
\draw ({2*\sx},-0.1) -- ({2*\sx},0.1);

\node at ({0*\sx},-0.1) [below] {$0$};
\node at ({0.5*\sx},-0.1) [below] {$\frac12$};
\node at ({1*\sx},-0.1) [below] {$1$};
\node at ({1.5*\sx},-0.1) [below] {$\frac{3}{2}$};
\node at ({2*\sx},-0.1) [below] {$2$};

\end{tikzpicture}
\end{document}



\draw[->] (-0.1,0) -- (12.6,0) coordinate[label={$c$}];
\draw[->] (0,-0.1) -- (0,7.5) coordinate[label={right:$\varrho(\tilde{A})$}];

\foreach \y in {1,...,7}{
	\draw (-0.1,\y)--(0.1,\y);
	\node at (-0.1,\y) [left] {$\y$};
}

\draw ({0.5*\sx},-0.1) -- ({0.5*\sx},0.1);
\draw ({1*\sx},-0.1) -- ({1*\sx},0.1);
\draw ({1.5*\sx},-0.1) -- ({1.5*\sx},0.1);
\draw ({2*\sx},-0.1) -- ({2*\sx},0.1);

\node at ({0*\sx},-0.1) [below] {$0$};
\node at ({0.5*\sx},-0.1) [below] {$\frac12$};
\node at ({1*\sx},-0.1) [below] {$1$};
\node at ({1.5*\sx},-0.1) [below] {$\frac{3}{2}$};
\node at ({2*\sx},-0.1) [below] {$2$};

\end{tikzpicture}
\end{document}

