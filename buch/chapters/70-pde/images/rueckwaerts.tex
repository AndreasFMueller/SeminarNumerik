%
% rueckwaerts.tex -- Rückwärts-Verfahren für die Wärmeleitungsgleichung
%
% (c) 2020 Prof Dr Andreas Müller, Hochschule Rapperswil
%
\documentclass[tikz]{standalone}
\usepackage{amsmath}
\usepackage{times}
\usepackage{txfonts}
\usepackage{pgfplots}
\usepackage{csvsimple}
\usetikzlibrary{arrows,intersections,math}
\begin{document}
\def\skala{1.2}
\begin{tikzpicture}[>=latex,thick,scale=\skala]

\fill[color=red!10]
	   (7,2.5) arc (90:0:0.5)
	-- (7.5,2) arc (180:360:0.5)
	-- (8.5,2) arc (180:90:0.5)
	-- (9,2.5) arc (-90:90:0.5)
	-- (7,3.5) arc (90:270:0.5)
	-- cycle;

\fill[color=red!10]
	   (0,2.5) arc (90:0:0.5)
	-- (0.5,2) arc (180:360:0.5)
	-- (1.5,2) arc (180:90:0.5)
	-- (2,2.5) arc (-90:90:0.5)
	-- (0,3.5) arc (90:270:0.5)
	-- cycle;

\draw[->] (-0.2,0) -- (10.8,0) coordinate[label={$x$}];
\draw[->] (0,-0.2) -- (0,4.5) coordinate[label={left:$y$}];

\fill[color=red!10] (0.45,3) arc (0:180:0.45) -- cycle;

\foreach \x in {0,...,10}{
	\foreach \y in {1,...,4}{
		\fill[color=white] (\x,\y) circle[radius=0.1];
		\fill (\x,\y) circle[radius=0.08];
	}
}
\foreach \x in {1,...,10}{
	\fill[color=white] (\x,0) circle[radius=0.1];
	\fill[color=blue] (\x,0) circle[radius=0.08];
	\node at (\x,-0.1) [below] {$i=\x$};
}
\foreach \y in {1,...,4}{
	\node at (-0.1,\y) [left] {$k=\y$};
}

\fill[color=white] (7,3) circle[radius=0.1];
\fill[color=white] (8,3) circle[radius=0.1];
\fill[color=white] (9,3) circle[radius=0.1];
\fill[color=white] (8,2) circle[radius=0.1];
\draw[color=red,line width=1.4pt] (7,3)--(9,3);
\draw[color=red,line width=1.4pt] (8,3)--(8,2);
\fill[color=red] (7,3) circle[radius=0.08];
\fill[color=red] (8,3) circle[radius=0.08];
\fill[color=red] (9,3) circle[radius=0.08];
\fill[color=red] (8,2) circle[radius=0.08];

\fill[color=white] (0,3) circle[radius=0.1];
\fill[color=white] (1,3) circle[radius=0.1];
\fill[color=white] (2,3) circle[radius=0.1];
\fill[color=white] (1,2) circle[radius=0.1];
\draw[color=red,line width=1.4pt] (0,3)--(2,3);
\draw[color=red,line width=1.4pt] (1,3)--(1,2);
\fill[color=red] (0,3) circle[radius=0.08];
\fill[color=red] (1,3) circle[radius=0.08];
\fill[color=red] (2,3) circle[radius=0.08];
\fill[color=red] (1,2) circle[radius=0.08];

\node at (8,3) [above] {$u_{i,k+1}\mathstrut $};
\node at (7,3) [above] {$u_{i-1,k+1}\mathstrut $};
\node at (9,3) [above] {$u_{i+1,k+1}\mathstrut $};
\node at (8,2) [below] {$u_{ik}\mathstrut $};

\node at (0,3) [above] {$u_{0,k+1}\mathstrut $};
\node at (1,3) [above] {$u_{1,k+1}\mathstrut $};
\node at (2,3) [above] {$u_{2,k+1}\mathstrut $};
\node at (1,2) [below] {$u_{1k}\mathstrut $};

\draw[<->] (4,3.1) -- (4,3.9);
\node at (4,3.5) [right] {$h_t\mathstrut$};

\draw[<->] (4.1,2) -- (4.9,2);
\node at (4.5,2) [above] {$h_x\mathstrut$};

\end{tikzpicture}
\end{document}

