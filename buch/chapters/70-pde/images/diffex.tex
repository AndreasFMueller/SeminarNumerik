%
% diffex.tex -- template for standalon tikz images
%
% (c) 2020 Prof Dr Andreas Müller, Hochschule Rapperswil
%
\documentclass[tikz]{standalone}
\usepackage{amsmath}
\usepackage{times}
\usepackage{txfonts}
\usepackage{pgfplots}
\usepackage{csvsimple}
\usetikzlibrary{arrows,intersections,math}
\begin{document}
\def\skala{1}
\begin{tikzpicture}[>=latex,thick,scale=\skala]

\def\xnull{0.5}
\def\schritt{2.2}

\pgfmathparse{\xnull-\schritt}
\xdef\xlinks{\pgfmathresult}

\pgfmathparse{\xnull-0.5*\schritt}
\xdef\xhlinks{\pgfmathresult}

\pgfmathparse{\xnull+0.5*\schritt}
\xdef\xhrechts{\pgfmathresult}

\pgfmathparse{\xnull+\schritt}
\xdef\xrechts{\pgfmathresult}

\pgfmathparse{(\xnull+\xhlinks)/3}
\xdef\mhlinks{\pgfmathresult}

\pgfmathparse{(\xnull+\xhrechts)/3}
\xdef\mhrechts{\pgfmathresult}

\pgfmathparse{(\xnull+\xlinks)/3}
\xdef\mlinks{\pgfmathresult}

\pgfmathparse{(\xnull+\xrechts)/3}
\xdef\mrechts{\pgfmathresult}

\pgfmathparse{2*\xnull/3}
\xdef\mnull{\pgfmathresult}

\pgfmathparse{\xnull*\xnull/3}
\xdef\ynull{\pgfmathresult}

\pgfmathparse{\xlinks*\xlinks/3}
\xdef\ylinks{\pgfmathresult}

\pgfmathparse{\xhlinks*\xhlinks/3}
\xdef\yhlinks{\pgfmathresult}

\pgfmathparse{\xrechts*\xrechts/3}
\xdef\yrechts{\pgfmathresult}

\pgfmathparse{\xhrechts*\xhrechts/3}
\xdef\yhrechts{\pgfmathresult}

\begin{scope}[xshift=-3.5cm]
	\draw[->] (-3.1,0)--(3.3,0) coordinate[label={$x$}];
	\draw[->] (-0,-0.1)--(-0,3.3) coordinate[label={left:$y$}];
	\draw[color=red,line width=1.4pt]
		plot[domain=-3:3,samples=100] (\x,{\x*\x/3});
	\draw[line width=0.2pt] (\xrechts,0)--(\xrechts,\yrechts);
	\draw[line width=0.2pt] (\xlinks,0)--(\xlinks,\ylinks);
	\draw (\xrechts,-0.1) -- (\xrechts,0.1);
	\node at (\xrechts,-0.1) [below] {$x_0+h$};
	\draw (\xlinks,-0.1) -- (\xlinks,0.1);
	\node at (\xlinks,-0.1) [below] {$x_0-h$};
	\clip (-3,-1) rectangle (3,3);
	\draw[color=blue]
		({\xlinks-5},{\ylinks-5*\mlinks})
		--
		({\xnull+5},{\ynull+5*\mlinks});
	\draw[color=blue]
		({\xnull-5},{\ynull-5*\mrechts})
		--
		({\xrechts+5},{\yrechts+5*\mrechts});
	\draw[color=blue,line width=0.2pt]
		({\xhlinks-5},{\yhlinks-5*\mlinks})
		--
		({\xhlinks+5},{\yhlinks+5*\mlinks});
	\draw[color=blue,line width=0.2pt]
		({\xhrechts-5},{\yhrechts-5*\mrechts})
		--
		({\xhrechts+5},{\yhrechts+5*\mrechts});
	\fill[color=blue] (\xlinks,\ylinks) circle[radius=0.08];
	\fill[color=blue] (\xnull,\ynull) circle[radius=0.08];
	\fill[color=blue] (\xrechts,\yrechts) circle[radius=0.08];
	\fill[color=blue] (\xhrechts,\yhrechts) circle[radius=0.06];
	\fill[color=blue] (\xhlinks,\yhlinks) circle[radius=0.06];
	\draw (\xnull,-0.1)--(\xnull,0.1);
	\node at (\xnull,-0.1) [below] {$x_0$};
\end{scope}

\begin{scope}[xshift=3.5cm]
	\draw[->] (-3.1,0)--(3.3,0) coordinate[label={$x$}];
	\draw[->] (-0,-0.1)--(-0,3.3) coordinate[label={left:$y$}];
	\draw[color=red,line width=1.4pt]
		plot[domain=-3:3,samples=100] (\x,{\x*\x/3});
	\draw[line width=0.2pt] (\xrechts,0)--(\xrechts,\yrechts);
	\draw[line width=0.2pt] (\xlinks,0)--(\xlinks,\ylinks);
	\draw (\xrechts,-0.1) -- (\xrechts,0.1);
	\node at (\xrechts,-0.1) [below] {$x_0+h$};
	\draw (\xlinks,-0.1) -- (\xlinks,0.1);
	\node at (\xlinks,-0.1) [below] {$x_0-h$};
	\clip (-3,-1) rectangle (3,3);
	\draw[color=blue]
		({\xnull-5},{\ynull-5*\mnull})
		--
		({\xnull+5},{\ynull+5*\mnull});
	\draw[color=blue]
		({\xlinks-5},{\ylinks-5*\mnull})
		--
		({\xrechts+5},{\yrechts+5*\mnull});
	\fill[color=blue] (\xlinks,\ylinks) circle[radius=0.08];
	\fill[color=blue] (\xnull,\ynull) circle[radius=0.08];
	\fill[color=blue] (\xrechts,\yrechts) circle[radius=0.08];
	\draw (\xnull,-0.1)--(\xnull,0.1);
	\node at (\xnull,-0.1) [below] {$x_0$};
\end{scope}

\end{tikzpicture}
\end{document}

