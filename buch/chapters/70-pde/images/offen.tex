%
% offen.tex -- Definition einer offenen Menge
%
% (c) 2020 Prof Dr Andreas Müller, Hochschule Rapperswil
%
\documentclass[tikz]{standalone}
\usepackage{amsmath}
\usepackage{times}
\usepackage{txfonts}
\usepackage{pgfplots}
\usepackage{csvsimple}
\usetikzlibrary{arrows,intersections,math}
\begin{document}
\def\skala{1}
\begin{tikzpicture}[>=latex,thick,scale=\skala]

\fill[color=red!20]
	(0,0) -- (8,0) -- (8,2) -- (5,5) -- (0,5) -- cycle;
\node at (0.5,4.5) {$\Omega$};

\def\umgebung#1#2#3{
	\fill[color=blue!10] (#1,#2) circle[radius=#3];
	\fill[color=blue] (#1,#2) circle[radius=0.08];
}

\umgebung{2}{2}{1}
\umgebung{6.5}{3.5}{1}
\umgebung{5.5}{0.5}{0.5}
\umgebung{0.5}{1}{0.5}
\umgebung{0.3}{2}{0.3}
\umgebung{0.2}{4}{0.2}
\umgebung{4}{3.8}{1.2}

\begin{scope}[xshift=2cm,yshift=2cm]
\draw[->,color=blue] (0,0) -- (30:1);
\node[color=blue] at (30:0.5) [above left] {$\varepsilon$};
\end{scope}

\begin{scope}[xshift=6.5cm,yshift=3.5cm]
\draw[->,color=blue] (0,0) -- (30:1);
\node[color=blue] at (30:0.5) [above left] {$\varepsilon$};
\end{scope}

\begin{scope}[xshift=5.5cm,yshift=0.5cm]
\draw[->,color=blue] (0,0) -- (30:0.5);
\node[color=blue] at (30:0.25) [above left] {$\varepsilon$};
\end{scope}

\begin{scope}[xshift=4cm,yshift=3.8cm]
\draw[->,color=blue] (0,0) -- (30:1.2);
\node[color=blue] at (30:0.6) [above left] {$\varepsilon$};
\end{scope}

\draw[color=red]
	(0,0) -- (8,0) -- (8,2) -- (5,5) -- (0,5) -- cycle;

\node at (6.5,3.5) [below left] {$P$};

\end{tikzpicture}
\end{document}

