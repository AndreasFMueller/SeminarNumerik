%
% gebiet.tex -- Gebietsdefinition für PDE
%
% (c) 2020 Prof Dr Andreas Müller, Hochschule Rapperswil
%
\documentclass[tikz]{standalone}
\usepackage{amsmath}
\usepackage{times}
\usepackage{txfonts}
\usepackage{pgfplots}
\usepackage{csvsimple}
\usetikzlibrary{arrows,intersections,math}
\begin{document}
\def\skala{0.9}
\begin{tikzpicture}[>=latex,thick,scale=\skala]

\pgfmathparse{2*tan(30)}
\xdef\r{\pgfmathresult}

\coordinate (A) at (0:2);
\coordinate (Aa) at (0:3);
\coordinate (B) at (60:2);
\coordinate (Bb) at (60:3);
\coordinate (C) at (120:2);
\coordinate (Cc) at (120:3);
\coordinate (D) at (180:2);
\coordinate (De) at (180:3);
\coordinate (E) at (240:2);
\coordinate (Ee) at (240:3);
\coordinate (F) at (300:2);
\coordinate (Ff) at (300:3);

\def\interior#1{
	\fill[color=red!#1]
		   (0:2) arc (-90:-210:\r)
		-- (60:2) arc (-30:-150:\r)
		-- (120:2) arc (30:-90:\r)
		-- (180:2) arc (90:-30:\r)
		-- (240:2) arc (150:30:\r)
		-- (300:2) arc (210:90:\r) -- cycle;
}

\def\spikes{
	\foreach \a in {0,60,...,300}{
		\draw[color=red,line width=1.4pt] (\a:2) -- (\a:3);
	}
}

\def\boundary{
	\draw[color=red,line width=1.4pt]
		   (0:2) arc (-90:-210:\r)
		-- (60:2) arc (-30:-150:\r)
		-- (120:2) arc (30:-90:\r)
		-- (180:2) arc (90:-30:\r)
		-- (240:2) arc (150:30:\r)
		-- (300:2) arc (210:90:\r) -- cycle;
}

\begin{scope}[xshift=-5cm]
	\interior{100}
	\boundary
	\spikes
	\node[color=white] at (0,0) {$A$};
\end{scope}

\begin{scope}[xshift=0.5cm]
	\interior{20}
	\node at (0,0) {$\mathring{A}$};
\end{scope}

\begin{scope}[xshift=5cm]
	\interior{20}
	\boundary
	\node at (0,0) {$\mathring{A}$};
	\node[color=red] at (0,1.2) [above] {$\partial\mathring{A}$};
\end{scope}

\end{tikzpicture}
\end{document}

