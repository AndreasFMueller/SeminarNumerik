%
% implizitspektrum.tex -- spectrum for the implizit method
%
% (c) 2020 Prof Dr Andreas Müller, Hochschule Rapperswil
%
\documentclass[tikz]{standalone}
\usepackage{amsmath}
\usepackage{times}
\usepackage{txfonts}
\usepackage{pgfplots}
\usepackage{csvsimple}
\usetikzlibrary{arrows,intersections,math}
\begin{document}
\def\skala{1}
\begin{tikzpicture}[>=latex,thick,scale=\skala]

\def\sx{6}
\def\sy{3}
\def\r{0.02}

\fill[color=gray!20] ({-1*\sx},0) rectangle ({1*\sx},{2*\sy});

\draw[->] (-6.1,0) -- (6.3,0) coordinate[label={$\lambda$}];
\draw[->] (0,-0.1) -- (0,{2*\sy+0.4}) coordinate[label={$c$}];

\foreach \x in {-1,...,1}{
	\draw ({\x*\sx},-0.1) -- ({\x*\sx},0.1);
	\node at ({\x*\sx},-0.1) [below] {$\x$};
}

\draw (-0.1,{0.5*\sy}) -- (0.1,{0.5*\sy});
\draw (-0.1,{1.0*\sy}) -- (0.1,{1.0*\sy});
\draw (-0.1,{1.5*\sy}) -- (0.1,{1.5*\sy});
\draw (-0.1,{2.0*\sy}) -- (0.1,{2.0*\sy});

\node at (-0.1,{0.5*\sy}) [left] {$c=\frac12$};
\node at (-0.1,{1.0*\sy}) [left] {$c=1$};
\node at (-0.1,{1.5*\sy}) [left] {$c=\frac32$};
\node at (-0.1,{2.0*\sy}) [left] {$c=2$};

%
% implizitspektrum.tex -- spectrum for the implizit method
%
% (c) 2020 Prof Dr Andreas Müller, Hochschule Rapperswil
%
\documentclass[tikz]{standalone}
\usepackage{amsmath}
\usepackage{times}
\usepackage{txfonts}
\usepackage{pgfplots}
\usepackage{csvsimple}
\usetikzlibrary{arrows,intersections,math}
\begin{document}
\def\skala{1}
\begin{tikzpicture}[>=latex,thick,scale=\skala]

\def\sx{6}
\def\sy{3}
\def\r{0.02}

\fill[color=gray!20] ({-1*\sx},0) rectangle ({1*\sx},{2*\sy});

\draw[->] (-6.1,0) -- (6.3,0) coordinate[label={$\lambda$}];
\draw[->] (0,-0.1) -- (0,{2*\sy+0.4}) coordinate[label={$c$}];

\foreach \x in {-1,...,1}{
	\draw ({\x*\sx},-0.1) -- ({\x*\sx},0.1);
	\node at ({\x*\sx},-0.1) [below] {$\x$};
}

\draw (-0.1,{0.5*\sy}) -- (0.1,{0.5*\sy});
\draw (-0.1,{1.0*\sy}) -- (0.1,{1.0*\sy});
\draw (-0.1,{1.5*\sy}) -- (0.1,{1.5*\sy});
\draw (-0.1,{2.0*\sy}) -- (0.1,{2.0*\sy});

\node at (-0.1,{0.5*\sy}) [left] {$c=\frac12$};
\node at (-0.1,{1.0*\sy}) [left] {$c=1$};
\node at (-0.1,{1.5*\sy}) [left] {$c=\frac32$};
\node at (-0.1,{2.0*\sy}) [left] {$c=2$};

%
% implizitspektrum.tex -- spectrum for the implizit method
%
% (c) 2020 Prof Dr Andreas Müller, Hochschule Rapperswil
%
\documentclass[tikz]{standalone}
\usepackage{amsmath}
\usepackage{times}
\usepackage{txfonts}
\usepackage{pgfplots}
\usepackage{csvsimple}
\usetikzlibrary{arrows,intersections,math}
\begin{document}
\def\skala{1}
\begin{tikzpicture}[>=latex,thick,scale=\skala]

\def\sx{6}
\def\sy{3}
\def\r{0.02}

\fill[color=gray!20] ({-1*\sx},0) rectangle ({1*\sx},{2*\sy});

\draw[->] (-6.1,0) -- (6.3,0) coordinate[label={$\lambda$}];
\draw[->] (0,-0.1) -- (0,{2*\sy+0.4}) coordinate[label={$c$}];

\foreach \x in {-1,...,1}{
	\draw ({\x*\sx},-0.1) -- ({\x*\sx},0.1);
	\node at ({\x*\sx},-0.1) [below] {$\x$};
}

\draw (-0.1,{0.5*\sy}) -- (0.1,{0.5*\sy});
\draw (-0.1,{1.0*\sy}) -- (0.1,{1.0*\sy});
\draw (-0.1,{1.5*\sy}) -- (0.1,{1.5*\sy});
\draw (-0.1,{2.0*\sy}) -- (0.1,{2.0*\sy});

\node at (-0.1,{0.5*\sy}) [left] {$c=\frac12$};
\node at (-0.1,{1.0*\sy}) [left] {$c=1$};
\node at (-0.1,{1.5*\sy}) [left] {$c=\frac32$};
\node at (-0.1,{2.0*\sy}) [left] {$c=2$};

%
% implizitspektrum.tex -- spectrum for the implizit method
%
% (c) 2020 Prof Dr Andreas Müller, Hochschule Rapperswil
%
\documentclass[tikz]{standalone}
\usepackage{amsmath}
\usepackage{times}
\usepackage{txfonts}
\usepackage{pgfplots}
\usepackage{csvsimple}
\usetikzlibrary{arrows,intersections,math}
\begin{document}
\def\skala{1}
\begin{tikzpicture}[>=latex,thick,scale=\skala]

\def\sx{6}
\def\sy{3}
\def\r{0.02}

\fill[color=gray!20] ({-1*\sx},0) rectangle ({1*\sx},{2*\sy});

\draw[->] (-6.1,0) -- (6.3,0) coordinate[label={$\lambda$}];
\draw[->] (0,-0.1) -- (0,{2*\sy+0.4}) coordinate[label={$c$}];

\foreach \x in {-1,...,1}{
	\draw ({\x*\sx},-0.1) -- ({\x*\sx},0.1);
	\node at ({\x*\sx},-0.1) [below] {$\x$};
}

\draw (-0.1,{0.5*\sy}) -- (0.1,{0.5*\sy});
\draw (-0.1,{1.0*\sy}) -- (0.1,{1.0*\sy});
\draw (-0.1,{1.5*\sy}) -- (0.1,{1.5*\sy});
\draw (-0.1,{2.0*\sy}) -- (0.1,{2.0*\sy});

\node at (-0.1,{0.5*\sy}) [left] {$c=\frac12$};
\node at (-0.1,{1.0*\sy}) [left] {$c=1$};
\node at (-0.1,{1.5*\sy}) [left] {$c=\frac32$};
\node at (-0.1,{2.0*\sy}) [left] {$c=2$};

\input{implizitspektrum.inc}

\end{tikzpicture}
\end{document}



\end{tikzpicture}
\end{document}



\end{tikzpicture}
\end{document}



\end{tikzpicture}
\end{document}

