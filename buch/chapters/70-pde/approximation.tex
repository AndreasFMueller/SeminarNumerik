%
% approximation.tex
%
% (c) 2020 Prof Dr Andreas Müller, Hochschule Rapperswil
%
\subsection{Approximation
\label{pde:fem:subsection:approximation}}
Das äquivalente Minimalproblem zu einer partiellen Differentialgleichung
hat das Problem etwas vereinfacht, die Ordnung der Ableitung ist
reduziert worden, aber ist ist immer noch ein Problem, in dem eine
Funktion bestimmt werden muss, also ein unendlichdimensionales Problem.
In dieser Form ist es daher immer noch nicht einer effizienten
numerischen Lösung zugänglich.

Zur Illustration soll in diesem Abschnitt das folgende Problem 
gelöst werden.
Auf dem Interval $\Omega=[0,\pi]$ ist eine Funktion $f(x)$ gegeben.
Gesucht ist eine Funktion $u\colon [0,\pi]\to\mathbb R$ derart, dass
\begin{equation}
\begin{aligned}
u''(x) &= f(x)
\\
u(0) &= u_0 & u(\pi)&= u_1.
\end{aligned}
\label{buch:pde:approx:beispiel}
\end{equation}
Als erstes müssen wir das äquivalente Minimalproblem finden.

\begin{problem}
Sei $F(x)$ eine Stammfunktion von $f(x)$.
Eine Lösung des Differentialgleichungsproblems~\eqref{buch:pde:approx:beispiel}
minimiert
\[
I(u)
=
\int_0^\pi (u'(x) - F(x))^2\,dx.
\]
\end{problem}

\begin{proof}[Beweis]
Die Richtungsableitung von $I(u)$ ist
\begin{align*}
\frac{dI}{d\varepsilon}(u+\varepsilon h)\bigg|_{\varepsilon=0}
&=
\int_0^\pi
2(u'(x)+\varepsilon h'(x)-F(x)) h'(x) 
\,dx
\\
&=
2
\int_0^\pi
(
u'(x)
-
F(x)
)
h'(x)
\,dx
\\
&=
2\biggl[(u'(x)-F(x))h(x)\biggr]_0^\pi
-
2\int_0^\pi
(u''(x)
-
F'(x))
h(x)
\,dx
\intertext{Der erste Term verschwindet wegen $h(0)=h(\pi)=0$ und es bleibt}
&=
2\int_0^\pi
(u''(x)
-
f(x))
h(x)
\,dx
=0.
\end{align*}
Dies muss für alle $h(x)$ gelten, so dass folgt
$u''(x) -f(x)=0$ oder $u''(x)=f(x)$.
\end{proof}

Die Stammfunktion $F(x)$ ist nicht eindeutig bestimmt, vielmehr
ist jede $F(x)+C$ ebenfalls eine Stammfunktion.
Nach obiger Rechnung führt sie jedoch auf die gleichen Minima.

Wir können aber davon ausgehen, dass die Lösungsfunktionen einigermassen
glatt sind, also sich gut durch ein Polynom approximieren lässt.
Wir approximieren jetzt $u(x)$ als Polynom und schreiben
\[
u(x) = a_0 + a_1x + a_2x^2 + \dots + a_nx^n.
\]
Das Minimalprinzip für $u(x)$ führt auf ein Minimalprinzip für 
die Koeffizienten $a_k$.
Zunächst brauchen wir die Ableitung von $u(x)$:
\begin{align*}
u'(x) &= \sum_{k=1}^n ka_kx^{k-1}.
\end{align*}
Jetzt müssen wir das Integral von
\[
(u'(x)-F(x))^2
=
u'(x)^2 - 2u'(x)F(x) + F(x)^2
\]
durch 
die Koeffizienten $a_k$ ausdrücken.
\begin{align*}
\int_0^\pi u'(x)^2\,dx
&=
\int_0^\pi
\sum_{i,j=1}^n ija_ia_jx^{i+j-2}
\,dx
=
\sum_{i,j=1}^n ij \int_0^\pi x^{i+j-2}\,dx\, a_ia_j
=
\sum_{i,j=1}^n
\underbrace{ij \frac{\pi^{i+j-1}}{i+j-1}}_{\displaystyle b_{ij}}
a_ia_j
=
a^tBa
\\
\int_0^\pi u(x)'F(x) \,dx
&=
\int_0^\pi \sum_{i=1}^n ia_i x^{i-1} F(x) \,dx
=
\sum_{i=1}^n
a_i
\underbrace{\int_0^\pi 
ix^{i-1} F(x)\,dx}_{\displaystyle =b_i}
=
b^t a
\\
\int_0^\pi F(x)^2\,dx &=: c
\end{align*}
Gesucht ist jetzt also ein Koeffizientenvektor $a$, der den
Ausdruck
\[
Q(a) = a^t B a + b^t a + c
\]
minimiert.
Der Summand $c$ kann natürlich weggelassen werden.

Die Randbedingungen müssen natürlich auch erfüllt sein, auch
diese können wir durch die Polynomkoeffizienten ausdrücken:
\begin{align*}
u_0=u(0)   &= a_0
\\
u_1=u(\pi) &= a_0 + a_1\pi + a_2\pi^2+\dots a_n\pi^n.
\end{align*}
Dies lässt sich auch in Matrixform mit der Matrix
\[
A=\begin{pmatrix}
1&0&0&\dots&0\\
1&\pi&\pi^2&\dots&\pi^n
\end{pmatrix}
\]
als
\[
Aa = \begin{pmatrix}u_0\\u_1\end{pmatrix}
\]
schreiben.

Wir machen das Problem noch etwas konkreter und verlangen $f(x)=1$
und $u_0=u_1=0$.
Dann ist $F(x)=x$ eine mögliche Stammfunktion und damit lässt sich
der Vektor $b$ berechnen als
\[
b_i
=
\int_0^\pi i F(x) x^{i-1}\,dx
=
\int_0^\pi i x^i\,dx
=
\frac{i}{i+1}
\pi^{i+1}.
\]
Die Matrix $B$ ist
\begin{align*}
B&=\begin{pmatrix}
\pi   & \pi^2            & \pi^3            \\
\pi^2 & \frac{4}{3}\pi^3 & \frac{3}{2}\pi^4 \\
\pi^3 & \frac{3}{2}\pi^4 & \frac{9}{5}\pi^5
\end{pmatrix}
\end{align*}
Aus den Randbedingung folgt $a_0=0$ und
\[
a_1\pi+a_2\pi^2=0
\qquad\Rightarrow\qquad
a_1 = \pi a_2.
\]
Die gesuchte Lösung muss sich also den Ausdruck
\[
a_2
\begin{pmatrix}
1\\\pi
\end{pmatrix}^t
\pi^3
\begin{pmatrix}
\frac43   &\frac32\pi\\
\frac32\pi&\frac95\pi^2
\end{pmatrix}
\begin{pmatrix}
1\\\pi
\end{pmatrix}
a_2
-
\begin{pmatrix}
\frac23\pi^3\\\frac34\pi^4
\end{pmatrix}
\begin{pmatrix}
1\\\pi
\end{pmatrix}
a_2
=
\biggl(
\frac43\pi^3+3\pi^5 +\frac95\pi^7
\biggr)
a_2^2
-\biggl(\frac23\pi^3+\frac34\pi^5\biggr)
a^2
\]
Gesucht wird also nur das Minimum eines quadratischen Ausdrucks 
in $a_2$, und dieses wird bei $a_2=\frac12$ angenommen.
Tatsächlich kann man nachprüfen, dass
\[
u(x)
=
\frac12\biggl(x-\frac{\pi}2\biggr)^2 -\frac{\pi^2}8
=
\frac12x^2 -\frac12 x\pi +\frac{\pi^2}8 - \frac{\pi^2}8
=
\frac12x(x-\frac{\pi}2)
\]
eine Lösung des Problems ist.







