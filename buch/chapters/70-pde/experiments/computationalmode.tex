%
% tikztemplate.tex -- template for standalon tikz images
%
% (c) 2020 Prof Dr Andreas Müller, Hochschule Rapperswil
%
\documentclass[tikz]{standalone}
\usepackage{amsmath}
\usepackage{times}
\usepackage{txfonts}
\usepackage{pgfplots}
\usepackage{csvsimple}
\usetikzlibrary{arrows,intersections,math}
\begin{document}
\def\skala{1}
\begin{tikzpicture}[>=latex,thick,scale=\skala]

\def\sx{1}
\def\sy{3}

\input{cm.tex}

\begin{scope}
\clip (0,0) rectangle (13.5,6);

%\draw[color=red] \floatthousand;
%\draw[color=red] \floattenthousand;
%\draw[color=red] \floathundredthousand;

%\draw[color=red] \doublethousand;
%\draw[color=red] \doubletenthousand;
%\draw[color=red] \doublehundredthousand;

\draw[color=red,line width=1.4pt] \longdoublehundred;
\draw[color=red,line width=1.4pt] \longdoublethousand;
\draw[color=red,line width=1.4pt] \longdoubletenthousand;
\draw[color=red,line width=1.4pt] \longdoublehundredthousand;

\end{scope}

\draw[color=blue,line width=1.4pt] plot[domain=0:13.5,samples=100]
	({\x},{3*exp(-\x)});

\draw[->] (-0.1,0) -- (13.7,0) coordinate[label={$x$}];
\draw[->] (0,-0.1) -- (0,6.5) coordinate[label={right:$y$}];

\foreach \x in {1,...,13}{
	\draw (\x,-0.1) -- (\x,0.1);
	\node at (\x,-0.1) [below] {$\x$};
}

\draw (-0.1,\sy) -- (0.1,\sy);
\draw (-0.1,{2*\sy}) -- (0.1,{2*\sy});
\node at (-0.1,{1*\sy}) [left] {$1$};
\node at (-0.1,{2*\sy}) [left] {$2$};
\node at (-0.1,-0.1) [below left] {$0$};

\node at (6.5,{1.6*\sy}) [left] {$h=10^{-2}$};
\node at (8.8,{1.6*\sy}) [left] {$h=10^{-3}$};
\node at (11.1,{1.6*\sy}) [left] {$h=10^{-4}$};
\node at (13.4,{1.6*\sy}) [left] {$h=10^{-5}$};

\end{tikzpicture}
\end{document}

