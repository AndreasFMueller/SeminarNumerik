%
% slides.tex -- XXX
%
% (c) 2017 Prof Dr Andreas Müller, Hochschule Rapperswil
%

\setbeamertemplate{itemize/enumerate body begin}{\large}
\setbeamertemplate{itemize/enumerate subbody begin}{\large}
\setbeamertemplate{itemize/enumerate subsubbody begin}{\large}
\setbeamertemplate{enumerate item}{\large\insertenumlabel.}
\setbeamertemplate{enumerate subitem}{\large\insertenumlabel.\insertsubenumlabel}
\setbeamertemplate{enumerate subsubitem}{%
  \large\insertenumlabel.\insertsubenumlabel.\insertsubsubenumlabel}

%\begin{frame}
%\frametitle{Vorträge: Ablauf}
%\begin{enumerate}
%\item<2-> Vortrag: Video oder BBB Präsentation. Für Video:
%\begin{itemize}
%\item<3-> Wiedergabe im Youtube-Stream  (Link im Moodle Block {\em Vorträge})
%\item<4-> Selber schauen via Youtube-Link im Moodle Block zum Vortrags-Thema
%\end{itemize}
%\item<5-> Fragerunde: Web-Konferenz in BBB
%\item<6-> Beurteilung: im Moodle, Block {\em Vorträge}
%\bigskip
%\item<7-> Fragezeit bis 18:40
%\end{enumerate}
%\end{frame}

\begin{frame}
\frametitle{Vorträge 25.~Mai 2020}
\begin{enumerate}
\item<2->
Joel Rechsteiner: Finite Elemente in der Ebene
\bigskip

\item<3->
Michael Schmid: Lösungen der Gleichung von Burgers
\bigskip

\item<4->
Cédric Renda: Padé-Approximation

\end{enumerate}
\end{frame}


