%
% sekantenfehler.tex
%
% (c) 2020 Prof Dr Andreas Müller, Hochschule Rapperswil
%
\begin{frame}
\frametitle{Fehler des Sekantenverfahrens}
\begin{block}{Verbesserung des Auslöschungsproblems}
\begin{align*}
x_{n+1}
&=
\frac{x_nf(x_{n-1}) {\color{red}\mathstrut\only<2->{-x_nf(x_n)+x_nf(x_n)}}- x_{n-1} f(x_n)}{f(x_{n-1})-f(x_n)}
\\
&\uncover<3->{=
x_n \only<3>{+}\only<4->{-} f(x_n) \frac{x_n-x_{n-1}}{
\only<3>{f(x_{n-1})-f(x_n)}
\only<4->{f(x_{n})-f(x_{n-1})}
}}
\\
&\uncover<5->{=
\underbrace{x_n}_{\displaystyle\text{beste Approximation}}
-\underbrace{
f(x_n)\frac{x_n-x_{n-1}}{f(x_n)-f(x_{n-1})}
}_{\displaystyle\text{Korrektur}}}
\end{align*}
\uncover<6->{Korrektor ist klein wenn $f(x_n)$ klein ist!}
\end{block}
\end{frame}
