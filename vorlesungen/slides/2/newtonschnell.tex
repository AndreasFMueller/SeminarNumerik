%
% newtonschnell.tex
%
% (c) 2020 Prof Dr Andreas Müller, Hochschule Rapperswil
%
\begin{frame}
\frametitle{Konvergenzgeschwindigkeit}
\vspace{-15pt}
\begin{columns}[t]
\begin{column}{0.48\hsize}
\begin{block}{Aufgabe}
Wie muss ein Iterationsverfahren
\begin{align}
x_{n+1}
&=x_n + \text{Korrektur}
\notag
\\
&\uncover<2->{=
x_n + a(x_n) \cdot f(x_n)}
\label{fixpunkt}
\end{align}
aufgebaut sein, damit die Konvergenz quadratisch wird?
\end{block}
\begin{block}{Fixpunktiteration}
\eqref{fixpunkt} ist ein Fixpunktiterationsverfahren 
\[
x_{n+1}=g(x_n)
\]
\vspace{-5pt}
mit
\vspace{-5pt}
\[
g(x) = x + a(x)\cdot f(x)
\]
\end{block}
\end{column}
\begin{column}{0.48\hsize}
\uncover<3->{%
\begin{block}{Kriterium}
Aus Kapitel 1: $g'(x^*)=0$
\uncover<4->{
\begin{align*}
g'(x) &= 1 + a'(x)\cdot f(x) + a(x) \cdot f'(x)
\\
\uncover<5->{0}&\uncover<5->{= 1 + a'(x^*)\cdot \underbrace{f(x^*)}_{\displaystyle=0} + a(x^*) \cdot f'(x^*)}
\\
\uncover<6->{a(x^*)}&\uncover<6->{=-\frac{1}{f'(x^*)}}
\intertext{\uncover<7->{$x^*$ unbekannt: }}
\uncover<8->{a(x)}&\uncover<8->{=-\frac1{f'(x)}}
\\
\uncover<9->{g(x)}&\uncover<9->{=x -\frac{f(x)}{f'(x)}}\quad\uncover<10->{\text{Newton!}}
\end{align*}}
\end{block}}
\end{column}
\end{columns}
\end{frame}
