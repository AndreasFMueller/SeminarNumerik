%
% beispiele.tex
%
% (c) 2020 Prof Dr Andreas Müller, Hochschule Rapperswil
%
\begin{frame}
\frametitle{Beispiele}
\begin{columns}[t]
\begin{column}{0.40\hsize}
\begin{block}{Poisson-Problem}
$u$ Potential, $\varrho$ Ladungsdichte
\[
\Delta u = 4\pi \varrho
\]
elliptisches Problem
\end{block}
\end{column}
\begin{column}{0.56\hsize}
\uncover<2->{%
\begin{block}{Wellen}
$u$ ``Auslenkung'', $a$ Wellengeschwindigkeit
\[
\frac{\partial^2 u}{\partial t^2} = a^2\Delta u
\]
hyperbolisches Problem
\end{block}}
\end{column}
\end{columns}
\begin{columns}[t]
\begin{column}{0.40\hsize}
\uncover<3->{%
\begin{block}{Wärmeleitung}
$u$ Temperatur
\[
\frac{\partial u}{\partial t}  = \Delta u
\]
parabolisches Problem
\end{block}}
\end{column}
\begin{column}{0.56\hsize}
\uncover<4->{%
\begin{block}{Gleichung von Burgers}
Nichtlineare Wellengleichung
\[
\frac{\partial u}{\partial t} + u\frac{\partial u}{\partial t} = 0
\]
quasilineares Problem 1. Ordnung
\end{block}}
\end{column}
\end{columns}
\end{frame}
