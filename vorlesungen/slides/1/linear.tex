%
% linear.texx
%
% (c) 2020 Prof Dr Andreas Müller, Hochschule Rapperswil
%
\begin{frame}
\frametitle{Lineare Konvergenz}
\begin{block}{Grundfrage}
Wie entwickelt sich der Fehler $\delta_n = x_n - x^*$?
\end{block}

\uncover<2->{
\begin{block}{Approximation}
\vspace{-15pt}
\[
x_{n+1} = f(x_n) = f(x^* + \delta_n) = f(x^*) + f'(x^*)\delta_n + \frac12 f''(x^*)\delta_n^2+\dots
\]
\end{block}}

\vspace{-10pt}

\uncover<3->{
\begin{block}{Fehlerentwicklung}
\vspace{-20pt}
\begin{align*}
x_{n+1} = x^* + \delta_{n+1}
&= f(x_n) = x^* + f'(x^*) \delta_n + \frac12f''(x^*)\delta_n^2 + O(\delta_n^3)
\\
\delta_{n+1}
&=
f'(x^*)\delta_n + \frac12f''(x^*)\delta_n^2 + O(\delta_n^3);
\end{align*}
\uncover<4->{
$\Rightarrow$ Fehler ändert sich in jeder Iteration um den Faktor $f'(x^*)$:
\begin{enumerate}
\item<5-> Konvergenz für $|f'(x^*)|<1$
\item<6-> Anzahl korrekter Stellen nimmt in jeder Iteration um gleich viel zu
\end{enumerate}
}
\end{block}

}
\end{frame}
