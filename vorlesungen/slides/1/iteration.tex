%
% iteration.tex
%
% (c) 2020 Prof Dr Andreas Müller, Hochschule Rapperswil
%
\begin{frame}
\frametitle{Iteration}
\vspace{-15pt}
\begin{columns}[t]
\begin{column}{0.47\hsize}
\uncover<1->{
\begin{block}{Analysis}
Formeln:
\begin{align*}
\Gamma(x)
&=
\int_0^\infty t^{x-1}e^{-t}\,dt
\\
n!
&=
\Gamma(n+1)
\\
y&=f(x)
\end{align*}
$\Rightarrow$
Problemlösung {\em in geschlossener Form}
\end{block}}
\end{column}
\begin{column}{0.47\hsize}
\uncover<2->{
\begin{block}{Numerik}
Problemlösung mit einem Programm
\begin{center}
\verbatiminput{../slides/1/numerikprogramm.txt}
\end{center}
\end{block}}
\end{column}
\end{columns}

\bigskip

\uncover<3->{
\begin{block}{Iteration}
$\Rightarrow$ Verhalten von Iterationen der Form $x=f(x)$ studieren
\end{block}
}
\end{frame}
