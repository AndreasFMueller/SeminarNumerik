%
% ausloeschung.tex
%
% (c) 2020 Prof Dr Andreas Müller, Hochschule Rapperswil
%

\begin{frame}
\frametitle{Auslöschung}
\begin{definition}
{\em Auslöschung} ist Genauigkeitsverlust, der bei der Subtraktion
fast gleich grosser Zahlen entsteht.
\end{definition}

\begin{block}{Beispiel}
\begin{center}
\ifthenelse{\boolean{presentation}}{\small}{}
\renewcommand\arraystretch{1.2}
\begin{tabular}{|>{$}c<{$}|>{\tt}r|>{\tt}r|}
\hline 
              & \textnormal{Gleikommawert}        &\textnormal{Festkommawert}\\
\hline
\sqrt{10}     & 0 10000000 10010100110001011000010&11.0010100110001011000010\\
\pi           & 0 10000000 10010010000111111011011&11.0010010000111111011011\\
\hline
\sqrt{10}-\pi & 0 01111001 01010010111001110000000& 0.0000010101001011100111\\
\hline
\end{tabular}
\end{center}
\uncover<2->{
$\Rightarrow$ Genauigkeit nur noch 17 statt 24 bit, 7 bit Verlust
}
\end{block}
\end{frame}

%
% normalverteilung
%
\begin{frame}
\frametitle{Auslöschung bei Normalverteilung}
\begin{block}{Beispiel}
$X$ standardnormalverteilte Zufallsvariable, berechnet 
\[
p
=
P(a\le X < b)
=
\Phi(b)- \Phi(a)
=
\frac{1}{\sqrt{2\pi}}
\int_a^b  e^{-x^2/2}\,dx
\]
\end{block}
\vspace{-20pt}
\uncover<2->{
\begin{block}{Lösung}
Mit Fehlerfunktion $\operatorname{erf}(x)$ oder komplementärer Fehlerfunktion
$\operatorname{erfc}(x)=1-\operatorname{erf}(x)$
\[
\operatorname{erf}(x)
=
\frac{2}{\sqrt{\pi}}\int_0^xe^{-t^2}\,dt\uncover<3->{,
\quad
\Phi(x) = \frac12 + \operatorname{erf}(x)}
\uncover<4->{
\quad\Rightarrow\quad
p=\begin{cases}
\operatorname{erf}(b)-\operatorname{erf}(a)&\\
\operatorname{erfc}(a)-\operatorname{erfc}(b)&
\end{cases}}
\]
\end{block}}
\vspace{-15pt}
\uncover<5->{
\[
\left.
\begin{aligned}
a&=4.18\\
b&=5.18
\end{aligned}\right\}
\Rightarrow
\begin{tabular}{|l|>{$}r<{$}|>{$}r<{$}|}
\hline
Datentyp       & \textnormal{Rechnung mit $\operatorname{erf}(x)$}&
\textnormal{Rechnung mit $\operatorname{erfc}(x)$}\\
\hline
\texttt{float} & 0.000000\phantom{\mathstrut\cdot10^{-00}}&
6.271826\cdot10^{-17}\\
\texttt{double}& 1.110223\mathstrut\cdot10^{-16}&
6.271826\cdot10^{-17}\\
\texttt{long}  & 6.272110\mathstrut\cdot10^{-17}&
6.271826\cdot10^{-17}\\
\hline
\end{tabular}
\]
}
\end{frame}
