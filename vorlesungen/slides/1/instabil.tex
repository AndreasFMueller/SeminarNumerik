%
% instabil.tex
%
% (c) 2020 Prof Dr Andreas Müller, Hochschule Rapperswil
%
\begin{frame}
\frametitle{Instabilität und Kondition}
\begin{columns}[t]
\begin{column}{0.48\hsize}
\begin{block}{Kondition}
``Gute'' Kondition: Rundungsfehler werden nicht
vergrössert
\end{block}
\uncover<2->{%
Vermeiden:
\begin{itemize}
\item<3-> Empfindliche Abhängigkeit der Resultate von Parametern,
\[
\frac{\partial f}{\partial x} = \text{gross}
\]
\item<4-> Division durch kleine Nenner
\end{itemize}}
\end{column}
\begin{column}{0.48\hsize}
\uncover<5->{%
\begin{block}{Instabilität}
Unbrauchbare Resultate, weil Rundungsfehler im Laufe der Rechnung
über alle sinnvollen Grenzen aufgebläht werden.
\end{block}}
\uncover<6->{%
Vermeiden:
\begin{itemize}
\item<7-> Mit stark von Auslöschung betroffenen Zahlen rechnen
\item<8-> Wiederholt mit grossen Zahlen multiplizieren
\end{itemize}}
\end{column}
\end{columns}
\end{frame}
