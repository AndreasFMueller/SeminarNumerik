%
% approximation.tex
%
% (c) 2020 Prof Dr Andreas Müller, Hochschule Rapperswil
%
\begin{frame}
\frametitle{Approximation des Integrals}
\uncover<2->{%
\begin{block}{Abschätzungen für Funktionswerte}
In jedem Teilintervall $[x_{k-1},x_k]$ gilt
\[
x\in[x_{k-1},x_k]
\qquad\Rightarrow\qquad
\underline{f}_k \le f(x) \le \overline{f}_k
\]
\end{block}}
\uncover<3->{%
\begin{block}{Abschätzungen für Ober- und Untersummen}
Für beliebige $\tilde{x}_k\in[x_{k-1},x_k]$ folgt
\[
\underline{I}
=
\sum_{k=1}^n \underline{f}_k\cdot(x_{k}-x_{k-1})
\le
\sum_{k=1}^n f(\tilde{x}_k)\cdot(x_{k}-x_{k-1})
\le
\sum_{k=1}^n \overline{f}_k\cdot(x_{k}-x_{k-1})
=
\overline{I}
\]
\end{block}}
\vspace{-5pt}
\uncover<4->{%
\begin{block}{Grenzwert}
\vspace{-10pt}
\[
I
=
\lim_{n\to\infty,\delta\to 0}
\sum_{k=1}^n f(\tilde{x}_k)\cdot(x_{k}-x_{k-1})
\]
\end{block}}
\end{frame}

