%
% korn.tex
%
% (c) 2020 Prof Dr Andreas Müller, Hochschule Rapperswil
%
\begin{frame}
\frametitle{Korn und Grenzwert}
\begin{block}{Korn}
Länge des längsten Teilintervalls der Unterteilung $x_0,x_1,\dots,x_{n-1},x_n$
\[
\delta = \max_{1\le k\le n} |x_{k}-x_{k-1}|
\]
\vspace{-15pt}
\end{block}
\uncover<2->{
\begin{block}{Riemann-integrierbar}
Die Funktion $f\colon [a,b]\to\mathbb R$
ist Riemann-integrierbar, wenn
\[
\lim_{n\to\infty,\delta\to 0} \underline{I}
=
\lim_{n\to\infty,\delta\to 0} \overline{I}
=
I
\]
\uncover<3->{%
Der gemeinsame Grenzwert ist das {\em Riemann-Integral von $f(x)$}}
\end{block}}
\uncover<4->{
\begin{block}{Beispiele}
\begin{itemize}
\item
$f\colon [a,b]\to\mathbb R$ stetig $\Rightarrow$ $f$ Riemann-integrierbar
\item
$f\colon [a,b]\to\mathbb R$ stetig ausser in endlich vielen Punkten
in $[a,b]$
\end{itemize}
\end{block}}

\end{frame}
