%
% rkfehler.tex
%
% (c) 2020 Prof Dr Andreas Müller, Hochschule Rapeprswil
%
\begin{frame}
\frametitle{Fehler des Runge-Kutta-Verfahrens}
\begin{block}{Runge-Kutta-Inkrementfunktion}
\vspace{-20pt}
\begin{align*}
k_1 &= f(x,y)
\\
k_2 &= f\biggl(x+\frac{h}2, y+\frac{h}2 k_1\biggr)
\\
k_3 &= f\biggl(x+\frac{h}2, y+\frac{h}2 k_2\biggr)
\\
k_4 &= f(x+h, y+hk_3)
\\
F(x,y,h)
&=
\frac{1}{6} ( k_1 +2k_2 + 2k_3 +k_4)
&&\uncover<2->{\Rightarrow}
&\uncover<2->{y(x+h)-y(x) - hF(x,y,h)}&\uncover<2->{=O(h^5)}
\end{align*}
\vspace{-18pt}
\end{block}
\uncover<3->{%
\begin{block}{Verfahrensfehler}
$k=t/h$ Schritte mit Schrittweite $h=t/k$
\vspace{-5pt}
\[
\uncover<4->{\text{Verfahrensfehler}}
\uncover<5->{ = k\cdot O(h^5) }
\uncover<6->{ = \frac{t}{h}\cdot O(h^5) }
\uncover<7->{ = O(h^4) }
\]
\end{block}}
\end{frame}
