%
% schiessmethode.tex
%
% (c) 2020 Prof Dr Andreas Müller, Hochschule Rapperswil
%
\begin{frame}
\frametitle{Schiessmethode}
\vspace{-15pt}
\begin{columns}[t]
\begin{column}{0.48\hsize}
\begin{block}{Randwertproblem}
$f\colon \mathbb R\times\mathbb R^n\times\mathbb R^n\to\mathbb R^n$:
\begin{align*}
\frac{d^2}{dt^2} x &= f(t,x,\dot{x})
\\
x(0)&=x_0
\\
{\color{red}
x(1)}&{\color{red}=x_1}
\end{align*}
\end{block}
\uncover<2->{%
\begin{block}{Umwandlung: Anfangswertproblem}
Finde $\color{red}v_0$ derart, dass
\[
x(1,x_0,{\color{red}v_0}) = x_1
\]
$n$ Gleichungen für ${\color{red}v_0}\in\mathbb R^n$ 
\end{block}}
\end{column}
\begin{column}{0.48\hsize}
\uncover<3->{%
\begin{block}{Lösung mit Newton}
Lösung $x(t,v_0)$ erfüllt
\vspace{-5pt}
\begin{align*}
\frac{d^2}{dt^2} x(t,v_0)
&= 
f(t,x(t,v_0),\dot{x}(t,v_0))
\\[-7pt]
\intertext{\uncover<4->{Ableiten nach $v_0$}}
\\[-27pt]
\uncover<5->{
\frac{d^2}{dt^2}\frac{\partial x}{\partial v_0}}
&\uncover<5->{=
\frac{\partial f}{\partial x}\cdot \frac{\partial x}{\partial v_0}
+
\frac{\partial f}{\partial v}\cdot \frac{\partial \dot{x}}{\partial v_0}}
\\
\uncover<6->{
\frac{d^2}{dt^2} X}
&\uncover<6->{=
\frac{\partial f}{\partial x}\cdot X
+
\frac{\partial f}{\partial v} \cdot \dot{X}}
\\
\uncover<7->{
\text{mit}\quad
X(0)}&\uncover<7->{= 0,\quad \dot{X}(0)=E}
\end{align*}
\uncover<8->{%
Newton-Verfahren mit $X = X(1)$:
\[
v_{0,n+1}
=
v_{0,n}
-
X^{-1}\cdot (x(1,v_{0,n}) - x_1)
\]}
\end{block}}
\end{column}
\end{columns}
\end{frame}
