%
% verbesserung.tex
%
% (c) 2020 Prof Dr Andreas Müller, Hochschule Rapperswil
%
\begin{frame}
\frametitle{Verbesserung der numerischen Lösung $\tilde{y}(x)$}
\vspace{-15pt}
\begin{columns}[t]
\begin{column}{0.40\hsize}
\begin{block}{Einschritt-Verfahren}
Berechung von $\tilde{y}(x+h)$ aus $\tilde{y}(x)$:
\uncover<2->{%
\[
\tilde{y}(x+h) = \tilde{y}(x) + h\cdot F(x,\tilde{y},h)
\]
$F$ heisst Inkrementfunktion
\[
F(x,\tilde{y},h) \simeq y'(x) + \frac{y''(x)}{2!}h+\dots
\]}
\end{block}
\uncover<6->{%
\vspace{-10pt}
\begin{block}{Fehler}
Der Fehler hängt davon ab, wie gut $hF(x,\tilde{y},h)$ die 
das Inkrement $\tilde{y}(x+h)-\tilde{y}(x)$ approximiert.
\end{block}}
\end{column}
\begin{column}{0.56\hsize}
\uncover<3->{%
\begin{block}{Euler-Verfahren}
\vspace{-15pt}
\begin{align*}
\tilde{y}(x+h)
&=
\tilde{y}(x) + h\cdot f(x,\tilde{y})
\\
F(x,\tilde{y},h)
&=
f(x,\tilde{y})
\end{align*}
\end{block}}
\uncover<4->{%
\begin{block}{Verbessertes Euler-Verfahren}
\vspace{-15pt}
\begin{align*}
F(x,\tilde{y},h)
&=
f\biggl(x+\frac{h}2, \tilde{y} + \frac{h}2 f(x,\tilde{y}) \biggr)
\end{align*}
\end{block}}
\uncover<5->{%
\begin{block}{Vereinfachtes Runge-Kutta-Verfahren}
\vspace{-15pt}
\begin{align*}
F(x,\tilde{y},h)
&=
\frac{1}2\bigl(
f(x,\tilde{y}) + f(x+h, \tilde{y}+hf(x,\tilde{y}))
\bigr)
\end{align*}
\end{block}}
\end{column}
\end{columns}
\end{frame}
