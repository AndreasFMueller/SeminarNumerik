%
% parameter.tex
%
% (c) 2020 Prof Dr Andreas Müller, Hochschule Rapperswil
%
\begin{frame}
\frametitle{Ableitung nach Parametern}
\vspace{-15pt}
\begin{columns}[t]
\begin{column}{0.48\hsize}
\begin{block}{Voraussetzungen}
Differentialgleichung
\vspace{-5pt}
\begin{align*}
\frac{dy}{dx} &= f(x,y,p),& y(0) &= y_0
\end{align*}
\vspace{-5pt}
hat Lösung $y=y(x,p)$
\end{block}
\uncover<2->{%
\begin{block}{Ableitungsmatrix $\partial y(x,p)/\partial p$}
DGL nach $p$ ableiten
\begin{align*}
\frac{dy(x,p)}{dx}
&=
f(x,y(x,p),p)
\\
\uncover<3->{
\Rightarrow\;
\frac{d}{dx}
\frac{\partial y}{\partial p}}
&\uncover<3->{=
\frac{\partial f}{\partial y} \cdot \frac{\partial y}{\partial p}
+
\frac{\partial f}{\partial p}}
\end{align*}
\uncover<4->{%
Anfangsbedingung: $\partial y(0,p)/\partial p=0$}%
\end{block}}%
\end{column}
\begin{column}{0.48\hsize}
\uncover<5->{%
\begin{block}{Beispiel}
Für das Eigenwertproblem, $p=\lambda$
\vspace{-5pt}
\[
u''(x)=\lambda u(x)
\uncover<6->{%
\Rightarrow
\frac{d}{dx}
\begin{pmatrix}u_1\\u_2\end{pmatrix}
=
\begin{pmatrix} u_2\\\lambda u_1
\end{pmatrix}}
\]
\vspace{-12pt}

\uncover<7->{%
Ableitungen:
\vspace{-5pt}
\begin{align*}
\frac{\partial f(x,y,\lambda)}{\partial y}
&=
\begin{pmatrix} 0&1\\ \lambda&0\end{pmatrix}
\\
\frac{\partial f(x,y,p)}{\partial \lambda}
&=
\begin{pmatrix} 0\\ 1 \end{pmatrix}
\end{align*}}%
\vspace{-10pt}

\uncover<8->{%
Differentialgleichung:
\begin{align*}
\frac{dP}{dx} 
&= 
\begin{pmatrix} 0&1\\ \lambda&0\end{pmatrix} P
+ 
\begin{pmatrix} 0\\ 1 \end{pmatrix}
\end{align*}}%
\end{block}}
\end{column}
\end{columns}
\end{frame}
