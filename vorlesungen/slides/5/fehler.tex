%
% fehler.tex
%
% (c) 2020 Prof Dr Andreas Müller, Hochschule Rapperswil
%
\begin{frame}
\frametitle{Fehler des Euler-Verfahrens}
\begin{block}{Beiträge}
Lösung zwischen $x=0$ und $x=t$ in $k$ Schritten: $h=\frac{t}{k}$
\begin{enumerate}
\item<2->
Fehler in jedem Schritt 
\vspace{-10pt}
\begin{align*}
&\uncover<3->{\text{Taylorreihe:}}&
\uncover<3->{y(x+h)}
&\uncover<3->{= 
y(x) + y'(x)\cdot h {\color<5->{red}\mathstrut + \frac{y''(x)}{2!}\cdot h^2 +\frac{y'''(x)}{3!}\cdot h^3+\dots}}
\\
&\uncover<4->{\text{Eulerschritt:}}&
\uncover<4->{\tilde{y}(x+h)}
&\uncover<4->{=
y(x) + y'(x)\cdot h}
\\
&\uncover<5->{\text{Fehler:}}&
\uncover<5->{\Delta y}
&\uncover<5->{=
\phantom{ y(x) + y'(x)\cdot h+\mathstrut}
O(h^2)}
\end{align*}
\vspace{-15pt}
\item<6->
Anzahl Schritte: $k=t/h$
\vspace{-5pt}
\[
\uncover<7->{
\text{Gesamter Fehler:}\qquad \frac{t}{h}\cdot O(h^2) = O(h)
}
\]
\end{enumerate}
\end{block}
\vspace{-25pt}
\uncover<8->{%
\begin{block}{Konvergenz}
Halbierung von $h$
$\Rightarrow$
doppelte Arbeit aber nur 1 bit Genauigkeitsgewinn
\uncover<9->{{\color{red}$\Rightarrow$ bad}}
\end{block}}

\end{frame}
