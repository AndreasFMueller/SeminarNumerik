%
% picard.tex
%
% (c) 2020 Prof Dr Andreas Müller, Hochschule Rapperswil
%
\begin{frame}
\frametitle{Der Satz von Picard-Lindelöf}
\begin{block}{Satz}
Die Lösung $\vec{y}\colon \mathbb R\to\mathbb R^n: x\mapsto\vec{y}$ der
Differentialgleichung 
\[
\frac{d}{dx} \vec{y} = F(x,\vec{y}),
\qquad
\vec{y}(0)=\vec{g}
\]
existiert in einer Umgebung von $x=0$ und
ist eindeutig bestimmt, wenn $F$ eine Lipshitz-Bedingung erfüllt:
\[
\exists L>0, \alpha>0\qquad
|F(x, \vec{a}) - F(x,\vec{b})|
\le L \,|\vec{a} - \vec{b}|
\quad
\forall \vec{a},\vec{b}\in \mathbb R^n
\]
\end{block}
\uncover<2->{%
{\usebeamercolor[fg]{title}
$\Rightarrow$}
Existenz einer Lösung ist praktisch immer garantiert.}
\end{frame}
