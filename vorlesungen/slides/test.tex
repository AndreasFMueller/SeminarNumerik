%
% test.tex collection of all slides
%
% (c) 2019 Prof Dr Andreas Müller, Hochschule Rapperswil
%

%\folie{3/problem.tex}
%\folie{3/linear.tex}
%\folie{3/polygon.tex}
%\folie{3/lagrange.tex}
%\folie{3/gleichungen.tex}
%\folie{3/nullstellenschachtelung.tex}
%\folie{3/fehlerformel.tex}
%\folie{3/runge.tex}
%\folie{3/tschebyscheff.tex}
%\folie{3/lissajous.tex}
%\folie{3/tschebyscheffpoly.tex}
%\folie{3/hermite.tex}
%\folie{3/baryzentrisch.tex}
%\folie{4/mittelpunkt.tex}
%\folie{4/trapez.tex}
%\folie{4/plan.tex}
%\folie{4/romberg.tex}
%\folie{4/rombergbeispiel.tex}
%\folie{5/reduktion.tex}
%\folie{5/reduktionbeispiel.tex}
%\folie{5/picard.tex}
%\folie{5/euler.tex}
%\folie{5/fehler.tex}
%\folie{5/verbesserung.tex}
%\folie{5/rungekutta.tex}
%\folie{5/rkfehler.tex}
%\folie{5/mehrschritt.tex}
%\folie{5/richtungsfeld.tex}
%\folie{5/sir.tex}
%\folie{5/faktoren.tex}
%\folie{5/sirsim.tex}

\folie{6/fem/introbeispiel.tex}
\folie{6/fem/integralprinzip.tex}
\folie{3/spline/fe.tex}
\folie{3/spline/forderungen.tex}
\folie{3/spline/variationsproblem.tex}
\folie{3/spline/eulerlagrange.tex}
\folie{3/spline/mechanik.tex}
\folie{3/spline/minimal.tex}
\folie{3/spline/zweiteabl.tex}
\folie{3/spline/hermite.tex}
\folie{3/spline/bedingungen.tex}
\folie{3/spline/gleichungssystem.tex}



