%
% tschebyscheffpoly.tex
%
% (c) 2020 Prof Dr Andreas Müller, Hochschule Rapperswil
%
\begin{frame}
\frametitle{Tschebyscheff-Polynome}
\vspace{-10pt}
\begin{columns}[t]
\begin{column}{0.38\hsize}
\begin{block}{Lissajous-Figur}
$t\in[0,\pi]$, $n\ge 0$
\begin{align*}
x&=\cos t,&
y&=\cos nt
\end{align*}
\end{block}
\vspace{-15pt}
\uncover<2->{%
\begin{block}{Aufgabe}
$y$ als Polynom $x$ ausdrücken: 
$y=T_n(x)$
\end{block}}
\vspace{-5pt}
\uncover<3->{%
\begin{block}{Beispiele}
\vspace{-20pt}
\begin{align*}
T_0(x)&=1\\
T_1(x)&=x\\
T_2(x)&=\cos 2t = 2\cos^2t-1\\
      &= 2x^2-1
\end{align*}
\end{block}}
\end{column}
\begin{column}{0.58\hsize}
\uncover<4->{%
\begin{block}{Rekursion}
Additionstheorem:
\begin{align*}
\cos (n+1)t+\cos(n-1)t&=2\cos nt\,\cos t
\\
\uncover<5->{%
\Rightarrow\quad
T_{n+1}(x) + T_{n-1}(x)}&\uncover<5->{= 2T_n(x) \, x}
\end{align*}
\uncover<6->{Rekursionsformel:
\[
T_{n+1}(x) = 2xT_n(x) - T_{n-1}(x)
\]}
\uncover<7->{$\Rightarrow$ alle $T_n(x)$ sind Polynome}
\end{block}}
\uncover<8->{%
\begin{block}{Nullstellen}
\[
x_k = \cos \frac{2k-1}{2n}\pi,\quad k=1,\dots,n
\]
\end{block}}
\end{column}
\end{columns}
\end{frame}

