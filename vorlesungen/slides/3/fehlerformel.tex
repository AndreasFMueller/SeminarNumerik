%
% fehlerformel.tex
%
% (c) 2020 Prof Dr Andreas Müller, Hochschule Rapperswil
%
\begin{frame}
\frametitle{Fehler des Interpolationspolynoms}
\vspace{-10pt}
\begin{columns}[t]
\begin{column}{0.34\hsize}
\begin{block}{Fehlerformel}
$p(x)$ ein Polynom vom Grad $n$ mit $f(x_i) = p(x_i)$, $x\in [a,b]$,
dann gibt es $\xi_x\in[a,b]$ derart, dass
\[
f(x)-p(x)
=
\frac{l(x)}{(n+1)!}f^{(n+1)}(\xi_x)
\]
\end{block}
\uncover<11->{
\begin{block}{Fehler limitiert durch}
\begin{itemize}
\item<12-> $\max_{x} |l(x)|$
\item<13-> $\max_{x} |f^{(n+1)}(x)|$
\item<14-> $1/(n+1)!$
\end{itemize}
\end{block}}
\end{column}
\begin{column}{0.62\hsize}
\uncover<2->{%
\begin{proof}[Beweis]
\begin{enumerate}
\item<3->
$f(x)-p(x)$ hat $n+1$ Nullstellen.
\item<4->
$\exists c$ mit $g(x)=f(x)-p(x)-cl(x)$ hat $n+2$ Nullstellen.
\item<5->
$n+1$ Ableitungen: $g^{(n+1)}(x)$ hat Nullstelle $\xi_x$:
\uncover<6->{%
\begin{align*}
p^{(n+1)}(x)&=0\\
\uncover<7->{l^{(n+1)}(x)}    &\uncover<7->{=(n+1)!}
\\
\uncover<8->{g^{(n+1)}(\xi_x)}&\uncover<8->{=f^{(n+1)}(\xi_x)-c(n+1)!=0}
\\
\uncover<9->{c}&\uncover<9->{=f^{(n+1)}(\xi_x)/(n+1)!}
\end{align*}}
\vspace{-15pt}
\item<10->
$
\displaystyle
f(x)-p(x)=\frac{f^{(n+1)}(\xi_x)}{(n+1)!}l(x)
$
\qedhere
\end{enumerate}
\end{proof}}
\end{column}
\end{columns}
\end{frame}
