%
% gleichungen.tex
%
% (c) 2020 Prof Dr Andreas Müller, Hochschule Rapperswil
%
\begin{frame}
\frametitle{Gleichung für Interpolationspolynom}
\begin{block}{Ansatz}
\vspace{-5pt}
\[
p(x) = a_nx^n + a_{n-1}x^{n-1}+\dots + a_2x^2 + a_1x + a_0\uncover<2->{,\qquad
p(x_i)=f_i}
\]
\end{block}
\vspace{-15pt}
\uncover<3->{
\begin{block}{Lösung mit Gleichungssytem}
\uncover<4->{
\[
\begin{linsys}{5}
{\color<5->{red}a_n}x_1^n &+& {\color<5->{red}a_{n-1}}x_1^{n-1} &+& \dots &+& {\color<5->{red}a_1}x_1 &+& {\color<5->{red}a_0}   &=& f_1 \\
{\color<5->{red}a_n}x_2^n &+& {\color<5->{red}a_{n-1}}x_2^{n-1} &+& \dots &+& {\color<5->{red}a_1}x_2 &+& {\color<5->{red}a_0}   &=& f_2 \\
{\color<5->{red}a_n}x_3^n &+& {\color<5->{red}a_{n-1}}x_3^{n-1} &+& \dots &+& {\color<5->{red}a_1}x_3 &+& {\color<5->{red}a_0}   &=& f_3 \\
\vdots\; & & \vdots\;         & & \ddots& &\vdots\;& &\vdots\;& & \vdots\;\\
{\color<5->{red}a_n}x_n^n &+& {\color<5->{red}a_{n-1}}x_n^{n-1} &+& \dots &+& {\color<5->{red}a_1}x_n &+& {\color<5->{red}a_0}   &=& f_n 
\end{linsys}
\]}
\end{block}}
\uncover<5->{%
\begin{itemize}
\item<6-> zu teuer: Berechnungsaufwand $O(n^3)$ prohitiv
\item<7-> ungelöst: Stabilität der Berechnung von $f(x)$
\end{itemize}
}
\end{frame}
