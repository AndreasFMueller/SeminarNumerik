%
% runge.tex
%
% (c) 2020 Prof Dr Andreas Müller, Hochschule Rapperswil
%
\begin{frame}
\frametitle{Runge-Phänomen}
\begin{block}{Beobachtung}
Äquidistante Stützstellen führen zu starken Oszillationen von $l(x)$
an den Intervallenden
\end{block}
\begin{center}
\begin{tikzpicture}[>=latex]
\begin{scope}
\clip (-6.1,-3) rectangle (6.1,3);
\draw[color=red,line width=1.4pt] (-6.0000,0.0000)
	--(-5.9900,0.4643)
	--(-5.9800,0.9001)
	--(-5.9700,1.3083)
	--(-5.9600,1.6902)
	--(-5.9500,2.0467)
	--(-5.9400,2.3788)
	--(-5.9300,2.6876)
	--(-5.9200,2.9740)
	--(-5.9100,3.2389)
	--(-5.9000,3.4832)
	--(-5.8900,3.7078)
	--(-5.8800,3.9136)
	--(-5.8700,4.1013)
	--(-5.8600,4.2717)
	--(-5.8500,4.4256)
	--(-5.8400,4.5638)
	--(-5.8300,4.6869)
	--(-5.8200,4.7957)
	--(-5.8100,4.8909)
	--(-5.8000,4.9730)
	--(-5.7900,5.0428)
	--(-5.7800,5.1008)
	--(-5.7700,5.1476)
	--(-5.7600,5.1838)
	--(-5.7500,5.2099)
	--(-5.7400,5.2265)
	--(-5.7300,5.2341)
	--(-5.7200,5.2331)
	--(-5.7100,5.2241)
	--(-5.7000,5.2075)
	--(-5.6900,5.1838)
	--(-5.6800,5.1534)
	--(-5.6700,5.1166)
	--(-5.6600,5.0740)
	--(-5.6500,5.0258)
	--(-5.6400,4.9725)
	--(-5.6300,4.9144)
	--(-5.6200,4.8518)
	--(-5.6100,4.7851)
	--(-5.6000,4.7146)
	--(-5.5900,4.6406)
	--(-5.5800,4.5633)
	--(-5.5700,4.4831)
	--(-5.5600,4.4002)
	--(-5.5500,4.3149)
	--(-5.5400,4.2274)
	--(-5.5300,4.1380)
	--(-5.5200,4.0469)
	--(-5.5100,3.9542)
	--(-5.5000,3.8603)
	--(-5.4900,3.7652)
	--(-5.4800,3.6693)
	--(-5.4700,3.5726)
	--(-5.4600,3.4753)
	--(-5.4500,3.3776)
	--(-5.4400,3.2797)
	--(-5.4300,3.1817)
	--(-5.4200,3.0836)
	--(-5.4100,2.9858)
	--(-5.4000,2.8882)
	--(-5.3900,2.7911)
	--(-5.3800,2.6944)
	--(-5.3700,2.5984)
	--(-5.3600,2.5031)
	--(-5.3500,2.4086)
	--(-5.3400,2.3150)
	--(-5.3300,2.2224)
	--(-5.3200,2.1308)
	--(-5.3100,2.0404)
	--(-5.3000,1.9511)
	--(-5.2900,1.8632)
	--(-5.2800,1.7765)
	--(-5.2700,1.6912)
	--(-5.2600,1.6073)
	--(-5.2500,1.5248)
	--(-5.2400,1.4439)
	--(-5.2300,1.3644)
	--(-5.2200,1.2866)
	--(-5.2100,1.2103)
	--(-5.2000,1.1357)
	--(-5.1900,1.0627)
	--(-5.1800,0.9914)
	--(-5.1700,0.9218)
	--(-5.1600,0.8538)
	--(-5.1500,0.7876)
	--(-5.1400,0.7231)
	--(-5.1300,0.6603)
	--(-5.1200,0.5992)
	--(-5.1100,0.5399)
	--(-5.1000,0.4823)
	--(-5.0900,0.4264)
	--(-5.0800,0.3723)
	--(-5.0700,0.3198)
	--(-5.0600,0.2691)
	--(-5.0500,0.2201)
	--(-5.0400,0.1727)
	--(-5.0300,0.1271)
	--(-5.0200,0.0831)
	--(-5.0100,0.0407)
	--(-5.0000,-0.0000)
	--(-4.9900,-0.0391)
	--(-4.9800,-0.0766)
	--(-4.9700,-0.1126)
	--(-4.9600,-0.1470)
	--(-4.9500,-0.1798)
	--(-4.9400,-0.2112)
	--(-4.9300,-0.2411)
	--(-4.9200,-0.2695)
	--(-4.9100,-0.2964)
	--(-4.9000,-0.3220)
	--(-4.8900,-0.3461)
	--(-4.8800,-0.3690)
	--(-4.8700,-0.3904)
	--(-4.8600,-0.4106)
	--(-4.8500,-0.4295)
	--(-4.8400,-0.4471)
	--(-4.8300,-0.4635)
	--(-4.8200,-0.4788)
	--(-4.8100,-0.4928)
	--(-4.8000,-0.5057)
	--(-4.7900,-0.5175)
	--(-4.7800,-0.5283)
	--(-4.7700,-0.5379)
	--(-4.7600,-0.5466)
	--(-4.7500,-0.5542)
	--(-4.7400,-0.5609)
	--(-4.7300,-0.5667)
	--(-4.7200,-0.5715)
	--(-4.7100,-0.5755)
	--(-4.7000,-0.5786)
	--(-4.6900,-0.5809)
	--(-4.6800,-0.5824)
	--(-4.6700,-0.5831)
	--(-4.6600,-0.5831)
	--(-4.6500,-0.5824)
	--(-4.6400,-0.5810)
	--(-4.6300,-0.5789)
	--(-4.6200,-0.5762)
	--(-4.6100,-0.5729)
	--(-4.6000,-0.5690)
	--(-4.5900,-0.5646)
	--(-4.5800,-0.5596)
	--(-4.5700,-0.5541)
	--(-4.5600,-0.5481)
	--(-4.5500,-0.5417)
	--(-4.5400,-0.5348)
	--(-4.5300,-0.5276)
	--(-4.5200,-0.5199)
	--(-4.5100,-0.5119)
	--(-4.5000,-0.5035)
	--(-4.4900,-0.4948)
	--(-4.4800,-0.4858)
	--(-4.4700,-0.4766)
	--(-4.4600,-0.4670)
	--(-4.4500,-0.4572)
	--(-4.4400,-0.4472)
	--(-4.4300,-0.4370)
	--(-4.4200,-0.4266)
	--(-4.4100,-0.4161)
	--(-4.4000,-0.4054)
	--(-4.3900,-0.3945)
	--(-4.3800,-0.3836)
	--(-4.3700,-0.3725)
	--(-4.3600,-0.3614)
	--(-4.3500,-0.3502)
	--(-4.3400,-0.3389)
	--(-4.3300,-0.3276)
	--(-4.3200,-0.3162)
	--(-4.3100,-0.3049)
	--(-4.3000,-0.2935)
	--(-4.2900,-0.2822)
	--(-4.2800,-0.2709)
	--(-4.2700,-0.2596)
	--(-4.2600,-0.2484)
	--(-4.2500,-0.2372)
	--(-4.2400,-0.2261)
	--(-4.2300,-0.2151)
	--(-4.2200,-0.2041)
	--(-4.2100,-0.1933)
	--(-4.2000,-0.1825)
	--(-4.1900,-0.1719)
	--(-4.1800,-0.1614)
	--(-4.1700,-0.1510)
	--(-4.1600,-0.1408)
	--(-4.1500,-0.1307)
	--(-4.1400,-0.1207)
	--(-4.1300,-0.1109)
	--(-4.1200,-0.1013)
	--(-4.1100,-0.0918)
	--(-4.1000,-0.0826)
	--(-4.0900,-0.0734)
	--(-4.0800,-0.0645)
	--(-4.0700,-0.0558)
	--(-4.0600,-0.0472)
	--(-4.0500,-0.0388)
	--(-4.0400,-0.0307)
	--(-4.0300,-0.0227)
	--(-4.0200,-0.0149)
	--(-4.0100,-0.0074)
	--(-4.0000,0.0000)
	--(-3.9900,0.0072)
	--(-3.9800,0.0141)
	--(-3.9700,0.0208)
	--(-3.9600,0.0274)
	--(-3.9500,0.0337)
	--(-3.9400,0.0398)
	--(-3.9300,0.0457)
	--(-3.9200,0.0513)
	--(-3.9100,0.0568)
	--(-3.9000,0.0620)
	--(-3.8900,0.0671)
	--(-3.8800,0.0719)
	--(-3.8700,0.0765)
	--(-3.8600,0.0809)
	--(-3.8500,0.0851)
	--(-3.8400,0.0891)
	--(-3.8300,0.0929)
	--(-3.8200,0.0965)
	--(-3.8100,0.0998)
	--(-3.8000,0.1030)
	--(-3.7900,0.1060)
	--(-3.7800,0.1088)
	--(-3.7700,0.1114)
	--(-3.7600,0.1138)
	--(-3.7500,0.1160)
	--(-3.7400,0.1180)
	--(-3.7300,0.1199)
	--(-3.7200,0.1216)
	--(-3.7100,0.1231)
	--(-3.7000,0.1244)
	--(-3.6900,0.1255)
	--(-3.6800,0.1265)
	--(-3.6700,0.1273)
	--(-3.6600,0.1280)
	--(-3.6500,0.1285)
	--(-3.6400,0.1289)
	--(-3.6300,0.1291)
	--(-3.6200,0.1291)
	--(-3.6100,0.1290)
	--(-3.6000,0.1288)
	--(-3.5900,0.1285)
	--(-3.5800,0.1280)
	--(-3.5700,0.1274)
	--(-3.5600,0.1266)
	--(-3.5500,0.1258)
	--(-3.5400,0.1248)
	--(-3.5300,0.1238)
	--(-3.5200,0.1226)
	--(-3.5100,0.1213)
	--(-3.5000,0.1199)
	--(-3.4900,0.1184)
	--(-3.4800,0.1168)
	--(-3.4700,0.1152)
	--(-3.4600,0.1134)
	--(-3.4500,0.1116)
	--(-3.4400,0.1097)
	--(-3.4300,0.1077)
	--(-3.4200,0.1056)
	--(-3.4100,0.1035)
	--(-3.4000,0.1013)
	--(-3.3900,0.0991)
	--(-3.3800,0.0968)
	--(-3.3700,0.0945)
	--(-3.3600,0.0921)
	--(-3.3500,0.0897)
	--(-3.3400,0.0872)
	--(-3.3300,0.0847)
	--(-3.3200,0.0821)
	--(-3.3100,0.0795)
	--(-3.3000,0.0769)
	--(-3.2900,0.0743)
	--(-3.2800,0.0717)
	--(-3.2700,0.0690)
	--(-3.2600,0.0663)
	--(-3.2500,0.0636)
	--(-3.2400,0.0609)
	--(-3.2300,0.0582)
	--(-3.2200,0.0555)
	--(-3.2100,0.0528)
	--(-3.2000,0.0501)
	--(-3.1900,0.0474)
	--(-3.1800,0.0447)
	--(-3.1700,0.0420)
	--(-3.1600,0.0394)
	--(-3.1500,0.0367)
	--(-3.1400,0.0341)
	--(-3.1300,0.0314)
	--(-3.1200,0.0288)
	--(-3.1100,0.0263)
	--(-3.1000,0.0237)
	--(-3.0900,0.0212)
	--(-3.0800,0.0187)
	--(-3.0700,0.0162)
	--(-3.0600,0.0138)
	--(-3.0500,0.0114)
	--(-3.0400,0.0090)
	--(-3.0300,0.0067)
	--(-3.0200,0.0044)
	--(-3.0100,0.0022)
	--(-3.0000,-0.0000)
	--(-2.9900,-0.0022)
	--(-2.9800,-0.0043)
	--(-2.9700,-0.0063)
	--(-2.9600,-0.0083)
	--(-2.9500,-0.0103)
	--(-2.9400,-0.0122)
	--(-2.9300,-0.0141)
	--(-2.9200,-0.0159)
	--(-2.9100,-0.0177)
	--(-2.9000,-0.0194)
	--(-2.8900,-0.0211)
	--(-2.8800,-0.0227)
	--(-2.8700,-0.0243)
	--(-2.8600,-0.0258)
	--(-2.8500,-0.0272)
	--(-2.8400,-0.0286)
	--(-2.8300,-0.0300)
	--(-2.8200,-0.0312)
	--(-2.8100,-0.0325)
	--(-2.8000,-0.0336)
	--(-2.7900,-0.0348)
	--(-2.7800,-0.0358)
	--(-2.7700,-0.0368)
	--(-2.7600,-0.0378)
	--(-2.7500,-0.0387)
	--(-2.7400,-0.0395)
	--(-2.7300,-0.0403)
	--(-2.7200,-0.0410)
	--(-2.7100,-0.0417)
	--(-2.7000,-0.0423)
	--(-2.6900,-0.0429)
	--(-2.6800,-0.0434)
	--(-2.6700,-0.0439)
	--(-2.6600,-0.0443)
	--(-2.6500,-0.0446)
	--(-2.6400,-0.0449)
	--(-2.6300,-0.0452)
	--(-2.6200,-0.0454)
	--(-2.6100,-0.0455)
	--(-2.6000,-0.0456)
	--(-2.5900,-0.0457)
	--(-2.5800,-0.0457)
	--(-2.5700,-0.0457)
	--(-2.5600,-0.0456)
	--(-2.5500,-0.0454)
	--(-2.5400,-0.0453)
	--(-2.5300,-0.0451)
	--(-2.5200,-0.0448)
	--(-2.5100,-0.0445)
	--(-2.5000,-0.0442)
	--(-2.4900,-0.0438)
	--(-2.4800,-0.0434)
	--(-2.4700,-0.0429)
	--(-2.4600,-0.0424)
	--(-2.4500,-0.0419)
	--(-2.4400,-0.0414)
	--(-2.4300,-0.0408)
	--(-2.4200,-0.0401)
	--(-2.4100,-0.0395)
	--(-2.4000,-0.0388)
	--(-2.3900,-0.0381)
	--(-2.3800,-0.0374)
	--(-2.3700,-0.0366)
	--(-2.3600,-0.0358)
	--(-2.3500,-0.0350)
	--(-2.3400,-0.0342)
	--(-2.3300,-0.0333)
	--(-2.3200,-0.0324)
	--(-2.3100,-0.0315)
	--(-2.3000,-0.0306)
	--(-2.2900,-0.0297)
	--(-2.2800,-0.0287)
	--(-2.2700,-0.0278)
	--(-2.2600,-0.0268)
	--(-2.2500,-0.0258)
	--(-2.2400,-0.0248)
	--(-2.2300,-0.0238)
	--(-2.2200,-0.0228)
	--(-2.2100,-0.0217)
	--(-2.2000,-0.0207)
	--(-2.1900,-0.0197)
	--(-2.1800,-0.0186)
	--(-2.1700,-0.0176)
	--(-2.1600,-0.0165)
	--(-2.1500,-0.0154)
	--(-2.1400,-0.0144)
	--(-2.1300,-0.0133)
	--(-2.1200,-0.0123)
	--(-2.1100,-0.0112)
	--(-2.1000,-0.0102)
	--(-2.0900,-0.0091)
	--(-2.0800,-0.0081)
	--(-2.0700,-0.0070)
	--(-2.0600,-0.0060)
	--(-2.0500,-0.0050)
	--(-2.0400,-0.0040)
	--(-2.0300,-0.0030)
	--(-2.0200,-0.0020)
	--(-2.0100,-0.0010)
	--(-2.0000,0.0000)
	--(-1.9900,0.0010)
	--(-1.9800,0.0019)
	--(-1.9700,0.0028)
	--(-1.9600,0.0038)
	--(-1.9500,0.0047)
	--(-1.9400,0.0056)
	--(-1.9300,0.0064)
	--(-1.9200,0.0073)
	--(-1.9100,0.0081)
	--(-1.9000,0.0089)
	--(-1.8900,0.0098)
	--(-1.8800,0.0105)
	--(-1.8700,0.0113)
	--(-1.8600,0.0120)
	--(-1.8500,0.0128)
	--(-1.8400,0.0135)
	--(-1.8300,0.0141)
	--(-1.8200,0.0148)
	--(-1.8100,0.0154)
	--(-1.8000,0.0161)
	--(-1.7900,0.0166)
	--(-1.7800,0.0172)
	--(-1.7700,0.0178)
	--(-1.7600,0.0183)
	--(-1.7500,0.0188)
	--(-1.7400,0.0193)
	--(-1.7300,0.0197)
	--(-1.7200,0.0201)
	--(-1.7100,0.0205)
	--(-1.7000,0.0209)
	--(-1.6900,0.0213)
	--(-1.6800,0.0216)
	--(-1.6700,0.0219)
	--(-1.6600,0.0222)
	--(-1.6500,0.0224)
	--(-1.6400,0.0227)
	--(-1.6300,0.0229)
	--(-1.6200,0.0231)
	--(-1.6100,0.0232)
	--(-1.6000,0.0233)
	--(-1.5900,0.0235)
	--(-1.5800,0.0235)
	--(-1.5700,0.0236)
	--(-1.5600,0.0236)
	--(-1.5500,0.0237)
	--(-1.5400,0.0236)
	--(-1.5300,0.0236)
	--(-1.5200,0.0236)
	--(-1.5100,0.0235)
	--(-1.5000,0.0234)
	--(-1.4900,0.0233)
	--(-1.4800,0.0231)
	--(-1.4700,0.0230)
	--(-1.4600,0.0228)
	--(-1.4500,0.0226)
	--(-1.4400,0.0223)
	--(-1.4300,0.0221)
	--(-1.4200,0.0218)
	--(-1.4100,0.0216)
	--(-1.4000,0.0213)
	--(-1.3900,0.0209)
	--(-1.3800,0.0206)
	--(-1.3700,0.0202)
	--(-1.3600,0.0199)
	--(-1.3500,0.0195)
	--(-1.3400,0.0191)
	--(-1.3300,0.0187)
	--(-1.3200,0.0182)
	--(-1.3100,0.0178)
	--(-1.3000,0.0173)
	--(-1.2900,0.0169)
	--(-1.2800,0.0164)
	--(-1.2700,0.0159)
	--(-1.2600,0.0154)
	--(-1.2500,0.0149)
	--(-1.2400,0.0143)
	--(-1.2300,0.0138)
	--(-1.2200,0.0132)
	--(-1.2100,0.0127)
	--(-1.2000,0.0121)
	--(-1.1900,0.0115)
	--(-1.1800,0.0110)
	--(-1.1700,0.0104)
	--(-1.1600,0.0098)
	--(-1.1500,0.0092)
	--(-1.1400,0.0086)
	--(-1.1300,0.0080)
	--(-1.1200,0.0074)
	--(-1.1100,0.0068)
	--(-1.1000,0.0061)
	--(-1.0900,0.0055)
	--(-1.0800,0.0049)
	--(-1.0700,0.0043)
	--(-1.0600,0.0037)
	--(-1.0500,0.0031)
	--(-1.0400,0.0024)
	--(-1.0300,0.0018)
	--(-1.0200,0.0012)
	--(-1.0100,0.0006)
	--(-1.0000,-0.0000)
	--(-0.9900,-0.0006)
	--(-0.9800,-0.0012)
	--(-0.9700,-0.0018)
	--(-0.9600,-0.0024)
	--(-0.9500,-0.0030)
	--(-0.9400,-0.0035)
	--(-0.9300,-0.0041)
	--(-0.9200,-0.0047)
	--(-0.9100,-0.0052)
	--(-0.9000,-0.0058)
	--(-0.8900,-0.0063)
	--(-0.8800,-0.0068)
	--(-0.8700,-0.0074)
	--(-0.8600,-0.0079)
	--(-0.8500,-0.0084)
	--(-0.8400,-0.0089)
	--(-0.8300,-0.0093)
	--(-0.8200,-0.0098)
	--(-0.8100,-0.0103)
	--(-0.8000,-0.0107)
	--(-0.7900,-0.0111)
	--(-0.7800,-0.0116)
	--(-0.7700,-0.0120)
	--(-0.7600,-0.0123)
	--(-0.7500,-0.0127)
	--(-0.7400,-0.0131)
	--(-0.7300,-0.0134)
	--(-0.7200,-0.0138)
	--(-0.7100,-0.0141)
	--(-0.7000,-0.0144)
	--(-0.6900,-0.0147)
	--(-0.6800,-0.0150)
	--(-0.6700,-0.0152)
	--(-0.6600,-0.0155)
	--(-0.6500,-0.0157)
	--(-0.6400,-0.0159)
	--(-0.6300,-0.0161)
	--(-0.6200,-0.0163)
	--(-0.6100,-0.0164)
	--(-0.6000,-0.0166)
	--(-0.5900,-0.0167)
	--(-0.5800,-0.0168)
	--(-0.5700,-0.0169)
	--(-0.5600,-0.0170)
	--(-0.5500,-0.0171)
	--(-0.5400,-0.0171)
	--(-0.5300,-0.0172)
	--(-0.5200,-0.0172)
	--(-0.5100,-0.0172)
	--(-0.5000,-0.0171)
	--(-0.4900,-0.0171)
	--(-0.4800,-0.0171)
	--(-0.4700,-0.0170)
	--(-0.4600,-0.0169)
	--(-0.4500,-0.0168)
	--(-0.4400,-0.0167)
	--(-0.4300,-0.0166)
	--(-0.4200,-0.0164)
	--(-0.4100,-0.0163)
	--(-0.4000,-0.0161)
	--(-0.3900,-0.0159)
	--(-0.3800,-0.0157)
	--(-0.3700,-0.0155)
	--(-0.3600,-0.0152)
	--(-0.3500,-0.0150)
	--(-0.3400,-0.0147)
	--(-0.3300,-0.0144)
	--(-0.3200,-0.0142)
	--(-0.3100,-0.0139)
	--(-0.3000,-0.0135)
	--(-0.2900,-0.0132)
	--(-0.2800,-0.0129)
	--(-0.2700,-0.0125)
	--(-0.2600,-0.0122)
	--(-0.2500,-0.0118)
	--(-0.2400,-0.0114)
	--(-0.2300,-0.0110)
	--(-0.2200,-0.0106)
	--(-0.2100,-0.0102)
	--(-0.2000,-0.0098)
	--(-0.1900,-0.0093)
	--(-0.1800,-0.0089)
	--(-0.1700,-0.0084)
	--(-0.1600,-0.0080)
	--(-0.1500,-0.0075)
	--(-0.1400,-0.0070)
	--(-0.1300,-0.0066)
	--(-0.1200,-0.0061)
	--(-0.1100,-0.0056)
	--(-0.1000,-0.0051)
	--(-0.0900,-0.0046)
	--(-0.0800,-0.0041)
	--(-0.0700,-0.0036)
	--(-0.0600,-0.0031)
	--(-0.0500,-0.0026)
	--(-0.0400,-0.0021)
	--(-0.0300,-0.0016)
	--(-0.0200,-0.0010)
	--(-0.0100,-0.0005)
	--(0.0000,0.0000)
	--(0.0100,0.0005)
	--(0.0200,0.0010)
	--(0.0300,0.0016)
	--(0.0400,0.0021)
	--(0.0500,0.0026)
	--(0.0600,0.0031)
	--(0.0700,0.0036)
	--(0.0800,0.0041)
	--(0.0900,0.0046)
	--(0.1000,0.0051)
	--(0.1100,0.0056)
	--(0.1200,0.0061)
	--(0.1300,0.0066)
	--(0.1400,0.0070)
	--(0.1500,0.0075)
	--(0.1600,0.0080)
	--(0.1700,0.0084)
	--(0.1800,0.0089)
	--(0.1900,0.0093)
	--(0.2000,0.0098)
	--(0.2100,0.0102)
	--(0.2200,0.0106)
	--(0.2300,0.0110)
	--(0.2400,0.0114)
	--(0.2500,0.0118)
	--(0.2600,0.0122)
	--(0.2700,0.0125)
	--(0.2800,0.0129)
	--(0.2900,0.0132)
	--(0.3000,0.0135)
	--(0.3100,0.0139)
	--(0.3200,0.0142)
	--(0.3300,0.0144)
	--(0.3400,0.0147)
	--(0.3500,0.0150)
	--(0.3600,0.0152)
	--(0.3700,0.0155)
	--(0.3800,0.0157)
	--(0.3900,0.0159)
	--(0.4000,0.0161)
	--(0.4100,0.0163)
	--(0.4200,0.0164)
	--(0.4300,0.0166)
	--(0.4400,0.0167)
	--(0.4500,0.0168)
	--(0.4600,0.0169)
	--(0.4700,0.0170)
	--(0.4800,0.0171)
	--(0.4900,0.0171)
	--(0.5000,0.0171)
	--(0.5100,0.0172)
	--(0.5200,0.0172)
	--(0.5300,0.0172)
	--(0.5400,0.0171)
	--(0.5500,0.0171)
	--(0.5600,0.0170)
	--(0.5700,0.0169)
	--(0.5800,0.0168)
	--(0.5900,0.0167)
	--(0.6000,0.0166)
	--(0.6100,0.0164)
	--(0.6200,0.0163)
	--(0.6300,0.0161)
	--(0.6400,0.0159)
	--(0.6500,0.0157)
	--(0.6600,0.0155)
	--(0.6700,0.0152)
	--(0.6800,0.0150)
	--(0.6900,0.0147)
	--(0.7000,0.0144)
	--(0.7100,0.0141)
	--(0.7200,0.0138)
	--(0.7300,0.0134)
	--(0.7400,0.0131)
	--(0.7500,0.0127)
	--(0.7600,0.0123)
	--(0.7700,0.0120)
	--(0.7800,0.0116)
	--(0.7900,0.0111)
	--(0.8000,0.0107)
	--(0.8100,0.0103)
	--(0.8200,0.0098)
	--(0.8300,0.0093)
	--(0.8400,0.0089)
	--(0.8500,0.0084)
	--(0.8600,0.0079)
	--(0.8700,0.0074)
	--(0.8800,0.0068)
	--(0.8900,0.0063)
	--(0.9000,0.0058)
	--(0.9100,0.0052)
	--(0.9200,0.0047)
	--(0.9300,0.0041)
	--(0.9400,0.0035)
	--(0.9500,0.0030)
	--(0.9600,0.0024)
	--(0.9700,0.0018)
	--(0.9800,0.0012)
	--(0.9900,0.0006)
	--(1.0000,-0.0000)
	--(1.0100,-0.0006)
	--(1.0200,-0.0012)
	--(1.0300,-0.0018)
	--(1.0400,-0.0024)
	--(1.0500,-0.0031)
	--(1.0600,-0.0037)
	--(1.0700,-0.0043)
	--(1.0800,-0.0049)
	--(1.0900,-0.0055)
	--(1.1000,-0.0061)
	--(1.1100,-0.0068)
	--(1.1200,-0.0074)
	--(1.1300,-0.0080)
	--(1.1400,-0.0086)
	--(1.1500,-0.0092)
	--(1.1600,-0.0098)
	--(1.1700,-0.0104)
	--(1.1800,-0.0110)
	--(1.1900,-0.0115)
	--(1.2000,-0.0121)
	--(1.2100,-0.0127)
	--(1.2200,-0.0132)
	--(1.2300,-0.0138)
	--(1.2400,-0.0143)
	--(1.2500,-0.0149)
	--(1.2600,-0.0154)
	--(1.2700,-0.0159)
	--(1.2800,-0.0164)
	--(1.2900,-0.0169)
	--(1.3000,-0.0173)
	--(1.3100,-0.0178)
	--(1.3200,-0.0182)
	--(1.3300,-0.0187)
	--(1.3400,-0.0191)
	--(1.3500,-0.0195)
	--(1.3600,-0.0199)
	--(1.3700,-0.0202)
	--(1.3800,-0.0206)
	--(1.3900,-0.0209)
	--(1.4000,-0.0213)
	--(1.4100,-0.0216)
	--(1.4200,-0.0218)
	--(1.4300,-0.0221)
	--(1.4400,-0.0223)
	--(1.4500,-0.0226)
	--(1.4600,-0.0228)
	--(1.4700,-0.0230)
	--(1.4800,-0.0231)
	--(1.4900,-0.0233)
	--(1.5000,-0.0234)
	--(1.5100,-0.0235)
	--(1.5200,-0.0236)
	--(1.5300,-0.0236)
	--(1.5400,-0.0236)
	--(1.5500,-0.0237)
	--(1.5600,-0.0236)
	--(1.5700,-0.0236)
	--(1.5800,-0.0235)
	--(1.5900,-0.0235)
	--(1.6000,-0.0233)
	--(1.6100,-0.0232)
	--(1.6200,-0.0231)
	--(1.6300,-0.0229)
	--(1.6400,-0.0227)
	--(1.6500,-0.0224)
	--(1.6600,-0.0222)
	--(1.6700,-0.0219)
	--(1.6800,-0.0216)
	--(1.6900,-0.0213)
	--(1.7000,-0.0209)
	--(1.7100,-0.0205)
	--(1.7200,-0.0201)
	--(1.7300,-0.0197)
	--(1.7400,-0.0193)
	--(1.7500,-0.0188)
	--(1.7600,-0.0183)
	--(1.7700,-0.0178)
	--(1.7800,-0.0172)
	--(1.7900,-0.0166)
	--(1.8000,-0.0161)
	--(1.8100,-0.0154)
	--(1.8200,-0.0148)
	--(1.8300,-0.0141)
	--(1.8400,-0.0135)
	--(1.8500,-0.0128)
	--(1.8600,-0.0120)
	--(1.8700,-0.0113)
	--(1.8800,-0.0105)
	--(1.8900,-0.0098)
	--(1.9000,-0.0089)
	--(1.9100,-0.0081)
	--(1.9200,-0.0073)
	--(1.9300,-0.0064)
	--(1.9400,-0.0056)
	--(1.9500,-0.0047)
	--(1.9600,-0.0038)
	--(1.9700,-0.0028)
	--(1.9800,-0.0019)
	--(1.9900,-0.0010)
	--(2.0000,0.0000)
	--(2.0100,0.0010)
	--(2.0200,0.0020)
	--(2.0300,0.0030)
	--(2.0400,0.0040)
	--(2.0500,0.0050)
	--(2.0600,0.0060)
	--(2.0700,0.0070)
	--(2.0800,0.0081)
	--(2.0900,0.0091)
	--(2.1000,0.0102)
	--(2.1100,0.0112)
	--(2.1200,0.0123)
	--(2.1300,0.0133)
	--(2.1400,0.0144)
	--(2.1500,0.0154)
	--(2.1600,0.0165)
	--(2.1700,0.0176)
	--(2.1800,0.0186)
	--(2.1900,0.0197)
	--(2.2000,0.0207)
	--(2.2100,0.0217)
	--(2.2200,0.0228)
	--(2.2300,0.0238)
	--(2.2400,0.0248)
	--(2.2500,0.0258)
	--(2.2600,0.0268)
	--(2.2700,0.0278)
	--(2.2800,0.0287)
	--(2.2900,0.0297)
	--(2.3000,0.0306)
	--(2.3100,0.0315)
	--(2.3200,0.0324)
	--(2.3300,0.0333)
	--(2.3400,0.0342)
	--(2.3500,0.0350)
	--(2.3600,0.0358)
	--(2.3700,0.0366)
	--(2.3800,0.0374)
	--(2.3900,0.0381)
	--(2.4000,0.0388)
	--(2.4100,0.0395)
	--(2.4200,0.0401)
	--(2.4300,0.0408)
	--(2.4400,0.0414)
	--(2.4500,0.0419)
	--(2.4600,0.0424)
	--(2.4700,0.0429)
	--(2.4800,0.0434)
	--(2.4900,0.0438)
	--(2.5000,0.0442)
	--(2.5100,0.0445)
	--(2.5200,0.0448)
	--(2.5300,0.0451)
	--(2.5400,0.0453)
	--(2.5500,0.0454)
	--(2.5600,0.0456)
	--(2.5700,0.0457)
	--(2.5800,0.0457)
	--(2.5900,0.0457)
	--(2.6000,0.0456)
	--(2.6100,0.0455)
	--(2.6200,0.0454)
	--(2.6300,0.0452)
	--(2.6400,0.0449)
	--(2.6500,0.0446)
	--(2.6600,0.0443)
	--(2.6700,0.0439)
	--(2.6800,0.0434)
	--(2.6900,0.0429)
	--(2.7000,0.0423)
	--(2.7100,0.0417)
	--(2.7200,0.0410)
	--(2.7300,0.0403)
	--(2.7400,0.0395)
	--(2.7500,0.0387)
	--(2.7600,0.0378)
	--(2.7700,0.0368)
	--(2.7800,0.0358)
	--(2.7900,0.0348)
	--(2.8000,0.0336)
	--(2.8100,0.0325)
	--(2.8200,0.0312)
	--(2.8300,0.0300)
	--(2.8400,0.0286)
	--(2.8500,0.0272)
	--(2.8600,0.0258)
	--(2.8700,0.0243)
	--(2.8800,0.0227)
	--(2.8900,0.0211)
	--(2.9000,0.0194)
	--(2.9100,0.0177)
	--(2.9200,0.0159)
	--(2.9300,0.0141)
	--(2.9400,0.0122)
	--(2.9500,0.0103)
	--(2.9600,0.0083)
	--(2.9700,0.0063)
	--(2.9800,0.0043)
	--(2.9900,0.0022)
	--(3.0000,-0.0000)
	--(3.0100,-0.0022)
	--(3.0200,-0.0044)
	--(3.0300,-0.0067)
	--(3.0400,-0.0090)
	--(3.0500,-0.0114)
	--(3.0600,-0.0138)
	--(3.0700,-0.0162)
	--(3.0800,-0.0187)
	--(3.0900,-0.0212)
	--(3.1000,-0.0237)
	--(3.1100,-0.0263)
	--(3.1200,-0.0288)
	--(3.1300,-0.0314)
	--(3.1400,-0.0341)
	--(3.1500,-0.0367)
	--(3.1600,-0.0394)
	--(3.1700,-0.0420)
	--(3.1800,-0.0447)
	--(3.1900,-0.0474)
	--(3.2000,-0.0501)
	--(3.2100,-0.0528)
	--(3.2200,-0.0555)
	--(3.2300,-0.0582)
	--(3.2400,-0.0609)
	--(3.2500,-0.0636)
	--(3.2600,-0.0663)
	--(3.2700,-0.0690)
	--(3.2800,-0.0717)
	--(3.2900,-0.0743)
	--(3.3000,-0.0769)
	--(3.3100,-0.0795)
	--(3.3200,-0.0821)
	--(3.3300,-0.0847)
	--(3.3400,-0.0872)
	--(3.3500,-0.0897)
	--(3.3600,-0.0921)
	--(3.3700,-0.0945)
	--(3.3800,-0.0968)
	--(3.3900,-0.0991)
	--(3.4000,-0.1013)
	--(3.4100,-0.1035)
	--(3.4200,-0.1056)
	--(3.4300,-0.1077)
	--(3.4400,-0.1097)
	--(3.4500,-0.1116)
	--(3.4600,-0.1134)
	--(3.4700,-0.1152)
	--(3.4800,-0.1168)
	--(3.4900,-0.1184)
	--(3.5000,-0.1199)
	--(3.5100,-0.1213)
	--(3.5200,-0.1226)
	--(3.5300,-0.1238)
	--(3.5400,-0.1248)
	--(3.5500,-0.1258)
	--(3.5600,-0.1266)
	--(3.5700,-0.1274)
	--(3.5800,-0.1280)
	--(3.5900,-0.1285)
	--(3.6000,-0.1288)
	--(3.6100,-0.1290)
	--(3.6200,-0.1291)
	--(3.6300,-0.1291)
	--(3.6400,-0.1289)
	--(3.6500,-0.1285)
	--(3.6600,-0.1280)
	--(3.6700,-0.1273)
	--(3.6800,-0.1265)
	--(3.6900,-0.1255)
	--(3.7000,-0.1244)
	--(3.7100,-0.1231)
	--(3.7200,-0.1216)
	--(3.7300,-0.1199)
	--(3.7400,-0.1180)
	--(3.7500,-0.1160)
	--(3.7600,-0.1138)
	--(3.7700,-0.1114)
	--(3.7800,-0.1088)
	--(3.7900,-0.1060)
	--(3.8000,-0.1030)
	--(3.8100,-0.0998)
	--(3.8200,-0.0965)
	--(3.8300,-0.0929)
	--(3.8400,-0.0891)
	--(3.8500,-0.0851)
	--(3.8600,-0.0809)
	--(3.8700,-0.0765)
	--(3.8800,-0.0719)
	--(3.8900,-0.0671)
	--(3.9000,-0.0620)
	--(3.9100,-0.0568)
	--(3.9200,-0.0513)
	--(3.9300,-0.0457)
	--(3.9400,-0.0398)
	--(3.9500,-0.0337)
	--(3.9600,-0.0274)
	--(3.9700,-0.0208)
	--(3.9800,-0.0141)
	--(3.9900,-0.0072)
	--(4.0000,0.0000)
	--(4.0100,0.0074)
	--(4.0200,0.0149)
	--(4.0300,0.0227)
	--(4.0400,0.0307)
	--(4.0500,0.0388)
	--(4.0600,0.0472)
	--(4.0700,0.0558)
	--(4.0800,0.0645)
	--(4.0900,0.0734)
	--(4.1000,0.0826)
	--(4.1100,0.0918)
	--(4.1200,0.1013)
	--(4.1300,0.1109)
	--(4.1400,0.1207)
	--(4.1500,0.1307)
	--(4.1600,0.1408)
	--(4.1700,0.1510)
	--(4.1800,0.1614)
	--(4.1900,0.1719)
	--(4.2000,0.1825)
	--(4.2100,0.1933)
	--(4.2200,0.2041)
	--(4.2300,0.2151)
	--(4.2400,0.2261)
	--(4.2500,0.2372)
	--(4.2600,0.2484)
	--(4.2700,0.2596)
	--(4.2800,0.2709)
	--(4.2900,0.2822)
	--(4.3000,0.2935)
	--(4.3100,0.3049)
	--(4.3200,0.3162)
	--(4.3300,0.3276)
	--(4.3400,0.3389)
	--(4.3500,0.3502)
	--(4.3600,0.3614)
	--(4.3700,0.3725)
	--(4.3800,0.3836)
	--(4.3900,0.3945)
	--(4.4000,0.4054)
	--(4.4100,0.4161)
	--(4.4200,0.4266)
	--(4.4300,0.4370)
	--(4.4400,0.4472)
	--(4.4500,0.4572)
	--(4.4600,0.4670)
	--(4.4700,0.4766)
	--(4.4800,0.4858)
	--(4.4900,0.4948)
	--(4.5000,0.5035)
	--(4.5100,0.5119)
	--(4.5200,0.5199)
	--(4.5300,0.5276)
	--(4.5400,0.5348)
	--(4.5500,0.5417)
	--(4.5600,0.5481)
	--(4.5700,0.5541)
	--(4.5800,0.5596)
	--(4.5900,0.5646)
	--(4.6000,0.5690)
	--(4.6100,0.5729)
	--(4.6200,0.5762)
	--(4.6300,0.5789)
	--(4.6400,0.5810)
	--(4.6500,0.5824)
	--(4.6600,0.5831)
	--(4.6700,0.5831)
	--(4.6800,0.5824)
	--(4.6900,0.5809)
	--(4.7000,0.5786)
	--(4.7100,0.5755)
	--(4.7200,0.5715)
	--(4.7300,0.5667)
	--(4.7400,0.5609)
	--(4.7500,0.5542)
	--(4.7600,0.5466)
	--(4.7700,0.5379)
	--(4.7800,0.5283)
	--(4.7900,0.5175)
	--(4.8000,0.5057)
	--(4.8100,0.4928)
	--(4.8200,0.4788)
	--(4.8300,0.4635)
	--(4.8400,0.4471)
	--(4.8500,0.4295)
	--(4.8600,0.4106)
	--(4.8700,0.3904)
	--(4.8800,0.3690)
	--(4.8900,0.3461)
	--(4.9000,0.3220)
	--(4.9100,0.2964)
	--(4.9200,0.2695)
	--(4.9300,0.2411)
	--(4.9400,0.2112)
	--(4.9500,0.1798)
	--(4.9600,0.1470)
	--(4.9700,0.1126)
	--(4.9800,0.0766)
	--(4.9900,0.0391)
	--(5.0000,-0.0000)
	--(5.0100,-0.0407)
	--(5.0200,-0.0831)
	--(5.0300,-0.1271)
	--(5.0400,-0.1727)
	--(5.0500,-0.2201)
	--(5.0600,-0.2691)
	--(5.0700,-0.3198)
	--(5.0800,-0.3723)
	--(5.0900,-0.4264)
	--(5.1000,-0.4823)
	--(5.1100,-0.5399)
	--(5.1200,-0.5992)
	--(5.1300,-0.6603)
	--(5.1400,-0.7231)
	--(5.1500,-0.7876)
	--(5.1600,-0.8538)
	--(5.1700,-0.9218)
	--(5.1800,-0.9914)
	--(5.1900,-1.0627)
	--(5.2000,-1.1357)
	--(5.2100,-1.2103)
	--(5.2200,-1.2866)
	--(5.2300,-1.3644)
	--(5.2400,-1.4439)
	--(5.2500,-1.5248)
	--(5.2600,-1.6073)
	--(5.2700,-1.6912)
	--(5.2800,-1.7765)
	--(5.2900,-1.8632)
	--(5.3000,-1.9511)
	--(5.3100,-2.0404)
	--(5.3200,-2.1308)
	--(5.3300,-2.2224)
	--(5.3400,-2.3150)
	--(5.3500,-2.4086)
	--(5.3600,-2.5031)
	--(5.3700,-2.5984)
	--(5.3800,-2.6944)
	--(5.3900,-2.7911)
	--(5.4000,-2.8882)
	--(5.4100,-2.9858)
	--(5.4200,-3.0836)
	--(5.4300,-3.1817)
	--(5.4400,-3.2797)
	--(5.4500,-3.3776)
	--(5.4600,-3.4753)
	--(5.4700,-3.5726)
	--(5.4800,-3.6693)
	--(5.4900,-3.7652)
	--(5.5000,-3.8603)
	--(5.5100,-3.9542)
	--(5.5200,-4.0469)
	--(5.5300,-4.1380)
	--(5.5400,-4.2274)
	--(5.5500,-4.3149)
	--(5.5600,-4.4002)
	--(5.5700,-4.4831)
	--(5.5800,-4.5633)
	--(5.5900,-4.6406)
	--(5.6000,-4.7146)
	--(5.6100,-4.7851)
	--(5.6200,-4.8518)
	--(5.6300,-4.9144)
	--(5.6400,-4.9725)
	--(5.6500,-5.0258)
	--(5.6600,-5.0740)
	--(5.6700,-5.1166)
	--(5.6800,-5.1534)
	--(5.6900,-5.1838)
	--(5.7000,-5.2075)
	--(5.7100,-5.2241)
	--(5.7200,-5.2331)
	--(5.7300,-5.2341)
	--(5.7400,-5.2265)
	--(5.7500,-5.2099)
	--(5.7600,-5.1838)
	--(5.7700,-5.1476)
	--(5.7800,-5.1008)
	--(5.7900,-5.0428)
	--(5.8000,-4.9730)
	--(5.8100,-4.8909)
	--(5.8200,-4.7957)
	--(5.8300,-4.6869)
	--(5.8400,-4.5638)
	--(5.8500,-4.4256)
	--(5.8600,-4.2717)
	--(5.8700,-4.1013)
	--(5.8800,-3.9136)
	--(5.8900,-3.7078)
	--(5.9000,-3.4832)
	--(5.9100,-3.2389)
	--(5.9200,-2.9740)
	--(5.9300,-2.6876)
	--(5.9400,-2.3788)
	--(5.9500,-2.0467)
	--(5.9600,-1.6902)
	--(5.9700,-1.3083)
	--(5.9800,-0.9001)
	--(5.9900,-0.4643)
	--(6.0000,0.0000);

\end{scope}
\draw[->] (-6.1,0)--(6.3,0) coordinate[label={$x$}];
\draw[->] (0,-3.1)--(0,3.2) coordinate[label={right:$y$}];
\foreach \x in {-6,...,6}{
	\fill[color=red] (\x,0) circle[radius=0.08];
}
\end{tikzpicture}
\end{center}
\end{frame}
