%
% baryzentrisch.tex
%
% (c) 2020 Prof Dr Andreas Müller, Hochschule Rapperswil
%
\begin{frame}
\frametitle{Baryzentrische Formeln}
\vspace{-10pt}
\begin{columns}[t]
\begin{column}{0.48\hsize}
\begin{block}{$l_i(x)$ instabil}
\begin{enumerate}
\item
$x$ nahe bei $x_i$ $\Rightarrow$ Auslöschung in $x-x_i$
\item
anderen Faktoren $x-x_j$ ``gross''
\end{enumerate}
\end{block}
\vspace{-10pt}
\begin{block}{Gewichte $w_i$}
\vspace{-20pt}
\begin{align*}
w_i
&=
\frac{1}{\prod_{k=0,k\ne i}^n (x_i-x_k)}
\\
l_i(x)
&=
\frac{l(x)}{x-x_i}\cdot w_i
\end{align*}
\end{block}
\vspace{-10pt}
\begin{block}{Interpolationspolynom}
\vspace{-10pt}
\[
p(x)
=
l(x) \sum_{i=0}^n \frac{w_if_i}{x-x_i}
\]
\end{block}
\end{column}
\begin{column}{0.48\hsize}
\begin{block}{Gewichtetes Mittel}
$f_i=1\;\forall i\quad\Rightarrow \quad p(x) = 1$
\vspace{-5pt}
\begin{align*}
l(x)
&=
\biggl(
\sum_{i=0}^n \frac{w_i}{x-x_i}
\biggr)^{-1}\\[-10pt]
\intertext{\vspace{-5pt}%
Interpolationspolynom
\vspace{-5pt}}
p(x)
&=
\frac{\displaystyle\sum_{i=0}^n \frac{w_if_i}{x-x_i}}{\displaystyle\sum_{i=0}^n \frac{w_i}{x-x_i}}
\end{align*}
$p(x)$ ist gewichtetes Mittel der $f_i$ mit Gewichten
\[
\frac{w_i}{x-x_i}
\]
\end{block}
\end{column}
\end{columns}
\end{frame}
