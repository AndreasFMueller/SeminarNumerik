%
% minimal.tex
%
% (c) 2020 Prof Dr Andreas Müller, Hochschule Rapperswil
%
\begin{frame}
\frametitle{Minimalproblem für Splines}
\begin{block}{Variation}
Bilde Variation von $\int_a^b g''(x)^2\,dx$ mit
$g(x)\to g(x)+\varepsilon h(x)$, $h(x_i)=0$
\end{block}
\uncover<2->{%
\begin{block}{In Teilintervallen $[x_i,x_{i+1}]$}
\vspace{-12pt}
\begin{align*}
0
&=
\left.
\frac{d}{d\varepsilon}
\int_{x_i}^{x_{i+1}} (g''(x) + \varepsilon h''(x))^2\,dx
\right|_{\varepsilon=0}
\uncover<3->{=
2\int_{x_i}^{x_{i+1}} g''(x)h''(x) \,dx}
\\
\uncover<4->{0}
&\uncover<4->{=
\left[ g''(x) h'(x) \right]_{x_i}^{x_{i+1}}
-
\int_{x_i}^{x_{i+1}} g'''(x) h'(x)\,dx}
\\
&\uncover<5->{=
\left[ g''(x) h'(x) \right]_{x_i}^{x_{i+1}}
-
\left[ g'''(x) h(x) \right]_{x_i}^{x_{i+1}}
+
\int_{x_i}^{x_{i+1}} g''''(x) h(x)\,dx}
\end{align*}
\end{block}}
\vspace{-14pt}
\uncover<6->{%
\begin{block}{Wähle $h$ geeignet:}
\vspace{-10pt}
\[
h(x_i)=h(x_{i+1})=0,\;
h'(x_i)=h'(x_{i+1})=0
\uncover<7->{%
\qquad\Rightarrow\qquad
g''''(x) = 0}
\]
\uncover<8->{%
$g$ ist ein kubisches Polynom in jedem Teilinterval $[x_i,x_{i+1}]$}
\end{block}}

\end{frame}
