%
% eulerlagrange.tex
%
% (c) 2020 Prof Dr Andreas Müller, Hochschule Rapperswi
%
\begin{frame}
\frametitle{Euler-Lagrange-Gleichung}
\vspace{-15pt}
\begin{columns}[t]
\begin{column}{0.4\hsize}
\begin{block}{``Nachbarfunktion''}
\vspace{-15pt}
\begin{align*}
x(t)
&\to
x(t) + \varepsilon h(t)
\\
0&=h(a)=h(b)
\end{align*}
\end{block}
\end{column}
\begin{column}{0.56\hsize}
\uncover<2->{%
\begin{block}{Wahlmöglichkeit für $h(t)$}
\vspace{-15pt}
\[
\int_a^b f(t)h(t)\,dt = 0\;\forall h \quad\Rightarrow\quad f(t)=0
\]
\end{block}}
\end{column}
\end{columns}
\vspace{-10pt}
\uncover<3->{%
\begin{block}{Stationär}
\vspace{-15pt}
\begin{align*}
0
&=
\left.
\frac{d}{d\varepsilon} W(x+\varepsilon h)
\right|_{\varepsilon=0}
\uncover<4->{=
\left.
\frac{d}{d\varepsilon}
\int_a^b L(t,x(t)+\varepsilon h(t), \dot{x}(t) + \varepsilon \dot{h}(t))\,dt
\right|_{\varepsilon=0}}
\\
&\uncover<5->{=
\int_a^b \frac{\partial L}{\partial x}(t,x(t), \dot{x}(t))\, h(t)}
\only<5>{+
\frac{\partial L}{\partial v}(t,x(t),\dot{x}(t))\, \dot{h}(t)\,dt}
\only<6->{
+
\biggl[ \frac{\partial L}{\partial v}(t,x(t),\dot{x}(t)) h(t)\biggr]_a^b}
\\
&\qquad\qquad
\uncover<6->{-
\int_a^b
\frac{d}{dt}
\frac{\partial L}{\partial v}(t,x(t),\dot{x}(t))\, h(t)\,dt}
\\
&\uncover<7->{=
\int_a^b
\biggl(\frac{\partial L}{\partial x}(t,x(t),\dot{x}(t))
-\frac{d}{dt}\frac{\partial L}{\partial v}(t,x(t),\dot{x}(t))\biggr)h(t)\,dt}
\uncover<8->{
\quad\Rightarrow\quad
\boxed{
\frac{d}{dt}
\frac{\partial L}{\partial v}
-\frac{\partial L}{\partial x}
=
0}}
\end{align*}
\end{block}}
\end{frame}
