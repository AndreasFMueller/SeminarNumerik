%
% gleichungssystem.tex
%
% (c) 2020 Prof Dr Andreas Müller, Hochschule Rapperswil
%
\begin{frame}
\frametitle{Zweite Ableitungen}
Bedingungen für zweite Ableitung
\[
g_i''(x)
=
{\color{red}f_i}h_{0,i}''(x) + {\color{red}f_{i+1}}h_{1,i}''(x)
+
{\color{red}s_i}h^{1\prime\prime}_{0,i}(x) + {\color{red}s_{i+1}}h_{1,i}^{1\prime\prime}(x)
\]
\uncover<2->{
Spezialfall: Stützstellen ganzzahlig
\begin{align*}
h_0''(0)               &=          - 6 &
h_1''(0)               &= \phantom{-}6 &
h_0^{1\prime\prime}(0) &=          - 4 &
h_1^{1\prime\prime}(0) &= \phantom{-}2
\\
h_0''(1)               &= \phantom{-}6 &
h_1''(1)               &=          - 6 &
h_1^{1\prime\prime}(1) &=          - 2 &
h_1^{1\prime\prime}(1) &= \phantom{-}4
\end{align*}
}
\end{frame}

\begin{frame}
\frametitle{Gleichungsssytem}
Für ganzzahlige Stützstellen:
\[
\setcounter{MaxMatrixCols}{20}
\begin{matrix}
4s_0 &+& 2s_1 & &      & &      & &        & &        & &      &=& 6f_1-6f_0 \\
2s_0 &+& 4s_1 &+& 2s_2 & &      & &        & &        & &      &=& 6f_2-6f_1 \\
     & & 2s_1 &+& 4s_2 &+& 2s_3 & &        & &        & &      &=& 6f_3-6f_2 \\
     & &      & & 2s_2 &\ddots& &\ddots &  & &        & &      &=& 6f_4-6f_3 \\
     & &      & &      &\ddots& & &        &\ddots&   & &      & & \vdots\hspace*{15pt}    \\
     & &      & &      & &      & &        & &        & &      & & \\
     & &      & &      & &      &\ddots&   &\ddots&2s_{n-1}& &      & & \vdots\hspace*{15pt}          \\
     & &      & &      & &      & &2s_{n-2}&+&4s_{n-1}&+& 2s_n &=& 6f_{n-1}-6f_{n-2} \\
     & &      & &      & &      & &        & &2s_{n-1}&+& 4s_n &=& 6f_n-6f_{n-1} 
\end{matrix}
\]
\end{frame}
