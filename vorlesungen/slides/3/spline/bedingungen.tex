%
% bedingungen.tex
%
% (c) 2020 Prof Dr Andreas Müller, Hochschule Rapperswil
%
\begin{frame}
\frametitle{Bedingungen}
\vspace{-15pt}
\begin{columns}[t]
\begin{column}{0.34\hsize}
\begin{block}{Aufgabe}
Für jedes Interval $[x_i,x_{i+1}]$, $i=1,\dots,n-1$, findet ein kubisches
Polynome $g_i(x)$
\end{block}
\end{column}
\begin{column}{0.62\hsize}
\begin{block}{Ansatz}
Mit Hermite-Polynomen $h_{ki}(x)$ und $h_{ki}^1(x)$ für das Intervall
$[x_i,x_{i+1}]$:
\[
g_i(x)
=
f_ih_{0,i}(x)
+
f_{i+1}h_{1,i}(x)
+
{\color{red}p_i}h_{0,i}^1(x)
+
{\color{red}q_i}h_{1,i}^1(x)
\]
\end{block}
\end{column}
\end{columns}
\vspace{-10pt}
\begin{columns}[t]
\begin{column}{0.44\hsize}
\begin{block}{Steigungen}
Für $i=1,\dots,n-1$:
\begin{align*}
g_{i-1}'(x_i)&=g_{i}'(x_i)
\\
{\color{red}q_{i-1}}&={\color{red}p_i}=s_i
\end{align*}
$n+1$ Steigungen $s_i$, $i=0,\dots,n$,
$g'(x_i)=s_i$.
\end{block}
\end{column}
\begin{column}{0.52\hsize}
\begin{block}{2. Ableitungen}
Stetigkeit der zweiten Ableitung
\begin{align*}
g_{i-1}''(x_i)&=g_{i}''(x_i) &i&=1,\dots,n-1\\
g_{0}''(x_0)&=g_{n}''(x_n)=0
\end{align*}
{\color{blue}$n+1$ Gleichungen} für $n+1$ Steigungen $s_i$
\end{block}
\end{column}
\end{columns}

\end{frame}
