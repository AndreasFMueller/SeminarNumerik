%
% linalg.tex
%
% (c) 2020 Prof Dr Andreas Müller, Hochschule Rapperswil
%
\begin{frame}
\frametitle{Warum reicht ``Gauss'' nicht?}
\begin{columns}[t]
\begin{column}{0.48\hsize}
\begin{block}{Gleichungen Lösen}
\uncover<3->{%
Aufwand für ``Gauss'': $O(n^3)$}
\begin{itemize}
\item<4->
Strömungssimulation: $n=N^d$, für $N=10^k$,
Laufzeit $O(10^{3dk})$, 214000 Jahre auf {\em Summit}
\item<5->
Viele Operationen, in denen Differenzen gebildet werden: Potential für Auslöschung
\item<6->
Division durch kleine Pivot-Elemente $\Rightarrow$ 
\end{itemize}
\end{block}
\end{column}
\begin{column}{0.48\hsize}
\uncover<2->{%
\begin{block}{Eigenwerte und Eigenvektoren}
\uncover<7->{%
``Standardmethode'': charakteristisches Polynom $\det(A-\lambda E)=\chi_A(\lambda)=0$
}
\begin{itemize}
\item<8->
Determinante viel zu aufwändig: $n > 5$ unpraktikabel
\item<9->
Für jeden Eigenwert ist ein Durchführung des Gauss-Algorithmus nötig mit
singulärer Matrix \uncover<10->{$\rightarrow$ numerisch ungünstig}
\end{itemize}
\end{block}}
\end{column}
\end{columns}
\end{frame}
