%
% konditionbeispiel.tex
%
% (c) 2020 Prof Dr Andreas Müller, Hochschule Rapperswil
%
\begin{frame}
\frametitle{Beispiel: Kahan-Matrix}
\vspace{-15pt}
\begin{columns}[t]
\begin{column}{0.32\hsize}
\begin{block}{Kahan-Matrix}
\vspace{-10pt}
\[
A=\begin{pmatrix}
1000000&999999\\
\phantom{0}999999&999998
\end{pmatrix}
\]
\end{block}
\end{column}
\begin{column}{0.64\hsize}
\uncover<3->{%
\begin{block}{Inverse der Kahan-Matrix}
\vspace{-10pt}
\[
A^{-1}=
\begin{pmatrix}
          -999990.385579839& \phantom{-}999991.385572224\\
\phantom{-}999991.385572224&           -999992.385564610
\end{pmatrix}
\]
\end{block}}
\end{column}
\end{columns}
\uncover<2->{%
Zeilen ``fast'' gleich, $A$ ``fast singulär''}
\begin{columns}[t]
\begin{column}{0.28\hsize}
\uncover<4->{%
\begin{block}{Singulärwerte}
\vspace{-20pt}
\begin{align*}
s_1&=\phantom{-}2.0000\cdot 10^6 \\
s_2&=  -5.0000\cdot 10^{-7}
\end{align*}
\end{block}}
\end{column}
\begin{column}{0.26\hsize}
\uncover<5->{%
\begin{block}{Konditionszahl}
Sehr schlecht:
\[
\kappa = 4.0000 \cdot 10^{12}
\]
\end{block}}
\end{column}
\begin{column}{0.4\hsize}
\uncover<6->{%
\begin{block}{$AA^{-1}\stackrel{?}{=}E$}
Trotz \texttt{double}-Rechnung
\[
\begin{pmatrix}
   1.00012207& -0.00012207 \\
   0\phantom{.00000000}& \phantom{-}1\phantom{.00000000}
\end{pmatrix}
\]
\end{block}}
\end{column}
\end{columns}
\end{frame}
