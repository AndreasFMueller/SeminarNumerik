%
% splitting.tex
%
% (c) 2020 Prof Dr Andreas Müller, Hochschule Rapperswil
%
\begin{frame}
\frametitle{Matrix-Splitting}
\vspace{-10pt}
\begin{columns}[t]
\begin{column}{0.3\hsize}
\begin{block}{Splitting}
$Ax=b$ mit $A=B+C$
\end{block}
\end{column}
\begin{column}{0.24\hsize}
\uncover<3->{%
\begin{block}{Iteration}
\vspace{-22pt}
\begin{align*}
\color{red}x&=B^{-1}b - B^{-1}Cx
\\
\uncover<4->{{\color{red}\tilde{x}}}&\uncover<4->{=B^{-1}C\tilde{x}}
\end{align*}
\end{block}}
\end{column}
\begin{column}{0.36\hsize}
\uncover<5->{%
\begin{block}{Konvergenz}
\vspace{-22pt}
\[
\varrho(B^{-1}C)
\begin{cases}
< 1&\text{konvergent}\\
>1&\text{instabil}
\end{cases}
\]
\end{block}}
\end{column}
\end{columns}
\hrule

\begin{columns}[t]
\begin{column}{0.36\hsize}
\uncover<6->{%
\begin{block}{Jacobi}
$B=D$ und $C=L+R$
\end{block}}
\uncover<2->{%
\begin{block}{Gauss-Seidel}
$B = D+L$ und $C=R$
\end{block}}
\uncover<7->{%
\begin{block}{Richardson}
$B=\tau E$ und $C=A-\tau E$
\uncover<8->{$\Rightarrow$ $B^{-1}C=\tau^{-1}A-E$}
\end{block}}
\end{column}
\begin{column}{0.6\hsize}
\uncover<9->{%
\begin{block}{Successive-Over-Relaxation (SOR)}
$B = (1/\omega)D + L$ und $C=R + (1-1/\omega)D$

\uncover<10->{Konvergiert für $A=A^t$ immer!}
\end{block}}
\uncover<11->{%
\begin{block}{Gauss-Nachiteration}
\begin{enumerate}
\item<11-> Gauss mit \texttt{float}, $x_{\texttt{float}}$, $B=L_{\texttt{float}}R_{\texttt{float}}$
\item<12-> $C = A-B$ sehr klein
\item<13-> $x_{\texttt{double}} = b_{0,\texttt{float}} - B^{-1}Cx_{\texttt{float}}$
\end{enumerate}
\end{block}}
\end{column}
\end{columns}



\end{frame}
