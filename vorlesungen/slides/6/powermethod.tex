%
% powermethod.tex
%
% (c) 2020 Prof Dr Andreas Müller, Hochschule Rapperswil
%
\begin{frame}
\frametitle{Potenzmethode}
\vspace{-15pt}
\begin{columns}[t]
\begin{column}{0.44\hsize}
\begin{block}{Aufgabe}
$n\times n$-Matrix $A$, finde grössten Eigenwert $\lambda\in\mathbb R$ und Eigenvektor $v$
\end{block}
\uncover<2->{%
\begin{block}{Methode}
\begin{enumerate}
\item<3-> $v_0\in \mathbb R^n$ zufällig
\item<4-> $v_{n+1} = Av_{n} / |Av_{n}|$
\item<5-> Wiederhole 2 bis Konvergenz
\item<7-> $v=v_{\infty}$
\item<8-> $\lambda = |Av_\infty|$
\end{enumerate}
\end{block}}
\uncover<8->{%
\begin{block}{Anwendung}
Google-Pagerank ist EV der Google-Matrix zum
(grössten) EW 1
\end{block}}
\end{column}
\begin{column}{0.54\hsize}
\uncover<9->{%
\begin{proof}[Beweis]
\begin{itemize}
\item<10->
$v_1,\dots,v_n$ Eigenbasis mit EW $\lambda_1>\dots \ge\lambda_n$
\item<11->
$v_0 = a_1v_1+\dots+a_nv_n$ mit $a_1\ne 0$
\item<12->
$A^kv_0 = \lambda_1^k a_1 v_1 + \dots + \lambda_n^k a_n v_n$
\item<13->
Division durch $\lambda_1^k$:
\[
\frac{1}{\lambda_1^k}A^kv_0
=
a_1v_1 \underbrace{+ \dots + \biggl(\underbrace{\frac{\lambda_i}{\lambda_n}}_{\displaystyle < 1}\biggr)^k a_kv_k+\dots}_{\displaystyle\to 0}
\]
\vspace{-10pt}
\item<14-> Konvergenz: $\displaystyle
\frac{A^kv_0}{|A^kv_0|} \to v_1
$
\end{itemize}
\end{proof}}
\end{column}
\end{columns}
\end{frame}
