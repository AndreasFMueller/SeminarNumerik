%
% drehung.tex
%
% (c) 2020 Prof Dr Andreas Müller, Hochschule Rapperswil
%
\begin{frame}
\frametitle{Drehung (Fall $n=2$)}
\begin{block}{Diagonalisierung mit {\em einer} Drehung}
\vspace{-5pt}
\[
\begin{pmatrix}
\cos\alpha& \sin\alpha\\
-\sin\alpha&\cos\alpha
\end{pmatrix}
\begin{pmatrix}
a_{11}&a_{12}\\
a_{21}&a_{22}
\end{pmatrix}
\begin{pmatrix}
\cos\alpha&-\sin\alpha\\
\sin\alpha&\cos\alpha
\end{pmatrix}
=
\begin{pmatrix}
\lambda_1&\color{red}0\\
\color{red}0&\lambda_2
\end{pmatrix}
\]
\end{block}
\vspace{-10pt}
\uncover<2->{%
\begin{block}{Bestimmung von $\alpha$}
\vspace{-20pt}
\begin{align*}
\Rightarrow
a_{11}\sin\alpha\cos\alpha + a_{12}(\cos^2\alpha-\sin^2\alpha) -a_{22}\sin\alpha\cos\alpha&={\color{red}0}
\\
\uncover<3->{
(a_{11}-a_{22}) \frac12 \sin2\alpha + a_{12} \cos2\alpha}&\uncover<3->{={\color{red}0}}
\\
\uncover<4->{
\Rightarrow\qquad
\cot 2\alpha}&\uncover<4->{= \frac{a_{22}-a_{11}}{2a_{12}} = \vartheta}
\end{align*}
\end{block}}
\vspace{-15pt}
\uncover<5->{%
\begin{block}{Bestimmung von $\cos\alpha$ und $\sin\alpha$}
\vspace{-20pt}
\[
\tan\alpha = \vartheta \pm\sqrt{\vartheta^2+1}
\uncover<6->{
\qquad\Rightarrow\qquad
\cos\alpha = \frac{1}{\sqrt{1+\tan^2\alpha}},
\quad
\sin\alpha = \frac{\tan\alpha}{\sqrt{1+\tan^2\alpha}}}
\]
\end{block}}
\end{frame}
