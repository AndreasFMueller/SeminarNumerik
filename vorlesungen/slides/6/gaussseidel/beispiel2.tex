%
% beispiel2.tex
%
% (c) 2020 Prof Dr Andreas Müller, Hochschule Rapperswil
%
\begin{frame}
\frametitle{Beispiel}
% A =
% 
%    1   1   0   1
%    1   2   1   0
%    0   1   2   1
%    1   0   1   1
% 
% spectralradius =  2.3315
% b =
% 
%    4
%    2
%    1
%    3

\def\s{\hspace*{-3mm}}
\[
\begin{linsys}{5}
\s  x_1&+&\s  x_2&+&\s     &+&\s  x_4&=&\s  4\\
\s  x_1&+&\s 2x_2&+&\s  x_3&+&\s     &=&\s  2\\
\s     & &\s  x_2&+&\s 2x_3&+&\s  x_4&=&\s  1\\
\s  x_1&+&\s     &+&\s  x_3&+&\s  x_4&=&\s  3\\
\end{linsys}
\]


\renewcommand{\arraystretch}{2.3}

\only<2>{
Iteration $n=1$:
\[
\begin{linsys}{6}
{\color{red}\phantom{00}\hbox{$-1.00$}}&=&\s\displaystyle\frac{1}{1}\cdot\bigl(4& &             &-&\s 1\cdot(\phantom{+00}2.00)&-&\s 0\cdot(\phantom{+00}1.00)&-&\s 1\cdot(\phantom{+00}3.00)\bigr)\\
{\color{red}\phantom{+00}1.00}&=&\s\displaystyle\frac{1}{2}\cdot\bigl(2&-&\s 1\cdot({\color{red}\phantom{00}\hbox{$-1.00$}})& &             &-&\s 1\cdot(\phantom{+00}1.00)&-&\s 0\cdot(\phantom{+00}3.00)\bigr)\\
{\color{red}\phantom{00}\hbox{$-1.50$}}&=&\s\displaystyle\frac{1}{2}\cdot\bigl(1&-&\s 0\cdot({\color{red}\phantom{00}\hbox{$-1.00$}})&-&\s 1\cdot({\color{red}\phantom{+00}1.00})& &             &-&\s 1\cdot(\phantom{+00}3.00)\bigr)\\
{\color{red}\phantom{+00}5.50}&=&\s\displaystyle\frac{1}{1}\cdot\bigl(3&-&\s 1\cdot({\color{red}\phantom{00}\hbox{$-1.00$}})&-&\s 0\cdot({\color{red}\phantom{+00}1.00})&-&\s 1\cdot({\color{red}\phantom{00}\hbox{$-1.50$}})& &             \bigr)\\
\end{linsys}
\]
}
\only<3>{
Iteration $n=2$:
\[
\begin{linsys}{6}
{\color{red}\phantom{00}\hbox{$-2.50$}}&=&\s\displaystyle\frac{1}{1}\cdot\bigl(4& &             &-&\s 1\cdot(\phantom{+00}1.00)&-&\s 0\cdot(\phantom{00}\hbox{$-1.50$})&-&\s 1\cdot(\phantom{+00}5.50)\bigr)\\
{\color{red}\phantom{+00}3.00}&=&\s\displaystyle\frac{1}{2}\cdot\bigl(2&-&\s 1\cdot({\color{red}\phantom{00}\hbox{$-2.50$}})& &             &-&\s 1\cdot(\phantom{00}\hbox{$-1.50$})&-&\s 0\cdot(\phantom{+00}5.50)\bigr)\\
{\color{red}\phantom{00}\hbox{$-3.75$}}&=&\s\displaystyle\frac{1}{2}\cdot\bigl(1&-&\s 0\cdot({\color{red}\phantom{00}\hbox{$-2.50$}})&-&\s 1\cdot({\color{red}\phantom{+00}3.00})& &             &-&\s 1\cdot(\phantom{+00}5.50)\bigr)\\
{\color{red}\phantom{+00}9.25}&=&\s\displaystyle\frac{1}{1}\cdot\bigl(3&-&\s 1\cdot({\color{red}\phantom{00}\hbox{$-2.50$}})&-&\s 0\cdot({\color{red}\phantom{+00}3.00})&-&\s 1\cdot({\color{red}\phantom{00}\hbox{$-3.75$}})& &             \bigr)\\
\end{linsys}
\]
}
\only<4>{
Iteration $n=3$:
\[
\begin{linsys}{6}
{\color{red}\phantom{00}\hbox{$-8.25$}}&=&\s\displaystyle\frac{1}{1}\cdot\bigl(4& &             &-&\s 1\cdot(\phantom{+00}3.00)&-&\s 0\cdot(\phantom{00}\hbox{$-3.75$})&-&\s 1\cdot(\phantom{+00}9.25)\bigr)\\
{\color{red}\phantom{+00}7.00}&=&\s\displaystyle\frac{1}{2}\cdot\bigl(2&-&\s 1\cdot({\color{red}\phantom{00}\hbox{$-8.25$}})& &             &-&\s 1\cdot(\phantom{00}\hbox{$-3.75$})&-&\s 0\cdot(\phantom{+00}9.25)\bigr)\\
{\color{red}\phantom{00}\hbox{$-7.62$}}&=&\s\displaystyle\frac{1}{2}\cdot\bigl(1&-&\s 0\cdot({\color{red}\phantom{00}\hbox{$-8.25$}})&-&\s 1\cdot({\color{red}\phantom{+00}7.00})& &             &-&\s 1\cdot(\phantom{+00}9.25)\bigr)\\
{\color{red}\phantom{+0}18.88}&=&\s\displaystyle\frac{1}{1}\cdot\bigl(3&-&\s 1\cdot({\color{red}\phantom{00}\hbox{$-8.25$}})&-&\s 0\cdot({\color{red}\phantom{+00}7.00})&-&\s 1\cdot({\color{red}\phantom{00}\hbox{$-7.62$}})& &             \bigr)\\
\end{linsys}
\]
}
\only<5>{
Iteration $n=4$:
\[
\begin{linsys}{6}
{\color{red}\phantom{0}\hbox{$-21.88$}}&=&\s\displaystyle\frac{1}{1}\cdot\bigl(4& &             &-&\s 1\cdot(\phantom{+00}7.00)&-&\s 0\cdot(\phantom{00}\hbox{$-7.62$})&-&\s 1\cdot(\phantom{+0}18.88)\bigr)\\
{\color{red}\phantom{+0}15.75}&=&\s\displaystyle\frac{1}{2}\cdot\bigl(2&-&\s 1\cdot({\color{red}\phantom{0}\hbox{$-21.88$}})& &             &-&\s 1\cdot(\phantom{00}\hbox{$-7.62$})&-&\s 0\cdot(\phantom{+0}18.88)\bigr)\\
{\color{red}\phantom{0}\hbox{$-16.81$}}&=&\s\displaystyle\frac{1}{2}\cdot\bigl(1&-&\s 0\cdot({\color{red}\phantom{0}\hbox{$-21.88$}})&-&\s 1\cdot({\color{red}\phantom{+0}15.75})& &             &-&\s 1\cdot(\phantom{+0}18.88)\bigr)\\
{\color{red}\phantom{+0}41.69}&=&\s\displaystyle\frac{1}{1}\cdot\bigl(3&-&\s 1\cdot({\color{red}\phantom{0}\hbox{$-21.88$}})&-&\s 0\cdot({\color{red}\phantom{+0}15.75})&-&\s 1\cdot({\color{red}\phantom{0}\hbox{$-16.81$}})& &             \bigr)\\
\end{linsys}
\]
}
\only<6>{
Iteration $n=5$:
\[
\begin{linsys}{6}
{\color{red}\phantom{0}\hbox{$-53.44$}}&=&\s\displaystyle\frac{1}{1}\cdot\bigl(4& &             &-&\s 1\cdot(\phantom{+0}15.75)&-&\s 0\cdot(\phantom{0}\hbox{$-16.81$})&-&\s 1\cdot(\phantom{+0}41.69)\bigr)\\
{\color{red}\phantom{+0}36.12}&=&\s\displaystyle\frac{1}{2}\cdot\bigl(2&-&\s 1\cdot({\color{red}\phantom{0}\hbox{$-53.44$}})& &             &-&\s 1\cdot(\phantom{0}\hbox{$-16.81$})&-&\s 0\cdot(\phantom{+0}41.69)\bigr)\\
{\color{red}\phantom{0}\hbox{$-38.41$}}&=&\s\displaystyle\frac{1}{2}\cdot\bigl(1&-&\s 0\cdot({\color{red}\phantom{0}\hbox{$-53.44$}})&-&\s 1\cdot({\color{red}\phantom{+0}36.12})& &             &-&\s 1\cdot(\phantom{+0}41.69)\bigr)\\
{\color{red}\phantom{+0}94.84}&=&\s\displaystyle\frac{1}{1}\cdot\bigl(3&-&\s 1\cdot({\color{red}\phantom{0}\hbox{$-53.44$}})&-&\s 0\cdot({\color{red}\phantom{+0}36.12})&-&\s 1\cdot({\color{red}\phantom{0}\hbox{$-38.41$}})& &             \bigr)\\
\end{linsys}
\]
}
\only<7>{
Iteration $n=6$:
\[
\begin{linsys}{6}
{\color{red}\hbox{$-126.97$}}&=&\s\displaystyle\frac{1}{1}\cdot\bigl(4& &             &-&\s 1\cdot(\phantom{+0}36.12)&-&\s 0\cdot(\phantom{0}\hbox{$-38.41$})&-&\s 1\cdot(\phantom{+0}94.84)\bigr)\\
{\color{red}\phantom{+0}83.69}&=&\s\displaystyle\frac{1}{2}\cdot\bigl(2&-&\s 1\cdot({\color{red}\hbox{$-126.97$}})& &             &-&\s 1\cdot(\phantom{0}\hbox{$-38.41$})&-&\s 0\cdot(\phantom{+0}94.84)\bigr)\\
{\color{red}\phantom{0}\hbox{$-88.77$}}&=&\s\displaystyle\frac{1}{2}\cdot\bigl(1&-&\s 0\cdot({\color{red}\hbox{$-126.97$}})&-&\s 1\cdot({\color{red}\phantom{+0}83.69})& &             &-&\s 1\cdot(\phantom{+0}94.84)\bigr)\\
{\color{red}\phantom{+}218.73}&=&\s\displaystyle\frac{1}{1}\cdot\bigl(3&-&\s 1\cdot({\color{red}\hbox{$-126.97$}})&-&\s 0\cdot({\color{red}\phantom{+0}83.69})&-&\s 1\cdot({\color{red}\phantom{0}\hbox{$-88.77$}})& &             \bigr)\\
\end{linsys}
\]
}
\only<8>{
Iteration $n=7$:
\[
\begin{linsys}{6}
{\color{red}\hbox{$-298.42$}}&=&\s\displaystyle\frac{1}{1}\cdot\bigl(4& &             &-&\s 1\cdot(\phantom{+0}83.69)&-&\s 0\cdot(\phantom{0}\hbox{$-88.77$})&-&\s 1\cdot(\phantom{+}218.73)\bigr)\\
{\color{red}\phantom{+}194.59}&=&\s\displaystyle\frac{1}{2}\cdot\bigl(2&-&\s 1\cdot({\color{red}\hbox{$-298.42$}})& &             &-&\s 1\cdot(\phantom{0}\hbox{$-88.77$})&-&\s 0\cdot(\phantom{+}218.73)\bigr)\\
{\color{red}\hbox{$-206.16$}}&=&\s\displaystyle\frac{1}{2}\cdot\bigl(1&-&\s 0\cdot({\color{red}\hbox{$-298.42$}})&-&\s 1\cdot({\color{red}\phantom{+}194.59})& &             &-&\s 1\cdot(\phantom{+}218.73)\bigr)\\
{\color{red}\phantom{+}507.59}&=&\s\displaystyle\frac{1}{1}\cdot\bigl(3&-&\s 1\cdot({\color{red}\hbox{$-298.42$}})&-&\s 0\cdot({\color{red}\phantom{+}194.59})&-&\s 1\cdot({\color{red}\hbox{$-206.16$}})& &             \bigr)\\
\end{linsys}
\]
}


\end{frame}
