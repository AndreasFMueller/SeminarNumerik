%
% konvergenz.tex
%
% (c) 2020 Prof Dr Andreas Müller, Hochschule Rapperswil
%
\begin{frame}
\frametitle{Konvergenz}
\vspace{-15pt}
\begin{columns}[t]
\begin{column}{0.48\hsize}
\begin{block}{Lösung}
Die Lösung von $Ax=b$ erfüllt
\begin{align*}
Dx&=b-Lx-Rx
\\
\uncover<2->{x}&\uncover<2->{=b_0-(D+L)^{-1}Rx}
\end{align*}
\end{block}
\vspace{-15pt}
\uncover<3->{%
\begin{block}{Iterationsfolge mit Startwert $x_0$}
\vspace{-10pt}
\[
x_{n+1}
=
b_0 - \underbrace{(D + L)^{-1}R}_{\displaystyle=B} x_{n}
\]
\end{block}}
\vspace{-15pt}
\uncover<4->{%
\begin{block}{Fehler}
$\tilde{x}_n = x_n-x$
\begin{align*}
B\tilde{x}_n
&=
x_{n+1}\underbrace{-b_0+Bx}_{\displaystyle=-x}
\uncover<5->{=
\tilde{x}_{n+1}}
\end{align*}
\end{block}}
\end{column}
\begin{column}{0.48\hsize}
\uncover<6->{%
\begin{block}{Wachstum des Fehlers}
$\lambda_1\ge \dots \ge\lambda_n$ die Eigenwerte von $B$, dann ist
\begin{align*}
\uncover<7->{|\tilde{x}_{n+1}|}&\uncover<8->{= |B\tilde{x}_n|}\uncover<9->{\le |\lambda_1| \cdot |\tilde{x}_n|}
\\
	&\uncover<10->{ \le |\lambda_1|^2\cdot |\tilde{x}_{n-1}|}\uncover<11->{\le\dots}
\\
	&\uncover<12->{\le |\lambda_1|^{n+1}\cdot |\tilde{x}_0|}
\end{align*}
\end{block}}
\vspace{-15pt}
\uncover<13->{%
\begin{block}{Konvergenzradius}
\vspace{-20pt}
\begin{align*}
\varrho(B) = |\lambda_1|
&\uncover<14->{=
\max\{|\lambda|\;|\;\text{$\lambda$ EW von $B$}\}}
\end{align*}
\end{block}}
\vspace{-10pt}
\uncover<15->{%
\begin{block}{Stabilität}
\begin{center}
\begin{tabular}{lcl}
\uncover<16->{\usebeamercolor[fg]{title}$\varrho(B)<1$}&\uncover<17->{$\Rightarrow$}&\uncover<17->{lineare Konvergenz}\\
\uncover<18->{\usebeamercolor[fg]{title}$\varrho(B)>1$}&\uncover<19->{$\Rightarrow$}&\uncover<19->{instabil}
\end{tabular}
\end{center}
\end{block}}
\end{column}
\end{columns}
\end{frame}
