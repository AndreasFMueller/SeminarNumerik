%
% iteration.tex
%
% (c) 2020 Prof Dr Andreas Müller, Hochschule Rapperswil
%
\begin{frame}[fragile]
\frametitle{Iteration\uncover<5->{: Jacobi}\uncover<7->{, Gauss-Seidel}}

\def\r#1{\bgroup\color<7->{red}#1\egroup}

\begin{block}{Idee}
 Gleichungssystem nach einzelnen Variablen auflösen: %
\uncover<2-6,8->{$\r{x_1}$, }%
\uncover<3-6,9->{$\r{x_2}$, }%
\uncover<4-6,10->{$\r{x_3}$, }%
\uncover<5-6,11->{$\r{x_4}$}%
%
\def\s{\hspace*{-3mm}}
\renewcommand{\arraystretch}{2.3}
%
\def\schrittnull{%
\begin{linsys}{9}
\s a_{11}x_1&+&\s a_{12}x_2&+&\s a_{13}x_3&+&\s a_{14}x_4&=&\s\phantom{\displaystyle\frac1{a_{33}}(}b_1&\phantom{-}&\s\phantom{a_{33}x_3}&\phantom{-}&\s\phantom{a_{33}x_3}&\phantom{-}&\s\phantom{a_{33}x_3}&\phantom{-}&\s\phantom{a_{33}x_3)}\\
\s a_{21}x_1&+&\s a_{22}x_2&+&\s a_{23}x_3&+&\s a_{24}x_4&=&\s                         b_2&           &\s                   &           &\s                   &           &\s                   &           &\s                    \\
\s a_{31}x_1&+&\s a_{32}x_2&+&\s a_{33}x_3&+&\s a_{34}x_4&=&\s                         b_3&           &\s                   &           &\s                   &           &\s                   &           &\s                    \\
\s a_{41}x_1&+&\s a_{42}x_2&+&\s a_{43}x_3&+&\s a_{44}x_4&=&\s                         b_4&           &\s                   &           &\s                   &           &\s                   &           &\s                    \\
\end{linsys}%
}
%
\def\schritteins{%
\begin{linsys}{9}
\s       \r{x_1}& &\s          & &\s          & &\s          &=&\s\displaystyle\frac1{a_{11}}(b_1&\phantom{-}&\s\phantom{a_{33}x_3}&-&\s a_{12}x_2&-&\s a_{13}x_3&-&\s a_{14}x_4)\\
\s a_{21}\r{x_1}&+&\s a_{22}x_2&+&\s a_{23}x_3&+&\s a_{24}x_4&=&\s               b_2&           &\s                   & &\s          & &\s          & &\s           \\
\s a_{31}\r{x_1}&+&\s a_{32}x_2&+&\s a_{33}x_3&+&\s a_{34}x_4&=&\s               b_3&           &\s                   & &\s          & &\s          & &\s           \\
\s a_{41}\r{x_1}&+&\s a_{42}x_2&+&\s a_{43}x_3&+&\s a_{44}x_4&=&\s               b_4&           &\s                   & &\s          & &\s          & &\s           \\
\end{linsys}%
}
%
\def\schrittzwei{%
\begin{linsys}{9}
\s       \r{x_1}& &\s              & &\s          & &\s          &=&\s\displaystyle\frac1{a_{11}}(b_1& &\s              &-&\s a_{12}x_2&-&\s a_{13}x_3&-&\s a_{14}x_4)\\
\s              & &\s       \r{x_2}& &\s          & &\s          &=&\s\displaystyle\frac1{a_{22}}(b_2&-&\s a_{21}\r{x_1}& &\s          &-&\s a_{23}x_3&-&\s a_{24}x_4)\\
\s a_{31}\r{x_1}&+&\s a_{32}\r{x_2}&+&\s a_{33}x_3&+&\s a_{34}x_4&=&b_3                 & &\s              & &\s          & &\s          & &\s           \\
\s a_{41}\r{x_1}&+&\s a_{42}\r{x_2}&+&\s a_{43}x_3&+&\s a_{44}x_4&=&b_4                 & &\s              & &\s          & &\s          & &\s           \\
\end{linsys}%
}
%
\def\schrittdrei{%
\begin{linsys}{9}
\s       \r{x_1}& &\s              & &\s              & &\s          &=&\s         \displaystyle\frac1{a_{11}}( b_1& &\s              &-&\s     a_{12}x_2&-&\s a_{13}x_3&-&\s a_{14}x_4)\\
\s              & &\s       \r{x_2}& &\s              & &\s          &=&\s         \displaystyle\frac1{a_{22}}( b_2&-&\s a_{21}\r{x_1}& &\s              &-&\s a_{23}x_3&-&\s a_{24}x_4)\\
\s              & &\s              & &\s       \r{x_3}& &\s          &=&\s         \displaystyle\frac1{a_{33}}( b_3&-&\s a_{31}\r{x_1}&-&\s a_{32}\r{x_2}& &\s          &-&\s a_{34}x_4)\\
\s a_{41}\r{x_1}&+&\s a_{42}\r{x_2}&+&\s a_{43}\r{x_3}&+&\s a_{44}x_4&=&\s\phantom{\displaystyle\frac1{a_{33}}(}b_4& &\s              & &\s              & &\s          & &\s           \\
\end{linsys}%
}
%
\def\schrittvier{%
\begin{linsys}{9}
\s            \r{x_1}&\phantom{+}&\s\phantom{a_{11}x_1}&\phantom{+}&\s\phantom{a_{11}x_1}&\phantom{+}&\s\phantom{a_{11}x_1}&=&\s\displaystyle\frac1{a_{11}}(b_1& &\s              &-&\s     a_{12}x_2&-&\s     a_{13}x_3&-&\s a_{14}x_4)\\
\s                   &           &\s            \r{x_2}&           &\s                   &           &\s                   &=&\s\displaystyle\frac1{a_{22}}(b_2&-&\s a_{21}\r{x_1}& &\s              &-&\s     a_{23}x_3&-&\s a_{24}x_4)\\
\s                   &           &\s                   &           &\s            \r{x_3}&           &\s                   &=&\s\displaystyle\frac1{a_{33}}(b_3&-&\s a_{31}\r{x_1}&-&\s a_{32}\r{x_2}& &\s              &-&\s a_{34}x_4)\\
\s\phantom{a_{33}x_3}&           &\s                   &           &\s                   &           &\s            \r{x_4}&=&\s\displaystyle\frac1{a_{44}}(b_4&-&\s a_{41}\r{x_1}&-&\s a_{42}\r{x_2}&-&\s a_{43}\r{x_3}& &\s          )\\
\end{linsys}%
}
%
\ifthenelse{\boolean{presentation}}{
\only<1>{\[
\schrittnull
\]}%
%
\only<2>{\[
\schritteins
\]}%
%
\only<3>{\[
\schrittzwei
\]}%
%
\only<4>{\[
\schrittdrei
\]}%
%
\only<5-6>{\[
\schrittvier
\]}%
%
\only<7>{\[
\schrittnull
\]}%
%
\only<8>{\[
\schritteins
\]}%
%
\only<9>{\[
\schrittzwei
\]}%
%
\only<10>{\[
\schrittdrei
\]}%
%
}{}%
%
\only<11>{\[
\schrittvier
\]}%
\uncover<6->{Aber: Variablen verändern sich im Laufe des Prozesses}%
\end{block}
\end{frame}
