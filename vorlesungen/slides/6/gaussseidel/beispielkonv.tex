%
% beispielkonv.tex
%
% (c) 2020 Prof Dr Andreas Müller, Hochschule Rapeprswil
%
\begin{frame}
\frametitle{Konvergenzradius der Beispiele}
\vspace{-10pt}

\begin{columns}[t]
\begin{column}{0.48\hsize}
\begin{block}{Beispiel 1: konvergent}
\vspace{-10pt}
\[
A = \begin{pmatrix}
8& 1& 1& 1\\
1& 6& 1& 1\\
1& 1& 5& 1\\
1& 1& 1& 7
\end{pmatrix}
\]
\[
\uncover<2->{
\setlength\arraycolsep{2pt}
B =\begin{pmatrix}
   0.000& \phantom{-}0.125& \phantom{-}0.125& \phantom{-}0.125\\
   0.000&           -0.021& \phantom{-}0.146& \phantom{-}0.146\\
   0.000&           -0.021&           -0.054& \phantom{-}0.146\\
   0.000&           -0.012&           -0.031&           -0.060\\
\end{pmatrix}}
\]
\end{block}
\vspace{-10pt}
\uncover<3->{%
\begin{block}{Spektralradius}
\vspace{-15pt}
\[
\varrho(B)
=
0.10230
\quad\Rightarrow\quad
\text{konvergent}
\]
\end{block}}
\end{column}
\begin{column}{0.48\hsize}
\begin{block}{Beispiel 2: instabil}
\vspace{-10pt}
\[
A = \begin{pmatrix}
1&  1&  0&  1\\
1&  2&  1&  0\\
0&  1&  2&  1\\
1&  0&  1&  1
\end{pmatrix}
\]
\[
\uncover<4->{
B = \begin{pmatrix}
0.00& \phantom{-}1.00& \phantom{-}0.00& \phantom{-}1.00\\
0.00&           -0.50& \phantom{-}0.50&           -0.50\\
0.00& \phantom{-}0.25&           -0.25& \phantom{-}0.70\\
0.00&           -1.25& \phantom{-}0.25&           -1.70
\end{pmatrix}}
\]
\end{block}
\vspace{-10pt}
\uncover<5->{%
\begin{block}{Spektralradius}
\vspace{-5pt}
\[
\varrho(B)
=
2.3315
\quad\Rightarrow\quad\text{instabil}
\]
\end{block}}
\end{column}
\end{columns}
\end{frame}
