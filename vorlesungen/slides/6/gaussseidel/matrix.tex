%
% matrix.tex -- Matrixformulierung des Gauss-Seidel Algorithmus
%
% (c) 2020 Prof Dr Andreas Müller, Hochschule Rapperswil
%
\begin{frame}[fragile]
\frametitle{Matrix-Formulierung}

\definecolor{farbed}{rgb}{0.2,0.2,0.8}
\definecolor{farbel}{rgb}{0,0.6,0}
\definecolor{farber}{rgb}{0,0.6,0.8}
\definecolor{farbeb}{rgb}{1.0,0.4,0}

\def\r#1{\bgroup\color{red}#1\egroup}
\def\D#1{\bgroup\color{farbed}#1\egroup}
\def\R#1{\bgroup\color{farber}#1\egroup}
\def\L#1{\bgroup\color{farbel}#1\egroup}
\def\B#1{\bgroup\color{farbeb}#1\egroup}

\def\s{\hspace*{-3mm}}
%\renewcommand{\arraystretch}{2.3}


\def\schrittvier{%
\begin{linsys}{9}
\s\D{a_{11}}\r{x_1}&=&\s\B{b_1}& &\s                 &-&\s    \R{a_{12}}x_2&-&\s    \R{a_{13}}x_3&-&\s\R{a_{14}}x_4\\
\s\D{a_{22}}\r{x_2}&=&\s\B{b_2}&-&\s\L{a_{21}}\r{x_1}& &\s                 &-&\s    \R{a_{23}}x_3&-&\s\R{a_{24}}x_4\\
\s\D{a_{33}}\r{x_3}&=&\s\B{b_3}&-&\s\L{a_{31}}\r{x_1}&-&\s\L{a_{32}}\r{x_2}& &\s                 &-&\s\R{a_{34}}x_4\\
\s\D{a_{44}}\r{x_4}&=&\s\B{b_4}&-&\s\L{a_{41}}\r{x_1}&-&\s\L{a_{42}}\r{x_2}&-&\s\L{a_{43}}\r{x_3}& &\s             \\
\end{linsys}%
}
%

\[
\left.
\schrittvier
\uncover<6->{
\right\}
\quad
\Rightarrow
\quad
\D{D}\r{x} = \B{b} - \L{L}\r{x} - \R{R}x}
\]
\vspace{-10pt}
\uncover<2->{%
\begin{block}{Aufteilung der Matrix $A$}
\vspace{-15pt}
\begin{align*}
\uncover<3->{%
D=\begin{pmatrix}
\D{a_{11}}&         0&         0&     0\\
         0&\D{a_{22}}&         0&     0\\
         0&         0&\D{a_{33}}&     0\\
         0&         0&         0&\D{a_{44}}
\end{pmatrix}}\uncover<4->{\!,
\;
L=\begin{pmatrix}
         0&         0&         0&     0\\
\L{a_{11}}&         0&         0&     0\\
\L{a_{11}}&\L{a_{12}}&         0&     0\\
\L{a_{11}}&\L{a_{12}}&\L{a_{13}}&     0
\end{pmatrix}}\uncover<5->{\!,
\;
R=\begin{pmatrix}
     0&\R{a_{12}}&\R{a_{13}}&\R{a_{14}}\\
     0&         0&\R{a_{13}}&\R{a_{14}}\\
     0&         0&         0&\R{a_{14}}\\
     0&         0&         0&         0
\end{pmatrix}}
\end{align*}
\end{block}}
\vspace{-15pt}
\uncover<7->{%
\begin{block}{Auflösen nach $\r{x}$}
\vspace{-10pt}
\[
\D{D}\r{x} = \B{b} - \L{L}\r{x} - \R{R}x
\uncover<8->{\quad\Rightarrow\quad
(\D{D}+\L{L})\r{x}=\B{b}-\R{R}x}
\uncover<9->{\quad\Rightarrow\quad
\r{x}=\underbrace{(\D{D}+\L{L})^{-1}\B{b}}_{\displaystyle\B{b_0}} - (\D{D}+\L{L})^{-1}\R{R}\, x}
\]
\end{block}}

\end{frame}
