%
% zuteilung.tex
%
% (c) 2020 Prof Dr Andreas Müller, Hochschule Rapperswil
%
\documentclass[handout]{beamer}
\usepackage{array}
\usepackage{german}
\usepackage{graphicx}
\usepackage[utf8]{inputenc}
\usepackage[T1]{fontenc}
\mode<beamer>{%
\usetheme{Copenhagen}
}
\usepackage[orientation=landscape,size=a4,debug,scale=2.9]{beamerposter}
\title[]{}
\begin{document}
\begin{frame}
\frametitle{Themenzuteilung}
\begin{center}
\renewcommand\arraystretch{1.25}
\begin{tabular}{|l|c|l|l|l|}
\hline
Name&&Thema&Kürzel&Vortragsdatum\\
\hline
\hline
% 1. Termin 20. April
Michael Schneeberger     &E  &Iteration der logistischen Gleichung&logistic&20. April\\
\hline
\hline
% 2. Termin 27. April
Mike Schmid              &I  &Gauss-Quadratur&gauss&27.~April\\
\hline
Joel Rechsteiner         &E  &Finite Elemente&fe&\\
\hline
\hline
% 3. Termin 5. Mai
Severin Weiss            &EEU&Numerische Laplace-Inversion&laplace&\phantom{0}5.~Mai\\
\hline
Manuel Tischhauser       &E  &Stabile QR-Zerlegung&qr&\\
\hline
\hline
% 4. Termin 11. Mai
Reto Fritsche            &E  &Schrittlängensteuerung&steps&11.~Mai\\
\hline
Fabio Marti              &E  &ODE mit Taylor&taylor&\\
\hline
\hline
% 5. Termin 18. Mai
Daniel Bucher            &I  &Störungstheorie&perturbation&18.~Mai\\
Thomas Kistler           &I  &&&\\
\hline
Manuel Catteneo          &E  &Van der Pol Oszillator&vanderpol&\\
Niccolo Galliani         &E  &&&\\
\hline
\hline
% 6. Termin: 25. Mai
Cédric Renda             &E  &Padé-Approximation&pade&25.~Mai\\
\hline
Michael Schmid           &E  &Numerische Lösung der Burgers-Gleichung&burgers&\\
\hline
%Reto Wildhaber           &B  &Kettenbrüche&{\tiny QR}&OK\\
%\hline
\end{tabular}
\end{center}
\end{frame}
\end{document}

