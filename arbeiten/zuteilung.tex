%
% zuteilung.tex
%
% (c) 2020 Prof Dr Andreas Müller, Hochschule Rapperswil
%
\documentclass[handout]{beamer}
\usepackage{array}
\usepackage{german}
\usepackage{graphicx}
\usepackage[utf8]{inputenc}
\usepackage[T1]{fontenc}
\mode<beamer>{%
\usetheme{Copenhagen}
}
\usepackage[orientation=landscape,size=a4,debug,scale=2.9]{beamerposter}
\title[]{}
\begin{document}
\begin{frame}
\frametitle{Themenzuteilung}
\begin{center}
\renewcommand\arraystretch{1.25}
\begin{tabular}{|p{6cm}|c|p{6cm}|p{6cm}|c|}
\hline
Name&&1. Wahl&2. Wahl&\\
\hline
\hline
Benjamin Bouhafs-Keller  &B  &&&\\
\hline
Daniel Bucher            &I  &Störungstheorie&Kepler, Taylor&\\
\hline
Manuel Catteneo          &E  &Laplace-Inversion&Van der Pol&\\
Niccolo Galliani         &E  &Laplace-Inversion&Van der Pol&\\
\hline
Reto Fritsche            &E  &Schrittlängensteuerung&{\tiny Kepler}&OK\\
\hline
Rahel Hertelendy         &B  &&&\\
\hline
Thomas Kistler           &I  &Störungstheorie&Gauss, Taylor&\\
\hline
Marc Kühne               &B  &&&\\
\hline
Fabio Marti              &E  &Taylor&logistische Gleichung&\\
\hline
Joel Rechsteiner         &E  &Finite Elemente&{\tiny Bessel}&OK\\
\hline
Cédric Renda             &E  &&&\\
\hline
Mike Schmid              &I  &Taylor&Gauss&\\
\hline
Michael Schmid           &E  &Störungstheorie&Burgers&\\
\hline
Michael Schneeberger     &E  &logistische Gleichung&{\tiny $z^3-1$}&OK\\
\hline
Manuel Tischhauser       &E  &QR-Zerlegung&{\tiny Bessel, Störungstheorie}&OK\\
\hline
Severin Weiss            &EEU&Laplace&&\\
\hline
Reto Wildhaber           &B  &Kettenbrüche&{\tiny QR}&OK\\
\hline
\end{tabular}
\end{center}
\end{frame}
\end{document}

